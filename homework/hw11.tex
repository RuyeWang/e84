\documentstyle[11pt]{article}
\usepackage{html}
\begin{document}
\begin{center}
{\Large \bf  E84 Home Work 11}
\end{center}
\begin{enumerate}

\item Find values of $R_C$ and $R_B$ in the circuit with $\beta=100$
and $V_{CC}=15V$ so that the Q-point is $I_C=25mA$ and $V_{CE}=7.5V$.
What is the Q point if $\beta=200$?

\htmladdimg{../hw9a.gif}


% {\bf Solution:}

% Find $I_B=I_C/\beta=25mA/100=0.25mA$, $R_B=(15-0.7)/0.25=57.2k\Omega$.
% Also as $V_{CE}=V_{CC}-I_C R_C$, i.e., $R_C=(V_{CC}-V_{CE})/I_C=
% (15-7.5)/25\times 10^{-3}=300\Omega$. 
% 
% $I_B=(V_{CC}-V_{BD})/R_B=0.25mA$, $I_C=\beta I_B=200\times 0.25=50mA$,
% $V_{CE}=V_{CC}-R_C I_C=15-300\times 50\times 10^{-3}=0$.

\item Design a stable self-biasing transistor circuit with $V_{CC}=15V$ 
and the DC operating point (Q point) of $I_C=2.5mA$ and $V_{CE}=7.5V$ in 
the middle of the load line. The $\beta$ of the ransistor ranges from 50 
to 200. 

{\bf Hint} This is a design problem with possibly multiple solutions, 
i.e., there may be more degrees of freedom than constraining conditions. 
One of such conditions is $(\beta+1)R_E \gg R_B$ (typically, 
$R_B \le 0.1\times \beta_{min} R_E$) for the DC operating point to be 
approximately independent of $\beta$ (see online notes). 

Start the design process from the desired Q-point, determine $R_C$ and
$R_E$, then find desired $V_{BB}$ and finally $R_1$ and $R_2$.

\htmladdimg{../hw9b.gif}

% {\bf Solution:}
% 
% \begin{itemize}
% \item Find $V_{CC}$: We want the Q-point to be in the middle of the
%   load line, so we set $V_{CC}=2V_{CE}=2\times 7.5=15V$.
% \item Find $R_C$ and $R_E$: As $V_{CE}=V_{CC}-I_CR_C-I_ER_E\approx
%   V_{CC}-I_C(R_C+R_E)$, we have $R_C+R_E=7.5/2.5\times 10^{-3}=3K\Omega$.
%   Choose $R_E=1K\Omega$ and $R_C=2K\Omega$.
% \item Find $R_B$: To satisfy $R_B \ll \beta_{min} R_E$, we
% 	let $R_B=0.1\times \beta_{min} R_E=0.1\times 50\times 1000=5\;K\Omega$
% \item Find $V_{BB}$: 
% 	\[ V_{BB}=V_{be}+I_ER_E=0.7+2.5\times 10^{-3}\times 10^3=3.2V \]
% \item Find $R_1$ and $R_2$:
% \[	R_B=\frac{R_1R_2}{R_1+R_2}=5\;K\Omega \;\;\;\;\;\;\;\;
% 	\frac{V_{BB}}{V_{CC}}=\frac{R_2}{R_1+R_2}=\frac{3.2}{15}=0.21	\]
% Solve these two equations (first divide the first equation by the second), 
% we obtain the two unknowns $R_1$ and $R_2$:
% \[	R_1=\frac{5\;K\Omega}{0.21}=24\;K\Omega \;\;\;\;\;\;\;\;\;
% 	R_2=6.4\; K\Omega	\]
% \end{itemize}

\item In the AC amplifier shown in the figure, $R_S=600\Omega$, $R_1=30K\Omega$, 
$R_2=20K\Omega$, $R_E=4K\Omega$, $R_C=3K\Omega$, $R_L=5.1K\Omega$, 
$\beta=100$, $V_{CC}=15V$. Also assume $r_{be}=1000\Omega$. And assume the
frequency of the AC signals to be amplified is high enough so that the
coupling capacitors and the emitter by-pass capacitor can be treated as 
AC short circuits. Find
\begin{itemize}
\item The DC operating point in terms of variables $I_B$, $I_C$, $V_C$, $V_E$ 
  and $V_{CE}$.
\item The AC equivalent diagram
\item The AC input and output impedances
\item The voltage gain of the AC amplifier
\end{itemize}

\htmladdimg{../hw9c.gif}

% {\bf Solution:} 
% \begin{itemize}
% \item Find DC operating point.
% \[ V_{BB}=V_{CC}\frac{R_2}{R_1+R_2}=15\frac{20}{30+20}=6V	\]
% \[ R_B=\frac{R_1R_2}{R_1+R_2}=\frac{600}{50}=12K\Omega 	\]
% \[ I_B=\frac{V_{BB}-V_{be}}{(\beta+1)R_E+R_B}
% 	=\frac{6-0.7}{101\times 4000+12000}=\frac{5.3}{26\times 10^4}
% 	=0.025 mA	\]
% \[ I_C=\beta I_B=100\times 0.025 mA=2.5 mA	\]
% \[ V_C=V_{CC}-I_C R_C=15-2.5\times 3=7.5V	\]
% \[ V_E=(\beta+1)I_C R_E=5V	\]
% \[ V_{CE}=V_C-V_E=2.5V \]
% \item Find input and output impedances.
% \[	r_{be}=V_T/I_B=0.026/0.025=1K\Omega	\]
% \[	r_{in}=R_1||R_2||r_{be} \approx 1K\Omega \]
% \[	r_{out}=R_C=3K\Omega	\]

% \item Find voltage gain.
% \[	G=-\beta \frac{(R_1||R_2||r_{be})}{(R_1||R_2||r_{be})+R_s}
% 	\frac{1}{r_{be}}(R_C||R_L) \approx -117	\]
% \end{itemize}

\item An emitter follower circuit is shown in the figure. Assume 
$r_{be}=1K\Omega$, $\beta=100$, $R_B=300K\Omega$, $R_E=4K\Omega$, 
$R_S=500\Omega$, $R_L=5.1K\Omega$, and $V_{CC}=12$. Find the DC operating 
point in terms of $I_B$, $I_C$, $I_E$, and $V_E$. Then find the input and 
output resistances and the voltage gain.

\htmladdimg{../hw9e.gif}

% {\bf Solution}
% 
% Find base current:
% \[ I_B=\frac{V_{CC}-V_{BE}}{R_B+(\beta+1)R_E}=16.1 \mu A \]
% \[ I_E=(\beta+1) I_B=1.61 mA \]
% \[ V_E=I_E R_E=6.48 V,  V_{CE}=V_{CC}-V_E=5.52 V \]
% \[ G=\frac{(\beta+1)R_E||R_L}{R_S+r_{be}+(\beta+1)R_E||R_L} 
%     =101\times 2.24/(1.5+101\times 2.24)=0.993 \]
% \[ r_{in}=(\beta+1)(R_E||R_L)=101\times 2.24K\Omega =226.4 K\Omega \]
% \[ r_{out}=(R_S+r_{be})/(\beta+1)=1.5K\Omega/101=15 \Omega \]


\end{enumerate}

\end{document}

\item In the circuit shown below, the two base resistors $R_B=43K\Omega$,
the collector $R_C=1K\Omega$. Assume each of the two input voltages $V_1$
and $V_2$ are either 0.2V or 5V. Find the output voltage $V_{out}$ in the
following combinations of the inputs. (Hint: assume 5V input to a transistor
will drive it to saturation.)

\begin{tabular}{cc|c}\hline
$V_1$ & $V_2$ & $V_{out}$ \\ \hline
 0.2  &  0.2  &           \\ \hline
 5.0  &  0.2  &           \\ \hline
 0.2  &  5.0  &           \\ \hline
 5.0  &  5.0  &           \\ \hline
\end{tabular}

\htmladdimg{../hw9d.gif}

\end{itemize}


\item In the AC amplifier shown in the figure, $R_S=1.5K\Omega$, $R_1=10K\Omega$, 
$R_2=200K\Omega$, $R_C=3.5K\Omega$, $R_L=2K\Omega$, $\beta=100$, $V_{CC}=20V$.
Also assume $r_{be}=200\Omega$. Find the DC variables $I_B$, $I_C$, $V_C$.

