\batchmode
\documentstyle[11pt]{article}

\let\providecommand\newcommand
\RequirePackage{ifthen}


\usepackage{html}






\usepackage[]{inputenc}



\makeatletter

\makeatletter
\count@=\the\catcode`\_ \catcode`\_=8 
\newenvironment{tex2html_wrap}{}{}%
\catcode`\<=12\catcode`\_=\count@
\newcommand{\providedcommand}[1]{\expandafter\providecommand\csname #1\endcsname}%
\newcommand{\renewedcommand}[1]{\expandafter\providecommand\csname #1\endcsname{}%
  \expandafter\renewcommand\csname #1\endcsname}%
\newcommand{\newedenvironment}[1]{\newenvironment{#1}{}{}\renewenvironment{#1}}%
\let\newedcommand\renewedcommand
\let\renewedenvironment\newedenvironment
\makeatother
\let\mathon=$
\let\mathoff=$
\ifx\AtBeginDocument\undefined \newcommand{\AtBeginDocument}[1]{}\fi
\newbox\sizebox
\setlength{\hoffset}{0pt}\setlength{\voffset}{0pt}
\addtolength{\textheight}{\footskip}\setlength{\footskip}{0pt}
\addtolength{\textheight}{\topmargin}\setlength{\topmargin}{0pt}
\addtolength{\textheight}{\headheight}\setlength{\headheight}{0pt}
\addtolength{\textheight}{\headsep}\setlength{\headsep}{0pt}
\setlength{\textwidth}{349pt}
\newwrite\lthtmlwrite
\makeatletter
\let\realnormalsize=\normalsize
\global\topskip=2sp
\def\preveqno{}\let\real@float=\@float \let\realend@float=\end@float
\def\@float{\let\@savefreelist\@freelist\real@float}
\def\liih@math{\ifmmode$\else\bad@math\fi}
\def\end@float{\realend@float\global\let\@freelist\@savefreelist}
\let\real@dbflt=\@dbflt \let\end@dblfloat=\end@float
\let\@largefloatcheck=\relax
\let\if@boxedmulticols=\iftrue
\def\@dbflt{\let\@savefreelist\@freelist\real@dbflt}
\def\adjustnormalsize{\def\normalsize{\mathsurround=0pt \realnormalsize
 \parindent=0pt\abovedisplayskip=0pt\belowdisplayskip=0pt}%
 \def\phantompar{\csname par\endcsname}\normalsize}%
\def\lthtmltypeout#1{{\let\protect\string \immediate\write\lthtmlwrite{#1}}}%
\newcommand\lthtmlhboxmathA{\adjustnormalsize\setbox\sizebox=\hbox\bgroup\kern.05em }%
\newcommand\lthtmlhboxmathB{\adjustnormalsize\setbox\sizebox=\hbox to\hsize\bgroup\hfill }%
\newcommand\lthtmlvboxmathA{\adjustnormalsize\setbox\sizebox=\vbox\bgroup %
 \let\ifinner=\iffalse \let\)\liih@math }%
\newcommand\lthtmlboxmathZ{\@next\next\@currlist{}{\def\next{\voidb@x}}%
 \expandafter\box\next\egroup}%
\newcommand\lthtmlmathtype[1]{\gdef\lthtmlmathenv{#1}}%
\newcommand\lthtmllogmath{\lthtmltypeout{l2hSize %
:\lthtmlmathenv:\the\ht\sizebox::\the\dp\sizebox::\the\wd\sizebox.\preveqno}}%
\newcommand\lthtmlfigureA[1]{\let\@savefreelist\@freelist
       \lthtmlmathtype{#1}\lthtmlvboxmathA}%
\newcommand\lthtmlpictureA{\bgroup\catcode`\_=8 \lthtmlpictureB}%
\newcommand\lthtmlpictureB[1]{\lthtmlmathtype{#1}\egroup
       \let\@savefreelist\@freelist \lthtmlhboxmathB}%
\newcommand\lthtmlpictureZ[1]{\hfill\lthtmlfigureZ}%
\newcommand\lthtmlfigureZ{\lthtmlboxmathZ\lthtmllogmath\copy\sizebox
       \global\let\@freelist\@savefreelist}%
\newcommand\lthtmldisplayA{\bgroup\catcode`\_=8 \lthtmldisplayAi}%
\newcommand\lthtmldisplayAi[1]{\lthtmlmathtype{#1}\egroup\lthtmlvboxmathA}%
\newcommand\lthtmldisplayB[1]{\edef\preveqno{(\theequation)}%
  \lthtmldisplayA{#1}\let\@eqnnum\relax}%
\newcommand\lthtmldisplayZ{\lthtmlboxmathZ\lthtmllogmath\lthtmlsetmath}%
\newcommand\lthtmlinlinemathA{\bgroup\catcode`\_=8 \lthtmlinlinemathB}
\newcommand\lthtmlinlinemathB[1]{\lthtmlmathtype{#1}\egroup\lthtmlhboxmathA
  \vrule height1.5ex width0pt }%
\newcommand\lthtmlinlineA{\bgroup\catcode`\_=8 \lthtmlinlineB}%
\newcommand\lthtmlinlineB[1]{\lthtmlmathtype{#1}\egroup\lthtmlhboxmathA}%
\newcommand\lthtmlinlineZ{\egroup\expandafter\ifdim\dp\sizebox>0pt %
  \expandafter\centerinlinemath\fi\lthtmllogmath\lthtmlsetinline}
\newcommand\lthtmlinlinemathZ{\egroup\expandafter\ifdim\dp\sizebox>0pt %
  \expandafter\centerinlinemath\fi\lthtmllogmath\lthtmlsetmath}
\newcommand\lthtmlindisplaymathZ{\egroup %
  \centerinlinemath\lthtmllogmath\lthtmlsetmath}
\def\lthtmlsetinline{\hbox{\vrule width.1em \vtop{\vbox{%
  \kern.1em\copy\sizebox}\ifdim\dp\sizebox>0pt\kern.1em\else\kern.3pt\fi
  \ifdim\hsize>\wd\sizebox \hrule depth1pt\fi}}}
\def\lthtmlsetmath{\hbox{\vrule width.1em\kern-.05em\vtop{\vbox{%
  \kern.1em\kern0.8 pt\hbox{\hglue.17em\copy\sizebox\hglue0.8 pt}}\kern.3pt%
  \ifdim\dp\sizebox>0pt\kern.1em\fi \kern0.8 pt%
  \ifdim\hsize>\wd\sizebox \hrule depth1pt\fi}}}
\def\centerinlinemath{%
  \dimen1=\ifdim\ht\sizebox<\dp\sizebox \dp\sizebox\else\ht\sizebox\fi
  \advance\dimen1by.5pt \vrule width0pt height\dimen1 depth\dimen1 
 \dp\sizebox=\dimen1\ht\sizebox=\dimen1\relax}

\def\lthtmlcheckvsize{\ifdim\ht\sizebox<\vsize 
  \ifdim\wd\sizebox<\hsize\expandafter\hfill\fi \expandafter\vfill
  \else\expandafter\vss\fi}%
\providecommand{\selectlanguage}[1]{}%
\makeatletter \tracingstats = 1 


\begin{document}
\pagestyle{empty}\thispagestyle{empty}\lthtmltypeout{}%
\lthtmltypeout{latex2htmlLength hsize=\the\hsize}\lthtmltypeout{}%
\lthtmltypeout{latex2htmlLength vsize=\the\vsize}\lthtmltypeout{}%
\lthtmltypeout{latex2htmlLength hoffset=\the\hoffset}\lthtmltypeout{}%
\lthtmltypeout{latex2htmlLength voffset=\the\voffset}\lthtmltypeout{}%
\lthtmltypeout{latex2htmlLength topmargin=\the\topmargin}\lthtmltypeout{}%
\lthtmltypeout{latex2htmlLength topskip=\the\topskip}\lthtmltypeout{}%
\lthtmltypeout{latex2htmlLength headheight=\the\headheight}\lthtmltypeout{}%
\lthtmltypeout{latex2htmlLength headsep=\the\headsep}\lthtmltypeout{}%
\lthtmltypeout{latex2htmlLength parskip=\the\parskip}\lthtmltypeout{}%
\lthtmltypeout{latex2htmlLength oddsidemargin=\the\oddsidemargin}\lthtmltypeout{}%
\makeatletter
\if@twoside\lthtmltypeout{latex2htmlLength evensidemargin=\the\evensidemargin}%
\else\lthtmltypeout{latex2htmlLength evensidemargin=\the\oddsidemargin}\fi%
\lthtmltypeout{}%
\makeatother
\setcounter{page}{1}
\onecolumn

% !!! IMAGES START HERE !!!

{\newpage\clearpage
\lthtmlinlinemathA{tex2html_wrap_inline225}%
$v(t)=110\sqrt{2} \;\cos(2\pi 60\;t)$%
\lthtmlinlinemathZ
\lthtmlcheckvsize\clearpage}

{\newpage\clearpage
\lthtmlinlinemathA{tex2html_wrap_inline227}%
$R_1=100\Omega$%
\lthtmlinlinemathZ
\lthtmlcheckvsize\clearpage}

{\newpage\clearpage
\lthtmlinlinemathA{tex2html_wrap_inline229}%
$R_2=50\Omega$%
\lthtmlinlinemathZ
\lthtmlcheckvsize\clearpage}

{\newpage\clearpage
\lthtmlinlinemathA{tex2html_wrap_inline231}%
$C=6.63\mu F$%
\lthtmlinlinemathZ
\lthtmlcheckvsize\clearpage}

{\newpage\clearpage
\lthtmlinlinemathA{tex2html_wrap_inline233}%
$L=0.53 H$%
\lthtmlinlinemathZ
\lthtmlcheckvsize\clearpage}

{\newpage\clearpage
\lthtmlinlinemathA{tex2html_wrap_inline235}%
$\phi=\tan^{-1}(X/R)<0$%
\lthtmlinlinemathZ
\lthtmlcheckvsize\clearpage}

{\newpage\clearpage
\lthtmlinlinemathA{tex2html_wrap_inline237}%
$\phi>0$%
\lthtmlinlinemathZ
\lthtmlcheckvsize\clearpage}

{\newpage\clearpage
\lthtmldisplayA{displaymath181}%
\begin{displaymath} Z_C=-\frac{j}{\omega C}=-\frac{j}{2\pi 60\times 6.63\times 10^{-6}}
  	=-j400\Omega\end{displaymath}%
\lthtmldisplayZ
\lthtmlcheckvsize\clearpage}

{\newpage\clearpage
\lthtmldisplayA{displaymath182}%
\begin{displaymath}Z_L=j\omega L=j 2\pi 60\times 0.53=j200\Omega,\;\;\;\; Z_{RL}=100+j200\end{displaymath}%
\lthtmldisplayZ
\lthtmlcheckvsize\clearpage}

{\newpage\clearpage
\lthtmldisplayA{displaymath183}%
\begin{displaymath} Z_{CRL}=\frac{Z_C Z_{RL}}{Z_C+Z_{RL}}=\frac{(100+j200)(-j400)}{100+j200-j400}
  	=\frac{800-j400}{1-j2}	\end{displaymath}%
\lthtmldisplayZ
\lthtmlcheckvsize\clearpage}

{\newpage\clearpage
\lthtmldisplayA{displaymath184}%
\begin{displaymath}
  Z_{total}=Z_{CRL}+R_2=\frac{800-j400}{1-j2}+50=370+j240=441\angle 33^\circ
  \end{displaymath}%
\lthtmldisplayZ
\lthtmlcheckvsize\clearpage}

{\newpage\clearpage
\lthtmldisplayA{displaymath185}%
\begin{displaymath} \dot{I}=\frac{\dot{V}}{Z_{total}}=\frac{110}{441\angle 33^\circ}
  	=0.25\angle -33^\circ	\end{displaymath}%
\lthtmldisplayZ
\lthtmlcheckvsize\clearpage}

{\newpage\clearpage
\lthtmlinlinemathA{tex2html_wrap_inline241}%
$\lambda=cos (-33^\circ)=0.839$%
\lthtmlinlinemathZ
\lthtmlcheckvsize\clearpage}

{\newpage\clearpage
\lthtmlinlinemathA{tex2html_wrap_inline243}%
$S=110\times 0.25=27.5 W$%
\lthtmlinlinemathZ
\lthtmlcheckvsize\clearpage}

{\newpage\clearpage
\lthtmlinlinemathA{tex2html_wrap_inline245}%
$P=S \cos 33^\circ=27.5\times 0.84=23 W$%
\lthtmlinlinemathZ
\lthtmlcheckvsize\clearpage}

{\newpage\clearpage
\lthtmlinlinemathA{tex2html_wrap_inline247}%
$Q=S \sin 33^\circ=27.5\times 0.54=15 W$%
\lthtmlinlinemathZ
\lthtmlcheckvsize\clearpage}

{\newpage\clearpage
\lthtmlinlinemathA{tex2html_wrap_inline249}%
$C$%
\lthtmlinlinemathZ
\lthtmlcheckvsize\clearpage}

{\newpage\clearpage
\lthtmlinlinemathA{tex2html_wrap_inline253}%
$1/j\omega C=-jX$%
\lthtmlinlinemathZ
\lthtmlcheckvsize\clearpage}

{\newpage\clearpage
\lthtmlinlinemathA{tex2html_wrap_inline255}%
$X=1/\omega C$%
\lthtmlinlinemathZ
\lthtmlcheckvsize\clearpage}

{\newpage\clearpage
\lthtmldisplayA{displaymath186}%
\begin{displaymath}	Z_{all}=-jX || Z_{total}=\frac{-jX(370+j240)}{-jX+(370+j240)}
  	=\frac{240X-j370X}{370-j(X-240)}=|Z|\angle Z=|Z|\angle \phi	\end{displaymath}%
\lthtmldisplayZ
\lthtmlcheckvsize\clearpage}

{\newpage\clearpage
\lthtmlinlinemathA{tex2html_wrap_inline257}%
$\phi=\cos^{-1} 0.9=25.84^\circ$%
\lthtmlinlinemathZ
\lthtmlcheckvsize\clearpage}

{\newpage\clearpage
\lthtmldisplayA{displaymath187}%
\begin{displaymath}	\tan^{-1}[\frac{-370X}{240X}]-\tan^{-1}[\frac{-(X-240)}{370}]=
  	-57^\circ+\tan^{-1}[\frac{X-240}{370}]=25.84^\circ \end{displaymath}%
\lthtmldisplayZ
\lthtmlcheckvsize\clearpage}

{\newpage\clearpage
\lthtmldisplayA{displaymath188}%
\begin{displaymath}	\tan^{-1}[\frac{X-240}{370}]=82.84^\circ, \;\;\;
  \frac{X-240}{370}=\tan \;82.84^\circ=7.96, \;\;\; X=3185.4 \end{displaymath}%
\lthtmldisplayZ
\lthtmlcheckvsize\clearpage}

{\newpage\clearpage
\lthtmldisplayA{displaymath189}%
\begin{displaymath} \frac{1}{\omega C}=X=3185.4,\;\;\;\;C=\frac{1}{2\pi 60\times 3185.4}
  =0.83 \mu F \end{displaymath}%
\lthtmldisplayZ
\lthtmlcheckvsize\clearpage}

{\newpage\clearpage
\lthtmldisplayA{displaymath190}%
\begin{displaymath}	\tan^{-1}[\frac{-370X}{240X}]-\tan^{-1}[\frac{-(X-240)}{370}]=
  	-57^\circ+\tan^{-1}[\frac{X-240}{370}]=0^\circ \end{displaymath}%
\lthtmldisplayZ
\lthtmlcheckvsize\clearpage}

{\newpage\clearpage
\lthtmldisplayA{displaymath191}%
\begin{displaymath}	\tan^{-1}[\frac{X-240}{370}]=57^\circ, \;\;\;
  \frac{X-240}{370}=\tan \;57^\circ=1.54, \;\;\; X=810 \end{displaymath}%
\lthtmldisplayZ
\lthtmlcheckvsize\clearpage}

{\newpage\clearpage
\lthtmldisplayA{displaymath192}%
\begin{displaymath} \frac{1}{\omega C}=X=810,\;\;\;\;C=\frac{1}{2\pi 60\times 810}
  =3.27 \mu F \end{displaymath}%
\lthtmldisplayZ
\lthtmlcheckvsize\clearpage}

{\newpage\clearpage
\lthtmlinlinemathA{tex2html_wrap_inline259}%
${\bf Z}^{-1}={\bf Y}$%
\lthtmlinlinemathZ
\lthtmlcheckvsize\clearpage}

{\newpage\clearpage
\lthtmlinlinemathA{tex2html_wrap_inline261}%
$I_2=0$%
\lthtmlinlinemathZ
\lthtmlcheckvsize\clearpage}

{\newpage\clearpage
\lthtmlinlinemathA{tex2html_wrap_inline263}%
$Z_{11}=V_1/I_1=1/j\omega C$%
\lthtmlinlinemathZ
\lthtmlcheckvsize\clearpage}

{\newpage\clearpage
\lthtmlinlinemathA{tex2html_wrap_inline265}%
$Z_{21}=V_2/I_1=1/j\omega C$%
\lthtmlinlinemathZ
\lthtmlcheckvsize\clearpage}

{\newpage\clearpage
\lthtmlinlinemathA{tex2html_wrap_inline267}%
$I_1=0$%
\lthtmlinlinemathZ
\lthtmlcheckvsize\clearpage}

{\newpage\clearpage
\lthtmlinlinemathA{tex2html_wrap_inline269}%
$Z_{12}=V_1/I_2=1/j\omega C$%
\lthtmlinlinemathZ
\lthtmlcheckvsize\clearpage}

{\newpage\clearpage
\lthtmlinlinemathA{tex2html_wrap_inline271}%
$Z_{22}=V_2/I_2=j\omega L+1/j\omega C$%
\lthtmlinlinemathZ
\lthtmlcheckvsize\clearpage}

{\newpage\clearpage
\lthtmlinlinemathA{tex2html_wrap_inline273}%
$V_2=0$%
\lthtmlinlinemathZ
\lthtmlcheckvsize\clearpage}

{\newpage\clearpage
\lthtmlinlinemathA{tex2html_wrap_inline275}%
$Y_{11}=I_1/V_1=j\omega C+1/j\omega L$%
\lthtmlinlinemathZ
\lthtmlcheckvsize\clearpage}

{\newpage\clearpage
\lthtmlinlinemathA{tex2html_wrap_inline277}%
$Y_{21}=I_2/V_1=1/j\omega L$%
\lthtmlinlinemathZ
\lthtmlcheckvsize\clearpage}

{\newpage\clearpage
\lthtmlinlinemathA{tex2html_wrap_inline279}%
$V_1=0$%
\lthtmlinlinemathZ
\lthtmlcheckvsize\clearpage}

{\newpage\clearpage
\lthtmlinlinemathA{tex2html_wrap_inline281}%
$Y_{12}=I_1/V_2=1/j\omega L$%
\lthtmlinlinemathZ
\lthtmlcheckvsize\clearpage}

{\newpage\clearpage
\lthtmlinlinemathA{tex2html_wrap_inline283}%
$Y_{22}=I_2/V_2=1/j\omega L$%
\lthtmlinlinemathZ
\lthtmlcheckvsize\clearpage}

{\newpage\clearpage
\lthtmldisplayA{displaymath193}%
\begin{displaymath} {\bf Z}^{-1}=\left[ \begin{array}{rr} 1/j\omega C & 1/j\omega C \\
  	1/j\omega C & j\omega L+1/j\omega C \end{array} \right]^{-1}
  	=\frac{C}{L}\left[ \begin{array}{rr} j\omega L+1/j\omega C & -1/j\omega C \\
  	-1/j\omega C & 1/j\omega C \end{array} \right]
  	=\left[ \begin{array}{rr} j\omega C+1/j\omega L & -1/j\omega L \\
  	-1/j\omega L & 1/j\omega L \end{array} \right]
  \end{displaymath}%
\lthtmldisplayZ
\lthtmlcheckvsize\clearpage}

{\newpage\clearpage
\lthtmlinlinemathA{tex2html_wrap_inline285}%
$Y_{11}=-4$%
\lthtmlinlinemathZ
\lthtmlcheckvsize\clearpage}

{\newpage\clearpage
\lthtmlinlinemathA{tex2html_wrap_inline287}%
$Y_{12}=3$%
\lthtmlinlinemathZ
\lthtmlcheckvsize\clearpage}

{\newpage\clearpage
\lthtmlinlinemathA{tex2html_wrap_inline289}%
$Y_{21}=3$%
\lthtmlinlinemathZ
\lthtmlcheckvsize\clearpage}

{\newpage\clearpage
\lthtmlinlinemathA{tex2html_wrap_inline291}%
$Y_{22}=-2$%
\lthtmlinlinemathZ
\lthtmlcheckvsize\clearpage}

{\newpage\clearpage
\lthtmlinlinemathA{tex2html_wrap_inline293}%
$V_0=5V$%
\lthtmlinlinemathZ
\lthtmlcheckvsize\clearpage}

{\newpage\clearpage
\lthtmlinlinemathA{tex2html_wrap_inline295}%
$Z_0=1\Omega$%
\lthtmlinlinemathZ
\lthtmlcheckvsize\clearpage}

{\newpage\clearpage
\lthtmlinlinemathA{tex2html_wrap_inline297}%
$Z_L=j1\Omega$%
\lthtmlinlinemathZ
\lthtmlcheckvsize\clearpage}

{\newpage\clearpage
\lthtmlinlinemathA{tex2html_wrap_inline299}%
$I_1$%
\lthtmlinlinemathZ
\lthtmlcheckvsize\clearpage}

{\newpage\clearpage
\lthtmlinlinemathA{tex2html_wrap_inline301}%
$I_2$%
\lthtmlinlinemathZ
\lthtmlcheckvsize\clearpage}

{\newpage\clearpage
\lthtmlinlinemathA{tex2html_wrap_inline303}%
$V_1$%
\lthtmlinlinemathZ
\lthtmlcheckvsize\clearpage}

{\newpage\clearpage
\lthtmlinlinemathA{tex2html_wrap_inline305}%
$V_2$%
\lthtmlinlinemathZ
\lthtmlcheckvsize\clearpage}

{\newpage\clearpage
\lthtmldisplayA{displaymath194}%
\begin{displaymath} {\bf Z}={\bf Y}^{-1}=\left[ \begin{array}{rr}-4 & 3 \\3 & -2\end{array} \right]^{-1}
    =\left[ \begin{array}{rr}2 & 3 \\3 & 4\end{array} \right] \end{displaymath}%
\lthtmldisplayZ
\lthtmlcheckvsize\clearpage}

{\newpage\clearpage
\lthtmldisplayA{displaymath195}%
\begin{displaymath} \left\{ \begin{array}{l} V_1=2I_1+3I_2 \\V_2=3I_1+4I_2 \end{array} \right. \end{displaymath}%
\lthtmldisplayZ
\lthtmlcheckvsize\clearpage}

{\newpage\clearpage
\lthtmldisplayA{displaymath196}%
\begin{displaymath} \left\{ \begin{array}{l} V_1=V_0-R_0I_1 \\V_2=-Z_L I_2 \end{array} \right. \end{displaymath}%
\lthtmldisplayZ
\lthtmlcheckvsize\clearpage}

{\newpage\clearpage
\lthtmldisplayA{displaymath197}%
\begin{displaymath} \left\{ \begin{array}{l} 2I_1+3I_2=V_0-R_0I_1\\3I_1+4I_2=-Z_LI_2\end{array}\right. 
      \;\;\;\;\;\mbox{i.e.}\;\;\;\;\;
    \left\{ \begin{array}{l} (2+R_0)I_1+3I_2=V_0\\3I_1+(4+Z_L)I_2=0\end{array}\right. \end{displaymath}%
\lthtmldisplayZ
\lthtmlcheckvsize\clearpage}

{\newpage\clearpage
\lthtmldisplayA{displaymath198}%
\begin{displaymath} I_1=\frac{5(4+j)}{3(1+j)},\;\;\;\;\;\;I_2=-\frac{5}{1+j}\end{displaymath}%
\lthtmldisplayZ
\lthtmlcheckvsize\clearpage}

{\newpage\clearpage
\lthtmldisplayA{displaymath199}%
\begin{displaymath} V_2=-I_2Z+L=\frac{j5}{1+j},\;\;\;\;\;\; V_1=-\frac{5(1-j2)}{3(1+j)} \end{displaymath}%
\lthtmldisplayZ
\lthtmlcheckvsize\clearpage}

{\newpage\clearpage
\lthtmlinlinemathA{tex2html_wrap_inline317}%
$R_L$%
\lthtmlinlinemathZ
\lthtmlcheckvsize\clearpage}

{\newpage\clearpage
\lthtmlinlinemathA{tex2html_wrap_inline319}%
$Z_{Th}$%
\lthtmlinlinemathZ
\lthtmlcheckvsize\clearpage}

{\newpage\clearpage
\lthtmlinlinemathA{tex2html_wrap_inline321}%
$V_1=V_0-R_0I_1=-I_1$%
\lthtmlinlinemathZ
\lthtmlcheckvsize\clearpage}

{\newpage\clearpage
\lthtmlinlinemathA{tex2html_wrap_inline323}%
$V_1=2I_1+3I_2$%
\lthtmlinlinemathZ
\lthtmlcheckvsize\clearpage}

{\newpage\clearpage
\lthtmlinlinemathA{tex2html_wrap_inline325}%
$ 2I_1+3I_2=-I_1$%
\lthtmlinlinemathZ
\lthtmlcheckvsize\clearpage}

{\newpage\clearpage
\lthtmlinlinemathA{tex2html_wrap_inline327}%
$I_1=-I_2$%
\lthtmlinlinemathZ
\lthtmlcheckvsize\clearpage}

{\newpage\clearpage
\lthtmlinlinemathA{tex2html_wrap_inline331}%
$V_2=3I_1+4I_2$%
\lthtmlinlinemathZ
\lthtmlcheckvsize\clearpage}

{\newpage\clearpage
\lthtmlinlinemathA{tex2html_wrap_inline333}%
$ V_2=I_2$%
\lthtmlinlinemathZ
\lthtmlcheckvsize\clearpage}

{\newpage\clearpage
\lthtmlinlinemathA{tex2html_wrap_inline335}%
$Z_{Th}=V_2/I_2=1$%
\lthtmlinlinemathZ
\lthtmlcheckvsize\clearpage}

{\newpage\clearpage
\lthtmlinlinemathA{tex2html_wrap_inline337}%
$V_{Th}$%
\lthtmlinlinemathZ
\lthtmlcheckvsize\clearpage}

{\newpage\clearpage
\lthtmlinlinemathA{tex2html_wrap_inline343}%
$V_1=2I_1$%
\lthtmlinlinemathZ
\lthtmlcheckvsize\clearpage}

{\newpage\clearpage
\lthtmlinlinemathA{tex2html_wrap_inline345}%
$V_2=3I_1$%
\lthtmlinlinemathZ
\lthtmlcheckvsize\clearpage}

{\newpage\clearpage
\lthtmlinlinemathA{tex2html_wrap_inline347}%
$I_1=(V_0-V_1)/R_0=5-V_1=5-2I_1$%
\lthtmlinlinemathZ
\lthtmlcheckvsize\clearpage}

{\newpage\clearpage
\lthtmlinlinemathA{tex2html_wrap_inline349}%
$I_1=5/3$%
\lthtmlinlinemathZ
\lthtmlcheckvsize\clearpage}

{\newpage\clearpage
\lthtmlinlinemathA{tex2html_wrap_inline351}%
$V_{Th}=V_2=3I_1=5$%
\lthtmlinlinemathZ
\lthtmlcheckvsize\clearpage}

{\newpage\clearpage
\lthtmldisplayA{displaymath202}%
\begin{displaymath}V_2=V_{Th}\;\frac{Z_L}{Z_{Th}+Z_L}=\frac{j5}{1+j} \end{displaymath}%
\lthtmldisplayZ
\lthtmlcheckvsize\clearpage}


\end{document}
