\documentstyle[11pt]{article}
\usepackage{html}
\begin{document}
\begin{center}
{\Large \bf E84 Home Work 4}
\end{center}
\begin{enumerate}

\begin{enumerate}

\item In the voltage divider circuit below, $V_0=20V$, $R_1=R_2=500\Omega$. 
Use Thevenin's theorem to find the current through and voltage across the 
load resistor $R_L$ when it is $100\Omega$, $200\Omega$, $300\Omega$, 
respectively.

\htmladdimg{../voltagedivider.gif}

% $R_T=250$, $V_T=10V$
% When $R_L=100$, $I_L=V_T/(R_T+R_L)=1/35$, $V_L=I_L R_L=20/7$
% When $R_L=200$, $I_L=V_T/(R_T+R_L)=1/45$, $V_L=I_L R_L=40/9$
% When $R_L=300$, $I_L=V_T/(R_T+R_L)=1/55$, $V_L=I_L R_L=60/11$

\item Using Thevenin's theorem to determine the current in the $3-\Omega$ 
resistor of the following figure.

\htmladdimg{../hw2c.gif}

% move two voltage sources to left, and 3-$\Omega$ resistor to the right as load
% find equivalent voltage Vo and internal resistance and Ro:
% current going clockwise around the loop (without load): (30-24)/(6+12)=1/3
% voltages across 6 ohm resistor and 12 ohm resistor are -2V and 4V, Vo=30-2=24+4=28V
% Ro=6//12=4ohm, current through load resistor is 28/(4+3)=7A

\item Find voltage across and the current through the 10-$\Omega$ resistor.

\htmladdimg{../hw2d.gif}

% use superposition principle. When 24V is acting alone with 1A open, 
% parallel resistors 15 and 10 become 15//10=6, V'ab=24 x 6/(6+6)=12V
% I'=12/10=1.2A. When 1A is acting alone with 24V closed, parallel resistors
% 6 and 15 become 90/21, I''=10/(10+90/21)=3/10=0.3 A, V''ab=I'' x 10=3V
% overall V=V'+V''=12+3=15, I=I'+I''=1.2+0.3=1.5A


\item 

Find all currents in the diagram in which V=120V, $R_1=2\Omega$, $R_2=20\Omega$. 
{\bf Hint:} it is very hard to solve the problem by finding the currents in the
order of $I_1$, $I_3$, $I_5$, as computing the resistances of the resistor 
network is tedious. However, it is much more straight forward to find the 
currents in the order of $I_5$, $I_3$, $I_1$, if you assume $I_5$ is known, e.g.,
$I_5=1A$. However, the voltage for the voltage source obtained based on this 
assumption is of course not as given (120V). In this case, the linearity property
$F(ax+by)=aF(x)+bF(y)$ can be applied. In particular, given $y=F(x)$, then 
$ay=F(ax)=aF(x)$. Use this relationship to find the actual values of the 
currents.

\htmladdimg{../hw2b.gif}

% use node c as reference (ground), assume $I_5=1$, then $V_b=(20+2)=22V$
% $I_4=22/20=1.1A$, $I_3=I_4+I_5=1.1+1=2.1A$, $V_a=2.1\times 2+V_b=4.2+22=26.2V$
% $I_2=V_a/20=26.2/20=1.31A$, $I_1=I_2+I_3=1.31+2.1=3.41A$, 
% $V_0=I_1\times 2+V_a=2\times 3.41+26.2=33.02$.
% But the given voltage source is 120V, all currents should be scaled up by
% a factor $120/33.02=3.634$, $I_1=12.39$, $I_2=4.76$, $I_3=7.63$, $I_4=4.00$,
% $I_5=3.63$

\item In the circuit below, $R_1=1\Omega$, $V_2=2V$, $R_3=1\Omega$, $R_4=3\Omega$
$I_5=5A$, $V_6=2V$, and $R_6=1\Omega$. Find
\begin{enumerate}
\item current through voltage source $V_2$. 
\item current through resistor $R_3$
\end{enumerate}
(Hint: consider superposition theorem.)

\htmladdimg{../hw3a.gif}

% consider each of the three sources alone:
% V2 alone (I5 open, V6 short): I'  =V2/[R1+R3//(R3+R6)]=2/(1+1//4)=2/(1+4/5)=10/9 (left)
% V6 alone (I5 open, V2 short): I'' =0.5 x V6/(R6+R4+R1//R3)=1/(1+3+0.5)=1/4.5=2/9 (left)
% I5 alone (V2, V6 both short): 
% current thru R6: I5xR4/[R4+(R6+R1//R3)]=5x3/4.5=10/3
% current thru R4: I5x(R6+R1//R3)/[R4+(R6+R1//R3)]=5x1.5/4.5=5/3
% current thru R1 is (half of that thru R6): 5/3
% I'''=current thru R4 + current thru R1=5/3+5/3=10/3 (right)
% current thru V2: I'+I''+I'''=10/9+2/9-10/3=-2 (left) or 2 (right)

% currents thru R3 due to V2 and V6 respectively are the same as that thru V2
% currents thru R3 due to I5 is (half of current thru R6 above): 5/3 (left)
% total current thru R3: 10/9+2/9+5/3=3 (left) 

% The sum of two currents is equal to I5: 2+3=5

\end{enumerate}
\end{document}

\item The circuit in the figure below has a voltage source $V_0=20V$ and a current 
source $I_0=3A$, and four resistors $R_1=20\Omega$, $R_2=10\Omega$, $R_3=30\Omega$ 
and $R_4=10\Omega$. Find the voltage across $R_4$ and the power dissipated by it.

\htmladdimg{../hw2a.gif}

%If voltage source acts alone (current source open), according to voltage divider,
%V'_4=V0 R4/(R2+R2) = 10V, power consumption is W'=V^2/R=10^2/10=10W
%If current source acts alone (voltage source short), according to current divider
%V''_4=(I0 R2/(R22+R4) R4=[3x10/(10+10)x10=15V, power consumption is W''=V^2/R=15^2/10=22.5W
%Total V_4=V'_4+V''_4=10+15=25V, total power W=25^2/10=62.5W not equal to 10+22.5=32.5




\item In the circuit below, $V_0=18V$, $R_1=R_2-3\Omega$, $R_3=6\Omega$,
$R_4=1.5\Omega$. Find the value of current $I$ when $R_5$ is $1\Omega$,
$1\Omega$, and $1\Omega$. Moreover, find the value for $R_5$ for the
desired current $I=0.5A$.

{\bf Hint:} Single out $R_5$ as the load of a network composed of all other
resistors $R_1$, $R_2$, $R_3$, $R_4$ and the voltage source $V_0=18V$, then
apply Thevenin's theorem to find the open-circuit voltage $V_T$ in series
with the internal resistance $R_T$.

% the open-circuit voltage $V_T=V_{ab}$ is the difference between nodes 
% a and b: V_{oc}=V_0 R_3/(R_1+R_3)-V_0 R_4/(R_2+R_4)=18(6/9-1.5/4.5)=6V
% the internal resistance between nodes a and b (with V_0 short circuit)
% is R_1//R_3+R_2//R_4=3x6/9+3x1.5/4.5=3. 
% R_5=1, I=6/(3+1)=1.5, R_5=2, I=6/(3+2)=1.2, R_5=3, I=6/(3+3)=1,
% I=0.5, R_5=V_T/I-R_T=6/0.5-3=9



\htmladdimg{../hw2i.gif}
