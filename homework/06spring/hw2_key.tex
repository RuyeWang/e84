\documentstyle[11pt]{article}
\usepackage{html}
\begin{document}
\begin{center}
{\Large \bf E84 Home Work 2}
\end{center}
\begin{enumerate}

The calculations of all problems in this problem set are straight forward.
However, the concepts involving source, load, internal (output) impedance
of the source and the input impedance of the load are very important. After
finding the numerical solutions of the problem, pause and reflect what they
mean. 

\item Find the equivalent resistance between the two terminals before and
after the switch is closed. (Note, the two diagonal branches are NOT
connected to each other in the middle.)

\htmladdimg{../hw2e.gif}

{\bf Solution:}
 before S is closed, $R=(3+2)//(6+2)=5//8=40/13 \Omega$

 after S is closed, $R=2//2+3//6=1+2=3 \Omega$

\item (a) In reality, an ammeter can be modeled by an ideal meter with zero impedance 
  in series with an internal impedance $r_a$. To minimize the influnce of the ammeter
  on the circuit during the measurement, should $r_a$ be minimized or maximized?

  (b) A voltmeter can be modeled by an ideal meter with infinite impedance in parallel
  with an internal (or input) impedance $r_v$. To minimize the influnce of the voltmeter 
  on the circuit during the measurement, should $r_v$ be minimized or maximized?

  (c) Calculate current $I$ and voltage $V$ in the simple circuit below, assuming 
  $V_0=10V$, $R=500\Omega$ and the measuring meters are not connected, i.e., the ammeter
  is short circuit, the voltmeter is open circuit.

  (d) What is the reading on the ammeter with $r_a=10\Omega$ inserted in the circuit 
  as shown? Assume the voltmeter is not connected.

  (e) What is the reading on the voltmeter with $r_v=10^4 \Omega = 10 K\Omega$, connected
  to the circuit as shown? Assume the ammeter is not connected.

  \htmladdimg{../meters.gif}

  {\bf Solution:} 

  (a) $r_a$ should be minimized, ideally, zero.

  (b) $r_v$ should be maximized, ideally, infinity.

  (c) \[ I=10V/(500+500)\Omega=10 mA,\;\;\;\;V=10V\frac{500}{500+500}=5 V \]

  (d) \[ I=10V/(500+500+10)\Omega=9.9 mA \]

  (e) \[ V=10V\frac{500||10000}{500||10000+500}=4.878 V \]

\item Ideally, a voltage source, such as a battery, should output constant voltage,
  independent of how much current will be drawn by its load, e.g., a resistor. However, 
  in reality, all voltage sources have some internal (output) impedance, modeled by an 
  ideal voltage source $V_0$ in series with an impedance $R_0$. Assume $V_0=10V$ and 
  $R_0=10\Omega$ in the following circuit, find the voltage $V_L$ the load resistor 
  $R_L$ gets when (a) $R_L=100 \Omega$ and then (b) $R_L=10K\Omage$. 
  (c) In general, if you want the load to get maximal voltage from the source, do you 
  want $R_L$ and $R_0$ to be minimized or maximized, individually?

  \htmladdimg{../source_load.gif}

  {\bf Solution:}

  (a) When $R_L=100 \Omega$, 
  \[V_L=V_0 \frac{R_L}{R_L+r_0}=10 \frac{100}{100+10}=9 V \]

  (b) $R_L=10K\Omage$.
  \[V_L=V_0 \frac{R_L}{R_L+r_0}=10 \frac{10000}{10000+10}=9.99 V \]

  (c) To maximize $V_L$, we want maximize $R_L$ and minimize $R_0$.

\item The circuit below with $R=10 \Omega$ can be used to measure the internal 
  impedance $R_0$ of a battery.
  \begin{itemize}
    \item When the battery is fresh, it is found $V_{out}=1.6V$ when the switch is open 
      and $V_{out}=1.4V$ when the switch is closed. Find the internal impedance $R_0$.
    \item The same battery is tested one year later, and $V_{out}=1.2V$ when the switch 
      is open and $V_{out}=0.3V$ when the switch is closed. Find the internal impedance 
      $R_0$.
  \end{itemize}
(Hint: assume the input impedance of the voltmeter is infinity, i.e., the voltmeter
  does not draw any current.)

  \htmladdimg{../battery_impedance.gif}

{\bf Solution:}
\[ V_{out}=1.6\times \frac{10}{10+R_0}=1.5, \longrightarrow R_0=1.43 \Omega \]
\[ V_{out}=1.2\times \frac{10}{10+R_0}=0.3, \longrightarrow R_0=30 \Omega \]

\item The circuit below can be used to find the input impedance $r_v=R_{in}$ of
  a voltmeter (or an oscilloscope that also measures voltages). Assume $V_0=9V$
  and $R_1=R_2=10 M\Omega$, and the reading of the voltmeter is $V_L=3V$. 
  What is the input impedance of the voltmeter?
  
  \htmladdimg{../meter_impedance.gif}

  {\bf Solution:}

  \[ V_L=V_0\frac{R_2||R_{in}}{R_2||R_{in}+R_1}
  =9 \frac{R_2 || R_{in}}{R_2 || R_{in}+R_1}=3 \]
  Solving this equation for $R_{in}$ we get $R_{in}=10 M\Omega$.


\end{enumerate}

\end{document}

\item Assume in the circuits in the figure, the voltage source is $V_0=2V$,
	the current source is $I_0=1A$, and the resistor is $R=3\Omega$.
	For both (a) and (b).

\htmladdimg{../hw1d.gif}

Find:
\begin{enumerate}
\item the current going through each of the three elements

{\bf Solution:}
\begin{itemize}
\item (a) I=1A going thru all 3 elements 
\item (b) 1A thru current source, 2/3A thru R, 1/3 thru voltage source.
\end{itemize}

\item the voltage across each of the three elements

{\bf Solution:}
\begin{itemize}
\item (a) 2V across voltage source, 3V across R, 5V across current source
\item (b) 2V across all 3 elements.
\end{itemize}

\item which of the power sources is delivering power and how much? which is 
	receiving power and how much? How much power is dissipated by the
	resistor $R$?

{\bf Solution:}
\begin{itemize}
\item (a) current source delivers 5W, R dissipates 3W, voltage source 
	receives 2W
\item (b) current source delivers 2W, R dissipates 4/3W, voltage source 
	receives 2/3W 
\end{itemize}
When receiving power, the voltage source is being charged.
\end{enumerate}

\end{enumerate}
\end{document}


\item Find the current I3 through the load resistor of 6$\Omega$ by the
following two methods:
\begin{enumerate}
\item Convert the two voltage sources to equivalent current sources, 
	simplify the circuit and solve for the current;
\item Use superposition theorem to consider one of the two sources at a
  time, then add the two partial results to get the final result.

{\bf Hint: Superposition theorem}

When there exist multiple energy sources, the currents and voltages in 
the circuit can be found as the algebraic sum of the corresponding values 
obtained by assuming only one source at a time, with all other sources 
turned off (voltage sources treated as short circuit, current sources 
treated as open circuit).

\end{enumerate}

\htmladdimg{../hw2g.gif}

% convert first voltage source (left):   I1=10V/1 ohm=10A, R1=1
% convert second voltage source (right): I2=6V/3 ohm=2A, R2=3
% combine two parallel current sources: I0=I1+I2=12A
% combine two parallel resistors: R0=(R1 x R2)/(R1+R2)=3/4=0.75
% use current divider to find current:
%       I=I0 R0/(R+R0)=12 x 0.75/(6+0.75)=4/3=1.33A

\item Find the three currents labeled as I1, I2 and I3 in the same figure 
	above by the following method:
\begin{enumerate}
\item Apply KCL to node a to get a current equation;
\item Apply KVL to the two loops to get two voltage equations;
\item Solve the three equations for the unknown currents I1, I2 and I3.
\end{enumerate}

% I1+I2-I3=0, 10V-I1=6V+3I2, 10V-I1=6I3
% solve to get: I1=2, I2=-2/3, I3=4/3

\item Find all currents labeled $I_k (k=1,2,3,4,5)$ in the circuit below. 
The resistance of a resistor labeled by 10K, for example, is 10,000
Ohms ($10,000 \Omega=10K\Omega$).

{\bf Hint:} follow these steps:
\begin{itemize}
\item Choose node 0 as the reference point or ``ground'', and let the 
	voltages at nodes 1 and 2 be $V_1$ and $V_2$ (with respect to
	node 0), respectively. 
\item Express all currents in terms of the node voltages $V_1$, $V_2$
	and other known quantaties such as the current and voltage sources
	and the resistors.
\item Apply KCL to node 1 and node 2 to get two current equations, 
	substitute all current expressions into the two equations.
\item Solve the two equations for $V_1$ and $V_2$.
\item Find all currents.
\end{itemize}

\htmladdimg{../hw2h.gif}

% KCL equations: I1+I2+I3-I0=0, -I3+I4+I5=0
% I1=V1/R1, I2=V1/R2, I3=(V1-V2)/R3, I4=V2/R4, I5=(V2+V0)/R5
% substitute I1 through I5 into KCL equations and solve for V1 and V2 
% to get: V1=21.8V, V2=-21.8V
% obtain all currents: I1=1.09, I2=0.545, I3=4.36, I4=-1.09, I5=5.46

\item Find all currents I1 through I5 in the circuit below. 

{\bf Hint:} use the loop current method: 

\begin{itemize}
\item Assume the currents around the four loops are Ia, Ib, Ic and Id 
  with Ia=10A and Id=6.5A (due to the two current sources).
\item Apply KVL to the two middle loops to get two equations.
\item solve them for two unknown currents Ib and Ic, get I1 through I5.
\end{itemize}

\htmladdimg{../hw2i.gif}

% KVL: (2+3+4)Ib-2Ia-4Ic=-10, (4+3+2)Ic-4Ib-2Id=10
% solve to get Ib=2.8A, Ic=3.8A, then find I1 thru I5:
% I1=Ia-Ib=7.2A, I2=Ib-Ic=-1A, I3=Ic-Id=-2.7A, I4=Ib=2.8A, I5=Ic=3.8A


\item Find the optimal load resistance $R_L$ so that it receives maximal
power from the current source $I_0=10A$ with internal resistance 
$R_0=1\Omega$ and power transmission line resistance $R_T=9\Omega$. 
Find the maximum load power and the power loss along the transmission 
line.

\htmladdimg{../hw2f.gif}

To verify your choice of load resistance, show that the power consumption
of the load will always be lower than this maximum when its resistance is 
either increased or decreased by ten percent.

% First convert current source to voltage source with $V_0=I_0 R_0=10V$
% and $R_0=1 \Omega$. To maximize load power consumption, let 
% $R_L=R_0+R_T=10 \Omega$. The current is $I=10V/20\Omega=0.5A$. Load power 
% is $I^2R_L=10/4=2.5W$, power loss on transmission line is $I^2R_T=9/4A$
% When $R_L=11\Omega$, $I=10V/21\Omega$, $W_L=I^2 R_L=2.494$
% When $R_L=9\Omega$, $I=10V/19\Omega$, $W_L=I^2 R_L=2.493$




\end{enumerate}
\end{document}


\item Using Thevenin's theorem to determine the current in the $3-\Omega$ 
resistor of the following figure.

\htmladdimg{../hw2c.gif}

% move two voltage sources to left, and 3-$\Omega$ resistor to the right as load
% find equivalent voltage Vo and internal resistance and Ro:
% current going clockwise around the loop (without load): (30-24)/(6+12)=1/3
% voltages across 6 ohm resistor and 12 ohm resistor are -2V and 4V, Vo=30-2=24+4=28V
% Ro=6//12=4ohm, current through load resistor is 28/(4+3)=7A

\item Find voltage across and the current through the 10-$\Omega$ resistor.

\htmladdimg{../hw2d.gif}

% use superposition principle. When 24V is acting alone with 1A open, 
% parallel resistors 15 and 10 become 15//10=6, V'ab=24 x 6/(6+6)=12V
% I'=12/10=1.2A. When 1A is acting alone with 24V closed, parallel resistors
% 6 and 15 become 90/21, I''=10/(10+90/21)=3/10=0.3 A, V''ab=I'' x 10=3V
% overall V=V'+V''=12+3=15, I=I'+I''=1.2+0.3=1.5A


\item The circuit in the figure below has a voltage source $V_0=20V$ and a current 
source $I_0=3A$, and four resistors $R_1=20\Omega$, $R_2=10\Omega$, $R_3=30\Omega$ 
and $R_4=10\Omega$. Find the voltage across $R_4$ and the power dissipated by it.

\htmladdimg{../hw2a.gif}

%If voltage source acts alone (current source open), according to voltage divider,
%V'_4=V0 R4/(R2+R2) = 10V, power consumption is W'=V^2/R=10^2/10=10W
%If current source acts alone (voltage source short), according to current divider
%V''_4=(I0 R2/(R22+R4) R4=[3x10/(10+10)x10=15V, power consumption is W''=V^2/R=15^2/10=22.5W
%Total V_4=V'_4+V''_4=10+15=25V, total power W=25^2/10=62.5W not equal to 10+22.5=32.5

\item 

Find all currents in the diagram in which V=120V, $R_1=2\Omega$, $R_2=20\Omega$. 
{\bf Hint:} it is very hard to solve the problem by finding the currents in the
order of $I_1$, $I_3$, $I_5$, as computing the resistances of the resistor 
network is tedious. However, it is much more straight forward to find the 
currents in the order of $I_5$, $I_3$, $I_1$, if you assume $I_5$ is known, e.g.,
$I_5=1A$. However, the voltage for the voltage source obtained based on this 
assumption is of course not as given (120V). In this case, the linearity property
$F(ax+by)=aF(x)+bF(y)$ can be applied. In particular, given $y=F(x)$, then 
$ay=F(ax)=aF(x)$. Use this relationship to find the actual values of the 
currents.

\htmladdimg{../hw2b.gif}





\end{enumerate}
\end{document}

