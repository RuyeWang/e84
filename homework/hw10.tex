\documentstyle[11pt]{article}
\usepackage{html}
\begin{document}
\begin{center}
{\Large \bf  E84 Home Work 10}
\end{center}

\begin{enumerate}

\item In the circuit shown below, the two base resistors $R_B=4.3K\Omega$,
the collector $R_C=1K\Omega$. Assume the two transistors have the same 
$\beta=100$ value and they each receive an input voltage ($V_1$ and $V_2$)
at either 0.2V or 5V. Find the output voltage $V_{out}$ for the following 
combinations of the inputs. (Hint: 5V input to a transistor will drive it 
to saturation.)

\begin{tabular}{cc|c}\hline
$V_1$ & $V_2$ & $V_{out}$ \\ \hline
 0.2  &  0.2  &           \\ \hline
 5.0  &  0.2  &           \\ \hline
 0.2  &  5.0  &           \\ \hline
 5.0  &  5.0  &           \\ \hline
\end{tabular}

\htmladdimg{../hw9d.gif}

% {\bf Solution:} When $V_1=0.2V$, $T_1$ is cut-off. When $V_1=5V$, 
% $I_{b1}=(5-0.7)/4.3=1mA$, $T_1$ is saturated with $V_{CE}=0.2V$ and 
% $I_C=(5-0.2)/1=4.8 mA$ (instead of $I_C=\beta I_B=100 mA$). The same is
% true for $T_2$. From the table below we see that the circuit is a NAND 
% (not AND) gate:
% 
% \begin{tabular}{cc|c}\hline
% $V_1$ & $V_2$ & $V_{out}$ \\ \hline
%  0.2  &  0.2  &    5      \\ \hline
%  5.0  &  0.2  &    0.2    \\ \hline
%  0.2  &  5.0  &    0.2    \\ \hline
%  5.0  &  5.0  &    0.2    \\ \hline
% \end{tabular}

\item A BJT transistor with $\beta=24$ is set up as a common-base 
configuration as shown in the figure below. 
\htmladdimg{../../lectures/figures/CB.gif}

\begin{itemize} 
\item If it is known that $I_{CB0}=10 nA$ and $I_E=2 mA$, find $I_C$ 
  and $I_B$.
\item If $V_{CB}=15V$ and $V_{EB}=0.7$, estimate $I_E$, $I_B$ and $I_C$ 
  based on the plots below. Re-estimate these currents if $V_{EB}$ is 
  increased to $0.8V$.

  Note: in the figure above, as the assumed polarities of both $I_E$ and 
  $V_{EB}$ are the opposite to those assumed in plot (b) below, the negative
  signs for both $I_E$ and $V_{EB}$ in the plot should be dropped.

\end{itemize}

\htmladdimg{../../lectures/figures/transistorCBplots.gif}

% {\bf Solution:} $\alpha=\beta/(1+\beta)=24/25=0.96$, 
% $I_C=I_E \alpha + I_{CB0}=2 mA \times 1.92+0.01=1.921 mA$,
% $I_B=I_C/\beta=1.92/24=0.08$, or $I_B=I_E-I_C=2-1.96=0.08 mA$.
% 
% If $V_{EB}=0.7 V$ and $V_{CB}=15V$, $I_E=5 mA$ can be estimated from 
% figure (b), and correspondingly, $I_C=4.8 mA$ and $I_B=0.2 mA$. When 
% $V_{EB}=0.8 V$, $I_E=20 mA$, $I_C=19.2 mA$, and $I_B=0.8 mA$.

\item In the figure below, the transistor with $\alpha=0.99$ and
  $I_{CB0}=10^{-11} A$ is set up as a common-emitter circuit. The
  base-emitter pn-junction is forward biased with $I_B=20 \mu A$, 
  $V_{CC}=10V$, and $R_C=2k\Omega$. Find $I_{CE0}$, $I_C$, $I_E$, 
  $V_{CE}$, and $V_{CB}$. Is the collector-base pn-junction forward
  or reverse biased? (Assume the voltage across a forward biased 
  pn-junction is $0.7 V$.)
  
\htmladdimg{../CEexample.gif}

% {\bf Solution:} 
% \[ \beta=\frac{\alpha}{1-\alpha}=99 \]
% \[ I_{CE0}=(1+\beta) I_{CB0}=(1+99) 10^{-11}=10^{-9} \]
% \[ I_C=\beta I_B+I_{CE0}=99 \times 2 \times 10^{-5} + 10^{-9}=1.98 mA \]
% \[ I_E=I_C+I_B=1.98+0.02=2 mA \]
% \[ V_{CE}=V_{CC}-R_C I_C=10-2 \times 10^3 \times 1.98 \times 10^{-3}
%    \approx 10-4=6 V \]
% Since the base-emitter pn-junction is forward biased with $V_{BE}=0.7V$,
% we have $V_{CB}=V_{CE}-V_{BE}=6-0.7=5.3 V$, the collector-base pn-junction
% is reverse biased.

\item The figure (A) below shows a common-emitter transistor 
applification circuit (silicon) with$\beta=100$, $V_{cc}=20V$, 
and $R_C=2\;K\Omega$. 
\begin{itemize}
\item The base current is $i_b(t)=(50+30 \cos \omega t)\;\mu A$.
Find $i_c(t)$ and $v_c(t)$ as functions of time.
\item Repeat the above for $i_b(t)=(90+30 \cos \omega t)\;\mu A$.
\item Repeat the above for $i_b(t)=(10+30 \cos \omega t)\;\mu A$.
\end{itemize}
For each of the three cases above, sketch the output (collector) 
characteristics ($i_c$ vs $v_c$) as shown in class show the wave
forms of $i_b(t)$, $i_c(t)$ and $v_c(t)$, following the example
in the 
\htmladdnormallink{lecture notes}{http://fourier.eng.hmc.edu/e84/lectures/ch4/node5.html}.

{\bf Note:} As the convention in the schematics of transistor circuits,
the bottom horizontal line is treated as the ground, and all voltages,
such as $V_b$, $V_c$ and $V_e$ are measured with respect to the 
ground as the reference point.

{\bf Hint:} The relationship $I_C=\beta I_B$ is only valid in the
linear region in the middle range of the load line. However, in 
the cut-off region (close to the horizontal axis) and the saturation
region (close to the vertical axis), the above relationship no
longer holds and the actual output current $I_c$ and $V_{ce}$ can
only be found graphically in the output characteristic plot.

\htmladdimg{../hw9f.gif}


% {\bf Solution:}
% \[ i_c(t)=\beta i_b(t)=(5+3\cos \omega t)\;mA \]
% \[ v_c(t)=V_{cc}-R_c i_c(t)=20-2000 (5+3\cos \omega t)\times 10^{-3}
%   =10-6\cos \omega t  \]
% \[ i_e(t)=(\beta+1) i_b(t)\approx i_c(t)=(5+3\cos \omega t)\;mA \]
% \[ v_e(t)=R_e i_e(t)=2000 (5+3\cos \omega t)\times 10^{-3}
%    =10+6\cos \omega t  \]

\item Now assume an emittor resistor $R_e=2\;K\Omega$ is added between 
emittor of the transistor and ground in the circuit above (figure (B)
above).  Find the voltages $v_e(t)$ as well as $v_c(t)$ when the base
current is $i_b(t)=(50+30 \cos \omega t)\;\mu A$. Sketch the waveformes
of the two voltages.

% {\bf Solution:}
% \begin{itemize}
% \item 
% \[ i_c(t)=\beta i_b(t)=(5+3\cos \omega t)\;mA \]
% \[ v_c(t)=V_{cc}-R_c i_c(t)=20-2000 (5+3\cos \omega t)\times 10^{-3}
%    =10-6\cos \omega t  \]
% \item
% \[ i_c(t)=\beta i_b(t)=(1+3\cos \omega t)\;mA \]
% \[ v_c(t)=V_{cc}-R_c i_c(t)=20-2000 (9+3\cos \omega t)\times 10^{-3}
%    =2-6\cos \omega t  \]
% Clipping happens during the negative half-cycle due to saturation.
% \item
% \[ i_c(t)=\beta i_b(t)=(10+3\cos \omega t)\;mA \]
% \[ v_c(t)=V_{cc}-R_c i_c(t)=20-2000 (1+3\cos \omega t)\times 10^{-3}
%    =18-6\cos \omega t  \]
% Clipping happens during the positive half-cycle due to cut-off.
% \end{itemize}

\end{enumerate}

\end{document}




\item Find values of $R_C$ and $R_B$ in the circuit with $\beta=100$
and $V_{CC}=15V$ so that the Q-point is $I_C=25mA$ and $V_{ce}=7.5V$
What is the Q point if $\beta=200$?

\htmladdimg{../hw9a.gif}


% {\bf Solution:}

% Find $I_B=I_C/\beta=25mA/100=0.25mA$, $R_B=(15=0.7)/0.25=57.2k\Omega$
% Also as $V_{CC}=I_C R_C+V_{ce}$, i.e., $R_C=(V_{CC}-V_{ce})/I_C=
% (15-7.5)/25\times 10^{-3}=300\Omega$. 
% 
% $I_B=(V_{CC}-V_{BD})/R_B=0.25mA$, $I_C=\beta I_B=200\times 0.25=50mA$,
% $V_{ce}=V_{CC}-R_C I_C=15-300\times 50\times 10^{-3}=0$.

\item Design a stable self-biasing transistor circuit with a DC operating
point (Q point) of $I_C=2.5mA$ and $V_{ce}=7.5V$ which should be in the 
middle of the load line. The $\beta$ of the ransistor ranges from 50 to
200. 

{\bf Hint} This is a design problem with possibly multiple solutions, 
i.e., there may be more degrees of freedom than constraining conditions. 
One of such conditions is $(\beta+1)R_E \gg R_B$ for the DC operating 
point to be approximately independent of $\beta$ (see online notes). 

\htmladdimg{../hw9b.gif}

% {\bf Solution:}
% 
% \begin{itemize}
% \item Find $V_{CC}$: We want the Q-point to be in the middle of the
%   load line, so we set $V_{CC}=2V_{ce}=2\times 7.5=15V$.
% \item Find $R_C$ and $R_E$: As $V_{ce}=V_{CC}-I_CR_C-I_ER_E\approx
%   V_{CC}-I_C(R_C+R_E)$, we have $R_C+R_E=7.5/2.5\times 10^{-3}=3K\Omega$.
%   Choose $R_E=1K\Omega$ and $R_C=2K\Omega$.
% \item Find $R_B$: To satisfy $R_B \ll \beta_{min} R_E$, we
% 	let $R_B=0.1\times \beta_{min} R_E=0.1\times 50\times 1000=5\;K\Omega$
% \item Find $V_{BB}$: 
% 	\[ V_{BB}=V_{be}+I_ER_E=0.7+2.5\times 10^{-3}\times 10^3=3.2V \]
% \item Find $R_1$ and $R_2$:
% \[	R_B=\frac{R_1R_2}{R_1+R_2}=5\;K\Omega \;\;\;\;\;\;\;\;
% 	\frac{V_{BB}}{V_{CC}}=\frac{R_2}{R_1+R_2}=\frac{3.2}{15}=0.21	\]
% Solve these two equations (first divide the first equation by the second), 
% we obtain the two unknowns $R_1$ and $R_2$:
% \[	R_1=\frac{5\;K\Omega}{0.21}=24\;K\Omega \;\;\;\;\;\;\;\;\;
% 	R_2=6.4\; K\Omega	\]
% \end{itemize}

\item In the AC amplifier shown in the figure, $R_S=600\Omega$, $R_1=30K\Omega$, 
$R_2=20K\Omega$, $R_E=4K\Omega$, $R_C=3K\Omega$, $R_L=5.1K\Omega$, 
$\beta=100$, $V_{CC}=15V$. 
%Also assume $r_{be}=200\Omega$. 
Assume the 
frequency of the AC signals to be amplified is high enough so that the
coupling capacitors and the emitter by-pass capacitor can be treated as 
AC short circuits. Find
\begin{itemize}
\item The DC operating point in terms of variables $I_B$, $I_C$, $V_C$, $V_E$ 
  and $V_{CE}$.
\item The AC equivalent diagram
\item The AC input and output impedances
\item The voltage gain of the AC amplifier
\end{itemize}

\htmladdimg{../hw9c.gif}

% {\bf Solution:} 
% \begin{itemize}
% \item Find DC operating point.
% \[ V_{BB}=V_{CC}\frac{R_2}{R_1+R_2}=15\frac{20}{30+20}=6V	\]
% \[ R_B=\frac{R_1R_2}{R_1+R_2}=\frac{600}{50}=12K\Omega 	\]
% \[ I_B=\frac{V_{BB}-V_{be}}{(\beta+1)R_E+R_B}
% 	=\frac{6-0.7}{101\times 4000+12000}=\frac{5.3}{26\times 10^4}
% 	=0.025 mA	\]
% \[ I_C=\beta I_B=100\times 0.025 mA=2.5 mA	\]
% \[ V_C=V_{CC}-I_C R_C=15-2.5\times 3=7.5V	\]
% \[ V_E=(\beta+1)I_C R_E=5V	\]
% \[ V_{CE}=V_C-V_E=2.5V \]
% \item Find input and output impedances.
% \[	r_{be}=V_T/I_B=0.026/0.025=1K\Omega	\]
% \[	r_{in}=R_1||R_2||r_{be} \approx 1K\Omega \]
% \[	r_{out}=R_C=3K\Omega	\]

% \item Find voltage gain.
% \[	G=-\beta \frac{(R_1||R_2||r_{be})}{(R_1||R_2||r_{be})+R_s}
% 	\frac{1}{r_{be}}(R_C||R_L) \approx -117	\]
% \end{itemize}

\item An emitter collower circuit is shown in the figure. Assume 
$\beta=50$, $R_B=300K\Omega$, $R_E=4K\Omega$, $R_S=500\Omega$, 
$R_L=5.1K\Omega$, and $V_{CC}=12$. Find the DC operating point in
terms of $I_B$, $I_C$, $I_E$, and $V_E$. Then find the input and
output resistances and the voltage gain.

\htmladdimg{../hw9e.gif}

\item In the circuit shown below, the two base resistors $R_B=43K\Omega$,
the collector $R_C=1K\Omega$. Assume each of the two input voltages $V_1$
and $V_2$ are either 0.2V or 5V. Find the output voltage $V_{out}$ in the
following combinations of the inputs. (Hint: assume 5V input to a transistor
will drive it to saturation.)

\begin{tabular}{cc|c}\hline
$V_1$ & $V_2$ & $V_{out}$ \\ \hline
 0.2  &  0.2  &           \\ \hline
 5.0  &  0.2  &           \\ \hline
 0.2  &  5.0  &           \\ \hline
 5.0  &  5.0  &           \\ \hline
\end{tabular}

\htmladdimg{../hw9d.gif}


\end{enumerate}

\end{document}

\end{enumerate}
