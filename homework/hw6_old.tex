\documentstyle[11pt]{article}
\usepackage{html}
\begin{document}
\begin{center}
{\Large \bf  Home Work 6 ---- E84, Fall, 2004}
\end{center}
\begin{enumerate}

\item A series circuit composed of a capacitor and an inductor is to be 
resonant at 800 kHz with voltage input. Specify the value of $C$ for the 
capacitor required for the given inductor with $L=40\mu H$ and an internal 
resistance $R_L=4.02\Omega$, and predict the bandwidth. Assume the capacitor 
is ideal, i.e., it introduces no resistance.


%{\bf Solution:}
% As $\omega_0=1/\sqrt{LC}=2\pi 8\times 10^5$, and $L=4\times 10^{-5}H$, we
% can find $C$ to be
% \[
% C=\frac{1}{\omega_0^2 L}=\frac{1}{(2\pi 8\times 10^5)^2\times 4\times 10^{-5}}
% =0.99\;nF	\]
% Next find the quality factor:
% \[
% Q=\frac{\omega_0L}{R}=\frac{2\pi 8\times 10^5\times 4\times 10^{-5}}{4.02}=50
% \]
% then the bandwidth is
% \[	f_2-f_2=\frac{f_0}{Q}=\frac{800\times 10^3}{50}=16\;kHz	\]

\item In reality, all inductors have a non-zero resistance, therefore a 
parallel resonance circuit should be modeled as shown in the figure:

\htmladdimg{../../lectures/figures/parallelRCL.gif}

Find the expression of the resonant frequency of this circuit. (Hint,
similar to our discussion of parallel and series RCL circuits in class, 
the resonant frequency $\omega_0$ is achieved when the imaginary part of
the impedance (or admittance) is zero.) If it is given that $R=5\Omega$,
$C=100\;\mu F=100\times 10^{-6}F$ and $L=5\; mH=5\times 10^{-3}H$, what
is the resonant frequency of this circuit? Does the current drawn from 
the voltage source reach maximum or minimum at this frequency?

% {\bf Solution:} The admittance is:
% \[	Y=\frac{1}{R+j\omega L}+j\omega C
% 	=\frac{R-j\omega L+j\omega C(R^2+\omega^2L^2)}{R^2+\omega^2L^2}
% \]
% When the imaginary part of $Y$ is zoro, the circuit is at resonance and 
% the resonant frequency $\omega_0$ can be found by solving $Im[Y]=0$:
% \[	\omega_0 L=\omega_0 C(R^2+\omega_0^2L^2)	\]
% Solving this for $\omega_0$, we get:
% \[	\omega_0=\sqrt{\frac{1}{LC}-(\frac{R}{L})^2}
% 	=\sqrt{\frac{1}{5\times 10^{-3}\times 100\times 10^{-6}}
% 	-(\frac{5}{5\times 10^{-3}})^2}=1000\;Hz
% 	\]
% As the admittance reaches minimum at $\omega=\omega_0$, the current drawn
% from the voltage source $\cdot{I}=Y\cdot{V}$ reaches minimum, i.e., it 
% behaves as a band stop filter.
% 
% Also note that for $\omega_0$ to be real, we must have
% \[	\frac{1}{LC} > (\frac{R}{L})^2, \;\;\;\;\;\;\mbox{i,e,}
% 	\;\;\;\;\;\;R<\sqrt{\frac{L}{C}}	\]
% When $R \ll \sqrt{L/C}$, the resonant frequency is
% \[
% \omega_0=\sqrt{\frac{1}{LC}-(\frac{R}{L})^2}\approx \frac{1}{\sqrt{LC}}	
% \]

\item Find the output voltage $v_{out}(t)$ when $\omega=0$ and 
$\omega\rightarrow \infty$ when the input $v_{in}(t)=10\;cos(\omega t)$, 
assuming $R_1=100\Omega$, $R_2=100\Omega$, $C=10\mu F$ and $L=10\;mH$.

\htmladdimg{../hw6a.gif}

% {\bf Solution}
% 
% When $\omega=0$, the inductor is short circuit, and the capacitor is
% open circuit, $v_{out}(t)=v_{in}(t)$.
% When $\omega=0$, the inductor is open circuit, and the capacitor is
% short circuit, $v_{out}(t)=v_{in}(t)/2=5\;cos(\omega_0 t)$.

\item The function of a loudspeaker crossover network is to channel 
frequencies higher than a given crossover frequency $f_c$ into the
high-frequency speaker (``tweeter'') and frequencies below $f_c$ into
the low-frequency speaker (``woofer''). One such circuit is shown below.
Assume the resistances of the tweeter is $R_1=8\Omega$ and that of the 
woofer is $R_2=8\Omega$, the voltage amplifier can be modeled as an
ideal voltage source, and the crossover frequency is $f_c=2000\; Hz$.
Design the network in terms of $L$ and $C$ so that $f_c$ is the
half-power point of each of the two speaker circuits. Give the expression
of the power $P_1(f)$ and $P_2(f)$ of the speakers as a function of 
frequency $f$ and crossover frequency $f_c$, and sketch them. Assume the
RMS of the input voltage is 1V.

\htmladdimg{../hw6b.gif}

% {\bf Solution}
% 
% The RMS voltage across the tweeter is
% \[	V_1=|\frac{R}{1/j2\pi f C+R}|
% 	=\frac{2\pi f CR}{\sqrt{1+(2\pi f CR)^2}}	\]
% If $f=f_c=2000$ is at half-power point ($V_1=V_{in}/\sqrt{2}$), the real and
% imaginary parts of the denominator should be equal and we get 
% \[	\frac{1}{j2\pi f C}=R;\;\;\;\mbox{i.e.}\;\;\;\;
% 	C=\frac{1}{2\pi f_c R}=9.95 \;\mu F	\]
% The RMS voltage across the woofer is
% \[	V_2=|\frac{R}{j2\pi f L+R}|
% 	=\frac{R}{\sqrt{R^2+(2\pi f L)^2}}	\]
% If $f=f_c=2000$ is at half-power point ($V_2=V_{in}/\sqrt{2}$), the real and
% imaginary parts of the denominator should be equal and we get 
% we get 
% \[	j2\pi f L=R;\;\;\;\mbox{i.e.}\;\;\;\;
% 	L=\frac{R}{2\pi f_c}=0.637 \;mH	\]
% The power plots:
% \[	P_1(f)=\frac{V_1^2(f)}{R}=\frac{1}{8}(\frac{1}{1+(f_c/f)^2}) \]
% \[	P_2(f)=\frac{V_1^2(f)}{R}=\frac{1}{8}(\frac{1}{1+(f/f_c)^2}) \]


\end{enumerate}
\end{document}

\item In the parallel resonant circuit shown in the figure, $L=0.25\;mH$,
$R=25\Omega$, $C=85\;pF$. Find resonant frequency, and the impedance at r
esonance.

\htmladdimg{../hw5c.gif}

\[	Y=\frac{1}{R+j\omega L}+j\omega C=	\]



