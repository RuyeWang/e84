\documentstyle[11pt]{article}
\usepackage{html}
\begin{document}
\begin{center}
{\Large \bf  E84 Home Work 10}
\end{center}
\begin{enumerate}

\item In a control system, the three inputs voltages $v_1$, $v_2$ 
and $v_3$ represent, respectively, temperature, pressure and velocity
of the process under control. According to the control strategy, the
output need to be $v_o=-2v_1-10v_2-0.5v_3$. Design a summer-inverter 
shown below as the controller.  It is usually desirable for both 
inputs of the opamp to have the same resistance to ground, i.e.,
$R_0=R_1||R_2||R_3$. It is given that $R_f=100k\Omega$.

\htmladdimg{../hw10i.gif}

{\bf Solution:}

The output of the circuit is
\[ v_o=-(\frac{R_f}{R_1}v_1+\frac{R_f}{R_2}v_2+\frac{R_f}{R_3}v_3) \]
For the controller, we need to have
\[	\frac{R_f}{R_1}=2,\;\;\;
	\frac{R_f}{R_2}=10,\;\;\;
	\frac{R_f}{R_3}=0.5	\]
i.e., $R_1=100/2=50k\Omega$, $R_2=100/10=10k\Omega$ and
$R_3=100/0.5=200k\Omega$. Moreover, for $R_0=R_1||R_2||R_3$, we need
\[ \frac{1}{R_0}=\frac{1}{R_1}+\frac{1}{R_2}+\frac{1}{R_3}
	=\frac{2+10+0.5}{100},\;\;\;\mbox{i.e.,}
	R_0=\frac{100}{12.5}=8	\]

\item Represent the output $v_o$ as a fuction of the inputs $v_1$, 
$v_2$, and $v_3$ in the following circuit. Denote the parallel 
connection of $R_0$, $R_1$, $R_2$, and $R_3$ by $R+=R_0||R_1||R_2||R_3$.

\htmladdimg{../hw10e.gif}

{\bf Solution:} 

Assume $v-=v+=v$, then we have:
\[ \frac{v}{R_4}=\frac{v_o-v}{R_f},\;\;\;\;\mbox{i.e.,}\;\;\;
	v_o=(1+\frac{R_f}{R_4}) v	\]
and
\[ \frac{v_1-v}{R_1}+\frac{v_2-v}{R_2}+\frac{v_3-v}{R_3}=\frac{v}{R_0} \]
i.e.,
\[ \frac{v_1}{R_1}+\frac{v_2}{R_2}+\frac{v_3}{R_3}
	=v(\frac{1}{R_0}+\frac{1}{R_1}+\frac{1}{R_2}+\frac{1}{R_3})
	=\frac{v}{R+}	\]
Then we have
\[	v_o=(1+\frac{R_f}{R_4})(\frac{R+}{R_1}v_1+\frac{R+}{R_2}v_2
	+\frac{R+}{R_3v_3})	\]

\item Find the output $v_o$ of the following circuit as a function 
of all four inputs $v_1$, $v_2$, $v_3$, and $v_4$. It is known that
$R_3 || R_4 || R_0=R_1 || R_2 || R_f$.

\htmladdimg{../hw10c.gif}

{\bf Solution:} Assume $v-=v+=v$, then we have
\[ \frac{v_1-v}{R_1}+\frac{v_2-v}{R_2}=-\frac{v_o-v}{R_f},\;\;\;\;
\mbox{i.e.,}\;\;\;\;
	\frac{v_1}{R_1}+\frac{v_2}{R_2}+\frac{v_o}{R_f}
	=v(\frac{1}{R_1}+\frac{1}{R_2}+\frac{1}{R_f}) \]
also
\[ \frac{v_3-v}{R_3}+\frac{v_4-v}{R_4}=\frac{v}{R_o},\;\;\;\;
\mbox{i.e.,}\;\;\;\;
	\frac{v_3}{R_3}+\frac{v_4}{R_4}
	=v(\frac{1}{R_3}+\frac{1}{R_4}+\frac{1}{R_o}) \]
solve for $v$:
\[ v=(\frac{v_3}{R_3}+\frac{v_4}{R_4})/(\frac{1}{R_3}+
	\frac{1}{R_4}+\frac{1}{R_o}) \]
plug into the other equation:
\[ \frac{v_1}{R_1}+\frac{v_2}{R_2}+\frac{v_o}{R_f}
	=(\frac{v_3}{R_3}+\frac{v_4}{R_4})
	(\frac{1}{R_1}+\frac{1}{R_2}+\frac{1}{R_f})
	/(\frac{1}{R_3}+\frac{1}{R_4}+\frac{1}{R_o}) 
	=\frac{v_3}{R_3}+\frac{v_4}{R_4}	 \]
i.e.,
\[  v_o=R_f (-\frac{v_1}{R_1}-\frac{v_2}{R_2}+\frac{v_3}{R_3}+
	\frac{v_4}{R_4})	\]

\item The circuit above can be used to find the algebraic sum of a set 
of voltages. However, the required relationship among the resistors
($R_3 || R_4 || R_0=R_1 || R_2 || R_f$) makes the adjustment of the 
coeficients difficult. In practice, the algebraic sum of several
input voltages can be found using multi-stage opamp circuits, as 
shown below. Find the output $v_o$ as a function of all four inputs 
$v_1$, $v_2$, $v_3$, and $v_4$.

\htmladdimg{../hw10f.gif}

{\bf Solution:}

For the first stage, we have
\[	v_{o1}=-R_{f1}(\frac{v_1}{R_1}+\frac{v_2}{R_2})	\]
For the second stage, we have
\[	v_o=-R_{f2}(\frac{v_3}{R_3}+\frac{v_4}{R_4}+\frac{v_{o1}}{R_5})	
	=-\frac{R_{f2}}{R_3}v_3-\frac{R_{f2}}{R_4}v_4
	+\frac{R_{f1}R_{f2}}{R_1R_5}v_1+\frac{R_{f1}R_{f2}}{R_2R_5}v_2	
\]

\item The input voltage $v_i(t)$ of the circuit below is a square wave 
as shown. The horizontal scale is time in millisecond and the vertical 
scale is voltage in volt. Plot the output voltage $v_o(t)$. Assume
$R=10k\Omega$, $C=0.1\mu F$, and the initial voltage across the 
capacitor is 0.

\htmladdimg{../hw10g.gif}

{\bf Solution:} 

$\tau=RC=10^4\times 10^{-7}=10^{-3}=0.001$, let $t_1=1\;ms$, $t_2=3\;ms$. 
First consider the output voltage of during the first three milliseconds.
During the time period $0 \le t < 1\;ms$, the output voltage is
\[	v_o(t)=-\frac{1}{\tau}\int_0^t v_i(t) dt+v_C(0)
	=-\frac{t}{0.001}=-1000 t	\]
In particular, when $t=t_1=1\;ms=0.1\;s$, $v_o(t)=v_o(t_1)=-1V$. 
During the time period $1\;ms \le t < 3\;ms$, the output voltage is
\[	v_o(t)=-\frac{1}{\tau}\int_{t_1}^t v_i(t) dt+v_C(t_1)
	=\frac{t-t_1}{0.001}+v_o(t_1)=1000(t-t_1)-1	\]
In particular, when $t=t_2=3\;ms=0.003\;s$, $t-t_1=t_2-t_1=0.002\;s$,
$v_o(t)=1000\times 0.002-1=1V$.
The output voltage $v_o(t)$ during the next three milliseconds is the
negative of that of the prevous three milliseconds.

\htmladdimg{../hw10h.gif}

\end{itemize}

\end{enumerate}

\end{document}

{\bf Solution:}
\item The circuit below is an voltage amplifier composed of three
opamps. The first two form the first stage and the third is the 
second stage. The values of the resistors are as follows: 
$R_2=R_3=1k\Omega$, $R_1=R_4=R_5=2k\Omega$, $R_6=R_7=100k\Omega$.
Find 
\begin{itemize}
\item voltage gain of the first stage: $A_1=(v_{o1}-v_{o2})/(v_{i1}-v_{i2})$;
\item voltage gain of the second stage: $A_2=v_o/(v_{o1}-v_{o2})$;
\item overall voltage gain $A=v_o/(v_{i1}-v_{i2})$;
\item $A_1$, $A_2$, and $A_3$ when $R_1=\infty$ (open circuit).
\end{itemize}

(Hint: $i=(v_{i1}-v_{i2})/R_1$)

\htmladdimg{../hw10d.gif}

{\bf Solution:}

\[ v_{o1}=iR_2+v_{i1}=(1+\frac{R_2}{R_1})v_{i1}-\frac{R_2}{R_1}v_{i2} \]
\[ v_{o2}=iR_3+v_{i2}=(1+\frac{R_3}{R_1})v_{i2}-\frac{R_3}{R_1}v_{i1} \]
As $R_2=R_3$, 
\[ v_{o1}-v_{o2}=(1+\frac{2R_2}{R_1})(v_{i1}-v_{i2})	\]
\[ A_1=\frac{v_{o1}-v_{o2}}{v_{i1}-v_{i2}}=1+\frac{2R_2}{R_1}	\]
As $R_4=R_5$, $R_6=R_7$, 
\[ A_2=\frac{v_o}{v_{o1}-v_{o2}}=-\frac{R_6}{R_4} \]
\[ A_o=A_1 A_2=-\frac{R_6}{R_4}(1+\frac{2R_2}{R_1})	\]

\item Design a summer with inputs of different signs to implement the 
same control circuit above but with a different control strategy 
$v_o=0.2 v_1-10 v_2+1.3 v_3$. 

{\bf Solution:}


