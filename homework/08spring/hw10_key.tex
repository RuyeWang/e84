\documentstyle[11pt]{article}
\usepackage{html}
\begin{document}
\begin{center}
{\Large \bf E84 Home Work 10}
\end{center}
\begin{enumerate}

\item The circuit shown in the figure contains a voltage source $V=10V$,
two resistors $R_1=300\Omega$ and $R_2=200\Omega$, and a silicon diode.
Find the voltage $V_D$ across and the current $I_D$ through the diode.
Solve this problem in two different methods: (a) assume voltage $V_D=0.7$ 
(as the diode is always forward biased), and (b) use the graphic approach
to find the intersection of the load line and the diode equation:
\[ I_D=I_0 ( e^{V_D/\eta V_T}-1 ) \]
Sketch the plot of the two curves and estimate the solution $(I_D,V_D)$
at their intersection. (Note that you can assume $I_0=10^{-10}$ and 
$V_T=0.026V$ at room temperature 300K.)

\htmladdimg{../hw8a.gif}

{\bf Solution:} As the diode is forward biased, the voltage across it
is 0.7V, and the current through $R_2$ is $I_2=0.7V/200\Omega=3.5mA$, 
the current through $R_1$ is $I_1=(10-0.7)/300=31 mA$, and 
$I_D=I_1-I_2=31-3.5=27.5 mA$.

Also, using Thevenin's theorem, we have $R_T=R_1||R_2=300||200=120$, 
$V_T=10\times 200/(300+200)=4$, and then the linear equation
\[
I_D=\frac{V_T-V_D}{R_T}
\]
Plotting both equations, we get

\htmladdimg{../hw8g.gif}

\item The circuit shown in the figure contains a voltage source $V=10V$,
three resistors $R_1=300\Omega$, $R_2=200\Omega$, and $R_3=100\Omega$, 
and two silicon diode. Find the voltage across the two parallel branches.

\htmladdimg{../hw8c.gif}

{\bf Solution:} As the diode is forward biased, the voltage across it
is 0.7V. Assume the voltage in question is $x$, then we have the following
equation:
\[
\frac{10-x}{300}=\frac{x-0.7}{200}+\frac{x-0.7}{100}	
\]
which can be solved for $x$ to get the voltage to be $2.39V$.
Alternatively, the voltage can also be found as
\[
\frac{R_2||R_3}{R_1+R_2||R_3}(10-0.7)+0.7=2.39
\]

\item The circuit shown in the figure is a converter (adaptor) based on
a full-wave rectifier, which gets an AC voltage input of 115V 60 Hz, and
produces a 12V DC output. The voltage variation or ripple of the output
should not exceed 5\% when the load current is no more than 2A. Design 
the converter in terms of the turn ratio of the transformer and the 
value of the capacitor. 

\htmladdimg{../hw8d.gif}

{\bf Solution:} This is a full-wave rectifier circuit, which turns the
negative half cycle of the input to positive, so that the period of the
output is $T=1/120=0.0083S$. The charge of the capacitor reduced during
this period is $Q=IT=2A\times 0.083 S=0.0167 C$. If the voltage across
the capacitor is dropped by less than $V_C=0.05\times 12=0.6V$, its 
capacitance has to be greater than $C=Q/V_C=0.0167/0.6=0.0278 F$.

The peak voltage of input is $115\times \sqrt{2}=162.6$, the peak voltage
of the output is $12V$. If the 1.4 V voltage drop across the two diodes 
is considered small enough when compared with the required 12 V, then
the turn ratio of the transformer is $162.6/12=13.55$. Otherwise, the 
the turn ratio is $162.6/(12+1.4)=12.13$.

\item The input voltage to the circuit in the following figure is
$v(t)=10 \sin (\omega t)$. The two DC voltages are both 5V. Sketch the
waveform of the output voltage $V_{out}$.

\htmladdimg{../hw8e.gif}

{\bf Solution:} When $v(t)$ is in the range from -5.7V to 5.7V, neither
of the two diode branches is conducting and no current is drawn from the
voltage source. As the result, the output is identical to the input.
When $v(t)>5.7$, the diode branch on the right is conducting and it
will hold the output at 5.7V. When $v(t)<5.7$, the diode branch on the
left is conducting and it will hold the output at -5.7V.

\htmladdimg{../hw_clipping.gif}
 
\item Assume each of the input voltages $V_1$ and $V_2$ takes one of
two values, either 0V or 5V. Find the output voltage $V_{out}$ in the
following combinations of two input. As always, all voltages are 
measured with respective to ground, although it is not explicitly 
shown.

{\bf Solution:}  If at least one of the inputs is 0V, the corresponding
diode is conducting and the output voltage is held to 0.7V. Only when
both inputs are 5V, both diodes are cut off and draw no current from the
voltage source. Since there is no voltage drop across the resistor, the
output voltage is the same as the voltage source of 5V. 

This circuit can be used to implement logic AND, i.e. only when both 
input $V_1$ AND input $V_2$ are high (for true or logical 1), will the
output be high (true or logical 1). Otherwise the output is low (false
or logical 0).

\begin{tabular}{c||c c c c}\hline\hline
$V_1$ & 0V & 0V & 5V & 5V \\ \hline
$V_1$ & 0V & 5V & 0V & 5V \\ \hline
$V_{out}$ & 0.7V & 0.7V & 0.7V & 5V  &   \\
\end{tabular}

\htmladdimg{../hw8f.gif}


\end{enumerate}

\end{document}
