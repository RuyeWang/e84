\documentstyle[11pt]{article}
\usepackage{html}
\begin{document}
\begin{center}
{\Large \bf  E84 Home Work}
\end{center}
\begin{enumerate}

\begin{enumerate}

\item {\bf Algebraic summer:}

  \htmladdimg{../../../lectures/figures/opamp5a.gif}

  Express the output voltage $v_{out}$ as a weighted sum of the four
  input voltages $v_1,\cdots,v_4$:
  \[
  v_{out}=\sum_{i=1}^4 k_i v_i
  \]
  Find the four coefficients (may be either positive or negative).

  Hint: Define $V=V^+ \approx V^- $, and apply KCL to $V^-$ and $V^+$ 
  to get two equations. Solve one of them for $V$, substitute it into 
  the other equation, and then write $v_{out}$ as a function of all 
  four input voltages.

%  \htmladdnormallink{\bf Solution}{../../../lectures/opamp_sub/opamp_sub.html}

\item {\bf Sallen-Key filter (HP/LP)}

  Derive the voltage gain ($V_{out}/V_{in}$) of the general Sallen-Key 
  fitler (HP or LP) in terms of the impedances $Z_1$ through $Z_4$:

  \htmladdimg{../../../lectures/figures/SallenKey.gif}

\item {\bf Sallen-Key filter (BP)}

  \htmladdimg{../../../lectures/figures/SallenKeyBP.png}

  Derive the voltage gain ($V_{out}/V_{in}$) of the general Sallen-Key 
  fitler (HP or LP) in terms of the impedances $Z_1$ through $Z_4$
  and $R_f$, $R_a$, $R_b$.

  ({\bf Hint:} For simplicity, assume $R_a/(R_a+R_b)=k$.

\item {\bf The Howland current source:}

  The circuit generates a constant current through the load, independent
  of the load resistance $R_L$ of the load. Give the expression of this 
  current $I_L$ through $R_L$ as a function of the inputs $V^+$ and $V^-$ 
  and the resistances in the circuit, and show it is independent of $R_L$.
  Assume $R_2/R_1=R_4/R_3$.
  
  \htmladdimg{../../../lectures/figures/HowlandCurrentSource.png}

%  \htmladdnormallink{\bf Solution}{../../../lectures/opamp_sub/node2.html}


  Hint: recall the relationship between the current through and voltage 
  across a diode and use the virtual ground assumption.

  \begin{comment}
  {\bf Solution:}
  \[
  \frac{V^--V}{R_1}+\frac{V_0-V}{R_2}=0,\;\;\;\;\;\;
  \frac{V^+-V}{R_3}+\frac{V_0-V}{R_4}=\frac{V}{R_L};
  \]
  Solving the first equation for $V_0-V$:
  \[
  V_0-V=(V-V^-)\frac{R_2}{R_1}=(V-V^-)\frac{R_4}{R_3}
  \]
  and substituting into the second equation, we get:
  \[
  \frac{V^+-V}{R_3}+\frac{V-V^-}{R_3}
  =\frac{V^+-V^-}{R_3}=\frac{V}{R_L}
  \]
  Solving the last equation for $V$ we get
  \[
  V=\frac{R_L}{R_3}(V^+-V^-)
  \]
  and the current through $R_L$ is:
  \[
  I_L=\frac{V}{R_L}=\frac{V^+-V^-}{R_3}
  \]
  which is a constant independent of $R_L$.
  \end{comment}

\item {\bf Exponential and logarithmic amplifiers:}

  \htmladdimg{../../../lectures/figures/ExpLogAmplifier.png}

  The current $I_D$ through and voltage $V_D$ across a diode are related 
  by the following:
  \[
  I_D=I_0 \left( e^{V_D/\eta V_T}-1 \right)
  \]
  where $\eta$ and $V_T$ are some parameters. The direction of the current
  $I_D$ through the diode is indicated by the arrow of the symbol, and the 
  polarity of the voltage $V_D$ is plus on the arrow side and minus on the
  other side. When $V_D/\eta V_T$ is large enough, $e^{V_D/\eta V_T}\gg 1$.

  Show that the output voltages $v_{out}$ of the exponential amplifier 
  (left) and logarithmic amplifier (right) are approximately an exponential
  and logarithmic function of the input voltage $v_{in}$, respectively:
  \[
  v_{out}\approx C \;\exp(v_{in}/a),\;\;\;\;\;\; v_{out}\approx D\; \ln (v_{in}/b)
  \]
  Derive these relationships and determine the coefficients $C,\;D$ and 
  $a, b$.

%  \htmladdnormallink{\bf Solution}{../../../lectures/opamp_sub/node3.html}

\end{enumerate}

\end{document}







\item The figure (A) below shows a common-emitter transistor 
applification circuit (silicon) with$\beta=100$, $V_{cc}=20V$, 
and $R_C=2\;K\Omega$. Find $i_c(t)$ and $v_c(t)$ as functions of time for each
  of the base currents given below:
\begin{itemize}

\item 
\
 is $i_b(t)=(50+30 \cos \omega t)\;\mu A$.

\item Repeat the above for $i_b(t)=(90+30 \cos \omega t)\;\mu A$.
\item Repeat the above for $i_b(t)=(10+30 \cos \omega t)\;\mu A$.
  However, note that the actual base current cannot be negative as 
  it can only flow from base to emitter but not the other direction,
  i.e., all negative parts of $i_b(t)$ are zero.
\end{itemize}
For each of the three cases above, sketch the output (collector) 
characteristics ($i_c$ vs $v_c$) as shown in class to show the wave
forms of $i_b(t)$, $i_c(t)$ and $v_{ce}(t)$, following the example
in the 
\htmladdnormallink{lecture notes:}{http://fourier.eng.hmc.edu/e84/lectures/ch4/node5.html}

{\bf Note:} As the convention in the schematics of transistor circuits,
the bottom horizontal line is treated as the ground, and all voltages,
such as $V_b$, $V_c$ and $V_e$ are measured with respect to the 
ground as the reference point.

{\bf Hint:} The relationship $I_C=\beta I_B$ is only valid in the
linear region in the middle range of the load line. However, in 
the cut-off region (close to the horizontal axis) and the saturation
region (close to the vertical axis), the above relationship no
longer holds and the actual output current $I_c$ and $V_{ce}$ can
only be found graphically in the output characteristic plot.

\htmladdimg{../hw9f.gif}

% {\bf Solution:}
%  \begin{itemize}
%  \item 
%  \[ i_c(t)=\beta i_b(t)=(5+3\cos \omega t)\;mA \]
%  \[ v_c(t)=V_{cc}-R_c i_c(t)=20-2000 (5+3\cos \omega t)\times 10^{-3}
%     =10-6\cos \omega t  \]
%  \item
%  \[ i_c(t)=\beta i_b(t)=(1+3\cos \omega t)\;mA \]
%  \[ v_c(t)=V_{cc}-R_c i_c(t)=20-2000 (9+3\cos \omega t)\times 10^{-3}
%     =2-6\cos \omega t  \]
%  Clipping happens during the negative half-cycle due to saturation.
%  \item
%  \[ i_c(t)=\beta i_b(t)=(10+3\cos \omega t)\;mA \]
%  \[ v_c(t)=V_{cc}-R_c i_c(t)=20-2000 (1+3\cos \omega t)\times 10^{-3}
%     =18-6\cos \omega t  \]
%  Clipping happens during the positive half-cycle due to cut-off.
%  \end{itemize}



\item Assume in the circuit below (figure B), $\beta=100$, $V_{cc}=20V$, 
$R_C=R_E=1\;K\Omega$.  Find the voltages $V_E(t)$ and $V_C(t)$ when 
the base current is $I_B(t)=(50+30 \cos \omega t)\;\mu A$. Sketch the
waveformes (over time) of the two voltages.

\htmladdimg{../hw9f.gif}

% {\bf Solution:}
%  \[ i_c(t)=\beta i_b(t)=(5+3\cos \omega t)\;mA \]
%  \[ v_c(t)=V_{cc}-R_c i_c(t)=20-1000 (5+3\cos \omega t)\times 10^{-3}
%    =15-3\cos \omega t  \]
%  \[ i_e(t)=(\beta+1) i_b(t)\approx i_c(t)=(5+3\cos \omega t)\;mA \]
%  \[ v_e(t)=R_e i_e(t)=1000 (5+3\cos \omega t)\times 10^{-3}
%     =5+3\cos \omega t  \]

\item Find values of $R_C$ and $R_B$ in the circuit with $\beta=100$
and $V_{CC}=15V$ so that the Q-point is $I_C=25mA$ and $V_{CE}=7.5V$.
What is the Q point if $\beta=200$?

\htmladdimg{../hw9a.gif}

% {\bf Solution:}

%  Find $I_B=I_C/\beta=25mA/100=0.25mA$, $R_B=(15-0.7)/0.25=57.2k\Omega$.
%  Also as $V_{CE}=v_{cc}-I_C R_C$, i.e., $R_C=(V_{CC}-V_{CE})/I_C=
%  (15-7.5)/25\times 10^{-3}=300\Omega$. 
%  
%  $I_B=(V_{CC}-V_{BD})/R_B=0.25mA$, $I_C=\beta I_B=200\times 0.25=50mA$,
%  $V_{CE}=V_{CC}-R_C I_C=15-300\times 50\times 10^{-3}=0$.
% This result indicate that the transistor is in saturation region and the
% linear relationship $I_C=\beta I_B$ no longer holds. In this case $V_{CE}$ 
% can be roughly estimated as being 0.2 V$ and $I_C$ is just a little bit
% smaller than $V_{CC}/R_C=5 mA$.

\item Design a stable self-biasing transistor circuit as shown below
so that the DC operating point (Q point) of $I_C=2.5mA$ and $V_{CE}=7.5V$ 
is in the middle of the load line for maximal dynamic range. Assume 
the smallest $\beta$ value allowed is $\beta_{min}=50$. To reduce the 
number of free parameters, we assume $R_C=2K\Omega$. 

Hint: 
\begin{itemize}
\item For the DC operating point to be approximately independent of the
  specific transistor used, we want $(\beta_{min}+1)R_E\approx 
  \beta_{min} R_E \ge 10 R_B$.
\item Start the design process from the desired Q-point, determine $R_E$, 
then find desired $V_{BB}$ and finally $R_1$ and $R_2$.
\end{itemize}

\htmladdimg{../hw9b.gif}

%  {\bf Solution:}
 
%  \begin{itemize}
%  \item Find $V_{CC}$: for the Q-point to be in the middle of the
%    load line, we set $V_{CC}=2V_{CE}=2\times 7.5=15V$.
%  \item Find $R_C$ and $R_E$: As $V_{CE}=V_{CC}-I_CR_C-I_ER_E\approx
%    V_{CC}-I_C(R_C+R_E)$, we have $R_C+R_E=7.5/2.5\times 10^{-3}=3K\Omega$,
%    i.e., $R_E=5K\Omega-R_C=1K\Omega$.
%  \item Find desired $V_{BB}$: 
%  	\[ V_{BB}=V_{be}+I_ER_E=0.7+2.5\times 10^{-3}\times 10^3=3.2V \]
%  \item Find $R_B$: To satisfy $10R_B \le \beta_{min}R_E$, we let 
%    $R_B=0.1\times \beta_{min} R_E=0.1\times 50\times 1000=5\;K\Omega$
%  \item Find $R_1$ and $R_2$:
%  \[	R_B=\frac{R_1R_2}{R_1+R_2}=5\;K\Omega \;\;\;\;\;\;\;\;
%        V_{BB}=3.2V=V_{CC}}\frac{R_2}{R_1+R_2}=15 \frac{R_2}{R_1+R_2} \]
%  Solve these two equations (first divide the first equation by the second), 
%  we obtain the two unknowns $R_1$ and $R_2$:
%  \[	R_1=\frac{5\;K\Omega}{0.21}=24\;K\Omega \;\;\;\;\;\;\;\;\;
%  	R_2=6.4\; K\Omega	\]

 \end{itemize}

\item Verify your design in previous problem by checking the DC operating
  point of the resulting circuit is approximately the same as the requirement.

% {\bf Solution:}
% \begin{itemize}
% \item Use Thevenin's theorem to find:
%   \[ R_B=\frac{R_1 R_2}{R_1+R_2}=\frac{24\times 6.4}{24+6.4}=5.05K\Omega \]
%   \[ V_{BB}=V_{CC}\frac{R_2}{R_1+R_2}=15\frac{6.4}{24+6.4}=3.16V \]
% \item Find $V_E=V_{BB}-V_{BE}=3.16V-0.7V=2.46V$.
% \item Find $I_C \approx I_E=V_E/R_E=2.46 mA$.
% \item Find $V_{CE}=V_{CC}-I_C R_C-I_E R_E\approx 15-2.46(2+1)=7.6V$
% \end{itemize}


\item The {\em Howland current source} shown below is an op-amp circuit 
  that generates a constant current through the load, independent of
  the resistance $R_L$ of the load. Give the expression of this current
  $I_L$ through $R_L$ as a function of the inputs $V^+$ and $V^-$ and the
  resistances in the circuit, assuming $R_2/R_1=R_4/R_3$, and show it is 
  independent of $R_L$.
  
  \htmladdimg{../../../lectures/figures/HowlandCurrentSource.png}

  \begin{comment}
  {\bf Solution:}
  \[
  \frac{V^--V}{R_1}+\frac{V_0-V}{R_2}=0,\;\;\;\;\;\;
  \frac{V^+-V}{R_3}+\frac{V_0-V}{R_4}=\frac{V}{R_L};
  \]
  Solving the first equation for $V_0-V$:
  \[
  V_0-V=(V-V^-)\frac{R_2}{R_1}=(V-V^-)\frac{R_4}{R_3}
  \]
  and substituting into the second equation, we get:
  \[
  \frac{V^+-V}{R_3}+\frac{V-V^-}{R_3}
  =\frac{V^+-V^-}{R_3}=\frac{V}{R_L}
  \]
  Solving the last equation for $V$ we get
  \[
  V=\frac{R_L}{R_3}(V^+-V^-)
  \]
  and the current through $R_L$ is:
  \[
  I_L=\frac{V}{R_L}=\frac{V^+-V^-}{R_3}
  \]
  which is a constant independent of $R_L$.
  \end{comment}

\end{enumerate}

\end{document}

\item In the circuit shown below, the two base resistors $R_B=4.3K\Omega$,
the collector resistor is $R_C=1K\Omega$. Assume the two transistors
have the same $\beta=100$ value and they each receive an input ($V_1$ and
$V_2$) at either $0.2V$ or $5V$. Find the output voltage $V_{out}$ for
the following combinations of inputs. (Hint: $5V$ input to a transistor
will drive it to saturation.)

\begin{tabular}{cc|c}\hline
$V_1$ & $V_2$ & $V_{out}$ \\ \hline
  $0.2$ & $0.2$ &         \\
  $5.0$ & $0.2$ &         \\
  $0.2$ & $5.0$ &         \\
  $5.0$ & $5.0$ &         \hline
\end{tabular}

\htmladdimg{../hw9d.gif}

% {\bf Solution:} When $V_1=0.2V$, $T_1$ is cut-off. When $V_1=5V$, 
% $I_{b1}=(5-0.7)/4.3=1\;mA$, $T_1$ is saturated with $V_{CE}=0.2V$
% and $I_C=(5-0.2)/1=4.8\;mA$ (instead of $I_C=\beta I_b=100\;mA$).
% The same is true for $T_2$. From the table below we see that the
% circuit is a NOR (not OR) gate (high voltage for True and low
% voltage for False). 
% \begin{tabular}{cc|c}\hline
% $V_1$ & $V_2$ & $V_{out}$ \\ \hline
%   $0.2$ & $0.2$ & $5.0V$ \\
%   $5.0$ & $0.2$ & $0.2V$ \\
%   $0.2$ & $5.0$ & $0.2V$ \\
%   $5.0$ & $5.0$ & $0.2V$ \hline
% \end{tabular}




\item In the AC amplifier shown in the figure, $R_S=600\Omega$, $R_1=30K\Omega$, 
$R_2=20K\Omega$, $R_E=4K\Omega$, $R_C=3K\Omega$, $R_L=5.1K\Omega$, 
$\beta=100$, $V_{CC}=15V$. Also assume $r_{be}=1000\Omega$. And assume the
frequency of the AC signals to be amplified is high enough so that the
coupling capacitors and the emitter by-pass capacitor can be treated as 
AC short circuits. Find
\begin{itemize}
\item The DC operating point in terms of variables $I_B$, $I_C$, $V_C$, $V_E$ 
  and $V_{CE}$.
\item The AC equivalent diagram
\item The AC input and output impedances
\item The voltage gain of the AC amplifier
\end{itemize}

\htmladdimg{../hw9c.gif}

 {\bf Solution:} 
 \begin{itemize}
 \item Find DC operating point.
 \[ V_{BB}=V_{CC}\frac{R_2}{R_1+R_2}=15\frac{20}{30+20}=6V	\]
 \[ R_B=\frac{R_1R_2}{R_1+R_2}=\frac{600}{50}=12K\Omega 	\]
 \[ I_B=\frac{V_{BB}-V_{be}}{(\beta+1)R_E+R_B}
 	=\frac{6-0.7}{101\times 4000+12000}=\frac{5.3}{26\times 10^4}
 	=0.025 mA	\]
 \[ I_C=\beta I_B=100\times 0.025 mA=2.5 mA	\]
 \[ V_C=V_{CC}-I_C R_C=15-2.5\times 3=7.5V	\]
 \[ V_E=(\beta+1)I_C R_E=5V	\]
 \[ V_{CE}=V_C-V_E=2.5V \]
 \item Find input and output impedances.
 \[	r_{be}=V_T/I_B=0.026/0.025=1K\Omega	\]
 \[	r_{in}=R_1||R_2||r_{be} \approx 1K\Omega \]
 \[	r_{out}=R_C=3K\Omega	\]

 \item Find voltage gain.
 \[	G=-\beta \frac{(R_1||R_2||r_{be})}{(R_1||R_2||r_{be})+R_s}
 	\frac{1}{r_{be}}(R_C||R_L) \approx -117	\]
 \end{itemize}

\item An emitter follower circuit is shown in the figure. Assume 
$r_{be}=1K\Omega$, $\beta=100$, $R_B=300K\Omega$, $R_E=4K\Omega$, 
$R_S=500\Omega$, $R_L=5.1K\Omega$, and $V_{CC}=12$. Find the DC operating 
point in terms of $I_B$, $I_C$, $I_E$, and $V_E$. Then find the input and 
output resistances and the voltage gain.

\htmladdimg{../hw9e.gif}

{\bf Solution}

Find base current:
\[ I_B=\frac{V_{CC}-V_{BE}}{R_B+(\beta+1)R_E}=16.1 \mu A \]
\[ I_E=(\beta+1) I_B=1.61 mA \]
\[ V_E=I_E R_E=6.48 V,  V_{CE}=V_{CC}-V_E=5.52 V \]
\[ G=\frac{(\beta+1)R_E||R_L}{R_S+r_{be}+(\beta+1)R_E||R_L} 
    =101\times 2.24/(1.5+101\times 2.24)=0.993 \]
\[ r_{in}=(\beta+1)(R_E||R_L)=101\times 2.24K\Omega =226.4 K\Omega \]
\[ r_{out}=(R_S+r_{be})/(\beta+1)=1.5K\Omega/101=15 \Omega \]

\end{enumerate}
\end{document}

\item In the circuit shown below, the two base resistors $R_B=43K\Omega$,
the collector $R_C=1K\Omega$. Assume each of the two input voltages $V_1$
and $V_2$ are either 0.2V or 5V. Find the output voltage $V_{out}$ in the
following combinations of the inputs. (Hint: assume 5V input to a transistor
will drive it to saturation.)

\begin{tabular}{cc|c}\hline
$V_1$ & $V_2$ & $V_{out}$ \\ \hline
 0.2  &  0.2  &           \\ \hline
 5.0  &  0.2  &           \\ \hline
 0.2  &  5.0  &           \\ \hline
 5.0  &  5.0  &           \\ \hline
\end{tabular}

\htmladdimg{../hw9d.gif}

\end{itemize}


\item In the AC amplifier shown in the figure, $R_S=1.5K\Omega$, $R_1=10K\Omega$, 
$R_2=200K\Omega$, $R_C=3.5K\Omega$, $R_L=2K\Omega$, $\beta=100$, $V_{CC}=20V$.
Also assume $r_{be}=200\Omega$. Find the DC variables $I_B$, $I_C$, $V_C$.



///////////


\documentstyle[11pt]{article}
\usepackage{html}
\begin{document}
\begin{center}
{\Large \bf  E84 Home Work 10}
\end{center}

\begin{enumerate}

\item In the circuit shown below, the two base resistors $R_B=4.3K\Omega$,
the collector $R_C=1K\Omega$. Assume the two transistors have the same 
$\beta=100$ value and they each receive an input voltage ($V_1$ and $V_2$)
at either 0.2V or 5V. Find the output voltage $V_{out}$ for the following 
combinations of the inputs. (Hint: 5V input to a transistor will drive it 
to saturation.)

\begin{tabular}{cc|c}\hline
$V_1$ & $V_2$ & $V_{out}$ \\ \hline
 0.2  &  0.2  &           \\ \hline
 5.0  &  0.2  &           \\ \hline
 0.2  &  5.0  &           \\ \hline
 5.0  &  5.0  &           \\ \hline
\end{tabular}

\htmladdimg{../hw9d.gif}

{\bf Solution:} When $V_1=0.2V$, $T_1$ is cut-off. When $V_1=5V$, 
$I_{b1}=(5-0.7)/4.3=1mA$, $T_1$ is saturated with $V_{CE}=0.2V$ and 
$I_C=(5-0.2)/1=4.8 mA$ (instead of $I_C=\beta I_B=100 mA$). The same is
true for $T_2$. From the table below we see that the circuit is a NOR
(Not OR) gate:

\begin{tabular}{cc|c}\hline
$V_1$ & $V_2$ & $V_{out}$ \\ \hline
 0.2  &  0.2  &    5      \\ \hline
 5.0  &  0.2  &    0.2    \\ \hline
 0.2  &  5.0  &    0.2    \\ \hline
 5.0  &  5.0  &    0.2    \\ \hline
\end{tabular}

\item A BJT transistor with $\beta=24$ is set up as a common-base 
configuration as shown in the figure below. 
\htmladdimg{../../lectures/figures/CB.gif}

\begin{itemize} 
\item If it is known that $I_{CB0}=10 nA$ and $I_E=2 mA$, find $I_C$ 
  and $I_B$.
\item If $V_{CB}=15V$ and $V_{EB}=0.7$, estimate $I_E$, $I_B$ and $I_C$ 
  based on the plots below. Re-estimate these currents if $V_{EB}$ is 
  increased to $0.8V$.

  Note: in the figure above, as the assumed polarities of both $I_E$ and 
  $V_{EB}$ are the opposite to those assumed in plot (b) below, the negative
  signs for both $I_E$ and $V_{EB}$ in the plot should be dropped.

\end{itemize}

\htmladdimg{../../lectures/figures/transistorCBplots.gif}

{\bf Solution:} $\alpha=\beta/(1+\beta)=24/25=0.96$, 
$I_C=I_E \alpha + I_{CB0}=2 mA \times 1.92+0.01=1.921 mA$,
$I_B=I_C/\beta=1.92/24=0.08$, or $I_B=I_E-I_C=2-1.96=0.08 mA$.

If $V_{EB}=0.7 V$ and $V_{CB}=15V$, $I_E=5 mA$ can be estimated from 
figure (b), and correspondingly, $I_C=4.8 mA$ and $I_B=0.2 mA$. When 
$V_{EB}=0.8 V$, $I_E=20 mA$, $I_C=19.2 mA$, and $I_B=0.8 mA$.

\item In the figure below, the transistor with $\alpha=0.99$ and
  $I_{CB0}=10^{-11} A$ is set up as a common-emitter circuit. The
  base-emitter pn-junction is forward biased with $I_B=20 \mu A$, 
  $V_{CC}=10V$, and $R_C=2k\Omega$. Find $I_{CE0}$, $I_C$, $I_E$, 
  $V_{CE}$, and $V_{CB}$. Is the collector-base pn-junction forward
  or reverse biased? (Assume the voltage across a forward biased 
  pn-junction is $0.7 V$.)
  
\htmladdimg{../CEexample.gif}

{\bf Solution:} 
\[ \beta=\frac{\alpha}{1-\alpha}=99 \]
\[ I_{CE0}=(1+\beta) I_{CB0}=(1+99) 10^{-11}=10^{-9} \]
\[ I_C=\beta I_B+I_{CE0}=99 \times 2 \times 10^{-5} + 10^{-9}=1.98 mA \]
\[ I_E=I_C+I_B=1.98+0.02=2 mA \]
\[ V_{CE}=V_{CC}-R_C I_C=10-2 \times 10^3 \times 1.98 \times 10^{-3}
   \approx 10-4=6 V \]
Since the base-emitter pn-junction is forward biased with $V_{BE}=0.7V$,
we have $V_{CB}=V_{CE}-V_{BE}=6-0.7=5.3 V$, the collector-base pn-junction
is reverse biased.

\item The figure (A) below shows a common-emitter transistor 
applification circuit (silicon) with$\beta=100$, $V_{cc}=20V$, 
and $R_C=2\;K\Omega$. 
\begin{itemize}
\item The base current is $i_b(t)=(50+30 \cos \omega t)\;\mu A$.
Find $i_c(t)$ and $v_c(t)$ as functions of time.
\item Repeat the above for $i_b(t)=(90+30 \cos \omega t)\;\mu A$.
\item Repeat the above for $i_b(t)=(10+30 \cos \omega t)\;\mu A$.
\end{itemize}
For each of the three cases above, sketch the output (collector) 
characteristics ($i_c$ vs $v_c$) as shown in class show the wave
forms of $i_b(t)$, $i_c(t)$ and $v_c(t)$, following the example
in the lecture notes:
http://fourier.eng.hmc.edu/e84/lectures/ch4/node7.html.

{\bf Note:} As the convention in the schematics of transistor circuits,
the bottom horizontal line is treated as the ground, and all voltages,
such as $V_b$, $V_c$ and $V_e$ are measured with respect to the 
ground as the reference point.

{\bf Hint:} The relationship $I_C=\beta I_B$ is only valid in the
linear region in the middle range of the load line. However, in 
the cut-off region (close to the horizontal axis) and the saturation
region (close to the vertical axis), the above relationship no
longer holds and the actual output current $I_c$ and $V_{CE}$ can
only be found graphically in the output characteristic plot.

\htmladdimg{../hw9f.gif}

 {\bf Solution:}
 \begin{itemize}
 \item 
 \[ i_c(t)=\beta i_b(t)=(5+3\cos \omega t)\;mA \]
 \[ v_c(t)=V_{cc}-R_c i_c(t)=20-2000 (5+3\cos \omega t)\times 10^{-3}
    =10-6\cos \omega t  \]
 \item
 \[ i_c(t)=\beta i_b(t)=(1+3\cos \omega t)\;mA \]
 \[ v_c(t)=V_{cc}-R_c i_c(t)=20-2000 (9+3\cos \omega t)\times 10^{-3}
    =2-6\cos \omega t  \]
 Clipping happens during the negative half-cycle due to saturation.
 \item
 \[ i_c(t)=\beta i_b(t)=(10+3\cos \omega t)\;mA \]
 \[ v_c(t)=V_{cc}-R_c i_c(t)=20-2000 (1+3\cos \omega t)\times 10^{-3}
    =18-6\cos \omega t  \]
 Clipping happens during the positive half-cycle due to cut-off.
 \end{itemize}

\item Now assume an emittor resistor $R_e=2\;K\Omega$ is added between 
emittor of the transistor and ground in the circuit above (figure (B)
above).  Find the voltages $v_e(t)$ as well as $v_c(t)$ when the base
current is $i_b(t)=(50+30 \cos \omega t)\;\mu A$. Sketch the waveformes
of the two voltages.

 {\bf Solution:}
 \[ i_c(t)=\beta i_b(t)=(5+3\cos \omega t)\;mA \]
 \[ v_c(t)=V_{cc}-R_c i_c(t)=20-2000 (5+3\cos \omega t)\times 10^{-3}
    =10-6\cos \omega t  \]
 \[ i_e(t)=(\beta+1) i_b(t)\approx i_c(t)=(5+3\cos \omega t)\;mA \]
 \[ v_e(t)=R_e i_e(t)=2000 (5+3\cos \omega t)\times 10^{-3}
    =10+6\cos \omega t  \]

\end{enumerate}

\end{document}



\item Find values of $R_C$ and $R_B$ in the circuit with $\beta=100$
and $V_{CC}=15V$ so that the Q-point is $I_C=25mA$ and $V_{CE}=7.5V$
What is the Q point if $\beta=200$?

\htmladdimg{../hw9a.gif}


% {\bf Solution:}

% Find $I_B=I_C/\beta=25mA/100=0.25mA$, $R_B=(15=0.7)/0.25=57.2k\Omega$
% Also as $V_{CC}=I_C R_C+V_{CE}$, i.e., $R_C=(V_{CC}-V_{CE})/I_C=
% (15-7.5)/25\times 10^{-3}=300\Omega$. 
% 
% $I_B=(V_{CC}-V_{BD})/R_B=0.25mA$, $I_C=\beta I_B=200\times 0.25=50mA$,
% $V_{CE}=V_{CC}-R_C I_C=15-300\times 50\times 10^{-3}=0$.

\item Design a stable self-biasing transistor circuit with a DC operating
point (Q point) of $I_C=2.5mA$ and $V_{CE}=7.5V$ which should be in the 
middle of the load line. The $\beta$ of the ransistor ranges from 50 to
200. 

{\bf Hint} This is a design problem with possibly multiple solutions, 
i.e., there may be more degrees of freedom than constraining conditions. 
One of such conditions is $(\beta+1)R_E \gg R_B$ for the DC operating 
point to be approximately independent of $\beta$ (see online notes). 

\htmladdimg{../hw9b.gif}

% {\bf Solution:}
% 
% \begin{itemize}
% \item Find $V_{CC}$: We want the Q-point to be in the middle of the
%   load line, so we set $V_{CC}=2V_{CE}=2\times 7.5=15V$.
% \item Find $R_C$ and $R_E$: As $V_{CE}=V_{CC}-I_CR_C-I_ER_E\approx
%   V_{CC}-I_C(R_C+R_E)$, we have $R_C+R_E=7.5/2.5\times 10^{-3}=3K\Omega$.
%   Choose $R_E=1K\Omega$ and $R_C=2K\Omega$.
% \item Find $R_B$: To satisfy $R_B \ll \beta_{min} R_E$, we
% 	let $R_B=0.1\times \beta_{min} R_E=0.1\times 50\times 1000=5\;K\Omega$
% \item Find $V_{BB}$: 
% 	\[ V_{BB}=V_{be}+I_ER_E=0.7+2.5\times 10^{-3}\times 10^3=3.2V \]
% \item Find $R_1$ and $R_2$:
% \[	R_B=\frac{R_1R_2}{R_1+R_2}=5\;K\Omega \;\;\;\;\;\;\;\;
% 	\frac{V_{BB}}{V_{CC}}=\frac{R_2}{R_1+R_2}=\frac{3.2}{15}=0.21	\]
% Solve these two equations (first divide the first equation by the second), 
% we obtain the two unknowns $R_1$ and $R_2$:
% \[	R_1=\frac{5\;K\Omega}{0.21}=24\;K\Omega \;\;\;\;\;\;\;\;\;
% 	R_2=6.4\; K\Omega	\]
% \end{itemize}

\item In the AC amplifier shown in the figure, $R_S=600\Omega$, $R_1=30K\Omega$, 
$R_2=20K\Omega$, $R_E=4K\Omega$, $R_C=3K\Omega$, $R_L=5.1K\Omega$, 
$\beta=100$, $V_{CC}=15V$. 
%Also assume $r_{be}=200\Omega$. 
Assume the 
frequency of the AC signals to be amplified is high enough so that the
coupling capacitors and the emitter by-pass capacitor can be treated as 
AC short circuits. Find
\begin{itemize}
\item The DC operating point in terms of variables $I_B$, $I_C$, $V_C$, $V_E$ 
  and $V_{CE}$.
\item The AC equivalent diagram
\item The AC input and output impedances
\item The voltage gain of the AC amplifier
\end{itemize}

\htmladdimg{../hw9c.gif}

% {\bf Solution:} 
% \begin{itemize}
% \item Find DC operating point.
% \[ V_{BB}=V_{CC}\frac{R_2}{R_1+R_2}=15\frac{20}{30+20}=6V	\]
% \[ R_B=\frac{R_1R_2}{R_1+R_2}=\frac{600}{50}=12K\Omega 	\]
% \[ I_B=\frac{V_{BB}-V_{be}}{(\beta+1)R_E+R_B}
% 	=\frac{6-0.7}{101\times 4000+12000}=\frac{5.3}{26\times 10^4}
% 	=0.025 mA	\]
% \[ I_C=\beta I_B=100\times 0.025 mA=2.5 mA	\]
% \[ V_C=V_{CC}-I_C R_C=15-2.5\times 3=7.5V	\]
% \[ V_E=(\beta+1)I_C R_E=5V	\]
% \[ V_{CE}=V_C-V_E=2.5V \]
% \item Find input and output impedances.
% \[	r_{be}=V_T/I_B=0.026/0.025=1K\Omega	\]
% \[	r_{in}=R_1||R_2||r_{be} \approx 1K\Omega \]
% \[	r_{out}=R_C=3K\Omega	\]

% \item Find voltage gain.
% \[	G=-\beta \frac{(R_1||R_2||r_{be})}{(R_1||R_2||r_{be})+R_s}
% 	\frac{1}{r_{be}}(R_C||R_L) \approx -117	\]
% \end{itemize}

\item An emitter follower circuit is shown in the figure. Assume 
$\beta=50$, $R_B=300K\Omega$, $R_E=4K\Omega$, $R_S=500\Omega$, 
$R_L=5.1K\Omega$, and $V_{CC}=12$. Find the DC operating point in
terms of $I_B$, $I_C$, $I_E$, and $V_E$. Then find the input and
output resistances and the voltage gain.

\htmladdimg{../hw9e.gif}

\item In the circuit shown below, the two base resistors $R_B=43K\Omega$,
the collector $R_C=1K\Omega$. Assume each of the two input voltages $V_1$
and $V_2$ are either 0.2V or 5V. Find the output voltage $V_{out}$ in the
following combinations of the inputs. (Hint: assume 5V input to a transistor
will drive it to saturation.)

\begin{tabular}{cc|c}\hline
$V_1$ & $V_2$ & $V_{out}$ \\ \hline
 0.2  &  0.2  &           \\ \hline
 5.0  &  0.2  &           \\ \hline
 0.2  &  5.0  &           \\ \hline
 5.0  &  5.0  &           \\ \hline
\end{tabular}

\htmladdimg{../hw9d.gif}


\end{enumerate}

\end{document}

\end{enumerate}
