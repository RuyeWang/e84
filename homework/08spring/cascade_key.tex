\documentstyle[11pt]{article}
\usepackage{html}
\begin{document}
\begin{center}
{\Large \bf E84 Homework 1}
\end{center}
\begin{enumerate}


\item Find the frequency response function (FRF) $H_C(j\omega)$ of the RC circuit 
  inside the dashed box in the figure below (E59!). 

  Then find the overall FRF of the entire circuit $H(j\omega)$ when the voltage across 
  the 2nd capacitor on the right is treated as the output $V_{out}$. 

  Compare your result $H(j\omega)$ with $H^2_C(j\omega)$. Are they the same or different, 
  why?

  If you want the overall FRF to be $H^2_C(j\omega)$, what can you do (E80 op-amp lab!) ?

  \htmladdimg{../../../lectures/figures/cascade.gif}

  {\bf Solution:} Find $H_C(j\omega)$ by voltage divider:
  \[  V_{out}=V_{in}\frac{1/j\omega C}{R+j\omega C}=V_{in}\frac{1}{j\omega RC+1},
  \;\;\;\;\;\;\mbox{i.e.}\;\;\;\;\;\;
  H_C(j\omega)=\frac{V_{out}}{V_{in}}=\frac{1}{j\omega RC+1}=\frac{1}{j\omega\tau+1} \]
  where $\tau=RC$. And we have:
  \[ H^2_C(j\omega)=\frac{1}{(j\omega\tau+1)^2}=\frac{1}{(j\omega\tau)^2+j2\omega\tau+1}
  =\frac{1}{1-(\omega\tau)^2+j2\omega\tau}
  \]
  To find the overall FRF, we first find the impedance of the 2nd RC circuit in
  parallel with the C of the first:
  \[ Z=C || (R+1/\j\omega C)=\frac{(R+1/j\omega C)/j\omega C}{(R+1/j\omega C)+1/j\omega C}
  =\frac{R+1/j\omega C}{j\omega RC+2} \]
  Then we find the FRF of the first RC circuit when the second RC circuit is treated as
  its load by voltage divider again:
  \[ H_1(j\omega)=\frac{Z}{R+Z}V_{in}
  =\frac{\frac{R+1/j\omega C}{j\omega RC+2}}{R+\frac{R+1/j\omega C}{j\omega RC+2}}
  =\frac{R+1/j\omega C}{j\omega R^2C+2R+R+1/j\omega C}
  =\frac{j\omega\tau+1}{(j\omega\tau)^2+j3\omega\tau+1}   
  =\frac{j\omega\tau+1}{1-(\omega\tau)^2+j3\omega\tau }
  \]
  Finally we get the overall FRF with the second RC circuit treated as a voltage divider:
  \[ H(j\omega)=H_1(j\omega) H_C(j\omega)
  =\left(\frac{j\omega\tau+1}{1-(\omega\tau)^2+j3\omega\tau}\right)\;
  \left(\frac{1}{j\omega\tau+1}\right)=\frac{1}{1-(\omega\tau)^2+j3\omega\tau}  \]
  Comparing this with $H^2_C(j\omega)$ above, we see that they are not the same.
  This is because the 2nd RC circuit will change the behavior of the first RC circuit, 
  i.e., the origianl FRF $H_C(j\omega)$ is no longer valid once the 2nd RC circuit is
  connected to the 1st as its load. If the overall $H(j\omega)$ is desired to be 
  $H^2_C(j\omega)$, an follower or buffer opamp circuit can be inserted in between 
  the two RC circuits to isolate the two.

  {\bf Moral of this problem:}

  {\bf In general, when two circuits with FRTs $H_1(j\omega)$ and $H_2(j\omega)$ are 
    cascaded together, the overall FRF $H(j\omega)\ne H_1(j\omega) H_2(j\omega)$,
    unless the second circuit has infinite input impedance, i.e., it does not affect
    the behavior $H_1(j\omega)$ of the first circuit.}

  {\bf As the follower (buffer) opamp circuit ideally has infinite input impedance and 
    zero output impedance, it does not affect the previoius circuit as its source and 
    it is not affected by the subsequent circuit as its load. Therefore inserting the 
    follower circuit between the two circuits with $H_1(j\omega)$ and $H_2(j\omega)$ 
    will isolate the two so that the overall FRF is $H(j\omega)=H_1(j\omega)H_2(j\omega)$.}

\end{enumerate}

\end{document}

