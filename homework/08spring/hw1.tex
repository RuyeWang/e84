\documentstyle[11pt]{article}
\usepackage{html}
\begin{document}
\begin{center}
{\Large \bf E84 Homework 1}
\end{center}
\begin{enumerate}

\item Given the basic relationship between the voltage across and
  current through each of the three types of components $R$, $C$,
  and $L$,
  \begin{tabular}{l|cc}
    Resistor $R$ & $i=v/R=Gv$ & $v=Ri=i/G$ \\
    Inductor $L$ & $i=\int v\,dt/L$ & $v=L\;di/dt$ \\
    Capacitor $C$ & $i=C\,dv/dt$ & $v=\int i\,dt/C$ \\
  \end{tabular}
  \begin{itemize}
  \item derive the expression for the equivalent resistance $R_s$
    of $n$ resistors $R_1,\cdots,R_n$ combined in series. Then derive 
    the expression for the equivalent resistance $R_p$ of the $n$ 
    resistors combined in parallel.
  \item Repeat the above for $n$ capacitors $C_1,\cdots,C_n$.
  \item Repeat the above for $n$ capacitors $L_1,\cdots,L_n$.
  \end{itemize}

%\item The electron gun of a cathode-ray tube (CRT) generates a beam
%  of electrons which is accelerated by a potential difference of
%  20,000 V over a distance of 4 cm. Find 
%\begin{enumerate}
%\item the strength of the electric field, and
%\item the power supplied to a beam of $50\times 10^{15}$ electrons
%  in one second.
%\end{enumerate}

\item (a) If two light bulbs both labeled as 110V and 40W in series are
connected to a socket outlet of 190V, what is the power consumption of 
each of the bulbs?

% R=V^2/W=110^2/40, V'=190/2=95, 
% W'=V'^2/R=W (V'/V)^2=40 (95/110)^2=40x0.746=29.83

(b) Replace one of the two bulbs by another bulb labeled as 110V 15W, and
find the power consumption of each of the bulbs. What will happen to each
of the two bulbs? (Note that when the power consumption by a bulb is larger
than the specified wattage, it will be burned out!)

% V=110, V'=190, R1=V^2/W1=110^2/40, R2=V^2/W2=110^2/15, 
% I=V'/R=V'/(R1+R2), W1'=I^2 R1=8.89, W2'=I^2 R2=23.67>15,  burned out!

\item Consider the circuit on the left. Give the expressions of voltage 
  $V_1$ and $V_2$ across $R_1$ and $R_2$, respectively, in terms of $R_1$ 
  and $R_2$ as well as the voltage source $V$.

  In the circuit on the right, give the expressions of the voltages across 
  $R_2$ and $R_4$ in terms of the circuit parameters ($R_1$ through $R_4$ 
  as well as the voltage source $V$).

  \htmladdimg{../../../lectures/figures/VDivider.png}

%  {\bf Solution:}
%  \[
%  V_2=\frac{R_2}{R_1+R_2}\;V
%  \]
%  \begin{eqnarray}
%  V_2&=&\frac{R_2||(R_3+R_4)}{R_2||(R_3+R_4)+R_1}\,V
%  =\frac{R_2(R_3+R_4)/(R_2+R_3+R_4)}{R_2(R_3+R_4)/(R_2+R_3+R_4)+R_1}\,V
%  \nonumber\\
%  &=&\frac{R_2(R_3+R_4)}{R_2(R_3+R_4)+R_1(R_2+R_3+R_4)}\,V
%  =\frac{R_2}{R_1+R_2+R_1R_2/(R_3+R_4)}\,V
%  \nonumber
%  \end{eqnarray}

%  \[
%  V_4=V_2\,\frac{R_4}{R_3+R_4}=\frac{R_2R_4}{(R_1+R_2)(R_3+R_4)+R_1R_2}\,V
%  \]

\item Measurement of a physical process by instruments may be tricky due 
  to the inevitable interfere on the process being caused by the instruments
  (remember what you learned in quantum mechanics?). The figure below shows 
  two possible configurations for the measurement of the voltage across and 
  the current through the load. 

  \htmladdimg{../hw1b.gif}

\begin{enumerate}
\item What are required of the ammeter and the voltmeter to minimize their
  influences on the measurements? 

% The ammeter should have minimum (ideally 0) impedance while the voltmeter
% should have maximum (ideally infinity) impedance. 

\item How would the ammeter and the voltmeter affect the measurement of the
  current and the voltage in either of the configurations (a and b)?
\end{enumerate}

% In (a) the voltmeter will by-pass some current so that the actual current
% through the load is smaller than the reading of the ammeter. In (b) the
% ammeter will cause some voltage drop and the actual voltage across the load
% is lower than the reading of the voltmeter. 

\item Use Kirchhoff's voltage and current laws to find voltage $V_{BD}$ and 
  resistance $R_2$ in the circuit shown below:

\htmladdimg{../hw1c.gif}

(Note: The direction of a current and the polarity of a voltage source can
be assumed arbitrarily. To determined the actual direction and polarity, the
sign of the values also should be considered. For example, a current labeled 
in left-to-right direction with a negative value is actually flowing 
right-to-left.)

% apply KCL to node B: 2+(-6)+I2=0, I2=-4, 
% apply KCL to node C: -4+1-I3=0, I3=-3;
% apply KCL to node D: 5-3+I4=0, I4=-2;
% apply KVL to loop ABCD: 10+2x1+(-4)xR2+6+(-3)x1-(-2)x5=0, R2=6.25
% apply KVL to loop BDA: V_{BD}+2x5+10+2x1=0, V_{BD}=-22V, or
% apply KVL to loop BDC: V_{BD}-(-3)x1-6-(-4x6.25)=0, V_{BD}=-22V

% \item Assume in the circuits in the figure, the voltage source is $V_0=2V$,
%	the current source is $I_0=1A$, and the resistor is $R=3\Omega$.
%	For both (a) and (b).
%
%\htmladdimg{../hw1d.gif}

%Find:
%\begin{enumerate}
%\item the current going through each of the three elements

% (a) I=1A going thru all 3 elements 
% (b) 1A thru current source, 2/3A thru R, 1/3 thru voltage source.

% \item the voltage across each of the three elements

% (a) 2V across voltage source, 3V across R, 5V across current source
% (b) 2V across all 3 elements.

%\item which of the power sources is delivering power and how much? which is 
%	receiving power and how much? How much power is dissipated by the
%	resistor $R$?

% (a) current source delivers 5W, R dissipates 3W, voltage source receives 2W
% (b) current source delivers 2W, R dissipates 4/3W, voltage source receives 2/3W
% When receiving power, the voltage source is being charged.
%\end{enumerate}

%\item A current flowing in an initially uncharged 1-$\mu$F capacitor is shown
%  in the figure as a function of time. Find and plot the voltage across the 
%  capacitor produced by this current.

%\htmladdimg{../hw1a.gif}


\item Find the equivalent resistance between 
the two terminals before and after the switch is closed. (Note, the two 
diagonal branches are NOT connected to each other in the middle.)

\htmladdimg{../hw2e.gif}

\item (Optional, extra credits) Find the equivalent resistance $R_{eq}$ 
between the two terminals in the figure, where $R_0=3\Omega$, $R_1=4\Omega$, 
$R_2=2\Omega$, $R_3=2\Omega$, $R_4=1\Omega$. What is $R_{eq}$ if $R_0=5\Omega$?

(Hint: apply a test voltage $V_{test}$ across the terminals and the 
equivalent resistance can be found to be $R_{eq}=V_{test}/I_{test}$.
The circuit can be solved by applying KCL to $V_1$ and $V_2$.)

\htmladdimg{../hw1f.gif}

%{\bf Solution:}
%\[ \frac{v_1-v_{test}}{4}+\frac{v_1-0}{2}+\frac{v_1-v_2}{3}=0 \]
%\[ \frac{v_2-v_{test}}{2}+\frac{v_2-0}{1}+\frac{v_2-v_1}{3}=0 \]
%solving these we get $v_1=v_2=v_{test}/3$. Then apply KCL to the
%positive terminal to get
%\[ i_{test}+\frac{v_1-v_{test}}{4}+\frac{v_2-v_{test}}{2}=0 \]
%solving to get
%\[R_{eq}=\frac{v_{test}}{i_{test}}=2 \Omega \]
%Note that in this case $v_1=v_2$, independent of the value of $R_0$.

%Alternatively, based on delta-Y conversion (to be considered later), the 
%triangle (delta) formed by $R_0$, $R1$, and $R_2$ can be coonverted to a Y 
%configuration with $R'_0=8/9$ (top), $R'_1=4/3$ (left), and $R'_2=2/3$. Then
%we have $R'_1+R_3=10/3$ and $R'_2+R_4=5/3$. Their parallel combination is 
%$10/3 || 5/3=10/9$, in series with $R'_0=8/9$, i.e., the total resistance
%Tnis $R_{eq}=10/9+8/9=2$. However, this approach does not reveal the fact that 
%$R_0$ can take any value without changing $R_{eq}$.

%\item The homework problem on \htmladdnormallink{this page}{http://fourier.eng.hmc.edu/e84/lectures/ch1/node6.html}.

\item Design a multimeter that can measure both DC and AC voltage, DC current,
  and resistance with different scales. Specifically, you are given an analog 
  meter $A$ with a needle display, which reaches full scale when a DC current 
  of $I=100\;\mu A=10^{-4}\;A$ goes through it. The internal resistance of the
  meter is 10 Ohms. In addition, you need some multi-position rotary switches 
  to select different scales for each of the three types of measurements, and 
  resistors with any values needed in your design.

  \begin{itemize}
    \item DC Voltage measurement: DC voltages in these ranges can be measured
      0-2.5, 0-10, 0-50, and 0-250 (all in volts). Use a 4-position rotary switch
      to select one of the four ranges as shown in the figure below. For example, 
      when the range of 0-10 is selected, the needle display will reach full scale 
      when the voltage being measured is 10 V. The circuit is shown below. Determine 
      all resistances labeled.

      \htmladdimg{../multimeterV.gif}

%      {\bf Solution:} $R_1=25\;K\Omega$, $R_2=75\;K\Omega$, $R_3=400\;K\Omega$
%      and $R_4=2\;M\Omega$.
    \item AC Voltage measurement: To measure an AC voltage (in terms of its
      RMS value), it first needs to be converted into a DC voltage. This can 
      be achieved by a 
      \htmladdnormallink{diode}{../../../lectures/ch4/node2.html}
      which only allows the current to pass in one
      direction (along the arrow) but not the other. This process is called
      rectification. 

      \htmladdimg{../../../lectures/figures/halfwaverectifier.gif}

      The diode will also cause a voltage drop of 0.7 volt 
      along the direction. The actual reading of the meter reflects the 
      \htmladdnormallink{average value}{../averagevalue/node1.html}
      of the rectified current. Find the resistance $R$ so 
      that when the incoming AC voltage is $V=10$ volt (RMS), the meter shows 
      a full scale display.

      \htmladdimg{../multimeterACV.gif}

%      {\bf Solution:} The peak value of voltage is $10\times \sqrt{2}=14.14$,
%      which is reduced (due to voltage drop of the diode) to $14.14-0.7=13.44$.
%      The average value of the rectified voltage is $13.44/\pi=4.28$. For the
%      meter to have a full scale display, the current need to be
%      \[ 10^{-4}=\frac{4.28}{R},\;\;\;\;\;\;\mbox{i.e.}\;\;\;\;\;\;R=42.8\,k\Omega \]
%      The internal resistance of $10\,\Omega$ can be neglected.

    \item DC current measurement: measure currents in these ranges (all in mA):
      0-0.5, 0-2.5, 0-10, 0-50. Use a 4-position rotary switch to select one 
      of the four ranges as shown in the figure below. For example, when the 
      range of 0-10 is selected, the needle display will reach full scale when 
      a 10 mA current is measured. Determine all resistances labeled. Use
      $R_0=1\;K\Omega$.

      \htmladdimg{../multimeterA1a.gif}

%      {\bf Solution:} Voltage across input is $V=0.1\;mA\times 1\;K\Omega=100\;mV$.
%      Therefore 
%      \[ R_1=100/(0.5-0.1)=250\;\Omega \]
%      \[ R_2=100/(2.5-0.1)=41.67\;\Omega \]
%      \[ R_1=100/(10-0.1)=10.1\;\Omega \]
%      \[ R_1=100/(50-0.1)=2\;\Omega \]

    \item Resistance measurement: The circuit for resistance measurement is
      provided as shown below, where $V_1=1.5V$. Determine the values for the 
      resistors labeled as $R_0$, $R_1$, $R_{10}$, $R_{100}$ and $R_{1000}$ 
      and $V_2$ so that the needle display of the meter is full scale 
      ($I=100\;\mu A$) when the resistor $R=0$ being measured (between the 
      two leads labeled + and -) is zero, or half scale ($I=50\;\mu A$) when 
      the value of $R$ and the position of he two synchronized rotary switches
      are given in each of the four case shown in the table:

      \begin{tabular}{l|llll} \hline
	positions  & $\times 1$ & $\times 10$ & $\times 100$ & $\times 1K$ \\
	$R$ values & 20$\Omega$ & 200$\Omega$ & 2000$\Omega$ & 20 $K\Omega$ 
      \end{tabular}

      \htmladdimg{../multimeterR.gif}

%      {\bf Solution:}
%
%      \begin{itemize}
%	\item First determine $R_0$: when $R=0$, we get 
%	  $R_0=1.5V/0.1\;mA=15\;K\Omega$.
%	\item When $R=20\;\Omega$, the current through meter $A$ should be:
%	  \begin{eqnarray}
%	    I&=&\frac{V_1}{R+R_1||R_0}\frac{R_1}{R_1+R_0}
%	  =\frac{V_1}{R+\frac{R_1R_0}{R_1+R_0}}\frac{R_1}{R_1+R_0}
%	  =V_1\frac{R_1}{RR_1+RR_0+R_1R_0} 	  \nonumber \\
%	  &=&5\times 10^{-4}\;A 
%	  \nonumber
%	  \end{eqnarray}
%	  Given $V_1=1.5V$, $R_0=15K$ and $R=20\Omega$, we can solve this 
%	  equation to get 
%	  \[ R_1=\frac{R_0RI}{V_1-I(R+R_0)}=20\Omega \]
%	\item When $R=200\;\Omega$, solving the above equation we get 
%	  $R_{10}=202.7\Omega$.
%	\item When $R=2\;K\Omega$, solving the above equation we get 
%	  $R_{10}=2307.7\Omega$.
%	\item When $R=20\;K\Omega$, we need to determine $V_2$ and 
%	  $R_{1k}$ so that $I=5\items 10^{-5} A$, and also when $R=0$, 
%	  $I=10^{-4}A$. $R_{1k}$ and $V_1+V_2$ can be found by solving
%	  these equations:
%	  \[ \left\{ \begin{array}{l}
%	    (V_1+V_2)/(R_{1k}+R_0)=10^{-4}A \\
%	    (V_1+V_2)/(R+R_{1k}+R_0)=5\times 10^{-5}A
%	  \end{array}\right. \]
%	  Solving we get $V_1+V_2=2V$, $R_{1k}+R_0=20\;K\Omega$, i.e.,
%	  $R_{1k}=5K\Omega$, $V_2=0.5V$
%      \end{itemize}
     
  \end{itemize}

\end{enumerate}
\end{document}

