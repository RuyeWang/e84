\documentstyle[11pt]{article}
\usepackage{html}
\begin{document}
\begin{center}
{\Large \bf E84 Home Work 7}
\end{center}
\begin{enumerate}

\item Find the current $i_1(t)$ through resistor $R_1$ after the
  switch is closed at $t=0$, assuming the input is a DC voltage and 
  the circuit is already in steady state before $t=0$.
  (Hint: current through an inductor cannot change instantaneously.)

  \htmladdimg{../hw5d.gif}

  {\bf Solution:}
 
  First, current through $L$ does not change instantaneously when the 
  switch is closed at $t=0$, and the voltage $v_L$ across $L$ (same as
  the voltage across $R_1$) is zero, tthe current through $R_1$ is zero:
  $i_1(0_+)=0$. Second, when the circuit reaches steady state after the 
  switch is closed at $t=0$, the inductor is a short-circuit, i.e., no 
  current goes through $R_1$. Therefore, $v_1(t)=0$ for all $t$.

\item In the circuit below, $V_s=6V$, $R_1=6\Omega$, $R_2=3\Omega$,
  $L=0.5H$, $I_s=2A$. Assume before the switch is closed at $t=0$, the
  system is already stablized. Find current $i_L(t)$ through $L$ and 
  voltage $v_{R_1}$ across $R_1$.

  \htmladdimg{../hw5a.gif}
 
  {\bf Solution:}
 
  $i(0)=V_s/R_1=6/6=1A$, $i(\infty)=V_s/R_1+I_s=1+2=3A$
  Find equivalent resistance (when both energy sources are turned off)
  $R=R_1 || R_2=3\times 6/(3+6)=2$, $\tau=L/R=0.5/2=0.25\;sec.$
  Final solution:
  $i_L(t)=3+(1-3)e^{-t/0.25}=3-2 e^{-4t} \; A$
  
  Find $v_{R_1}$: 
  $V_{R_1}(\infty)=6$, To find $V_{R_1}(0+)$, we combine the current source
  $I_S=2$ (up) in parallel with $L$ treated as a current source with 
  current 1 (down) as a current source of $I_L=1$ (up). Then the voltage 
  $V_{R_1}(0)$ across $R_1$ can be found by superposition. Due to $V_S$,
  $I'_{R_1}=-V_s/(R_1+R_2)=6/(3+6)=2/3\;A$ (up), due to $I_S$ (current
  divider with $R_2$), $I''_{R_1}=1/3\;A$ (down). Then 
  $I''_{R_1}(0)=V'_{R_1}+V''_{R_1}=1/3\;A$ (up), and $V_{R_1}=R_1I_{R_1}=6/3=2\;V$.
  Now we have
  \[ v_{R_1}=V_{R_1}(\infty)+[V_0-V_{R_1}(\infty)]e^{-t/\tau}=6+(2-6)e^{-4t}
  =6-4e^{-4t}	\]
  Alternatively, apply KVL to the loop of $V_s$, $R_1$ and $L$, and get
  \[ v_{R_1}=V_s-L di_L/dt=6-0.5 \frac{d(3-2 e^{-4t})}{dt}=6-4e^{-4t}	\]
  
\item In the circuit below, $V_s=12V$, $R_1=5\Omega$, $R_2=20\Omega$,
  $R_3=6\Omega$, $C_1=10\mu F$, $C_2=30\mu F$. Assume before the switch 
  is closed at $t=0$, the system is already stablized. Find voltages
  $v_1(t)$ and $v_2(t)$ across capacitors $C_1$ and $C_2$, respectively.
  (Hint, $C_1$ and $C_2$ are two capacitors in series with an equivalent
  capacitance is $C=C_1 C_2/(C_1+C_2)$. $C_1$ and $C_2$ have share the same
  time constant $\tau=RC$.)

  \htmladdimg{../hw5b.gif}

  {\bf Solution:}
 
  \[ v_1(0)=v_2(0)=0	\]
  \[ v_1(\infty)+v_2(\infty)=V_s \frac{R_2}{R_1+R_2}
  	=12 \frac{20}{5+20}=9.6V	\]
  As voltage across capacitor is inversely proportional to $C$ (same charge 
  $Q$), we have
  \[	\frac{v_1(\infty)}{v_2(\infty)}=\frac{C_2}{C_1}=3	\]
  i.e., $v_1(\infty)=3v_2(\infty)$, and we get $v_1(\infty)=7.2V$,
  $v_2(\infty)=2.4V$.
  Find equivalent resistance: 
  \[ R=R_3+\frac{R_1 R_2}{R_1+R_2}=6+\frac{5\times 20}{5+20}=10\Omega \]
  Find equivalent capacitance: $C_1 C_2/(C_1+C_2)=30\times 10/(30+10)=7.5$. 
  Find time constant: $\tau=RC=10\times 7.5\times 10^{-6}=7.5\times 10^{-5}$.
  Find $v_1(t)$ and $v_2(t)$:
  $v_1(t)=7.2(1-e^{-t/(7.5\times 10^{-5})})$
  $v_2(t)=2.4(1-e^{-t/(7.5\times 10^{-5})})$

\item In the circuit below, $R_1=100\Omega$, $R_2=150\Omega$, $R_3=100\Omega$,
  $L=0.1H$, $C=20\mu F$, $V=60V$. The circuit is in steady state initially when 
  the switch is at position 2 (not connected). Find $v_C(t)$ and $i_L(t)$ for the 
  following two independent cases:
  \begin{itemize}
    \item after the switch is changed to position 1 at $t=0$;
    \item after the switch is changed to position 3 at $t=0$;
  \end{itemize}
  

  \htmladdimg{../hw6e.gif}

  {\bf Solution}

  Find initial values:
  \[ v_C(0)=\frac{100}{150+100}\times 60=24V,\;\;\;\;\;\;\;i_L(0)
  =\frac{60}{150+100}=0.24A\]
  \begin{itemize}
    \item after the switch is put on position 1 at $t=0$, 
      \[ v_C(\infty)=60V,\;\;\;\;\;\;\;i_L(\infty)=60/100=0.6A \]
      Find the time constants:
      \[ \tau_1=R_1C=2\times 10^{-3} s,\;\;\;\;\;\;\tau_2=L/R_2=10^{-3} s \]
      The complete responses:
      \[ v_C(t)=v_C(\infty)+[v_C(0)-v_C(\infty)]e^{-t/\tau_1}=60-36 e^{-500 t},\;\;\;\;
	 i_L(t)=i_L(\infty)+[i_L(0)-i_L(\infty)]e^{-t/\tau_2}=0.6-0.36 e^{-1000 t} \]
    \item after the switch is put on position 3 at $t=0$, 
      \[ v_C(\infty)=0,\;\;\;\;\;\;\;i_L(\infty)=0 \]      
      The time constants are the same as above. The complete responses:
      \[ v_C(t)=v_C(\infty)+[v_C(0)-v_C(\infty)]e^{-t/\tau_1}=24 e^{-500 t},\;\;\;\;
	 i_L(t)=i_L(\infty)+[i_L(0)-i_L(\infty)]e^{-t/\tau_2}=0.24 e^{-1000 t} \]
  \end{itemize}

\end{enumerate}
\end{document}

\item An RCL parallel circuit containing $R=20\Omega$, $L=400 \mu H$,
$C=1 \mu F$ is driven by a current source $I=50 \mu A$ with variable 
frequency. Find resonant frequency, the quality factor, and the voltage
across and the currents through each of the three components at resonance.

% {\bf Solution:}
% 
%  \[ \omega_0=\frac{1}{\sqrt{LC}}=\frac{1}{\sqrt{10^{-6}\times 400 \times 10^{-6}}}
%  	=\frac{1}{20 \times 10^{-6}}=5 \times 10^4	\]
%  \[ Q=\frac{1}{G}\sqrt{\frac{C}{L}}=R\sqrt{\frac{C}{L}}
%  =20\sqrt{\frac{10^{-6}}{400\times 10^{-6}}}=1	\]
%  $v=i\times R=50 \mu A \times 20 \Omega=1 mV$, $i_R=50 \mu A$,
%  $i_L=i_C=Q i_R=50 \mu A$.
 
 
\item An RCL series circuit is driven by a voltage source $V=10V$ of
frequency $\omega=10^4 \;rad/sec.$. The variable capacitor $C$ is
adjusted so the the maximual current of $100\;mA$ is achieved. The
voltage across this capacitor is $600V$. Find the values for $R$, $L$,
$C$ and the quality factor $Q$.

% {\bf Solution:}
% 
%  At resonance, $v_L=-v_C=-600V$. $R=V/I=10V/0.1A=100\Omega$, $v_R=V=10V$.
%  $Q=600/10=60$.  
%  \[ \frac{1}{R}\sqrt{\frac{L}{C}}=Q=60	\]
%  \[ \frac{1}{\sqrt{LC}}=\omega_0=10000 \]
%  Solve for $L$ and $C$: $L=0.6H$, $C=10^{-7}/6 F$


\item A series circuit composed of a capacitor and an inductor is to be 
resonant at 800 kHz with voltage input. Specify the value of $C$ for the 
capacitor required for the given inductor with $L=40\mu H$ and an internal 
resistance $R_L=4.02\Omega$, and predict the bandwidth. Assume the capacitor 
is ideal, i.e., it introduces no resistance.


%{\bf Solution:}
% As $\omega_0=1/\sqrt{LC}=2\pi 8\times 10^5$, and $L=4\times 10^{-5}H$, we
% can find $C$ to be
% \[
% C=\frac{1}{\omega_0^2 L}=\frac{1}{(2\pi 8\times 10^5)^2\times 4\times 10^{-5}}
% =0.99\;nF	\]
% Next find the quality factor:
% \[
% Q=\frac{\omega_0L}{R}=\frac{2\pi 8\times 10^5\times 4\times 10^{-5}}{4.02}=50
% \]
% then the bandwidth is
% \[	f_2-f_2=\frac{f_0}{Q}=\frac{800\times 10^3}{50}=16\;kHz	\]

\item In reality, all inductors have a non-zero resistance, therefore a 
parallel resonance circuit should be modeled as shown in the figure:

\htmladdimg{../../lectures/figures/parallelRCL.gif}

Find the expression of the resonant frequency of this circuit. (Hint,
similar to our discussion of parallel and series RCL circuits in class, 
the resonant frequency $\omega_0$ is achieved when the imaginary part of
the impedance (or admittance) is zero.) If it is given that $R=5\Omega$,
$C=100\;\mu F=100\times 10^{-6}F$ and $L=5\; mH=5\times 10^{-3}H$, what
is the resonant frequency of this circuit? Does the current drawn from 
the voltage source reach maximum or minimum at this frequency?

% {\bf Solution:} The admittance is:
% \[	Y=\frac{1}{R+j\omega L}+j\omega C
% 	=\frac{R-j\omega L+j\omega C(R^2+\omega^2L^2)}{R^2+\omega^2L^2}
% \]
% When the imaginary part of $Y$ is zoro, the circuit is at resonance and 
% the resonant frequency $\omega_0$ can be found by solving $Im[Y]=0$:
% \[	\omega_0 L=\omega_0 C(R^2+\omega_0^2L^2)	\]
% Solving this for $\omega_0$, we get:
% \[	\omega_0=\sqrt{\frac{1}{LC}-(\frac{R}{L})^2}
% 	=\sqrt{\frac{1}{5\times 10^{-3}\times 100\times 10^{-6}}
% 	-(\frac{5}{5\times 10^{-3}})^2}=1000\;Hz
% 	\]
% As the admittance reaches minimum at $\omega=\omega_0$, the current drawn
% from the voltage source $\cdot{I}=Y\cdot{V}$ reaches minimum, i.e., it 
% behaves as a band stop filter.
% 
% Also note that for $\omega_0$ to be real, we must have
% \[	\frac{1}{LC} > (\frac{R}{L})^2, \;\;\;\;\;\;\mbox{i,e,}
% 	\;\;\;\;\;\;R<\sqrt{\frac{L}{C}}	\]
% When $R \ll \sqrt{L/C}$, the resonant frequency is
% \[
% \omega_0=\sqrt{\frac{1}{LC}-(\frac{R}{L})^2}\approx \frac{1}{\sqrt{LC}}	
% \]

%\item Find the output voltage $v_{out}(t)$ when $\omega=0$ and 
%$\omega\rightarrow \infty$ when the input $v_{in}(t)=10\;cos(\omega t)$, 
%assuming $R_1=100\Omega$, $R_2=100\Omega$, $C=10\mu F$ and $L=10\;mH$.
%
%\htmladdimg{../hw6a.gif}
%
% {\bf Solution}
% 
% When $\omega=0$, the inductor is short circuit, and the capacitor is
% open circuit, $v_{out}(t)=v_{in}(t)$.
% When $\omega=0$, the inductor is open circuit, and the capacitor is
% short circuit, $v_{out}(t)=v_{in}(t)/2=5\;cos(\omega_0 t)$.


\item Design a parallel circuit to be resonant at 800 kHz with a bandwidth
of 32 kHz. The inductor has $L=40 \mu H$ and $R_L=4.02 \Omega$. Find the
capacitance $C$ needed for the desired resonant frequency. In order to
satisfy the desired bandwidth, you may also need to include a resistor 
in the circuit. (Hint, note that if the quality factor of the circuit
$Q=\omega_0 L/R > 20$, all relations for resonant circuit with ideal
inductor still hold.)


% {\bf Solution:} Based on the desired resonant frequency and bandwidth, 
% the quality factor needs to be
% \[ Q=\frac{f_0}{\Triangle f}=\frac{800\times 10^3}{32 \times 10^3}=25 \]
% Since $Q>20$, the resonant frequency is approximately
% \[ \omega_0=\frac{1}{\sqrt{LC}},\;\;\;\;\mbox{i.e.,}\;\;\;\;
% C=\frac{1}{\omega_0^2 L}=\frac{1}{(2\pi 8\times 10^5)^2\times 4\times 10^{-5}}
% =0.99 nF \]
% However, the quality factor of the parallel circuit is
% \[ Q=\frac{\omega_0 L}{R}=\frac{2\pi 8 \times 10^5 \times 4\times 10^{-5}}
% 	{4.02}=50 \]
% twice the desired $Q=25$, we have to double the resistance $R=4.02$ to 
% $R=8.04$ to reduce $Q$ by half.

\item The function of a loudspeaker crossover network is to channel 
frequencies higher than a given crossover frequency $f_c$ into the
high-frequency speaker (``tweeter'') and frequencies below $f_c$ into
the low-frequency speaker (``woofer''). One such circuit is shown below.
Assume the resistances of the tweeter is $R_1=8\Omega$ and that of the 
woofer is $R_2=8\Omega$, the voltage amplifier can be modeled as an
ideal voltage source, and the crossover frequency is $f_c=2000\; Hz$.
Design the network in terms of $L$ and $C$ so that $f_c$ is the corner
frequency or half-power point of each of the two speaker circuits. Give 
the expression of the power $P_1(f)$ and $P_2(f)$ of the speakers as a 
function of frequency $f$ and crossover frequency $f_c$, and sketch them.
Assume the RMS of the input voltage is 1V.

\htmladdimg{../hw6b.gif}

% {\bf Solution}
% 
% The RMS voltage across the tweeter is
% \[	V_1=|\frac{R}{1/j2\pi f C+R}|
% 	=\frac{2\pi f CR}{\sqrt{1+(2\pi f CR)^2}}	\]
% If $f=f_c=2000$ is at half-power point ($V_1=V_{in}/\sqrt{2}$), the real and
% imaginary parts of the denominator should be equal and we get 
% \[	\frac{1}{j2\pi f C}=R;\;\;\;\mbox{i.e.}\;\;\;\;
% 	C=\frac{1}{2\pi f_c R}=9.95 \;\mu F	\]
% The RMS voltage across the woofer is
% \[	V_2=|\frac{R}{j2\pi f L+R}|
% 	=\frac{R}{\sqrt{R^2+(2\pi f L)^2}}	\]
% If $f=f_c=2000$ is at half-power point ($V_2=V_{in}/\sqrt{2}$), the real and
% imaginary parts of the denominator should be equal and we get 
% we get 
% \[	j2\pi f L=R;\;\;\;\mbox{i.e.}\;\;\;\;
% 	L=\frac{R}{2\pi f_c}=0.637 \;mH	\]
% The power plots:
% \[	P_1(f)=\frac{V_1^2(f)}{R}=\frac{1}{8}(\frac{1}{1+(f_c/f)^2}) \]
% \[	P_2(f)=\frac{V_1^2(f)}{R}=\frac{1}{8}(\frac{1}{1+(f/f_c)^2}) \]


\end{enumerate}
\end{document}


\item The load of a voltage soruce of $v(t)=110\sqrt{2} \;cos(2\pi 60\;t)$
is shown in the figure, where $R_1=100\Omega$, $R_2=50\Omega$, $C=6.63\mu F$, 
$L=0.53 H$. Is the load capacitive ($\phi<0$) or inductive ($\phi>0$)?
Find the power factor, the apparent power, the real power and the
reactive power. To improve the power factor to 0.9, a shunt capacitor is added.
What should the capacitance $C$ be? What should $C$ be if the power factor is 
required to be 1?

\htmladdimg{../hw6c.gif}

% {\bf Solution:}
% 
% \[ Z_C=-\frac{j}{\omega C}=-\frac{j}{2\pi 60\times 6.63\times 10^{-6}}
% 	=-j400\Omega\]
% \[Z_L=j\omega L=j 2\pi 60\times 0.53=j200\Omega,\;\;\;\; Z_{RL}=100+j200\]
% 
% \[ Z_{CRL}=\frac{Z_C Z_{RL}}{Z_C+Z_{RL}}=\frac{(100+j200)(-j400)}{100+j200-j400}
% 	=\frac{800-j400}{1-j2}	\]
% \[
% Z_{total}=Z_{CRL}+R_2=\frac{800-j400}{1-j2}+50=370+j240=441\angle 33^\circ
% \]
% The load is inductive as $\phi>0$.
% \[ \dot{I}=\frac{\dot{V}}{Z_{total}}=\frac{110}{441\angle 33^\circ}
% 	=0.25\angle -33^\circ	\]
% power factor is $\lambda=cos (-33^\circ)=0.839$, 
% the apparent power is $S=110\times 0.25=27.5 W$, 
% the real power is $P=S \cos 33^\circ=27.5\times 0.84=23 W$
% the reactive power is $Q=S \sin 33^\circ=27.5\times 0.54=15 W$
% Adding a shunt capacitor with impedance $1/j\omega C=-jX$ ($X=1/\omega C$), 
% the overall load impedance is
% \[	Z_{all}=-jX || Z_{total}=\frac{-jX(370+j240)}{-jX+(370+j240)}
% 	=\frac{240X-j370X}{370-j(X-240)}=|Z|\angle Z=|Z|\angle \phi	\]
% For the power factor to be 0.9, this impedance need to have a phase angle 
% $\phi=\cos^{-1} 0.9=25.84^\circ$, and we need to have:
% \[	\tan^{-1}[\frac{-370X}{240X}]-\tan^{-1}[\frac{-(X-240)}{370}]=
% 	-57^\circ+\tan^{-1}[\frac{X-240}{370}]=25.84^\circ \]
% \[	\tan^{-1}[\frac{X-240}{370}]=82.84^\circ, \;\;\;
% \frac{X-240}{370}=\tan \;82.84^\circ=7.96, \;\;\; X=3185.4 \]
% \[ \frac{1}{\omega C}=X=3185.4,\;\;\;\;C=\frac{1}{2\pi 60\times 3185.4}
% =0.83 \mu F \]
% For the power factor to be 1, we need to have
% \[	\tan^{-1}[\frac{-370X}{240X}]-\tan^{-1}[\frac{-(X-240)}{370}]=
% 	-57^\circ+\tan^{-1}[\frac{X-240}{370}]=0^\circ \]
% i.e., 
% \[	\tan^{-1}[\frac{X-240}{370}]=57^\circ, \;\;\;
% \frac{X-240}{370}=\tan \;57^\circ=1.54, \;\;\; X=810 \]
% \[ \frac{1}{\omega C}=X=810,\;\;\;\;C=\frac{1}{2\pi 60\times 810}
% =3.27 \mu F \]

\item In the parallel resonant circuit shown in the figure, $L=0.25\;mH$,
$R=25\Omega$, $C=85\;pF$. Find resonant frequency, and the impedance at r
esonance.

\htmladdimg{../hw5c.gif}

\[	Y=\frac{1}{R+j\omega L}+j\omega C=	\]



