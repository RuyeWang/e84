\documentstyle[11pt]{article}
\usepackage{html}
\begin{document}
\begin{center}
{\Large \bf E84 Homework 3}
\end{center}
\begin{enumerate}

\item While various voltage sources such as batteries are very common 
in everyday life, current sources are not widely available. One type of
current source is the photocells, which generates current proportional 
to the intensity of the incoming light. Also, certain specially designed
transistor circuits can generate to output constant current. Moreover, 
as discussed in class, any current source can be obtained by converting
a corresponding voltage source. Design a current source with $I_0=10$ mA 
and $R_0=1\;K\Omega$ by converting a voltage source. Find its voltate 
$V_0$ and internal resistance $R_0$. 

% {\bf Solution:} $R_0=1\;K\Omega$, $V_0=I_0\times R_0=10\;mA \times 1000\;
% \Omega=10\;V$.

\item The circuit below (assuming $I_0=1A$, $R=3\Omega$, $V_0=2V$) was
discussed in class in terms of power delivery, absoption or dissipation 
of each element in the circuit. Now redo the problem assuming the polarity
of the voltage source is reversed (negative on top). 

\htmladdimg{../../../lectures/figures/powerdeliveryexample.gif}

% {\bf Solution:} The current through the resistor (upward( is $2/3\; A$,
% the current through the voltage source (downward) is $1+2/3=5/3\;A$. 
% The power delivered by the voltage source is therefore $5/3\times 2=10/3$
% Watts, the power dissipated by the resistor is $(2/3)^2\times 3=4/3$ 
% Watts, and the power absorbed by the current source is 2 Watts, i.e.,
% $10/3=2+4/3$.

\item The output resistance of the power amplification circuit of
  a Hi-Fi system is $R_{out}=8\Omega$ and the output voltage is 
  $V_{out}=20V$. Find the power received by the speaker, the total
  power consumption, and the power efficiency of the circuit, for 
  each of the three speakers with resistance: $R_L=4\Ogema$, 
  $R_L=8\Ogema$, and $R_L=16\Ogema$

% {\bf Solution:} The power received by the load (speaker) is 
% \[ P_L=\frac{V_{out}^2}{(R_{out}+R_L)^2} R_L \]
% and the power efficiency is
% \[ \eta=\frac{P_L}{P_{total}}=\frac{R_L}{R_{out}+R_L} \]
% where $V_{out}=20V$, $R_{out}=8\Omega$. 
% \begin{itemize}
% \item When $R_L=4\Omega$, $P_L=100/9=11.1$ Watts, $\eta=1/3$, 
%   the total power is $P_{total}=100/3=33.3$ Watts.
% \item When $R_L=8\Omega$, $P_L=100/8=12.5$ Watts, $\eta=1/2$,
%   the total power is $P_{total}=100/4=25$ Watts.
% \item When $R_L=16\Omega$, $P_L=100/9=11.1$ Watts, $\eta=2/3$,
%   the total power is $P_{total}=100/3=16.7$ Watts.
% \end{itemize}

\item Find the optimal load resistance $R_L$ so that it receives maximal
power from the current source $I_0=10A$ with internal resistance 
$R_0=1\Omega$ and power transmission line resistance $R_T=9\Omega$. 
Find the maximum load power and the power loss along the transmission 
line.

\htmladdimg{../hw2f.gif}

To verify your choice of load resistance, show that the power consumption
of the load will always be lower than this maximum when its resistance is 
either increased or decreased by ten percent.

% {\bf Solution:}

%  First convert current source to voltage source with $V_0=I_0 R_0=10V$
%  and $R_0=1 \Omega$. To maximize load power consumption, let 
%  $R_L=R_0+R_T=10 \Omega$. The current is $I=10V/20\Omega=0.5A$. Load power 
%  is $I^2R_L=10/4=2.5W$, power loss on transmission line is $I^2R_T=9/4A$
%  When $R_L=11\Omega$, $I=10V/21\Omega$, $W_L=I^2 R_L=2.494$
%  When $R_L=9\Omega$, $I=10V/19\Omega$, $W_L=I^2 R_L=2.493$

\item Convert the following circuit into (a) an equivalent current 
  source $(I_{cs}, R_{cs})$ and then (b) an equivalent voltage source
  $(V_{vs}, R_{vs})$. Give an expression for the load $R_L$ so that it 
  will receive maximum power from the source.

\htmladdimg{../../../lectures/figures/source_parallel.gif}

% {\bf Solution:}
% 
% (a) Convert voltage source $(V_0,R_1)$ on the left to a current source
%     $(I'_0=V_0/R_1, R_1)$ in parallel with the current source $(I_0,R_2)$.
%     The overall current source is therefore:
% \[ I_{cs}=I_0+\frac{V_0}{R_1},\;\;\;\;\;
%    R_{cs}=R_1||R_2=\frac{R_1 R_2}{R_1+R_2} \]
% (b) Convert the overall current source above to a voltage source:
% \[ V_{vs}=I_{cs} R_{cs}=(I_0+\frac{V_0}{R_1})\frac{R_1 R_2}{R_1+R_2} 
%    =I_0\frac{R_1 R_2}{R_1+R_2}+V_0\frac{R_2}{R_1+R_2},\;\;\;\;
%    R_{vs}=R_{cs}
% \]
% (c) For $R_L$ to receive maximum power, we need $R_L=R_{cs}=R_{vs}=R_1||R_2$
   
\item Solve the homework problem (Example 4) on 
  \htmladdnormallink{this page}{http://fourier.eng.hmc.edu/e84/lectures/ch2/node2.html}.

\item Find all node voltages in the circuit with respect to the bottom node as
  ground, where $R_1=100\Omega$, $R_2=5\Omega$, $R_3=200\Omega$, $R_4=50\Omega$, 
  $V=50V$, $I=0.2A$.  Use both node voltage and loop current methods to solve 
  this circuit. Choose independent loops and nodes wisely to simplify your
  computation.  

  \htmladdimg{../../../lectures/figures/problembase1.gif}

\end{enumerate}
\end{document}

\item Find the current I3 through the load resistor of 6$\Omega$ by the
following two methods:
\begin{enumerate}
\item Convert the two voltage sources to equivalent current sources, 
	simplify the circuit and solve for the current;
\item Use superposition theorem to consider one of the two sources at a
  time, then add the two partial results to get the final result.

{\bf Hint: Superposition theorem}

When there exist multiple energy sources, the currents and voltages in 
the circuit can be found as the algebraic sum of the corresponding values 
obtained by assuming only one source at a time, with all other sources 
turned off (voltage sources treated as short circuit, current sources 
treated as open circuit).

\end{enumerate}

\htmladdimg{../hw2g.gif}

% convert first voltage source (left):   I1=10V/1 ohm=10A, R1=1
% convert second voltage source (right): I2=6V/3 ohm=2A, R2=3
% combine two parallel current sources: I0=I1+I2=12A
% combine two parallel resistors: R0=(R1 x R2)/(R1+R2)=3/4=0.75
% use current divider to find current:
%       I=I0 R0/(R+R0)=12 x 0.75/(6+0.75)=4/3=1.33A

\item Find the three currents labeled as I1, I2 and I3 in the same figure 
	above by the following method:
\begin{enumerate}
\item Apply KCL to node a to get a current equation;
\item Apply KVL to the two loops to get two voltage equations;
\item Solve the three equations for the unknown currents I1, I2 and I3.
\end{enumerate}

% I1+I2-I3=0, 10V-I1=6V+3I2, 10V-I1=6I3
% solve to get: I1=2, I2=-2/3, I3=4/3

\item Find all currents labeled $I_k (k=1,2,3,4,5)$ in the circuit below. 
The resistance of a resistor labeled by 10K, for example, is 10,000
Ohms ($10,000 \Omega=10K\Omega$).

{\bf Hint:} follow these steps:
\begin{itemize}
\item Choose node 0 as the reference point or ``ground'', and let the 
	voltages at nodes 1 and 2 be $V_1$ and $V_2$ (with respect to
	node 0), respectively. 
\item Express all currents in terms of the node voltages $V_1$, $V_2$
	and other known quantaties such as the current and voltage sources
	and the resistors.
\item Apply KCL to node 1 and node 2 to get two current equations, 
	substitute all current expressions into the two equations.
\item Solve the two equations for $V_1$ and $V_2$.
\item Find all currents.
\end{itemize}

\htmladdimg{../hw2h.gif}

% KCL equations: I1+I2+I3-I0=0, -I3+I4+I5=0
% I1=V1/R1, I2=V1/R2, I3=(V1-V2)/R3, I4=V2/R4, I5=(V2+V0)/R5
% substitute I1 through I5 into KCL equations and solve for V1 and V2 
% to get: V1=21.8V, V2=-21.8V
% obtain all currents: I1=1.09, I2=0.545, I3=4.36, I4=-1.09, I5=5.46

\item Find all currents I1 through I5 in the circuit below. 

{\bf Hint:} use the loop current method: 

\begin{itemize}
\item Assume the currents around the four loops are Ia, Ib, Ic and Id 
  with Ia=10A and Id=6.5A (due to the two current sources).
\item Apply KVL to the two middle loops to get two equations.
\item solve them for two unknown currents Ib and Ic, get I1 through I5.
\end{itemize}

\htmladdimg{../hw2i.gif}

% KVL: (2+3+4)Ib-2Ia-4Ic=-10, (4+3+2)Ic-4Ib-2Id=10
% solve to get Ib=2.8A, Ic=3.8A, then find I1 thru I5:
% I1=Ia-Ib=7.2A, I2=Ib-Ic=-1A, I3=Ic-Id=-2.7A, I4=Ib=2.8A, I5=Ic=3.8A


\item Find the optimal load resistance $R_L$ so that it receives maximal
power from the current source $I_0=10A$ with internal resistance 
$R_0=1\Omega$ and power transmission line resistance $R_T=9\Omega$. 
Find the maximum load power and the power loss along the transmission 
line.

\htmladdimg{../hw2f.gif}

To verify your choice of load resistance, show that the power consumption
of the load will always be lower than this maximum when its resistance is 
either increased or decreased by ten percent.

% First convert current source to voltage source with $V_0=I_0 R_0=10V$
% and $R_0=1 \Omega$. To maximize load power consumption, let 
% $R_L=R_0+R_T=10 \Omega$. The current is $I=10V/20\Omega=0.5A$. Load power 
% is $I^2R_L=10/4=2.5W$, power loss on transmission line is $I^2R_T=9/4A$
% When $R_L=11\Omega$, $I=10V/21\Omega$, $W_L=I^2 R_L=2.494$
% When $R_L=9\Omega$, $I=10V/19\Omega$, $W_L=I^2 R_L=2.493$




\end{enumerate}
\end{document}


\item Using Thevenin's theorem to determine the current in the $3-\Omega$ 
resistor of the following figure.

\htmladdimg{../hw2c.gif}

% move two voltage sources to left, and 3-$\Omega$ resistor to the right as load
% find equivalent voltage Vo and internal resistance and Ro:
% current going clockwise around the loop (without load): (30-24)/(6+12)=1/3
% voltages across 6 ohm resistor and 12 ohm resistor are -2V and 4V, Vo=30-2=24+4=28V
% Ro=6//12=4ohm, current through load resistor is 28/(4+3)=7A

\item Find voltage across and the current through the 10-$\Omega$ resistor.

\htmladdimg{../hw2d.gif}

% use superposition principle. When 24V is acting alone with 1A open, 
% parallel resistors 15 and 10 become 15//10=6, V'ab=24 x 6/(6+6)=12V
% I'=12/10=1.2A. When 1A is acting alone with 24V closed, parallel resistors
% 6 and 15 become 90/21, I''=10/(10+90/21)=3/10=0.3 A, V''ab=I'' x 10=3V
% overall V=V'+V''=12+3=15, I=I'+I''=1.2+0.3=1.5A


\item The circuit in the figure below has a voltage source $V_0=20V$ and a current 
source $I_0=3A$, and four resistors $R_1=20\Omega$, $R_2=10\Omega$, $R_3=30\Omega$ 
and $R_4=10\Omega$. Find the voltage across $R_4$ and the power dissipated by it.

\htmladdimg{../hw2a.gif}

%If voltage source acts alone (current source open), according to voltage divider,
%V'_4=V0 R4/(R2+R2) = 10V, power consumption is W'=V^2/R=10^2/10=10W
%If current source acts alone (voltage source short), according to current divider
%V''_4=(I0 R2/(R22+R4) R4=[3x10/(10+10)x10=15V, power consumption is W''=V^2/R=15^2/10=22.5W
%Total V_4=V'_4+V''_4=10+15=25V, total power W=25^2/10=62.5W not equal to 10+22.5=32.5

\item 

Find all currents in the diagram in which V=120V, $R_1=2\Omega$, $R_2=20\Omega$. 
{\bf Hint:} it is very hard to solve the problem by finding the currents in the
order of $I_1$, $I_3$, $I_5$, as computing the resistances of the resistor 
network is tedious. However, it is much more straight forward to find the 
currents in the order of $I_5$, $I_3$, $I_1$, if you assume $I_5$ is known, e.g.,
$I_5=1A$. However, the voltage for the voltage source obtained based on this 
assumption is of course not as given (120V). In this case, the linearity property
$F(ax+by)=aF(x)+bF(y)$ can be applied. In particular, given $y=F(x)$, then 
$ay=F(ax)=aF(x)$. Use this relationship to find the actual values of the 
currents.

\htmladdimg{../hw2b.gif}





\end{enumerate}
\end{document}

\item Find the equivalent resistance between the two terminals before and
after the switch is closed. (Note, the two diagonal branches are NOT
connected to each other in the middle.)

\htmladdimg{../hw2e.gif}

% before S is closed, $R=(3+2)//(6+2)=5//8=40/13 \Omega$
% after S is closed, $R=2//2+3//6=1+2=3 \Omega$

