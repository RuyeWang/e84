\documentstyle[11pt]{article}
\usepackage{html}
\begin{document}
\begin{center}
{\Large \bf E84 Home Work 5}
\end{center}
\begin{enumerate}

\item Example 5 at the end of \htmladdnormallink{this page}{http://fourier.eng.hmc.edu/e84/lectures/ch2/node5.html}

\item Example 7 at the end of \htmladdnormallink{this page}{http://fourier.eng.hmc.edu/e84/lectures/ch2/node5.html}

\item The non-inverting amplifier at the end of \htmladdnormallink{this page}{http://fourier.eng.hmc.edu/e84/lectures/opamp/node2.html}.

\end{enumerate}
\end{document}


\item The figure below shows a two-dimensional network of infinite number 
of 1$\Omega$ resistors forming a grid extending to infinity in all four 
directions. Find the resistance between any two neighboring nodes, e.g., 
nodes a and b.

\htmladdimg{../resistornetwork.gif}

{\bf Hint:} Add two 1A current sources to the two nodes, as shown in the 
figure, and use superposition theorem to find the current $i_{ab}$ from 
a to b, and thereby the voltage drop between a and b. Note that as the
network is infinite, the four branches of each node are totally symmetric
and therefore the currents through them are identical. The resistance 
between these nodes can be found by Ohm's law.

\item Convert the one-port circuit below to an equivalent Thevenin voltage
source, find $V_T$ and $R_T$. Find the voltage $V$ and current $I$ associated
with a load resistor with resistance $R_L=1\Omega, 2\Omega, 3\Omega$.

\htmladdimg{../theveninsource.gif}

\item Convert the one-port circuit below to an equivalent Norton current
source, find $I_N$ and $R_N$. Find the voltage $V$ and current $I$ associated
with a load resistor with resistance $R_L=1\Omega, 2\Omega, 3\Omega$.

\htmladdimg{../nortonsource.gif}

\item In the circuit below, each of the six branches has a $1\Omega$
resistor and a voltage source of unknow voltage and polarity. The current
from points a to b is known to be $I_{ab}=1A$. If the resistor between 
points a and b is changed from $1\Omega$ to $3\Omega$, what is the current
$I_{ab}$?

\htmladdimg{../trianglenetwork.gif}

\item Find the currents ($I_1$ through $I_5$) flowing through each of the 
five resistors ($R_1$ through $R_5$ in the circuit below (same as shown in
class), where $V_0=18V$, $R_1=R_2=3\Omega$, $R_3=5\Omega$, $R_4=1.5\Omega$
and $R_5=2\Omega$. (Hint: convert delta formed by by $R_1$, $R_2$ and $R_5$
to a Y composed of $R_a$, $R_b$ and $R_c$, where $c$ is the positive 
terminal of voltage source, find $V_a$ and $V_b$ and then all the currents.)

\htmladdimg{../bridgeproblem.gif}

%  (1) Convert the delta formed by $R_1$, $R_2$ and $R_5$ to a Y: 
%  $R_a=R_b=3/4$, $R_c=9/8$
% 
%  (2) Find overall load resistance of voltage source:
%  $R_{total}=R_c+(R_a+R_3)||(R_b+R_4)=9/8+(27/4)||(9/4)=45/16$
% 
%  (3) Find current through voltage source: $I_{total}=V_0/R_{total}=18/(45/16)=32/5$
% 
%  (4) Find currents through two branches by current dividor:
%  $I_a=I_3=(32/5)(9/4)/(9/4+27/4)=8/5$, $I_b=I_4=(32/5)(27/4)/(9/4+27/4)=24/5$
% 
% (5) Find voltages $V_a$ and $V_b$ with respect to negative terminal of
%   voltage source treated as ground: $V_a=R_3 I_a=48/5$, $V_b=R_4 I_b=36/5$
% 
%  (6) Find currents through $R_1$, $R_2$ and $R_5$:
%  $I_1=(18-48/5)/3=14/5$, $R_2=(18-36/5)/3=18/5$, $I_5=(V_a-V_b)/R_5=2.4/2=6/5$
% 
%  (7) Verify: $I_1+I_2=I_{total}=32/5$, $I_1-I_5=I_3=8/5$, $I_2+I_5=I_4=24/5$.

\end{enumerate}
\end{document}

\item The resistance $R$ of a circuit is a real value which can be measured
by a multimeter. However, the impedance $Z$ is complex which cannot be 
measured directly. Instead, one can use an oscilloscope to find sinusoidal 
voltage $v(t)$ across and current $i(t)$ through the circuit, and the obtain
the impedance as the ratio between the complex representations of the voltage
and current. Suppose we find:
\[ v(t)=12 cos(1000t-30^\circ),\;\;\;\; i(t)=6\;cos(1000t+15^\circ) \]
Find the impedance (both resistance and reactance) and the admittance (both
conductance and susceptance) of the circuit.

% Represent voltage and current in complex forms:
% $v(t)=Re[12e^{j(1000 t-30^\circ)}],\;\;\;\;i(t)=Re[6e^{j(1000 t+15^\circ)}]$
% $Z=2e^{-j45^\circ}=2\angle{-45^\circ}=\sqrt{2}-j\sqrt{2}$
% $R=\sqrt{2},\;\;\;\;X=-\sqrt{2}$.
% $Y=1/Z=G+jB=G+jB$, $G=R/(R^2+X^2)=\sqrt{2}/4$, $B=-X/\sqrt{R^2+X^2}=\sqrt{2}/4$

\item A voltage $v(t)=120\sqrt{2} cos(1000t+90^\circ) V$ (volt) is applied to 
a resistor $R=15\Omega$, a capacitor $C=83.3\mu F$ and an inductor $L=30\; mH$ 
connected in parallel. Find the over all steady state current $i=i_R+i_C+i_L$ 
by phasor method.

%  Express input voltage as a phasor $\dot{V}=120\angle{90^\circ}$.  
% Then $\dot{I}_R=\dot{V}/R=120\angle{90^\circ}/15=0+j8 A$
%  $\dot{I}_C=\dot{V}/Z_C=j\omega C\dot{V}=(0.00833\angle 90^\circ)(120\angle{90^\circ})
%  	=10\angle 180^\circ=-10+j0 A$
%  $\dot{I}_L=\dot{V}/Z_L=\dot{V}/j\omega L=(120\angle 90^\circ)/(30\angle{90^\circ})
%  	=4+j0 A$.
%  By KCL, we have $\dot{I}=\dot{I}_R+\dot{I}_C+\dot{I}_L=(0-10+4)+j(8+0+0)
%  	=-6+j8=10\angle{127^\circ} A$
%  $i(t)=10\sqrt{2}\;cos(1000t+127^\circ) A$

\item A voltage $v(t)=12\sqrt{2} \cos 5000 t$ (volt V) is applied to a circuit
composed of two branches in parallel. One branch has a capacitor $C=10\mu F$,
while the other has a resistor $R=20\Omega$ and an inductor $L=3 mH$ in series.
Using phasor method, find the impedances $Z_C$ and $Z_{RL}$ of the two branches,
and then the overall combined impedance $Z_{all}$ of the circuit. Then find
the steady state current $i(t)$ through the circuit.

%  $Z_R=R=20+j0=20\angle 0^\circ \Omega$, $Z_L=j\omega L=j5000\times 0.003
%  =15\angle 90^\circ \Omega$, $Z_C=1/j\omega C=-j/5000 \times 10^{-5}=-j20
%  =20\angle -90^\circ \Omega$
%  $Z_{RL}=Z_R+Z_L=20+j15=25\angle 37^\circ \Omega$
%  $Z_{all}=Z_C//Z_{RL}=Z_C Z_{RL}/(Z_C+Z_{RL})=25\angle 37^\circ \; 
%  20\angle -90^\circ/(20+j15-j20)=500\angle -53^\circ/20.6\angle -14^\circ
%  =24.3\angle -39^\circ=18.9-j15.3 \Omega $
%  $\dot{I}=\dot{V}{Z}=12\angle 0^\circ/24.3\angle -39^\circ=0.49\angle 39^\circ A$
%  $i(t)=0.49\sqrt{2} cos(5000t+39^\circ) A$

\item Solve the problem above again but this time use the admittances 
$Y_C=1/Z_C$, $Y_{RL}=1/Z_{RL}$, $Y_{all}=1/Z_{all}$ (instead of the
impedances $Z_C$, $Z_{RL}$, $Z_{all}$). Recall that Ohm's law becomes
$\dot{I}=\dot{V}/Z=\dot{V}Y$. (Make sure all impedances you found in 
previous problem are correct before you find the admittances as their 
reciprocals.)

% $Y_C=1/Z_C=1/20\angle{0^\circ}=0.05\angle{90^\circ}=0+j0.05\;S$
% $Y_{RL}=1/Z_{RL}=1/25\angle{37^\circ}=0.04\angle{-37^\circ}=0.032-j0.024\;S$
% $Y_{all}=Y_C+Y_{RL}=0+j0.05 + 0.032-j0.024=0.032+j0.026=0.04\angle{39^\circ}$
% $I_C=Y_C V=0.05\angle{90^\circ} \times 12\angle{0^\circ}=0.6\angle{90^\circ}A$
% $I_{RL}=Y_{RL} V=0.04\angle{-37^\circ} \times 12\angle{0^\circ}=0.48\angle{-37^\circ}A$
% $I=I_C+I_{RL}=0+j0.6+0.384-j0.288=0.384+j0.312=0.49\angle{39^\circ}$
% Alternatively,
% $I=Y_{all}V=0.041\angle{39^\circ} \times 12\angle{0^\circ}=0.49\angle{39^\circ}$


\end{enumerate}
\end{document}

