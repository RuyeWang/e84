\documentstyle[11pt]{article}
\usepackage{html}
\begin{document}
\begin{center}
{\Large \bf E84 Home Work 8}
\end{center}
\begin{enumerate}

\item Use Matlab to plot the frequency response functions (FRF) of the 
  RC first-order low-pass (voltage across C is the output) and high-pass
  (voltage across R is the output) filters. Assume $C=1\,\mu F$, 
  $R=1\,k\Omega$. Choose the range of frequency properly so that the
  corner frequency is around the middle region of the plots.
  \begin{itemize}
    \item Plot the linear gain and phase as a function of $\omega$ of 
      the two filters.
    \item Make the Bode plots (without using the built-in function Bode)
      of both the gain (in log magnitude $20\log_{10}|H(\omega)|$ and 
      phase (between $-\pi$ and $\pi$) of the two filters (log scale in 
      frequency). The Matlab function semilogx can plot a function $f(x)$
      with log scale in $x$.
  \end{itemize}

\item Use Matlab to plot the frequency response functions (FRF) of the 
  RLC second-order low-pass (voltage across C as the output), band-pass
  (voltage across R as the output), and high-pass (voltage across L as
  the output) filters. Assume $L=1\,mH$, $C=1\,\mu F$, $R=1\,k\Omega$. 
  Choose the range of frequency properly so that the natural frequency 
  is around the middle region of the plots.
  \begin{itemize}
    \item Plot the linear gain and phase as a function of $\omega$ of the 
      three filters.
    \item Make the Bode plots (without using the built-in function Bode)
      of both the gain and phase of the three filters (log scale in frequency).
  \end{itemize}

\item The RCL series circuit can be used as a filter when the voltage across
  all three components is the input and the voltage across any of the 
  three components is treated as output. Find the resonant frequency
  $\omega_r$ (at which the output is maximized) in terms of the natural
  frequency $\omega_n=1/\sqrt{LC}$, when (1) voltage $v_C(t)$ across C
  is treated as output, and (2) voltage $v_L(t)$ across L is treated as
  output.

  The frequency response functions $H_C(\omega)$ and $H_L(\omega)$
  may have a peak, i.e., $|H(\omega_r)|>|H(\omega)|$ for any $\omega\ne 
  \omega_r$, when $\zeta$ is small enough, but it may not have such a
  peak if $\zeta$ is too large. Find the critical value $\zeta_c$ so 
  that for any $\zeta<\zeta_c$ $|H_C(\omega)|$ and $H_L(\omega)|$ will 
  have a peak at $\omega=\omega_r$, but such a peak no longer exist when 
  $\zeta>\zeta_c$.  

\item An RCL series circuit composed of $R=10\Omega$, $L=10 mH$ and
  $C=1 \mu F$ is connected to an input AC voltage $v_{in}(t)=cos\omega t$.
  \begin{itemize}
  \item Find the quality factor $Q$ and natural frequency $\omega_n$.
  \item Assume the voltage $v_R(t)$ across $R$ is taken as the output voltage.
    Find the bandwidth of the circuit. Also, use any software (e.g., Matlab) 
    to plot the ratio of the magnitudes between the output and input voltages 
    $v_R/v_{in}$ as a function of frequency $\omega$.
  \item Assume the voltage $v_L(t)$ across $L$ is taken as the output voltage.
    plot the ratio $v_L/v_{in}$ as a function of frequency $\omega$.
  \item Assume the voltage $v_C(t)$ across $C$ is taken as the output voltage.
    plot the ratio $v_C/v_{in}$ as a function of frequency $\omega$.
  \end{itemize}

%  {\bf Solution:}
% \begin{itemize}
%  \item 
%  \[ Q=\frac{1}{R}\sqrt{\frac{L}{C}}=\frac{1}{10}\sqrt{\frac{10^{-2}}{10^{-6}}}
%     =10 \]
%  \[ \omega_0=\frac{1}{\sqrt{LC}}=\frac{1}{\sqrt{10^{-2}10^{-6}}}
%     =\frac{1}{\sqrt{10^{-8}}}=10^4 \]
%  \item 
%  \[ \triangle \omega=\frac{\omega_0}{Q}=\frac{10^4}{10}=1,000 \]
%  \end{itemize}
%  \htmladdimg{../rcl_plots_hw.gif}

\item A series circuit composed of a capacitor $C$ and an inductor $L$ is to 
  be resonant at 800 kHz with a voltage input. Specify the value of $C$ for the 
  capacitor required for the given inductor with $L=40\mu H$ and an internal 
  resistance $R_L=4.02\Omega$, and predict the bandwidth. Assume the capacitor 
  is ideal, i.e., it introduces no resistance.

%  {\bf Solution:}
%   As $\omega_0=1/\sqrt{LC}=2\pi 8\times 10^5$, and $L=4\times 10^{-5}H$, we
%   can find $C$ to be
%   \[
%   C=\frac{1}{\omega_0^2 L}=\frac{1}{(2\pi 8\times 10^5)^2\times 4\times 10^{-5}}
%   =0.99\;nF	\]
%   Next find the quality factor:
%   \[
%   Q=\frac{\omega_0L}{R}=\frac{2\pi 8\times 10^5\times 4\times 10^{-5}}{4.02}=50
%   \]
%   then the bandwidth is
%   \[	f_2-f_2=\frac{f_0}{Q}=\frac{800\times 10^3}{50}=16\;kHz	\]

\item Design a series circuit to be resonant at 800 kHz with a bandwidth
  of 32 kHz. The inductor has $L=40 \mu H$ and $R_L=4.02 \Omega$. Find the
  capacitance $C$ needed for the desired resonant frequency. In order to 
  satisfy the desired bandwidth, you may also need to include a resistor in 
  the circuit. 

%  {\bf Solution:} Based on the desired resonant frequency and bandwidth, 
%  the quality factor needs to be
%  \[ Q=\frac{f_0}{\Triangle f}=\frac{800\times 10^3}{32 \times 10^3}=25 \]
%  Since $Q>20$, the resonant frequency is approximately
%  \[ \omega_0=\frac{1}{\sqrt{LC}},\;\;\;\;\mbox{i.e.,}\;\;\;\;
%  C=\frac{1}{\omega_0^2 L}=\frac{1}{(2\pi 8\times 10^5)^2\times 4\times 10^{-5}}
%  =0.99 nF \]
%  However, the quality factor of the parallel circuit is
%  \[ Q=\frac{\omega_0 L}{R}=\frac{2\pi 8 \times 10^5 \times 4\times 10^{-5}}
% 	{4.02}=50 \]
%  twice the desired $Q=25$, we have to double the resistance $R=4.02$ to 
%  $R=8.04$ to reduce $Q$ by half.

\end{enumerate}
\end{document}




\item In the circuit below, $R_1=20\Omega$, $R_2=10\Omega$, $C=318\mu F$,
and the sinusoidal voltage source is $v_0(t)=14.1 \cos(314 t-45^{\circ})$.
Find the complete voltage response $v(t)=v_C(t)$ across $C$ and $R_2$ after
the switch closes at $t=0$.

\htmladdimg{../completeresponse.gif}

% {\bf Solution:} 
% \begin{itemize}
% \item First find phasor representation of the voltage source:
% $\dot{V}_0=10\angle{-45^{\circ}}$ and the impedance of the capacitor:
% $Z_C=1/j\omega C=-j/(314\times 318\times 10^{-6})=-j10$. 

% \item Find $v_\infty(t)$:
%  \[ \dot{V}=\dot{V}_0 \frac{R_2|| Z_C}{R_1+R_2||Z_C}
%  =10\angle{-45^\circ}\frac{(-j10 || 10)}{20+(-j10 || 10)}
%  =2.77\angle{-79^\circ} \]
%  \[ v(t)=2.77\sqrt{2} \cos(314 t-79^\circ)=3.92 \cos(314 t-79^\circ) \]
%\item Find $v(0)=v_C(0)=v_C(0+)=0$

% \item Find $\tau=R_{eq}C=(R_1||R_2) C=20 || 10 \times 318\times 
%  10^{-6}=2.12\times 10^{-3}$
% \end{itemize}
% Now we get:
% \begin{eqnarray}
%  v(t)&=&v_\infty(t)+[v(0)-v_\infty(0)]e^{-t/\tau}
%  =2.77\sqrt{2} \cos(314 t-79^\circ)+[0-2.77\sqrt{2}cos(-79^\circ)] e^{-10^3 t/2.12}
%  \nonumber \\ 
%  &=& 3.92 \cos(314 t-79^\circ)-0.75 e^{-10^3 t/2.12}
%  \nonumber
%\end{eqnarray}



\item The function of a loudspeaker crossover network is to channel 
  frequencies higher than a given crossover frequency $f_c$ into the
  high-frequency speaker (``tweeter'') and frequencies below $f_c$ into
  the low-frequency speaker (``woofer''). One such circuit is shown below.
  Assume the resistances of the tweeter is $R_1=8\Omega$ and that of the 
  woofer is $R_2=8\Omega$, the voltage amplifier can be modeled as an
  ideal voltage source, and the crossover frequency is $f_c=2000\; Hz$.
  Design the network in terms of $L$ and $C$ so that $f_c$ is the corner
  freqnency or half-power point of each of the two speaker circuits. Give 
  the expression of the power $P_1(f)$ and $P_2(f)$ of the speakers as a 
  function of frequency $f$ and crossover frequency $f_c$, and sketch them.
  Assume the RMS of the input voltage is 1V.

\htmladdimg{../hw6b.gif}

%   {\bf Solution}
%   
%   The RMS voltage across the tweeter is
%   \[	V_1=|\frac{R}{1/j2\pi f C+R}|
%   	=\frac{2\pi f CR}{\sqrt{1+(2\pi f CR)^2}}	\]
%   If $f=f_c=2000$ is at half-power point ($V_1=V_{in}/\sqrt{2}$), the real and
%   imaginary parts of the denominator should be equal and we get 
%   \[	\frac{1}{2\pi f C}=R;\;\;\;\mbox{i.e.}\;\;\;\;
%   	C=\frac{1}{2\pi f_c R}=9.95 \;\mu F	\]
%  The RMS voltage across the woofer is
%  \[	V_2=|\frac{R}{j2\pi f L+R}|
%   	=\frac{R}{\sqrt{R^2+(2\pi f L)^2}}	\]
%   If $f=f_c=2000$ is at half-power point ($V_2=V_{in}/\sqrt{2}$), the real and
%   imaginary parts of the denominator should be equal and we get 
%   \[	2\pi f L=R;\;\;\;\mbox{i.e.}\;\;\;\;
%   	L=\frac{R}{2\pi f_c}=0.637 \;mH	\]
%   The power plots:
%   \[	P_1(f)=\frac{V_1^2(f)}{R}=\frac{1}{8}(\frac{1}{1+(f_c/f)^2}) \]
%   \[	P_2(f)=\frac{V_1^2(f)}{R}=\frac{1}{8}(\frac{1}{1+(f/f_c)^2}) \]

\end{enumerate}
\end{document}

\item In the circuit below, $V_s=6V$, $R_1=6\Omega$, $R_2=3\Omega$,
$L=0.5H$, $I_s=2A$. Assume before the switch is closed at $t=0$, the
system is already stablized. Find current $i_L(t)$ through $L$ and 
voltage $v_{R_1}$ across $R_1$.

\htmladdimg{../hw5a.gif}

{\bf Solution:}

 $i(0)=V_s/R_1=6/6=1A$, $i(\infty)=V_s/R_1+I_s=1+2=3A$

 Find equivalent resistance (when both energy sources are turned off):
 $R=R_1 || R_2=3\times 6/(3+6)=2$, $\tau=L/R=0.5/2=0.25\;sec.$

 Final solution:
 $i_L(t)=3+(1-3)e^{-t/0.25}=3-2 e^{-4t} \; (A)$

 Find $v_{R_1}$: apply KVL to the loop of $V_s$, $R_1$ and $L$, get
 \[ L\frac{di_L}{dt}+v_{R_1}=V_s	\]
 \[ v_{R_1}=V_s-L di_L/dt=6-0.5 \frac{d(3-2 e^{-4t})}{dt}=6-4e^{-4t}	\]

\item In the circuit below, $V_s=12V$, $R_1=5\Omega$, $R_2=20\Omega$,
$R_3=6\Omega$, $C_1=10\mu F$, $C_2=30\mu F$. Assume before the switch 
is closed at $t=0$, the system is already stablized. Find voltages
$v_1(t)$ and $v_2(t)$ across capacitors $C_1$ and $C_2$, respectively.
(Hint, $C_1$ and $C_2$ are two capacitors in series with an equivalent
capacitance is $C=C_1 C_2/(C_1+C_2)$. $C_1$ and $C_2$ have share the same
time constant $\tau=RC$.)

\htmladdimg{../hw5b.gif}

{\bf Solution:}

 \[ v_1(0)=v_2(0)=0	\]
 \[ v_1(\infty)+v_2(\infty)=V_s \frac{R_2}{R_1+R_2}
 	=12 \frac{20}{5+20}=9.6V	\]
 As voltage $V$ across a capacitor is inversely proportional to $C$, and 
 $C1$ and $C2$ store the same charge $Q=v_1C_1=v_2C_2$, we have
 \[	\frac{v_1(\infty)}{v_2(\infty)}=\frac{C_2}{C_1}=3	\]
 i.e., $v_1(\infty)=3v_2(\infty)$, and we get $v_1(\infty)=7.2V$,
 $v_2(\infty)=2.4V$.
 Find equivalent resistance: 
 \[ R=R_3+\frac{R_1 R_2}{R_1+R_2}=6+\frac{5\times 20}{5+20}=10\Omega \]
 Find equivalent capacitance: $C_1 C_2/(C_1+C_2)=30\times 10/(30+10)=7.5$. 
 Find time constant: $\tau=RC=10\times 7.5\times 10^{-6}=7.5\times 10^{-5}$.
 Find $v_1(t)$ and $v_2(t)$:
 $v_1(t)=7.2(1-e^{-t/(7.5\times 10^{-5})})$
 $v_2(t)=2.4(1-e^{-t/(7.5\times 10^{-5})})$

\item An RLC parallel circuit containing $R=20\Omega$, $L=400 \mu H$,
  $C=1 \mu F$ is driven by a current source $I=50 \mu A$ with variable 
  frequency. Find natural frequency, the quality factor, and the voltage
  across and the currents through each of the three components at resonance.

{\bf Solution:}

 \[ \omega_0=\frac{1}{\sqrt{LC}}=\frac{1}{\sqrt{10^{-6}\times 400 \times 10^{-6}}}
 	=\frac{1}{20 \times 10^{-6}}=5 \times 10^4	\]
 \[ Q=\frac{1}{G}\sqrt{\frac{C}{L}}=R\sqrt{\frac{C}{L}}
 =20\sqrt{\frac{10^{-6}}{400\times 10^{-6}}}=1	\]
 $v=i\times R=50 \mu A \times 20 \Omega=1 mV$, $i_R=50 \mu A$,
 $i_L=i_C=Q i_R=50 \mu A$.
 
 
\item An RCL series circuit is driven by a voltage source $V=10V$ of
frequency $\omega=10^4 \;rad/sec.$. The variable capacitor $C$ is
adjusted so the the maximual current of $100\;mA$ is achieved. The
voltage across this capacitor is $600V$. Find the values for $R$, $L$,
$C$ and the quality factor $Q$.

{\bf Solution:}

 At resonance, $v_L=-v_C=-600V$. $R=V/I=10V/0.1A=100\Omega$, $v_R=V=10V$.
 $Q=600/10=60$.  
 \[ \frac{1}{R}\sqrt{\frac{L}{C}}=Q=60	\]
 \[ \frac{1}{\sqrt{LC}}=\omega_0=10000 \]
 Solve for $L$ and $C$: $L=0.6H$, $C=10^{-7}/6 F$

\end{enumerate}
\end{document}

\item Resonant circuuit is widely used in radio and TV receivers to 
select a desired station from many stations available. The circuit and
its model are shown in the figure. Assume $L=0.3mH$, $R=16\Omega$, and 
$C$ is variable capacitor, which can be adjusted to match the resonant 
frequency of the circuit to the frequency of the desired station. Assume 
the frequency of the desired station is $640\;kHz$, find the value of $C$. 
If the induced voltage in the circuit is $e=2 \mu V$ (rms), find the 
current (rms) in the parallel resonant circuit, and the output voltage 
(rms) across the capacitor.

{\bf Hint:} First check if $Q=\omega L/R$ is larger than 20. If so, the
resonant frequency can still be found approximately as 
$\omega_0=1/\sqrt{LC}$ and $jX_L-jX_C=0$.

\htmladdimg{../hw7e.gif}

{\bf Solution:} At the desired resonant frequency $f=640\;kHz$, the 
reactance of the inductor is
\[ X_L=\omega L=2\pi f L=2\times 3.14\times 640\times 10^3\times 0.3
	\times 10^{-3}=1206\Omega \]
and the quality factor $Q$ of this circuit is 
\[	Q=\frac{\omega L}{R}=\frac{2\times 3.14\times 640\times 
	10^3\times 0.3\times 10^{-3}}{16}=75 \gg 20	\]
The resonant frequency can be expressed approximately as
\[	f_0=\frac{1}{2\pi\sqrt{LC}}=640\;kHz	\]
Solving this we get
\[	C=\frac{1}{(2\pi f_0)^2L}=206\times 10^{-12}F=206\;pF	\]
The current iin circuit is
\[	I_{rms}=e/R=2\times 10^{-6}/16=0.125\times 10^{-6} \]
The output voltage across $C$ is
\[ V_C\approx V_L=I_{rms} X_L=0.125\times 10^{-6}\times 1206=151\;\muV \]


\item When a circuit composed of an inductor and a resistor in series
is connected to a DC voltage source of 36V, the current through the 
circuit can be measured to be 6A; when it is connected to an AC voltage 
$v(t)=110 \sqrt{2}\;\cos(2\pi 60\;t)$, the current through the circuit 
is 11A (RMS value). Find the resistance $R$ and the inductance $L$
of the load, and the power factor of the circuit as the load of the AC
voltage. 

{\bf Solution:} The resistance can be found to be $36V/6A=6\Omega$.
When $\omega=2\pi 60=377\;Rad/sec$, the impedance of the RL series load 
is 
\[ Z=R+j\omega L=|Z|\angle Z=\sqrt{R^2+( \omega L)^2}\;\tan^{-1} (\omega L/R) \]
and the current is
\[ I_{rms}=\frac{V_{rms}}{|Z|}=\frac{110}{\sqrt{36+(377 L)^2}}=11 \]
Solving this for $L$, we get $L=0.0212\;H=21.2\;mH$. The power factor is
$\cos (\tan^{-1} (\omega L/R))=\cos (\tan^{-1} 4/3)=0.6$.

\item The load of a voltage soruce of $v(t)=110\sqrt{2} \;\cos(2\pi 60\;t)$
is shown in the figure, where $R_1=100\Omega$, $R_2=50\Omega$, $C=6.63\mu F$, 
$L=0.53 H$. Is the load capacitive ($\phi=\tan^{-1}(X/R)<0$) or inductive 
($\phi>0$)? Find the power factor, the apparent power, the real power and 
the reactive power. 

\htmladdimg{../hw6d.gif}

 {\bf Solution:}
 \[ Z_C=-\frac{j}{\omega C}=-\frac{j}{2\pi 60\times 6.63\times 10^{-6}}
 	=-j400\Omega\]
 \[Z_L=j\omega L=j 2\pi 60\times 0.53=j200\Omega,\;\;\;\; Z_{RL}=100+j200\]
 
 \[ Z_{CRL}=\frac{Z_C Z_{RL}}{Z_C+Z_{RL}}=\frac{(100+j200)(-j400)}{100+j200-j400}
 	=\frac{800-j400}{1-j2}	\]
 \[
 Z_{total}=Z_{CRL}+R_2=\frac{800-j400}{1-j2}+50=370+j240=441\angle 33^\circ
 \]
 The load is inductive as $\phi>0$.
 \[ \dot{I}=\frac{\dot{V}}{Z_{total}}=\frac{110}{441\angle 33^\circ}
 	=0.25\angle -33^\circ	\]
 power factor is $\lambda=cos (-33^\circ)=0.839$, 
 the apparent power is $S=110\times 0.25=27.5 W$, 
 the real power is $P=S \cos 33^\circ=27.5\times 0.84=23 W$
 the reactive power is $Q=S \sin 33^\circ=27.5\times 0.54=15 W$

\item To improve the power factor of the circuit above to 0.9, a shunt 
capacitor is added. What should the capacitance $C$ be? What should $C$ be 
if the power factor is required to be 1?

\htmladdimg{../hw6c.gif}

 {\bf Solution:}
 
 Adding a shunt capacitor with impedance $1/j\omega C=-jX$ ($X=1/\omega C$), 
 the overall load impedance is
 \[	Z_{all}=-jX || Z_{total}=\frac{-jX(370+j240)}{-jX+(370+j240)}
 	=\frac{240X-j370X}{370-j(X-240)}=|Z|\angle Z=|Z|\angle \phi	\]
 For the power factor to be 0.9, this impedance need to have a phase angle 
 $\phi=\cos^{-1} 0.9=25.84^\circ$, and we need to have:
 \[	\tan^{-1}[\frac{-370X}{240X}]-\tan^{-1}[\frac{-(X-240)}{370}]=
 	-57^\circ+\tan^{-1}[\frac{X-240}{370}]=25.84^\circ \]
 \[	\tan^{-1}[\frac{X-240}{370}]=82.84^\circ, \;\;\;
 \frac{X-240}{370}=\tan \;82.84^\circ=7.96, \;\;\; X=3185.4 \]
 \[ \frac{1}{\omega C}=X=3185.4,\;\;\;\;C=\frac{1}{2\pi 60\times 3185.4}
 =0.83 \mu F \]
 For the power factor to be 1, we need to have
 \[	\tan^{-1}[\frac{-370X}{240X}]-\tan^{-1}[\frac{-(X-240)}{370}]=
 	-57^\circ+\tan^{-1}[\frac{X-240}{370}]=0^\circ \]
 i.e., 
 \[	\tan^{-1}[\frac{X-240}{370}]=57^\circ, \;\;\;
 \frac{X-240}{370}=\tan \;57^\circ=1.54, \;\;\; X=810 \]
 \[ \frac{1}{\omega C}=X=810,\;\;\;\;C=\frac{1}{2\pi 60\times 810}
 =3.27 \mu F \]

\end{enumerate}

\end{document}



\item The load of a voltage soruce of $v(t)=110\sqrt{2} \;cos(2\pi 60\;t)$
is shown in the figure, where $R_1=100\Omega$, $R_2=50\Omega$, $C=6.63\mu F$, 
$L=0.53 H$. Is the load capacitive ($\phi<0$) or inductive ($\phi>0$)?
Find the power factor, the apparent power, the real power and the
reactive power. To improve the power factor to 0.9, a shunt capacitor is added.
What should the capacitance $C$ be? What should $C$ be if the power factor is 
required to be 1?

\htmladdimg{../hw6c.gif}

 {\bf Solution:}
 
 \[ Z_C=-\frac{j}{\omega C}=-\frac{j}{2\pi 60\times 6.63\times 10^{-6}}
 	=-j400\Omega\]
 \[Z_L=j\omega L=j 2\pi 60\times 0.53=j200\Omega,\;\;\;\; Z_{RL}=100+j200\]
 
 \[ Z_{CRL}=\frac{Z_C Z_{RL}}{Z_C+Z_{RL}}=\frac{(100+j200)(-j400)}{100+j200-j400}
 	=\frac{800-j400}{1-j2}	\]
 \[
 Z_{total}=Z_{CRL}+R_2=\frac{800-j400}{1-j2}+50=370+j240=441\angle 33^\circ
 \]
 The load is inductive as $\phi>0$.
 \[ \dot{I}=\frac{\dot{V}}{Z_{total}}=\frac{110}{441\angle 33^\circ}
 	=0.25\angle -33^\circ	\]
 power factor is $\lambda=cos (-33^\circ)=0.839$, 
 the apparent power is $S=110\times 0.25=27.5 W$, 
 the real power is $P=S \cos 33^\circ=27.5\times 0.84=23 W$
 the reactive power is $Q=S \sin 33^\circ=27.5\times 0.54=15 W$
 Adding a shunt capacitor with impedance $1/j\omega C=-jX$ ($X=1/\omega C$), 
 the overall load impedance is
 \[	Z_{all}=-jX || Z_{total}=\frac{-jX(370+j240)}{-jX+(370+j240)}
 	=\frac{240X-j370X}{370-j(X-240)}=|Z|\angle Z=|Z|\angle \phi	\]
 For the power factor to be 0.9, this impedance need to have a phase angle 
 $\phi=\cos^{-1} 0.9=25.84^\circ$, and we need to have:
 \[	\tan^{-1}[\frac{-370X}{240X}]-\tan^{-1}[\frac{-(X-240)}{370}]=
 	-57^\circ+\tan^{-1}[\frac{X-240}{370}]=25.84^\circ \]
 \[	\tan^{-1}[\frac{X-240}{370}]=82.84^\circ, \;\;\;
 \frac{X-240}{370}=\tan \;82.84^\circ=7.96, \;\;\; X=3185.4 \]
 \[ \frac{1}{\omega C}=X=3185.4,\;\;\;\;C=\frac{1}{2\pi 60\times 3185.4}
 =0.83 \mu F \]
 For the power factor to be 1, we need to have
 \[	\tan^{-1}[\frac{-370X}{240X}]-\tan^{-1}[\frac{-(X-240)}{370}]=
 	-57^\circ+\tan^{-1}[\frac{X-240}{370}]=0^\circ \]
 i.e., 
 \[	\tan^{-1}[\frac{X-240}{370}]=57^\circ, \;\;\;
 \frac{X-240}{370}=\tan \;57^\circ=1.54, \;\;\; X=810 \]
 \[ \frac{1}{\omega C}=X=810,\;\;\;\;C=\frac{1}{2\pi 60\times 810}
 =3.27 \mu F \]

\end{enumerate}

\end{document}


\item Find the Z-model and Y-model of the circuit shown in the figures, by
assuming one of the two known variables (currents or voltages) is zero at
a time. Then verify your results by checking whether ${\bf Z}^{-1}={\bf Y}$.

\htmladdimg{../networkCL.gif}

 {\bf Solution:}

 For the Z-model, 
 \begin{itemize}
 item Assume $I_2=0$ (open-circuit), then
 	$Z_{11}=V_1/I_1=1/j\omega C$, $Z_{21}=V_2/I_1=1/j\omega C$
 \item Assume $I_1=0$ (open-circuit), then
 	$Z_{12}=V_1/I_2=1/j\omega C$, $Z_{22}=V_2/I_2=j\omega L+1/j\omega C$
 \end{itemize}
 For the Y-model, 
 \begin{itemize}
 \item Assume $V_2=0$ (short-circuit), then
 	$Y_{11}=I_1/V_1=j\omega C+1/j\omega C$, $Y_{21}=I_2/V_1=1/j\omega L$
 \item Assume $V_1=0$ (short-circuit), then
 	$Y_{12}=I_1/V_2=1/j\omega L$, $Y_{22}=I_2/V_2=1/j\omega L$
 \end{itemize}
 {\bf verify:}
 \[ {\bf Z}^{-1}=\left[ \begin{array}{rr} 1/j\omega C & 1/j\omega C \\
 	1/j\omega C & j\omega L+1/j\omega C \end{array} \right]^{-1}
 	=\frac{C}{L}\left[ \begin{array}{rr} j\omega L+1/j\omega C & -1/j\omega C \\
 	-1/j\omega C & 1/j\omega C \end{array} \right]
 	=\left[ \begin{array}{rr} j\omega C+1/j\omega L & -1/j\omega L \\
 	-1/j\omega L & 1/j\omega L \end{array} \right]
 \]

\item Represent the two-port system shown in the figure as a T-model with
$Z_1$, $Z_2$ and $Z_3$, and then a $\Pi$-model with $Y_1$, $Y_2$ and 
$Y_3$. Also confirm the system is reciprocal. All resistors are 1 $\Omega$.

\htmladdimg{../hw7b.gif}

 {\bf Solution:}
 \begin{itemize}
 \item Convert the $Y$ containing three $1\Omega$ resistors in the circuit 
 into a delta with three $1/3\Omega$ resistors.
 \item Get the T-model of the system with $Z_1=Z_2=1/3$, $Z_3=4/3$.
 \item Convert the T-model into a Z-network with
 \[ Z_{11}=Z_1+Z_3=5/3,\;\;\;\;Z_{22}=Z_2+Z_3=5/3,
 	\;\;\;\;Z_{12}=Z_{21}=Z_3=4/3 \]
 \item Find ${\bf Y}$:
 \[ {\bf Y}={\bf Z}^{-1}=\frac{1}{3}\left[ \begin{array}{rr}
 	5 & 4 \\ 4 & 5 \end{array} \right]^{-1}
 	=\frac{1}{3}\left[ \begin{array}{rr} 5 & -4 \\ -4 & 5 \end{array} \right]
 \]
 	with $Y_{11}=Y_{22}=5/3$, $Y_{12}=Y_{21}=-4/3$.
 \item Convert the Y-model to a $\Pi$-network:
 \[ Y_1=Y_{11}+Y_{12}=1/3,\;\;\;\;Y_2=Y_{22}+Y_{21}=1/3,\;\;\;\;Y_3=-Y_{12}=4/3 \]
 \item The $\Pi$-model can be also expressed In terms of impedances:
 \[ Z_1=3,\;\;\;\;Z_2=3,\;\;\;\;Z_3=4/3 \]
 \end{itemize}
 As both ${\bf Z}$ and ${\bf Y}$ are symmetric, the system is reciprocal.

\item The figure below shows a voltage source $V$ and its load containing
two resistors $R_C=3\Omega$, $R_L=2\Omega$ and an ideal transformer with 
turn ratio $n=2$. Find the equivalent load resistance $R=V/I$.
(Hint: find current $I_1$ and $I_4$ in terms of $R_C$, $R_L$, the turn 
ratio $n$, as well as $V$, then find $I=I_1+I_4$ and $R=V/I$.)

\htmladdimg{../hw7a.gif}

 {\bf Solution:}
 
 \begin{itemize}
 \item Find $V_2=V/2$
 \item Find $I_4=(V-V_2)/3=V/6$
 \item Find $I_3=V_2/2=V_1/4$
 \item Find $I_2=I_4-I_3=V(1/6-1/4)=-V/12$
 \item Find $I_1=-I_2/2=V/24$
 \item Find $I=I_1+I_4=V(1/24+1/6)=5V/24)$
 \item Find $R=V/I=24/5=4.8\Omega$
 \end{itemize}

\item The parameters of the Y-model of the two-port network are
$Y_{11}=-4$, $Y_{12}=3$, $Y_{21}=3$, and $Y_{22}=-2$. The voltage source
is $V_0=5V$, $Z_0=1\Omega$, $Z_L=j1\Omega$. Find variables $I_1$, $I_2$, $V_1$, $V_2$.

\htmladdimg{../hw7c.gif}

 {\bf Solution:} 
 \begin{itemize}
 \item Convert Y-model to Z-model:
   \[ {\bf Z}={\bf Y}^{-1}=\left[ \begin{array}{rr}-4 & 3 \\ 3 & -2\end{array} \right]^{-1}
   =\left[ \begin{array}{rr}2 & 3 \\ 3 & 4\end{array} \right] \]
   \[ \left\{ \begin{array}{l} V_1=2I_1+3I_2 \\ V_2=3I_1+4I_2 \end{array} \right. \]
 \item Setup additional equations:
   \[ \left\{ \begin{array}{l} V_1=V_0-R_0I_1 \\ V_2=-Z_L I_2 \end{array} \right. \]
 \item Find $I_1$ and $I_2$:
   \[ \left\{ \begin{array}{l} 2I_1+3I_2=V_0-R_0I_1\\ 3I_1+4I_2=-Z_LI_2\end{array}\right. 
     \;\;\;\;\;\mbox{i.e.}\;\;\;\;\;
   \left\{ \begin{array}{l} (2+R_0)I_1+3I_2=V_0\\ 3I_1+(4+Z_L)I_2=0\end{array}\right. \]
     These equations can be solved to get
     \[ I_1=\frac{5(4+j)}{3(1+j)},\;\;\;\;\;\;I_2=-\frac{5}{1+j}\]
 \item Find $V_1$, $V_2$:
   \[ V_2=-I_2Z+L=\frac{j5}{1+j},\;\;\;\;\;\; V_1=-\frac{5(1-j2)}{3(1+j)} \]
 \end{itemize}

\item Repeat the previous problem but this time use Thevenin's theorem to find
$V_2$ across load $R_L$.

 {\bf Solution:} 
   \[ \left\{ \begin{array}{l} V_1=2I_1+3I_2 \\ V_2=3I_1+4I_2 \end{array} \right. \]
   \[ \left\{ \begin{array}{l} V_1=V_0-R_0I_1 \\ V_2=-Z_L I_2 \end{array} \right. \]
 First, find $Z_{Th}$ when the voltage souce is short circuit:
 \begin{itemize}
   \item Equating $V_1=V_0-R_0I_1=-I_1$ to $V_1=2I_1+3I_2$, we get
     $ 2I_1+3I_2=-I_1$, i.e., $I_1=-I_2$.
   \item Substituting $I_1=-I_2$ into $V_2=3I_1+4I_2$, we get $ V_2=I_2$, i.e.,
     $Z_{Th}=V_2/I_2=1$.
 \end{itemize}
 Second, find $V_{Th}$ when the load is open circuit, i.e., $I_2=0$:
 \begin{itemize}
   \item Since $I_2=0$, the Z-model equations become $V_1=2I_1$, $V_2=3I_1$.
   \item We also get $I_1=(V_0-V_1)/R_0=5-V_1=5-2I_1$, which can be solved to 
     get $I_1=5/3$.
   \item Then $V_{Th}=V_2=3I_1=5$.
 \end{itemize}
 Finally, we can find voltage across $R_L$ as
 \[V_2=V_{Th}\;\frac{Z_L}{Z_{Th}+Z_L}=\frac{j5}{1+j} \]

\end{enumerate}

\end{document}


\item In reality, all inductors have a non-zero resistance, therefore a 
parallel LC resonance circuit should be modeled as shown in the figure:

\htmladdimg{../../../lectures/figures/parallelRCL.gif}

This mixed RCL circuit is quite different from the pure series or parallel 
RCL circuit in the sense that the real part of its admittance is also a function
of $\omega$ as well as the imaginary part. Consequently, we cannot minimize 
the admittance of the circuit by letting $\omega=1/\sqrt{LC}$. However, if the
quality factor of the inductor is large enough, e.g., $Q=\omega_0 L/R > 20$, 
we could still find the resonant frequency approximately by requiring the 
imaginary part of the admittance to be zero $Im[Y(\omega_0)]=0$.

\begin{itemize}
\item Assuming $\omega_0 L/R >20$, give the expression of the resonant 
  frequency at which the admittance of the circuit will be minimized.  

\item If $C=1\;nF=10^{-9}F$, $L=25 \mu H=2.5\times 10^{-5}H$, and 
  $R=5\Omega$, confirm the assumption $\omega_0 L/R >20$ is valid, and find 
  the resonant frequency.
\end{itemize}

% {\bf Solution:} 
% 
% The admittance is:
% \[ Y(\omega)=\frac{1}{R+j\omega L}+j\omega C
%    =\frac{R-j\omega L+j\omega C(R^2+\omega^2L^2)}{R^2+\omega^2L^2}
% \]
% Assuming $\omega_0L/R>20$, the resonant frequency can be found by letting
% the imaginary part of $Y$ be zoro $Im[Y]=0$:
% \[ \omega_0 L=\omega_0 C(R^2+\omega_0^2L^2)	\]
% Solving this for $\omega_0$, we get:
% \[ \omega_0=\sqrt{\frac{1}{LC}-(\frac{R}{L})^2}
%    =\sqrt{\frac{1}{2.5\times 10^{-14}}-(\frac{5}{2.5\times 10^{-5}})^2}
%    =\sqrt{4\times 10^{13}-4\times 10^{10}}\approx 6.3\times 10^6 \]
% i.e.,$f_0=\omega_0/2\pi=6.3\times 10^6/6.283 \;Hz=10^6=1000 KHz$.
% Check to see the validity of the approximation:
% \[ \frac{\omega_0L}{R}=\frac{6.3\times 10^6\times 2.5\times 10^{-5}}{5}=31.5 >20 \]
% As the admittance reaches minimum at $\omega=\omega_0$, the current drawn
% from the voltage source $\dot{I}=Y\dot{V}$ reaches minimum, i.e., it 
% behaves as a band stop filter.
%  
%   Also note that for $\omega_0$ to be real, we must have
%   \[	\frac{1}{LC} > (\frac{R}{L})^2, \;\;\;\;\;\;\mbox{i,e,}
%   	\;\;\;\;\;\;R<\sqrt{\frac{L}{C}}	\]
%   When $R \ll \sqrt{L/C}$, the resonant frequency is
%   \[
%   \omega_0=\sqrt{\frac{1}{LC}-(\frac{R}{L})^2}\approx \frac{1}{\sqrt{LC}}	
%   \]

% \item Find the output voltage $v_{out}(t)$ when $\omega=0$ and 
% $\omega\rightarrow \infty$ when the input $v_{in}(t)=10\;cos(\omega t)$, 
% assuming $R_1=100\Omega$, $R_2=100\Omega$, $C=10\mu F$ and $L=10\;mH$.
%\htmladdimg{../hw6a.gif}
%
% {\bf Solution}
% 
% When $\omega=0$, the inductor is short circuit, and the capacitor is
% open circuit, $v_{out}(t)=v_{in}(t)$.
% When $\omega=0$, the inductor is open circuit, and the capacitor is
% short circuit, $v_{out}(t)=v_{in}(t)/2=5\;cos(\omega_0 t)$.

