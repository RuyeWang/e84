\documentstyle[11pt]{article}
\usepackage{html}
\begin{document}
\begin{center}
{\Large \bf E84 Home Work 9}
\end{center}
\begin{enumerate}

\item The load of a voltage soruce of $v(t)=110\sqrt{2} \;\cos(2\pi 60\;t)$
is shown in the figure, where $R_1=100\Omega$, $R_2=50\Omega$, $C=6.63\mu F$, 
$L=0.53 H$. Is the load capacitive ($\phi=\tan^{-1}(X/R)<0$) or inductive 
($\phi>0$)? Find the power factor, the apparent power, the real power and 
the reactive power. 

\htmladdimg{../hw6d.gif}

%   {\bf Solution:}
%  \[ Z_C=-\frac{j}{\omega C}=-\frac{j}{2\pi 60\times 6.63\times 10^{-6}}
%   	=-j400\Omega\]
%   \[Z_L=j\omega L=j 2\pi 60\times 0.53=j200\Omega,\;\;\;\; Z_{RL}=100+j200\]
%   
%   \[ Z_{CRL}=\frac{Z_C Z_{RL}}{Z_C+Z_{RL}}=\frac{(100+j200)(-j400)}{100+j200-j400}
%   	=\frac{800-j400}{1-j2}	\]
%   \[
%   Z_{total}=Z_{CRL}+R_2=\frac{800-j400}{1-j2}+50=370+j240=441\angle 33^\circ
%   \]
%   The load is inductive as $\phi>0$.
%   \[ \dot{I}=\frac{\dot{V}}{Z_{total}}=\frac{110}{441\angle 33^\circ}
%   	=0.25\angle -33^\circ	\]
%   power factor is $\lambda=cos (-33^\circ)=0.839$, 
%   the apparent power is $S=110\times 0.25=27.5 W$, 
%   the real power is $P=S \cos 33^\circ=27.5\times 0.84=23 W$
%   the reactive power is $Q=S \sin 33^\circ=27.5\times 0.54=15 W$

\item To improve the power factor of the circuit above so that $\lambda=0.9$, a 
shunt capacitor is added. What should the capacitance $C$ be? What should $C$ be 
if the power factor is required to be 1?

\htmladdimg{../hw6c.gif}

%   {\bf Solution:}
%  
%   Adding a shunt capacitor with impedance $1/j\omega C=-jX$ ($X=1/\omega C$), 
%   the overall load impedance is
%   \[	Z_{all}=-jX || Z_{total}=\frac{-jX(370+j240)}{-jX+(370+j240)}
%   	=\frac{240X-j370X}{370-j(X-240)}=|Z|\angle Z=|Z|\angle \phi	\]
%   For the power factor to be 0.9, this impedance need to have a phase angle 
%   $\phi=\cos^{-1} 0.9=25.84^\circ$, and we need to have:
%   \[	\tan^{-1}[\frac{-370X}{240X}]-\tan^{-1}[\frac{-(X-240)}{370}]=
%   	-57^\circ+\tan^{-1}[\frac{X-240}{370}]=25.84^\circ \]
%   \[	\tan^{-1}[\frac{X-240}{370}]=82.84^\circ, \;\;\;
%   \frac{X-240}{370}=\tan \;82.84^\circ=7.96, \;\;\; X=3185.4 \]
%   \[ \frac{1}{\omega C}=X=3185.4,\;\;\;\;C=\frac{1}{2\pi 60\times 3185.4}
%   =0.83 \mu F \]
%   For the power factor to be 1, we need to have
%   \[	\tan^{-1}[\frac{-370X}{240X}]-\tan^{-1}[\frac{-(X-240)}{370}]=
%   	-57^\circ+\tan^{-1}[\frac{X-240}{370}]=0^\circ \]
%   i.e., 
%   \[	\tan^{-1}[\frac{X-240}{370}]=57^\circ, \;\;\;
%   \frac{X-240}{370}=\tan \;57^\circ=1.54, \;\;\; X=810 \]
%   \[ \frac{1}{\omega C}=X=810,\;\;\;\;C=\frac{1}{2\pi 60\times 810}
%   =3.27 \mu F \]

\item 
In the circuit shown below, the voltage source $v(t)=100\;\sqrt{2}\;sin\;314t\;$ volts,
and the effective values of the three currents $i$, $i_L$ and $i_C$ are the same. The
total real energy comsumed by the circuit is 866 W. Find the values of $R$, $L$ and $C$.
(Hint: represent all currents $i(t)$, $i_C(t)$, $i_L(t)$ and voltage $v(t)$ as phasors 
$\dot{I}$, $\dot{I}_C$, $\dot{I}_L$, $\dot{V}$, and draw them as vectors to figure out
how they are related.)


\htmladdimg{../midterm2h.gif}

%{\bf Solution:}
%Let $\dot{V}=100\angle 0$ be the phasor representation of $v(t)$ so that 
%\[ v(t)=Im[\sqrt{2} \dot{V} e^{j\omega t}] =Im[\sqrt{2} \dot{V} e^{j314t}] \]
%where $\omega=2\pi f=314$, i.e., $f=50$ Hz. We have
%\[ \dot{I}_C=j\omega C \dot{V}=100\omega C \angle 90^\circ,
%\;\;\;\;\;\dot{I}_L=\frac{\dot{V}}{R+j\omega L}=\frac{100}{\sqrt{R^2+\omega^2 L^2}}\angle -\phi \]
%where $\phi=tan^{-1}(\omega L/R)$. Due to KCL, we hav
%\[ \dot{I}=\dot{I}_C+\dot{I}_L \]
%but also as given
%\[ |\dot{I}|=|\dot{I}_C|=|\dot{I}_L|=I \]
%we conclude that these three currents are equal in magnitude and $60^\circ$ apart in 
%phase angle, as shown below:
%\htmladdimg{../midterm2h1.gif}
%
%where $\dot{I}_C$ is $90^\circ$ ahead of $\dot{V}$, $\dot{I}$ is $\theta=30^\circ$
%ahead of $\dot{V}$, which in turn is $\phi=30^\circ$ ahead of $\dot{I}_L$. Therefore,
%\[ \dot{I}=I\angle \theta=I\angle 30^\circ,\;\;\;\;\;\dot{I}_L=I\angle \phi=I\angle -30^\circ,
%\;\;\;\;\;\dot{I}_C=I\angle 90^\circ  \]
%As the real power is 
%\[ P=866=V I \cos\phi =100 I \cos (-30^\circ)=86.6 I \]
%we get
%\[ I=10 A \]
%therefore we also get
%\[ P=866=RI^2=R 100,\;\;\;\;\;R=8.66 \Omega \]
%and
%\[ \dot{I}=10\angle 30^\circ,\;\;\;\;\dot{I}_L=10\angle -30^\circ,\;\;\;\dot{I}_C=10\angle 90^\circ  \]
%But from above
%\[ \dot{I}_C=100\omega C \angle 90^\circ=10\angle 90^\circ \]
%we get:
%\[ \omega C=0.1,\;\;\;\;C=0.1/314=3.18\times 10^{-4} \; F\]
%Since we also have:
%\[ \dot{I}_L=\frac{100}{\sqrt{R^2+\omega^2 L^2}}\angle -\phi=10\angle -30^\circ \]
%solving this we get
%\[ \omega L=5,\;\;\;\;\;\;L=5/\omega=5/314=0.0159\;H \]


\item An audio amplification circuit with an output voltage $V_0$ and internal
  resistance $R_0=200\Omega$ is used to drive a speaker with resistance 
  $R_L=8\Omega$. (You may want to read the notes 
  \htmladdnormallink{here}{http://fourier.eng.hmc.edu/e84/lectures/ch3/node11.html}.)
  \begin{itemize}
  \item Find the real power $P_L$ received by the speaker
  \item Find the power delivery efficiency $\eta=P_L/P_T$, where
    $P_T=P_L+P_0$ is the total power delivered by the voltage source
    and $P_0$ is the power consumed by the internal resistance $R_0$.
  \item To maximize the power delivered to the speaker, a matching circuit
    composed of $C_1=C_2$ and $L$ is inserted between the source and load.
    Find the reactance $X$, the power received by the speaker, and the
    efficiency.
  \item Repeat the previous part with a matching circuit composed of
    $L_1=L_2$ and $C$.
  \end{itemize}

\end{enumerate}

\end{document}

\item The circuit shown in the figure contains a voltage source $V=10V$,
two resistors $R_1=300\Omega$ and $R_2=200\Omega$, and a silicon diode.
Find the voltage $V_D$ across and the current $I_D$ through the diode.
Solve this problem in two different methods: (a) assume voltage $V_D=0.7$ 
(as the diode is always forward biased), and (b) use the graphic approach
to find the intersection of the load line and the diode equation:
\[ I_D=I_0 ( e^{V_D/\eta V_T}-1 ) \]
Sketch the plot of the two curves and estimate the solution $(I_D,V_D)$
at their intersection. (Note that you can assume $I_0=10^{-10}$ and 
$V_T=0.026V$ at room temperature 300K.)

\htmladdimg{../hw8a.gif}

% {\bf Solution:} 
% Method 1: As the diode is forward biased, the voltage across it
% is 0.7V, and the current through $R_2$ is $0.7V/200\Omega=3.5mA$, and
% the current through $R_1$ is $(10-0.7)/300=31 mA$. The current through
% the diode is therefore $I_D=31-3.5=27.5 mA$.
% Method 2: set up the equation for $V_D$ and $I_D$:
% \[ \frac{V_D}{R_2}+I_D=\frac{V-V_D}{R_1} \]
% if $V_D=0$, $I_D=\frac{V}{R_1}=\frac{10}{300}=33.3 mA$
% if $I_D=0$, $V_D=V\;\frac{R_2}{R_1+R_2}=10\frac{200}{300+200}=4V$
% the intersection of the two curves is aprroximately the same as method 1.
% \htmladdimg{../hw8aa.gif}

\item The circuit shown in the figure contains a voltage source $V=10V$,
three resistors $R_1=300\Omega$, $R_2=200\Omega$, and $R_3=100\Omega$, 
and two silicon diode. Find the voltage across the two parallel branches.
(Hint: assume $V_D=0.7$ when the diode is forward biased.)

\htmladdimg{../hw8c.gif}

% {\bf Solution:} As the diode is forward biased, the voltage across it
% is 0.7V. Assume the voltage in question is $x$, then we have the following
% equation:
% \[	\frac{10-x}{300}=\frac{x-0.7}{200}+\frac{x-0.7}{100}	\]
% which can be solved for $x$ to get the voltage to be $2.39V$.

\item The circuit shown in the figure is a converter (adaptor) based on
a full-wave rectifier, which gets an AC voltage input of 115V 60 Hz, and
produces a 12V DC output. The voltage variation or ripple of the output
should not exceed 5\% when the load current is no more than 2A. Design 
the converter in terms of the turn ratio of the transformer and the 
value of the capacitor. 

\htmladdimg{../hw8d.gif}

\end{enumerate}

\end{document}



\item Read and understand the notes about two-port networks on 
\htmladdnormallink{this page}{http://fourier.eng.hmc.edu/e84/lectures/ch2/node4.html}
(before the title ``Principle of reciprocity'', which will not be covered), especially
the two examples, and then you should be able to do the following three problems.

Find the Z-model and Y-model of the circuit shown in the figures, by
assuming one of the two known variables (currents or voltages) is zero at
a time. Then verify your results by checking whether ${\bf Z}^{-1}={\bf Y}$.

\htmladdimg{../networkCL.gif}

%   {\bf Solution:}
% 
%   For the Z-model, 
%   \begin{itemize}
%   item Assume $I_2=0$ (open-circuit), then
%   	$Z_{11}=V_1/I_1=1/j\omega C$, $Z_{21}=V_2/I_1=1/j\omega C$
%   \item Assume $I_1=0$ (open-circuit), then
%   	$Z_{12}=V_1/I_2=1/j\omega C$, $Z_{22}=V_2/I_2=j\omega L+1/j\omega C$
%   \end{itemize}
%   For the Y-model, 
%   \begin{itemize}
%   \item Assume $V_2=0$ (short-circuit), then
%   	$Y_{11}=I_1/V_1=j\omega C+1/j\omega L$, $Y_{21}=I_2/V_1=1/j\omega L$
%   \item Assume $V_1=0$ (short-circuit), then
%   	$Y_{12}=I_1/V_2=1/j\omega L$, $Y_{22}=I_2/V_2=1/j\omega L$
%   \end{itemize}
%   {\bf verify:}
%   \[ {\bf Z}^{-1}=\left[ \begin{array}{rr} 1/j\omega C & 1/j\omega C \\
%   	1/j\omega C & j\omega L+1/j\omega C \end{array} \right]^{-1}
%   	=\frac{C}{L}\left[ \begin{array}{rr} j\omega L+1/j\omega C & -1/j\omega C \\
%   	-1/j\omega C & 1/j\omega C \end{array} \right]
%   	=\left[ \begin{array}{rr} j\omega C+1/j\omega L & -1/j\omega L \\
%   	-1/j\omega L & 1/j\omega L \end{array} \right]
%   \]

\item The parameters of the Y-model of the two-port network are $Y_{11}=-4$, 
$Y_{12}=3$, $Y_{21}=3$, and $Y_{22}=-2$. The voltage source is $V_0=5V$, 
$Z_0=1\Omega$, $Z_L=j1\Omega$. Find variables $I_1$, $I_2$, $V_1$, $V_2$. 
(Hint: refer to Method 1 in the example shown in the
\htmladdnormallink{web notes}{http://fourier.eng.hmc.edu/e84/lectures/ch2/node4.html})

\htmladdimg{../hw7c.gif}

%   {\bf Solution:} 
%   \begin{itemize}
%   \item Convert Y-model to Z-model:
%     \[ {\bf Z}={\bf Y}^{-1}=\left[ \begin{array}{rr}-4 & 3 \\ 3 & -2\end{array} \right]^{-1}
%     =\left[ \begin{array}{rr}2 & 3 \\ 3 & 4\end{array} \right] \]
%     \[ \left\{ \begin{array}{l} V_1=2I_1+3I_2 \\ V_2=3I_1+4I_2 \end{array} \right. \]
%   \item Setup additional equations:
%     \[ \left\{ \begin{array}{l} V_1=V_0-R_0I_1 \\ V_2=-Z_L I_2 \end{array} \right. \]
%   \item Find $I_1$ and $I_2$:
%     \[ \left\{ \begin{array}{l} 2I_1+3I_2=V_0-R_0I_1\\ 3I_1+4I_2=-Z_LI_2\end{array}\right. 
%       \;\;\;\;\;\mbox{i.e.}\;\;\;\;\;
%     \left\{ \begin{array}{l} (2+R_0)I_1+3I_2=V_0\\ 3I_1+(4+Z_L)I_2=0\end{array}\right. \]
%       These equations can be solved to get
%       \[ I_1=\frac{5(4+j)}{3(1+j)},\;\;\;\;\;\;I_2=-\frac{5}{1+j}\]
%   \item Find $V_1$, $V_2$:
%     \[ V_2=-I_2Z+L=\frac{j5}{1+j},\;\;\;\;\;\; V_1=-\frac{5(1-j2)}{3(1+j)} \]
%   \end{itemize}

\item Repeat the previous problem but this time use Thevenin's theorem to find
$V_2$ across load $R_L$. (Hint: refer to Method 2 in the example shown in the
\htmladdnormallink{web notes}{http://fourier.eng.hmc.edu/e84/lectures/ch2/node4.html})

%   {\bf Solution:} 
%     \[ \left\{ \begin{array}{l} V_1=2I_1+3I_2 \\ V_2=3I_1+4I_2 \end{array} \right. \]
%     \[ \left\{ \begin{array}{l} V_1=V_0-R_0I_1 \\ V_2=-Z_L I_2 \end{array} \right. \]
%   First, find $Z_{Th}$ when the voltage souce is short circuit:
%   \begin{itemize}
%     \item Equating $V_1=V_0-R_0I_1=-I_1$ to $V_1=2I_1+3I_2$, we get
%       $ 2I_1+3I_2=-I_1$, i.e., $I_1=-I_2$.
%     \item Substituting $I_1=-I_2$ into $V_2=3I_1+4I_2$, we get $ V_2=I_2$, i.e.,
%       $Z_{Th}=V_2/I_2=1$.
%   \end{itemize}
%   Second, find $V_{Th}$ when the load is open circuit, i.e., $I_2=0$:
%   \begin{itemize}
%     \item Since $I_2=0$, the Z-model equations become $V_1=2I_1$, $V_2=3I_1$.
%     \item We also get $I_1=(V_0-V_1)/R_0=5-V_1=5-2I_1$, which can be solved to 
%       get $I_1=5/3$.
%     \item Then $V_{Th}=V_2=3I_1=5$.
%   \end{itemize}
%   Finally, we can find voltage across $R_L$ as
%   \[V_2=V_{Th}\;\frac{Z_L}{Z_{Th}+Z_L}=\frac{j5}{1+j} \]

\end{enumerate}
\end{document}

\item The circuit shown in the figure contains a voltage source $V=10V$,
two resistors $R_1=300\Omega$ and $R_2=200\Omega$, and a silicon diode.
Find the voltage $V_D$ across and the current $I_D$ through the diode.
Solve this problem in two different methods: (a) assume voltage $V_D=0.7$ 
(as the diode is always forward biased), and (b) use the graphic approach
to find the intersection of the load line and the diode equation:
\[ I_D=I_0 ( e^{V_D/\eta V_T}-1 ) \]
Sketch the plot of the two curves and estimate the solution $(I_D,V_D)$
at their intersection. (Note that you can assume $I_0=10^{-10}$ and 
$V_T=0.026V$ at room temperature 300K.)

\htmladdimg{../hw8a.gif}

% {\bf Solution:} 
% Method 1: As the diode is forward biased, the voltage across it
% is 0.7V, and the current through $R_2$ is $0.7V/200\Omega=3.5mA$, and
% the current through $R_1$ is $(10-0.7)/300=31 mA$. The current through
% the diode is therefore $I_D=31-3.5=27.5 mA$.
% Method 2: set up the equation for $V_D$ and $I_D$:
% \[ \frac{V_D}{R_2}+I_D=\frac{V-V_D}{R_1} \]
% if $V_D=0$, $I_D=\frac{V}{R_1}=\frac{10}{300}=33.3 mA$
% if $I_D=0$, $V_D=V\;\frac{R_2}{R_1+R_2}=10\frac{200}{300+200}=4V$
% the intersection of the two curves is aprroximately the same as method 1.
% \htmladdimg{../hw8aa.gif}

\item The circuit shown in the figure contains a voltage source $V=10V$,
three resistors $R_1=300\Omega$, $R_2=200\Omega$, and $R_3=100\Omega$, 
and two silicon diode. Find the voltage across the two parallel branches.
(Hint: assume $V_D=0.7$ when the diode is forward biased.)

\htmladdimg{../hw8c.gif}

% {\bf Solution:} As the diode is forward biased, the voltage across it
% is 0.7V. Assume the voltage in question is $x$, then we have the following
% equation:
% \[	\frac{10-x}{300}=\frac{x-0.7}{200}+\frac{x-0.7}{100}	\]
% which can be solved for $x$ to get the voltage to be $2.39V$.

\end{enumerate}

\end{document}

\item The circuit shown in the figure is a converter (adaptor) based on
a full-wave rectifier, which gets an AC voltage input of 115V 60 Hz, and
produces a 12V DC output. The voltage variation or ripple of the output
should not exceed 5\% when the load current is no more than 2A. Design 
the converter in terms of the turn ratio of the transformer and the 
value of the capacitor. 

\htmladdimg{../hw8d.gif}

\item The input voltage to the circuit in the following figure is
$v(t)=10 \sin (\omega t)$. The two DC voltages are both 5V. Sketch the
waveform of the output voltage $V_{out}$.

\htmladdimg{../hw8e.gif}

\item Assume each of the input voltages $V_1$ and $V_2$ takes one of
two values, either 0V or 5V. Find the output voltage $V_{out}$ in the
following combinations of three input:
 
\begin{tabular}{c||c c c c}\hline\hline
$V_1$ & 0V & 0V & 5V & 5V \\ \hline
$V_1$ & 0V & 5V & 0V & 5V \\ \hline
$V_{out}$ &   &   &   &   \\
\end{tabular}

\htmladdimg{../hw8f.gif}


\end{enumerate}

\end{document}


\item Represent the two-port system shown in the figure as a T-model with
$Z_1$, $Z_2$ and $Z_3$, and then a $\Pi$-model with $Y_1$, $Y_2$ and 
$Y_3$. Also confirm the system is reciprocal. All resistors are 1 $\Omega$.

\htmladdimg{../hw7b.gif}

 {\bf Solution:}
 \begin{itemize}
 \item Convert the $Y$ containing three $1\Omega$ resistors in the circuit 
 into a delta with three $1/3\Omega$ resistors.
 \item Get the T-model of the system with $Z_1=Z_2=1/3$, $Z_3=4/3$.
 \item Convert the T-model into a Z-network with
 \[ Z_{11}=Z_1+Z_3=5/3,\;\;\;\;Z_{22}=Z_2+Z_3=5/3,
 	\;\;\;\;Z_{12}=Z_{21}=Z_3=4/3 \]
 \item Find ${\bf Y}$:
 \[ {\bf Y}={\bf Z}^{-1}=\frac{1}{3}\left[ \begin{array}{rr}
 	5 & 4 \\ 4 & 5 \end{array} \right]^{-1}
 	=\frac{1}{3}\left[ \begin{array}{rr} 5 & -4 \\ -4 & 5 \end{array} \right]
 \]
 	with $Y_{11}=Y_{22}=5/3$, $Y_{12}=Y_{21}=-4/3$.
 \item Convert the Y-model to a $\Pi$-network:
 \[ Y_1=Y_{11}+Y_{12}=1/3,\;\;\;\;Y_2=Y_{22}+Y_{21}=1/3,\;\;\;\;Y_3=-Y_{12}=4/3 \]
 \item The $\Pi$-model can be also expressed In terms of impedances:
 \[ Z_1=3,\;\;\;\;Z_2=3,\;\;\;\;Z_3=4/3 \]
 \end{itemize}
 As both ${\bf Z}$ and ${\bf Y}$ are symmetric, the system is reciprocal.

\item The figure below shows a voltage source $V$ and its load containing
two resistors $R_C=3\Omega$, $R_L=2\Omega$ and an ideal transformer with 
turn ratio $n=2$. Find the equivalent load resistance $R=V/I$.
(Hint: find current $I_1$ and $I_4$ in terms of $R_C$, $R_L$, the turn 
ratio $n$, as well as $V$, then find $I=I_1+I_4$ and $R=V/I$.)

\htmladdimg{../hw7a.gif}

 {\bf Solution:}
 
 \begin{itemize}
 \item Find $V_2=V/2$
 \item Find $I_4=(V-V_2)/3=V/6$
 \item Find $I_3=V_2/2=V_1/4$
 \item Find $I_2=I_4-I_3=V(1/6-1/4)=-V/12$
 \item Find $I_1=-I_2/2=V/24$
 \item Find $I=I_1+I_4=V(1/24+1/6)=5V/24)$
 \item Find $R=V/I=24/5=4.8\Omega$
 \end{itemize}
