\documentstyle[11pt]{article}
\usepackage{html}
\begin{document}
\begin{center}
{\Large \bf E84 Homework 1}
\end{center}
\begin{enumerate}

\item Given the basic relationship between the voltage across and
  current through each of the three types of components $R$, $C$,
  and $L$,
  \begin{tabular}{l|cc}
    Resistor $R$ & $i=v/R=Gv$ & $v=Ri=i/G$ \\
    Inductor $L$ & $i=\int v\,dt/L$ & $v=L\;di/dt$ \\
    Capacitor $C$ & $i=C\,dv/dt$ & $v=\int i\,dt/C$ \\
  \end{tabular}
  \begin{itemize}
  \item derive the expression for the equivalent resistance $R_s$
    of $n$ resistors $R_1,\cdots,R_n$ combined in series. Then derive 
    the expression for the equivalent resistance $R_p$ of the $n$ 
    resistors combined in parallel.
  \item Repeat the above for $n$ capacitors $C_1,\cdots,C_n$.
  \item Repeat the above for $n$ capacitors $L_1,\cdots,L_n$.
  \end{itemize}

\item (a) If two light bulbs both labeled as 110V and 40W in series are
connected to a socket outlet of 190V, what is the power consumption of 
each of the bulbs?

(b) Replace one of the two bulbs by another bulb labeled as 110V 15W, and
find the power consumption of each of the bulbs. What will happen to each
of the two bulbs? (Note that when the power consumption by a bulb is larger
than the specified wattage, it will be burned out!)

\item Measurement of a physical process by instruments may be tricky due 
  to the inevitable interfere on the process being caused by the instruments
  (remember what you learned in quantum mechanics?). The figure below shows 
  two possible configurations for the measurement of the voltage across and 
  the current through the load. 

  \htmladdimg{../hw1b.gif}

\begin{enumerate}
\item What are required of the ammeter and the voltmeter to minimize their
	influences on the measurements? 

\item How would the ammeter and the voltmeter affect the measurement of the
	current and the voltage in either of the configurations (a and b)?
\end{enumerate}

\item Use Kirchhoff's voltage and current laws to find voltage $V_{BD}$ and resistance 
$R_2$ in the circuit shown below:

\htmladdimg{../hw1c.gif}

(Note: The direction of a current and the polarity of a voltage source can
be assumed arbitrarily. To determined the actual direction and polarity, the
sign of the values also should be considered. For example, a current labeled 
in left-to-right direction with a negative value is actually flowing 
right-to-left.)

\item Find the equivalent resistance between the two terminals before and
after the switch is closed. (Note, the two diagonal branches are NOT
connected to each other in the middle.)

\htmladdimg{../hw2e.gif}

\item Find the equivalent resistance $R_{eq}$ between the two terminals
in the figure, where $R_0=3\Omega$, $R_1=4\Omega$, $R_2=2\Omega$, $R_3=2\Omega$, 
$R_4=1\Omega$. What is $R_{eq}$ if $R_0=5\Omega$?

(Hint: apply a test voltage $V_{test}$ across the terminals and the 
equivalent resistance can be found to be $R_{eq}=V_{test}/I_{test}$.
The circuit can be solved by applying KCL to $V_1$ and $V_2$.)

\htmladdimg{../hw1f.gif}

\item Design a multimeter that can measure both DC and AC voltage, DC current,
  and resistance with different scales. Specifically, you are given an analog 
  meter $A$ with a needle display, which reaches full scale when a DC current 
  of $I=100\;\mu A=10^{-4}\;A$ goes through it. The internal resistance of the
  meter is 10 Ohms. In addition, you need some multi-position rotary switches 
  to select different scales for each of the three types of measurements, and 
  resistors with any values needed in your design.

  \begin{itemize}
    \item DC Voltage measurement: DC voltages in these ranges can be measured
      0-2.5, 0-10, 0-50, and 0-250 (all in volts). Use a 4-position rotary switch
      to select one of the four ranges as shown in the figure below. For example, 
      when the range of 0-10 is selected, the needle display will reach full scale 
      when the voltage being measured is 10 V. The circuit is shown below. Determine 
      all resistances labeled.

      \htmladdimg{../multimeterV.gif}

    \item AC Voltage measurement: To measure an AC voltage (in terms of its
      RMS value), it first needs to be converted into a DC voltage. This can 
      be achieved by a 
      \htmladdnormallink{diode}{../../../lectures/ch4/node2.html}
      which only allows the current to pass in one
      direction (along the arrow) but not the other. This process is called
      rectification. 

      \htmladdimg{../../../lectures/figures/halfwaverectifier.gif}

      The diode will also cause a voltage drop of 0.7 volt 
      along the direction. The actual reading of the meter reflects the 
      \htmladdnormallink{average value}{../averagevalue/node1.html}
      of the rectified current. Find the resistance $R$ so 
      that when the incoming AC voltage is $V=10$ volt (RMS), the meter shows 
      a full scale display.

      \htmladdimg{../multimeterACV.gif}

    \item DC current measurement: measure currents in these ranges (all in mA):
      0-0.5, 0-2.5, 0-10, 0-50. Use a 4-position rotary switch to select one 
      of the four ranges as shown in the figure below. For example, when the 
      range of 0-10 is selected, the needle display will reach full scale when 
      a 10 mA current is measured. Determine all resistances labeled. Use
      $R_0=1\;K\Omega$.

      \htmladdimg{../multimeterA1a.gif}

    \item Resistance measurement: The circuit for resistance measurement is
      provided as shown below, where $V_1=1.5V$. Determine the values for the 
      resistors labeled as $R_0$, $R_1$, $R_{10}$, $R_{100}$ and $R_{1000}$ 
      and $V_2$ so that the needle display of the meter is full scale 
      ($I=100\;\mu A$) when the resistor $R=0$ being measured (between the 
      two leads labeled + and -) is zero, or half scale ($I=50\;\mu A$) when 
      the value of $R$ and the position of he two synchronized rotary switches
      are given in each of the four case shown in the table:

      \begin{tabular}{l|llll} \hline
	positions  & $\times 1$ & $\times 10$ & $\times 100$ & $\times 1K$ \\
	$R$ values & 20$\Omega$ & 200$\Omega$ & 2000$\Omega$ & 20 $K\Omega$ 
      \end{tabular}

      \htmladdimg{../multimeterR.gif}

  \end{itemize}

\end{enumerate}
\end{document}

