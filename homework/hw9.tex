\documentstyle[11pt]{article}
\usepackage{html}
\begin{document}
\begin{center}
{\Large \bf  E84 Home Work 9}
\end{center}
\begin{enumerate}


\item The circuit shown in the figure contains a voltage source $V=10V$,
two resistors $R_1=300\Omega$ and $R_2=200\Omega$, and a silicon diode.
Find the voltage $V_D$ across and the current $I_D$ through the diode.
Solve this problem in two different methods: (a) assume voltage $V_D=0.7$ 
(as the diode is always forward biased), and (b) use the graphic approach
to find the intersection of the load line and the diode equation:
\[ I_D=I_0 ( e^{V_D/\eta V_T}-1 ) \]
Sketch the plot of the two curves and estimate the solution $(I_D,V_D)$
at their intersection. (Note that you can assume $I_0=10^{-10}$ and 
$V_T=0.026V$ at room temperature 300K.)

\htmladdimg{../hw8a.gif}

% {\bf Solution:} 
% Method 1: As the diode is forward biased, the voltage across it
% is 0.7V, and the current through $R_2$ is $0.7V/200\Omega=3.5mA$, and
% the current through $R_1$ is $(10-0.7)/300=31 mA$. The current through
% the diode is therefore $I_D=31-3.5=27.5 mA$.
% Method 2: set up the equation for $V_D$ and $I_D$:
% \[ \frac{V_D}{R_2}+I_D=\frac{V-V_D}{R_1} \]
% if $V_D=0$, $I_D=\frac{V}{R_1}=\frac{10}{300}=33.3 mA$
% if $I_D=0$, $V_D=V\;\frac{R_2}{R_1+R_2}=10\frac{200}{300+200}=4V$
% the intersection of the two curves is aprroximately the same as method 1.
% \htmladdimg{../hw8aa.gif}

\item The circuit shown in the figure contains a voltage source $V=10V$,
three resistors $R_1=300\Omega$, $R_2=200\Omega$, and $R_3=100\Omega$, 
and two silicon diode. Find the voltage across the two parallel branches.
(Hint: assume $V_D=0.7$ when the diode is forward biased.)

\htmladdimg{../hw8c.gif}

% {\bf Solution:} As the diode is forward biased, the voltage across it
% is 0.7V. Assume the voltage in question is $x$, then we have the following
% equation:
% \[	\frac{10-x}{300}=\frac{x-0.7}{200}+\frac{x-0.7}{100}	\]
% which can be solved for $x$ to get the voltage to be $2.39V$.

\item The circuit shown in the figure is a converter (adaptor) based on
a full-wave rectifier, which gets an AC voltage input of 115V 60 Hz, and
produces a 12V DC output. The voltage variation or ripple of the output
should not exceed 5\% when the load current is no more than 2A. Design 
the converter in terms of the turn ratio of the transformer and the 
value of the capacitor. 

\htmladdimg{../hw8d.gif}

\item The input voltage to the circuit in the following figure is
$v(t)=10 \sin (\omega t)$. The two DC voltages are both 5V. Sketch the
waveform of the output voltage $V_{out}$.

\htmladdimg{../hw8e.gif}

\item Assume each of the input voltages $V_1$ and $V_2$ takes one of
two values, either 0V or 5V. Find the output voltage $V_{out}$ in the
following combinations of three input:
 
\begin{tabular}{c||c c c c}\hline\hline
$V_1$ & 0V & 0V & 5V & 5V \\ \hline
$V_1$ & 0V & 5V & 0V & 5V \\ \hline
$V_{out}$ &   &   &   &   \\
\end{tabular}

\htmladdimg{../hw8f.gif}


\end{enumerate}

\end{document}


\item The figure (A) below shows a common-emitter transistor 
applification circuit (silicon) with$\beta=100$, $V_{cc}=20V$, 
and $R_C=2\;K\Omega$. 
\begin{itemize}
\item The base current is $i_b(t)=(50+30 \cos \omega t)\;\mu A$.
Find $i_c(t)$ and $v_c(t)$ as functions of time.
\item Repeat the above for $i_b(t)=(90+30 \cos \omega t)\;\mu A$.
\item Repeat the above for $i_b(t)=(10+30 \cos \omega t)\;\mu A$.
\end{itemize}
For each of the three cases above, sketch the output (collector) 
characteristics ($i_c$ vs $v_c$) as shown in class show the wave
forms of $i_b(t)$, $i_c(t)$ and $v_c(t)$, following the example
in the lecture notes:
http://fourier.eng.hmc.edu/e84/lectures/ch4/node7.html.

{\bf Note:} As the convention in the schematics of transistor circuits,
the bottom horizontal line is treated as the ground, and all voltages,
such as $V_b$, $V_c$ and $V_e$ are measured with respect to the 
ground as the reference point.

{\bf Hint:} The relationship $I_C=\beta I_B$ is only valid in the
linear region in the middle range of the load line. However, in 
the cut-off region (close to the horizontal axis) and the saturation
region (close to the vertical axis), the above relationship no
longer holds and the actual output current $I_c$ and $V_{ce}$ can
only be found graphically in the output characteristic plot.

\htmladdimg{../hw9f.gif}


% {\bf Solution:}
% \begin{itemize}
% \item 
% \[ i_c(t)=\beta i_b(t)=(5+3\cos \omega t)\;mA \]
% \[ v_c(t)=V_{cc}-R_c i_c(t)=20-2000 (5+3\cos \omega t)\times 10^{-3}
%    =10-6\cos \omega t  \]
% \item
% \[ i_c(t)=\beta i_b(t)=(1+3\cos \omega t)\;mA \]
% \[ v_c(t)=V_{cc}-R_c i_c(t)=20-2000 (9+3\cos \omega t)\times 10^{-3}
%    =2-6\cos \omega t  \]
% Clipping happens during the negative half-cycle due to saturation.
% \item
% \[ i_c(t)=\beta i_b(t)=(10+3\cos \omega t)\;mA \]
% \[ v_c(t)=V_{cc}-R_c i_c(t)=20-2000 (1+3\cos \omega t)\times 10^{-3}
%    =18-6\cos \omega t  \]
% Clipping happens during the positive half-cycle due to cut-off.
% \end{itemize}

\item Now assume an emittor resistor $R_e=2\;K\Omega$ is added between 
emittor of the transistor and ground in the circuit above (figure (B)
above).  Find the voltages $v_e(t)$ as well as $v_c(t)$ when the base
current is $i_b(t)=(50+30 \cos \omega t)\;\mu A$. Sketch the waveformes
of the two voltages.

% {\bf Solution:}
% \[ i_c(t)=\beta i_b(t)=(5+3\cos \omega t)\;mA \]
% \[ v_c(t)=V_{cc}-R_c i_c(t)=20-2000 (5+3\cos \omega t)\times 10^{-3}
%   =10-6\cos \omega t  \]
% \[ i_e(t)=(\beta+1) i_b(t)\approx i_c(t)=(5+3\cos \omega t)\;mA \]
% \[ v_e(t)=R_e i_e(t)=2000 (5+3\cos \omega t)\times 10^{-3}
%    =10+6\cos \omega t  \]

\end{enumerate}

\end{document}

\item Find values of $R_C$ and $R_B$ in the circuit with $\beta=100$
and $V_{CC}=15V$ so that the Q-point is $I_C=25mA$ and $V_{ce}=7.5V$
What is the Q point if $\beta=200$?

\htmladdimg{../hw9a.gif}


% {\bf Solution:}

% Find $I_B=I_C/\beta=25mA/100=0.25mA$, $R_B=(15=0.7)/0.25=57.2k\Omega$
% Also as $V_{CC}=I_C R_C+V_{ce}$, i.e., $R_C=(V_{CC}-V_{ce})/I_C=
% (15-7.5)/25\times 10^{-3}=300\Omega$. 
% 
% $I_B=(V_{CC}-V_{BD})/R_B=0.25mA$, $I_C=\beta I_B=200\times 0.25=50mA$,
% $V_{ce}=V_{CC}-R_C I_C=15-300\times 50\times 10^{-3}=0$.

\item Design a stable self-biasing transistor circuit with a DC operating
point (Q point) of $I_C=2.5mA$ and $V_{ce}=7.5V$ which should be in the 
middle of the load line. The $\beta$ of the ransistor ranges from 50 to
200. 

{\bf Hint} This is a design problem with possibly multiple solutions, 
i.e., there may be more degrees of freedom than constraining conditions. 
One of such conditions is $(\beta+1)R_E \gg R_B$ for the DC operating 
point to be approximately independent of $\beta$ (see online notes). 

\htmladdimg{../hw9b.gif}

% {\bf Solution:}
% 
% \begin{itemize}
% \item Find $V_{CC}$: We want the Q-point to be in the middle of the
%   load line, so we set $V_{CC}=2V_{ce}=2\times 7.5=15V$.
% \item Find $R_C$ and $R_E$: As $V_{ce}=V_{CC}-I_CR_C-I_ER_E\approx
%   V_{CC}-I_C(R_C+R_E)$, we have $R_C+R_E=7.5/2.5\times 10^{-3}=3K\Omega$.
%   Choose $R_E=1K\Omega$ and $R_C=2K\Omega$.
% \item Find $R_B$: To satisfy $R_B \ll \beta_{min} R_E$, we
% 	let $R_B=0.1\times \beta_{min} R_E=0.1\times 50\times 1000=5\;K\Omega$
% \item Find $V_{BB}$: 
% 	\[ V_{BB}=V_{be}+I_ER_E=0.7+2.5\times 10^{-3}\times 10^3=3.2V \]
% \item Find $R_1$ and $R_2$:
% \[	R_B=\frac{R_1R_2}{R_1+R_2}=5\;K\Omega \;\;\;\;\;\;\;\;
% 	\frac{V_{BB}}{V_{CC}}=\frac{R_2}{R_1+R_2}=\frac{3.2}{15}=0.21	\]
% Solve these two equations (first divide the first equation by the second), 
% we obtain the two unknowns $R_1$ and $R_2$:
% \[	R_1=\frac{5\;K\Omega}{0.21}=24\;K\Omega \;\;\;\;\;\;\;\;\;
% 	R_2=6.4\; K\Omega	\]
% \end{itemize}

\item In the AC amplifier shown in the figure, $R_S=600\Omega$, $R_1=30K\Omega$, 
$R_2=20K\Omega$, $R_E=4K\Omega$, $R_C=3K\Omega$, $R_L=5.1K\Omega$, 
$\beta=100$, $V_{CC}=15V$. 
%Also assume $r_{be}=200\Omega$. 
Assume the 
frequency of the AC signals to be amplified is high enough so that the
coupling capacitors and the emitter by-pass capacitor can be treated as 
AC short circuits. Find
\begin{itemize}
\item The DC operating point in terms of variables $I_B$, $I_C$, $V_C$, $V_E$ 
  and $V_{CE}$.
\item The AC equivalent diagram
\item The AC input and output impedances
\item The voltage gain of the AC amplifier
\end{itemize}

\htmladdimg{../hw9c.gif}

% {\bf Solution:} 
% \begin{itemize}
% \item Find DC operating point.
% \[ V_{BB}=V_{CC}\frac{R_2}{R_1+R_2}=15\frac{20}{30+20}=6V	\]
% \[ R_B=\frac{R_1R_2}{R_1+R_2}=\frac{600}{50}=12K\Omega 	\]
% \[ I_B=\frac{V_{BB}-V_{be}}{(\beta+1)R_E+R_B}
% 	=\frac{6-0.7}{101\times 4000+12000}=\frac{5.3}{26\times 10^4}
% 	=0.025 mA	\]
% \[ I_C=\beta I_B=100\times 0.025 mA=2.5 mA	\]
% \[ V_C=V_{CC}-I_C R_C=15-2.5\times 3=7.5V	\]
% \[ V_E=(\beta+1)I_C R_E=5V	\]
% \[ V_{CE}=V_C-V_E=2.5V \]
% \item Find input and output impedances.
% \[	r_{be}=V_T/I_B=0.026/0.025=1K\Omega	\]
% \[	r_{in}=R_1||R_2||r_{be} \approx 1K\Omega \]
% \[	r_{out}=R_C=3K\Omega	\]

% \item Find voltage gain.
% \[	G=-\beta \frac{(R_1||R_2||r_{be})}{(R_1||R_2||r_{be})+R_s}
% 	\frac{1}{r_{be}}(R_C||R_L) \approx -117	\]
% \end{itemize}

\item An emitter collower circuit is shown in the figure. Assume 
$\beta=50$, $R_B=300K\Omega$, $R_E=4K\Omega$, $R_S=500\Omega$, 
$R_L=5.1K\Omega$, and $V_{CC}=12$. Find the DC operating point in
terms of $I_B$, $I_C$, $I_E$, and $V_E$. Then find the input and
output resistances and the voltage gain.

\htmladdimg{../hw9e.gif}

\item In the circuit shown below, the two base resistors $R_B=43K\Omega$,
the collector $R_C=1K\Omega$. Assume each of the two input voltages $V_1$
and $V_2$ are either 0.2V or 5V. Find the output voltage $V_{out}$ in the
following combinations of the inputs. (Hint: assume 5V input to a transistor
will drive it to saturation.)

\begin{tabular}{cc|c}\hline
$V_1$ & $V_2$ & $V_{out}$ \\ \hline
 0.2  &  0.2  &           \\ \hline
 5.0  &  0.2  &           \\ \hline
 0.2  &  5.0  &           \\ \hline
 5.0  &  5.0  &           \\ \hline
\end{tabular}

\htmladdimg{../hw9d.gif}

\end{itemize}

\end{enumerate}

\end{document}

\item In the AC amplifier shown in the figure, $R_S=1.5K\Omega$, $R_1=10K\Omega$, 
$R_2=200K\Omega$, $R_C=3.5K\Omega$, $R_L=2K\Omega$, $\beta=100$, $V_{CC}=20V$.
Also assume $r_{be}=200\Omega$. Find the DC variables $I_B$, $I_C$, $V_C$.

