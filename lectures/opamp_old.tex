\documentstyle[12pt]{article}
\usepackage{html}
% \usepackage{graphics}  
\begin{document}

\section*{Chapter 5: Operational Amplifiers (Op-amps)}

\subsection*{Operational Amplifier}

The circuit schematic of the typical 
\htmladdnormallink{741 op-amp}{http://en.wikipedia.org/wiki/Operational_amplifier#Internal_circuitry_of_741_type_op-amp} is shown below:


\htmladdimg{../figures/opamp741b.gif}

A component-level diagram of the common 741 op-amp. Dotted lines outline: 
\begin{itemize}
\item \htmladdnormallink{Current mirrors/current source (red)}{http://en.wikipedia.org/wiki/Current_mirror}
\item \htmladdnormallink{Differential amplifier (blue)}{http://en.wikipedia.org/wiki/Differential_amplifier}
\item \htmladdnormallink{Class A gain stage (magenta)}{http://en.wikipedia.org/wiki/Electronic_amplifier#Class_A}
\item Voltage level shifter (green); 
\item \htmladdnormallink{output stage (cyan)}{http://en.wikipedia.org/wiki/Electronic_amplifier#Class_B}
\end{itemize}


Like all op-amps, the circuit basically consists of three stages:
\begin{itemize}
\item {\bf Differential amplifier} with high input impedance that generates
  a voltage signal, the amplified voltage difference $v^+-v^-$.
\item {\bf Voltage amplifier} (class A amplification) with a high voltage gain 
  to further amplify the voltage.
\item {\bf Output amplifier} (class AB push-pull emitter follower) with low 
  output impedance and high current driving capability.
\end{itemize}
The op-amp requires two voltage supplies $\pm V_{cc}$ of both polarities (typically 
$V_{CC}=15$ V). 

Although the op-amp circuit may look complicated, its operation can be simply 
modeled by a voltage amplifier with three parameters, as shown below:

\htmladdimg{../figures/OpAmp0a.gif}

\begin{itemize}
\item
  {\bf Input impedance $r_{out}$:} very large, typically a few mega-Ohms or
  higher ($10^6\sim 10^{12}\,\Omega$, e.g., 741 $6\;M\Omega$), depending on 
  the frequency and specific components used (e.g., BJT or FET).

\item {\bf Output resistance $r_{out}$:}, very small, typically a few tens of
  ohms, e.g., 75 $\Omega$.

\item {\bf Open-circuit gain $A$:}, based on both the inverting input $v^-$ 
  and the non-inverting input $v^+$:
  \[
  v_{internal}=A_d (v^+ - v^-)+A_c \frac{1}{2}(v^+ + v^-)\approx A_d (v^+ - v^-) 
  \]
  where $A_d$ is the differential-mode gain and $A_c$ is the common-mode gain.
  It is desired that $A_d\rightarrow \infty$ and $A_c\rightarrow 0$, i.e., the 
  output is only proportional to the difference $v^+-v^-$ between the two inputs. 
  The common-mode rejection ratio (CMRR) is defined as the ratio between 
  differential-mode gain and common-mode gain:
  \[ CMRR=20\;\log_{10} \left(\frac{A_d}{A_c}\right)>100\;dB,
  \;\;\;\;\;\;\;\;(A_d>10^5 A_c) \]
  As $A_d$ is large, typically $A=A_d=10^5 \sim 10^9$, the output is approximately
  \[ v_{out}=A(v^+-v^-) \]
\end{itemize}

Also, as the output $v_{out}=A(v^+-v^-)$ is in the range between $-V_{CC}$ 
and $V_{CC}$ and $A>10^5$ is large, $v^+-v^-=v_{out}/A$ is small (in the 
micro-volt range), i.e., $v^-\approx v^+$. If, as in some op-amp circuits, 
$v^+=0$ is grounded, then $v^-\approx v^+=0$ is very close to zero, i.e., 
it is almost the same as ground, or 
\htmladdnormallink{{\em virtual ground}}{http://en.wikipedia.org/wiki/Virtual_ground}.
The analysis of various op-amp circuits can be much simplified by this virtual
ground assumption.

As $A$ is large, $V_{out}=A(v^+-v^-)$ is usually saturated, equal to either 
$V_{CC}$ or $-V_{CC}$ (called the ``rails''), depending on whether or not 
$v^+$ is greater than $v^-$. For $v_{out}$ to be meaningful, some kind of
negative feedback is needed. In the following, we consider some typical 
op-amp circuits to show how to analyze an Op-amp circuit to find its input 
resistance $R_{in}$, output resistance $R_{out}$, and open-circuit voltage 
gain $G_{oc}$.

\begin{itemize}

\item {\bf Voltage follower:} The input is connected to the positive 
  input while the output is directly connected to the negative input 
  (100\% negative feedback), as shown in (A) in the figure below. The 
  op-amp can be modeled by its input impedance $r_{in}$, output impedance 
  $r_{out}$ and voltage gain $A$, as shown in (B). Then the voltage follower 
  can be modeled by its input impedance $R_{in}$, output impedance $R_{out}$,
  and voltage gain $G_{oc}$, as shown in (C). 

  \htmladdimg{../figures/FollowerModel.png}
  
  Specifically, $R_{in}$, $R_{out}$ and $G_{oc}$ can be found below. Here 
  the voltage source in the op-amp is $A(v^+-v^-)=A \;i_{in} r_{in}$.
  
  \begin{itemize}
  \item {\bf Input impedance $R_{in}$:} Applying KVL to the loop we get
    \[
    v_{in}=i_{in}(r_{in}+r_{out})+A\;(v^+-v^-)
    =i_{in}(r_{in}+r_{out})+A\;i_{in}r_{in}
    =i_{in}((A+1)r_{in}+r_{out})
    \]
    Dividing both sides by $i_{in}$ we get the input impedance:
    \[
    R_{in}=\frac{v_{in}}{i_{in}}=(A+1)r_{in}+r_{out} \approx Ar_{in}
    \]
  \item {\bf Open-circuit voltage gain $G_{oc}$:} The open-circuit output 
    voltage is
    \[
    v_{out}=v_{oc}=A(v^+-v^-)+i_{in}r_{out}=Ai_{in}r_{in}+i_{in}r_{out}
    =i_{in}(Ar_{in}+r_{out})
    \]
    The open-circuit gain is
    \[
    G=\frac{v_{out}}{v_{in}}=\frac{Ar_{in}+r_{out}}{(A+1)r_{in}+r_{out}}\approx 1
    \]
    i.e., 
    \[
    v_{out}=v_{oc}\approx v_{in}
    \]
  \item {\bf Output impedance $R_{out}$:} 
    With a short-circuit load, we have $v^+-v^-=v_{in}$, and the 
    short-circuit current can be found by superposition:
    \[
    i_{sc}=\frac{v_{in}}{r_{in}}+\frac{A(v^+-v^-)}{r_{out}}
    =\frac{v_{in}}{r_{in}}+\frac{Av_{in}}{r_{out}}
    =v_{in}\left(\frac{r_{out}+Ar_{in}}{r_{in}r_{out}}\right)
    \approx v_{in}\frac{A}{r_{out}}
    \]
    As we also know $v_{oc}=v_{out}\approx v_{in}$, we get the output impedance
    (Thevenin's model):
    \[
    R_{out}=\frac{v_{oc}}{i_{sc}}=\left(\frac{r_{in}r_{out}}{r_{out}+Ar_{in}}\right)
    \approx \frac{r_{out}}{A}
    \]
    The approximation is due to the fact that $r_{out}\ll Ar_{in}$.
  \end{itemize}

  In summary, we see that the voltage follower has a unit voltage gain, but
  much increased input resistance $R_{in}\approx A r_{in}$ (e.g., $10^{10}\Omega$)
  and much reduced output resistance $R_{out}\approx r_{out}/A$ (e.g., $10^{-3} \Omega$).
  In practice we could simply assume $R_{in}=\infty$ and $R_{out}=0$.



  \begin{comment}
    \htmladdimg{../figures/VoltageFollowerModel.png}

  \begin{itemize}
    \item {\bf Open-circuit voltage gain $G_{oc}$:} Assume an ideal source 
      voltage $v_s$ ($R_s=0$) is applied to the input of the circuit and 
      the output port is open circuit $R_L=\infty$. Then applying KVL to 
      the loop, we get
      \[
      v_s=(r_{in}+r_{out})i_{in}+A(v^+-v^-)=(r_{in}+r_{out})i_{in}+Ar_{in}i_{in}
      =[(A+1)r_{in}+r_{out}]i_{in} 
      \]
      Note that the internal voltage source is $A(v^+-v^-)=r_{in}i_{in}$. 
      The output voltage is:
      \[
      v_{out}=Ar_{in}i_{in}+r_{out}i_{in}=(Ar_{in}+r_{out})i_{in} 
      \]
      and the open-circuit voltage gain is:
      \[ 
      G_{oc}=\frac{v_{out}}{v_s}=\frac{Ar_{in}+r_{out}}{(A+1)r_{in}+r_{out}} 
      \approx 1
      \]
      Since $A>>1$, $G_{oc}\approx 1$ is approximately unity.
    \item {\bf Input resistance $R_{in}$:} 
      We now connect a load $R_L$ to the output port, while still keeping $R_S=0$ 
      (as it is irrelevant to $R_{in}$). Applying KVL to the two loops we get:
      \[ 
      \left\{ \begin{array}{l}
	v_s=(r_{in}+r_{out})i_{in}-r_{out}i_{out}+A r_{in}i_{in} \\
	A r_{in}i_{in}=(r_{out}+R_L)i_{out}-r_{out}i_{in} \right. \end{array} 
      \]
      Solving these two equations for the loop currents $i_{in}$ and $i_{out}$ we get
      \[ 
      i_{in}=\frac{v_s}{(A+1)r_{in}+r_{out}(R_L-Ar_{in})/(R_L+r_{out})} 
      \]
      The input resistance can be found as the ratio of the input voltage and current:
      \begin{eqnarray}
	R_{in}&=&\frac{v_s}{i_{in}}
	=(A+1)r_{in}+r_{out}\frac{R_L-Ar_{in}}{R_L+r_{out}} 
	=\frac{(A+1)r_{in}(R_L+r_{out})}{R_L+r_{out}}+r_{out}\frac{R_L-Ar_{in}}{R_L+r_{out}} 
	=r_{in}\frac{r_{out}+(A+1)R_L}{R_L+r_{out}}+\frac{R_L r_{out}}{R_L+r_{out}}
	\nonumber \\
	&=&r_{in}\frac{r_{out}+(A+1)R_L}{R_L+r_{out}}+R_L|| r_{out}      
	\nonumber 
      \end{eqnarray}
      Note that $R_{in}$ is affected by the load $R_L$. As usually $r_{out}$ is very
      small in comparison with all other resistances in the expression, it can be 
      dropped and the above becomes approximately
      \[ R_{in}\approx Ar_{in} \]
    \item {\bf Output resistance $R_{out}$:} This is the ratio $R_{out}=v_{oc}/i_{sc}$
      between the open-circuit voltage $v_{oc}$ and the short-circuit current $i_{sc}$.

      \begin{itemize}
      \item The open-circuit voltage $v_{oc}$ (with $R_L=\infty$ and $i_{out}=0$) 
        can be found by applying KVL to the loop:
        \[ 
        v_s=(R_s+r_{in}+r_{out})i_{in}+Ar_{in}i_{in},\;\;\;\mbox{i.e.,}\;\;\;\;
        i_{in}=\frac{v_s}{R_s+(A+1)r_{in}+r_{out}}      
        \]
        The voltage across the output port is
        \[ 
        v_{oc}=Ar_{in}i_{in}+r_{out}i_{in}
        =v_s\frac{Ar_{in}+r_{out}}{R_s+(A+1)r_{in}+r_{out}} 
        \]
      \item The short-circuit current $i_{sc}=i_{out}$ ($R_L=0$) can be found
        by applying KVL to the two loops:
        \[ 
        \left\{ \begin{array}{l}
          v_s=(R_s+r_{out}+(A+1)r_{in})i_{in}-r_{out} i_{out} \\
          Ar_{in}i_{in}=r_{out} i_{out}-r_{out} i_{in} \right. \end{array} 
        \]
        Solving these two equation for the two loop currents we get:
        \[ 
        i_{in}=\frac{v_s}{R_s+r_{in}},\;\;\;\;\;\;\;
        i_{out}=i_{sc}=v_s\frac{Ar_{in}+r_{out}}{r_{out}(R_s+r_{in})} 
        \]
      \item Find the output resistance:
        \[ 
        R_{out}=\frac{v_{oc}}{i_{sc}}
        =\frac{r_{out}(R_s+r_{in})}{R_s+(A+1)r_{in}+r_{out}} 
        \]
        Note that $R_{out}$ is affected by the internal resistance $R_s$ of the
        source. As $R_s+r_{out}<<(A+1)r_{in}$ and $R_s<<r_{in}$, we have
        \[ 
        R_{out}\approx \frac{r_{out}}{A} 
        \]
      \end{itemize}

  \end{itemize}


  \end{comment}

  {\bf Example:} 

  \htmladdimg{../figures/OpAmpBuffer.png}

  The figure on the left shows a circuit represented by an ideal voltage
  source $V_s$ in series with an internal resistance $R_s$ (Thevenin's
  theorem), with a load $R_L$. The voltage delivered to the load by this
  non-ideal source is
  \[
  v_{out}=v_s \;\frac{R_L}{R_L+R_s} =v_s - v_s\left(\frac{R_s}{R_L+R_s} \right)<v_s
  \]
  The output voltage across the load is only a fraction of the voltage due
  to the voltage drop $v_s R_s/(R_L+R_s)$ across the internal resistance $R_s$.
  If it is desired for the output voltage to be as close to the source as 
  possible, the internal resistance $R_s$ needs to be small compared to the
  load resistance $R_L$. However, if a voltage follower is used as a buffer
  between the source and the load, as shown in the figure on the right,
  the output voltage across $R_L$ can be very close to source voltage,
  i.e.,$v_{out}\approx v_s$.

  The voltage follower is modeled by its input and output resistances 
  $R_{in}$ and $R_{out}$, as well as its voltage gain $G_{oc}$, and the 
  output voltage can be obtained after two levels of voltage dividers:
  \[ 
  v_{out}=G_{oc} v_{in} \left(\frac{R_L}{R_{out}+R_L}\right)=
  G_{oc} v_s \left(\frac{R_{in}}{R_s+R_{in}}\right)\left(\frac{R_L}{R_{out}+R_L}\right)
  \approx v_s
  \]
  The approximation is due to 
  \begin{itemize}
  \item $G_{oc} \approx 1$
  \item 
    $R_{in}\gg R_s \;\;\Longrightarrow \;\;R_{in}/(R_s+R_{in})\approx 1$
  \item 
    $R_{out}\ll R_L\;\;\Longrightarrow \;\;R_L/(R_{out}+R_L) \approx 1$
  \end{itemize}

\item {\bf Inverting Amplifier}

%  \htmladdimg{../figures/InvertingAmplifier.png}
  \htmladdimg{../figures/InvertingAmplifier1a.png}

  As the analysis of the circuit using full model of the op-amp is very involved,
  certain approximation is made to simplify the analysis.
  \begin{itemize}
    \item {\bf Open-circuit voltage gain} $(R_s=0,\;R_L=\infty)$:

      As $r_{in}\gg R_1,\,R_f$ and $r_{out}\ll R_1,\,R_f$, we approximate
      $r_{in}\approx \infty$ and $r_{out}\approx 0$. Also, we have
      $v^+=0$, $v^-=v_s-R_1 i_{in}$ and $v^+-v^-=-v^-=R_1 i_{in}-v_s$.
      Now we have
      \[
      i_{in}=\frac{v_s-A(v_+-v_-)}{R_1+R_f}
      =\frac{v_s-A(R_1 i_{in}-v_s)}{R_1+R_f}
      \]
      Solving for $i_{in}$ we get
      \[
      i_{in}=\frac{(A+1)v_s}{(A+1)R_1+R_f}
      \]
      The output voltage is:
      \[
      v_{out}=v_s-(R_1+R_f) i_{in}
      =v_s\left[1-\frac{(A+1)(R_1+R_f)}{(A+1)R_1+R_f}\right] 
      =v_s\frac{-AR_f}{(A+1)R_1+R_f}
      \] 
      Now we have the open-circuit voltage gain:
      \[
      G_{oc}=\frac{v_{out}}{v_s}=\frac{-AR_f}{(A+1)R_1+R_f}
      \approx - \frac{R_f}{R_1} 
      \]
      The approximation is due to the fact that $A$ is very large. 

      \begin{comment}
      This approximate result can also be obtained under the virtual 
      ground assumption $v^-\approx v^+=0$. Applying KCL at the node of 
      $v^-$, we get
      \[
      \frac{v_s}{R_1}+\frac{v_{out}}{R_f}=0,\;\;\;\;\;\mbox{i.e.}\;\;\;\;
      G_{oc}=\frac{v_{out}}{v_s}=-\frac{R_f}{R_1}
      \]
      \end{comment}

   \item {\bf Input resistance:} 

      We assume $R_s=0$, and find the input resistance $R_{in}$ as the
      ratio of $v_s$ and the input current $i_{in}$. By KCL applied to 
      the node of $v^-$:
      \[
      \frac{v^--v_s}{R_1}+\frac{v^--(-Av^-)}{R_f}=0 
      \]
      Solving for $v^-$:
      \[
      v^-=v_s\frac{R_f}{R_f+(A+1)R_1} 
      \]
      The input current is
      \begin{eqnarray}
	i_{in}&=&\frac{v_s-v^-}{R_1}
        =\frac{v_s}{R_1}\left[1-\frac{R_f}{R_f+(A+1)R_1}\right]
	\nonumber \\
	&=&v_s\frac{1}{R_1}\frac{(A+1)R_1}{R_f+(A+1) R_1}
        =v_s\;\frac{A+1}{R_f+(A+1) R_1}
	\nonumber \end{eqnarray}
      Dividing $v_s$ by $i_{in}$ we get
      \[
      R_{in}=\frac{R_f+(A+1) R_1}{A+1}\approx R_1
      \]
      
    \begin{comment}
    \item {\bf Input resistance:} 

      We assume $R_s=0$, and find the input resistance $R_{in}$ as the
      ratio of $v_s$ and the input current $i_{in}$. By KCL applied to 
      the node of $v^-$:
      \[
      \frac{v^--v_s}{R_1}+\frac{v^-}{r_{in}}+\frac{v^--(-Av^-)}{R_f+r_{out}}=0 
      \]
      Solving for $v^-$:
      \[
      v^-=v_s\frac{r_{in}(R_f+r_{out})}{(R_1+r_{in})(R_f+r_{out})+(A+1)R_1r_{in}} 
      \]
      The input current is
      \begin{eqnarray}
	i_{in}&=&\frac{v_s-v^-}{R_1}
        =\frac{v_s}{R_1}\left[1-\frac{r_{in}(R_f+r_{out})}{(R_1+r_{in})(R_f+r_{out})
            +(A+1)R_1r_{in}}\right] 
	\nonumber \\
	&=&v_s\frac{R_f+r_{out}+(A+1)r_{in}}{(R_1+r_{in})(R_f+r_{out})+(A+1)R_1r_{in}} 
	\nonumber \end{eqnarray}
      Dividing $v_s$ by $i_{in}$ we get
      \begin{eqnarray}
      R_{in}&=&\frac{v_s}{i_{in}}
      =\frac{[R_f+r_{out}+(A+1)R_1]r_{in}+R_1(R_f+r_{out})}{R_f+r_{out}+(A+1)r_{in}} 
      \nonumber\\
      &\approx&\frac{(R_f+A\,R_1)r_{in}+R_1R_f}{R_f+A\,r_{in}} 
      \approx \frac{A\,R_1\,r_{in}}{A\,r_{in}}=R_1 
      \nonumber
      \end{eqnarray}
      The first approximation is based on $r_{out}\ll R_1,\,R_f$ and $A\gg 1$,
      the second approximation is based on $A\,r_{in}\gg R_1,\,R_f$.
    \end{comment}

    \item {\bf Output resistance:} Here we assume $r_{in}\rightarrow \infty$
      to simplify the analysis. 
      \begin{itemize}
      \item Find short-circuit output current: Applying KCL to the 
        output node we get
        \[
        i_{sc}=\frac{-A v^-}{r_{out}}+\frac{v^-}{R_f}
        =v^-\;\left(\frac{r_{out}-AR_f}{r_{out}R_f} \right)
        \]
        but as $v^-=v_s \,R_f/(R_s+R_1+R_f)$, the above becomes
        \[
        i_{sc}=v_s\left(\frac{R_f}{R_s+R_1+R_f}\right)
        \left(\frac{r_{out}-AR_f}{r_{out}R_f} \right)
        =v_s\,\frac{r_{out}-AR_f}{(R_s+R_1+R_f)r_{out}}  
        \]
      \item Find open-circuit output voltage: Applying KCL to the node 
        of $v^-$ we get
        \[
        \frac{v_s-v^-}{R_s+R_1}+\frac{(-Av^-)-v^-}{R_f+r_{out}}=0 
        \]
        Solving for $v^-$ we get
        \[
        v^-=v_s \left(\frac{R_f+r_{out}}{(A+1)(R_s+R_1)+R_f+r_{out}} \right)
        \]
        The open-circuit output voltage can be found to be (voltage divider)
        \begin{eqnarray}
          v_{oc}&=&[v^--(-Av^-)]\frac{r_{out}}{R_f+r_{out}}-Av^-
          =v^- \left(\frac{r_{out}-AR_f}{R_f+r_{out}}\right)
          \nonumber\\
          &=&v_s \;\frac{r_{out}-AR_f}{(A+1)(R_s+R_1)+R_f+r_{out}} 
          \nonumber
        \end{eqnarray}
      \item Find output resistance:
        \begin{eqnarray}
          R_{out}&=&\frac{v_{oc}}{i_{sc}}
          =\frac{(R_s+R_1+R_f)r_{out}}{(A+1)(R_s+R_1)+R_f+r_{out}}
          \nonumber\\
          &\approx& \frac{(R_s+R_1+R_f)r_{out}}{A(R_s+R_1)} 
          \approx \left(\frac{R_1+R_f}{R_1}\right) \frac{r_{out}}{A} 
          \nonumber
        \end{eqnarray}
        The approximation is based on $A\gg 1$ and $R_s\ll R_1,\,R_f$.

      \end{itemize}
  \end{itemize}


\item {\bf Non-Inverting Amplifier} (Homework)

%  \htmladdimg{../figures/noninverteramplifier.gif}
  \htmladdimg{../figures/NonInvertingAmplifer1.png}

  The three parameters of this non-inverting amplifier can be found 
  to be (see 
  \htmladdnormallink{here}{../noninvertingopamp/index.html}):
  \begin{itemize}
    \item open-circuit voltage gain:
      \[
      G_{oc} \approx \frac{R_1+R_f}{R_1}>1
      \]
    \item input resistance:
      \[
      R_{in} \approx \left(\frac{R_1}{R_1+R_f}\right)A r_{in}
      \approx\frac{A}{G_{oc}}r_{in}<A r_{in}
      \]
    \item output resistance:
      \[
      R_{out}\approx \left(\frac{R_1+R_f}{R_1} \right)\,\frac{r_{out}}{A} 
      \approx\frac{G_{oc}}{A}r_{out}>r_{out}/A
      \]
  \end{itemize}
  Comparing these results with those of the voltage follower, we see 
  that $G_{oc}>1$ is a little better, but both $R_{in}$ and $R_{out}$ are 
  a little worse. In particular if $R_f=0$, this non-inverting amplifier
  becomes a voltage follower with $G_{oc}\approx 1$, $R_{in}\approx Ar_{in}$,
  and $R_{out}\approx r_{out}/A$.

\end{itemize}


\subsection*{Op-Amp Circuits }

%\htmladdimg{../figures/opam1.gif}

To simplify the analysis of the op-amp circuits, we further make the following 
assumptions:
\begin{itemize}
\item The huge input resistance $r_{in}$ can be treated as infinity 
  $r_{in}\rightarrow \infty$.
\item The input current drawn by an op-amp is small ($10^{-9}\sim10^{-12}\;A$), 
  and could be approximated to be zero $I^+=I^-=0$.
\item The small output impedance $r_{out}$ can be treated as zero $r_{out}\approx 0$,
  i.e., the output $V_{out}$ is not affected by the load $R_L$ (so long as it is
  much greater than $r_{out}$).
\item Based on the assumptions that $V_{out}=A(V^+-V^-)$ is in the range between
  the positive and negative voltage supplies (e.g., $\pm 15\,V$, the {\em rails})
  and $A\rightarrow\infty$, we can assume $V^+-V^-=V_{out}/A\rightarrow 0$, i.e.,
  $V^-\approx V^+$. We can further assume if one of the two inputs is grounded, 
  the other one is {\em virtually grounded}. Based on this virtual ground assumption, 
  the analysis of all op-amp circuits is significantly simplified.
\item The bandwidth is large ($1 \sim 20MHz$), i.e., the property of the op-amp 
  remain unchanged for all frequencies of interest.
\end{itemize}
Based on these approximations, an op-amp can be further simplified as the modeled
shown on the right of the figure below, based on which the analysis of op-amp 
circuits can be much simplified, as shown in the following examples.

\htmladdimg{../figures/OpAmp0.gif}

\begin{itemize}
\item {\bf Voltage follower (buffer)}

\htmladdimg{../figures/voltagefollowermodel1.gif}

\[
V_{out}=V^-\approx V^+=V_{in} 
\]
As the output $V_{out}=V_{in}$ is the same as the input, why can't we replace this
op-amp circuit by a piece of wire?

\item {\bf Inverter}

\htmladdimg{../figures/opam2.gif}

Current into the op-amp is negligible, and $V^-\approx V^+=0$. Applying KCL to
the node of $V^-$, we have
\[ 
\frac{V_{in}}{R_1}+\frac{V_{out}}{R_f}=0,\;\;\;\;\;\;\mbox{i.e.}\;\;\;\;\;\;
V_{out}=-\frac{R_f}{R_1}V_{in}	
\]

In general, $R_1$ and $R_2$ of the inverter can be replaced by two networks (with
impedances $Z_1$ and $Z_2$ respectively) containing resistors and capacitors and
the analysis of the circuit can be carried out easily in frequency domain:
\[
H(j\omega)=\frac{V_{out}(j\omega)}{V_{in}(j\omega)}=-\frac{Z_2(j\omega)}{Z_1(j\omega)}
\]
This is a convenient way to design filters of various frequency characteristics.

\htmladdimg{../figures/opam11.gif}


\item {\bf Non-Inverting Amplifier}

  \htmladdimg{../figures/NonInverterOpAmp.gif}

  \[
  V_{in}=V^+\approx V^-=V_{out}\frac{R_1}{R_1+R_f},\;\;\;\;\mbox{i.e.}
  \;\;\;\;\;\;
  V_{out}=\left(\frac{R_1+R_f}{R_1}\right)V_{in}
  \]

\item {\bf Summer-inverter}

  \htmladdimg{../figures/opam3.gif}

  Apply KCL to $V^-$:
  \[ \sum_{k=1}^n \frac{V_k}{R_k}+\frac{V_{out}}{R_f}=0,\;\;\;\;\;\;\;
  \mbox{i.e.}\;\;\;\;\;\;\;V_{out}=-R_f \sum_{k=1}^n \frac{V_k}{R_k}
  =- \sum_{k=1}^n \frac{R_f}{R_k} \;V_k  \]

\item {\bf Differential amplifier}

  \htmladdimg{../figures/opamp10.gif}

  We first define $V=V^-\approx V^+$, and then apply KCL to both $V^-$ 
  and $V^+$ to get:
  \[ 
  \frac{V_1-V}{R_1}+\frac{V_{out}-V}{R_2}=0,\;\;\;\;\mbox{i.e.}\;\;\;\;
  V_{out}=-\frac{R_2}{R_1}V_1+\left(1+\frac{R_2}{R_1}\right) V 
  \]
  and
  \[ 
  V\approx V^+=\frac{R_4}{R_3+R_4}V_2 
  \]
  Substituting the second into the first we get:
  \[
  V_{out}=-\frac{R_2}{R_1}V_1+\left(\frac{R_1+R_2}{R_1}\right)\frac{R_4}{R_3+R_4}V_2 
  \]

  If one of the two inputs, e.g., $V_2=V_{ref}$ is connected to a constant
  voltage treated as a reference voltage, then the differential amplifier
  can also be used as a level shifter. The output is $-R_2/R_1$ times the 
  input $V_1$, shifted by a constant value 
  \[
  V_{shift}=\left(\frac{R_1+R_2}{R_1}\right)\frac{R_4}{R_3+R_4}V_2 
  \]
  Alternatively, if $V_1=V_{ref}$, the output is some scaled version of
  $V_2$ shifted by
  \[
  V_{shift}=-\frac{R_2}{R_1}V_{ref}
  \]

  Consider some special cases:
  \begin{itemize}
  \item If $R_1=R_3$ and $R_2=R_4$, then we get
    \[
    V_{out}=-\frac{R_2}{R_1}V_1+\left(\frac{R_1+R_2}{R_1}\right)\frac{R_4}{R_3+R_4}V_2 
    =\frac{R_2}{R_1}\;(V_2-V_1)	
    \]
  \item If $R_4=\infty$ (open circuit, and $R_3$ can be any value), then $V=V_2$,
    and the circuit is a combination of inverter and a non-inverter amplifiers:
    \[
    V_{out}=-\frac{R_2}{R_1}V_1+\left(1+\frac{R_2}{R_1}\right)V_2 
    \]
  \item If $R_1=R_4=\infty$, then $V_{out}=V_2$, the circuit becomes the 
    follower:
    \[
    V_{out}=-\frac{R_2}{R_1}V_1+\left(\frac{R_1+R_2}{R_1}\right)\frac{R_4}{R_3+R_4}V_2 
    =V_2
    \]
  \item If $R_4=0$ and $R_3=\infty$, then the circuit becomes the inverter:
    \[ 
    V_{out}=-\frac{R_2}{R_1}V_1+\left(\frac{R_1+R_2}{R_1}\right)\frac{R_4}{R_3+R_4}V_2 
    =-\frac{R_2}{R_1} V_1
    \]
  \item If $R_3=0$, $R_4=\infty$, and $V_1=0$, then the circuit becomes the 
    non-inverter:
    \[
    V_{out}=-\frac{R_2}{R_1}V_1+\left(\frac{R_1+R_2}{R_1}\right)\frac{R_4}{R_3+R_4}V_2 
    =\left(1+\frac{R_2}{R_1}\right)V_2 
    \]
  \end{itemize}

  It is likely that both inputs are subjected to some common noise
  $n(t)$ (e.g., the interference of 60Hz power supply):
  \[
  V'_1=V_1+n, \;\;\;\;\;V'_2=V_2+n
  \]
  In this case the output is 
  \[
  V_{out}=\frac{R_2}{R_1}\;(V'_2-V'_1)=\frac{R_2}{R_1}\;(V_2-V_1) 
  \]
  not affected by the common noise at all, i.e., the differential amplifier 
  can suppress {\em common-mode signal} (e.g., the noise signal $n(t)$)
  while amplify the {\em differential-mode signal} (e.g., $V_1$ and $V_2$).

\item {\bf Instrumentation Amplifier}

%\htmladdimg{../figures/instrumentamplifier.gif}

  The main drawback of the differential amplifier is that its input 
  impedance ($R_3+R_4$) may not be high enough if the output impedance 
  of the source is high. To overcome this problem, two non-inverting
  amplifiers with high input resistance are used each for one of the
  two inputs to the differential amplifier. The resulting circuit is 
  called the instrumentation amplifier.

  \htmladdimg{../figures/InstrumentationOpAmp.gif}

  The analysis of this circuit is very simple. As the output impedance
  resistance of the non-inverting amplifiers is very low, the output 
  voltage of the source circuit is not affected by the load circuit, 
  and the input voltage received by the load circuit is the same as 
  the open-circuit output voltage produced by the source circuit. 
  Therefore the outputs of the two non-inverters are:
  \[
  V'_1=V_1\left(1+\frac{R_f}{R_1}\right),\;\;\;\;\;\;\;\;\;
  V'_2=V_2\left(1+\frac{R_f}{R_1}\right) 
  \]
  The output voltage of the differential amplifier is:
  \[ 
  V_{out}=\frac{R_4}{R_3}(V'_2-V'_1)
  =\frac{R_4}{R_3}\left(1+\frac{R_f}{R_1}\right)(V_2-V_1) 
  \]
  Of course the two resistors $R_1$ can be combined to become $R_0=2R_1$,
  i.e., $R_1=R_0/2$, then the output can be written as:
  \[
  V_{out}=\frac{R_4}{R_3}\left(1+\frac{2R_f}{R_0}\right)(V_2-V_1) 
  \]

  Alternatively, we consider the current going from $V'_1$ to $V'_2$
  through $R_f$, $R_1$, $R_1$ and $R_f$:
  \[
  \frac{V'_1-V_1}{R_f}=\frac{V_1-V_0}{R_1}=\frac{V_0-V_2}{R_1}
  =\frac{V_2-V'_2}{R_f} 
  \]
  From the equation of the first two terms we get:
  \[
  V'_1=\left(1+\frac{R_f}{R_1}\right)V_1-\frac{R_f}{R_1}V_0 
  \]
  From the equation of the second two terms we get:
  \[
  V'_2=\left(1+\frac{R_f}{R_1}\right)V_2-\frac{R_f}{R_1}V_0 
  \]
  Using the equation of the differential amplifier above, we get the
  same result as above:
  \[
  V_{out}=\frac{R_4}{R_3}(V'_2-V'_1)
  =\frac{R_4}{R_3}\left(1+\frac{R_f}{R_1}\right)(V_2-V_1) 
  \]
  In the instrumentation circuit 
  \htmladdnormallink{AD623}{http://www.analog.com/media/en/technical-documentation/data-sheets/AD623.pdf},
  $R_3=R_4=R_f=50\,k\Omega$, $R_1=\infty$ (open-circuit), i.e., the 
  circuit has a unit voltage gain. However, if an external resistor 
  $R_G$ ($2R_1$) is connected to the circuit, the gain can be greater 
  up to 1000.
  

%Express the output voltage $V_{out}$ as a function of both inputs $V_1$ 
%and V_2$. Find the gain $A=V_{out}/(V_1-V_2)$.

%{\bf Hint:} Analyze the three op-amps separately. Assume the voltage at
%the middle point of $R_1$ is zero, i.e., the $v^-$ input of each of the
%two op-amps is grounded through $R_2/2$.
%\htmladdnormallink{Answer}{../instrumentamplifier/index.html}

\item {\bf Algebraic summer (inputs of different signs)}

  \htmladdimg{../figures/opamp5a.gif}

  
  
  \begin{comment}
  Define $V\stackrel{\triangle}{=}V^+ \approx V^- $. Apply KCL to $V^-$ and
  $V^+$ we get:
  \[
  \frac{V_1-V}{R_1}+\frac{V_2-V}{R_2}+\frac{V_{out}-V}{R_f}=0,\;\;\;\;\;\;
  \mbox{and}\;\;\;\;\;\;\;\;
  \frac{V_3-V}{R_3}+\frac{V_4-V}{R_4}=0 
  \]
  Solving the 2nd equation for $V$ we get:
  \[
  V=\frac{R_4}{R_3+R_4} V_3 + \frac{R_3}{R_3+R_4} V_4	
  \]
  and substitute it into the first equation to get
  \[
  V_{out}=-\frac{R_f}{R_1}V_1-\frac{R_f}{R_2}V_2
  +\left(\frac{R_f}{R_1}+\frac{R_f}{R_2}+1\right)
  \left(\frac{R_4}{R_3+R_4} V_3+\frac{R_3}{R_3+R_4} v_4\right) 
  \]
  \end{comment}

  It can be shown that (see \htmladdnormallink{here}{../opamp_sub/opamp_sub.html})
  the output is some algebraic sum of the inputs with both positive and
  negative coefficients:
  \[
  V_{out}=-\frac{R_f}{R_1}V_1-\frac{R_f}{R_2}V_2
  +\left(\frac{R_f}{R_1}+\frac{R_f}{R_2}+1\right)
  \left(\frac{R_4}{R_3+R_4} V_3+\frac{R_3}{R_3+R_4} v_4\right) 
  \]

\item {\bf Square Wave converter}

  Without feedback, the output of an op-amp is $V_{out}=A(V^+-V^-)$. As $A$ is
  large, $V_{out}$ is saturated, equal to either the positive or the negative
  voltage supply, depending on whether or not $V^+$ is greater than $V^-$. 
  When an input of any waveform is compared with a reference voltage, the
  output is a square wave:

  \htmladdimg{../figures/SquareConverter1.png}
  \htmladdimg{../figures/SquareConverter2.png}


\item {\bf A/D converter}

  These two possible outputs, positive and negative, can be treated as ``1'' 
  and ``0'' of the binary system. The figure shows an A/D converter built by
  three op-amps to measure voltage $V_{in}$ from 0 to 3 volts with resolution 
  1 V.

  \htmladdimg{../figures/opam12.gif}

  Due to the voltage divider, the input voltages to the three op-amps are, 
  respectively, 2.5V, 1.5V and 0.5V. The output of these op-amps are listed
  below for each of the input voltage levels. A digital logic circuit (a
  decoder) can convert the 3-bit output of the op-amps to the 2-bit binary 
  representation.

  \[
  \begin{array}{c||c|c|c|c}\hline
    \mbox{Input voltage}        & 0 	& 1	& 2	& 3	\\\hline
    \mbox{Op-amps Outputs}	& 000	& 001	& 011	& 111	\\\hline
    \mbox{Binary Representation}	& 00	& 01	& 10	& 11	\\ \hline
  \end{array}
  \]


\item {\bf Integrator and differentiator}

  \htmladdimg{../figures/opamp4.gif}

  {\bf Integrator}

  In time domain, as $v^-=v^+=0$ and $i_R+i_C=0$, we have (KCL)
  \[
  i_R+i_C=\frac{v_i}{R}+C\frac{d v_{out}}{dt}=0,
  \;\;\;\; \mbox{i.e.,}\;\;\;\;v_{out}=-\frac{1}{\tau} \int v_i dt	
  \]
  where $\tau \stackrel{\triangle}{=}RC$. In frequency domain, we have:
  \[
  H(j\omega)=-\frac{Z_2(j\omega)}{Z_1(j\omega)}=-\frac{1/j\omega C}{R}
  =-\frac{1}{j\omega RC}=-\frac{1}{j\omega \tau}	
  \]

  {\bf Differentiator}

  If we swap the resistor and the capacitor, we get in time domain:
  \[ 
  i_R+i_C=\frac{v_{out}}{R}+C\frac{d v_i}{dt}=0,\;\;\;\;
  \mbox{i.e.,}\;\;\;\;v_{out}=-RC \frac{d v_i}{dt}=-\tau \frac{d v_i}{dt}	
  \]
  In frequency domain, we have:
  \[ 
  H(j\omega)=-\frac{Z_2(j\omega)}{Z_1(j\omega)}=-\frac{R}{1/j\omega C}
  =-j\omega \tau 
  \]

\item {\bf Logarithmic and Exponential Amplifiers}

  \htmladdimg{../figures/ExpLogAmplifier.png}

  Based on the relationship between the current through and voltage 
  across a diode and the virtual ground assumption, we can show 
  that the output voltage of the exponential amplifier (left) is
  approximately an exponential function of the input voltage, and 
  the output voltage of the logarithmic amplifier (right) is 
  approximately a logarithmic function of the input voltage:
  \[
  v_{out}\approx C \;\exp(v_{in}/a),\;\;\;\;\;\; v_{out}= D\; \ln (v_{in}/b)
  \]
  (Homework: Derive this relationship and determine the coefficients 
  $C,\;D$ and $a,\,b$.)

\item {\bf PID controller}

  A proportional-integral-derivative (PID) controller can be implemented
  as shown. The output of the circuit is a linear combination of the
  signal together with its integral and derivative:
  \[
  v_{out}(t)=c_p v_{in}(t)+c_i \int v_{in}(t) dt +c_d\frac{d\,v_{in}}{dt}
  \]

  \htmladdimg{../figures/OpampPID.png}

\item {\bf Howland Current Source}

  \htmladdimg{../figures/HowlandCurrentSource.png}

  Assuming $R_2/R_1=R_4/R_3$, we can show that the output current
  through the load $R_L$ is a constant determined by the input
  voltages $V^-$ and $V^+$, as well as the circuit parameters
  (see \htmladdnormallink{here}{../HowlandCurrentSource/index.html}):
  \[
  I_L=\frac{V^+-V^-}{R_3}=\frac{V}{R_L}
  \]

  \begin{comment}
  \[
  \frac{V^--V}{R_1}+\frac{V_0-V}{R_2}=0,\;\;\;\;\;\;
  \frac{V^+-V}{R_3}+\frac{V_0-V}{R_4}=\frac{V}{R_L};
  \]
  Solving the first equation for $V_0-V$:
  \[
  V_0-V=(V-V^-)\frac{R_2}{R_1}=(V-V^-)\frac{R_4}{R_3}
  \]
  and substituting into the second equation, we get:
  \[
  \frac{V^+-V}{R_3}+\frac{V-V^-}{R_3}
  =\frac{V^+-V^-}{R_3}=\frac{V}{R_L}=I_L
  \]
  We see that the current through the load resistor $R_L$
  is constant, independent of $R_L$, i.e., the circuit is
  a current source.
  \end{comment}


\end{itemize}



\begin{comment}
\subsection*{Active Filters}

\subsubsection{First and Second Order Low/High/Band-Pass filters}

\htmladdimg{../figures/opamp4a.gif}

\begin{itemize}
\item {\bf Low-pass filter:}
  \begin{eqnarray}
    H(j\omega)&=&-\frac{Z_2}{Z_1}=-\frac{R_2\;||\;(1/j\omega C)}{R_1}
    =\frac{1}{R_1}\frac{R_2/j\omega C}{R_2+1/j\omega C}
    =-\frac{R_2}{R_1}\frac{1}{j\omega R_2C+1}
    \nonumber \\
    &=&-H(0)\frac{1}{1+j\omega \tau}=-H(0)\frac{1/\tau}{1/\tau+j\omega} 
    =-H(0)\frac{\omega_c}{j\omega+\omega_c} 
    \nonumber
  \end{eqnarray}
  where $H(0)=R_2/R_1$ is the DC gain when $\omega=0$,
  $\omega_c=1/\tau=1/R_2C$ is cut-off or corner frequency, at which 
  $|H(j\omega_c)|=H(0)/\sqrt{2}$. Intuitively, when frequency is high, 
  $Z_2(j\omega)$ is small and the negative feedback becomes strong,
  and the output is low. For example, when
  $\tau=10^{-3}$, $\omega_c=1/\tau=10^3$, the Bode plots are shown  
  below:

  \htmladdimg{../figures/BodeLP.gif}

\item {\bf High-pass filter:}
  \begin{eqnarray}
  H(j\omega)&=&-\frac{Z_2(j\omega)}{Z_1(j\omega)}
  =-\frac{R_2}{R_1+1/j\omega C}
  =-\frac{R_2}{R_1}\;\frac{1}{1+1/j\omega R_1C}
  \nonumber \\
  &=&-\frac{R_2}{R_1}\;\frac{j\omega}{j\omega +1/R_1C}
  =-H(0)\;\frac{j\omega}{j\omega+1/\tau} 
  =-H(0)\frac{j\omega}{j\omega+\omega_c} 
  \nonumber
  \end{eqnarray}
  where $H(0)=R_2/R_1$ is the DC gain, $\omega_c=1/\tau=1/R_1C$ is the 
  cut-off or corner frequency, at which $|H(j\omega_c)|=H(0)/\sqrt{2}$. 
  Intuitively, when frequency is low $Z_1(j\omega)$ is large and the 
  signal is difficult to pass, therefore the output is low. For example,
  when $\tau=10^{-6}$, $\omega_c=1/\tau=10^6$, the Bode plots are shown
  below:
  
  \htmladdimg{../figures/BodeHP.gif}

  If we let $R_1=R_2$, i.e., $H(0)=1$, and ignore the negative sign ($180^\circ$
  phase shift), the low-pass and high-pass filters can be represented by their
  transfer functions with $s=j\omega$:
  \[
  H_{lp}(j\omega)=\frac{\omega_c}{j\omega+\omega_c},
  \;\;\;\;\;\;\;H_{hp}(j\omega)=\frac{j\omega}{j\omega+\omega_c}
  \]

\item {\bf Second Order Band-pass Filters:}

  \htmladdimg{../figures/opamp4b.gif}

  \begin{eqnarray}
  H(j\omega)&=&-\frac{Z_2(j\omega)}{Z_1(j\omega)}
  =-\frac{R_2||(1/j\omega C_2)}{R_1+1/j\omega C_1}
  =-\frac{R_2/j\omega C_2}{(R_1+1/j\omega C_1)(R_2+1/j\omega C_2)}
  \nonumber \\
  &=&-\frac{j\omega R_2C_1}{(j\omega R_1C_1+1)(j\omega C_2R_2+1)}
  =-\frac{j\omega \tau_3}{(1+j\omega \tau_1)(1+j\omega \tau_2)} 
  \nonumber \\
  &=&\left(\frac{\omega_{c_1}}{j\omega+\omega_{c_1}}\right)\;
  \left(\frac{\omega_{c_2}}{j\omega+\omega_{c_2}}\right)\;
  \left(\frac{j\omega}{\omega_{c_3}}\right)
  \nonumber
  \end{eqnarray}
  where $\omega_{c_1}=1/\tau_1=1/R_1C_1$, $\omega_{c_2}=1/\tau_2=1/R_2C_2$, 
  and $\omega_{c_3}=1/\tau_3=1/R_2C_1$.

  For example, when $\tau_1=10^{-6}$, $\tau_2=10^{-8}$, $\tau_3=10^{-3}$, the Bode
  plots are shown below:

  \htmladdimg{../figures/BodeBP.gif}

\end{itemize}

\subsubsection{The Sallen-Key filters}

The
\htmladdnormallink{Sallen-Key filters}{http://en.wikipedia.org/wiki/Sallen-Key_topology}
are second-order active filters (low-pass, high-pass, and band-pass) that 
can be easily implemented using the configuration below:

\htmladdimg{../figures/SallenKey.gif}

We represent all voltages in phasor form. Due to the virtual ground 
assumption, $V_b$ at non-inverting input is virtually the same as that
at the inverting input, which is connected to the output $V_{out}$. 
Applying KCL to nodes a and b to get:
\[ 
\frac{V_a-V_{in}}{Z_1}+\frac{V_a-V_{out}}{Z_3}+\frac{V_a-V_{out}}{Z_2}=0 
\]
\[
\frac{V_{out}-V_a}{Z_2}+\frac{V_{out}}{Z_4}=0 
\]
Solving the second equation for $V_a$ we get:
\[
V_a=V_{out}\;\frac{Z_2+Z_4}{Z_4},
\;\;\;\;\;\;\mbox{i.e.}\;\;\;\;\;
V_a-V_{out}=V_{out}\frac{Z_2}{Z_4} 
\]
Substituting these into the first equation we get 
\[
V_{out}\;\frac{Z_2+Z_4}{Z_1Z_4}-\frac{V_{in}}{Z_1}
+V_{out}\frac{Z_2}{Z_3Z_4}+V_{out}\frac{Z_2}{Z_2Z_4} 
\]
Now the frequency response of the Sallen-Key filter can be found as
the ratio of $V_{out}$ and $V_{in}$:
\begin{eqnarray}  
H&=&\frac{V_{out}}{V_{in}}=
\frac{Z_3Z_4}{Z_1Z_2+Z_1Z_3+Z_2Z_3+Z_3Z_4} 
=\frac{Z_3Z_4}{Z_1Z_2+(Z_1+Z_2)Z_3+Z_3Z_4} 
\nonumber
\end{eqnarray}

\begin{itemize}
\item {\bf Second order low-pass filter}
  
  We let $Z_1=R_1$, $Z_2=R_2$, $Z_3=1/j\omega C_1$, $Z_4=1/j\omega C_2$, 
  and get
    
  \begin{eqnarray}
    H&=&\frac{1/(j\omega)^2C_1C_2}
    {R_1R_2+(R_1+R_2)/j\omega C_1+1/(j\omega)^2C_1C_2}
    \nonumber \\
    &=&\frac{1/R_1R_2C_1C_2}{(j\omega)^2+j\omega(R_1+R_2)/R_1R_2C_1+1/R_1R_2C_1C_2}
    \nonumber \\
    &=&\frac{1/R_1R_2C_1C_2}{(j\omega)^2+j\omega/R_pC_1+1/R_1R_2C_1C_2}
    \nonumber \\
    &=&\frac{\omega_n^2}{(j\omega)^2+2\zeta\omega_n\;j\omega+\omega_n^2} 
    =\frac{\omega_n^2}{(j\omega)^2+\omega_n/Q \;j\omega+\omega_n^2}
    =\frac{\omega_n^2}{(j\omega)^2+\Delta\omega\;j\omega+\omega_n^2} 
    \nonumber
  \end{eqnarray}
  where
  \[
  R_p=R_1||R_2=\frac{R_1R_2}{R_1+R_2},\;\;\;\;
  \omega_n=\frac{1}{\sqrt{R_1R_2C_1C_2}},\;\;\;\;\;\;\;
  \Delta\omega=\frac{1}{R_pC_1}
  \]
  and
  \[
  Q=\frac{\omega_n}{\Delta\omega}=\frac{\sqrt{R_1R_2C_1C_2}}{(R_1+R_2)C_2},
  \;\;\;\;\;
  \zeta=\frac{1}{2Q}=\frac{(R_1+R_2)C_2}{2\sqrt{R_1R_2C_1C_2}}
  \]
  As there are only two parameters $\omega_n$ and $\zeta$ or $Q=1/2\zeta$ 
  to satisfy, we can arbitrarily set any two of the four variables $R_1$, 
  $R_2$, $C_1$, and $C_2$, and then solve for the other two. For example, 
  for convenience, if we let $R_1=R_2=1$, we get
  \[
  \omega_n=\frac{1}{\sqrt{C_1C_2}},\;\;\;\;\;\Delta\omega=\frac{1}{C_1}
  \]
%  \begin{comment}
    \[
    \zeta=\sqrt{\frac{C_2}{C_1}}
    \]
    Solving these we get
    \[
    C_1=\frac{1}{\omega_n\zeta},\;\;\;\;\;\;C_2=\frac{\zeta}{\omega_n}
    \]
%  \end{comment}

\item {\bf Second order high-pass filter}
    
  We let $Z_1=1/j\omega C_1$, $Z_2=1/j\omega C_2$, $Z_3=R_1$, $Z_4=R_2$, 
  and get 
  \begin{eqnarray}
    H&=&\frac{R_1R_2}
    {1/(j\omega)^2C_1C_2+R_1(1/j\omega C_1+1/j\omega C_2)+R_1R_2}
    \nonumber \\
    &=&\frac{(j\omega)^2}{1/R_1R_2C_1C_2+j\omega\;(C_1+C_2)/C_1C_2R_2+(j\omega)^2}
    \nonumber \\
    &=&\frac{(j\omega)^2}{(j\omega)^2+j\omega\;/C_sR_2+1/R_1R_2C_1C_2}
    \nonumber \\
    &=&\frac{(j\omega)^2}{(j\omega)^2+2\zeta\omega_n\;j\omega+\omega_n^2} 
    =\frac{(j\omega)^2}{(j\omega)^2+\omega_n/Q \;j\omega+\omega_n^2} 
    =\frac{(j\omega_n)^2}{(j\omega)^2+\Delta\omega\;j\omega+\omega_n^2}
    \nonumber
  \end{eqnarray}
  where 
  \[
  C_s=\frac{C_1C_2}{C_1+C_2},\;\;\;\;\;
  \omega_n=\frac{1}{\sqrt{R_1R_2C_1C_2}},\;\;\;\;\;\;\;
  \Delta\omega=\frac{1}{C_sR_2}
  \]
  and
  \[
  Q=\frac{\sqrt{R_1R_2C_1C_2}}{(C_1+C_2)R_1},\;\;\;\;\;\;
  \zeta=\frac{1}{2Q}=\frac{(C_1+C_2)R_1}{2\sqrt{R_1R_2C_1C_2}}
  \]

\item {\bf Band-pass filter}

  \htmladdimg{../figures/SallenKeyBP.png}

  By voltage divider and virtual ground, we get
  \[
  V_2=\frac{R_a}{R_a+R_b}V_{out}=kV_{out},\;\;\;\;\;\;\;\left(k=\frac{R_a}{R_a+R_b}\right)
  \]
  Apply KCL to node $V_2$ to get:
  \[
  \frac{V_1-V_2}{Z_2}=\frac{V_2}{Z_4},\;\;\;\;\;\;\;\mbox{i.e.}\;\;\;\;\;
  V_1=V_2\left(\frac{1}{Z_2}+\frac{1}{Z_4}\right)Z_2
  =V_2\left(1+\frac{Z_2}{Z_4}\right)
  \]
  Apply KCL to node $V_1$ to get:
  \[
  \frac{V_{in}-V_1}{Z_1}+\frac{V_{out}-V_1}{R_f}+\frac{V_2-V_1}{Z_2}
  =\frac{V_1}{Z_3}
  \]
  Rearrange the terms, and replace $V_1$ by $V_2(1+Z_2/Z_4)$ to get
  \[
  \frac{V_{in}}{Z_1}+\frac{V_{out}}{R_f}+\frac{V_2}{Z_2}
  =V_1\left(\frac{1}{Z_1}+\frac{1}{R_f}+\frac{1}{Z_2}+\frac{1}{Z_3}\right)
  =V_2\left(1+\frac{Z_2}{Z_4}\right)\left(\frac{1}{Z_1}
  +\frac{1}{R_f}+\frac{1}{Z_2}+\frac{1}{Z_3}\right)
  \]
  Further rearrange the terms and replace $V_2$ by $kV_{out}$ to get
  \[
  \frac{V_{in}}{Z_1}+\frac{V_{out}}{R_f}
  =kV_{out}\left[\left(1+\frac{Z_2}{Z_4}\right)\left(\frac{1}{Z_1}+\frac{1}{R_f}
    +\frac{1}{Z_2}+\frac{1}{Z_3}\right)-\frac{1}{Z_2}\right]
  \]
  Further rearrange the terms
  \begin{eqnarray}
    \frac{V_{in}}{Z_1}&=&kV_{out}\left[\left(1+\frac{Z_2}{Z_4}\right)
\left(\frac{1}{Z_1}
      +\frac{1}{R_f}+\frac{1}{Z_2}+\frac{1}{Z_3}\right)-\frac{1}{Z_2}\right]-\frac{V_{out}}{R_f}
    \nonumber \\
    &=&
    V_{out}\left\{k\left[\left(1+\frac{Z_2}{Z_4}\right)\left(\frac{1}{Z_1}
      +\frac{1}{R_f}+\frac{1}{Z_2}+\frac{1}{Z_3}\right)-\frac{1}{Z_2}\right]-\frac{1}{R_f}\right\}
    \nonumber 
  \end{eqnarray}
  Finally we get the frequency response function:
  \begin{eqnarray}
    H=\frac{V_{out}}{V_{in}}&=&
    \frac{1}{Z_1\left\{k\left[\left(1+\frac{Z_2}{Z_4}\right)\left(\frac{1}{Z_1}+\frac{1}{R_f}+\frac{1}{Z_2}+\frac{1}{Z_3}\right)-\frac{1}{Z_2}\right]-\frac{1}{R_f}\right\}}
    \nonumber \\
    &=&
    \frac{1/k}{Z_1\left[\left(1+\frac{Z_2}{Z_4}\right)\left(\frac{1}{Z_1}+\frac{1}{R_f}+\frac{1}{Z_2}+\frac{1}{Z_3}\right)-\frac{1}{Z_2}\right]-\frac{Z_1}{kR_f}}
    \nonumber \\
    &=&
    \frac{1/k}{1+\frac{Z_1}{R_f}+\frac{Z_1}{Z_3}+\frac{Z_2}{Z_4}+\frac{Z_1Z_2}{Z_4R_f}+\frac{Z_1}{Z_4}+\frac{Z_1Z_2}{Z_3Z_4}-\frac{Z_1}{kR_f}}
    \nonumber
  \end{eqnarray}
  Now if we let
  \[
  Z_1=R_1,\;\;\;\;\;Z_4=R_2,\;\;\;\;Z_2=\frac{1}{j\omega C_2},
  \;\;\;\;Z_3=\frac{1}{j\omega C_1}
  \]
  the frequency response function becomes
  \[
  H=\frac{\left(1+\frac{R_b}{R_a}\right)\frac{1}{R_1C_1}j\omega}{(j\omega)^2+\left(\frac{1}{R_2C_1}+\frac{1}{R_2C_2}+\frac{1}{R_1C_1}-\frac{R_b}{R_aR_fC_1}\right)j\omega+\frac{R_1+R_f}{R_fR_1R_2C_1C_2}}
  =\frac{Aj\omega}{(j\omega)^2+\Delta\omega j\omega+\omega_n^2}
  \]
  This is a band-pass filter with the peak frequency equal to the natural
  frequency:
  \[
  \omega_n=\sqrt{\frac{R_1+R_f}{R_fR_1R_2C_1C_2}}
  \]
  the bandwidth
  \[
  \Delta\omega=\frac{1}{R_2C_1}+\frac{1}{R_2C_2}+\frac{1}{R_1C_1}-\frac{R_b}{R_aR_fC_1}
  \]
  The gain of the filter is controlled by $1+R_b/R_a$.
  
\end{itemize}


\subsubsection{The Twin-T notch (band-stop) filter}

{\bf The twin-T filter}

\htmladdimg{../figures/TwinT.png}

%\begin{comment}
The twin-T network is composed of two T-networks: 
\begin{itemize}
\item The RCR network is formed by two resistors $R_1=R_2=R$ and one
  capacitor $C_3=2C$. This T (or Y) network can be converted to a
  $\pi$ (or $\Delta$) network (see 
  \htmladdnormallink{here}{http://fourier.eng.hmc.edu/e84/lectures/ch2/node3.html}):
  :
  \[
  Z'_1=Z'_2=R+\frac{1}{2j\omega C}+\frac{R/2j\omega C}{R}=R+\frac{1}{j\omega C}
  =\frac{j\omega RC+1}{j\omega C}
  \]
  \[
  Z'_3=R+R+\frac{R^2}{1/2j\omega C}=2R+2R^2\omega C=2R(1+j\omega RC)
  \]
  The frequency response function of the voltage divider formed by $Z'_3$ and 
  $Z'_2$ is:
  \[
  H'(j\omega)=\frac{Z'_2}{Z'_2+Z'_3}
  =\frac{(1+j\omega RC)/j\omega C}{(1+j\omega RC)/j\omega C+2R(1+j\omega RC)}
  =\frac{1}{1+2j\omega RC}=\frac{1}{1+2j\omega\tau}
  \]
  where $\tau=RC$. 
  \[
  |H'(j\omega)|=\left\{\begin{array}{ll}1 & \omega=0\\1/\sqrt{2}& 
  \omega=1/2\tau\\ 0&\omega\rightarrow\infty\end{array}\right.
  \]
  This is a first-order low-pass filter with cut-off frequency at
  $\omega_c=1/2\tau$.

\item The CRC network is formed by two capacitors $C_1=C_2=C$ and one
  resistor $R_3=R/2$. This T (or Y) network can be converted to a
  $\pi$ (or $\Delta$) network:
  \[
  Z''_1=Z''_2=\frac{R}{2}+\frac{1}{j\omega C}+\frac{R/2j\omega C}{1/j\omega C}
  =R+\frac{1}{j\omega C}  =\frac{j\omega RC+1}{j\omega C}
  \]
  \[
  Z''_3=\frac{1}{j\omega C}+\frac{1}{j\omega C}+\frac{1/(j\omega C)^2}{R/2}
  =\frac{2}{j\omega C}+\frac{2}{R(j\omega C)^2}
  =\frac{2(1+j\omega RC)}{R(j\omega C)^2}
  \]
  The frequency response function of the voltage divider formed by $Z''_3$ 
  and $Z''_2$ is:
  \begin{eqnarray}
  H'(j\omega)&=&\frac{Z''_2}{Z''_2+Z''_3}
  =\frac{(1+j\omega RC)/j\omega C}
  {(1+j\omega RC)/j\omega C+2(1+j\omega RC)/R(j\omega C)^2}
  \nonumber \\
  &=&\frac{j\omega RC}{j\omega RC+2}=\frac{j\omega\tau}{j\omega\tau+2}
  \nonumber 
  \end{eqnarray}
  and
  \[
  |H'(j\omega)|=\left\{\begin{array}{ll}0 & \omega=0\\1/\sqrt{2}& 
  \omega=2/\tau\\ 1&\omega\rightarrow\infty\end{array}\right.
  \]
  This is a first-order high-pass filter with cut-off frequency at
  $\omega_c=2/\tau$.

\end{itemize}

As these two $\pi$-networks are combined in parallel, they form a single
$\pi$-network with three branches $Z_1=Z'_1||Z''_1$, $Z_2=Z'_2||Z''_2$, 
and $Z_3=Z'_3||Z''_3$:
\[
Z_1=Z'_1||Z''_1=Z_2=Z'_2||Z''_2=\frac{1}{2}\left(R+\frac{1}{j\omega C}\right)
\]
\[
  Z_3=Z'_3||Z''_3=\frac{Z'_3 Z''_3}{Z'_3+Z''_3}
  =\frac{4(1+j\omega RC)^2/(j\omega C)^2}{2R(1+j\omega RC)(1/R(j\omega C)^2+1)}
  =\frac{2R(1+j\omega RC)}{1+(j\omega RC)^2}
\]
The frequency response function of this $\pi$-network (a voltage divider) is:
\begin{eqnarray}
  H(j\omega)&=&\frac{Z_2}{Z_2+Z_3}=\frac{R+1/j\omega C}
  {R+1/j\omega C+4R(1+j\omega RC)/(1+(j\omega RC)^2)}
  \nonumber \\
  &=&\frac{(1+j\omega RC)/j\omega C}{(1+j\omega RC)/j\omega C+4R(1+j\omega RC)/(1+(j\omega RC)^2)}
  \nonumber \\
  &=&\frac{1/j\omega C}{1/j\omega C+4R/(1+(j\omega RC)^2)}
  =\frac{1}{1+4j\omega RC/(1+(j\omega RC)^2)}
  \nonumber \\
  &=&\frac{1+(j\omega\tau)^2}{1+(j\omega\tau)^2+4j\omega\tau}
  =\frac{(j\omega)^2+(1/\tau)^2}{(j\omega)^2+4j\omega/\tau+(1/\tau)^2}
  \nonumber
\end{eqnarray}
We define
\[
\omega_n=\frac{1}{RC}=\frac{1}{\tau},\;\;\;\;\;\mbox{i.e.,}\;\;\;\;\;
f_0=\frac{1}{2\pi RC}=\frac{1}{2\pi\tau}
\]
and express the second order denominator in the canonical form as
\[
H(j\omega)=\frac{V_{out}}{V_{in}}=\frac{(j\omega)^2+\omega_n^2}{(j\omega)^2+4\omega_nj\omega+\omega_n^2}
=\frac{(j\omega)^2+\omega_n^2}{(j\omega)^2+\omega_nj\omega/Q+\omega_n^2}
=\frac{\omega_n^2-\omega^2}{\omega_n^2-\omega^2+j\Delta\omega\;\omega}
\]
where $Q=1/4=0.25$ is the quality factor, and 
$\Delta\omega=\omega_n/Q=4\omega_n$ is the bandwidth of the filter.

\begin{itemize}
\item When $\omega=0$, 
  \[
  H(j\omega)\big|_{\omega=0}=H(0)=\frac{\omega_n}{\omega_n}=1
  \]
\item When $\omega\rightarrow \infty$, 
  \[
  H(j\omega)\big|_{\omega\rightarrow\infty}=\lim\limits_{\omega\rightarrow\infty}
  \left( \frac{\omega_n^2-\omega^2}{-\omega^2+j\omega_n\omega/Q+\omega_n^2}
  \right)
  =\lim\limits_{\omega\rightarrow\infty}\left(\frac{\omega^2}{\omega^2}\right)=1
  \]
\item When $\omega=\omega_n=1/\tau$, $H(j\omega_n)=0$
\end{itemize}
This twin-T network is a band-stop filter (notch filter) which attenuates 
the frequency $\omega_n=1/\tau$ to zero. This result can also be reached
by noticing the following
\[
H'(\omega)\big|_{\omega=1/\tau}=\frac{1}{1+j2},\;\;\;\;\;\;\;
H''(\omega)\big|_{\omega=1/\tau}=\frac{1}{1-j2}
\]
As they are equal in magnitude but opposite in phase, their outputs 
cancel each other to produce zero output.

When this notch filter
is used in a negative feedback loop of an amplifier, it becomes an oscillator.

{\bf The active twin-T filter}

The bandwidth $\Delta\omega=\omega_n/Q=4\omega_n$ is too large for most 
applications due to the small quality factor $Q=1/4$. To overcome this 
problem, an active filter containing two opam followers (with unity gain
$A=1$) can be used to introduce a positive feedback loop as shown below:

\htmladdimg{../figures/TwinTActive.png}

Now the common terminal of the twin-T filter is no longer grounded, instead
it is connected a potentiometer, to the voltage divider composed of $R_4$ 
and $R_5$, to form a feedback loop by which a fraction of the output 
$V_{out}$ is fed back:
\[
V_1=\frac{R_5}{R_4+R_5}\;V_{out}
\]
The input and output of the twin-T network are respectively $V_{in}-V_1$ and
$V_{out}-V_1$, and they are now related by the frequency response function 
$H(j\omega)$ of the twin-T network:
\[
V_{out}-V_1=H(V_{in}-V_1)
\]
Substituting $V_1=V_{out}\;R_5/(R_4+R_5)$ and rearranging, we get
\begin{eqnarray}
  HV_{in}&=&V_{out}+(H-1) V_1=V_{out}+(H-1)\frac{R_5}{R_4+R_5}V_{out}
  \nonumber \\
  &=&V_{out}\;\left(1+(H-1)\;\frac{R_5}{R_4+R_5}\right)
  \nonumber
\end{eqnarray}
The frequency response function of this active filter with feedback can be 
found to be
\[
H_{active}(j\omega)=\frac{V_{out}}{V_{in}}=\frac{H}{1+(H-1)R_5/(R_4+R_5)}
=\frac{H(R_4+R_5)}{(R_4+R_5)+(H-1)R_5}
\]
Substituting $H(j\omega)=((j\omega)^2+\omega_n^2)/((j\omega)^2+4\omega_n j\omega
+\omega_n^2)$ and
$H(j\omega)-1=-4\omega_nj\omega/((j\omega)^2+4\omega_nj\omega+\omega_n^2)$, we
get
\begin{eqnarray}
  H_{active}(j\omega)&=&\frac{\omega_n^2-\omega^2)(R_4+R_5)}{(R_4+R_5)
    (\omega_n^2-\omega^2+j4\omega_n \omega)- 4\omega_n j\omega R_5}
  \nonumber \\
  &=&\frac{\omega_n^2-\omega^2}{(\omega_n^2-\omega^2+j4\omega_n\omega)
    -4\omega_n j\omega\;R_5/(R_4+R_5)}
  \nonumber \\
  &=&\frac{\omega_n^2-\omega^2}{\omega_n^2-\omega^2+\omega_n j4\omega(1-R_5/(R_4+R_5))
}
  \nonumber \\
  &=&\frac{\omega_n^2-\omega^2}{\omega_n^2-\omega^2+\omega_n j4\omega R_4/(R_4+R_5)}
  \nonumber \\
  &=&\frac{\omega_n^2-\omega^2}{\omega_n^2-\omega^2+\omega_n/Q_{active} j\omega}
  =\frac{\omega_n^2-\omega^2}{\omega_n^2-\omega^2+\Delta\omega_{active} j\omega}
  \nonumber
\end{eqnarray}
where $Q_{active}$ and $\Delta\omega_{active}$ are respectively the quality
factor and the bandwidth of the active filter with feedback:
\[
Q_{active}=\frac{R_4+R_5}{4R_4},\;\;\;\;\;\;
\Delta\omega_{active}=\frac{\omega_n}{Q'}
\]
By varying $R_4$ and $R_5$, the quality factor $Q_{active}$ and bandwidth 
$\Delta\omega_{active}$ can be adjusted. In particular, 
\begin{itemize}
\item when $R_5=0$, $V_1=0$ (no feedback), $Q_{active}=1/4=Q$, 
  $\Delta\omega=\omega_n/Q=4\omega_n$; 
\item when $R_4=0$, $V_1=V_{out}$ (one hundred percent feedback), 
  $Q_{active}=\infty$, $\Delta\omega_{active}=\omega_n/Q_{active}=0$.
\end{itemize}


{\bf The bridged T filter}

If in the CRC T-network the vertical capacitor branch is dropped, i.e.,
$C=0$, while the RCR T-network is still the same, we get a bridged T 
network. Now $Z''_3=2R$, and we have
\[
  Z_3=Z'_3||Z''_3=\frac{Z'_3 Z''_3}{Z'_3+Z''_3}
  =\frac{2R(1+j\omega RC)}{1+j\omega RC+(j\omega RC)^2}
\]
The frequency response function of this bridged T network (a voltage 
divider) is:
\begin{eqnarray}
  H(j\omega)&=&\frac{Z_2}{Z_2+Z_3}=\frac{R+1/j\omega C}{R+1/j\omega C+2R(1+j\omega RC)/(1+j\omega RC+(j\omega RC)^2)}
  \nonumber \\
  &=&\frac{1/C}{1/j\omega C+2R/(1+j\omega RC+(j\omega RC)^2)}
  =\frac{1}{1+2j\omega RC/(1+j\omega RC+(j\omega RC)^2)}
  \nonumber \\
  &=&\frac{1+j\omega RC+(j\omega RC)^2}{1+3j\omega RC+(j\omega RC)^2}
  =\frac{(j\omega)^2+j\omega /RC+1/(RC)^2}{(j\omega)^2+3j\omega /RC+1/(RC)^2}
  \nonumber
\end{eqnarray}
We let $\omega_n=1/RC$, and express the second order denominator in the
canonical form as
\[
H(j\omega)=\frac{(j\omega)^2+\omega_n j\omega +\omega_n^2}{(j\omega)^2+3\omega_nj\omega +\omega_n^2}
=\frac{(j\omega)^2+\Delta\omega_n j\omega+\omega_n^2}{(j\omega)^2+\Delta\omega_dj\omega +\omega_n^2}
=\frac{\omega_n^2-\omega^2+\Delta\omega_n j\omega }{\omega_n^2-\omega^2+\Delta\omega_dj\omega}
\]
where 
\[
\Delta\omega_n=\omega_n,\;\;\;\;\;\;\;\Delta\omega_d=3\omega_n
\]
are the bandwidth of the 2nd-order systems of the numerator and the
denominator, respectively. 
\begin{itemize}
\item If $\omega=0$, $H(j\omega)=H(0)=1$
\item If $\omega\rightarrow \infty$, $H(j\omega)=1$
\item If $\omega=\omega_0=1/RC$, $H(j\omega_0)=1/3$
\end{itemize}
We see that this is a band-stop filter.

%\end{comment}

\subsubsection*{Wien bridge}

The Wien bridge is a particular type of the Wheatstone bridge of which
two of the four arms are composed of a capacitor as well as a resistor
in parallel and series:

%\htmladdimg{../figures/WienBridge.png}
\htmladdimg{../figures/WienBridge2.png}

For this bridge to balance, the ratios of the left and right branches
should be the same:
\[
\frac{R_3}{R_4}=\frac{R_2+1/j\omega C_2}{R_1||1/j\omega C_1}
=\frac{(j\omega R_1C_1+1)(j\omega R_2C_2+1)}{j\omega R_1C_2}
=\frac{1-\omega^2R_1R_2C_1C_2+j\omega(R_1C_1+R_2C_2)}{j\omega R_1C_2}
\]
For this equation to hold, the right-hand size needs to be real, i.e.,
\[
1-\omega^2R_1R_2C_1C_2=0,\;\;\;\;\;\mbox{i.e.,}\;\;\;\;\;
\omega=\frac{1}{\sqrt{R_1R_2C_1C_2}}
\]
and the equation above becomes
\[
\frac{R_3}{R_4}=\frac{R_1C_1+R_2C_2}{R_1C_2}=\frac{C_1}{C_2}+\frac{R_2}{R_1}
\]
In particular, if $R_1=R_2=R$ and $C_1=C_2=C$, we have:
\[
\omega=\frac{1}{\sqrt{R_1R_2C_1C_2}}
=\frac{1}{\sqrt{R^2C^2}}=\frac{1}{RC}
\]
and
\[
\frac{R_3}{R_4}=\frac{C_1}{C_2}+\frac{R_2}{R_1}=1+1=2,
\;\;\;\;\;\;\mbox{i.e.}\;\;\;\;\;R_4=2R_3
\]

\subsection{Colpitts Oscillators}

An oscillator is a feedback system composed of a forward path with
gain $G(j\omega)$ and a feedback path with gain $F(j\omega)$:

\htmladdimg{../figures/OscillatorModel.png}

For the system to oscillate, the feed back needs to be positive for
the feedback signal $FY$ to positively reinforce the signal going 
through the forward path in order to sustain the output $Y$ with
zero input $X=0$. Specifically, the output $Y$ and the input $X$ 
of a feedback system are related by
\[
Y=G(X+FY)=GX+GFY,\;\;\;\;\;\;\;\;\;\frac{Y}{X}=H=\frac{G}{1-GF},
\;\;\;\;\;\;\;Y=HX=\frac{G}{1-GF}\,X
\]
where $GF$ is the open-loop gain and $H$ is the closed-loop gain.
For this system to oscillate, i.e., for it to produce an output with
zero input, its closed-loop gain needs to be infinite, i.e., its 
open-loop gain $GF$ need to be real, with zero phase $\angle(GF)=0$
and unit gain $|GF|=1$. 

Oscillation is desirable if the system is used as a sinusoidal source, 
but it is undesirable if the system is a amplifier or part of a
control system which needs to be stable without oscillation.

%(In practice, $|GF|\ge 1$ to compensate energy attenuation in the physical system.)



%\begin{comment}
The op-amp circuit shown in the figure has both positive and negative
feedback branches. If the voltage to non-inverting input $V_+$ is 
considered as the input, the circuit is a non-inverting amplifier 
with gain:
\[
A=\frac{V_o}{V_+}=1+\frac{R_f}{R_3}=K
\]
On the other hand, the positive feedback gain is
\[
B(j\omega)=\frac{V_+}{V'_o}=\frac{Z_1}{Z_1+Z_2}
=\frac{R_1||1/j\omega C_1}{R_1||1/j\omega C_1+R_2+1/j\omega C_2}
\]
%\end{comment}

There exist many different configurations of oscillators based on a
single transistor. Shown below are three typical Colpitts oscillators:
common-base (CB, left), common emitter (CE, middle), and common 
collector (CC, right). All such circuits contain a ``tank'' LC circuit
composed of an inductor $L$ in parallel with $C_1$ and $C_2$ in series, 
with a resonant frequency 
\[
\omega_0=\frac{1}{\sqrt{LC_s}},\;\;\;\;\;\;\mbox{where}\;\;\;\;\;
C_s=\left(\frac{1}{C_1}+\frac{1}{C_2}\right)^{-1}
\]
where $C_s$ is the equivalent capacitance of the series combination 
of $C_1$ and $C_2$. All other $C$s (without a subscript) are coupling
capacitors that have a large enough capacitance and can therefore be 
treated as short circuit for AC signals. 

%\htmladdimg{../figures/ColpittsOscillators.png}
\htmladdimg{../figures/Colpitts3a.png}

Here are the requirements for these circuits to oscillate:
\begin{enumerate}
\item an LC tank tuning circuit that generates sinusoidal oscillation 
  at its resonant frequency $\omega_0=1/\sqrt{LC_s}$
\item a positive feedback loop that sustains the oscillation.
\end{enumerate}
How each of these circuits works can be qualitatively understood as 
below:
\begin{itemize}
\item CB with the base AC grounded: The collector voltage $V_c$ is the 
  output, a fraction of which at the middle point between the two 
  capacitors, ``tap point'', is fed-back to the emitter to a positive 
  feedback loop:
  \[
  V_c\uparrow\Longrightarrow V_e\uparrow\Longrightarrow \V_{be}\downarrow
  \Longrightarrow I_b\downarrow \Longrightarrow I_c\downarrow
  \Longrightarrow V_c\uparrow
  \]
\item CE with the emitter AC grounded: The collector voltage $V_c$ is 
  the output, which is fed-back through the LC tank circuit to the base. 
  As the tap point is grounded, the sinusoidal voltage across the LC
  tank produces opposite voltage polarities at the far ends of $C_1$ 
  and $C_2$, i.e., $V_{C_1}=V_b$ and $V_{C_2}=V_c$ have opposite phases 
  and thereby form a positive feedback loop:  
  \[
  V_c\uparrow\Longrightarrow V_b\downarrow\Longrightarrow \V_{be}\downarrow
  \Longrightarrow I_b\downarrow \Longrightarrow I_c\downarrow
  \Longrightarrow V_c\uparrow
  \]
\item CC with the collector AC grounded: This a voltage follower circuit
  in which the emitter voltage $V_e$ is the output that follows the input
  voltage $V_b$. The feedback from the emitter through the LC tank circuit 
  to the base form a positive feedback loop:
  \[
  V_e\uparrow\Longrightarrow V_t\uparrow\Longrightarrow V_b\uparrow
  \Longrightarrow I_b\uparrow\Longrightarrow I_e\uparrow
  \Longrightarrow V_e\uparrow
  \]
  where $V_t$ is the voltage at the tap point.

\end{itemize}

More specifically, we consider the common-collector circuit as an example.
To find out why the circuit oscillates and the resonant frequency, we 
disconnect the base path of the circuit and consider the open-loop gain 
of $H=V_o/V_i$ of the feedback loop. We further model the transistor 
by a Thevenin voltage source $V_i$ in series with an internal $R$, as 
shown in the figure:

\htmladdimg{../figures/ColpittsModel.png}

As the load of the Thevenin source, the tank circuit receives an input 
$V_t$ at the tap point, and produces an output $V_o$ across the parallel
combination of $L$ and $C_1$ in series with $C_2$. Applying KCL at the tap 
point we get:
\[
\frac{V_t-V_i}{R}+\frac{V_t}{1/j\omega C_2}+\frac{V_t}{j\omega L+1/j\omega C_1}=0
\]
i.e.,
\[
V_t\left(\frac{1}{R}+j\omega C_2+\frac{j\omega C_1}{1-\omega^2LC_1}\right)
=\frac{V_i}{R}
\]
Solving for $V_t$ we get
\[
V_t=\frac{1}{R(\frac{1}{R}+j\omega C_2+\frac{j\omega C_1}
{1-\omega^2LC_1})}\;V_i
=\frac{1}{1+j\omega R(C_2+C_1/(1-\omega^2LC_1)}\;V_i
\]
which is maximized if the frequency is such that the imaginary part of 
the denominator is zero:
\[
C_2+\frac{C_1}{1-\omega_0^2LC_1}=0,\;\;\;\;\;\;\;\mbox{i.e.}\;\;\;\;\;
\omega_0=\frac{1}{\sqrt{LC_1C_2/(C_1+C_2)}}=\frac{1}{\sqrt{LC_s}}
\]
This frequency $\omega_0$ is the resonant frequency, at which the voltage 
$V_t$ become the same as the source voltage $V_t=V_i$. Also note that the 
impedance of the tank circuit as the load of the Thevenin source is
\[
Z_{tank}=Z_{C_2}||(Z_{C_1}+Z_L)=\frac{(1/j\omega C_1+j\omega L)/j\omega C_2}
{1/j\omega C_1+1/j\omega C_2+j\omega L}
=\frac{(1/j\omega C_1+j\omega L)/C_2}{1/C_1+1/C_2-\omega^2 L}
=\frac{(1/j\omega C_1+j\omega L)/C_2}{1/C_s-\omega^2 L}
\]
At the resonant frequency, the denominator becomes zeros and $Z_{tank}=\infty$,
i.e., there is no current drawn from the source by the tank circuit. 
Consequently, the voltage drop across $R$ is zero and the voltage received 
by the tank circuit is $V_t=V_i$. Now the output voltage $V_o$ can be found 
by voltage divider:
\[
V_t=\frac{Z_{C_2}}{Z_{C_1}+Z_{C_2}}\;V_o=\frac{C_1}{C_1+C_2}\,V_o,
\;\;\;\;\;\mbox{i.e.}\;\;\;\;\;\;\;\;
V_o=\frac{C_1+C_2}{C_1}\;V_t=\frac{C_1+C_2}{C_1}\;V_i
\]
The open-loop gain (from $V_i$ to $V_o$) is:
\[
H=\frac{V_o}{V_i}=\frac{C_1+C_2}{C_1}
\]

%\begin{comment}
\begin{eqnarray}
  V_o&=&\frac{j\omega L}{j\omega L+1/j\omega C}\;V_t
  =\frac{\omega^2LC_1}{\omega^2LC_1-1}\;V_t
  \nonumber \\
  &=&\left(\frac{1-\omega^2LC_1}{1-\omega^2LC_1+j\omega R(C_1+C_2-\omega^2LC_1C_2)}\right)\;
  \left(\frac{\omega^2LC_1}{\omega^2LC_1-1}\right)\;V_i
  \nonumber \\
  &=&\frac{-\omega^2LC_1}{1-\omega^2LC_1+j\omega R(C_1+C_2-\omega^2LC_1C_2)}\;V_i
  \nonumber 
\end{eqnarray}

The open-loop gain (from $V_i$ to $V_o$) is:
\[
H=\frac{V_o}{V_i}=\frac{-\omega^2LC_1}{1-\omega^2LC_1+j\omega R(C_1+C_2-\omega^2LC_1C_2)}
\]
At the resonant frequency $\omega_0$, the imaginary part is zero, we have
\[
H=\frac{-\omega^2LC_1}{1-\omega^2LC_1}
=\frac{\omega_0^2LC_1}{\omega_0^2LC_1-1}
=\frac{LC_1/LC_s}{LC_1/LC_s-1}
=\frac{C_1}{C_1-C_s}
=\frac{C_1+C_2}{C_1}
\]
%\end{comment}

We see that at the resonant frequency $\omega_0$, the magnitude of the open-loop
gain is $|H|>1$ and its phase is $\angle H=0$ (zero phase, i.e., $H$ is real).
As both conditions for oscillation are satisfied, the circuit is an oscillator
with frequency at $\omega_0=1/\sqrt{LC_s}$.

%\begin{comment}
http://seit.unsw.adfa.edu.au/staff/sites/hrp/teaching/Electronics4/docs/PLL/colpitts.pdf

http://users.ece.gatech.edu/mleach/ece3050/notes/osc/wienbr.pdf

http://www.ece.msstate.edu/~winton/classes/ece3144/labs/Exp10.pdf

http://www.drp.fmph.uniba.sk/ESM/twin.pdf
%\end{comment}
\end{comment}


\subsection*{Single-Supply Op Amps and Circuits}

\begin{itemize}
\item \htmladdnormallink{Virtual ground}{http://tangentsoft.net/elec/vgrounds.html}
\item \htmladdnormallink{TI Application Bulletin}{../SingleSupplyOpAmp.pdf}
\item \htmladdnormallink{TI Single supply opamp design}{http://www.ti.com/lit/an/slyt189/slyt189.pdf}
\item \htmladdnormallink{SwarthmoreNotes}{http://www.swarthmore.edu/NatSci/echeeve1/Ref/SingleSupply/SingleSupply.html}
\end{itemize}


\begin{comment}

\subsubsection{Butterworth filters}

  The transition between the pass-band and stop-band of a first order 
  filter with cut-off frequency $\omega_c=1/\tau$ is characterized by the 
  the slope of 20 dB per decade of frequency change. To achieve better 
  selectivity, we can cascade a set of $n$ such first order filters to
  form an nth order filter with a slope of 20n dB per decade.

  \htmladdimg{../figures/Cascade1stOrder.png}

  The FRF of a first-order low-pass filter of unit gain is:  
  \[
  H(j\omega)=\frac{1}{j\omega\tau+1}
  =\frac{1}{j\omega/\omega_c+1}=\frac{\omega_c}{j\omega+\omega_c}
  \]
  The FRF of $n$ such filters in series is (assuming they are well
  buffered with no loading effect):
  \[
  H(j\omega)=\left(\frac{\omega_c}{\omega_c+j\omega} \right)^n
  \]
  The cut-off frequency of this nth order filter $\omega_{cn}$ can be
  found by solving the following equation
  \[
  \left|\frac{\omega_c}{\omega_c+j\omega}\right|^n
  =\left|\frac{\omega_c}{\sqrt{\omega_c^2+\omega^2}}\right|^n
  =\frac{1}{\sqrt{2}}=2^{-1/2}
  \]
  to get
  \[
  \omega=\omega_{cn}=\omega_c\sqrt{2^{1/n}-1}
  \]

  {\bf Example:} Design an 4th order LP filter with 
  $\omega_{c4}=2\pi 1000=6.2832\times 10^3$. The cut-off frequency of the 
  first-order LP filter can be found to be
  \[
  \omega_c=\frac{\omega_{c4}}{\sqrt{2^{1/4}-1}}
  =\frac{6.2832\times 10^3}{0.435}=1.445\times 10^4
  \]
  The time constant of the first-order filter is 
  $\tau=RC=1/\omega_c=6.92\times 10^{-5}$. If $C=0.1\;\mu F=10^{-7}\;F$, 
  then
  \[
  R=\frac{\omega_c}{C}=\frac{6.92\times 10^{-5}}{10^{-7}}
  =6.92\times 10^2=692\;\Omega
  \]

  The {\em Butterworth filters} have the property that the passing
  band is flat. The magnitude of the FRF of an nth order low-pass 
  Butterworth filter with cut-off frequency $\omega_c$ is
  \[
  \left| H_{lp}(j\omega)\right|=\frac{1}{\sqrt{1+(\omega/\omega_c)^{2n}}}
  =\left\{\begin{array}{ll}1 & \omega=0\\1/\sqrt{2}&\omega=\omega_c\\
  0 & \omega=\infty\end{array}\right.
  \]
  where $\omega_c$ is the cut-off frequency at which $|H_{lp}(j\omega_c)|
  =1/\sqrt{2}$.
  The transition between the pass-band and stop-band is controlled by the
  order $n$. In general, higher order corresponds to more rapid transition. 
  Specially, when $n=0$, $n=1$, and $n=\infty$, we have
  \begin{itemize}
  \item $n=0$, $|H_{lp}(j\omega)|=1$ is an all-pass filter
  \item $n=1$, the Butterworth filter is the regular first-order filter:
    \[
    |H_{lp}(j\omega)|=\frac{1}{\sqrt{1+(\omega/\omega_c)^2}}
    =\frac{\omega_c}{\sqrt{\omega^2+\omega_c^2}}
    =\left|\frac{\omega_c}{j\omega+\omega_c}\right|
    \]
  \item $n=\infty$, the Butterworth filter becomes an ideal low-pass
    filter:
    \[
    |H_{lp}(j\omega)|=\frac{1}{\sqrt{1+(\omega/\omega_c)^\infty}}  
    =\left\{\begin{array}{ll}1&\omega<\omega_c\\0&\omega>\omega_c\end{array}
    \right.
    \]
  \end{itemize}
  
  The magnitude of the FRF of an nth order high-pass Butterworth  filter 
  with cut-off frequency $\omega_c$ is
  \[
  \left| H_{hp}(j\omega)\right|=\frac{1}{\sqrt{1+(\omega_c/\omega)^{2n}}}
  =\frac{(\omega/\omega_c)^n}{\sqrt{1+(\omega/\omega_c)^{2n}}}
  =\left\{\begin{array}{ll}0 & \omega=1\\1/\sqrt{2}&\omega=\omega_c\\
  1 & \omega=\infty\end{array}\right.
  \]

  \htmladdimg{../figures/ButterworthFilters1.png}

  Now we consider the implementation of a low-pass Butterworth filter. We 
  need to first get its frequency response function (FRF) $H(j\omega)$, 
  and its transfer function $H(s)$ with $s=j\omega$, from its magnitude 
  $|H(j\omega)|$ given above. To do so, we find the poles of the transfer 
  function in the s-domain. For simplicity, in the following we assume 
  the frequency is normalized by the cut-off frequency $\omega_c$, i.e., 
  $\omega=\omega/\omega_c$, or $\omega_c=1$. Consider the low-pass case:
  \[
  |H(j\omega)|^2=\frac{1}{1+(\omega/\omega_c)^{2n}}
  =\frac{1}{1+\omega^{2n}}=\frac{1}{1+(\omega^2)^n}
  \]
  As $s=j\omega$, $s^2=(j\omega)^2=-\omega^2$, the FRF can be converted to
  the transfer function $H(s)$:
  \[
  |H(j\omega)|^2=H(j\omega) H(-j\omega)
  =\frac{1}{1+(\omega^2)^n}=\frac{1}{1+(-s^2)^n}
  =\frac{1}{1+(-1)^ns^{2n}}=|H(s)|^2=H(s) H(-s)
  \]
  We can now find all roots of the denominator, the poles of both $H(s)$ 
  and $H(-s)$, and separate them so that those on the left half of the 
  s-plane are the poles of $H(s)$ (stable and causal), while those on 
  the right s-plane belong to $H(-s)$ (stable and anti-causal).

  The roots of the denominator can be found be solving the equation
  \[
  1+(-1)^ns^{2n}=0,\;\;\;\;\;\;\;
  \left\{\begin{array}{ll}
  1+s^{2n}=0 & \mbox{$n$ is even}\\
  1-s^{2n}=0 & \mbox{$n$ is odd}\end{array}\right.
  \]
  The solution takes either of the two different forms depending on whether
  $n$ is even or odd, as illustrated below.

  \htmladdimg{../figures/ButterworthInSPlane.png}

  \begin{itemize}
  \item If $n$ is even, 
    \[
    s^{2n}=-1=e^{j(2k+1)\pi)},\;\;\;\;\;s_k=e^{j(2k+1)\pi/2n},\;\;\;\;\;(k=0,\cdots,2n-1)
    \]
    These $2n$ roots form $n$ complex conjugate pairs around the unit 
    circle of the s-plane. Corresponding to each of the $n$ roots 
    $s_k=e^{j(2 k+1)\pi/2n}$ ($k=0,\cdots,n-1$), there is another root
    $s_{2n-1-k}$ that is its complex conjugate:
    \[
    s_{2n-1-k}=e^{j(2(2n-1-k)+1)\pi/2n}=e^{j\pi/2n} e^{-j\pi/n} e^{-jk\pi/n}
    =e^{-j\pi/2n} e^{-jk\pi/n}=e^{-j(2 k+1)\pi/2n}=s_k^*
    \]
    Also, for $s_k=e^{j(2k+1)\pi/2n}$ to be a pole of $H(s)$, it needs to be on 
    the left s-plane, i.e.,
    \[
    \frac{(2k+1)\pi}{2n}>\frac{\pi}{2},\;\;\;\;\;\;\mbox{i.e.}\;\;\;\;\;\;
    k>\frac{n-1}{2}
    \]
    Now $H(s)$ can be found in terms of its $n$ poles:
    \begin{eqnarray}
    H(s)&=&\frac{1}{1+s^{2n}}
    =\frac{1}{\prod_{k=\left\lceil (n-1)/2 \right\rceil}^{n-1} (s-s_k)(s-s_{2n-1-k})}
    \nonumber \\
    &=&\frac{1}{\prod_{k=\left\lceil (n-1)/2 \right\rceil}^{n-1} (s-s_k)(s-s_k^*)}
    =\frac{1}{\prod_{k=\left\lceil (n-1)/2 \right\rceil}^{n-1} (s^2-(s_k+s_k^*)s+s_ks_k^*)}
    \nonumber \\
    &=&\frac{1}{\prod_{k=\left\lceil (n-1)/2 \right\rceil}^{n-1} (s^2-2\cos((2 k+1)\pi/2n)+1)}
    \nonumber 
    \end{eqnarray}
    where $\left\lceil x\right\rceil$ is the ceiling of $x$, and we have 
    used the fact that    
    \[
    s_k+s_k^*=e^{j(2 k+1)\pi/2n}+e^{-j(2 k+1)\pi/2n} =2\cos((2 k+1)\pi/2n),
    \;\;\;\;\;\;
    s_k s_k^*=1
    \]

  \item If $n$ is odd, 
    \[
    s^{2n}=1=e^{j2k\pi},\;\;\;\;\;s_k=e^{j2k\pi/2n}=e^{jk\pi/n},\;\;\;\;\;(k=0,\cdots,2n-1)
    \]
    These $2n$ roots contain $s_0=1$ and $s_n=-1$, as well as $n-1$ complex 
    conjugate pairs around the unit circle of the s-plane. Corresponding to 
    each root $s_k=e^{jk\pi/n}$ ($k=1,\cdots,n-1$), there is another root $s_{2n-k}$
    that is its complex conjugate:
    \[
    s_{2n-k}=e^{j(2n-k)\pi/n}=e^{-jk\pi/n}=s_k^*
    \]
    For $s_k$ to be a pole of $H(s)$, it needs to be on the left s-plane, i.e.,
    \[
    \frac{k\pi}{n}>\frac{\pi}{2},\;\;\;\;\;\;\mbox{i.e.}\;\;\;\;\;\;k>\frac{n}{2}
    \]
    Now $H(s)$ can be found in terms of its $n$ poles:
    \begin{eqnarray}
    H(s)&=&\frac{1}{1-s^{2n}}
    =\frac{1}{(s-s_n)\prod_{i=\left\lceil n/2 \right\rceil}^{n-1}(s-s_k)(s-s_{2n-k})}
    \nonumber \\
    &=&\frac{1}{(s+1)\prod_{i=\left\lceil n/2 \right\rceil}^{n-1}(s-s_k)(s-s_k^*)}
    =\frac{1}{(s+1)\prod_{i=\left\lceil n/2 \right\rceil}^{n-1}(s^2-(s_k+s_k^*)s+s_k s_k^*)}
    \nonumber \\
    &=&\frac{1}{(s+1)\prod_{i=\left\lceil n/2 \right\rceil}^{n-1}(s^2-2\cos(k\pi/n)+1)}
    \nonumber
    \end{eqnarray}

  \end{itemize}

  Specifically, here we find the transfer function $H(s)$ of the nth order
  Butterworth filter for $n=2,\cdots,6$:
  \begin{itemize}
  \item $n=2$, $1+s^4=0$, $s^4=-1=e^{j\pi}$, the four roots are
    $s_k=e^{j(\pi+k*2\pi)/4}$ ($k=0,\cdots,3$):
    \[
    s_0=e^{j\pi/4}=\frac{1+j}{\sqrt{2}},\;\;\;\;
    s_1=e^{j3\pi/4}=\frac{-1+j}{\sqrt{2}},\;\;\;\;
    s_2=e^{j5\pi/4}=\frac{-1-j}{\sqrt{2}},\;\;\;\;
    s_3=e^{j7\pi/4}=\frac{1-j}{\sqrt{2}}
    \]
    out of which $s_1$ and $s_2=s_1^*$ on the left-hand side of the s-plane are 
    the roots of $H(s)$:
    \begin{eqnarray}
    H(s)&=&\frac{1}{(s-s_1)(s-s_2)}=\frac{1}{(s-(-1+j)/\sqrt{2})(s-(-1-j)/\sqrt{2})}
    \nonumber \\
    &=&\frac{1}{s^2+\sqrt{2}s+1}
    \nonumber
    \end{eqnarray}
    Note that the coefficient of the first order term is $-2\cos(3\pi/2)=\sqrt{2}$.
  \item $n=3$, $1-s^6=0$, $s^6=1=e^{j2\pi}$, the six roots are
    $s_k=e^{j k2\pi/6}=e^{j k\pi/3}$ ($k=0,\cdots,5$)
    \[
    s_0=e^{j0}=1,\;\;\;\;
    s_1=e^{j\pi/3}=\frac{1+j\sqrt{3}}{2},\;\;\;\;s_2=e^{j2\pi/3}=\frac{-1+j\sqrt{3}}{2}
    \]
    \[
    s_3=e^{j3\pi/3}=e^{j\pi}=-1,\;\;\;\;
    s_4=e^{j4\pi/3}=\frac{-1-j\sqrt{3}}{2},\;\;\;\;s_5=e^{j5\pi/3}=\frac{1-j\sqrt{3}}{2}
    \]
    out of which $s_2$, $s_3=-1$, and $s_4=s_2^*$ on the left-hand side of s-plane 
    are the roots of $H(s)$:
    \begin{eqnarray}
    H(s)&=&\frac{1}{(s-s_2)(s-s_3)(s-s_4)}
    =\frac{1}{(s+1)(s-(-1+j\sqrt{3})/2)(s-(-1-j\sqrt{3})/2)}
    \nonumber \\
    &=&\frac{1}{(s+1)(s^2+s+1)}
    \nonumber
    \end{eqnarray}
    Note that the coefficient of the first order term is $-2\cos(2\pi/3)=1$.
  \item $n=4$, $1+s^8=0$, $s^8=-1=e^{j\pi}$, the eight roots are
    $s_k=e^{j(\pi+k2\pi)/8}$ ($k=0,\cdots,7$).
    Evaluating $-2\cos((2k+1)\pi/2n)$ for $k=2$ and $k=3$, we get the
    coefficients of the two first order terms
    $-2\cos(5\pi/8)=0.7654$ and $-2\cos(7\pi/8)=1.8478$.
    \[
    H(s)=\frac{1}{(s^2+0.765 s+1)(s^2+1.848 s+1)}
    \]
  \item $n=5$, $1-s^{10}=0$, $s^{10}=1=e^{j2\pi}$, the 10 roots are
    \[
    s_k=e^{j(k2\pi)/10}=e^{j(k\pi)/5},\;\;\;\;\;\;(k=0,\cdots,9)
    \]
    Evaluating $-2\cos(k\pi/n)$ for $k=2$ and $k=3$, we get the
    coefficients of the two first order terms
    $-2\cos(2\pi/5)=0.618$ and $-2\cos(3\pi/5)=1.618$, and we get
    \[
    H(s)=\frac{1}{(s+1)(s^2+0.618 s+1)(s^2+1.618 s+1)}
    \]
  \item $n=6$, $1+s^{12}=0$, $s^{12}=-1=e^{j\pi}$, the 12 roots are
    \[
    s_k=e^{j(\pi+k2\pi)/12},\;\;\;\;\;\;(k=0,\cdots,11)
    \]
    Evaluating $-2\cos((2k+1)\pi/2n)$ for $k=3$, $k=4$, and $k=5$,
    we get the coefficients of the three first order terms
    $-2\cos(7\pi/12)=0.5176$, $-2\cos(9\pi/12)=1.4142$, and
    $-2\cos(9\pi/12)=1.319$.
    \[
    H(s)=\frac{1}{(s^2+0.5176 s+1)(s^2+1.4142 s+1)(s^2+1.9319 s+1)}
    \]
  \end{itemize}
  In summary, we see that a Butterworth filter can be implemented as
  a cascade of second order systems in the form of $1/(s^2+a\,s+1)$ 
  if $n$ is even, and an additional first order system in the form of
  $1/(s+1)$ if $n$ is odd. The block diagrams below are for the 5th 
  and 6th order Butterworth filters:

  \htmladdimg{../figures/ButterworthDiagram.png}

  The first order filter in the cascade of the Butterworth filter 
  can be realized by the first order op-amp low-pass circuit shown
  above with
  \[
  H(s)=\frac{1/\tau}{s+1/\tau}=\frac{\omega_c}{s+\omega_c}=\frac{1}{s+1}
  \]
  where $\omega_c=1/\tau=1/RC$. If we let $R=1$, we get $C=1/\omega_c=1$. 

  The second order systems in the cascade can be implemented as a 
  Sallen-Key low-pass filter with 
  \[
  H(s)=\frac{1/R_1C_1R_2C_2}{s^2+s(R_1+R_2)/R_1R_2C_1+1/R_1C_1R_2C_2}
  =\frac{1}{s^2+\Delta\omega s+\omega_n^2}=\frac{1}{s^2+a s+1}
  \]
  where $\omega_n^2=1/R_1R_2C_1C_2$. If we let $R_1=R_2=1$ for simplicity, 
  we get
  \[
  2/C_1=a,\;\;\;\;\;1/C_1C_2=\omega_n^2=1
  \]
  Solving these we get
  \[
  C_1=2/a,\;\;\;\;\; C_2=1/C_1=a/2
  \]

  A High-pass Butterworth filter can be similarly implemented with the only
  difference that all first and second order systems in the cascade are
  high-pass filters
  \[
  H(s)=\frac{s}{s+\omega_c}=\frac{s}{s+1},\;\;\;\;\;\;\;
  H(s)=\frac{s^2}{s^2+\Delta\omega s+\omega_c^2}=\frac{s^2}{s^2+a s+1}
  \]
  so that the transfer function of the cascade is high-pass filter:
  \[
  H(s)=\left\{\begin{array}{cc}s^n/(1+s^{2n}) & \mbox{$n$ is even}\\
  s^n/(1-s^{2n}) & \mbox{$n$ is odd}\end{array}\right.
  \]

  To convert the results obtained above for normalized cut-off frequency 
  $\omega_n=1$ to unnormalized cut-off frequency $\omega_n\ne 1$, all we 
  need to do is to scale all capacitances $C$ to $C'=C/\omega_c$. The 
  capacitor in the first order filter becomes $C'=C/\omega_c$ so that
  $1/RC'=\omega_c/C'=\omega_c$; while the two capacitors in the second order
  filter become $C_1'=C_1/\omega_n$ and $C_2'=C_2/\omega_n$ so that
  $1/C_1'C_2'=\omega_n^2/C_1C_2=\omega_n^2$.

\subsubsection{Higher order systems}

Higher than first order systems can be built with multiple integrators, 
as shown here for a third order system:

\htmladdimg{../figures/opam7.gif}

From the diagram, we can get
\[
\left\{ \begin{array}{l}
  Y_3(s)=Y_2(s)/s \Longrightarrow Y_2(s)=Y_3(s)s	\\
  Y_2(s)=Y_1(s)/s \Longrightarrow Y_1(s)=Y_2(s)s=Y_3(s)s^2	\\
  Y_1(s)=Y_0(s)/s \Longrightarrow Y_0(s)=Y_1(s)s=Y_3(s)s^3	
\end{array} \right.
\]
But we also have
\[
Y_0(s)=X(s)-(k_1Y_1(s)+k_2Y_2(s)+k_3Y_3(s))	
\]
i.e., 
\[
X(s)=Y_0(s)+k_1Y_1(s)+k_2Y_2(s)+k_3Y_3(s)=(s^3+k_1s^2+k_2s+k_3) Y_3(s)	
\]
we get the transfer function
\[
H(s)=\frac{Y_3(s)}{X(s)}=\frac{1}{s^3+k_1s^2+k_2s+k_3}
\]


{\bf Second order system by 2 integrators}

\htmladdimg{../figures/opam8.gif}

From the diagram, we can get
\[
\left\{ \begin{array}{ll}
  Y_2(s)=-c_2Y_1(s)/s  \Longrightarrow  Y_1(s)=-sY_2(s)/c_2 \\
  Y_1(s)=-c_1Y_0(s)/s  \Longrightarrow Y_0(s)=-sY_1(s)/c_1=s^2Y_2(s)/c_1c_2 \\
  Y_0(s)=k_0 X(s)+k_1Y_1(s)+k_2Y_2(s) 
\end{array} \right.
\]
substituting the first two equations into the last one, we get
\[
\frac{s^2}{c_1c_2} Y_2(s)=k_0X(s)+k_1(-\frac{s}{c_2})Y_2(s)+k_2Y_2(s) 
\]
from which we obtain the transfer function as
\[
H(s)=\frac{Y_2(s)}{X(s)}=\frac{k_o}{\frac{s^2}{c_1c_2}+\frac{s}{c_2}s-k_2}
=\frac{k_oc_1c_2}{s^2+k_1c_1s-c_1c_2k_2}
\]
which is a second order system. In particular, if $c_1=c_2=c$, we have
\[
H(s)=k_0\frac{c^2}{s^2+c k_1s-k_2c^2}
\]
Comparing this with the canonical 2nd order system transfer function
\[
H(s)=\frac{\omega_n^2}{s^2+2\zeta \omega_n s+\omega_n^2}
\]
we see that we can let $c=\omega_n$ and $k_1=2\zeta$. Moreover, $k_2<0$, 
i.e., the feedback from the output should be negative. $k_0$ is a constant
scalar which can take any value.
	
%\htmladdimg{../figure/opam9.gif}


\end{itemize}
\end{comment}

\end{document}


	

	












