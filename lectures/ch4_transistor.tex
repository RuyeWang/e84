\documentstyle[12pt]{article}
\usepackage{html}
\textwidth 6.0in
\topmargin -0.5in
\oddsidemargin -0in
\evensidemargin -0.5in
% \usepackage{graphics}  
\begin{document}

\section*{Chapter 4: Semiconductor Devices}

\subsection*{Semiconductor materials}

\begin{itemize}
\item {\bf Conductors and Insulators:} 
Good conductors such as copper and silver can conduct electricity with
little resistance because their crystal structure allows a loosely
bound valence electron per atom to move freely throughout the lattice. 
Insulators do not conduct electric current as no free electrons exist
in the material.

\item {\bf Semiconductors:} 
The two semiconductors of great importance are silicon (Si 14) and 
germanium (Ge 32), which both have four valence electrons. In crystal
structure (lattice)  is a tetrahedral pattern with each atom sharing one 
valence electron with each of four neighbors (covalent bonds). 

If an electron gains enough thermal energy (1.1 eV for Si or 0.7 eV for
Ge), it may break the covalent bond and becomes a free electron of negative
charge, while leaving a vacancy or a hole of positive charge. In an electric
field, a free electron may move to a new location to fill a hole there, i.e.,
both such electrons and holes contribute to electrical conduction. Such 
crystal is called intrinsic semiconductor.

At room temperature, relatively few electrons gain enough energy to become
free electrons, the over all conductivity of such materials is low, thereby
their name semiconductors. 

\htmladdimg{../figures/elements.gif}

\item {\bf Doped Semiconductors:}
The conductivity of semiconductor material can be improved by doping, i.e.,
by adding an impurity element with either three or five valence electrons,
called, respectively, trivalent and pentavalent elements. 

\begin{itemize}
\item {\bf n-type semiconductor:}
When a small amount of pentavalent element is added, a silicon atom in the
lattice may be replaced by a pentavalent atom with four of its valent electrons
forming the covalent bounds and one extra free electron. This is an {\bf n-type}
semiconductor whose conductivity is much improved compared to the intrinsic
semiconductors, due to the extra free electrons in the lattice, which are called
{\bf predominant or majority current carriers}. There also exist some tiny number 
of holes called {\bf minority carriers}.

\item {\bf p-type semiconductor:}
When a small amount of trivalent element is added, a silicon atom in the
lattice may be replaced by a trivalent atom with only three valent electrons
forming three covalent bounds and a hole in the lattice. This is a {\bf p-type}
semiconductor whose conductivity is also much improved compared to the intrinsic
semiconductors, due to the holes in the lattice, which are called {\bf predominant 
or majority current carriers}. There also exist some tiny number of free electrons
called {\bf minority carriers}.
\end{itemize}

\htmladdimg{../figures/lattice.gif}

\item {\bf pn Junction}

\htmladdimg{../figures/pn_junction0.gif}

When p-type and n-type materials in contact with each other, a p-n junction is
formed due to two effects:
\begin{itemize}
\item {\bf Diffusion:}
Although both sides are electrically neutral, but they have different
concentration of electrons (the n-type) and holes (the p-type), and the 
free electrons in the n-type material begin to diffuse across the p-n junction
between the two materials, due to their thermal motion, and to fill some of the
holes in the p-type material. Equivalently, the holes are also drifting from
the p-type side to the n-type side.

\item {\bf Electric Field}
If no other forces were involved, the diffusion would carry out 
continuously until the free electrons and holes are uniformly distributed
across both materials. However, as the result of the diffusion process,
electrical field is gradually established, negative on the side of p-type 
material due to the extra electrons, positive on the side of n-type 
material due to the loss of free electrons. This electrical field prevents
further diffusion as the electrons on the n-type side are expelled from
the p-type side by the electrical field.

\end{itemize}

The effects of both diffusion and electric field eventually lead to an 
equilibrium where the two effects balance each other so that there are
no more charge carriers (free electrons or holes) crossing the p-n junction.
This region around the p-n junction, called the {\bf depletion region} as 
there no longer exist freely movable charge carriers, becomes a barrier 
between the two ends of the material that prevent current to flow through.

\htmladdimg{../figures/pn_junction1.gif}

\end{itemize}

\subsection*{Diodes}

Due to the fact that there exist few freely movable charge carriers in the
depletion region around the p-n junction, the conductivity is very poor.
However, if certain voltage is applied to the two ends of the material, 
the conductivity may change, depending one the polarity of the applied
voltage:

\htmladdimg{../figures/diode0.gif}

\begin{itemize}
\item {\bf Reverse bias} (negative to p-type, positive to n-type)

  The negative voltage applied to the p-type will repel electrons in n-type
  and attract holes in p-type so that both carriers are moving away from 
  the p-n junction. As the depletion region becomes thicker than before, 
  there is no current through the p-n junction and the conductivity is zero.

\item {\bf Forward bias} (positive to p-type, negative to n-type)

  The positive voltage applied to the p-type will attract electrons in n-type
  and repel holes in p-type so that both carriers are moving towards the p-n
  junction. As the depletion region becomes thinner, the conductivity is 
  improved and there is current through the p-n junction. The conductivity
  increases as the voltage becomes higher.
  
\end{itemize}
The voltage-current behavior of a p-n junction is described by
\[ I_D=I_0 ( e^{V_D/\eta V_T}-1 ), \;\;\;\;\mbox{or}\;\;\;\;
	V_D=\eta V_T\;ln (\frac{I_D}{I_0}+1)=\eta V_T\;ln (\frac{I_D+I_0}{I_0})	\]

where 
\begin{itemize}
\item $I_0$ is the {\em reverse saturation current}, a tiny current that 
  flows in the reverse direction when $V_D \ll 0$, due to the minority 
  carriers. $I_0$  is about $10^{-10} \sim 10^{-12}$ A for Si and $10^{-4}$
  A for Ge.
\item $ V_T=kT/e $ is the voltage equivalent temperature, where 
  $k=1.38\times 10^{-23}$ Joules/Kelvin is Boltzmann's constant, 
  $e=1.602\times 10^{-19}$ coulomb is the charge of an electron, and
  $T$ is the temperature in degree K. For room temperature $T=300K$, 
  $V_T=26\; mV$.
\item $\eta$ is the ideality factor which is 1 for Ge and 1.4 for Si
\end{itemize}
In particular, when $V_D=0$, $I_D=0$, when $V_D\ll 0$, $I_D=I_0$, when
$V_D\gg 0$, $I_D=I_0 e^{V_D/V_T}$.

\htmladdimg{../figures/diode1.gif}

The resistance of an electrical device is defined as $r=\Delta V/\Delta I$.
For a diode, as $V_D(I_D)$ is not a linear function, the resistance 
$R_0=dV_D/dI_D$ is not a constant, but a function of $I_D$:
\[
 R_0=\frac{d}{dI_D}V_D=\frac{d}{dI_D} [\eta V_T\;ln (\frac{I_D+I_0}{I_0})]
=\eta V_T \frac{I_0}{I_D+I_0}\frac{1}{I_0}=\eta \; \frac{V_T}{I_D+I_0}	\]
As $I_D \gg I_0$, i.e., $I_D+I_0\approx I_D$, we have
\[	 R_0=\frac{dV_D}{dI_D}=\eta\; \frac{V_T}{I_D}	\]
If we let $V_T=26\;mV$, $\eta=1$, and $I_D=2\;mA$, then we have 
$R_0=13\;\Omega$.

{\bf Models of diodes:}

\htmladdimg{../figures/diode3.gif}

\begin{itemize}
\item Ideal model: if $V_D<0$, then $I_D=0$, else $I_D=V/R$
\item Diode with a voltage threshold $V_0=0.7V$:
	if $V_D<V_0$, then $I_D=0$, else $I_D=(V-V_0)/R$
\item Diode with a voltage threshold $V_0=0.7V$ and a resistance $R_0=20\Omega$
	if $V_D<V_0$, then $I_D=0$, else $I_D=(V-V_0)/(R+R_0)$
\item A current source can be added to simulate the reverse saturation current.
\end{itemize}

\htmladdimg{../figures/diodemodel.gif}

\begin{tabular}{c||c|c|c}\\ \hline
 $I_0$	& 1 mA & 10 mA & 100 mA	\\ \hline
$V_D$ for Si ($I_0=10^{-10}$, $\eta=1.4$) & 0.58 V & 0.67 V & 0.75 V \\
$V_D$ for Ge ($I_0=10^{-4}$, $\eta=1.0$) & 0.06 V & 0.12 V & 0.18 V \\
\end{tabular}

In general, when the forward voltage applied to a diode exceeds 0.7V (or 
0.3V) for silicon (or germanium) material, the diode is assumed to be 
conducting with very little resistance.

{\bf Example 1: } In the half-wave rectifier circuit shown below, 
$R=1000\Omega$, $V=3V$, and $D$ is a silicon diode. Find the current
$I_D$ through and voltage $V_D$ across $D$.

\htmladdimg{../figures/diode2.gif}

\begin{itemize}
\item {\bf Method 1: } Since the diode is forward biased, we can assume 
the voltage across the diode is $0.7V$ and the current can be determined 
by Ohm's law to be $I_D=(V-0.7)/R=2.3/1000=2.3mA$.

\item {\bf Method 2: } Applying KVL to the loop, we get a transcendental 
equation $V=V_D+I_DR=V_D+RI_0(e^{V_D/V_T}-1)$ which can be solved for
$V_D$ numerically. 

\item {\bf Method 3: } The current $I_D$ and voltage $V_D$ have to satisfy 
two equations simultaneously. First, we have the same equation above 
relating the current $I_D$ through and voltage $V_D$ across diode
$I_D=I_0(e^{V_D/V_T}-1)$. Second, according Ohm's law applied to $R$, we
have $V=V_D+I_DR$. These two simultaneous equations can be solved 
graphically to the two unknowns $I_D=2.4mA$ and $V_D=0.75V$.
\end{itemize}

{\bf Example 2: } Design a converter (adaptor) that converts AC power 
supply of 115V and 60 Hz to a DC voltage source of 14 V. When the load is
$10\Omega$, the variation (ripple) of the output DC voltage must be 5\% or
less.

\htmladdimg{../figures/halfwaverectifier.gif}

\htmladdimg{../figures/diode4.gif}

\begin{itemize}
\item The peak of the secondary output is $14V$ with RMS value
$14/\sqrt{2}=10V$, the ratio of the transformer should be 115:10. 
\item When the load is $R_L=10\Omega$, the load current is $I=V/R_L=14/10=1.4A$.
\item During the period between two peaks $T=1/f=1/60=16.7\;ms$, the charge 
	on the capacitor is reduced by $\Delta Q=IT=1.4A\times 16.7\;ms
	=23.4\;mC$. 	
\item The voltage across the capacitor is therefore dropped by
	$\Delta V=\Delta Q/C < 14\times 5/\%=0.7V $. 
\item Solve above equation for $C$, we get $C=\Delta Q/\Delta V=23.4\;mC/0.7V
	=33,400\;\mu F$.
\end{itemize}
This is an approximation based on the assumption that the load current is
constant, as the voltage drop is small. Otherwise the exponential decay of
the voltage across capacitor should be used, and the current is:
\[	i(t)=\frac{V}{R_L} e^{-t/\tau}	\]

\subsection*{Bipolar Junction Transistor (BJT)}

A Bipolar Junction Transistor (BJT) has three terminals connected to three
doped semiconductor regions. In an npn transistor, a thin and lightly doped 
p-type material is sandwiched between two thicker n-type materials; while 
in a pnp transistor, a thin and lightly doped n-type material is sandwiched 
between two thicker p-type materials. In the following we will only consider
npn BJTs.

\htmladdimg{../figures/transistors1.gif}

\htmladdimg{../figures/transistorBJT1.gif}

In many schematics of transistor circuits (especially when there exist a
large number of transistors in the circuit), the circle in the symbol of
a transistor is omitted.

\htmladdimg{../figures/transistorBJT2a.gif}

\htmladdimg{../figures/transistorBJT2b.gif}

The three terminals of a transistor are typically used as the input, output 
and the common terminal of both input and output. Depending on which of the
three terminals is used as common terminal, there are three different 
configurations: common emitter (CE), common base (CB) and common collector 
(CC). The common emitter (CE) is the most typical configuration:

\htmladdimg{../figures/transistors2.gif}

\begin{itemize}
\item {\bf Common-Base (CB)}

In the {\em normal operation}, the EB junction is forward biased while the 
CB junction is reverse biased.

\htmladdimg{../figures/CB.gif}

%\htmladdimg{../figures/transistorDC.gif}

\htmladdimg{../figures/CBnpn.gif}

%\htmladdimg{../figures/transistorDC1.gif}

The behavior of the npn-transistor is determined by its two pn-junctions:
\begin{itemize}
\item {\bf The emitter-base (EB) junction:}

The forward biased EB junction allows a current $I_E=I_{EN}+I_{EP}$ to flow 
through, where 
\begin{itemize}
\item $I_{EN}=\gamma I_E$ is due to electrons from E to B, 
\item $I_{EP}=(1-\gamma) I_E$ is due to the holes from B to E. 
\end{itemize}
As p-type base is thin and lightly doped, the following result:
\begin{itemize}
\item The fraction $\gamma$ is very close to unity $\gamma \approx 1$, i.e.,
\[	I_E=I_{EN}+I_{EP}\approx I_{EN}	\]
\item Most of the electrons from emitter $\alpha I_E$ ($\alpha \approx 0.99$) 
go through base to reach the CB junction, with only a small number of the 
electrons $(\gamma-\alpha)I_E$ combined with the hole in base.
\[	I_{CN}=\alpha I_{EN}	\]
\end{itemize}

\item {\bf The base-collector (BC) junction:}

The reverse biased CB junction blocks the majority carriers (holes in
p-type base, electrons in n-type collector), but lets through the 
minority carriers, the electrons in base and holes in collector, including
\begin{itemize}
\item most of the electrons from emitter $I_{CN}=\alpha I_E$,
\item the reverse saturate current of the CB junction $I_{CP}=I_{CB0}$,
\end{itemize}
These two currents form the overall collector current
\[	I_C=I_{CN}+I_{CP}=\alpha I_E+I_{CB0}	\]

Combining the three equations above (and assuming $\gamma=1$), we get:
\[	I_C\approx \alpha I_E + I_{CB0},\;\;\;\;\mbox{and}\;\;\;\;
	I_B=I_E-I_C=I_{EN}-I_{CN}-I_{CB0}	\]
The base current $I_B$ is the small difference between two nearly equal
currents.
\end{itemize}

\item {\bf Common-Emitter}

\htmladdimg{../figures/CE.gif}

\htmladdimg{../figures/CEnpn.gif}

The input current is $I_B$, $I_E=I_B+I_C$, and the output current is 
\[ I_C=\alpha I_E+I_{CB0}=\alpha (I_C+I_B) + I_{CB0}	\]
Solving for $I_C$, we get
\[ I_C=\frac{\alpha}{1-\alpha} I_B+\frac{1}{1-\alpha} I_{CB0}
	=\beta I_B + (\beta+1) I_{CB0}=\beta I_B + I_{CE0}	\]
Here $\beta\stackrel{\triangle}{=}\alpha/(1-\alpha)$ is the 
{\bf current-transfer ratio} for CE (typically, $\alpha=0.99$ and $\beta=99$),
and $I_{CE0}=(\beta+1) I_{CB0}$ is the reverse saturation current between
collector and emitter.

\end{itemize}


\subsection*{Field-Effect Transistors (FET)}

\begin{itemize}

\item {\bf JFET}

A junction-gate field-effect transistor (JFET) has three terminals, the 
source, gate and drain (corresponding to the emitter, base and collector 
of a BJT). In a JFET, the source and drain is connected by a conducting 
channel (n- or p-doped), which is insulated by reverse biased pn junctions 
(the p- or n-type substrate and gate) . The current flowing through the
channel is due to the voltage $V_{DS}$ applied across drain and source, 
and controlled by the voltage $V_{GS}$ applied to the gate. In the following,
we only consider n-channel FETs.

\htmladdimg{../figures/JFET.gif}

The drain-source current $I_D$ is affected by two voltages $V_{GS}$
and $V_{GS}$. 
\begin{itemize}
\item When $V_{GS}$ is fixed (e.g., $V_{GS}=0$) and $V_{DS}$ is small
	and positive ($V_D>V_S$), the conducting channel behaves like a
	resistor and $I_D$ increases linearly with $V_{DS}$.
\item As $V_{DS}$ continues to increase, the pn-junction close to the 
	drain is highly reverse biased and the depletion region widens
	and the channel narrows, until eventually the channel is pinched 
	off at $V_{GS}=V_P$ and $I_D$ no longer increases with $V_{DS}$ 
	to reach saturation.
\item After pinch-off point, $I_D$ is controlled by $V_{GS}$, which
	determines how much the pn-junction between the gate and the 
	channel is reverse biased and thereby the width of the channel.
\end{itemize}

This is the transfer characteristics:
\htmladdimg{../figures/JFETplot1.gif}

This is the output characteristics:
\htmladdimg{../figures/JFETplot2.gif}

As the pn-junction between G and S has to be reverse biased all the time, 
i.e., $V_{GS}<0$. 

\item {\bf MOSFET}

In the metal-oxide-semiconductor FEAT (MOSFET), also called insulated gate
FET (IGFET), the gate and the conducting channel is insulated by a thin 
layer of $SiO_2$ instead of a reverse biased pn-junction as in JFET. Both
the source S and drain D are n-type and the substrate between them is 
p-type. Due to this insulation, the voltage applied to gate can be either 
positive or negative (while it has to be negative to keep the pn-junction
reverse biased for an n-channel JFET). 

The current $I_D$ is controlled by input voltage $V_G$. 
\begin{itemize}
\item When input voltage $V_G$ is smaller than a threshold $V_G<V_T$, no 
	current flows through S and D, due to the two pn-junctions.
\item When $V_G>V_T$, the holes in p-type substrate are repelled away from
	the gate and some electrons are pulled close to the gate. These
	electrons (minority carriers in the p-type substrate) form a 
	{\em inversion layer} close to the gate, which becomes an n-type 
	channel between S and D so that the conductivity between them is 
	enhanced. 
\item When $I_{DS}$ is small, the n-channel behaves like a resistor and 
	$I_{DS}$ increases proportionally to $V_{DS}$. But when $I_{DS}$ 
	further increases, it cancels the effect of the positive gate 
	voltage $V_G$ at the drain end, and the n-channel narrows, $I_{DS}$
	no longer increase with $V_{DS}$.
\end{itemize}

\htmladdimg{../figures/MOSFET1.gif}

{\bf Depletion MOSFET}
\htmladdimg{../figures/MOSFET2.gif}

{\bf Enhancement MOSFET}

There exist two types of MOSFETs, depletion and enhancement, depending on
where the pinch voltage $V_T$ is, as shown in the figure below:

\htmladdimg{../figures/MOSFETplot1.gif}

\end{itemize}

{\bf The Transfer Characteristics:}

The drain current $I_D$ in the constant-current region expressed as a 
function of input voltage $V_{GS}$. For JFET the function is:
\[ I_D=I_{DSS}(1-V_{GS}/V_P)^2	\]
where $I_{DSS}$ is the value of $I_D$ with gate shorted with source,
i.e., $V_{GS}=0$.
and for enhancement MOSFET, it is:
\[ I_D=K(V_{GS}-V_T)^2	\]

{\bf Small Signal Model:}

Similar to BJT, an FET can also represented by a small-signal model. As
$i_G=0$, the model for FET is simpler than BJT. The model for common-source
configuration is
\[	i_g=0	\]
and
\[	i_d=f(v_{gs}, v_{ds})=\frac{\partial i_d}{\partial v_{gs}} v_{gs}
	+\frac{\partial i_d}{\partial v_{ds}} v_{ds}
	=g_m v_{gs}+v_{ds}/r_{ds}
\]
where $g_m=\partial i_d/\partial v_{gs}$ is the transfer admittance
representing how input voltage $v_{gs}$ effects the current $i_{ds}$,
$r_{ds}=\partial v_{ds}/\partial i_d$ is the output resistance (reciprocal 
of the slope of the i-v curve in the output characteristics), representing
how $v_{ds}$ effects $I_D$. As $r_{ds}$ is large, it can be omitted to 
simplify the model to 
\[	i_{ds}=g_m v_{gs}	\]

\htmladdimg{../figures/smallsignalmodelFET.gif}


{\bf Biasing}

Similar to BJT, FET can also used as common-source, common-gate or
common-drain circuit. The common-source configuration is most typical
and there are two different biasing methods to set up the DC operating
point:
\begin{itemize}
\item {\bf Self-Biasing}

	\htmladdimg{../figures/JFETbiasing1.gif}
\[	V_{GS}=-I_D R	\]
\item {\bf Voltage Divider Biasing}

	\htmladdimg{../figures/JFETbiasing2.gif}
\[	V_{GS}=\frac{R_{g2}}{R_{g1}+R_{g2}} \;V_{DD}-I_D R \]
\end{itemize}
Also, common to both biasing methods, we have
\[	V_{DS}=V_{DD}-I_D(R+R_d)	\]
and from the transfer characteristics:
\[	I_D=I_{DSS}(1-V_{GS}/V_P)^2	\]
These three equations can be solved for $V_{GS}$, $V_D$ and $V_{DS}$
for the DC operating point. 


{\bf Comparison between BJT and FET}

\begin{itemize}
\item The input resistance $r_{in}$ is low for BJT but high for FET.
	The gate pn-junction of a JFET is always reverse biased with
	$r_{in}>10^8 \Omega$, the gate of a MOSFET is totally insulated
	from the channel with $r_{in}>10^{11} \Omega$.
\item FET is voltage ($V_{GS}$) controlled, while BJT is current ($I_b$) 
	controlled, consequently, the power consumption of FETs is lower
	than BJTs.
\item Both majority and minority carriers are used in BJTs, but only 
	majority carriers are used in FETs. Consequently FETs have better
	temperature stability (minority carriers are more sensitive to
	temperature).
\item FETs are easy to fabricate in large scale and have higher element
	density than BJTs.
\item MOSFETs have thin insulation layer which is more prone to statics
	and requires special protection. 
\item BJTs have higher cutoff frequency and higher maximum current than 
	FETs.
\item FETs are much more widely used (especially in computers and digital 
	systems) than BJTs.
\end{itemize}

\subsection*{Input/Output Characteristics and AC Behavior} 

\begin{itemize}
\item {\bf Common-Base:}

\begin{itemize}
\item {\bf Input characteristics:} 

The EB junction is essentially the same as a forward biased diode, as the 
effect of collector-base voltage $V_{CB}$ is small. The current-voltage 
characteristics is essentially the same as that of a diode:
\[ I_E=I_0 ( e^{V_{BE}/V_T}-1 )	\]

\item {\bf Output characteristics:} 

The CB junction is reverse biased, the current $I_C$ depends on the current
$I_E$. When $I_E=0$, $I_C=I_{CB0}$, the current caused by the minority 
carriers crossing the pn-junction. This is similar to the diode current-voltage 
characteristics seen before, except both axes are reversed (rotated 180 
degrees) as both voltage $I_{CB}$ and current $I_C$ are defined in the 
opposite direction. When $I_E$ is increased, $I_C=\alpha I_E+I_{CB0}$ is 
increased correspondingly. The voltage $V_{CB}$ only slightly effects $I_C$ 
as $\alpha$ is slightly increased when $V_{CB}$ is increased.

As $I_C<I_E$, CB configuration has not current amplification effect.

\htmladdimg{../figures/transistorCBplots.gif}

\end{itemize}

\item {\bf Common-Emitter:}

\begin{itemize}
\item {\bf Input characteristics:} 

As in CB case, the EB junction of CE is essentially the same as a forward
biased diode, as the effect of collector-emitter voltage $V_{CE}$ is small. 
The current-voltage characteristics is essentially the same as that of a diode:
\[ I_B=I_0 ( e^{V_{BE}/V_T}-1 )	\]

\item {\bf Output characteristics:} 

The CB junction is reverse biased, the current $I_C=\beta I_B+(\beta+1)I_{CB0}$
depends on the current $I_B$. When $I_B=0$, $I_C=I_{CE0}$, the current caused 
by the minority carriers crossing the pn-junctions. When $I_B$ is increased,
$I_C$ is correspondingly increased by $\beta$ fold. Also, as increased $V_{CE}$
will slightly increase $\alpha$ but much greater increase in 
$\beta=\alpha/(1-\alpha)$, $I_C$ will increase more significantly as $V_{CE}$
increases.

\htmladdimg{../figures/transistorCEplots.gif}

\end{itemize}

\htmladdimg{../figures/transistortemp.gif}
It is seen that various parameters of a transistor change as functions of
temperature. In particular, $\beta$ value increases along with temperature.

{\bf Example:} Assume in the CE circuit shown above, $V_{CC}=20V$, $V_1=1V$, 
$R_B=R_C=1K\Omega$, $\alpha=0.98$. Find output voltage $V_2$.
\begin{itemize}
\item Find $\beta=\alpha/(1-\alpha)=49$
\item Find $I_B$. As the BE junction is forward biased, the voltage drop is
	about $V_{BE}=0.7\;V$, and $I_B=(V_1-V_{BE})/R_B=(1-0.7)/1=0.3\;mA$
\item Find $I_C=\beta I_B=49\times 0.3\;mA=14.7\; mA$
\item Find $V_2=V_{CC}-I_C R_C=20\;V-14.7\;mA \times 1\;K\Omega=5.3\;V$
\end{itemize}

{\bf Load Line:} The last equation above $V_{CE}=V_{CC}-I_C R_C$ is a 
straight line, called the load line, on the output (collector) current-voltage
characteristics plot, which passes through the following two points:
\begin{itemize}
\item When $I_C=0$, $V_{CE}=V_{CC}$. 
\item When $I_C=V_{CC}/R_C$, $V_{CE}=0$.
\end{itemize}
The actual collector current $I_C$ and voltage $V_{CE}$ can be determined 
as the intersection of the load line and the curve in the current-voltage
characteristics, given a specific base current $I_B$, as they have to satisfy 
both the internal I-V characteristics of the transistor and the external 
voltage source ($V_{CC}$ and load resistor $R_C$. The intersection is called
the {\bf operating point} or {\bf $Q$-point}.

\subsection*{Dynamics with AC input}

As $I_C\approx \beta I_B$, common-emitter configuration is commonly used for
amplification. Consider the following example:

\htmladdimg{../figures/transistorCEplots1.gif}

The $\beta$ value of the transistor can be estimated from the plot to be:
\[ \beta=\frac{\triangle I_C}{\triangle I_B}=\frac{2\;mA}{0.05\;mA}=40 \]
Assume $V_{CC}=15V$ and the load resistance is $R_L=R_C=1.5K\Omega$. The 
two points that determine the load line are
\begin{itemize}
\item $I_C=0$, $V_{CE}=V_{CC}=15V$. 
\item $V_{CE}=0$, $I_C=V_{CC}/R_C=15V/1.5K\Omega=10 mA$
\end{itemize}
Also assume the input current is a superposition of DC current $I_B=0.1mA$ 
and a sinusoidal signal $i_b(t)=0.05mA\;cos(\omega t)$. The overall input
base current is therefore $I_B+i_b=0.1mA+0.05mA cos(\omega t)$, i.e.,
$0.05 mA \le I_B+i_b \le 0.15 mA$. 

From the I-V characteristics plot, we can find graphically 
\begin{itemize}
\item DC component of the output (collector) current is $I_C=4 mA$; 
\item DC component of output (collector) voltage is 
	$V_C=V_{CC}-R_C I_C=15V-4 mA \times 1.5 K\Omega=9V$;
\item Sinusoidal component of output current is (between 2 mA and 6 mA):
  \[ i_c=\beta\;i_b=40\times 0.05\;cos(\omega t)=2mA\;cos(\omega t) \]
\item Sinusoidal component of the output voltage is 
  \[ v_c=R_L i_c=1.5 K\Omega \times 2mA cos(\omega t)=3 cos(\omega t) V \]
\item Overall output current is 
  \[I_C+i_c=4+2 cos(\omega t) V \] 
\item Overall output voltage is 
  \[ V_C+v_c=9+3 cos(\omega t) V \]
\end{itemize}
The magnitude (peak-to-peak) of the input current is $0.15-0.05=0.1\;mA$, 
the magnitude of the output current is $6-2=4mA$, i.e., the current is
amplified 40 fold.

\end{itemize}

{\bf Switch}

From the current-voltage plot of the output characteristics, we see that
the operation of a transistor can be in one of the three possible regions:
\begin{itemize}
\item {\bf Linear region:} When the input voltage is about $V_{BE}=0.7V$,
the transistor works in the linear range where the collector current 
$I_C=\beta I_B$ is proportional to base current $I_B$. Amplification 
takes place in the linear region due to this relationship.
\item {\bf Cutoff region:} When the input voltage $V_{BE}<0.7$ (possibly
negative), $I_B\approx 0$ and $I_C=\beta I_B+I_{CE0}=I_{CE0}\approx 0$ 
is close to zero. The transistor is said to be cut off. 
\item {\bf Saturation region:} When the input voltage $V_{BE}$ is higher 
than $0.7V$, $I_B$ will significantly increase (due to the exponential
relationship between $I_B$ and $V_{BE}$). But as the maximum value of
$I_C$ is restricted by the voltage supply and the collector and emitter
resistors ($V_{CC}/(R_C+R_E)$), the linear relationship $I_C=\beta I_B$
no longer holds. In this case, the transistor is said to be saturated
and $V_{CE}\approx 0.2V$.
\end{itemize}
Severe distortion in output $v_c$ will be caused if a transistor 
amplification circuit is working near either the cutoff or the
saturation region, as can be seen in the following sections.

\htmladdimg{../figures/OperatingRegins.gif}


{\bf Example}
\htmladdimg{../figures/CEswitch.gif}

Assume $V_{CC}=15V$, $R_C=1.5\;K\Omega$, $\beta=40$, $I_{CE0}=10^{-9}A=1\;nA$, 
find output voltage $V_2$ when the input voltage $V_1$ is 0.2V, 0.7V and 0.8V.
\begin{itemize}
\item $V_1=V_{BE}=0.2V < 0.7V$, the forward bias of BE junction is 
	insufficient, $I_B=0$, $I_C=\beta I_B+I_{CE0}=I_{CE0}=1\;nA$, 
	$V_2=V_{CC}-I_C R_C=15V-1.5\times 10^{3}\times 10^{-9}\approx 15V$. 
	The transistor is cutoff, or the switch is {\bf open}.
\item $V_1=V_{BE}=0.7V$, the BE junction is forward biased, from the input
	characteristics, we find $I_B\approx 0.15\;mA$, and $I_C=\beta I_B=6
	\;mA$, $V_C=V_{CC}-I_C R_C=15-6\times 10^{-3} 10^{3}=9\; V$. The
	transistor is in linear range.
\item $V_1=V_{BE}=0.8V>0.7V$, the BE junction is forward biased, from the input
	characteristics, we find $I_B\approx 0.4\;mA$, and $I_C=\beta I_B=16
	\;mA$, $V_C=V_{CC}-I_C R_C=15-16\times 10^{-3} 10^{3}=-1 V$. This 
	means it is impossible for the transistor to draw $16mA$ from the 
	power source of $15V$, as the maximum current is 
	$V_{CC}/R_C=15\;V/1.5\;K\Omega=10\;mA$. In this case, $V_{CE}$ can 
	be determined on the output characteristics to be about 0.2V 
	(intersection of load line and the curve corresponding to 
	$I_B=0.4\;mA$), and $I_C=(15-0.2)/1.5\approx 10\;mA$. The transistor
	is in saturation range, or the switch is {\bf closed}.

\end{itemize}
{\bf Conclusion: } a change in input from 0.2 to 0.8 switches the output 
	current from 0 to about 10 mA, and the output voltage from 15 to 0.2 V,
	and the transistor is in cut-off, linear, and saturation region,
	respectively. 
	$I_C=\beta I_B$ is only valid when the transistor is in the linear
	region.

\subsection*{DC Biasing}

The DC operating point of a transistor circuit need to be set up for it to 
work properly. The operating point is determined by the biasing circuit:

\htmladdimg{../figures/transistorbiasing.gif}

\begin{itemize}
\item {\bf Fixed current biasing} This is the simplest biasing. 

	As the voltage $V_{BE}$ (0.7V) is small compared to $V_{CC}$ (>10V),
	the base current can be estimated to be:
\[	I_B=\frac{V_{CC}-V_{BE}}{R_B} \approx \frac{V_{CC}}{R_B}	\]
	The collector current is
\[	I_C=\beta I_B+(\beta+1)I_{CB0}=\beta I_B+I_{CE0}
	\approx \beta \frac{V_{CC}-V_{BE}}{R_B} \approx \beta \frac{V_{CC}}{R_B}
\]
which is directly proportional to $\beta$. The output voltage is 
\[	V_{CE}=V_{CC}-I_C R_C	\]

As $I_C$ and $V_{CE}$ depend on $\beta$, which is different for different
transistors and changes as a function of temperature, the operating point 
is unstable and inconsistent. 

{\bf Example 1:} In the circuit of fixed current biasing, $V_{CC}=15V$,
$R_B=2M\Omega=2\times 10^6\Omega$, $R_C=10K\Omega=10^4\Omega$. Assume 
$V_{BE}=0.7$ and $\beta$ changes from $20$ to $220$. Find the operating
points for $\beta=100$ and the two extreme values of $\beta$.

\begin{itemize}

\item When $\beta=20$,
\[ I_C=\beta I_B=\beta \frac{V_{CC}-V_{BE}}{R_B}=
	20 \frac{15-0.7}{2\times 10^6}=20\times 7.15 \mu A=0.143 mA \]
\[ V_C=V_{CC}-I_CR_C=15-0.143\times 10^{-3} \times 10^4=14.57 V	\]

\item When $\beta=100$, 
\[ I_C=\beta I_B=\beta \frac{V_{CC}-V_{BE}}{R_B}=
	100 \frac{15-0.7}{2\times 10^6}=100\times 7.15 \mu A=0.715 mA \]
\[ V_C=V_{CC}-I_CR_C=15-0.715\times 10^{-3} \times 10^4=7.85 V	\]

\item When $\beta=220$,
\[ I_C=\beta I_B=\beta \frac{V_{CC}-V_{BE}}{R_B}=
	220 \frac{15-0.7}{2\times 10^6}=220\times 7.15 \mu A=1.57mA \]
\[ V_C=V_{CC}-I_CR_C=15-1.57\times 10^{-3} \times 10^4=-0.7 V	\]
How can there be a negative voltage while the voltage supply is 15V?
The collector current can no longer be determined by $I_C=\beta I_B$,
as the maximum $I_C$ corresponding to the transistor fully saturated
with $V_{CE}=0.2$ is $(V_{CC}-V_{CE})/R_C=14.8/10=1.48\;mA$. 

\end{itemize}

\item {\bf Self-biasing} 

To correct the problem above, self-biasing circuit is used to decrease the 
effect of changing $\beta$ by negative feed back. Qualitatively, if $I_C$ is 
increased due to increased $\beta$ or temperature, the following happens:
\[	I_C \uparrow \Longrightarrow V_E \uparrow \Longrightarrow V_{BE} 
	\downarrow \Longrightarrow I_B \downarrow \Longrightarrow 
	I_C=\beta I_B \downarrow	\]
This is a negative feedback loop which tends to stabilize the operating point.

To analyze this circuit quantitatively, we first find the base voltage 
$V_B$ and base current $I_B$. Note that only when the base current is much 
smaller than the current through $R_2$ ($I_B \ll I_2$), can we approximate
$V_B$ by voltage divider as:
\[	V_B=V_{CC} \frac{R_2}{R_1+R_2}	\]
If the condition $I_B \ll I_2$ is not satisfied, we have to use Thevenin's 
theorem to replace the base circuit by a voltage source 
\[	V_{BB}=\frac{R_2}{R_1+R_2} V_{CC}	\]
in series with a resistance
\[	R_B=\frac{R_1R_2}{R_1+R_2}	\]

\htmladdimg{../figures/transistorbiasing1.gif}

Next we use KVL to the base loop to get
\[ V_{BB}-I_BR_B-V_{BE}-(I_C+I_B)R_E=V_{BB}-V_{BE}-I_CR_E-I_B(R_B+R_E)=0 \]
Substituting 
\[	I_B=\frac{I_C}{\beta}-\frac{\beta+1}{\beta} I_{CB0}\approx
	\frac{I_C}{\beta}	\]
and solving for $I_B$, we get
\[	I_B=\frac{V_{BB}-V_{BE}}{(\beta+1) R_E+R_B}	\]
\[	I_C =\beta I_B=\frac{\beta(V_{BB}-V_{BE})}{(\beta+1) R_E+R_B}
            \approx \frac{\beta(V_{BB}-V_{BE})}{\beta R_E+R_B}
            \approx \frac{V_{BB}-V_{BE}}{R_E} \]
The last approximation is based on the further assumption that even for 
the minimum possible $\beta$ value of the transistor in the circuit,
it is still true that $R_B \ll \beta_{min} R_E$, e.g., 
$R_B =0.1\times \beta_{min} R_E$. If this condition is 
satisfied, $I_C$, and thereby the Q operation point is totally determined 
by the resistors independent of the $\beta$ value of the transistor. 
Comparing this with fixed biasing, where 
$I_C \approx \beta (V_{CC}-V_{BE})/R_B$ is directly proportional to the
$\beta$ value of the transistor, circuit has a much more stable operating
point.

{\bf Example 2:} In the circuit of self-biasing, $V_{CC}=28V$, 
$R_1=90 K\Omega$, $R_2=10 K\Omega$, $R_E=2 K\Omega$, $R_C=14 K\Omega$, 
Assume $V_{BE}=0.7$ and $\beta$ changes from $20$ to $200$. Find the 
operating points for $\beta=100$ and the two extreme values of $\beta$.
\[	R_B=R_1R_2/(R_1+R_2)=10\times 90/(10+90)=9 K\Omega	\]
\[	V_{BB}=V_{CC} R_2/(R_1+R_2)=28\times 10/(10+90)=2.8V	\]
\begin{itemize}
\item When $\beta=20$, 
\[ I_C=\beta I_B=\beta \frac{V_{BB}-V_{BE}}{(\beta+1) R_E+R_B}
	=\frac{20\times (2.8-0.7)}{21\times 2000+9000}
	\approx 0.82 mA	\]
\[ V_C=V_{CC}-I_CR_C=28-0.82\times 10^{-3} \times 14\times 10^3=16.5 V	\]
\[ V_E=I_E R_E=(I_C+I_E)R_E=(\beta+1)I_B R_E\approx 0.82 mA 2K\Omega=1.64V \]
\[ V_{CE}=V_C-V_E=16.5-1.64=14.8V \]

\item When $\beta=100$, 
\[ I_C=\beta I_B=\beta \frac{V_{BB}-V_{BE}}{(\beta+1) R_E+R_B}
	=\frac{100\times (2.8-0.7)}{101\times 2000+9000}
	=2.1/2110 \approx 1 mA	\]
\[ V_C=V_{CC}-I_CR_C=28-10^{-3} \times 14\times 10^3=14 V	\]
\[ V_E=I_E R_E=(I_C+I_E)R_E=(\beta+1)I_B R_E\approx 1mA 2K\Omega=2V \]
\[ V_{CE}=V_C-V_E=12V \]

\item When $\beta=200$, 
\[ I_C=\beta I_B=\beta\frac{V_{BB}-V_{BE}}{(\beta+1) R_E+R_B}
	=\frac{200\times (2.8-0.7)}{201\times 2000+9000}
	\approx 1.02 mA	\]
\[ V_C=V_{CC}-I_CR_C=28-1.02\times 10^{-3} \times 14\times 10^3=13.7 V	\]
\[ V_E=I_E R_E=(I_C+I_E)R_E=(\beta+1)I_B R_E\approx 1.02mA 2K\Omega=2V \]
\[ V_{CE}=V_C-V_E=13.7-2=11.7V \]
\end{itemize}

\end{itemize}

\subsection*{Small Signal Model and H parameters}

{\bf Two-port circuit:} 

\htmladdimg{../figures/twoportmodel.gif}
\htmladdimg{../figures/transistorHmodel.gif}

A transistor circuit can be treated as a two-port circuit with input and output
ports with four variables $(v_1, i_1, v_2, i_2)$. In general two of the four
variables are independent and the rest two can be expressed as their functions:
\[
	\left\{ \begin{array}{l} v_1=f_1(i_1,i_2) \\ v_2=f_2(i_1,i_2)
	\end{array} \right.
	\;\;\;\;\mbox{or}\;\;\;\;
	\left\{ \begin{array}{l} i_1=f_3(_1,v_2) \\ i_2=f_4(v_1,v_2)
	\end{array} \right.
	\;\;\;\;\mbox{or}\;\;\;\;
	\left\{ \begin{array}{l} v_1=f_5(i_1,v_2) \\ i_2=f_6(i_1,v_2)
	\end{array} \right.
\]
We use the third {\em hybrid} model to describe the CE transistor circuit with 
$v_1=v_{be}$, $i_1=i_b$, $v_2=v_{ce}$, and $i_2=i_{ce}$:
\[
	\left\{ \begin{array}{l} v_{be}=v_{be}(i_b,v_{ce}) \\ i_c=i_c(i_b,v_{ce})
	\end{array} \right.
\]
Taking the total derivative, we get:
\[
	dv_{be}=\frac{\partial v_{be}}{\partial i_b} d i_b
	+\frac{\partial v_{be}}{\partial v_{ce}} d v_{ce} 
	=h_i d i_b+h_r d v_{ce}	
	\;\;\;\;\;\;
	di_c=\frac{\partial i_c}{\partial i_b} d i_b
	+\frac{\partial i_c}{\partial v_{ce}} d v_{ce} 
	=h_f d i_b+h_o d v_{ce}
\]
where $h_i, h_f, h_r, h_o$ are the hybrid model parameters:
\begin{itemize}
\item $h_i=\partial v_{be}/\partial i_b}=r_{be}$: input impedance with 
  $v_{ce}=0$ (output short-circuit). This is AC resistance between base and 
  emitter, the reciprocal of the slope of the current-voltage curve of the 
  input characteristics. 

\item $h_r=\partial v_{be}/\partial v_{ce}}$: reverse transfer voltage ratio
  with $i_b=0$ (input open-circuit), representing how $V_{CE}$ affects $V_{BE}$.
  In general $h_i$ is small and can be ignored.

\item $h_f=\partial i_c}/\partial i_b}=\beta$: forward transfer current 
    ratio or current amplification factor  with $v_{ce}=0$ (output 
    short-circuit). Typically, $h_f=\beta$ is in the range of 20 to 200.

\item $h_o=\partial i_c/\partial v_{ce}}=1/r_{ce}$: output admittance with 
  $i_b=0$ (input open-circuit). It is slope of the current-voltage curve in 
  the output characteristics. In general $h_o$ is small and can be ignored.
\end{itemize}
If all variables $i_b$, $v_{be}$, $i_c$ and $v_{ce}$ are small signals (around 
the DC operating point $Q$ and far away from either the cut-off or the saturation 
region), these differential quantities can be rewritten as 
\[
	v_{be}=\frac{\partial v_{be}}{\partial i_b}
	i_b+\frac{\partial v_{be}}{\partial v_{ce}} v_{ce}=h_i i_b+h_r v_{ce}	
	\approx h_i i_b
	\;\;\;\;\;\;
	i_c=\frac{\partial i_c}{\partial i_b} i_b
	+\frac{\partial i_c}{\partial v_{ce}} v_{ce}=h_f i_b+h_o v_{ce}
	\approx h_f i_b
\]


\htmladdimg{../figures/hparameters.gif}

\htmladdimg{../figures/smallsignalmodelBJT.gif}

In general, $h_r$ and $h_o$ are small and could be assumed zero to further 
simplify the model, as shown on the left of the figure.

The equivalent AC resistance between base and emitter $r_{be}$ can be found 
as below. Assume small signal $(v_B(t), i_B(t))$ around the DC operating point 
$(V_B, I_B)$, the base voltage is $V_B+v_B(t)$ and the base current is
\begin{eqnarray}
&& I_0 [e^{(V_B+v_B(t))/V_T}-1]\approx I_0 e^{(V_B+v_B(t))/V_T}
	=I_0 e^{V_B/V_T} e^{v_B(t)/V_T}
	\nonumber \\
&=& I_B e^{v_B(t)/V_T} \approx I_B(1+\frac{v_B(t)}{V_T})=I_B+i_B(t)	
\end{eqnarray}
Note that $e^x=1+x+x^2/2!+x^3/3!+\cdots $, and 
\[	i_B(t)=I_B \frac{v_B(t)}{V_T}	\]
Then we get
\[	r_{be}=\frac{v_B(t)}{i_B(t)}=\frac{V_T}{I_B}	\]

Typically, $\beta=100$, $I_B=50\mu A$, $I_C=\beta I_B=100\times 50\mu A=5mA$.
Based on the small signal model, a transistor can be analyzed as a two-port 
circuit containing four elements, as shown in the figure below:


\subsection*{AC Amplification}

Based on the previous discussion, a single transistor AC amplification circuit 
is given as shown in the figure.

\htmladdimg{../figures/ACamplification1.gif}

If the capacitances of the coupling capacitors and the emitter by-pass 
capacitor are large enough with respect to the frequency of the AC signal 
in the circuit is high enough, these capacitors can all be approximated as 
short circuit. Moreover, note that the AC voltage of the voltage supply
$V_{CC}$ is zero, it can be treated the same as the ground. Now the AC
behavior of the transistor amplification circuit can be modeled by the 
following small signal equivalent circuit:

\htmladdimg{../figures/ACamplification2.gif}

{\bf AC Input Impedance:} For AC signals, the input of the amplification
circuit is shown below, where $R_s$ is the internal resistance of the signal 
source, and the input impedance of the circuit is the three resistances $R_1$, 
$R_2$ and $r_{be}$ in parallel:
\[	R_{in}=R_1||R_2||r_{be}
	=(\frac{1}{R_1}+\frac{1}{R_2}+\frac{1}{r_{be}})^{-1}
	=\frac{R_1 R_2 r_{be}}{R_1R_2+R_1r_{be}+R_2r_{be}}	\]
where $r_{be}$ is the resistance of the PN-junction) between the base and 
the emitter of the transistor, as discussed before:
\[	r_{be}=\eta V_T/I_B	\]

{\bf AC Output Impedance:} This is simply the resistance of the resistor 
\[	R_{out}=R_C	\]

{\bf AC Amplification Gain:} Given the AC input voltage $v_{in}$, the base
voltage can be found (voltage divider) to be
\[	v_B=v_{in}\; \frac{(R_1||R_2||r_{be})}{(R_1||R_2||r_{be})+R_s}	\]
and the base current is
\[	i_B=\frac{v_B}{r_{be}}=v_{in}\;\frac{(R_1||R_2||r_{be})}{(R_1||R_2||r_{be})+R_s}\frac{1}{r_{be}}		\]
The collector current is $i_C=\beta \;i_B$ and the output voltage is
\[	v_C=-i_C\;(R_C||R_L)=-\beta\;i_B\;(R_C||R_L)	\]
Here the negative sign indicates the fact that $v_C$ is $180^\circ$ out of
phase with $v_B$, as 
\[	v_B \uparrow \Longrightarrow i_B \uparrow \Longrightarrow i_C \uparrow
	\Longrightarrow v_C \downarrow	\]
The voltage gain is therefore
\[
G=\frac{v_{out}}{v_{in}}=\frac{v_C}{v_{in}}=
	-\beta \frac{(R_1||R_2||r_{be})}{(R_1||R_2||r_{be})+R_s}
	\frac{1}{r_{be}}(R_C||R_L)	\]
In particular, if the input resistance is much larger than the internal 
resistance of the voltage source, i.e., 
\[	R_{in} R_1||R_2||r_{be} \gg  R_s	\]
and the output resistance is much smaller than the load resistance, i.e.,
\[	R_{out}=R_C \ll R_L	\]
then the gain can be approximated as
\[	G=\frac{v_{out}}{v_{in}}=-\beta \frac{R_C}{r_{be}}	\]

{\bf Example:} 

\htmladdimg{../figures/transistorBJTexample1.gif}

This figure shows a common-emitter amplification circuit of npn BJT. Assume
$V_{CC}=V_{BB}=12V$, 

We assume the capacitors are large enough so that they can be considered
as short circuit for the AC signals. The DC and AC circuits are shown
below:

\htmladdimg{../figures/transistorBJTexample1a.gif}
\htmladdimg{../figures/transistorBJTexample1b.gif}

\begin{itemize}
\item Find base current:
\[	I_B=(V_{BB}-V_{BE})/R_B \approx V_{BB}/R_B=12/(300\times 10^3)
	=40\times 10^{-6}\; A=40\;\mu A	\]
\item Find DC load line: when $I_C=0$, $V_{CE}=12V$, when $V_{CE}=0$,
\[	I_C=V_{CC}/R_C=12V/(4\times 10^3)=3\times 10^{-3}A=3 \;mA	\]
	The DC load line is determined by these two points
\item Find DC operating point $Q$: 
	The intersection of the DC load line and the $I_C$ curve 
	corresponding to $I_B=40 \;\mu A$ is the DC operating point with 
	$V_{CE}=6V$ and $I_C=1.5 mA$ (i.e., $\beta=1.5ma/40 \mu A=37.5$).
\item Find AC load line: 
	The AC load is $R'_L=R_C//R_L=(R_C+R_L)/R_CR_L=2\;K\Omega$. The AC load
	line is a straight line passing the DC operating point $Q$ with its
	slope equal to $-1/R'_L=-1/2$. The intersections of AC load line 
	with $V_{CE}$ and $I_C$ axes can be found by
\[	\frac{1.5}{6-v_0}=-\frac{1}{2}\;\;\;\;\Longrightarrow v_0=9V \]
	and 
\[	\frac{i_0-1.5}{-6}=-\frac{1}{2}\;\;\;\;\Longrightarrow i_0=4.5\;mA \]
\item Find input current:
	Assume input voltage is $v_i=20\; sin\;\omega\;t\;mV$ and $V_{BE}=0.6V$, 
	the overall base voltage is $v_{BE}=V_{BE}+v_i=0.6+0.02\;sin\;\omega\;t\;V$, 
	and the corresponding base current can be found from the input i-v 
	characteristics to be $i_B=I_B+i_i=40+20\; sin\;\omega\;t\;\mu A$ 
	between 20 and 60 $\mu A$.
\item Find AC output voltage: This can be found graphically from the 
	output i-v characteristics, based on $I_B$, to be 
	$V_o=v_{CE}+v_o=6-1.5 \;sin\;\omega\; t\; V$, and the current is
	$I_o=i_C+i_o=1.5+ 0.75 \;sin\omega \;t\; mA$.
	Note that the output is in opposite phase (180 phase shift) with 
	the input. 
\item Find the voltage gain:
	\[ A_v=\frac{|v_o|}{|v_i|}=\frac{1.5}{0.02}=75	\]
\end{itemize}

\htmladdimg{../figures/transistorBJTexample1c.gif}

For a transistor to work properly, its DC operating point has to be set right,
otherwise distortion may be caused, as shown below. So to avid distortion, the
dynamic range should be maximized by setting the DC operating point at the 
middle point of the load line.

\htmladdimg{../figures/transistorQpoint.gif}

The circuit above can also be analyzed using the small-signal model. 

\htmladdimg{../figures/BJTexamplesmallsignal.gif}
The DC variables:
\[	I_B=(V_{BB}-V_{BE})/R_B \approx V_{BB}/R_B=40 \;\mu A	\]
\[	I_C=\beta\;I_B=37.5\times 40\;\mu A=1.5\;mA	\]
\[	V_{CE}=V_{CC}-R_C I_C=12-1.5\times 4000=6\;V	\]
\[	r_{be}=V_T/I_B=26\;mV/40\mu A=650 \;\Omega	\]
The AC variables:
\[	v_i=i_b r_{be}, \;\;\;\;v_o=-i_c R_C//R_L=-\beta i_b R_C//R_L	\]
The voltage gain is:
\[	A_v=\frac{v_o}{v_i}=-\frac{\beta R_C//R_L}{r_{be}}
	=-\frac{37.5\times 2000}{650}=-115	\]
The input resistance is $R_i=R_B//r_{be}\approx 650 \Omega$, the output 
resistance is $R_o=R_C//r_{ce}\approx R_C=4\;K\Omega$.

\subsection*{Emitter Follower}

An emitter follower circuit shown in the figure is widely used in AC 
amplification circuits. The input and output of the emitter follower are
the base and the emitter, respectively, therefore this circuit is also
called common-collector circuit.

\htmladdimg{../figures/emitterfollower.gif}

% Assume $\beta=50$, $R_B=300K\Omega$, $R_E=4K\Omega$, $R_S=500\Omega$, 
% $R_L=5.1K\Omega$, and $V_{CC}=12$.

{\bf DC operating point}

\[ \left\{ \begin{array}{l} V_{CC}=R_B I_B+V_{be}+R_E I_E	\\
	I_E=(\beta+1) I_B	\\
	V_{ce}=V_{CC}-R_E I_E \end{array} \right. \]
Solving these equations, we can get $I_B$, $I_E$ and $V_{ce}$.

{\bf AC small-signal equivalent circuit}

\htmladdimg{../figures/emitterfollower1.gif}

We assume $R_B \gg r_{be}+R_E$ and therefore can be ignored, and have
\[ \left\{ \begin{array}{l} 
	v_{out}=i_e (R_E||R_L)=(\beta+1) i_b (R_E||R_L) 	\\
	v_{in}=i_b (R_S+r_{be})+(\beta+1) i_b (R_E||R_L)=i_b (R_S+r_{be})+v_{out}
\end{array} \right. \]

{\bf Voltage gain:}
\[	G=\frac{v_{out}}{v_{in}}
	=\frac{(\beta+1) (R_E||R_L)}{(R_S+r_{be})+(\beta+1) (R_E||R_L)} 
\]
As $R_S+r_{be} \ll (\beta+1) (R_E||R_L)$, $G$ is smaller than but approximately
equal to 1.

{\bf The input resistance:} 

The input resistance is $R_B$ in parallel with the resistances of the circuit
to its right including the load $R_L$, which can be found by $v_b/i_b$. But as
\[ v_b=i_b(r_{be}+(\beta+1)(R_E||R_L) \]
we have
\[ r_{in}=R_B || (r_{be}+(\beta+1) (R_E||R_L) )\approx (\beta+1)(R_E||R_L) \]
Comparing this with the input resistance of the common-emitter circuit
$r_{in}=R_1||R_2|| r_{be} \approx r_{be}$, we see that the emitter
follower has a very large input resistance.

{\bf The output resistance:}

The output resistance is $R_E$ in parallel with the resistances of the circuit
to its left including the source, which can be found by $v_e/i_e$, where
\[ v_e=i_b(r_{be}+R_S||R_B), \;\;\;\;\;\mbox{and}\;\;\;\;\; i_e=(\beta+1)i_b\]
and we have
\[	\frac{v_e}{i_e}=\frac{r_{be}+R_S||R_B}{\beta+1}	\]
Alternatively, this resistance can be found as $v_{open}/i_{short}$, where 
$v_{open}$ is the output voltage with load open-circuit $R_L=\infty$ and
$i_{short}$ is the current with load short-circuit $R_L=0$. To find $v_{open}$
and $i_{short}$, the voltage source $v_{in}$ with internal resistance $R_S$ 
and a load $R_B$ can be converted by Thevenin's theorem to 
\[ R_{th}=R_S||R_B=\frac{R_B R_S}{R_B+R_S},	\;\;\;\;
	v_{th}=v_{in} \frac{R_B}{R_B+R_S}=v_{open}	\]
and $i_{short}$ can be found to be
\[	i_{short}=i_e=(\beta+1)i_b=(\beta+1)\frac{v_{th}}{r_{be}+R_{th}} \]
now we have
\[ \frac{v_{open}}{i_{short}}=\frac{v_{th}(r_{be}+R_{th})}{(\beta+1) v_{th}}
	=\frac{r_{be}+R_{th}}{\beta+1} =\frac{r_{be}+R_S||R_B}{\beta+1} \]
The overall output resistance is
\[	r_{out}	= R_E || \frac{r_{be}+R_S||R_B}{\beta+1}	
	\approx \frac{r_{be}+R_S}{\beta+1}	\]

%with voltage $v_{in}=0$ (source short)

Comparing this with the output resistance of the common-emitter circuit
$r_{out}=R_C$, we see that the emitter follower circuit has very small
output resistance.

{\bf Conclusion:} 

Emitter follower does not amplify voltage. However, due to its large input 
resistance drawing little current from the source, and its small output 
resistance capable of driving heavy load, it is widely used as both the 
input and output stages for a multi-stage voltage amplification circuit 
due to its property of very favorable input/output resistances.

\subsection*{Multi-stage Amplification}

In order to have an amplification gain, multi-stage amplification circuits
are needed. Such a circuit is typically composed of two or more cascaded
transistor amplifiers, coupled in one of three possible ways:
\begin{itemize}
\item Capacitor coupling:
\htmladdimg{../figures/coupling_capacitor.gif}
	\begin{itemize}
	\item Independent DC operating point;
	\item AC amplification of high gain if coupling capacitors are
		larger enough;
	\item Cannot amplify DC and low frequency signals;
	\item Difficult implementation on IC.
	\end{itemize}

\item Transformer coupling:
\htmladdimg{../figures/coupling_transformer.gif}
	\begin{itemize}
	\item Independent DC operating point;
	\item Can achieve maximal power by impedance match;
	\item Cannot amplify DC and low frequency signals;
	\item Difficult implementation on IC.
	\end{itemize}

\item Direct coupling:
\htmladdimg{../figures/coupling_direct.gif}
	\begin{itemize}
	\item DC operating point not independent;
	\item Can amplify both DC and AC signals;
	\item Easy implementation on IC.
	\end{itemize}

\end{itemize}





\end{document}


	

	












