\documentstyle[12pt]{article}
\usepackage{html}
% \usepackage{graphics}  
\begin{document}

\section*{Chapter 6: Active Filter Design}

\subsection{First and Second Order Low/High/Band-Pass filters}

\htmladdimg{../figures/opamp4a.gif}

\begin{itemize}
\item {\bf Low-pass filter:}
  \begin{eqnarray}
    H(j\omega)&=&-\frac{Z_2}{Z_1}=-\frac{R_2\;||\;(1/j\omega C)}{R_1}
    =-\frac{1}{R_1}\frac{R_2/j\omega C}{R_2+1/j\omega C}
    =-\frac{R_2}{R_1}\frac{1}{j\omega R_2C+1}
    \nonumber \\
    &=&-H(0)\frac{1}{1+j\omega \tau}=-H(0)\frac{1/\tau}{1/\tau+j\omega} 
    =-H(0)\frac{\omega_c}{j\omega+\omega_c} 
    \nonumber
  \end{eqnarray}
  where $H(0)=R_2/R_1$ is the DC gain when $\omega=0$, and
  $\omega_c=1/\tau=1/R_2C$ is cut-off or corner frequency, at which 
  $|H(j\omega_c)|=H(0)/\sqrt{2}$. Intuitively, when frequency is high, 
  $Z_2(j\omega)$ is small and the negative feedback becomes strong,
  and the output is low. For example, when
  $\tau=10^{-3}$, $\omega_c=1/\tau=10^3$, the Bode plots are shown  
  below:

  \htmladdimg{../figures/BodeLP.gif}

\item {\bf High-pass filter:}
  \begin{eqnarray}
    H(j\omega)&=&-\frac{Z_2(j\omega)}{Z_1(j\omega)}
    =-\frac{R_2}{R_1+1/j\omega C}
    =-\frac{R_2}{R_1}\;\frac{1}{1+1/j\omega R_1C}
    \nonumber \\
    &=&-\frac{R_2}{R_1}\;\frac{j\omega}{j\omega +1/R_1C}
    =-H(\infty)\;\frac{j\omega}{j\omega+1/\tau} 
    =-H(\infty)\frac{j\omega}{j\omega+\omega_c} 
    \nonumber
  \end{eqnarray}
  where $H(\infty)=R_2/R_1$ is the gain when $\omega\rightarrow\infty$, 
  $\omega_c=1/\tau=1/R_1C$ is the cut-off or corner frequency, at which
  $|H(j\omega_c)|=H(\infty)/\sqrt{2}$. Intuitively, when frequency is 
  low $Z_1(j\omega)$ is large and the signal is difficult to pass, 
  therefore the output is low. For example, when $\tau=10^{-6}$, 
  $\omega_c=1/\tau=10^6$, the Bode plots are shown below:
  
  \htmladdimg{../figures/BodeHP.gif}

  If we let $R_1=R_2$, i.e., $H(0)=1$, and ignore the negative sign ($180^\circ$
  phase shift), the low-pass and high-pass filters can be represented by their
  transfer functions with $s=j\omega$:
  \[
  H_{lp}(j\omega)=\frac{\omega_c}{j\omega+\omega_c},
  \;\;\;\;\;\;\;H_{hp}(j\omega)=\frac{j\omega}{j\omega+\omega_c}
  \]

\item {\bf Second Order Band-pass Filters:}

  \htmladdimg{../figures/opamp4b.gif}

  We let 
  \[
  Z_1(\omega)=R_1+(1/j\omega C_1),\;\;\;\;\;\;\;
  Z_2(\omega)=R_2||(1/j\omega C_2)
  \]
  and get the FRF of this inverting amplifier as
  \begin{eqnarray}
    H_{BP}(j\omega)&=&-\frac{Z_2(\omega)}{Z_1(\omega)}
    =-\frac{R_2||(1/j\omega C_2)}{R_1+1/j\omega C_1}
    =-\frac{R_2/j\omega C_2}{(R_1+1/j\omega C_1)(R_2+1/j\omega C_2)}
    \nonumber \\
    &=&-\frac{j\omega R_2C_1}{(j\omega R_1C_1+1)(j\omega C_2R_2+1)}
    =-\frac{j\omega \tau_3}{(1+j\omega \tau_1)(1+j\omega \tau_2)} 
    \nonumber \\
    &=&\left(\frac{\omega_{c_1}}{j\omega+\omega_{c_1}}\right)\;
    \left(\frac{\omega_{c_2}}{j\omega+\omega_{c_2}}\right)\;
    \left(\frac{j\omega}{\omega_{c_3}}\right)
    \nonumber
  \end{eqnarray}
  where $\omega_{c_1}=1/\tau_1=1/R_1C_1$, $\omega_{c_2}=1/\tau_2=1/R_2C_2$, 
  and $\omega_{c_3}=1/\tau_3=1/R_2C_1$. We see from the Bode plot above
  that this is indeed a band-pass filter.

  For example, when $\tau_1=10^{-6}$, $\tau_2=10^{-8}$, $\tau_3=10^{-3}$, 
  the Bode plots are shown below:

  \htmladdimg{../figures/BodeBP.gif}

\item {\bf Second Order Band-stop Filters:}

  If the log-magnitude $Lm$ of the Bode plot of a band-pass filter is 
  vertically flipped, we get a band-stop filter:
  \[
  -Lm_{BP}(j\omega)=-20\log |H_{BP}(j\omega)|
  =-20\log \bigg| \frac{Z_2(\omega)}{Z_1(\omega)} \bigg|
  =20\log \bigg| \frac{Z_1(\omega)}{Z_2(\omega)} \bigg|
  =Lm_{BS}(j\omega)
  \]
  We therefore see that if we simply swap $Z_1(\omega)$ and $Z_2(\omega)$ 
  in the op-ammp circuit of the band-pass filter, we get a band-stop filter:

  \begin{eqnarray}
    H_{BS}(j\omega)&=&-\frac{Z_1(\omega)}{Z_2(\omega)}
    =-\frac{R_2+1/j\omega C_2}{R_1||1/j\omega C_1}
    =-\frac{R_2+1/j\omega C_2}{R_1/j\omega C_1/(R_1+1/j\omega C_1)}
    \nonumber \\
    &=& \frac{(1+j\omega C_1R_1)(1+j\omega C_2R_2)}{j\omega R_1C_2}
    =\frac{(1+j\omega\tau_1)(1+j\omega\tau_1)}{j\omega\tau_3}
    \nonumber
  \end{eqnarray}

\end{itemize}

\begin{comment}

{\bf Example}

\htmladdimg{../figures/opampMF.png}

Apply KCL to the point between $C_1$ and $C_2$ denoted by $V$:
\[
\frac{V_{in}-V}{1/j\omega C_1}+\frac{V_{out}-V}{1/j\omega C_2}
=\frac{V}{R_2}+\frac{V}{1/j\omega C_1}
\]
i.e.,
\[
V_{in}j\omega C_1+V_{out}j\omega C_2
=\left(\frac{1}{R_2}+2j\omega C_1+j\omega C_2\right) V
=\left(\frac{1}{R_2}+j\omega(2C_1+C_2)\right) V
\]
Apply KCL to the inverting input of the op-amp which is virtually
grounded:
\[
\frac{V}{1/j\omega C_1}+\frac{V_{out}}{R_1}=0
\;\;\;\;\;\;\mbox{i.e.}\;\;\;\;\;\;
V=-\frac{1}{j\omega R_1C_1}V_{out}
\]
Substitute into the first equation
\begin{eqnarray}
  V_{in}j\omega C_1+V_{out}j\omega C_2
  &=&-\left(\frac{1}{R_2}+j\omega(2C_1+C_2)\right) \frac{1}{j\omega R_1C_1}V_{out}
  \nonumber\\
  &=&-\left(\frac{1}{j\omega R_1R_2C_1}+\frac{2C_1+C_2}{R_1C_1}\right) V_{out}
  \nonumber
\end{eqnarray}
Rearrange:
\[
V_{in}
=-\frac{1}{j\omega C_1}\left(j\omega C_2+\frac{1}{j\omega R_1R_2C_1}+\frac{2C_1+C_2}{R_1C_1}\right) V_{out}
=\left(\frac{C_2}{C_1}+\frac{1}{(j\omega)^2 R_1R_2C^2_1}
+\frac{2C_1+C_2}{j\omega R_1C_1^2}\right) V_{out}
\]
Now we can find the FRF:
\begin{eqnarray}
H(j\omega)&=&\frac{V_{out}}{V_{in}}
=-\left(\frac{C_2}{C_1}+\frac{1}{(j\omega)^2 R_1R_2C_1^2}
+\frac{2C_1+C_2}{j\omega R_1C_1^2}\right)^{-1}
=-\frac{(j\omega)^2R_1R_2C_1^2}{(j\omega)^2R_1R_2C_1C_2+j\omega(2C_1+C_2)R_2+1}
\nonumber\\
&=&-\frac{C_1}{C_2}\;\frac{(j\omega)^2}{(j\omega)^2+j\omega(2C_1+C_2)/R_1C_1C_2+1/R_1R_2C_1C_2}
=-\frac{C_1}{C_2}\;\frac{(j\omega)^2}{(j\omega)^2+j\omega\omega_n/Q+\omega_n^2}
\nonumber
\end{eqnarray}
where
\[
\omega_n=\frac{1}{\sqrt{R_1R_2C_1C_2}},\;\;\;\;\;\;
\frac{2C_1+C_2}{R_1C_1C_2}=\frac{\omega_n}{Q}=\Delta\omega,
\;\;\;\mbox{i.e.,}\;\;\;\;
Q=\omega_n\frac{R_1C_1C_2}{2C_1+C_2}=\frac{\sqrt{R_1R_2C_1C_2}}{(2C_1+C_2)R_2}
\]
This is a 2nd-order high-pass filter, when $\omega=\infty$, we have
\[
H(j\omega)\big|_{\omega=\infty}=H(j\infty)=-\frac{C_1}{C_2}
\]


{\bf Example}

\htmladdimg{../figures/opampMF1.png}

\[
\frac{V_{in}-V}{R_1}+\frac{V_{out}-V}{R_2}=\frac{V}{R_1}+\frac{V}{1/j\omega C_2}
\;\;\;\;\mbox{i.e.,}\;\;\;\;
\frac{V_{in}}{R_1}+\frac{V_{out}}{R_2}=\left( \frac{2}{R_1}+\frac{1}{R_2}+j\omega C_2 \right)V
\]
\[
\frac{V}{R_1}+\frac{V_{out}}{1/j\omega C_1}=0,\;\;\;\;\mbox{i.e.,}\;\;\;\;
V=-j\omega R_1C_1 V_{out}
\]
Substitute into the first equation:
\[
\frac{V_{in}}{R_1}+\frac{V_{out}}{R_2}
=-\left( \frac{2}{R_1}+\frac{1}{R_2}+j\omega C_2 \right)j\omega R_1C_1 V_{out}
=-\left( j\omega C_1 \left(\frac{2R_2+R_1}{R_2}\right)+(j\omega)^2 R_1C_1C_2 \right)V_{out}
\]
Rearrange:
\begin{eqnarray}
\frac{V_{in}}{R_1}&=&-\left(\frac{1}{R_2}+j\omega C_1\left(\frac{2R_2+R_1}{R_2}\right)+(j\omega)^2 R_1C_1C_2 \right)V_{out}
\nonumber\\
&=&-R_1C_1C_2\left(\frac{1}{R_1R_2C_1C_2}
+j\omega \left(\frac{2R_2+R_1}{R_1R_2C_2}\right)+(j\omega)^2 \right)V_{out}
\nonumber
\end{eqnarray}
Now we can find the FRF:
\[
H(j\omega)=\frac{V_{out}}{V_{in}}
=-\frac{1}{R_1^2C_1C_2}\;\;\frac{1}{(j\omega)^2+j\omega \omega_n/Q+\omega_n^2}
\]
where
\[
\omega_n=\frac{1}{\sqrt{R_1R_2C_1C_2}},\;\;\;\;
\frac{2R_2+R_1}{R_1R_2C_2}=\frac{\omega_n}{Q}=\Delta\omega,\;\;\;\mbox{i.e.,}\;\;\;
Q=\omega_n\frac{R_1R_2C_2}{2R_2+R_1}=\frac{\sqrt{R_1R_2C_1C_2}}{C_1(2R_2+R_1)}
\]
This is a 2nd-order low-pass filter, when $\omega=0$, we have
\[
H(j\omega)\big|_{\omega=0}=H(j0)=-\frac{1}{R_1^2C_1C_2}\;\frac{1}{\omega_n^2}
=-\frac{R_1R_2C_1C_2}{R_1^2C_1C_2}=-\frac{R_2}{R_1}
\]
\end{comment}

\subsection{The Sallen-Key filters}

The
\htmladdnormallink{Sallen-Key filters}{http://en.wikipedia.org/wiki/Sallen-Key_topology}
are second-order active filters (low-pass, high-pass, and band-pass) that 
can be easily implemented using the configuration below:

\htmladdimg{../figures/SallenKey.gif}

We represent all voltages in phasor form. Due to the virtual ground 
assumption, $V_b\approx V_{out}$ at non-inverting input is virtually 
the same as that at the inverting input, which is connected to the 
output $V_{out}$. Applying KCL to nodes a and b to get:
\[ 
\frac{V_a-V_{in}}{Z_1}+\frac{V_a-V_{out}}{Z_3}+\frac{V_a-V_{out}}{Z_2}
=\frac{V_a-V_{in}}{Z_1}+(V_a-V_{out})\left(\frac{1}{Z_3}+\frac{1}{Z_2}\right)=0 
\]
and
\[
\frac{V_{out}-V_a}{Z_2}+\frac{V_{out}}{Z_4}=0,
\]
The second equation can also be written as
\[
V_a-V_{out}=V_{out}\frac{Z_2}{Z_4}, \;\;\;\;\;\;\;V_a=V_{out}\frac{Z_2+Z_4}{Z_4}
\]
Substituting these into the first equation we get 
\[
V_{out}\;\left(\frac{Z_2+Z_4}{Z_1Z_4}+\frac{Z_2}{Z_3Z_4}+\frac{Z_2}{Z_2Z_4} \right)
=\frac{V_{in}}{Z_1}
\]
Now the frequency response of the Sallen-Key filter can be found as
the ratio of $V_{out}$ and $V_{in}$:
\begin{eqnarray}  
  H&=&\frac{V_{out}}{V_{in}}=\frac{Z_3Z_4}{Z_1Z_2+Z_1Z_3+Z_2Z_3+Z_3Z_4} 
\nonumber
\end{eqnarray}

\begin{itemize}
\item {\bf Second order low-pass filter}
  
  If we let $Z_1=R_1$, $Z_2=R_2$, $Z_3=1/j\omega C_1$, $Z_4=1/j\omega C_2$, 
  the FRF becomes (see
  \htmladdnormallink{here}{../ActiveFilters_sub/index.html}):
  \[
  H(j\omega)=\frac{\omega_n^2}{(j\omega)^2+\Delta\omega\;j\omega+\omega_n^2} 
  \]
  \begin{comment}
  \begin{eqnarray}
    H&=&\frac{1/(j\omega)^2C_1C_2}
    {R_1R_2+(R_1+R_2)/j\omega C_1+1/(j\omega)^2C_1C_2}
    \nonumber \\
    &=&\frac{1/R_1R_2C_1C_2}{(j\omega)^2+j\omega(R_1+R_2)/R_1R_2C_1+1/R_1R_2C_1C_2}
    \nonumber \\
    &=&\frac{1/R_1R_2C_1C_2}{(j\omega)^2+j\omega/R_pC_1+1/R_1R_2C_1C_2}
    \nonumber \\
    &=&\frac{\omega_n^2}{(j\omega)^2+2\zeta\omega_n\;j\omega+\omega_n^2} 
    =\frac{\omega_n^2}{(j\omega)^2+\omega_n/Q \;j\omega+\omega_n^2}
    =\frac{\omega_n^2}{(j\omega)^2+\Delta\omega\;j\omega+\omega_n^2} 
    \nonumber
  \end{eqnarray}
  \end{comment}
  where
  \[
  \omega_n=\frac{1}{\sqrt{R_1R_2C_1C_2}},\;\;\;\;\;\;\;
  \Delta\omega=\frac{1}{(R_1||R_2)C_1}=\frac{1}{R_pC_1},\;\;\;\;
  R_p=R_1||R_2=\frac{R_1R_2}{R_1+R_2}
  \]
  and
  \[
  Q=\frac{\omega_n}{\Delta\omega}=\frac{\sqrt{R_1R_2C_1C_2}}{(R_1+R_2)C_2},
  \;\;\;\;\;
  \zeta=\frac{1}{2Q}=\frac{(R_1+R_2)C_2}{2\sqrt{R_1R_2C_1C_2}}
  \]
  As there are only two parameters $\omega_n$ and $\zeta$ or $Q=1/2\zeta$ 
  to satisfy, we can arbitrarily set any two of the four variables $R_1$, 
  $R_2$, $C_1$, and $C_2$, and then solve for the other two. For example, 
  for convenience, if we let $R_1=R_2=1$, we get
  \[
  \omega_n=\frac{1}{\sqrt{C_1C_2}},\;\;\;\;\;\Delta\omega=\frac{1}{C_1}
  \]
  \begin{comment}
    \[
    \zeta=\sqrt{\frac{C_2}{C_1}}
    \]
    Solving these we get
    \[
    C_1=\frac{1}{\omega_n\zeta},\;\;\;\;\;\;C_2=\frac{\zeta}{\omega_n}
    \]
  \end{comment}

\item {\bf Second order high-pass filter}
    
  If we let $Z_1=1/j\omega C_1$, $Z_2=1/j\omega C_2$, $Z_3=R_1$, $Z_4=R_2$, 
  the FRF becomes (see
  \htmladdnormallink{here}{../ActiveFilters_sub/index.html}):
  \[
  H(j\omega)=\frac{(j\omega)^2}{(j\omega)^2+\Delta\omega\;j\omega+\omega_n^2}
  \]
  \begin{comment}
  \begin{eqnarray}
    H&=&\frac{R_1R_2}
    {1/(j\omega)^2C_1C_2+R_1(1/j\omega C_1+1/j\omega C_2)+R_1R_2}
    \nonumber \\
    &=&\frac{(j\omega)^2}{1/R_1R_2C_1C_2+j\omega\;(C_1+C_2)/C_1C_2R_2+(j\omega)^2}
    \nonumber \\
    &=&\frac{(j\omega)^2}{(j\omega)^2+j\omega\;/C_sR_2+1/R_1R_2C_1C_2}
    \nonumber \\
    &=&\frac{(j\omega)^2}{(j\omega)^2+2\zeta\omega_n\;j\omega+\omega_n^2} 
    =\frac{(j\omega)^2}{(j\omega)^2+\omega_n/Q \;j\omega+\omega_n^2} 
    =\frac{(j\omega_n)^2}{(j\omega)^2+\Delta\omega\;j\omega+\omega_n^2}
    \nonumber
  \end{eqnarray}
  \end{comment}
  where 
  \[
  \omega_n=\frac{1}{\sqrt{R_1R_2C_1C_2}},\;\;\;\;\;\;\;
  \Delta\omega=\frac{1}{C_sR_2},\;\;\;\;\;
  C_s=\frac{C_1C_2}{C_1+C_2}
  \]
  and
  \[
  Q=\frac{\sqrt{R_1R_2C_1C_2}}{(C_1+C_2)R_1},\;\;\;\;\;\;
  \zeta=\frac{1}{2Q}=\frac{(C_1+C_2)R_1}{2\sqrt{R_1R_2C_1C_2}}
  \]

\item {\bf Band-pass filter}

  \htmladdimg{../figures/SallenKeyBP.png}

  By voltage divider and virtual ground, we get
  \[
  V_2=\frac{R_a}{R_a+R_b}V_{out}=kV_{out},\;\;\;\;\;\;\;\left(k=\frac{R_a}{R_a+R_b}\right)
  \]
  Apply KCL to node $V_2$ to get:
  \[
  \frac{V_1-V_2}{Z_2}=\frac{V_2}{Z_4},\;\;\;\;\;\;\;\mbox{i.e.}\;\;\;\;\;
  V_1=V_2\left(\frac{1}{Z_2}+\frac{1}{Z_4}\right)Z_2
  =V_2\left(1+\frac{Z_2}{Z_4}\right)
  \]
  Apply KCL to node $V_1$ to get:
  \[
  \frac{V_{in}-V_1}{Z_1}+\frac{V_{out}-V_1}{R_f}+\frac{V_2-V_1}{Z_2}
  =\frac{V_1}{Z_3}
  \]
  Rearrange the terms, and replace $V_1$ by $V_2(1+Z_2/Z_4)$ to get
  \begin{eqnarray}
    &&\frac{V_{in}}{Z_1}+\frac{V_{out}}{R_f}+\frac{V_2}{Z_2}
    =V_1\left(\frac{1}{Z_1}+\frac{1}{R_f}+\frac{1}{Z_2}+\frac{1}{Z_3}\right)
    \nonumber\\
    &=&V_2\left(1+\frac{Z_2}{Z_4}\right)\left(\frac{1}{Z_1}
    +\frac{1}{R_f}+\frac{1}{Z_2}+\frac{1}{Z_3}\right)
    \nonumber
  \end{eqnarray}
  Further rearrange the terms and replace $V_2$ by $kV_{out}$ to get
  \[
  \frac{V_{in}}{Z_1}+\frac{V_{out}}{R_f}
  =kV_{out}\left[\left(1+\frac{Z_2}{Z_4}\right)\left(\frac{1}{Z_1}+\frac{1}{R_f}
    +\frac{1}{Z_2}+\frac{1}{Z_3}\right)-\frac{1}{Z_2}\right]
  \]
  Further rearrange the terms
  \begin{eqnarray}
    \frac{V_{in}}{Z_1}&=&kV_{out}\left[\left(1+\frac{Z_2}{Z_4}\right)
\left(\frac{1}{Z_1}
      +\frac{1}{R_f}+\frac{1}{Z_2}+\frac{1}{Z_3}\right)-\frac{1}{Z_2}\right]-\frac{V_{out}}{R_f}
    \nonumber \\
    &=&
    V_{out}\left\{k\left[\left(1+\frac{Z_2}{Z_4}\right)\left(\frac{1}{Z_1}
      +\frac{1}{R_f}+\frac{1}{Z_2}+\frac{1}{Z_3}\right)-\frac{1}{Z_2}\right]-\frac{1}{R_f}\right\}
    \nonumber 
  \end{eqnarray}
  Finally we get the frequency response function:
  \begin{eqnarray}
    H=\frac{V_{out}}{V_{in}}&=&
    \frac{1}{Z_1\left\{k\left[\left(1+\frac{Z_2}{Z_4}\right)\left(\frac{1}{Z_1}+\frac{1}{R_f}+\frac{1}{Z_2}+\frac{1}{Z_3}\right)-\frac{1}{Z_2}\right]-\frac{1}{R_f}\right\}}
    \nonumber \\
    &=&
    \frac{1/k}{Z_1\left[\left(1+\frac{Z_2}{Z_4}\right)\left(\frac{1}{Z_1}+\frac{1}{R_f}+\frac{1}{Z_2}+\frac{1}{Z_3}\right)-\frac{1}{Z_2}\right]-\frac{Z_1}{kR_f}}
    \nonumber \\
    &=&
    \frac{1/k}{1+\frac{Z_1}{R_f}+\frac{Z_1}{Z_3}+\frac{Z_2}{Z_4}+\frac{Z_1Z_2}{Z_4R_f}+\frac{Z_1}{Z_4}+\frac{Z_1Z_2}{Z_3Z_4}-\frac{Z_1}{kR_f}}
    \nonumber
  \end{eqnarray}
  Now if we let
  \[
  Z_1=R_1,\;\;\;\;\;Z_4=R_2,\;\;\;\;Z_2=\frac{1}{j\omega C_2},
  \;\;\;\;Z_3=\frac{1}{j\omega C_1}
  \]
  the frequency response function becomes
  \begin{eqnarray}
  H&=&\frac{\left(1+\frac{R_b}{R_a}\right)\frac{1}{R_1C_1}j\omega}{(j\omega)^2+\left(\frac{1}{R_2C_1}+\frac{1}{R_2C_2}+\frac{1}{R_1C_1}-\frac{R_b}{R_aR_fC_1}\right)j\omega+\frac{R_1+R_f}{R_fR_1R_2C_1C_2}}
  \nonumber\\
  &=&\frac{Aj\omega}{(j\omega)^2+\Delta\omega j\omega+\omega_n^2},
  \;\;\;\;\;\;\;\;\;
  A=(1+R_b/R_a)/R_1C_1
  \nonumber
  \end{eqnarray}
  This is a band-pass filter with the peak frequency equal to the natural
  frequency:
  \[
  \omega_n=\sqrt{\frac{R_1+R_f}{R_fR_1R_2C_1C_2}}
  \]
  the bandwidth
  \[
  \Delta\omega=\frac{1}{R_2C_1}+\frac{1}{R_2C_2}+\frac{1}{R_1C_1}-\frac{R_b}{R_aR_fC_1}
  \]
  The gain of the filter is controlled by $1+R_b/R_a$.
  
\end{itemize}


\subsection{The Twin-T notch (band-stop) filter}

{\bf The twin-T filter}

\htmladdimg{../figures/TwinT.png}

%\begin{comment}
The twin-T network is composed of two T-networks: 
\begin{itemize}
\item The RCR network is formed by two resistors $R_1=R_2=R$ and one
  capacitor $C_3=2C$. 

  \begin{comment}
  This T (or Y) network can be converted to a $\pi$ (or $\Delta$) network (see 
  \htmladdnormallink{here}{http://fourier.eng.hmc.edu/e84/lectures/ch2/node3.html}) of three branches:
  \[
  Z'_1=Z'_2=R+\frac{1}{2j\omega C}+\frac{R/2j\omega C}{R}
  =R+\frac{1}{j\omega C}=\frac{j\omega RC+1}{j\omega C}
  \]
  \[
  Z'_3=R+R+\frac{R^2}{1/2j\omega C}=2R+2R^2j\omega C=2R(1+j\omega RC)
  \]
  The frequency response function of the voltage divider formed by 
  $Z'_3$ and $Z'_2$ is:
  \begin{eqnarray}
  H'(j\omega)&=&\frac{Z'_2}{Z'_2+Z'_3}
  =\frac{(1+j\omega RC)/j\omega C}{(1+j\omega RC)/j\omega C+2R(1+j\omega RC)}
  \nonumber\\
  &=&\frac{1}{1+2j\omega RC}=\frac{1}{1+2j\omega\tau}
  \nonumber
  \end{eqnarray}
  where $\tau=RC$. This is a first-order low-pass filter with cut-off
  frequency at $\omega_c=1/2\tau$:
  \[
  |H'(j\omega)|=\left\{\begin{array}{ll}1 & \omega=0\\1/\sqrt{2}& 
  \omega=1/2\tau\\ 0&\omega\rightarrow\infty\end{array}\right.
  \]
  \end{comment}

\item The CRC network is formed by two capacitors $C_1=C_2=C$ and one
  resistor $R_3=R/2$. 

  \begin{comment}
  This T (or Y) network can be converted to a
  $\pi$ (or $\Delta$) network of three branches:
  \[
  Z''_1=Z''_2=\frac{R}{2}+\frac{1}{j\omega C}+\frac{R/2j\omega C}{1/j\omega C}
  =R+\frac{1}{j\omega C}  =\frac{j\omega RC+1}{j\omega C}
  \]
  \[
  Z''_3=\frac{1}{j\omega C}+\frac{1}{j\omega C}+\frac{1/(j\omega C)^2}{R/2}
  =\frac{2}{j\omega C}+\frac{2}{R(j\omega C)^2}
  =\frac{2(1+j\omega RC)}{R(j\omega C)^2}
  \]
  The frequency response function of the voltage divider formed by $Z''_3$ 
  and $Z''_2$ is:
  \begin{eqnarray}
  H''(j\omega)&=&\frac{Z''_2}{Z''_2+Z''_3}
  =\frac{(1+j\omega RC)/j\omega C}
  {(1+j\omega RC)/j\omega C+2(1+j\omega RC)/R(j\omega C)^2}
  \nonumber \\
  &=&\frac{j\omega RC}{j\omega RC+2}=\frac{j\omega\tau}{j\omega\tau+2}
  =\frac{1}{1-j2/\omega\tau}
  \nonumber 
  \end{eqnarray}
  This is a first-order high-pass filter with cut-off frequency at
  $\omega_c=2/\tau$:
  \[
  |H'(j\omega)|=\left\{\begin{array}{ll}0 & \omega=0\\1/\sqrt{2}& 
  \omega=2/\tau\\ 1&\omega\rightarrow\infty\end{array}\right.
  \]
  \end{comment}

\end{itemize}

\begin{comment}

As these two $\pi$-networks are combined in parallel, they form a single
$\pi$-network with three branches $Z_1=Z'_1||Z''_1$, $Z_2=Z'_2||Z''_2$, 
and $Z_3=Z'_3||Z''_3$:
\[
Z_1=Z'_1||Z''_1=Z_2=Z'_2||Z''_2=\frac{1}{2}\left(R+\frac{1}{j\omega C}\right)
\]
\[
  Z_3=Z'_3||Z''_3=\frac{Z'_3 Z''_3}{Z'_3+Z''_3}
  =\frac{2R(1+j\omega RC)}{1+(j\omega RC)^2}
\]
The frequency response function of this $\pi$-network (a voltage divider) is:
\begin{eqnarray}
  H(j\omega)&=&\frac{Z_2}{Z_2+Z_3}=\frac{R+1/j\omega C}
  {R+1/j\omega C+4R(1+j\omega RC)/(1+(j\omega RC)^2)}
  \nonumber \\
  &=&\frac{(1+j\omega RC)/j\omega C}{(1+j\omega RC)/j\omega C+4R(1+j\omega RC)/(1+(j\omega RC)^2)}
  \nonumber \\
  &=&\frac{1/j\omega C}{1/j\omega C+4R/(1+(j\omega RC)^2)}
  =\frac{1}{1+4j\omega RC/(1+(j\omega RC)^2)}
  \nonumber \\
  &=&\frac{1+(j\omega\tau)^2}{1+(j\omega\tau)^2+4j\omega\tau}
  =\frac{(j\omega)^2+(1/\tau)^2}{(j\omega)^2+4j\omega/\tau+(1/\tau)^2}
  \nonumber
\end{eqnarray}

\end{comment}

When the output is open-circuit, i.e., $Z_L=\infty$, the frequency 
response function of the twin-T network can be found to be
(see \htmladdnormallink{here}{../TwinT/TwinT.html}):
\begin{eqnarray}
H(j\omega)&=&\frac{v_{out}}{v_{in}}
=\frac{(j\omega)^2+\omega_n^2}{(j\omega)^2+4\omega_nj\omega+\omega_n^2}
\nonumber\\
&=&\frac{(j\omega)^2+\omega_n^2}{(j\omega)^2+\omega_nj\omega/Q+\omega_n^2}
=\frac{\omega_n^2-\omega^2}{\omega_n^2-\omega^2+j\omega\Delta\omega}
\nonumber
\end{eqnarray}
where
\[
\omega_n=\frac{1}{RC}=\frac{1}{\tau}
\]
$Q=1/4=0.25$ is the quality factor, and $\Delta\omega=\omega_n/Q=4\omega_n$ 
is the bandwidth of the filter. This twin-T network is a band-stop filter 
(notch filter) which attenuates the frequency $\omega_n=1/\tau$ to zero:
\[
|H(j\omega)|=\left\{\begin{array}{ll} 
H(0)=\omega_n^2/\omega_n^2=1 & \omega=0 \\
H(j\omega_n)=0 & \omega=\omega_n=1/\tau\\
H(\infty)=\lim\limits_{\omega\rightarrow\infty}H(j\omega)=\omega^2/\omega^2=1 & \omega\rightarrow \infty
\end{array}\right.
\]

\begin{comment}
This result can also be reached by noticing the following
\[
H'(j\omega)\big|_{\omega=1/\tau}=\frac{1}{1+j2},\;\;\;\;\;\;\;
H''(j\omega)\big|_{\omega=1/\tau}=\frac{1}{1-j2}
\]
As they are equal in magnitude but opposite in phase, their outputs 
cancel each other to produce zero output.
\end{comment}

When this notch filter is used in a negative feedback loop of an 
amplifier, it becomes an oscillator.

\htmladdimg{../figures/TwinTPlots1.png}

{\bf The active twin-T filter}

The bandwidth $\Delta\omega=\omega_n/Q=4\omega_n$ may be too large for 
most applications due to the small quality factor $Q=1/4$. To overcome 
this problem, an active filter containing two op-amp followers (with 
unity gain $A=1$) can be used to introduce a positive feedback loop as 
shown below:

\htmladdimg{../figures/TwinTActive.png}

Now the common terminal of the twin-T filter is no longer grounded, 
instead it is connected a potentiometer, a voltage divider composed 
of $R_4$ and $R_5$, to form a feedback loop by which a fraction of the
output $V_{out}$ is fed back:
\[
V_1=\frac{R_5}{R_4+R_5}\;V_{out}=\alpha v_{out}
\]
where $\alpha=R_5/(R_4+R_5)$, i.e., $1-\alpha=R_4/(R_4+R_5)$.

%\begin{comment}

The input and output of the twin-T network are respectively $V_{in}-V_1$ and
$V_{out}-V_1$, and they are now related by the frequency response function 
$H(j\omega)$ of the twin-T network:
\[
V_{out}-V_1=H(j\omega)(V_{in}-V_1)
\]
Rearranging and substituting $V_1=V_{out}\;R_5/(R_4+R_5)$, we get
\begin{eqnarray}
  H(j\omega)V_{in}&=&V_{out}+(H(j\omega)-1) V_1
  =V_{out}+(H(j\omega)-1)\frac{R_5}{R_4+R_5}V_{out}
  \nonumber \\
  &=&\left(1+(H(j\omega)-1)\;\frac{R_5}{R_4+R_5}\right)\,V_{out}
  =\frac{R_4+H(j\omega)R_5}{R_4+R_5}\;V_{out}
  \nonumber
\end{eqnarray}
Now the frequency response function of this active filter with feedback 
can be found to be
\[
H_{active}(j\omega)=\frac{V_{out}}{V_{in}}
=\frac{H(j\omega)(R_4+R_5)}{R_4+H(j\omega)R_5}
\]
Substituting $H(j\omega)=((j\omega)^2+\omega_n^2)/((j\omega)^2+4\omega_n j\omega
+\omega_n^2)$ we get
\begin{eqnarray}
  H_{active}(j\omega)&=&\frac{(\omega_n^2-\omega^2)(R_4+R_5)}
  {R_4(\omega_n^2-\omega^2+4\omega_n j\omega)+(\omega_n^2-\omega^2)R_5}
  \nonumber \\
  &=&\frac{(\omega_n^2-\omega^2)(R_4+R_5)}
  {4\omega_nR_4 j\omega+(\omega_n^2-\omega^2)(R_4+R_5)}
  \nonumber \\
  &=&\frac{\omega_n^2-\omega^2}{j\omega 4\omega_n R_4/(R_4+R_5)+\omega_n^2-\omega^2}
  \nonumber \\
  &=&\frac{\omega_n^2-\omega^2}{\omega_n^2-\omega^2+\omega_n/Q_{active} j\omega}
  =\frac{\omega_n^2-\omega^2}{\omega_n^2-\omega^2+\Delta\omega_{active} j\omega}
  \nonumber
\end{eqnarray}
where 
\[
Q_{active}=\frac{R_4+R_5}{4R_4},\;\;\;\;\;\;
\Delta\omega_{active}=\frac{\omega_n}{Q}_{active}
\]
are respectively the quality factor and the bandwidth of the active 
filter with feedback. 
%\end{comment}

It can be shown (see \htmladdnormallink{here}{../TwinT/TwinT.html})
that the frequency response function of this active twin-T filter is
\begin{eqnarray}
H_{active}(j\omega)
&=&\frac{\omega_n^2-\omega^2}{\omega_n^2+4j\omega\omega_n R_4/(R_4+R_5)-\omega^2}
\nonumber\\
&=&\frac{\omega_n^2-\omega^2}{\omega_n^2-\omega^2+\omega_n/Q_{active} j\omega}
=\frac{\omega_n^2-\omega^2}{\omega_n^2-\omega^2+\Delta\omega_{active} j\omega}
\nonumber
\end{eqnarray}
where 
\[
Q_{active}=\frac{R_4+R_5}{4R_4},\;\;\;\;\;\;
\Delta\omega_{active}=\frac{\omega_n}{Q}_{active}
\]
are respectively the quality factor and the bandwidth of the active 
filter with feedback. By changing $R_4$ and $R_5$, the bandwidth 
$\Delta\omega_{active}$ can be adjusted. In particular, 
\begin{itemize}
\item when $R_5=0$, $V_1=0$ (no feedback), $Q_{active}=1/4=Q$, 
  $\Delta\omega=\omega_n/Q=4\omega_n$; 
\item when $R_4=0$, $V_1=V_{out}$ (one hundred percent feedback), 
  $Q_{active}=\infty$, $\Delta\omega_{active}=\omega_n/Q_{active}=0$.
\end{itemize}


{\bf The bridged T filter}

If in the RCR T-network the vertical capacitor branch is dropped, 
i.e., $C=0$, the twin-T network becomes a bridged T network. Now
we have $Z'_3=2R$, while the CRC T-network is still the same with 
$Z''_3=2(1+j\omega RC)/R(j\omega C)^2$, we get:
\[
  Z_3=Z'_3||Z''_3=\frac{Z'_3 Z''_3}{Z'_3+Z''_3}
  =\frac{2R(1+j\omega RC)}{1+j\omega RC+(j\omega RC)^2}
\]
The frequency response function of this bridged T network (a voltage 
divider) is:
\begin{eqnarray}
  H(j\omega)&=&\frac{Z_2}{Z_2+Z_3}=\frac{R+1/j\omega C}{R+1/j\omega C+2R(1+j\omega RC)/(1+j\omega RC+(j\omega RC)^2)}
  \nonumber \\
  &=&\frac{1/C}{1/j\omega C+2R/(1+j\omega RC+(j\omega RC)^2)}
  =\frac{1}{1+2j\omega RC/(1+j\omega RC+(j\omega RC)^2)}
  \nonumber \\
  &=&\frac{1+j\omega RC+(j\omega RC)^2}{1+3j\omega RC+(j\omega RC)^2}
  =\frac{(j\omega)^2+j\omega /RC+1/(RC)^2}{(j\omega)^2+3j\omega /RC+1/(RC)^2}
  \nonumber
\end{eqnarray}
We let $\omega_n=1/RC$, and express both the numerator and the
denominator in the canonical form as
\[
H(j\omega)=\frac{(j\omega)^2+\omega_n j\omega +\omega_n^2}{(j\omega)^2+3\omega_nj\omega +\omega_n^2}
=\frac{(j\omega)^2+\Delta\omega_n j\omega+\omega_n^2}{(j\omega)^2+\Delta\omega_dj\omega +\omega_n^2}
=\frac{\omega_n^2-\omega^2+\Delta\omega_n j\omega }{\omega_n^2-\omega^2+\Delta\omega_dj\omega}
\]
where 
\[
\Delta\omega_n=\omega_n,\;\;\;\;\;\;\;\Delta\omega_d=3\omega_n
\]
are the bandwidth of the 2nd-order systems of the numerator and the
denominator, respectively. 
\begin{itemize}
\item If $\omega=0$, $H(j\omega)=H(0)=1$
\item If $\omega\rightarrow \infty$, $H(j\omega)=1$
\item If $\omega=\omega_n=1/RC$, $H(j\omega_n)=1/3$
\end{itemize}
We see that this is a band-stop filter.

%\end{comment}



\subsection*{Wien bridge}

The Wien bridge is a particular type of the Wheatstone bridge of which
two of the four arms are composed of a capacitor as well as a resistor
in parallel and series:

%\htmladdimg{../figures/WienBridge.png}
\htmladdimg{../figures/WienBridge2.png}

For this bridge to balance, the ratios of the left and right branches
should be the same:
\[
\frac{R_3}{R_4}=\frac{R_2+1/j\omega C_2}{R_1||1/j\omega C_1}
=\frac{(j\omega R_1C_1+1)(j\omega R_2C_2+1)}{j\omega R_1C_2}
=\frac{1-\omega^2R_1R_2C_1C_2+j\omega(R_1C_1+R_2C_2)}{j\omega R_1C_2}
\]
For this equation to hold, the right-hand side needs to be real, i.e.,
\[
1-\omega^2R_1R_2C_1C_2=0,\;\;\;\;\;\mbox{i.e.,}\;\;\;\;\;
\omega=\frac{1}{\sqrt{R_1R_2C_1C_2}}
\]
and the equation above becomes
\[
\frac{R_3}{R_4}=\frac{R_1C_1+R_2C_2}{R_1C_2}=\frac{C_1}{C_2}+\frac{R_2}{R_1}
\]
In particular, if $R_1=R_2=R$ and $C_1=C_2=C$, we have:
\[
\omega=\frac{1}{\sqrt{R_1R_2C_1C_2}}
=\frac{1}{\sqrt{R^2C^2}}=\frac{1}{RC}
\]
and
\[
\frac{R_3}{R_4}=\frac{C_1}{C_2}+\frac{R_2}{R_1}=1+1=2,
\;\;\;\;\;\;\mbox{i.e.}\;\;\;\;\;R_4=2R_3
\]


{\bf Wien-Robinson Filter}

\htmladdimg{../figures/WienRobinson2.png}

\begin{itemize}
\item
  \[
  \frac{V_{in}}{R_4}+\frac{V_{out}}{R_3}+\frac{V_1}{R_2}=0\;\;\;\;\;\;\;\;(1)
  \]
\item
  \[
  V_2=\frac{R+1/j\omega C}{R+1/j\omega C+R/j\omega C/(R+1/j\omega C)}V_1
  =\frac{(j\omega\tau+1)^2}{(j\omega\tau+1)^2+j\omega\tau}V_1,
  \]
  where $\tau=RC$, i.e.,
  \[
  V_1=\left(1+\frac{j\omega\tau}{(j\omega\tau+1)^2}\right)V_2,\;\;\;\;\;\;(2)
  \]
\item
  \[
  \frac{V_1-V_2}}{R_1}+\frac{V_{out}-V_2}{2R_1}=0,
    \;\;\;\;\;\;\;\mbox{i.e.}\;\;\;\;\;\; V_{out}=3V_2-2V_1,\;\;\;\;\;\;(3)
  \]
\end{itemize}
\[
V_{out}=3V_2-2V_1=3V_2-2\left(1+\frac{j\omega\tau}{(j\omega\tau+1)^2}\right)V_2
=V_2-\frac{j\omega 2\tau}{(j\omega\tau+1)^2}\;V_2
=\frac{(j\omega\tau)^2+1}{(j\omega\tau+1)^2}\;V_2
\]
or
\[
V_2=\frac{(j\omega\tau+1)^2}{(j\omega\tau)^2+1}V_{out}
\]
Substituting this into (2) we get
\[
V_1=\left(1+\frac{j\omega\tau}{(j\omega\tau+1)^2}\right)V_2
=\left(1+\frac{j\omega\tau}{(j\omega\tau+1)^2}\right)\frac{(j\omega\tau+1)^2}{(j\omega\tau)^2+1}V_{out}
=\frac{(j\omega\tau)^2+3j\omega\tau+1}{(j\omega\tau)^2+1}\;V_{out}
\]
Substituting this into (1) we get
\[
\frac{V_{in}}{R_4}+\frac{V_{out}}{R_3}
+\frac{(j\omega\tau)^2+3j\omega\tau+1}{(j\omega\tau)^2+1}\;\frac{V_{out}}{R_2}=0
\]
We rearrange to get
\[
\frac{V_{in}}{R_4}=-\left(\frac{1}{R_3}
+\frac{(j\omega\tau)^2+3j\omega\tau+1}{(j\omega\tau)^2+1}\;\frac{1}{R_2}\right)V_{out}
\]
\[
\frac{R_2}{R_4}=-\left(\frac{R_2}{R_3}
+\frac{(j\omega\tau)^2+3j\omega\tau+1}{(j\omega\tau)^2+1}\right)
\frac{V_{out}}{V_{in}}
\]
\[
H(j\omega)=\frac{V_{out}}{V_{in}}
=-\frac{R_2R_3}{R_4(R_2+R_3)}\;\frac{(j\omega\tau)^2+1}{(j\omega\tau)^2+3j\omega\tau R_3/(R_2+R_3)+1}
\]
\[
H(j\omega)=A\;\frac{(j\omega)^2+\omega_n^2}{(j\omega)^2+\Delta\omega/Q\;j\omega+\omega_n^2}
=\left\{\begin{array}{ll}A&\omega=0\\0&\omega=\omega_n=1/\tau
\\A&\omega\rightarrow\infty\end{array}\right.
\]
This is a band-stop filter with passband gain $A=-R_2R_3/R_4(R_2+R_3)=(R_2//R_3)/R_4$,
stop-band $\omega_n=1/\tau$ and $Q=(R_2+R_3)/3R_3$.


Further reading for 
\begin{itemize}
\item \htmladdnormallink{Oscillator circuits}{http://www.ti.com/lit/an/sloa060/sloa060.pdf}
\item \htmladdnormallink{Active filters}{http://www.ti.com/lit/ml/sloa088/sloa088.pdf}
\end{itemize}

\begin{comment}
\subsection{Colpitts Oscillators}

An oscillator is a feedback system composed of a forward path with
gain $G(j\omega)$ and a feedback path with gain $F(j\omega)$:

\htmladdimg{../figures/OscillatorModel.png}

For the system to oscillate, the feed back needs to be positive for
the feedback signal $FY$ to positively reinforce the signal going 
through the forward path in order to sustain the output $Y$ with
zero input $X=0$. Specifically, the output $Y$ and the input $X$ 
of a feedback system are related by
\[
Y=G(X+FY)=GX+GFY,\;\;\;\;\;\;\;\;\;\frac{Y}{X}=H=\frac{G}{1-GF},
\;\;\;\;\;\;\;Y=HX=\frac{G}{1-GF}\,X
\]
where $GF$ is the open-loop gain and $H$ is the closed-loop gain.
For this system to oscillate, i.e., for it to produce an output with
zero input, its closed-loop gain needs to be infinite, i.e., its 
open-loop gain $GF$ need to be real, with zero phase $\angle(GF)=0$
and unit gain $|GF|=1$. 

Oscillation is desirable if the system is used as a sinusoidal source, 
but it is undesirable if the system is a amplifier or part of a
control system which needs to be stable without oscillation.

%(In practice, $|GF|\ge 1$ to compensate energy attenuation in the physical system.)

There exist many different configurations of oscillators based on a
single transistor. Shown below are three typical Colpitts oscillators:
common-base (CB, left), common emitter (CE, middle), and common 
collector (CC, right). All such circuits contain a ``tank'' LC circuit
composed of an inductor $L$ in parallel with $C_1$ and $C_2$ in series, 
with a resonant frequency 
\[
\omega_0=\frac{1}{\sqrt{LC_s}},\;\;\;\;\;\;\mbox{where}\;\;\;\;\;
C_s=\left(\frac{1}{C_1}+\frac{1}{C_2}\right)^{-1}
\]
where $C_s$ is the equivalent capacitance of the series combination 
of $C_1$ and $C_2$. All other $C$s (without a subscript) are coupling
capacitors that have a large enough capacitance and can therefore be 
treated as short circuit for AC signals. 

%\htmladdimg{../figures/ColpittsOscillators.png}
\htmladdimg{../figures/Colpitts3a.png}

Here are the requirements for these circuits to oscillate:
\begin{enumerate}
\item an LC tank tuning circuit that generates sinusoidal oscillation 
  at its resonant frequency $\omega_0=1/\sqrt{LC_s}$
\item a positive feedback loop that sustains the oscillation.
\end{enumerate}
How each of these circuits works can be qualitatively understood as 
below:
\begin{itemize}
\item CB with the base AC grounded: The collector voltage $V_c$ is the 
  output, a fraction of which at the middle point between the two 
  capacitors, ``tap point'', is fed-back to the emitter to a positive 
  feedback loop:
  \[
  V_c\uparrow\Longrightarrow V_e\uparrow\Longrightarrow \V_{be}\downarrow
  \Longrightarrow I_b\downarrow \Longrightarrow I_c\downarrow
  \Longrightarrow V_c\uparrow
  \]
\item CE with the emitter AC grounded: The collector voltage $V_c$ is 
  the output, which is fed-back through the LC tank circuit to the base. 
  As the tap point is grounded, the sinusoidal voltage across the LC
  tank produces opposite voltage polarities at the far ends of $C_1$ 
  and $C_2$, i.e., $V_{C_1}=V_b$ and $V_{C_2}=V_c$ have opposite phases 
  and thereby form a positive feedback loop:  
  \[
  V_c\uparrow\Longrightarrow V_b\downarrow\Longrightarrow \V_{be}\downarrow
  \Longrightarrow I_b\downarrow \Longrightarrow I_c\downarrow
  \Longrightarrow V_c\uparrow
  \]
\item CC with the collector AC grounded: This a voltage follower circuit
  in which the emitter voltage $V_e$ is the output that follows the input
  voltage $V_b$. The feedback from the emitter through the LC tank circuit 
  to the base form a positive feedback loop:
  \[
  V_e\uparrow\Longrightarrow V_t\uparrow\Longrightarrow V_b\uparrow
  \Longrightarrow I_b\uparrow\Longrightarrow I_e\uparrow
  \Longrightarrow V_e\uparrow
  \]
  where $V_t$ is the voltage at the tap point.

\end{itemize}

More specifically, we consider the common-collector circuit as an example.
To find out why the circuit oscillates and the resonant frequency, we 
disconnect the base path of the circuit and consider the open-loop gain 
of $H=V_o/V_i$ of the feedback loop. We further model the transistor 
by a Thevenin voltage source $V_i$ in series with an internal $R$, as 
shown in the figure:

\htmladdimg{../figures/ColpittsModel2a.png}

As the load of the Thevenin source, the tank circuit receives an input 
$V_t$ at the tap point, and produces an output $V_o$ across the parallel
combination of $L$ and $C_1$ in series with $C_2$. Applying KCL at the tap 
point we get:
\[
\frac{V_t-V_i}{R}+\frac{V_t}{1/j\omega C_2}+\frac{V_t}{j\omega L+1/j\omega C_1}=0
\]
i.e.,
\[
V_t\left(\frac{1}{R}+j\omega C_2+\frac{j\omega C_1}{1-\omega^2LC_1}\right)
=\frac{V_i}{R}
\]
Solving for $V_t$ we get
\[
V_t=\frac{1}{R(\frac{1}{R}+j\omega C_2+\frac{j\omega C_1}
{1-\omega^2LC_1})}\;V_i
=\frac{1}{1+j\omega R(C_2+C_1/(1-\omega^2LC_1)}\;V_i
\]
which is maximized if the frequency is such that the imaginary part of 
the denominator is zero:
\[
C_2+\frac{C_1}{1-\omega_0^2LC_1}=0,\;\;\;\;\;\;\;\mbox{i.e.}\;\;\;\;\;
\omega_0=\frac{1}{\sqrt{LC_1C_2/(C_1+C_2)}}=\frac{1}{\sqrt{LC_s}}
\]
This frequency $\omega_0$ is the resonant frequency, at which the voltage 
$V_t$ become the same as the source voltage $V_t=V_i$. Also note that the 
impedance of the tank circuit as the load of the Thevenin source is
\[
Z_{tank}=Z_{C_2}||(Z_{C_1}+Z_L)=\frac{(1/j\omega C_1+j\omega L)/j\omega C_2}
{1/j\omega C_1+1/j\omega C_2+j\omega L}
=\frac{(1/j\omega C_1+j\omega L)/C_2}{1/C_1+1/C_2-\omega^2 L}
=\frac{(1/j\omega C_1+j\omega L)/C_2}{1/C_s-\omega^2 L}
\]
At the resonant frequency, the denominator becomes zeros and $Z_{tank}=\infty$,
i.e., there is no current drawn from the source by the tank circuit. 
Consequently, the voltage drop across $R$ is zero and the voltage received 
by the tank circuit is $V_t=V_i$. Now the output voltage $V_o$ can be found 
by voltage divider:
\[
V_t=\frac{Z_{C_2}}{Z_{C_1}+Z_{C_2}}\;V_o=\frac{C_1}{C_1+C_2}\,V_o,
\;\;\;\;\;\mbox{i.e.}\;\;\;\;\;\;\;\;
V_o=\frac{C_1+C_2}{C_1}\;V_t=\frac{C_1+C_2}{C_1}\;V_i
\]
The open-loop gain (from $V_i$ to $V_o$) is:
\[
H=\frac{V_o}{V_i}=\frac{C_1+C_2}{C_1}
\]

%\begin{comment}
\begin{eqnarray}
  V_o&=&\frac{j\omega L}{j\omega L+1/j\omega C}\;V_t
  =\frac{\omega^2LC_1}{\omega^2LC_1-1}\;V_t
  \nonumber \\
  &=&\left(\frac{1-\omega^2LC_1}{1-\omega^2LC_1+j\omega R(C_1+C_2-\omega^2LC_1C_2)}\right)\;
  \left(\frac{\omega^2LC_1}{\omega^2LC_1-1}\right)\;V_i
  \nonumber \\
  &=&\frac{-\omega^2LC_1}{1-\omega^2LC_1+j\omega R(C_1+C_2-\omega^2LC_1C_2)}\;V_i
  \nonumber 
\end{eqnarray}

The open-loop gain (from $V_i$ to $V_o$) is:
\[
H=\frac{V_o}{V_i}=\frac{-\omega^2LC_1}{1-\omega^2LC_1+j\omega R(C_1+C_2-\omega^2LC_1C_2)}
\]
At the resonant frequency $\omega_0$, the imaginary part is zero, we have
\[
H=\frac{-\omega^2LC_1}{1-\omega^2LC_1}
=\frac{\omega_0^2LC_1}{\omega_0^2LC_1-1}
=\frac{LC_1/LC_s}{LC_1/LC_s-1}
=\frac{C_1}{C_1-C_s}
=\frac{C_1+C_2}{C_1}
\]
%\end{comment}

We see that at the resonant frequency $\omega_0$, the magnitude of the open-loop
gain is $|H|>1$ and its phase is $\angle H=0$ (zero phase, i.e., $H$ is real).
As both conditions for oscillation are satisfied, the circuit is an oscillator
with frequency at $\omega_0=1/\sqrt{LC_s}$.

%\begin{comment}
http://seit.unsw.adfa.edu.au/staff/sites/hrp/teaching/Electronics4/docs/PLL/colpitts.pdf

http://users.ece.gatech.edu/mleach/ece3050/notes/osc/wienbr.pdf

http://www.ece.msstate.edu/~winton/classes/ece3144/labs/Exp10.pdf

http://www.drp.fmph.uniba.sk/ESM/twin.pdf
%\end{comment}

\end{comment}

\subsection{Butterworth filters}

  The transition between the pass-band and stop-band of a first order 
  filter with cut-off frequency $\omega_c=1/\tau$ is characterized by the 
  the slope of 20 dB per decade of frequency change. To achieve better 
  selectivity, we can cascade a set of $n$ such first order filters to
  form an nth order filter with a slope of 20n dB per decade.

  \htmladdimg{../figures/Cascade1stOrder.png}

  The FRF of a first-order low-pass filter of unit gain is:  
  \[
  H(j\omega)=\frac{1}{j\omega\tau+1}
  =\frac{1}{j\omega/\omega_c+1}=\frac{\omega_c}{j\omega+\omega_c}
  \]
  The FRF of $n$ such filters in series is (assuming they are well
  buffered with no loading effect):
  \[
  H(j\omega)=\left(\frac{\omega_c}{\omega_c+j\omega} \right)^n
  \]
  The cut-off frequency of this nth order filter $\omega_{cn}$ can be
  found by solving the following equation
  \[
  \left|\frac{\omega_c}{\omega_c+j\omega}\right|^n
  =\left|\frac{\omega_c}{\sqrt{\omega_c^2+\omega^2}}\right|^n
  =\frac{1}{\sqrt{2}}=2^{-1/2}
  \]
  to get
  \[
  \omega=\omega_{cn}=\omega_c\sqrt{2^{1/n}-1}
  \]

  {\bf Example:} Design an 4th order LP filter with 
  $\omega_{c4}=2\pi 1000=6.2832\times 10^3$. The cut-off frequency of the 
  first-order LP filter can be found to be
  \[
  \omega_c=\frac{\omega_{c4}}{\sqrt{2^{1/4}-1}}
  =\frac{6.2832\times 10^3}{0.435}=1.445\times 10^4
  \]
  The time constant of the first-order filter is 
  $\tau=RC=1/\omega_c=6.92\times 10^{-5}$. If $C=0.1\;\mu F=10^{-7}\;F$, 
  then
  \[
  R=\frac{\omega_c}{C}=\frac{6.92\times 10^{-5}}{10^{-7}}
  =6.92\times 10^2=692\;\Omega
  \]

  The {\em Butterworth filters} have the property that the passing
  band is flat. The magnitude of the FRF of an nth order low-pass 
  Butterworth filter with cut-off frequency $\omega_c$ is
  \[
  \left| H_{lp}(j\omega)\right|=\frac{1}{\sqrt{1+(\omega/\omega_c)^{2n}}}
  =\left\{\begin{array}{ll}1 & \omega=0\\1/\sqrt{2}&\omega=\omega_c\\
  0 & \omega=\infty\end{array}\right.
  \]
  where $\omega_c$ is the cut-off frequency at which $|H_{lp}(j\omega_c)|
  =1/\sqrt{2}$.
  The transition between the pass-band and stop-band is controlled by the
  order $n$. In general, higher order corresponds to more rapid transition. 
  Specially, when $n=0$, $n=1$, and $n=\infty$, we have
  \begin{itemize}
  \item $n=0$, $|H_{lp}(j\omega)|=1/\sqrt{2}$ is an all-pass filter.
  \item $n=1$, the Butterworth filter is the regular first-order filter:
    \[
    |H_{lp}(j\omega)|=\frac{1}{\sqrt{1+(\omega/\omega_c)^2}}
    =\frac{\omega_c}{\sqrt{\omega^2+\omega_c^2}}
    =\left|\frac{\omega_c}{j\omega+\omega_c}\right|
    \]
  \item $n=\infty$, the Butterworth filter becomes an ideal low-pass
    filter:
    \[
    |H_{lp}(j\omega)|=\frac{1}{\sqrt{1+(\omega/\omega_c)^\infty}}  
    =\left\{\begin{array}{ll}1&\omega<\omega_c\\0&\omega>\omega_c\end{array}
    \right.
    \]
  \end{itemize}
  
  The magnitude of the FRF of an nth order high-pass Butterworth  filter 
  with cut-off frequency $\omega_c$ is
  \[
  \left| H_{hp}(j\omega)\right|=\frac{1}{\sqrt{1+(\omega_c/\omega)^{2n}}}
  =\frac{(\omega/\omega_c)^n}{\sqrt{1+(\omega/\omega_c)^{2n}}}
  =\left\{\begin{array}{ll}0 & \omega=1\\1/\sqrt{2}&\omega=\omega_c\\
  1 & \omega=\infty\end{array}\right.
  \]

  \htmladdimg{../figures/ButterworthFilters1.png}

  Now we consider the implementation of a Butterworth filter. For 
  simplicity, in the following we assume the frequency is normalized 
  by the cut-off frequency $\omega_c$, i.e., $\omega=\omega/\omega_c$, 
  or $\omega_c=1$. Consider the low-pass case:
  \[
  |H(j\omega)|^2=H(j\omega)H(-j\omega)
  =\frac{1}{1+(\omega/\omega_c)^{2n}}
  =\frac{1}{1+\omega^{2n}}=\frac{1}{1+(\omega^2)^n}
  \]
  We need to get its FRF $H(j\omega)$ from its magnitude $|H(j\omega)|$.
  To do so, we first consider the transfer function (TF) $H(s)$ in the 
  s-domain corresponding to the FRF, which is the same as $H(j\omega)$ 
  when $s=j\omega$, i.e., $s^2=(j\omega)^2=-\omega^2$. Now the equation 
  above can be written as
  \[
  |H(j\omega)|^2=H(j\omega)\,H(-j\omega)
  =\frac{1}{1+(\omega^2)^n}=\frac{1}{1+(-s^2)^n}
  =\frac{1}{1+(-1)^ns^{2n}}=|H(s)|^2=H(s) H(-s)
  \]
  We further find the roots of the denominator, the poles of both $H(s)$
  and $H(-s)$, and separate them so that those on the left s-plane are 
  the poles of $H(s)$ (stable and causal), while those on the right 
  s-plane belong to $H(-s)$ (stable and anti-causal).

  The roots of the denominator can be found by solving the equation
  \[
  1+(-1)^ns^{2n}=0,\;\;\;\;\;\;\;\mbox{i.e.}\;\;\;\;\;
  \left\{\begin{array}{ll}
  1+s^{2n}=0 & \mbox{$n$ is even}\\
  1-s^{2n}=0 & \mbox{$n$ is odd}\end{array}\right.
  \]
  Solving these we get the $2n$ solutions on the unit circle in
  either of the two different forms depending on whether $n$ is even 
  or odd:
  \[
  \left\{\begin{array}{ll}
  s=(-1)^{1/2n}=(e^{j(2k+1)\pi})^{1/2n}=e^{j(2k+1)\pi/2n} & \mbox{$n$ is even}\\
  s=1^{1/2n}=(e^{j2k\pi})^{1/2n}=e^{jk\pi/n} & \mbox{$n$ is odd}
  \end{array}\right.\;\;\;\;\;\;(k=0,\cdots,2n-1)
  \]

  \htmladdimg{../figures/ButterworthInSPlane.png}

  \begin{itemize}
  \item If $n$ is even, 
    \[
    s_k=e^{j(2k+1)\pi/2n},\;\;\;\;\;(k=0,\cdots,2n-1)
    \]
    These $2n$ roots form $n$ complex conjugate pairs around the unit 
    circle of the s-plane. Corresponding to each of the $n$ roots 
    $s_k=e^{j(2 k+1)\pi/2n}$ ($k=0,\cdots,n-1$), there is another root
    $s_{2n-1-k}$ that is its complex conjugate:
    \[
    s_{2n-1-k}&=&e^{j(2(2n-1-k)+1)\pi/2n}=e^{j\pi(4n-2k-1)/2n}
    =e^{j2\pi} \;e^{-j(2k+1)\pi/2n} =e^{-j(2 k+1)\pi/2n}=s_k^*
    \]
    Also, for $s_k=e^{j(2k+1)\pi/2n}$ to be a pole of $H(s)$, it needs to be on 
    the left s-plane, i.e.,
    \[
    \frac{(2k+1)\pi}{2n}>\frac{\pi}{2},\;\;\;\;\;\;\mbox{i.e.}\;\;\;\;\;\;
    k>\frac{n-1}{2}
    \]
    Now $H(s)$ can be found in terms of its $n$ poles on the left s-plane:
    \begin{eqnarray}
    H(s)&=&\frac{1}{\prod_{k=\left\lceil (n-1)/2 \right\rceil}^{n-1} (s-s_k)(s-s_{2n-1-k})}
    =\frac{1}{\prod_{k=\left\lceil (n-1)/2 \right\rceil}^{n-1}(s-s_k)(s-s_k^*)}
    \nonumber \\
    &=&\frac{1}{\prod_{k=\left\lceil (n-1)/2 \right\rceil}^{n-1}(s^2-(s_k+s_k^*)s+s_ks_k^*)}
    =\frac{1}{\prod_{k=\left\lceil (n-1)/2 \right\rceil}^{n-1}(s^2-2\cos((2 k+1)\pi/2n)+1)}
    \nonumber 
    \end{eqnarray}
    where $\left\lceil x\right\rceil$ is the ceiling of $x$, and we have 
    used the fact that    
    \[
    s_k+s_k^*=e^{j(2 k+1)\pi/2n}+e^{-j(2 k+1)\pi/2n} =2\cos((2 k+1)\pi/2n),
    \;\;\;\;\;\;
    s_k s_k^*=1
    \]

  \item If $n$ is odd, 
    \[
    s_k=e^{j2k\pi/2n}=e^{jk\pi/n},\;\;\;\;\;(k=0,\cdots,2n-1)
    \]
    These $2n$ roots contain $s_0=e^{j0}=1$ and $s_n=e^{j\pi}=-1$, as well as
    $n-1$ complex conjugate pairs. Corresponding to each root $s_k=e^{jk\pi/n}$ 
    ($k=1,\cdots,n-1$), there is another root $s_{2n-k}$ that is its complex 
    conjugate:
    \[
    s_{2n-k}=e^{j(2n-k)\pi/n}=e^{-jk\pi/n}=s_k^*
    \]
    For $s_k$ to be a pole of $H(s)$, it needs to be on the left s-plane, i.e.,
    \[
    \frac{k\pi}{n}>\frac{\pi}{2},\;\;\;\;\;\;\mbox{i.e.}\;\;\;\;\;\;k>\frac{n}{2}
    \]
    Now $H(s)$ can be found in terms of its $n$ poles on the left s-plane:
    \begin{eqnarray}
    H(s)&=&\frac{1}{(s-s_n)\prod_{i=\left\lceil n/2 \right\rceil}^{n-1}(s-s_k)(s-s_{2n-k})}
    =\frac{1}{(s+1)\prod_{i=\left\lceil n/2 \right\rceil}^{n-1}(s-s_k)(s-s_k^*)}
    \nonumber \\
    &=&\frac{1}{(s+1)\prod_{i=\left\lceil n/2 \right\rceil}^{n-1}(s^2-(s_k+s_k^*)s+s_k s_k^*)}
    =\frac{1}{(s+1)\prod_{i=\left\lceil n/2 \right\rceil}^{n-1}(s^2-2\cos(k\pi/n)+1)}
    \nonumber
    \end{eqnarray}

  \end{itemize}

  Specifically, here we find the transfer function $H(s)$ of the nth order
  Butterworth filter for $n=2,\cdots,6$:
  \begin{itemize}
  \item $n=2$, $1+s^4=0$, $s^4=-1=e^{j(2k+1)\pi}$, the four roots are
    $s_k=e^{j(2k+1)\pi)/4}$ ($k=0,\cdots,3$):
    \[
    s_0=e^{j\pi/4}=\frac{1+j}{\sqrt{2}},\;\;\;\;
    s_1=e^{j3\pi/4}=\frac{-1+j}{\sqrt{2}},\;\;\;\;
    s_2=e^{j5\pi/4}=\frac{-1-j}{\sqrt{2}},\;\;\;\;
    s_3=e^{j7\pi/4}=\frac{1-j}{\sqrt{2}}
    \]
    out of which $s_1$ and $s_2=s_1^*$ on the left s-plane are the roots of 
    $H(s)$:
    \begin{eqnarray}
    H(s)&=&\frac{1}{(s-s_1)(s-s_2)}=\frac{1}{(s-(-1+j)/\sqrt{2})(s-(-1-j)/\sqrt{2})}
    \nonumber \\
    &=&\frac{1}{s^2+\sqrt{2}s+1}
    \nonumber
    \end{eqnarray}
    Note that the coefficient of the first order term is $-2\cos(3\pi/2)=\sqrt{2}$.
  \item $n=3$, $1-s^6=0$, $s^6=1=e^{j2k\pi}$, the six roots are
    $s_k=e^{j k2\pi/6}=e^{j k\pi/3}$ ($k=0,\cdots,5$)
    \[
    s_0=e^{j0}=1,\;\;\;\;
    s_1=e^{j\pi/3}=\frac{1+j\sqrt{3}}{2},\;\;\;\;s_2=e^{j2\pi/3}=\frac{-1+j\sqrt{3}}{2}
    \]
    \[
    s_3=e^{j3\pi/3}=e^{j\pi}=-1,\;\;\;\;
    s_4=e^{j4\pi/3}=\frac{-1-j\sqrt{3}}{2},\;\;\;\;s_5=e^{j5\pi/3}=\frac{1-j\sqrt{3}}{2}
    \]
    out of which $s_2$, $s_3=-1$, and $s_4=s_2^*$ on the left s-plane 
    are the roots of $H(s)$:
    \begin{eqnarray}
    H(s)&=&\frac{1}{(s-s_2)(s-s_3)(s-s_4)}
    =\frac{1}{(s+1)(s-(-1+j\sqrt{3})/2)(s-(-1-j\sqrt{3})/2)}
    \nonumber \\
    &=&\frac{1}{(s+1)(s^2+s+1)}
    \nonumber
    \end{eqnarray}
    Note that the coefficient of the first order term is $-2\cos(2\pi/3)=1$.
  \item $n=4$, $1+s^8=0$, $s^8=-1=e^{j(2k+1)\pi}$, the eight roots are
    $s_k=e^{j(2k+1)pi/8}$ ($k=0,\cdots,7$).
    Evaluating $-2\cos((2k+1)\pi/2n)$ for $k=2$ and $k=3$, we get the
    coefficients of the two first order terms
    $-2\cos(5\pi/8)=0.7654$ and $-2\cos(7\pi/8)=1.8478$.
    \[
    H(s)=\frac{1}{(s^2+0.765 s+1)(s^2+1.848 s+1)}
    \]
  \item $n=5$, $1-s^{10}=0$, $s^{10}=1=e^{j2k\pi}$, the 10 roots are
    \[
    s_k=e^{j(2k\pi)/10}=e^{j(k\pi)/5},\;\;\;\;\;\;(k=0,\cdots,9)
    \]
    Evaluating $-2\cos(k\pi/n)$ for $k=2$ and $k=3$, we get the
    coefficients of the two first order terms
    $-2\cos(2\pi/5)=0.618$ and $-2\cos(3\pi/5)=1.618$, and we get
    \[
    H(s)=\frac{1}{(s+1)(s^2+0.618 s+1)(s^2+1.618 s+1)}
    \]
  \item $n=6$, $1+s^{12}=0$, $s^{12}=-1=e^{j(2k+1)\pi}$, the 12 roots are
    \[
    s_k=e^{j(2k+1)\pi)/12},\;\;\;\;\;\;(k=0,\cdots,11)
    \]
    Evaluating $-2\cos((2k+1)\pi/2n)$ for $k=3$, $k=4$, and $k=5$,
    we get the coefficients of the three first order terms
    $-2\cos(7\pi/12)=0.5176$, $-2\cos(9\pi/12)=1.4142$, and
    $-2\cos(9\pi/12)=1.319$.
    \[
    H(s)=\frac{1}{(s^2+0.5176 s+1)(s^2+1.4142 s+1)(s^2+1.9319 s+1)}
    \]
  \end{itemize}
  In summary, we see that a Butterworth filter can be implemented as
  a cascade of second order systems in the form of $1/(s^2+a\,s+1)$ 
  if $n$ is even, and an additional first order system in the form of
  $1/(s+1)$ if $n$ is odd. The block diagrams below are for the 5th 
  and 6th order Butterworth filters:

  \htmladdimg{../figures/ButterworthDiagram.png}

  The first order filter in the cascade of the Butterworth filter 
  can be realized by the first order op-amp low-pass circuit shown
  above with
  \[
  H(s)=\frac{1/\tau}{s+1/\tau}=\frac{\omega_c}{s+\omega_c}=\frac{1}{s+1}
  \]
  where $\omega_c=1/\tau=1/RC$. If we let $R=1$, we get $C=1/\omega_c=1$. 

  The second order systems in the cascade can be implemented as a 
  Sallen-Key low-pass filter with 
  \[
  H(s)=\frac{1/R_1C_1R_2C_2}{s^2+s(R_1+R_2)/R_1R_2C_1+1/R_1C_1R_2C_2}
  =\frac{1}{s^2+\Delta\omega s+\omega_n^2}=\frac{1}{s^2+a s+1}
  \]
  where $\omega_n^2=1/R_1R_2C_1C_2$. If we let $R_1=R_2=1$ for simplicity, 
  we get
  \[
  2/C_1=a,\;\;\;\;\;1/C_1C_2=\omega_n^2=1
  \]
  Solving these we get
  \[
  C_1=2/a,\;\;\;\;\; C_2=1/C_1=a/2
  \]

  A High-pass Butterworth filter can be similarly implemented with the only
  difference that all first and second order systems in the cascade are
  high-pass filters
  \[
  H(s)=\frac{s}{s+\omega_c}=\frac{s}{s+1},\;\;\;\;\;\;\;
  H(s)=\frac{s^2}{s^2+\Delta\omega s+\omega_c^2}=\frac{s^2}{s^2+a s+1}
  \]
  so that the transfer function of the cascade is high-pass filter:
  \[
  H(s)=\left\{\begin{array}{cc}s^n/(1+s^{2n}) & \mbox{$n$ is even}\\
  s^n/(1-s^{2n}) & \mbox{$n$ is odd}\end{array}\right.
  \]

  To convert the results obtained above for normalized cut-off frequency 
  $\omega_n=1$ to unnormalized cut-off frequency $\omega_n\ne 1$, all we 
  need to do is to scale all capacitances $C$ to $C'=C/\omega_c$. The 
  capacitor in the first order filter becomes $C'=C/\omega_c$ so that
  $1/RC'=\omega_c/C'=\omega_c$; while the two capacitors in the second order
  filter become $C_1'=C_1/\omega_n$ and $C_2'=C_2/\omega_n$ so that
  $1/C_1'C_2'=\omega_n^2/C_1C_2=\omega_n^2$.

\subsection{Higher order systems}

Higher than first order systems can be built with multiple integrators, 
as shown here for a third order system:

\htmladdimg{../figures/opam7.gif}

From the diagram, we can get
\[
\left\{ \begin{array}{l}
	Y_3(s)=Y_2(s)/s \Longrightarrow Y_2(s)=Y_3(s)s	\\
	Y_2(s)=Y_1(s)/s \Longrightarrow Y_1(s)=Y_2(s)s=Y_3(s)s^2	\\
	Y_1(s)=Y_0(s)/s \Longrightarrow Y_0(s)=Y_1(s)s=Y_3(s)s^3	
\end{array} \right.
\]
But we also have
\[	Y_0(s)=X(s)-(k_1Y_1(s)+k_2Y_2(s)+k_3Y_3(s))	\]
i.e., 
\[	X(s)=Y_0(s)+k_1Y_1(s)+k_2Y_2(s)+k_3Y_3(s)=(s^3+k_1s^2+k_2s+k_3) Y_3(s)	\]
we get the transfer function
\[
	H(s)=\frac{Y_3(s)}{X(s)}=\frac{1}{s^3+k_1s^2+k_2s+k_3}
\]


{\bf Second order system by 2 integrators}

\htmladdimg{../figures/opam8.gif}

From the diagram, we can get
\[
\left\{ \begin{array}{ll}
	Y_2(s)=-c_2Y_1(s)/s  \Longrightarrow  Y_1(s)=-sY_2(s)/c_2 \\
	Y_1(s)=-c_1Y_0(s)/s  \Longrightarrow Y_0(s)=-sY_1(s)/c_1=s^2Y_2(s)/c_1c_2 \\
	Y_0(s)=k_0 X(s)+k_1Y_1(s)+k_2Y_2(s) 
	\end{array} \right.
\]
substituting the first two equations into the last one, we get
\[	\frac{s^2}{c_1c_2} Y_2(s)=k_0X(s)+k_1(-\frac{s}{c_2})Y_2(s)+k_2Y_2(s) \]
from which we obtain the transfer function as
\[
H(s)=\frac{Y_2(s)}{X(s)}=\frac{k_o}{\frac{s^2}{c_1c_2}+\frac{s}{c_2}s-k_2}
	=\frac{k_oc_1c_2}{s^2+k_1c_1s-c_1c_2k_2}
\]
which is a second order system. In particular, if $c_1=c_2=c$, we have
\[
	H(s)=k_0\frac{c^2}{s^2+c k_1s-k_2c^2}
\]
Comparing this with the canonical 2nd order system transfer function
\[
	H(s)=\frac{\omega_n^2}{s^2+2\zeta \omega_n s+\omega_n^2}
\]
we see that we can let $c=\omega_n$ and $k_1=2\zeta$. Moreover, $k_2<0$, 
i.e., the feedback from the output should be negative. $k_0$ is a constant
scalar which can take any value.
	
%\htmladdimg{../figure/opam9.gif}


\end{itemize}

\end{document}
