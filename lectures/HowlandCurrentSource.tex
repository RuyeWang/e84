\documentclass{article}
\usepackage{amsmath}
\usepackage{amssymb}
\usepackage{graphics}
\usepackage{comment}
\usepackage{html,makeidx}

\begin{document}

\section*{Howland Current Source}

  \htmladdimg{../figures/HowlandCurrentSource.png}

  Assuming $R_2/R_1=R_4/R_3$, we can show that the output current
  through the load $R_L$ is a constant determined by the input
  voltages $V^-$ and $V^+$, as well as the circuit parameters.

  \begin{equation}
  \frac{V^--V}{R_1}+\frac{V_0-V}{R_2}=0,\;\;\;\;\;\;
  \frac{V^+-V}{R_3}+\frac{V_0-V}{R_4}=\frac{V}{R_L}=I_L
  \end{equation}
  Solving the first equation for $V_0-V$, we get
  \begin{equation}
  V_0-V=(V-V^-)\frac{R_2}{R_1}=(V-V^-)\frac{R_4}{R_3}
  \end{equation}
  and substituting into the second equation, we get:
  \begin{equation}
  \frac{V^+-V}{R_3}+\frac{V-V^-}{R_3}
  =\frac{V^+-V^-}{R_3}=\frac{V}{R_L}=I_L
  \end{equation}
  We see that the load current $I_L$ through the load resistor 
  $R_L=V/R_L=(V^+-V^-)/R_3$, independent of $R_L$, i.e., the circuit 
  is a current source.


  {\bf Logarithmic and Exponential Amplifiers}

  \htmladdimg{../figures/ExpLogAmplifier.png}

  Based on the relationship between the current through and voltage 
  across a diode and the virtual ground assumption, we can show 
  that the output voltage of the exponential amplifier (left) is
  approximately an exponential function of the input voltage, and 
  the output voltage of the logarithmic amplifier (right) is 
  approximately a logarithmic function of the input voltage.
  
  First, the current $I_D$ through and voltage $V_D$ across a diode 
  are related by:
  \begin{equation}           
  I_D=I_0 \left( e^{V_D/\eta V_T}-1 \right)\approx I_0 \; e^{V_D/\eta V_T},
  \end{equation}        
  where $I_0$, $\eta$ and $V_T$ are some parameters, and $V_D/\eta V_T$ 
  is large enough so that $e^{V_D/\eta V_T}\gg 1$.    

  \begin{itemize}  
  \item Applying KCL to inverting input of the circuit on the left, 
  we get
  \begin{equation}
  I_0\;e^{V_{in}/\eta V_T}+\frac{V_{out}}{R}=0,\;\;\;\;\mbox{i.e.}\;\;\;\;
  V_{out}=-I_0R\;e^{V_{in}/\eta V_T}
  \end{equation}      
  \item Applying KCL to inverting input of the circuit on the right, 
  we get
  \begin{equation}
  \frac{V_{in}}{R}=I_0\;e^{-V_{out}/\eta V_T},\;\;\;\;\mbox{i.e.}\;\;\;\;
  V_{out}=-\eta V_T \ln\frac{V_{in}}{RI_0}
  \end{equation}      
  \end{itemize}
  Comparing the above with the desired:
  \begin{equation}
  V_{out}=C \;\exp(V_{in}/a),\;\;\;\;\;V_{out}=D\;\ln(V_{in}/b)
  \end{equation}
  we see that
  \begin{equation}
  C=-I_0R,\;\;\;\;a=\eta V_T,\;\;\;\;D=-\eta V_T,\;\;\;\;b=RI_0
  \end{equation}  

\end{document}
