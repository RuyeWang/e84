\documentstyle[12pt]{article}
% \usepackage{html}
\textwidth 6.0in
\topmargin -0.5in
\oddsidemargin -0in
\evensidemargin -0.5in
% \usepackage{graphics}  
\begin{document}




{\bf Finding internal resistance:}

The internal resistance $R_0$ of either the voltage or current source 
above can be found by:
\[ R_0=\frac{\mbox{open-circuit voltage}}{\mbox{short-circuit current}}
  =\frac{V_{oc}}{I_{sc}} \]
For a voltage source with $(V_0, R_0)$, the open-circuit voltage is
$V_{oc}=V_0$ and the short-circuit current is $I_{sc}=V_0/R_0$ and
their ratio is $R_0$.

For a current source with $(I_0, R_0)$, the open-circuit voltage is
$V_{oc}=I_0R_)$ and the short-circuit current is $I_{sc}=I_0$ and
their ratio is $R_0$.

{\bf Example:} A voltage source of $V_0=1V$ and $R_0=1\Omega$ can be
converted to a current source of $I_0=V_0/R_0=1A$ with the same $R_0$
(and vice versa). For a load of $R_L=1\Omega$, both energy sources will
provide the load current $I_L=0.5A$ and load voltage $V_L=0.5V$.



\end{document}


	

	

