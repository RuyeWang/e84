\documentstyle[12pt]{article}
\usepackage{html}

\begin{document}

\begin{itemize}
\item {\bf RC first-order system with constant input:}

Taking the unilateral Laplace transform of the original DE:
\[
\tau\frac{d}{dt} v_c(t)+v_c(t)=v_s(t)=V_s\,u(t)
\]
we get
\[
{\cal L}\left[\tau\frac{d}{dt} v_c(t)+v_c(t)\right]
=s\tau V_c(s)-\tau V_0+V_c(s)={\cal L}\right[V_s\,u(t)\right]=\frac{V_s}{s}
\]
Here we have used the property ${\cal L}(dx/dt)=s X(s)-x(0)$,
where $x(0)=\lim_{t\rightarrow 0}x(t)$ is the initial condition.
Solving for $V_c(s)$, we further get
\[
V_c(s)=\frac{V_s}{s(s\tau+1)}+\frac{\tau V_0}{s\tau+1}
=V_s\left(\frac{1}{s}-\frac{\tau}{s\tau+1}\right)+\frac{\tau V_0}{s\tau+1}
\]
Taking the inverse Laplace trannsform we get the solution of the DE in time domain:
\[
v_c(t)={\cal L}^{-1}V_c(s)=V_s-V_s e^{-t/\tau}+V_0e^{-t/\tau}
=V_s+(V_0-V_s)e^{-t/\tau}
\]
To find the steady state solution, we can use the final-value theorem:
\[
\lim_{t\rightarrow\infty}v_c(t)=\lim_{s\rightarrow 0} s V_c(s)
=\lim_{s\rightarrow 0} \left[\frac{V_s}{(s\tau+1)}+\frac{\tau sV_0}{s\tau+1}\right]=V_s
\]

\item {\bf RC first-order system with simultaneous input:}

First consider a complex input $V_s e^{j\omega t}\,u(t)=V_s(\cos\omega t+j\sin\omega t)u(t)$
on the right-hand side. Taking the bilateral Laplace transform of the DE:
\[
\tau\frac{d}{dt} v_c(t)+v_c(t)=V_s \cos(\omega t)=V_s Re(e^{j\omega t})\,u(t)
\]
we get
\[
{\cal L}\left[\tau\frac{d}{dt} v_c(t)+v_c(t)\right]
=s\tau V_c(s)-\tau V_0+V_c(s)={\cal L}\left[V_s\,e^{j\omega t}u(t)\right]=\frac{V_s}{s-j\omega}
\]
Solving for $V_c(s)$, we further get
\[
V_c(s)=\frac{1}{s\tau+1}\left(\frac{V_s}{s-j\omega}+\tau V_0}\right)
=\frac{V_s}{j\omega\tau+1}\left(\frac{1}{s-j\omega}-\frac{1}{s+1/\tau}\right)
+\frac{\tau V_0}{s\tau+1}
\]
Taking the inverse Laplace trannsform we get the solution of the DE in time domain:
\begin{eqnarray}
  v_c(t)&=&{\cal L}^{-1}V_c(s)=\frac{V_s}{j\omega\tau+1}\left[{\cal L}^{-1}
    \left(\frac{1}{s-j\omega}\right)-{\cal L}^{-1}\left(\frac{1}{s+1/\tau}\right)\right]+{\cal L}^{-1}\left( \frac{\tau V_0}{s\tau+1} \right)
  \nonumber\\
  &=&\frac{V_se^{-j\phi}}{\sqrt{\omega^2\tau^2+1}}\left[e^{j\omega t}-e^{-t/\tau}\right]+V_0 e^{-t/\tau}
  =\frac{V_s}{\sqrt{\omega^2\tau^2+1}}e^{j(\omega t-\phi)}
  -\frac{V_se^{-j\phi}}{\sqrt{\omega^2\tau^2+1}}e^{-t/\tau}+V_0 e^{-t/\tau}
  \nonumber
\end{eqnarray}
Taking the real part we get the solution:
\begin{eqnarray}
  v_c(t)&=&\frac{V_s}{\sqrt{\omega^2\tau^2+1}}\cos(\omega t-\phi)
  -\frac{V_se^{-t/\tau}}{\sqrt{\omega^2\tau^2+1}}\cos(-\phi)
  +V_0 e^{-t/\tau}
  \nonumber\\
  &=&\frac{V_s}{\sqrt{\omega^2\tau^2+1}}\cos(\omega t-\phi)
  +\left[V_0 -\frac{V_s}{\sqrt{\omega^2\tau^2+1}}\cos(-\phi)\right] e^{-t/\tau}
  \nonumber
\end{eqnarray}

\end{itemize}

\end{document}

