\documentstyle[12pt]{article}
\usepackage{html}
\textwidth 6.0in
\topmargin -0.5in
\oddsidemargin -0in
\evensidemargin -0.5in
% \usepackage{graphics}  
\begin{document}

\section*{Chapter 2: Circuit Principles}

\subsection*{Solving Circuits with Kirchoff Laws}

{\bf Example 1:} Find the three unknown currents ($I_1,I_2,I_3$) and three 
	unknown voltages ($V_{ab}, V_{bd}, V_{cb}$) in the circuit below:

\htmladdimg{../figures/branchcurrentmethod.gif}

{\bf Note:} The direction of a current and the polarity of a voltage source 
can be assumed arbitrarily. To determine the actual direction and polarity, 
the sign of the values also should be considered. For example, a current 
labeled in left-to-right direction with a negative value is actually flowing 
right-to-left.

\begin{itemize}
\item {\bf The Branch-Current Method:} 
  \begin{itemize}
  \item {\bf Step 1:} Label each unknown branch current arbitrarily with a 
    reference direction. If the calculated value is negative, the actual
    direction of the current is opposite to the reference direction.
  \item {\bf Step 2:} Define unknown voltages in terms of the assumed currents
    to obtain three {\em element equations}:
    \[ \begin{array}{l}
	V_{ab}=2I_1,\;\;\mbox{or}\;\;\;V_{ba}=-2I_1 \\
	V_{bd}=8I_3,\;\;\mbox{or}\;\;\;V_{db}=-8I_3 \\
	V_{cb}=4I_2,\;\;\mbox{or}\;\;\;V_{bc}=-4I_2 \end{array} \]
  \item {\bf Step 3:} Apply KCL to node b:
    \[ \sum I_b=0=+I_1+I_2-I_3\;\;\;\;\;\;\;\;\;\;(1)		\]
    We could also apply KCL to node d:
    \[ \sum I_d=0=-I_1-I_2+I_3	\]
    but this equation is not independent as it adds no information.
  \item {\bf Step 4:} Apply KVL to loops abda and bcdb:	
    \[ \begin{array}{r} 
	\sum V_{abda}=0=V_{ab}+V_{bd}+V_{da}\;\;\;\;\;\;\;\;\;\;(2)	\\
	\sum V_{bcdb}=0=V_{cb}+V_{bd}+V_{dc}\;\;\;\;\;\;\;\;\;\;(3)	
    \end{array} \]
    We could also apply KVL to loop abcda:
    \[ \sum V_{abcda}=0=V_{ab}+V_{bc}+V_{cd}+V_{da}	\]
    But this equation is not independent as it can be obtained as the
    difference between the previous two equations. 
    The above three independent equations (one for current, two for
    voltage) are called {\em connection equations}.
  \item {\bf Step 5:} We now have three simultaneous equations with three 
    unknown branch currents:
    \[ \begin{array}{r}
        \sum I_b=0=+I_1+I_2-I_3\;\;\;\;\;\;\;\;\;\;(1)	\\
	\sum V_{abda}=0=V_{ab}+V_{bd}+V_{da}=2I_1+8I_3-32\;\;\;\;\;\;\;\;\;\;(2') \\
	\sum V_{bcdb}=0=V_{cb}+V_{bd}+V_{dc}=4I_2+8I_3-20\;\;\;\;\;\;\;\;\;\;(3')
	\end{array} \]
    which can be rewritten as
    \[ \left\{ \begin{array}{rrrrr}
	  I_1 & +I_2 & -I_3& = &  0\;\;\;\;\;\;\;\;\;\;(1) \\
	 2I_1 &    & +8I_3 & = & 32\;\;\;\;\;\;\;\;\;\;(2') \\
	      &4I_2& +8I_3 & = & 20\;\;\;\;\;\;\;\;\;\;(3') \end{array} \right. \]
  \item {\bf Step 6: } Solving the equations, we get the three unknown currents:
    \[ I_1=4,\;\;\;I_2=-1,\;\;\;I_3=3	\]
    and then we get the three unknown voltages:
    \[ V_{ab}=8,\;\;\;V_{bd}=24,\;\;\;V_{cb}=-4	\]
  \end{itemize}

  Note that equations from KCL at node d and KVL to other loops are not
  independent. In general, if a circuit has n nodes and b branches, then
  there are $(n-1)$ independent node equations and $l=b-(n-1)$ independent 
  loop equations. In other words, the sum of the number of independent loops
  and the number of independent nodes is always the same as the number of 
  branches, i.e., the number of equations is always equal to the number of 
  unknowns in the branch current method.

\item {\bf The Loop/Mesh-Current Method:}
  \begin{enumerate} 
  \item Define a loop current around each loop in clockwise direction
    (although it could be arbitrary). Assume there are $l$ independent
    loops in the circuit, then we have $l$ loop currents as the unknown
    variables.
  \item Apply KVL around each of the loops in the same clockwise 
    direction to obtain $l$ equations. While calculating the voltage 
    drop across each resistor shared by two loops, both loop currents 
    (in opposite positions) should be considered.
  \item Solve the equation system with $l$ equations for the $l$ unknown
    loop currents.
  \end{enumerate}
  
  Now we can resolve the problem above using loop-current method:

  \htmladdimg{../figures/loopcurrentmethod.gif}

  For the first circuit, we apply KVL to the two loops to get
  \[	\sum V_{abda} =0=-32+2I_a+8(I_a-I_b) \]
  \[	\sum V_{bcdb} =0=8(I_b-I_a)+4I_b+20  \]
  Rewrite these as:
  \[ \left\{ \begin{array}{rrrr} 10I_a&-8I_b&=&32 \\ -8I_a &+12I_b&=&-20
  \end{array} \right. \]
  which can be solved to get $I_a=I_1=4$, $I_b=-I_2=1$, and $I_3=I_a-I_b=3$.

\item {\bf The Node-Voltage Method:}
  \begin{enumerate} 
  \item Assume there are n nodes in the circuit. Select one node as
    the ground, i.e., the reference point for all voltages of the circuit.
    The voltage at each of the remaining n-1 nodes is an unknown to be 
    obtained.
  \item Apply KCL to each of the n-1 nodes to obtain n-1 equations. 
  \item Solve the equation system with n-1 equations for the n-1 unknown
    node voltages.
  \end{enumerate}

  Assume node $d$ is the ground, and consider just $V_b=V_{bd}$ as the 
  only unknown in the problem. Apply KCL to node $b$, we have
  \[	\sum I_b=0=I_1+I_2-I_3	\]
  where
  \[ I_1=\frac{V_a-V_b}{R_{ab}}=\frac{32-V_b}{2},\;\;\;\;
  I_2=\frac{V_c-V_b}{R_{cb}}=\frac{20-V_b}{4},\;\;\;\;
  I_3=\frac{V_b}{R_{bd}}=\frac{V_b-0}{8}  \]
  This equation of one unknown $V_b$ can be solved to get $V_b=24$, and 
  all other unknown currents and voltages can be found easily.

As special case of the node-voltage method with only two nodes, we have the
following theorem:

{\bf Millman's theorem} 

If there are multiple parallel branches between two
nodes $a$ and $b$, then the voltage $V$ at node $a$ can be found as shown
below if the other node $b$ is treated as the reference point.

Assume there are three types of branches:
\begin{itemize}
  \item voltage branches with $V_i$ sources in series with $R_i$. The
    polarity of each $V_i$ is + on the node a side.
  \item current branches with $I_j$ (independent of resistors in series).
    The direction of each $I_j$ is toward node a.
  \item resistor branches with $R_k$.
\end{itemize}

Applying KCL to node $a$, we have:
\[ \sum_i \frac{V-V_i}{R_i}-\sum_j I_j+\sum_k \frac{V}{R_k}=0 \]
Solving for $V$, we get

\[ V=\frac{\sum_i V_i/R_i+\sum_j I_j}{\sum_i 1/R_i+\sum_k 1/R_k}=\frac{\sum I}{\sum G} \]
where the reciprocal of the resistance $G=1/R$ is called the conductance.

\htmladdimg{../figures/millmanthm.gif}

\end{itemize}

In summary, 
\begin{itemize}
  \item Branch current method: each equation is for one of the branches.
  \item Loop  current method: each equation is for one of the independent loops.
  \item Node  voltage method: each equation is for one of the independent nodes.
\end{itemize}


{\bf Example 2:} Solve the following circuit:

\htmladdimg{../figures/branchmethod.gif}

\begin{itemize}
\item {\bf Branch current method:}

  There are $n=4$ nodes (a,b,c,d) and $b=6$ branches in the circuit, 
  therefore we can get $n-1=4-1=3$ independent node equations (first
  three) and $b-(n-1)=6-3=3$ independent loop equations (second three):
  \[ \begin{array}{l}
    \mbox{node a:}\;\;\;\;\; I_1+I_2-I_3=0 	\\
    \mbox{node b:}\;\;\;\;\;-I_1-I_4-I_6=0 	\\
    \mbox{node d:}\;\;\;\;\;-I_2+I_5+I_6=0 	\\
    \mbox{loop 1:}\;\;\;\;\;-V_{s1}+I_1R_1+I_3R_3-I_4R_4+V_{s4}=0 \\
    \mbox{loop 2:}\;\;\;\;\;-I_2R_2+V_{s2}-I_5R_5-I_3R_3=0	 \\
    \mbox{loop 3:}\;\;\;\;\;-V_{s4}+I_4R_4+I_5R_5-I_6R_6-V_{s6}=0 
  \end{array} \]
  Solving these 6 equations we get the 6 branch currents.

\item {\bf Loop current method:}
  Let the three loop currents in the example above be $I_a$, $I_b$ and $I_c$ 
  for loops 1, 2, and 3, respectively, and applying KVL to the three loops,
  we get
  \[ \begin{array}{l}
    \mbox{loop 1:}\;\;\;\;\;-V_{s1}+R_1I_a+R_3(I_a-I_b)+R_4(I_a-I_c)+V_{s4}=0 \\
    \mbox{loop 2:}\;\;\;\;\;R_2I_b+V_{s2}+R_5(I_b-I_c)+R_3(I_b-I_a)=0 \\
    \mbox{loop 3:}\;\;\;\;\;-V_{s4}+R_4(I_c-I_a)+R_5(I_c-I_b)+R_6I_c-V_{s6}=0 
  \end{array} \]
  We can then solve these 3 loop equations to find the 3 loop currents.
\item {\bf Node voltage method:}
  Choose node d as ground, and then apply KCL to the remaining 3 nodes to get:
  \[ \begin{array}{l}
    \mbox{node a:}\;\;\;\;\;(V_a-V_b-V_{s1})/R_1+(V_a-V_{s2})/R_2+(V_a-V_c)/R_3=0 \\
    \mbox{node b:}\;\;\;\;\;(V_b-V_a+V_{s1})/R_1+(V_b-V_c+V_{s4})/R_4+(V_b-V_{s6})/R_6=0 \\
    \mbox{node c:}\;\;\;\;\;(V_c-V_b-V_{s4})/R_4+(V_c-V_a)/R_3+V_c/R_5=0 
  \end{array} \]
  We can then solve these 3 node equations to find the 3 node voltages.

\end{itemize}

{\bf Example 3:} Solve the following circuit.

\htmladdimg{../figures/currentmethod2.gif}

$I=0.5 A$, $V=6 V$, $R_1=3\Omega$, $R_2=8\Omega$, $R_3=6\Omega$, $R_4=4\Omega$.

{\bf Loop current method:}

Assume three loop currents $I_1$ (left), $I_2$ (right), $I_3$ (top). We have
\[ I_1=I=0.5A \]
\[ \left\{ \begin{array}{l}
  R_2(I_2-I_1)+R_3(I_2-I_3)-V=0 \\
  R_1(I_3-I_1)+R_4I_3+R_3(I_3-I_2)=0 \end{array} \right. 
  \;\;\;\;\;\;\;\mbox{i.e.}\;\;\;\;\;\;\;
  \left\{ \begin{array}{l}
  14 I_2-6 I_3=10 \\
  -6 I_2+13 I_3=1.5 \end{array} \right. \]
Solving to get: $I_2=0.952 A$, $I_3=0.555 A$. We can also get the three node
voltages: $V_3=-6V$ (right), $V_2=R_2(I_1-I_2)=8(0.5-0.952)=-3.616V$ (middle),
and $V_1=V_3+R_4 I_3=-3.78 V$. 

{\bf Node voltage method:}

Assume the bottom node is ground and the three node voltages are $V_1$ (left),
$V_2$ (middle), $V_3=V=-6V$ (right). 
\[ \left\{ \begin{array}{l}
  (V_1-V_3)/R_4+(V_1-V_2)/R_1=I \\
  (V_2-V_1)/R_1+(V_2-V_3)/R_3+V_2/R_2=0 \end{array} \right. 
  \;\;\;\;\;\;\;\mbox{i.e.}\;\;\;\;\;\;\;
  \left\{ \begin{array}{l}
  7V_1-4V_2=-12 \\
  -8V_1+15V_2=-24 \end{array} \right. \]
Solving for $V_1$ and $V_2$, we get: $V_1=-3.78$, $V_2=-3.616$, same as before.

{\bf Example 4:} (Homework) Find all node voltages with respect to the top-left
corner treated as reference node:

\htmladdimg{../figures/currentmethod1.gif}

$V_1=12 V$, $V_2=6 V$, $R_1=3 \Omega$, $R_2=8 \Omega$, $R_3=6 \Omega$, $R_4=4\Omega$.

\htmladdnormallink{Answer}{../ch2_sub0/index.html}

{\bf Note:} To simplify the analysis while using node voltage or loop current 
method, it is preferable to
\begin{itemize}
  \item choose independent loops so that no current source is shared by two 
    loops;
  \item choose ground node so that one of the voltage sources is connected
    to ground.
\end{itemize}

{\bf Example 5:} The two circuits shown below are equivalent, but you may want to
choose wisely in terms of which is easier to analyze. Solve this circuit using
both node voltage and loop current methods. Assume $R_1=100\Omega$, $R_2=5\Omega$, 
$R_3=200\Omega$, $R_4=50\Omega$, $V=50V$, and $I=0.2A$.

\htmladdimg{../figures/problembase1.gif}  \htmladdimg{../figures/problembase1a.gif}

\htmladdnormallink{Answer}{../ch2_sub4/index.html}

\subsection*{Network Theorems}

\begin{itemize}
\item {\bf Superposition Theorem:}

A generic system with input (stimulus, cause) $u$ and output (response,
effect) $v$ can be described mathematically as a function $v=F(u)$. 
The function is linear is it satisfies the following:
\begin{itemize}
\item {\bf Homogeneity}
  \[	F(ax)=aF(x) \]
\item {\bf Superposition}
  \[ F(x+y)=F(x)+F(y)	\]
\end{itemize}
Combining these two aspects, we have
\[	F(ax+by)=aF(x)+bF(y)	\]
The interpretation is that when the response is directly proportional to 
the cause in a system $F$, then we can consider several causes ($x$ and $y$) 
individually and then combine the individual responses ($F(x)$ and $F(y)$). 
An electrical system of linear components (with linear voltage-current 
relation) is a linear system (if only voltage and current are of interest).

When the system is linear, the superposition principle applies:

{\bf When there exist multiple energy sources, the currents and voltages in 
the circuit can be found as the algebraic sum of the corresponding values 
obtained by assuming only one source at a time, with all other sources turned 
off:
\begin{itemize}
\item A voltage source is treated as short circuit so that $V=0$.
\item A current source is treated as open circuit so that $I=0$.
\end{itemize}}

As superposition principle only applies to linear functions, it cannot be
applied to nonlinear functions such as power (e.g., $V^2/R$ or $I^2R$). 

\htmladdimg{../figures/superposition0.gif}

{\bf Superposition of voltage sources:}
\[	I_{12}=\frac{aV_1+bV_2}{R}=a\frac{V_1}{R}+b\frac{V_2}{R}=aI_1+bI_2 \]
where $I_1=V_1/R$ ($V_2$ short circuit) and $I_2=V_2/R$ ($V_1$ short circuit).
However, note that superposition principle does not apply to power:
\[
P_{12}=\frac{(V_1+V_2)^2}{R}\ne\frac{V_1^2}{R}+\frac{V_2^2}{R}=P_1+P_2
\]
{\bf Superposition of current sources:}
\[	V=R(aI_1+bI_2)=aI_1R+bI_2R=aV_1+bV_2 \]
where $V_1=I_1R$ ($I_2$ open circuit) and $V_2=I_2R$ ($I_1$ open circuit).
Again, superposition principle does not apply to power:
\[	P_{12}=R(I_1+I_2)^2 \ne RI_1^2+RI_2^2=P_1+P_2	\]

{\bf Example 1:} The previous example can also be solved by superposition 
theorem. 

\htmladdimg{../figures/kirchhoffexample1.gif}

First turn the voltage source of 20V off (short-circuit with 0V), and get
\[ I'_1=\frac{32}{2+8 || 4}=\frac{48}{7},\;\;\;I'_2=-I'_1\frac{8}{8+4}=-\frac{32}{7},
   \;\;\;I'_3=I'_1\frac{4}{8+4}=\frac{16}{7} \]
Second turn the voltage source of 32V off and get
\[ I''_2=\frac{20}{4+8 || 2}=\frac{25}{7},\;\;\;I''_1=-I''_2\frac{8}{2+8}=-\frac{20}{7},
   \;\;\;I''_3=I''_2\frac{2}{8+2}=\frac{5}{7} \]
The overall currents can then be found as the algebraic sums of the
corresponding values obtained with one voltage source turned on at a time:
\[ I_1=I'_1+I''_1=\frac{48}{7}-\frac{20}{7}=4,\;\;\;\;
   I_2=I''_2+I'_2=\frac{25}{7}-\frac{32}{7}=-1,\;\;\;\;
   I_3=I'_3+I''_3=\frac{16}{7}+\frac{5}{7}=3 \]

{\bf Example 2:} Find voltage $V$ and current $I$.

\htmladdimg{../figures/superposition.gif}

First, we solve this problem using node-voltage method. Assume the currents
$I_1$ (left branch), $I_0$ and $I_2$ (right branch) all leave the top node,
where the voltage is $V$ (with respect to the bottom treated as ground). 
By KCL, we have
\[ I_0+I_1+I_2=I_0+\frac{V-V_0}{R_1}+\frac{V}{R_2}=0,\;\;\;\;\mbox{i.e.,}\;\;\;\;
I_0R_1R_2+R_2(V-V_0)+R_1V=0 \]
Solving for $V$, we get:
\[ V=\frac{R_2V_0-I_0R_1R_2}{R_1+R_2}
  =V_0\frac{R_2}{R_1+R_2}-I_0\frac{R_1R_2}{R_1+R_2} \]
and
\[ I=I_2=\frac{V}{R_2}=\frac{V_0}{R_1+R_2}-I_0\frac{R_1}{R_1+R_2} \]
Next, using superposition theorem, we get
\begin{itemize}
\item Find $V'$ and $I'$ with the current source off (open circuit with zero 
  current):
  \[ I'=\frac{V_0}{R_1+R_2},\;\;\;\;V'=I'R_2=V_0\frac{R_2}{R_1+R_2} \]
\item Find $V''$ and $I''$ with the voltage source off (short circuit with 
  zero voltage):
  \[ I''=-I_0\frac{R_1}{R_1+R_2},\;\;\;\;V''=-I''R_2=-I_0\frac{R_1R_2}{R_1+R_2} \]
  Both $I''$ and $V''$ have a negative sign as their direction and polarity are
  opposite to those of the assumed current and voltage.  
\item Find the sum of the two:
  \[ I=I'+I''=\frac{V_0-I_0R_1}{R_1+R_2},\;\;\;\;
  V=V'+V''=R_2I=(V_0-I_0R_1)\frac{R_2}{R_1+R_2}	\]
\end{itemize}

\item {\bf Thevenin's Theorem:}

In principle, all currents and voltages of an arbitrary network of linear 
components and voltage/current sources can be found by any of the three methods 
discussed previous, namely, the branch current method, the loop current method 
and the node voltage method.

However, if only the current and/or voltage associated with one component 
are of interest, it is unnecessary to find voltages and currents elsewhere
in the circuit. The methods considered below can be used in such situations.

{\bf Any one-port (two-terminal) network of resistance elements and energy 
sources is equivalent to an ideal voltage source $V_T$ in series with a 
resistor $R_T$, where}
\begin{itemize}
\item $V_T$ {\bf is the open-circuit voltage of the network, and}
\item $R_T$ {\bf is the equivalent resistance when all energy sources are turned
off (short-circuit for voltage sources, open-circuit for current sources).}
\end{itemize}

\htmladdimg{../figures/theveninnortonfigure.gif}

If we are only interested in finding the voltage $V$ across and current $I$ 
through one particular resistor in a complex circuit containing a large number
of resistors, voltage and current sources, we can ``pull'' the resistor out 
and treat the rest of the circuit as a Thevenin voltage source $(V_T, R_T)$, 
and use Thevenin's theorem to find $V$ and $I$.

{\bf Proof:} 

\htmladdimg{../figures/TheveninProof.gif}

Assume with the load the network's terminal voltage and current are $V$ 
and $I$ respectively. 
\begin{itemize}
\item Replace the load by an ideal current source $I$ while keeping the 
  terminal voltage $V$ the same (b), the voltage or current anywhere in 
  the system should not be affected. 
	
\item Find the terminal voltage $V$ in terms of the internal energy sources 
  inside the network and the external current source by superposition
  principle:
  \begin{itemize}
  \item When the external current source is off (open circuit), the 
    terminal voltage $V'=V_{oc}$ is due only to the sources internal 
    to the network (c).
  \item When all internal sources are turned off (short-circuit for
    voltage sources, open-circuit for current sources), the terminal 
    voltage $V''=-R_0I$ where $R_0$ is the equivalent resistance of 
    the network with all energy sources off (d).
  \end{itemize}
  The overall terminal voltage is $V=V'+V''=V_{oc}-R_0I$. 
\end{itemize}
As far as the port voltage $V$ and current $I$ are concerned, a one-port
network is equivalent to an ideal voltage source $V_T=V{oc}$ equal to the
open-circuit voltage across the port, in series with an internal resistance
$R_T=R_0$, which can be obtained as the ratio of the open-circuit voltage 
and the short-circuit current.

\item {\bf Norton's Theorem:}

{\bf Any one-port (two-terminal) network of resistance elements and energy 
sources is equivalent to an ideal current source $I_N$ in parallel with a 
resistor $R_N$, where}
\begin{itemize}
\item {\bf $I_N$ is the short-circuit current of the network, and}
\item {\bf $R_N$ is the equivalent resistance when all energy sources are 
  turned off (short-circuit for voltage sources, open-circuit for current 
  sources).}
\end{itemize}

{\bf Proof:} The proof of this theorem is in parallel with the proof of the 
Thevenin's theorem. Again, assume with the load the network's terminal voltage 
and current are $V$ and $I$ respectively. 

\htmladdimg{../figures/NortonProof.gif}

\begin{itemize}
\item Replace the load by an ideal voltage source $V$ while keeping
  the terminal current $I$ the same, the voltage or current anywhere
  in the system should not be affected. 
	
\item Find the terminal current $I$ in terms of the internal energy sources 
  inside the network and the external voltage source by superposition principle:
  \begin{itemize}
  \item When the external voltage source is off (short circuit), the 
    terminal current $I'=I_{sc}$ is due only to the sources internal 
    to the network.
  \item When all internal sources are turned off (short-circuit for
    voltage sources, open-circuit for current sources), the terminal 
    current $I''=-V/R_0$, where $R_0$ is the equivalent resistance of 
    the network with all energy sources off.
  \end{itemize}
  The overall terminal current is $I=I'+I''=I_{sc}-V/R_0$. 
\end{itemize}
As far as the port voltage $V$ and current $I$ are concerned, a one-port
network is equivalent to an ideal current source $I_T=I{sc}$ equal to the
short-circuit current through the port, in parallel with an internal resistance
$R_N=R_0$, which can be obtained as the ratio of the open-circuit voltage
and the short-circuit current.

Of course, the Norton's theorem can also be easily proven by converting 
Thevenin's equivalent circuit of an ideal voltage source $V_T=V{oc}$ in 
series with a resistance $R_T$ to an equivalent circuit of an ideal current 
source $I_N=I_{sc}=V_{os}/R_T$ in parallel with a resistance $R_N=R_T$.

{\bf Load Line and Output Resistance}

Due to the Thevenin's and Norton's theorems, any one-port network of 
resistors and energy sources can be converted into a simple voltage or 
current source with an internal or output resistance $R_0$. Moreover, 
the relationship between the voltage $V$ across and the current $I$ 
through the load is a straight line referred to as the {\bf load line}.
The slope $-\triangle V/\triangle I=R_0$ of the load line indicates 
the internal or output resistance of the network, as shown in the
figure below. One the other hand, the input resistance $R_L$ of the 
load can also be represented on the graph as a straight line with its
slope $\triangle V/\triangle I=R_L$. The intersection of these two
lines indicates the actual voltage $V$ and current $I$ with the load
$R_L$.

\htmladdimg{../figures/loadline.gif}
% \htmladdimg{../figures/IVcharacteristic.gif}

The output resistance $R_0$ of a network can also be determined 
experimentally by varying the load $R_L$. Assume $(V_1, I_1)$ are 
associated with load $R_1$ and $(V_2, I_2)$ with load $R_2$, then 
the output resistance of the source network can be found to be:

\[ R_0=-\frac{\triangle V}{\triangle I}=-\frac{V_1-V_2}{I_1-I_2} \]

To show this, we assume the source network is converted to a voltage
source with $V_0$ and $R_0$, and use two different loads $R_1$ and
$R_2$ with
\[ I_1=\frac{V_0}{R_0+R_1},\;\;\;\;V_1=V_0\;\frac{R_1}{R_0+R_1},\;\;\;\;
   I_2=\frac{V_0}{R_0+R_2},\;\;\;\;V_2=V_0\;\frac{R_2}{R_0+R_2} \]
and
\[ \triangle I=I_1-I_2=V_0\;(\frac{1}{R_0+R_1}-\frac{1}{R_0+R_2}),\;\;\;
   \triangle V=V_1-V_2=V_0\;(\frac{R_1}{R_0+R_1}-\frac{R_2}{R_0+R_2}) \]
The output resistance can therefore be found to be $R_0$ as expected:
\[ -\frac{\triangle V}{\triangle I}=-\frac{V_1-V_2}{I_1-I_2}=R_0 \]

Consider two extreme cases for the two loads $R_1$ and $R_2$:
\begin{itemize}
\item when $R_1=0$ (short circuit), then we get the short-circuit 
  current: 
  \[ I_1=I_{sc}=V_0/R_0,\;\;\;\;\;\; V_1=0 \]
\item when $R_2=\infty$ (open circuit), then we get the open-circuit
  voltage: 
  \[ V_2=V_{oc}=V_0,\;\;\;\;\;\;\;I_2=0 \]
\end{itemize}
and we have $\triangle V=V_1-V_2=-V_0$, $\triangle I=V_1-V_2=V_0/R_0$,
and the output resistance is found to be:
\[ R_0=-\frac{\triangle V}{\triangle I}=\frac{V_{oc}}{I_{sc}} \]

{\bf Example:}

\htmladdimg{../figures/TheveninNorton.gif}

\begin{itemize}
\item Thevenin's equivalent circuit:
	\begin{itemize}
	\item Find open-circuit voltage $V_T=V_{oc}=V_{ab}=V_0-I_0R_1$
	\item Find equivalent internal resistance $R_T=R_1$:
	\item The Thevenin's equivalent circuit is shown in (b).
	\end{itemize}
\item Norton's equivalent circuit:
	\begin{itemize}
	\item Find short-circuit current. As there are two sources, 
	superposition principle is used to get $I_N=I_{sc}=I_{ab}=V_0/R_1-I_0$
	\item Find equivalent internal resistance $R_N=R_1$:
	\item The Norton's equivalent circuit is shown in (c).
	\end{itemize}
\end{itemize}
Note that the resistor $R_2$ does not appear in either of the equivalent
circuit. This is because $R_2$ is in series with an ideal current source
which drives a constant current $I_0$ through the branch, independent of
the resistance along the branch.

Also note that the Thevenin's voltage source and the Norton's current source
can be converted into each other:
\begin{itemize}
	\item Convert Thevenin's voltage source to Norton's current source:
	\[ I_N=V_T/R_T=V_0/R_1-I_0	\]
	\item Convert Norton's current source to Thevenin's voltage source:
	\[ V_T=I_N R_N=V_0-I_0R_1	\]
\end{itemize}

\item {\bf $\Delta$-Y Transformation}

\htmladdimg{../figures/DeltaY.gif}

The $\Delta$ configuration can be converted to $Y$ and vice versa. 
To relate $\Delta$ and $Y$, the resistances $R'_{ab}$ between terminals
a and b of $Y$ should be equal to $R_{ab}$ of $\Delta$, and the same is
true for the other two resistances, i.e.,
\[ \left\{ \begin{array}{rr}
	R'_{ab}=R_a+R_b=R_{ab}//(R_{ac}+R_{bc})	\\
	R'_{ac}=R_a+R_c=R_{ac}//(R_{ab}+R_{bc}) \\
	R'_{bc}=R_b+R_c=R_{bc}//(R_{ab}+R_{ac}) \end{array} \right. \]
\begin{itemize}
  \item {\bf Convert $\Delta$ to $Y$}: 

    Given $R_{ab}$, $R_{ac}$ and $R_{bc}$ of a $\Delta$, the three equations 
    can be solved for $R_a$, $R_b$ and $R_c$ of the corresponding $Y$. For 
    example, subtracting the 3rd equation from the sum of the first two, we 
    get expression for $R_a$. The solutions are:
    \[ \left\{ \begin{array}{rr}
      R_a=R_{ab}R_{ac}/(R_{ab}+R_{ac}+R_{bc}) \\
      R_b=R_{ab}R_{bc}/(R_{ab}+R_{ac}+R_{bc}) \\
      R_c=R_{ac}R_{bc}/(R_{ab}+R_{ac}+R_{bc}) \\
    \end{array} \right. \]

  \item {\bf Convert $Y$ to $\Delta$}: 

    Reversely, given $R_a$, $R_b$ and $R_c$ of a $Y$, the same three 
    equations can also be solved for $R_{ab}$, $R_{ac}$ and $R_{bc}$ of
    the corresponding $\Delta$ to get:
    \[ \left\{ \begin{array}{rr}
      R_{ab}=R_a+R_b+R_aR_b/R_c	\\
      R_{ac}=R_a+R_c+R_aR_c/R_b	\\
      R_{bc}=R_b+R_c+R_bR_c/R_a	\end{array} \right. \]
    The top circuit in the figure below can be converted into either of 
    the two equivalent circuits below.
\end{itemize}

\htmladdimg{../figures/DeltaY1.gif}

The $\Delta$ formed by $Z_1$, $Z_1$, and $Z_3$ can
be converted to a Y, which can then be combined with $Z_4$ to get 
a Y (bottom left) with:
\[ \left\{ \begin{array}{l}
  Y_1=Z_1Z_3/(Z_1+Z_2+Z_3) \\
  Y_2=Z_2Z_3/(Z_1+Z_2+Z_3) \\
  Y_3=Z_1Z_2/(Z_1+Z_2+Z_3)+Z_4 \\
\end{array} \right. \]
Alternatively, the Y formed by $Z_1$, $Z_2$, and $Z_4$ can
be converted to a $\Delta$, which can then be combined with $Z_3$ 
to get a $\Delta$ (bottom right) with:
\[ \left\{ \begin{array}{l}
  X_1=Z_1+Z_4+Z_1Z_4/Z_2 \\
  X_2=Z_2+Z_4+Z_2Z_4/Z_1 \\
  X_3=(Z_1+Z_2+Z_1Z_2/Z_4) || Z_3 \\
\end{array} \right. \]
The resulting $\Delta$ and Y circuits are equivalent as
it can be shown they can also be converted to each other with the
same system variables.

{\bf Example 0:} (Homework)

The conversion from $\Delta$ to $Y$ is more useful as $Y$ is easier to
analyze than $\Delta$. For example, the circuit in (a) below can be 
converted to that in (b) to find all the currents in the circuit:

\htmladdimg{../figures/DeltaYEx.gif}

Assume $V_0=225 V$, $R_0=1\Omega$, $R_1=40\Omega$, $R_2=36\Omega$, 
$R_3=50\Omega$, $R_4=55\Omega$, $R_5=10\Omega$, find the currents 
$I_0$, $I_1$, $I_2$, $I_3$, $I_4$, and $I_5$. 

\end{itemize}

{\bf Example 1: } 

In the circuit below, $V_0=18V$, $R_1=R_2=3\Omega$, 
$R_3=6\Omega$, $R_4=1.5\Omega$. Find the value of current $I$ when $R_5$
is $1\Omega$, $2\Omega$, and $3\Omega$. Moreover, find the value for 
$R_5$ for the desired current $I=0.5A$.

\htmladdimg{../figures/bridge_example.gif}

{\bf Method 1, $\Delta-Y$ conversion}

Find $I$ when $R_5=2\Omega$. First convert the $\Delta$ composed of
$R_1$, $R_2$ and $R_5$ into a $Y$ composed of $R_a$, $R_b$ and $R_c$:
\[ 	R_c=\frac{R_1R_2}{R_1+R_2+R_5}=\frac{9}{8},\;\;\;\;
	R_a=\frac{R_1R_5}{R_1+R_2+R_5}=\frac{3}{4}=R_b	\]
Find overall resistance:
\[ R_{total}=R_c+(R_a+R_3) || (R_b+R_4)=\frac{45}{16}	\]
Find overall current:
\[ I_0=\frac{V_0}{R_{total}}=\frac{18}{45/16}=\frac{32}{5}\]
Find currents through $R_3$ and $R_4$ (current divider):
\[ I_a=I_0\;\frac{R_b+R_4}{(R_a+R_3)+(R_b+R_4)}=\frac{8}{5} \]
\[ I_b=I_0\;\frac{R_a+R_3}{(R_a+R_3)+(R_b+R_4)}=\frac{24}{5}\]
Find voltage at points $a$ and $b$ (assuming negative end of voltage
source is ground):
\[ V_a=I_a \times R_3=\frac{8 }{5} \times 6=\frac{48}{5},\;\;\;\;\;
   V_b=I_b \times R_4=\frac{24}{5} \times 1.5=\frac{36}{5}	\]
Find current $I$ through $R_5=2\Omega$:
\[	I=\frac{V_a-V_b}{R_5}=\frac{12}{5}\times \frac{1}{2}=1.2\]
The same steps can be repeated for $R_5=1\Omega$ and $R_5=3\Omega$. 
But it is hard to find a value of $R_5$ given the require current $I=0.5$.

{\bf Method 2, Thevenin's theorem}

Solve the problem using Thevenin's theorem by the following steps:
\begin{itemize}
\item remove the branch in question from the circuit and treat the rest as
	a one-port network.
\item simplify the one-port network by Thevenin's theorem, find the open
	circuit voltage $V_T$ and the equivalent internal resistance $R_T$.
\item put the branch in equation back as the load of the Thevenin equivalent
	network and find the current/voltage.
\end{itemize}

Here, we remove $R_5$ as the load of a network composed of all other
resistors $R_1$, $R_2$, $R_3$, $R_4$ and the voltage source $V_0=18V$, then
apply Thevenin's theorem to find the open-circuit voltage between the two
terminals a and b:
\[	V_T=V_{oc}=V_0\frac{R_3}{R_1+R_3}-V_0\frac{R_4}{R_2+R_4}
	=18 (\frac{6}{9}-\frac{1.5}{4.5})=6V \]
and the internal resistance between a and b (with voltage source $V_0$ short 
circuit):
\[ R_T=R_1//R_3+R_2//R_4=\frac{3\times 6}{3+6}+\frac{3\times 1.5}{3+1.5}=3\Omega \]
Now find current $I$ for different $R_5$
\begin{itemize}
\item $R_5=1$, $I=6/(3+1)=1.5$
\item $R_5=2$, $I=6/(3+2)=1.2$
\item $R_5=3$, $I=6/(3+3)=1.0$
\end{itemize}
and when $I=0.5$, $R_5=V_T/I-R_T=6/0.5-3=9$


{\bf Example 2: } The circuit below, often used in some control system, 
is composed of two voltages, two potentiometers, and a load resistor.
Assume $V_1=72V$, $V_2=80V$, $R_1=1.5\Omega$, $R_2=3\Omega$, $R_3=1.5\Omega$,
and $R_4=2.5\Omega$. Find the current $I_L$ through the load resistor 
$R_L=1.5\Omega$. We denote the current through $R_i$ by $I_i$, and the 
voltage at the left and right ends of $R_L$ by $V_a$ and $V_b$, respectively,
with respect to the bottom wire assumed to be the ground.

\htmladdimg{../figures/potentiometers.gif}

{\bf Method 1, Superposition theorem}

Find $I'_L$ caused by voltage $V_1=72$, and then $I''_L$ caused by voltage
$V_2=80$, then get $I_L=I'_L+I''_L$.

\begin{itemize}

\item Short circuit $V_2$. Assume $I'_L=8$, so that currents through
	$R_3=1.5$ and $R_4=2.5$ are, respectively, $I_3=5$ and $I_4=3$
	(current divider), and $V_b=I_3R_3+I_4R_4=3\times 5+5\times 1.5=7.5$.
\item $V_a=R_L I'_L+V_b=1.5\times 8+7.5=19.5$, current through $R_2=3$
	is $I_2=V_a/R_2=19.5/3=6.5$, current through $R_1$ is 
	$I_1=I_2+I'_L=6.5+8=14.5$.
\item $V'_1=R_1\times I_1+V_a=1.5\times 14.5+19.5=41.25$. But $V_1=72$,
	we get scaling factor $V_1/V'_1=72/41.25=96/55$, and 
	$I'_L=8 \times 96/55$.

\item Short circuit $V_1$. Assume $I''_L=3$, so that currents through
	$R_1=1.5$ and $R_2=3$ are, respectively, $I_1=2$ and $I_2=1$
	(current divider), and $V_a=R_1I_1=R_2I_2=1.5\times 2+3\times 1=3$.
\item $V_b=R_L I''_L+V_a=1.5\times 3+3=7.5$, current through $R_3=1.5$
	is $I_3=V_b/R_3=7.5/1.5=5$, current through $R_4$ is 
	$I_4=I_3+I''_L=5+3=8$.
\item $V'_2=R_4\times I_4+V_b=2.5\times 8+7.5=27.5$. But $V_2=80$,
	we get scaling factor $V_2/V'_2=80/27.5=32/11$, and 
	$I''_L=3\times 32/11=96/11$.

\item Finally, we get the load current:
\[	I_L=I'_L-I''_L=\frac{96\times 8}{55}-\frac{96}{11}=\frac{288}{55} \]

\end{itemize}


{\bf Method 2, Thevenin's theorem}

Remove $R$, find open-circuit voltage $V_{ab}=V_a-V_b$ and equivalent
resistance $R$, then find $I_L=V_{ab}/(R+R_L)$.

\begin{itemize}
\item 
  $V_a=V_1\times R_2/(R_1+R_2)=72\times 3/(1.5+3)=48$

  $V_b=V_2\times R_3/(R_3+R_4)=80\times 1.5/(1.5+2.5)=30$

  $V_T=V_{ab}=V_a-V_b=48-30=18$
\item 
  $R_T=R_1 || R_2 + R_3 || R_4=3||1.5+2.5||1.5=7.75/4$
\item 
  \[ I_L=\frac{V_{ab}}{R+R_L}=\frac{18}{1.5+7.75/4}=\frac{288}{55} \]
\end{itemize}

\subsection*{Two-Port Networks}

{\bf Models of two-port networks}

Many complex passive and linear circuits can be modeled by a two-port network
model as shown below. A two-port network is represented by four external 
variables: voltage $V_1$ and current $I_1$ at the input port, and voltage 
$V_2$ and current $I_2$ at the output port, so that the two-port network 
can be treated as a black box modeled by the relationships between the 
four variables $V_1$, $V_2$, $I_1$ and $I_2$. There exist six different ways 
to describe the relationships between these variables, depending on which 
two of the four variables are given, while the other two can always be derived.

{\bf Note: } All voltages and currents below are complex variables and
represented by phasors containing both magnitude and phase angle. However, 
for convenience the phasor notation $\dot{V}$ and $\dot{I}$ are replaced by 
$V$ and $I$ respectively.

\htmladdimg{../figures/twoportmodel.gif}

\begin{itemize}
\item {\bf Z or impedance model:} Given two currents $I_1$ and $I_2$ 
  find voltages $V_1$ and $V_2$ by:
  \[ \left\{ \begin{array}{l} V_1=Z_{11}I_1+Z_{12}I_2 \\
    V_2=Z_{21}I_1+Z_{22}I_2 \end{array} \right.\;\;\;\;\;
  \left[ \begin{array}{l} V_1 \\ V_2\end{array} \right]=
  \left[ \begin{array}{ll} Z_{11} & Z_{12}\\Z_{21} & Z_{22}\end{array}\right]
  \left[ \begin{array}{l} I_1 \\ I_2\end{array} \right]
  ={\bf Z}\left[ \begin{array}{l} I_1 \\ I_2\end{array} \right] \]
  Here all four parameters $Z_{11}$, $Z_{12}$, $Z_{21}$, and $Z_{22}$ represent
  impedance. In particular, $Z_{21}$ and $Z_{12}$ are {\bf transfer impedances}, 
  defined as the ratio of a voltage $V_1$ (or $V_2$) in one part of a network to 
  a current $I_2$ (or $I_1$) in another part $Z_{12}=V_1/I_2$. ${\bf Z}$ is a 2 
  by 2 matrix containing all four parameters.

\item {\bf Y or admittance model:} Given two voltages $V_1$ and $V_2$,
  find currents $I_1$ and $I_2$ by:
  \[ \left\{ \begin{array}{l} I_1=Y_{11}V_1+Y_{12}V_2 \\
    I_2=Y_{21}V_1+Y_{22}V_2 \end{array} \right.\;\;\;\;\;
  \left[ \begin{array}{l} I_1 \\ I_2\end{array} \right]=
  \left[ \begin{array}{ll} Y_{11} & Y_{12}\\Y_{21} & Y_{22}\end{array}\right]
  \left[ \begin{array}{l} V_1 \\ V_2\end{array} \right]
  ={\bf Y}\left[ \begin{array}{l} V_1 \\ V_2\end{array} \right]  \]
  Here all four parameters $Y_{11}$, $Y_{12}$, $Y_{21}$, and $Y_{22}$ represent
  admittance. In particular, $Y_{21}$ and $Y_{12}$ are {\bf transfer admittances}. 
  ${\bf Y}$ is the corresponding parameter matrix.

\item {\bf A or transmission model:} Given $V_2$ and $I_2$, find 
  $V_1$ and $I_1$ by:
  \[\left\{ \begin{array}{l} 
    V_1=A_{11}V_2+A_{12}(-I_2) \\
    I_1=A_{21}V_2+A_{22}(-I_2) \end{array} \right.\;\;\;\;\;
  \left[ \begin{array}{l} V_1 \\ I_1\end{array} \right]=
  \left[ \begin{array}{ll} A_{11} & A_{12}\\A_{21} & A_{22}\end{array}\right]
  \left[ \begin{array}{r} V_2 \\ -I_2\end{array} \right]
  ={\bf A}\left[ \begin{array}{r} V_2 \\ -I_2\end{array} \right]  \]
  Here $A_{11}$ and $A_{22}$ are dimensionless coefficients, $A_{12}$ is impedance 
  and $A_{21}$ is admittance. A negative sign is added to the output current $I_2$ 
  in the model, so that the direction of the current is out-ward, for easy
  analysis of a cascade of multiple network models.

\item {\bf H or hybrid model:} Given $V_2$ and $I_1$, find $V_1$ and $I_2$ by:
  \[\left\{ \begin{array}{l} 
    V_1=H_{11}I_1+H_{12}V_2 \\
    I_2=H_{21}I_1+H_{22}V_2 \end{array} \right.\;\;\;\;\;
  \left[ \begin{array}{l} V_1 \\ I_2\end{array} \right]=
  \left[ \begin{array}{ll} H_{11} & H_{12}\\H_{21} & H_{22}\end{array}\right]
  \left[ \begin{array}{l} I_1 \\ V_2\end{array} \right]
  ={\bf H}\left[ \begin{array}{l} I_1 \\ V_2\end{array} \right]  \]
  Here $H_{12}$ and $H_{21}$ are dimensionless coefficients, $H_{11}$ is impedance
  and $H_{22}$ is admittance. 

\end{itemize}

{\bf Generalization to nonlinear circuits}

The two-port models can also be applied to a nonlinear circuit if the 
variations of the variables are small and therefore the nonlinear behavior 
of the circuit can be piece-wise linearized. Assume $z=f(x,y)$ is a nonlinear
function of variables $x$ and $y$. If the variations $\triangle x$ and 
$\triangle y$ are small, the function can be approximated by a linear 
model 
\[ \triangle z=\triangle f(x,y)=\frac{\partial f}{\partial x}\triangle x +
                    \frac{\partial f}{\partial y}\triangle y  \]
with the linear coefficients
\[ \frac{\partial f}{\partial x}=\lim_{\triangle x\rightarrow 0}
   \frac{\triangle f}{\triangle x}|_{\triangle y=0}
   \;\;\;\;\;\;\;\;
   \frac{\partial f}{\partial y}=\lim_{\triangle y\rightarrow 0}
   \frac{\triangle f}{\triangle y}|_{\triangle x=0}
\]

{\bf Finding the model parameters}

For each of the four types of models, the four parameters can be found
from variables $V_1$, $V_2$, $I_1$ and $I_2$ of a network by the following.
\begin{itemize}
\item For Z-model:
\[	Z_{11}=\frac{V_1}{I_1}|_{I_2=0},\;\;\;\;
	Z_{12}=\frac{V_1}{I_2}|_{I_1=0},\;\;\;\;
	Z_{21}=\frac{V_2}{I_1}|_{I_2=0},\;\;\;\;
	Z_{22}=\frac{V_2}{I_2}|_{I_1=0}
\]
\item For Y-model:
\[	Y_{11}=\frac{I_1}{V_1}|_{V_2=0},\;\;\;\;
	Y_{12}=\frac{I_1}{V_2}|_{V_1=0},\;\;\;\;
	Y_{21}=\frac{I_2}{V_1}|_{V_2=0},\;\;\;\;
	Y_{22}=\frac{I_2}{V_2}|_{Y_1=0},\;\;\;\;
\]
\item For A-model:
\[	A_{11}=\frac{V_1}{V_2}|_{I_2=0},\;\;\;\;
	A_{12}=\frac{V_1}{I_2}|_{V_2=0},\;\;\;\;
	A_{21}=\frac{I_1}{V_2}|_{I_2=0},\;\;\;\;
	A_{22}=\frac{I_1}{I_2}|_{V_2=0}
\]
\item For H-model:
\[	H_{11}=\frac{V_1}{I_1}|_{V_2=0},\;\;\;\;
	H_{12}=\frac{V_1}{V_2}|_{I_1=0},\;\;\;\;
	H_{21}=\frac{I_2}{I_1}|_{V_2=0},\;\;\;\;
	H_{22}=\frac{I_2}{V_2}|_{I_1=0}
\]
\end{itemize}

If we further define
\[	{\bf V}=[V_1, V_2]^T,\;\;\;\;\;\;{\bf I}=[I_1, I_2]^T	\]
then the Z-model and Y-model above can be written in matrix form:
\[ {\bf V}={\bf Z} {\bf I},\;\;\;\;\;\;{\bf I}={\bf Y} {\bf V},\;\;\;\;
	{\bf Y}={\bf Z}^{-1}		\]

{\bf Example: } 

\htmladdimg{../figures/networkLC.gif}

Find the Z-model and Y-model of the circuit shown.
\[ \left\{ \begin{array}{l} V_1=Z_{11}I_1+Z_{12}I_2 \\
	V_2=Z_{21}I_1+Z_{22}I_2 \end{array} \right.
\]
\begin{itemize}
\item First assume $I_2=0$, we get

\[ Z_{11}=V_1/I_1=j\omega L+1/j\omega C,\;\;\;\;\;Z_{21}=V_2/I_1=1/j\omega C \]
\item Next assume $I_1=0$, we get
\[ Z_{22}=V_2/I_2=1/j\omega C,\;\;\;\;\;\;Z_{12}=V_1/I_2=1/j\omega C \]
\end{itemize}
The parameters of the Y-model can be found as the inverse of $Z$:
\[ \left[\begin{array}{cc}Y_{11}&Y_{12}\\Y_{21}&Y_{22}\end{array}\right]
  =\left[\begin{array}{cc}Z_{11}&Z_{12}\\Z_{21}&Z_{22}\end{array}\right]^{-1}
  =\left[\begin{array}{cc}j\omega L+1/j\omega C & 1/j\omega C\\
	1/j\omega C & 1/j\omega \end{array}\right]^{-1}
  =\left[\begin{array}{cc}1/j\omega L & -1/j\omega L\\
	-1/j\omega L & j\omega C+1/j\omega L\end{array}\right] \]
	
{\bf Note:} 
\[	 \left[ \begin{array}{rr} a & b \\ c & d \end{array} \right]^{-1}
=\frac{1}{ad-bc}\left[ \begin{array}{rr} d & -b \\ -c & a \end{array} \right]
\]

{\bf Combinations of two-port models}

\begin{itemize}
\item Series connection of two 2-port networks: 
	${\bf Z}={\bf Z}_1+{\bf Z}_2$
\item Parallel connection of two 2-port networks: 
	${\bf Y}={\bf Y}_1+{\bf Y}_2$
\item Cascade connection of two 2-port networks:
	${\bf A}={\bf A}_1 \cdot {\bf A}_2$
\end{itemize}

\htmladdimg{../figures/networkmodels1.gif}

{\bf Example:} A The circuit shown below contains a two-port network (e.g., a 
filter circuit, or an amplification circuit) represented by a Z-model:
\[ {\bf Z}=\left[\begin{array}{ll} Z_{11} & Z_{12} \\ 
	Z_{21} & Z_{22} \end{array}\right]
	=\left[\begin{array}{rr} 4\Omega & j3\Omega \\ 
	j3\Omega & 2\Omega \end{array}\right]	\]
The input voltage is $V_0=3\angle 0^\circ$ with an internal impedance 
$Z_0=5\Omega$ and the load impedance is $R_L=4\Omega$. Find the two voltages 
$V_1$, $V_2$ and two currents $I_1$, $I_2$.

\htmladdimg{../figures/networkexample1.gif}

{\bf Method 1:} 
\begin{itemize}
\item First, according the Z-model, we have
\[ \left\{ \begin{array}{l} V_1=Z_{11}I_1+Z_{12}I_2= 4I_1+j3I_2 \\
	V_2=Z_{21}I_1+Z_{22}I_2=j3I_1+ 2I_2 \end{array} \right.	\]
\item Second, two more equations can be obtained from the circuit:
\[ \left\{ \begin{array}{l} V_1=V_0-Z_0 I_1=3-5I_1 \\
	V_2=-Z_L I_2=-4 I_2 \end{array} \right.	\]
\item Substituting the last two equations for $V_1$ and $V_2$ into the 
	first two, we get
\[ \left\{ \begin{array}{l} 9I_1+j3I_2=3 \\ j3I_1+6I_2=0 \end{array} \right. \]
\item Solving these we get 
\[	I_1=\frac{2}{7},\;\;\;\;\;I_2=-\frac{j}{7} \]
\item Then we can get the voltages
\[	V_1=\frac{11}{7},\;\;\;\;\;V_2=\frac{j4}{7} \]
\end{itemize}

{\bf Method 2:} We can also use Thevenin's theorem to treat everything before
the load impedance as an equivalent voltage source with Thevenin's voltage
$V_T$ and resistance $R_T$, and the output voltage $V_2$ and current 
$I_2$ can be found.

\begin{itemize}
\item Find $Z_T=V_2/I_2$ with voltage $V_0$ short-circuit:
  \begin{itemize}
  \item The Z-model:
    \[ \left\{ \begin{array}{l} V_1=Z_{11}I_1+Z_{12}I_2=4I_1+j3I_2	\\
      V_2=Z_{21}I_1+Z_{22}I_2=j3I_1+2I_2 \end{array} \right. \]
  \item Also due to the short circuit of voltage source $V_0=0$, we have
    \[ V_1=V_0-I_1 Z_0=0-5I_1	\]
  \item equating the two expressions for $V_1$, we get
    \[ 4I_1+j3I_2=-5I_1,\;\;\;\;\mbox{i.e.}\;\;\;\;I_1=-\frac{j}{3} I_2	\]
  \item Substituting this $I_1$ into the equation for $V_2$ above, we get
    \[ V_2=j3I_1+2I_2=I_2+2I_2=3I_2 \]
  \item Find $Z_T=V_2/I_2$:
    \[ Z_T=\frac{V_2}{I_2}=3	\]
  \end{itemize}
\item Find open-circuit voltage $V_T$ with $I_2=0$:
  \begin{itemize}
  \item Since the load is an open-circuit, $I_2=0$, we have
    \[ \left\{ \begin{array}{l} V_1=Z_{11}I_1=4I_1 \\
      V_2=Z_{21}I_1=j3 I_1 \end{array} \right. \]
    \item Find $I_1$:
      \[	I_1=\frac{V_0-V_1}{Z_0}=\frac{3-4I_1}{5}	\]
      Solving this to get $I_1=1/3$
    \item Find open-circuit voltage $V_T=V_2$:
      \[	V_2=Z_{21}I_1=j3 \frac{1}{3}=j	\]
  \end{itemize}
\item Find load voltage $V_2$:
  \[ V_2=\frac{V_T}{Z_T+Z_L}\;Z_L=\frac{j}{3+4}\;4=\frac{j4}{7} \]
\item Find load voltage $I_2$:
  \[ I_2=-\frac{V_2}{Z_L}=-\frac{j4}{7}\;\frac{1}{4}=-\frac{j}{7} \]
\end{itemize}


{\bf Principle of reciprocity}:
 
\htmladdimg{../figures/reciprocal.gif}

Consider the example circuit on the left above, which can be simplified 
as the network in the middle. The voltage source is in the branch on the
left, while the current $\dot{I}_2$ is in the branch on the right, which
can be found to be (current divider):
\[ \dot{I}_2=\frac{V}{Z_1+Z_2 Z_3/(Z_2+Z_3)}\;\frac{Z_3}{Z_2+Z_3}
=V \frac{Z_3}{Z_1Z_2+Z_1Z_3+Z_2Z_3} \]
We next interchange the positions of the voltage source and the current, 
so that the voltage source is in the branch on the right and the current 
to be found is in the branch on the left, as shown on the right of the 
figure above. The current can be found to be
\[ \dot{I}_1=\frac{V}{Z_2+Z_1 Z_3/(Z_1+Z_3)}\frac{Z_3}{Z_1+Z_3}
=V \frac{Z_3}{Z_1Z_2+Z_1Z_3+Z_2Z_3} \]
The two currents $\dot{I}_1$ and $\dot{I}_2$ are exactly the same! This
result illustrates the following {\bf reciprocity principle}, which can 
be proven in general:

{\em 
In any passive (without energy sources), 
linear network, if a voltage $V$ applied in branch 1 causes a current $I$ in 
branch 2, then this voltage $V$ applied in branch 2 will cause the same current
$I$ in branch 1.}

This reciprocity principle can also be stated as:

{\em In any passive, linear network, the transfer impedance $Z_{12}$ is equal 
to the reciprocal transfer impedance $Z_{21}$.}

Based on this reciprocity principle, any complex passive linear network can
be modeled by either a T-network or a $\Pi$-network:

\begin{itemize}
  \item {\bf T-Network Model:}
    \htmladdimg{../figures/Tmodel.gif}

    From this T-model, we get
    \[	\left\{ \begin{array}{l} V_1=(Z_1+Z_3)I_1+Z_3I_2=Z_{11}I_1+Z_{12}I_2 \\
      V_2=Z_3I_1+(Z_2+Z_3)I_2=Z_{21}I_1+Z_{22}I_2  \end{array} \right. \]
    Comparing this with the Z-model, we get
    \[ Z_{11}=Z_1+Z_3,\;\;\;\;\;Z_{22}=Z_2+Z_3,\;\;\;\;\;Z_{12}=Z_{21}=Z_3	\]
    Solving these equations for $Z_1$, $Z_2$ and $Z_3$, we get
    \[ Z_1=Z_{11}-Z_{12},\;\;\;\;Z_2=Z_{22}-Z_{21},\;\;\;\;Z_3=Z_{12}=Z_{21} \]

  \item {\bf $\Pi$-Network Model:}
    \htmladdimg{../figures/Pmodel.gif}

    From this $\Pi$-model, we get:
    \[ \left\{ \begin{array}{l} 
      I_1=Y_1V_1+Y_3(V_1-V_2)=(Y_1+Y_3)V_1-Y_3V_2=Y_{11}V_1+Y_{12}V_2 \\
      I_2=Y_3(V_2-V_1)+Y_2V_2=-Y_3V_1+(Y_2+Y_3)V_2=Y_{21}V_1+Y_{22}V_2 \end{array} \right. \]
    Comparing this with the Y-model, we get
    \[ Y_{11}=Y_1+Y_3,\;\;\;\;\;Y_{22}=Y_2+Y_3,\;\;\;\;\;Y_{12}=Y_{21}=-Y_3	\]
    Solving these equations for $Y_1$, $Y_2$ and $Y_3$, we get
    \[ Y_1=Y_{11}+Y_{12},\;\;\;\;Y_2=Y_{22}+Y_{21},\;\;\;\;Y_3=-Y_{12}=-Y_{21} \]
\end{itemize}

{\bf Example 1:} Convert the given T-network to a $\Pi$ network.

\htmladdimg{../figures/TPmodelexample.gif}

{\bf Solution:} Given $Z_1=j5$, $Z_2=-j5$, $Z_3=1$, we get its Z-model:
\[  Z_{11}=Z_1+Z_3=1+j5,\;\;\;\;Z_{22}=Z_2+Z_3=1-j5,\;\;\;\;Z_{12}=Z_{21}=Z_3=1 \]
The Z-model can be expressed in matrix form:
\[	\left[ \begin{array}{l} V_1 \\ V_2\end{array} \right]=
	\left[ \begin{array}{cc} Z_{11} & Z_{12} \\ Z_{21} & Z_{22} \end{array} \right]
	\left[ \begin{array}{l} I_1 \\ I_2\end{array} \right]
=	\left[ \begin{array}{cc} 1+j5 & 1 \\ 1 & 1-j5 \end{array} \right]
	\left[ \begin{array}{l} I_1 \\ I_2\end{array} \right]
\]
This Z-model can be converted into a Y-model:
\[	\left[ \begin{array}{l} I_1 \\ I_2\end{array} \right]=
	\left[ \begin{array}{cc} 1+j5 & 1 \\ 1 & 1-j5 \end{array} \right]^{-1}
	\left[ \begin{array}{l} V_1 \\ V_2\end{array} \right]
=\frac{1}{25}
	\left[ \begin{array}{cc} 1-j5 & -1 \\ -1 & 1+j5 \end{array} \right]
	\left[ \begin{array}{l} V_1 \\ V_2\end{array} \right]
=	\left[ \begin{array}{cc} Y_{11} & Y_{12} \\ Y_{21} & Y_{22} \end{array} \right]
	\left[ \begin{array}{l} V_1 \\ V_2\end{array} \right]
\]
This Y-model can be converted to a $\Pi$ network:
\[ Y_1=Y_{11}+Y_{12}=-j/5,\;\;\;\;Z_2=Y_{22}+Y_{21}=j/5,\;\;\;\;
	Y_3=-Y_{21}=-Y_{12}=1/25 \]
These admittances can be further converted into impedances:
\[ Z_1=1/Y_1=5j,\;\;\;\;Z_2=1/Y_2=-j5,\;\;\;\;Z_3=1/Y_3=25	\]
The same results can be obtained by Y to delta conversion.


{\bf Example 2:} Consider the ideal transformer shown in the figure below. 
Assume $R_0=10\Omega$, $R_L=5\Omega$, and the turn ratio is $n=n_1/n_2=2$. 
Describe this circuit as a two-port network.

\htmladdimg{../figures/transformerexample2.gif}

\begin{itemize}
\item Set up basic equations:
\[	\left\{ \begin{array}{l} 
	V_1=10\;I_1+2\;V_2 \\ I_2-V_2/5=-2\;I_1 \end{array} \right. \]
\item Rearrange the equations in the form of a Z-model. The second equation is
\[	V_2=10\;I_1+5\;I_2	\]
Substituting into the first equation, we get
\[	V_1=30\;I_1+10\;I_2	\]
The Z-model is:
\[	\left[ \begin{array}{l} V_1 \\ V_2 \end{array} \right]=
	\left[ \begin{array}{rr} Z_{11} & Z_{12} \\ Z_{21} & Z_{22} \end{array} \right]
	\left[ \begin{array}{l} I_1 \\ I_2 \end{array} \right]
=	\left[ \begin{array}{rr} 30 & 10 \\ 10 & 5 \end{array} \right]
	\left[ \begin{array}{l} I_1 \\ I_2 \end{array} \right]
\]
As $Z_{12}=Z_{21}=10$, this is a reciprocal network.
\end{itemize}
Alternatively, we can set up the equations in terms of the currents:

\begin{itemize}
\item
\[	\left\{ \begin{array}{l} 
	I_1=(V_1-2\;V_2)/10 \\ I_2=-2\;I_1+V_2/5 \end{array} \right. \]
\item Rearrange the equations in the form of a Y-model. The first equation is
\[	I_1=V_1/10-V_2/5	\]
\item Substituting into the second equation, we get
\[	I_2=-V_1/5+2V_2/5+V_2/5=-V_1/5+3V_2/5	\]
The Y-model is:
\[	\left[ \begin{array}{l} I_1 \\ I_2 \end{array} \right]=
	\left[ \begin{array}{rr} Y_{11} & Y_{12} \\ Y_{21} & Y_{22} \end{array} \right]
	\left[ \begin{array}{l} V_1 \\ V_2 \end{array} \right]
=	\left[ \begin{array}{rr} 1/10 & -1/5 \\ -1/5 & 3/5 \end{array} \right]
	\left[ \begin{array}{l} V_1 \\ V_2 \end{array} \right] \]
\end{itemize}
Finally, we can verify that ${\bf Z}^{-1}={\bf Y}$
\[ {\bf Z}^{-1}=\left[ \begin{array}{rr} 30 & 10 \\ 10 & 5 \end{array} \right]^{-1}
	=\frac{1}{50}\left[ \begin{array}{rr} 5 & -10 \\ -10 & 30 \end{array} \right]
	=\left[ \begin{array}{rr} 1/10 & -1/5 \\ -1/5 & 3/5 \end{array} \right]
	={\bf Y}	\]

\subsection*{Active Circuits}

So far all circuits we have discussed are composed of passive components 
(resistors, capacitors and inductors) driven by current and voltage sources.
In the future we will be considering active components such as bipolar junction
transistors (BJT) and field-effect transistors (FET), operational amplifiers 
(Op-Amps), as well as voltage/current amplification circuits. These active 
components (as simple as single transistors and as complicated as some Op-Amps) 
can be considered as controlled current or voltage sources that generate current 
or voltage depending on the input current or voltage.

For the purpose of describing the overall function and performance of such 
components and circuits (instead of its internal structure and implementation), 
a general model can be used with the following three parameters:
\begin{itemize}
\item {\bf Gain $A$:} The output voltage $v_{out}$ is related to the input voltage
  $v_{in}$ by $v_{out}=A v_{in}$, usually $A>1$.
\item {\bf Input impedance (resistance) $r_{in}$:} It is desirable to have a
  large $r_{in}$ so that little input current is drawn from the source, i.e., 
  the source is minimally affected by the amplifier as a load. Ideally 
  $r_{in}=\infty$. 
\item {\bf Output impedance (resistance) $r_{out}$:} It is desirable to have
  a small $r_{out}$ so that little voltage drop across this resistance will 
  result when the load of the amplifier draws a current from the amplifier, i.e.,
  the load will minimally affect the output voltage of the amplifier.
\end{itemize}

\htmladdimg{../figures/voltageamplifiermodel.gif}

{\bf Example 1:}

\htmladdimg{../figures/voltageamplifierex1.gif}

\[ v_{in}=i_s (R_s || r_{in})=i_s \frac{R_s r_{in}}{R_s+r_{in}} \]
\[ v_{out}=A v_{in} \frac{R_L}{R_L+r_{out}}=
A i_s \frac{R_s r_{in}}{R_s+r_{in}} \frac{R_L}{R_L+r_{out}} \]
Ideally when $r_{in}=\infty$ and $r_{out}=0$, $v_{out}=A i_s R_s$.

{\bf Example 2:}

\htmladdimg{../figures/voltageamplifierex2.gif}
\[ v_{in}=v_s \frac{r_{in}}{R_s+r_{in}} \]
\[ v_{out}=A v_{in} \frac{R_L}{R_L+r_{out}}
=A v_s \frac{r_{in}}{R_s+r_{in}} \frac{R_L}{R_L+r_{out}} \]
The terms ``voltage gain'' $G_V$, ``current gain'' $G_I$, and ``power gain'' $G_P$
need to be specifically defined for different circuit configurations. In this case,
they can be defined as below:
\begin{itemize}
\item {\bf Voltage gain $G_V$:} defined as the ratio of the output voltage
to the source voltage:
\[ G_V=\frac{v_{out}}{v_s}=A\frac{r_{in}}{R_s+r_{in}} \frac{R_L}{R_L+r_{out}} \]
Ideally, when $r_{in}=\infty$, $r_{out}=0$, we have $v_{out}=A v_s$ and $G_V=A$.

\item {\bf Current gain $G_I$:} defined as the ratio of the output current to
the input current:
\[ G_I=\frac{i_{out}}{i_{in}}=\frac{A v_{in}/(r_{out}+R_L)}{v_s/(R_s+r_{in})}
=\frac{A v_s r_{in}/(R_s+r_{in})(r_{out}+R_L)}{v_s/(R_s+r_{in})}
=\frac{A r_{in}}{r_{out}+R_L} \]
Ideally, when $r_{in}=\infty$, $r_{out}=0$, we have $G_I=\infty$.

\item {\bf Power gain $G_P$: } defined as the ratio of the power delivered to the
load to that to the amplifier:
\[ G_P=\frac{v_{out}^2/R_L}{v_{in}^2/r_{in}}=G_V^2 \frac{r_{in}}{R_L} \]
\end{itemize}

The alternative definitions of these voltage, current, and power gains may be
used, depending on the specific applications.

The voltage amplifier can be used as a component (a building block) in a 
larger circuit, such as two-port network with input port between terminals 
A and B and output port between terminals C and D. This network can be in turn 
described in terms of the three parameters, the open-circuit voltage gain,
the input resistance and output resistance, as shown below:
\begin{itemize}
\item {\bf Input resistance $R_{in}$:} This is the resistance between the two 
  terminals A and B of the input port, while a load $R_L$ is connected to the 
  output port between terminals C and D. $R_{in}$ can be obtained as the ratio:
  \[ R_{in}=\frac{\mbox{ideal voltage source $v_s$ applied to the input port}}{\mbox{input current $i_{in}$ through the input port}} \]
  In general $R_{in}$ is affected by the load $R_L$.
\item {\bf Output resistance $R_{out}$:} According to the Thevenin's theorem, 
  any one-port network can be treated as an ideal voltage source $v_T$ in series 
  with a resistance $R_T$. We apply this theorem to the output port and define
  the output resistance as the Thevenin resistance 
  \[ R_{out}=R_T=\frac{\mbox{open-circuit voltage ($v_{oc}=v_T$)}}{\mbox{short-circuit current ($i_{sc}=v_T/R_T$)}} \]
  while a voltage source $v_s$ with an internal resistance $R_s$ is applied to 
  the input port. In general, $R_{out}$ is affected by $R_s$ of the source.
\item {\bf The open-circuit voltage gain $A_{oc}$:} This is the ratio of the 
  open-circuit output voltage $v_{oc}$ to an ideal voltage source $v_s$ without 
  a load ($R_L=\infty$).
  \[ A_{oc}=\frac{\mbox{open-circuit voltage ($v_{oc}$)}}{\mbox{ideal voltage source ($v_s=v_{AB}$)}}\]
\end{itemize}
Note that a non-ideal source with internal resistance $R_s$ is used in the definition
of $R_{out}$ as it is affected by $R_s$, while an ideal source with $R_s=0$ is 
assumed in the definition of $R_{in}$ and $A_{oc}$. In case the source is not
ideal with $R_s\ne 0$, we will use the voltage $v_{AB}$ appearing across the input
port as the input voltage.

{\bf Example 3:} 

\htmladdimg{../figures/voltageamplifierex3.gif}

Find $R_{in}$, $R_{out}$, and $A_{oc}$ of this two-port network containing
$R_1$ and $R_2$ as well as the amplifier modeled by $r_{in}$, $r_{out}$ and $A$. 

\begin{itemize}
\item {\bf Input resistance:} By inspection, the input resistance of this
  2-port network can be found to be $R_{in}=R_1+r_{in}$.

\item {\bf Output resistance:} We assume the internal voltage source is 
  Thevenin voltage $v_T$, and get the open-circuit voltage 
  $v_{oc}=v_T R_2/(r_{out}+R_2)$ and the short-circuit current
  $i_{sc}=v_T/r_{out}$. The output resistance is
  \[ R_{out}=\frac{v_{oc}}{i_{sc}}=\frac{r_{out}R_2}{r_{out}+R_2}=r_{out}||R_2 \]
  Alternatively, $R_{out}$ can be found as the resistance between the two terminals
  C and D of the output port when the voltage source of the amplifier is turned off 
  (short-circuit), i.e., $R_{out}=R_2 || r_{out}$.

\item {\bf Open-circuit voltage gain:} This is the ratio of the voltage $V_{CD}$
  across the output port to the voltage $V_{AB}$ across the input port, when the
  output port is an open circuit, i.e., $R_L=\infty$. 
  \begin{eqnarray}
    v_{in}&=&V_{AB}\frac{r_{in}}{R_1+r_{in}}  \nonumber \\
    V_{CD}&=&A v_{in} \frac{R_2}{r_{out}+R_2}=A V_{AB}\frac{r_{in}}{R_1+r_{in}} 
    \frac{R_2}{r_{out}+R_2} \nonumber \\
    A_{oc}&=&\frac{V_{CD}}{V_{AB}}=\frac{A R_2 r_{in}}{(R_1+r_{in})(r_{out}+R_2)} 
    \nonumber 
  \end{eqnarray}
\end{itemize}
This 2-port network modeled as a voltage amplifier with $R_{in}$, $R_{out}$
and $A_{oc}$ can be used in more complicated circuits.

{\bf Example 4:} 

\htmladdimg{../figures/voltageamplifierex4.gif}

Find the parameters $R_{in}$, $R_{out}$ and $A_{oc}$ of the two-port network
with the voltage amplifier embedded.
\begin{itemize}
\item {\bf Open-circuit voltage gain:} As the output port is open circuit, 
  the output current is zero and so is the voltage drop across $r_{out}$.
  Applying an ideal voltage source $v_s$ to the input, we get the voltage 
  $v_{in}$ across $r_{in}$ and $v_1$ across $R_1$, respectively:
  \[ v_{in}=v_s \frac{r_{in}}{R_1+r_{in}},\;\;\;\;\;
  v_1=v_s \frac{R_1}{R_1+r_{in}} \]
  The voltage across output port is therefore:
  \[ v_{oc}=v_1+A v_{in}=v_s \frac{R_1}{R_1+r_{in}}+Av_s\frac{r_{in}}{R_1+r_{in}}
  =v_s \frac{R_1+A r_{in}}{R_1+r_{in}}  \]
  The open-circuit voltage gain can be found as the ratio of the open-circuit
  output voltage to the input voltage:
  \[ A_{oc}=\frac{v_{oc}}{v_s}=\frac{R_1+A r_{in}}{R_1+r_{in}}<A  \]
  if $R_1=0$, the circuit is reduced to the original voltage amplifier and we
  have $A_{oc}=A$.
\item {\bf Output resistance:} We first find the short-circuit current $i_{sc}$ 
  at the output port. Assume a voltage source $v_s$ with internal resistance 
  $R_s$ is applied to the input port while the output port is short-circuited.
  Applying KVL to the two loops of the circuit, we get:
  \[ v_s=(R_s+R_1+r_{in})i_{in}-R_1 i_{sc} \]
  \[ A v_{in}=A r_{in} i_{in}=(r_{out}+R_1) i_{sc}-R_1 i_{in} \]
  Solving these two equations for the two unknowns $i_{in}$ and $i_{sc}$, we get
  \[ i_{sc}=v_s \frac{Ar_{in}+R_1}{(1-A)R_1 r_{in} +r_{out}(R_s+r_{in}+R_1)
    +R_sR_1} \]
  The open-circuit output voltage $v_{oc}$ is 
  \[ v_{oc}=Ar_{in}i_{in}+R_1i_{in}=(Ar_{in}+R_1)\frac{v_s}{R_s+r_{in}+R_1} \]
  Now the output resistance can be found to be:
  \[ R_{out}=\frac{v_{oc}}{i_{sc}}
  =\frac{(1-A)r_{in}R_1+r_{out}(R_s+r_{in}+R_1)+R_sR_1}{R_s+r_{in}+R_1}
  =r_{out}+\frac{(1-A) r_{in}R_1 +R_sR_1}{R_1+r_{in}+R_s}\]
  Note that $R_{out}$ is affected by internal resistance $R_s$ of the source.
  When $R_s=0$, 
  \[ R_{out}\stackrel{R_s=0}{\Longrightarrow} r_{out}-(A-1) r_{in}||R_1 < r_{out} \]
  i.e., the output resistance is much reduced. Moreover, when $R_1=0$, 
  $R_{out}=r_{out}$.
\item {\bf Input resistance:} This can be found by applying an ideal voltage
  source $v_s$ to the input port, while the output port is connected to a 
  load $R_L$. The input resistance is $R_{in}=v_s/i_{in}$ where $i_{in}$ is
  the input current. Applying the KVL to the two loops of this circuit, we
  get
  \[ v_s=(r_{in}+R_1) i_{in}-R_1 i_{out} \]
  \[ A v_{in}=Ar_{in}i_{in}=(r_{out}+R_L+R_1)i_{out}-R_1i_{in} \]
  Solving these two equations for the two unknowns $i_{in}$ and $i_{out}$,
  we get
  \[ i_{in}=v_s \frac{R_1+R_L+r_{out}}{(R_L+r_{out})(R_1+r_{in})+(1-A)R_1r_{in}}
  \stackrel{R_L\rightarrow \infty}{\Longrightarrow} \frac{v_s}{r_{in}+R_1} \]
  Now the input resistance can be found to be
  \[ R_{in}=\frac{v_s}{i_{in}}=\frac{(R_L+r_{out})(R_1+r_{in})+(1-A)R_1r_{in}}{R_1+R_L+r_{out}}
  \stackrel{R_L\rightarrow \infty}{\Longrightarrow} R_1+r_{in}>r_{in} \]  
  Note that $R_{in}$ is affected by the load $R_L$. When $R_L=\infty$,
  $R_{in}=R_1+r_{in}$, i.e., the input resistance is much increased. 
  Moreover, if $R_1=0$, the circuit is reduced to the original voltage 
  amplifier with $R_{in}=r_{in}$.
\end{itemize}
In summary, the resistor $R_1$ shared by both the input and output loops
serves as a negative feedback:
\[ v_s\uparrow \rightarrow i_{in}, v_{out}\uparrow \rightarrow v_1\uparrow 
\rightarrow i_{in}, v_{out}\downarrow \]
As the result, the voltage gain $A_{oc}$ is reduced but both the input and
output resistances are improved, i.e., $R_{in}$ is increased and the $R_{out}$
is reduced.

{\bf Example 5:} (Homework)

\htmladdimg{../figures/voltageamplifierex5.gif}

A voltage amplifier, denoted by the box (solid line) with three internal 
parameters $r_{in}$, $r_{out}$ and $A$, is used as a component in a two-port
network as shown. Fine the input resistance $R_{in}$, output resistance 
$R_{out}$, and the open-circuit gain $A_{oc}$ of the two-port network.
Note that when considering the input and output resistances of the network,
you should take into account the load $R_L$ and the internal resistance $R_s$
of the source voltage.

\htmladdnormallink{Answer}{../ch2_sub1/index.html}

{\bf Example 6:} (Homework)

\htmladdimg{../figures/voltageamplifierex6.gif}

Two amplifiers with parameters $A_1$, $r_{i1}$, $r_{o1}$ and $A_2$, $r_{i2}$, 
$r_{o2}$, respectively, can be connected in cascade as shown in the figure. 
Given a voltage source $v_s$ in series with an internal resistance $R_s$,
find the output voltage. To maximize the output $v_{out}$, how would you 
change the values of the six parameters?

Find the power gain $G_p$ of the system.

\htmladdnormallink{Answer}{../ch2_sub2/index.html}


{\bf Example 7:} (Homework)

The input and output resistances $R_{in}$ and $R_{out}$, as well as the voltage 
gain $A_{oc}$ of a two-port network can be obtained experimentally. First,
connect an ideal voltage source $v_s$ (a new battery with very low internal
resistance) in series with a resistor $R_s$, and then connect load $R_L$ of
two different resistances to the output port. Now the three parameters can
be derived from the known values of $v_s$, $R_s$ and the two measurements of
the load voltage $v_{out}$, corresponding to the two resistance values used.

Assume $v_s=1.5V$, $R_s=5 k\Omega$, and the input voltage is measured to be 
$v_{in}=1.25 V$; also, assume the two different load resistors used are 
$R_1=150 \Omega$ and $R_2=200 \Omega$ respectively, with the two corresponding
output voltage $v_1=18.75V$ and $v_2=20$. Find $R_{in}$, $R_{out}$ and $A_{oc}$.

\htmladdnormallink{Answer}{../ch2_sub3/index.html}

\end{document}


\subsection*{Two-Ports}

Many circuits can be modeled by the two-port model shown below:
\htmladdimg{../figures/twoportmodel.gif}


	

\subsection*{Examples}
	

