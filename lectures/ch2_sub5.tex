\documentstyle[12pt]{article}
\usepackage{html}

\begin{document}

{\bf Solution for Example 5} (using Thevenin's theorem)

Remove $R$ as the load, model the rest of the circuit as a non-ideal
Thevenin model in terms of $R_T$ and $V_T$. To find $R_T$, we short $V$
and find:
\begin{equation}
  R_T=R_1||(R_3+R_2||R_5)=6||(3+2||2)=6||4=2.4\,\Omega
\end{equation}
To find $V_T$, we first find $V_5$ across $R_5$:
\begin{equation}
  V_5=V\frac{R_5}{R_5+R_2||(R_1+R_3)}=5 \frac{2}{2+2||9}=2.75\,V
\end{equation}
$V_2$ across $R_2$:
\begin{equation}
  V_5=V\frac{R_2||(R_1+R_3)}{R_5+R_2||(R_1+R_3)}=5 \frac{2}{2+2||9}=2.25\,V
\end{equation}
$V_3$ across $R_3$:
\begin{equation}
  V_3=V_2\frac{R_3}{R_1+R_3}=2.25\frac{3}{3+6}=0.75
\end{equation}
Now we get $V_T=V_3+V_5=3.5$. To get $R$ so that the current through it is
$I=0.5$, we solve the equation based on the Thevenin model
\begin{equation}
  \frac{V_T}{R_T+R}=\frac{3.5}{2.4+R}=0.5
\end{equation}
and get $R=4.6\,\Omega$.

{\bf Solution for Example 6}

\begin{itemize}
\item $I_2$ goes through $R_4$ alone with a voltage drop $R_4I_2=10V$.
\item $I_1$ splits between two parallel branches $R_2=15\Omega$ and 
  $R_1+R_3=21\Omega$. The current through $R_1$ is 
  \begin{equation}
    I=I_1\frac{R_2}{R_1+R_2+R_3}=\frac{2\times 15}{15+21}=\frac{5}{6}A
  \end{equation}
\item Voltage drop across $R_1$ is $R_1I=5V$.
\item The open-circuit voltage across the output port is
  $V_T=V_{oc}=10+5+6-9=12V$.
\item The total resistance between the two terminals is
  $R_T=R_{eq}=R_4+R_1||(R_2+R_3)=10+6||(15+15)=15\Omega$.
\end{itemize}


\end{document}





