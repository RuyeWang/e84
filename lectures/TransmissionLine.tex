\documentstyle[11pt]{article}
\usepackage{html}
\begin{document}

{\huge \bf Transmission and Delay Lines}

\section*{The voltage and current wave equations}

As shown in the figure, a transmission line can be modeled by its 
resistance and inductance in series, and the conductance and capacitance
in parallel, all distributed along its length in $x$ direction. Here $R$,
$L$, $G$ and $C$ represent, respectively, the resistance, inductance, 
conductance and capacitance per unit length.

\htmladdimg{../TransmissionLine.gif}

In the following analysis, we will use phasors $V(x)$ and $I(x)$ to represent
the voltage and current, respectively. Note that these variables are also
functions of location $x$ along the transmission line, which takes the value
$x=0$ and $x=l$ at the front and end of the transmission line of length $l$.

The voltage $v(x,t)$ and current $i(x,t)$ along the transmission
line are variables of both time $t$ and 1D space $x$. 
Across an
infinitesimal section $\triangle x$ along the line, the voltage and 
current change from $v$ and $i$ to $v+\triangle v$ and 
$i+\triangle i$, respectively: 
\[ v+\triangle v=v-R\triangle x\; i-L\triangle x \;\frac{\partial i}{\partial t}	\]
\[ i+\triangle i=i-G\triangle x\; v-C\triangle x\; \frac{\partial v}{\partial t}	\]
From these equations, we get
\[
\frac{\partial v}{\partial x}+L \frac{\partial i}{\partial t}+Ri=0
\]
\[
\frac{\partial i}{\partial x}+C \frac{\partial v}{\partial t}+Gv=0
\]
Differentiating the first equation with respect to $t$ and the second one
with respect to $x$, the two equations can be combined to get 
\[	\frac{\partial^2 i}{\partial x^2}-LC\frac{\partial^2 i}{\partial t^2}
	-RC\frac{\partial i}{\partial t}+G\frac{\partial v}{\partial x}=0
\]
Replacing $\partial v/\partial x$ by $-L \partial i/\partial t-Ri$, we get
the {\em transmission line equation} for current $i(x,t)$:
\[	\frac{\partial^2 i}{\partial x^2}=LC\frac{\partial^2 i}{\partial t^2}
	+(RC+GL)\frac{\partial i}{\partial t}+GRi
\]
Similarly, differentiating the first equation above with respect to $x$ and 
the second one with respect to $t$, the two equations can be combined 
to get the {\em transmission line equation} for voltage $v(x,t)$:
\[	\frac{\partial^2 v}{\partial x^2}=LC\frac{\partial^2 v}{\partial t^2}
	+(RC+GL)\frac{\partial v}{\partial t}+GRv
\]
At high frequency, the inductance and capacitance are dominant and the 
resistance and conductance can be ignored, i.e., $R=G=0$. In this case the
transmission line is {\em loss-less} and the above equations become
\[
\frac{\partial^2 i}{\partial x^2}=LC \frac{\partial^2 i}{\partial t^2}=\frac{1}{\mu^2} \frac{\partial^2 v}{\partial t^2}
\]
\[
\frac{\partial^2 v}{\partial x^2}=LC \frac{\partial^2 v}{\partial t^2}=\frac{1}{\mu^2} \frac{\partial^2 v}{\partial t^2}
\]
These high frequency transmission line equations are actually {\em wave 
equations} with the {\em velocity of the waves} defined as:
\[	\mu \stackrel{\triangle}{=} \frac{1}{\sqrt{LC}}	\]

\newpage
\section*{Forward and backward traveling waves}

The above wave equations are partial differential equations (PDEs) of two 
variables $t$ and $x$ and can be solved by Laplace transform method. By
taking Laplace transform of both $v(x,t)$ and $i(x,t)$ with respect to $t$:
\[ 	V(x,s)={\cal L}[v(x,t)]=\int_0^\infty v(x,t)e^{-st}dt	\]
\[ 	I(x,s)={\cal L}[i(x,t)]=\int_0^\infty i(x,t)e^{-st}dt	\]
and assuming zero initial conditions, the two PDEs become ordinary 
differential equations (ODEs) with respect to variable $x$ (with $s$ treated
as a parameter):
\[	\frac{dV}{dx}=-sLI\;\;\;\mbox{and}\;\;\;\;\frac{dI}{dx}=-sCV	\]
Combining these two first order ODEs, we get the voltage wave equation as
a second order ODE:
\[	\frac{d^2 V}{d x^2}-s^2LCV
	=\frac{d^2 V}{d x^2}-(\frac{s}{\mu})^2V=0	\]
with the general solution
\[	V(x,s)=V_b(s)e^{sx/\mu}+V_f(s)e^{-sx/\mu}	\]
where $e^{sx/\mu}$ and $e^{-sx/\mu}$ are the two particular solutions, and 
$V_b(s)$ and $V_f(s)$ are two arbitrary constants (with respect to variable 
$x$), although they are in general functions of the other variable $s$ (or
$t$ in time domain) which is treated as a parameter here. These two constants
can be determined by boundary conditions as discussed later. If $l$ is the 
length of the transmission line, then the time for the wave to travel with 
speed $\mu=1/\sqrt{LC}$ the whole length $l$ of the line is:
\[	T\stackrel{\triangle}{=}\frac{l}{\mu} = l\sqrt{LC} \]
The above solution can now be written as
\[	V(x,s)=V_B(x,s)+V_F(x,s)	\]
where
\[	V_B(x,s)\stackrel{\triangle}{=}V_b(s)e^{ sx/\mu}=V_b(s)e^{ sTx/l} \]
\[	V_F(x,s)\stackrel{\triangle}{=}V_f(s)e^{-sx/\mu}=V_f(s)e^{-sTx/l} \]
Note that $V_b(s)$ and $V_f(s)$ are just the backward and forward voltages
at the front end of the line
\[	V_b(s)=V_B(0,s),\;\;\;\;\mbox{and}\;\;\;\;V_f(s)=V_F(0,s)	\]
The time domain solution can be obtained as
\[
v(x,t)={\cal L}^{-1}[V(x,s)]={\cal L}^{-1}[V_b(x,s)]+{\cal L}^{-1}[V_f(x,s)]
=v_b(t+x/\mu)+v_f(t-x/\mu)	
\]
where
\[ v_b(t)={\cal L}^{-1}[V_b(s)],\;\;\;\;\;v_f(t)={\cal L}^{-1}[V_f(s)] \]
The solution is composed of two voltage waves $v_b(t+x/\mu)$ and $v_f(t-x/\mu)$
traveling along the transmission line in backward and forward directions, 
respectively. Similar result can be obtained for the current equation:
\[	I(x,s)=I_B(x,s)+I_F(x,s)=I_b(s)e^{sx/\mu}+I_f(s)e^{-sx/\mu} \]
or in the time domain
\[	i(x,t)=i_b(t+x/\mu)+i_f(t-x/\mu)	\]

\newpage
\section*{Characteristic Impedance}

Substituting $V(x,s)=V_b(s)e^{sx/\mu}+V_f(s)e^{-sx/\mu}$ and
$I(x,s)=I_b(s)e^{sx/\mu}+I_f(s)e^{-sx/\mu}$ into the equation
\[	\frac{dV}{dx}=-sLI	\]
we have
\[	\frac{s}{\mu}[V_f(s)e^{sx/\mu}-V_b(s)e^{-sx/\mu}]
	=-sL[I_f(s)e^{sx/\mu}+I_b(s)e^{-sx/\mu}]	\]
or 
\[ [\frac{s}{\mu}V_b(s)+sLI_b(s)]e^{sx/\mu}=
	[\frac{s}{\mu}V_f(s)-sLI_f(s)]e^{-sx/\mu}	\]
For this equation to hold for all $x$, the coefficients of the
exponentials have to be zero:
\[	\frac{s}{\mu}V_b(s)+sLI_b(s)=\frac{s}{\mu}V_f(s)-sLI_f(s)=0 \]
From these equations we can find the {\em characteristic impedance} 
of the transmission line defined as the ratio between forward voltage 
and current
\[	Z_0(s)\stackrel{\triangle}{=}\frac{V_f(s)}{I_f(s)}
	=L\mu=\frac{L}{\sqrt{LC}}=\sqrt{\frac{L}{C}}	\]
and the ratio between backward voltage and current is
\[	\frac{V_b(s)}{I_b(s)}=-L\mu=-\frac{L}{\sqrt{LC}}
	=-\sqrt{\frac{L}{C}}=-Z_0(s)	\]
We also have
\[	\frac{V_F(x,s)}{I_F(x,s)}
	=\frac{V_f(s)e^{-sTx/l}}{I_f(s)e^{-sTx/l}}
	=\frac{V_f(s)}{I_f(s)}=Z_0(s)	\]
\[	\frac{V_B(x,s)}{I_B(x,s)}
	=\frac{V_b(s)e^{-sTx/l}}{I_b(s)e^{-sTx/l}}
	=\frac{V_b(s)}{I_b(s)}=-Z_0(s)	\]
Note that the characteristic impedance $Z_0(s)$ is solly determined by the
distributive inductance $L$ and capacitance $C$ of the transmission
line, and it is independent of the total length $l$ of the line.

\newpage
\section*{Termination}

The transmission line is typically used to connect a voltage source
$v_0(t)$ or $V_0(s)$ with output (internal) impedance $Z_1(s)$ and a
load of input impedance $Z_2(s)$, as shown in the figure. 

\begin{itemize}
\item At the front of the line ($x=0$), we have
\[	V_1(s)=V_0(s)-I_1(s)Z_1(s)	\]
but as
\[ V_1(s)=V(x,s)|_{x=0}=[V_b(s)e^{sTx/l}+V_f(s)e^{-sTx/l}]_{x=0}
	=V_b(s)+V_f(s)	\]
\[ I_1(s)=I(x,s)|_{x=0}=[I_b(s)e^{sTx/l}+I_f(s)e^{-sTx/l}]_{x=0}
	=I_b(s)+I_f(s)	\]
we have
\[ V_b(s)+V_f(s)=V_0(s)-[I_b(s)+I_f(s)]Z_1(s)
	=V_0(s)-[-V_b(s)+V_f(s)]\frac{Z_1(s)}{Z_0(s)}	\]
Solving for $V_f(s)$, we have
\[
V_f(s)=\frac{Z_0(s)}{Z_0(s)+Z_1(s)}V_0(s)+\frac{Z_1(s)-Z_0(s)}{Z_1(s)+Z_0(s)}V_b(s)
=A(s)V_0(s)+\eta_1(s)V_b(s)	\]
where
\[	A(s)\stackrel{\triangle}{=}\frac{Z_0(s)}{Z_0(s)+Z_1(s)}	\]
is the {\em voltage attenuation ratio} describing the transmission 
line and the output impedance of the source as a voltage divider, and
\[ \eta_1(s)\stackrel{\triangle}{=}\frac{Z_1(s)-Z_0(s)}{Z_1(s)+Z_0(s)} \]
is the {\em reflection coefficient at the front of the line}
representing how much of the backward voltage wave is reflected at
the front. Note, in particular, that when the output impedance of the
source is the same as the characteristic impedance
\[	Z_1(s)=Z_0(s)	\]
the backward voltage is not reflected at the front end.

\item At the end of the transmission line ($x=l$), we have
\[	V_2(s)=I_2(s)Z_2(s)	\]
but as
\[ V_2(s)=V(x,s)|_{x=l}=[V_b(s)e^{sx/\mu}+V_f(s)e^{-sx/\mu}]_{x=l}
	=V_b(s)e^{sT}+V_f(s)e^{-sT}	\]
\[ I_2(s)=I(x,s)|_{x=l}=[I_b(s)e^{sx/\mu}+I_f(s)e^{-sx/\mu}]_{x=l}
	=I_b(s)e^{sT}+I_f(s)e^{-sT}	\]
we have
\[ V_b(s)e^{sT}+V_f(s)e^{-sT}=[I_b(s)e^{sT}+I_f(s)e^{-sT}]Z_2(s)
   =[-V_b(s)e^{sT}+V_f(s)e^{-sT}]\frac{Z_2(s)}{Z_0(s)} \]
From this we can find the ratio between the backward voltage and
the forward voltage at the end of the line:
\[	\eta_2(s)\stackrel{\triangle}{=}\frac{V_B(l,s)}{V_F(l,s)}
	=\frac{V_be^{sT}}{V_fe^{-sT}}
	=\frac{Z_2(s)-Z_0(s)}{Z_2(s)+Z_0(s)}		\]
which is defined as the {\em reflection coefficient at the end of the line}
representing how much of the forward voltage wave is reflected at the 
end. Note, in particular, that when the load impedance is equal to the 
characteristic impedance
\[	Z_2(s)=Z_0(s)	\]
the forward voltage is not reflected and the backward voltage is zero.
\end{itemize}

\newpage
\section*{Determination of $V_b(s)$ and $V_f(s)$}

Summarizing the above, we have two simultaneous equations with 
$V_b(s)$ and $V_f(s)$ as two unknowns:
\[ \left\{ \begin{array}{ll}
V_f(s)=A(s)V_0(s)+\eta_1(s)V_b(s) & \mbox{(at the front $x=0$)} \\
V_b(s)e^{sT}=\eta_2(s)V_f(s)e^{-sT} & \mbox{(at the end $x=l$)}
	\end{array} \right. \]
Given the voltage source $V_0(s)$, we can solve these simultaneous
equations to get the forward and backward voltage waves
\[ V_f(s)=\frac{A(s)}{1-\eta_1(s)\eta_2(s)e^{-2sT}}V_0(s) \]
\[ V_b(s)=\frac{A(s)\eta_2(s)e^{-2sT}}{1-\eta_1(s)\eta_2(s)e^{-2sT}}V_0(s) \]
Moreover, the voltages at the two ends of the transmission can be 
obtained. At the front, we have
\[ V_1(s)=V_b(s)+V_f(s)=\frac{A(s)[1+\eta_2(s)e^{-2sT}]}{1-\eta_1(s)\eta_2(s)e^{-2sT}}V_0(s) = T_1(s)V_0(s) \]
where $T_1(s)$ is the {\em voltage transfer function at the front} 
defined as
\[ T_1(s)\stackrel{\triangle}{=}\frac{A(s)[1+\eta_2(s)e^{-2sT}]}{1-\eta_1(s)\eta_2(s)e^{-2sT}}	\]
At the end, we have
\[  V_2(s)=V_b(s)e^{sT}+V_f(s)e^{-sT}=\frac{A(s)[1+\eta_2(s)]e^{-sT}}{1-\eta_1(s)\eta_2(s)e^{-2sT}}V_0(s) = T_2(s)V_0(s) \]
where $T_2(s)$ is the {\em voltage transfer function at the end} 
defined as
\[ T_2(s)\stackrel{\triangle}{=}\frac{A(s)[1+\eta_2(s)]e^{-sT}}{1-\eta_1(s)\eta_2(s)e^{-2sT}}	\]
The input current is
\begin{eqnarray}
I_1(x,s) &=& I(x,s)|_{x=0}=I_b(s)+I_f(s)
	\nonumber \\
 & = &\frac{-V_b(s)+V_f(s)}{Z_0(s)}=\frac{A(s)[1-\eta_2(s)e^{-2sT}]}{1-\eta_1(s)\eta_2(s)e^{-2sT}}
	\frac{V_0(s)}{Z_0(s)}	\nonumber 
\end{eqnarray}
And the {\em input impedance} of the transmission line can be 
obtained:
\[
Z_{in}(s)\stackrel{\triangle}{=}\frac{V_1(s)}{I_1(s)}=\frac{1+\eta_2e^{-2sT}}{1-\eta_2e^{-2sT}}
Z_0(s) \]

\newpage
{\large
\section*{Summary}
\begin{itemize}
\item {\bf Characteristic impedance:}
\[	Z_0(s)=\sqrt{L/C}	\]
\item {\bf Velocity of traveling wave:}
\[	\mu = \frac{1}{\sqrt{LC}}	\]
\item {\bf Voltage attenuation ratio:}
\[ A(s)=\frac{Z_0(s)}{Z_0(s)+Z_1(s)}	\]
\item {\bf Reflection coefficient at front:}
\[ \eta_1(s)=\frac{Z_1(s)-Z_0(s)}{Z_1(s)+Z_0(s)} \]
\item {\bf Reflection coefficient at end:}
\[ \eta_2(s)=\frac{Z_2(s)-Z_0(s)}{Z_2(s)+Z_0(s)} \]
\item {\bf Voltage transfer function at front:}
\[ T_1(s)=\frac{A(s)[1+\eta_2(s)e^{-2sT}]}{1-\eta_1(s)\eta_2(s)e^{-2sT}} \]
\item {\bf Voltage transfer function at end:}
\[ T_2(s)=\frac{A(s)[1+\eta_2(s)]e^{-sT}}{1-\eta_1(s)\eta_2(s)e^{-2sT}} \]
\item {\bf Input impedance:}
\[ Z_{in}(s)=\frac{1+\eta_2e^{-2sT}}{1-\eta_2e^{-2sT}}Z_0(s) \]
\end{itemize}
}
\newpage
\section*{Physical interpretation}

To find the physical interpretation of the voltage transfer functions
$T_1$ and $T_2$ at both ends of the line, we first consider their common
denominator. Assuming $Re[s]>0$ (and also $|\eta_1|<1$ and $|\eta_2|<1$), 
we have $|\eta_1\eta_2e^{-2sT}|<1$ and the common denominator can be 
expanded to become:
\[	\frac{1}{1-\eta_1\eta_2e^{-2sT}}=1+\eta_1\eta_2e^{-2sT}
	+(\eta_1\eta_2e^{-2sT})^2+(\eta_1\eta_2e^{-2sT})^3+\cdots 
\]
The voltage at front of the line is
\[ V_1=T_1V_0=AV_0(1+\eta_2e^{-s2T})[1+\eta_1\eta_2e^{-2sT}
	+(\eta_1\eta_2)^2e^{-4sT}+\cdots ]	\]
Here $AV_0$ is the voltage first entering the line at $t=0$, the term 
$(1+\eta_2 e^{-s2T})$ represents the sum of the forward wave and backward
wave reflected by the end of the line arriving at front after $2T$ time
delay. The general term $(\eta_1\eta_2)^ke^{-2ksT}$ ($k=1,2,3, \cdots$) 
inside the brackets represents the signal arriving at the front of the line
after traveling forward and backward $k$ times along the line and being 
reflected $k$ times at both ends. When the impedances are not well matched 
($Z_0\ne Z_1$, $Z_0\ne Z_2$), there will be multiple (infinite in principle)
reflections in the transmission line.

The output voltage at end of the line is 
\begin{eqnarray}
V_2 &=& T_2V_0=AV_0(1+\eta_2)e^{-sT}[1+\eta_1\eta_2e^{-2sT}
	+(\eta_1\eta_2)^2e^{-4sT}+\cdots ]
	\nonumber \\
 &=& AV_0 [ (1+\eta_2)e^{-sT}+(1+\eta_2)\eta_1\eta_2e^{-3sT}+
	(1+\eta_2)(\eta_1\eta_2)^2e^{-5sT}+ \cdots ]
	\nonumber 
\end{eqnarray}
Again $AV_0$ is the voltage first entering the line at $t=0$. The term 
$(1+\eta_2)$ represents the sum of the forward wave and its reflection at
the end of the line, and the factor $e^{-sT}$ represents the delay time for 
the wave to travel forward from front to end of the line. The meaning of the
general term $(\eta_1\eta_2)^ke^{-2ksT}$ is the same as above. 

\newpage
\section*{Examples}

\begin{itemize}
\item Assume the output impedance of the voltage source, the input
	impedance of the load and the characteristic impedance of
	the transmission line are all the same $Z_1=Z_2=Z_0$. Then we
	have:

\[ A(s)=\frac{Z_0(s)}{Z_0(s)+Z_1(s)}=\frac{1}{2}   \]	

\[ \eta_1(s)=\frac{Z_1(s)-Z_0(s)}{Z_1(s)+Z_0(s)}=0 \]

\[ \eta_2(s)=\frac{Z_2(s)-Z_0(s)}{Z_2(s)+Z_0(s)}=0 \]

\[ T_1(s)=\frac{A(s)[1+\eta_2(s)e^{-2sT}]}{1-\eta_1(s)\eta_2(s)e^{-2sT}}=A(s)=\frac{1}{2}	\]

\[ T_2(s)=\frac{A(s)[1+\eta_2(s)]e^{-sT}}{1-\eta_1(s)\eta_2(s)e^{-2sT}}=A(s)e^{-sT}=\frac{1}{2}e^{-sT}	\]

\[ Z_{in}(s)=\frac{1+\eta_2e^{-2sT}}{1-\eta_2e^{-2sT}}Z_0(s)=Z_0(s) \]

Under a step input 
\[	V_0(s)=\frac{V}{s}	\]
i.e.,
\[	v_0(t)=V u(t)=\left\{ \begin{array}{ll}0 & t<0 \\ V & t \ge 0
	\end{array} \right. \]
we have
\[	V_1(s)=T_1(s)V_0(s)=\frac{V}{2}	\]
i.e., 
\[	v_1(t)={\cal L}^{-1}[V_1(s)]=\frac{V}{2}u(t)	\]
and
\[	V_2(s)=T_2(s)V_0(s)=\frac{V}{2}e^{-sT}	\]
i.e.,
\[	v_2(t)={\cal L}^{-1}[V_2(s)]=\frac{V}{2}u(t-T)	\]
The load and the transmission line behave like a voltage divider with
a pure delay of $T$. In particular, if the internal impedance is zero
$Z_1=0$, we have $A(s)=1$, $\eta_1(s)=-1$, $\eta_2(s)=0$, $T_1(s)=1$,
$T_2(s)=e^{-sT}$, $Z_{in}(s)=Z_0(s)$, and consequently, 
\[	V_1(s)=T_1(s)V_0(s)=V_0(s)	\]
\[	V_2(s)=T_2(s)V_0(s)=V_0(s)e^{-sT}	\]
i.e., the input voltage $V_0(s)$ is transmitted to the output (the 
load) without distortion but a pure delay of $T$. 

\item Assume $Z_1(s)=Z_0(s)$ but the load is an open circuit, i.e., 
$Z_2(s)\rightarrow \infty$, we have
\[	A(s)=\frac{1}{2} \]
\[	\eta_1(s)=0	\]
\[	\eta_2(s)=1	\]
\[	T_1(s)=\frac{1}{2}(1+e^{-2sT})	\]
\[	T_2(s)=e^{-sT}	\]
\[ Z_{in}(s)=\frac{1+e^{-2sT}}{1-e^{-2sT}}Z_0(s)=\frac{Z_0(s)}{tanh(sT)} \]
In particular, if $|sT| << 1$, we have
\[ Z_{in}(s)\approx \frac{Z_0(s)}{sT}=\frac{\sqrt{L/C}}{sl\sqrt{LC}}
	=\frac{1}{slC}=\frac{1}{sC_{line}}	\]
i.e., if $T$ is short compared to times of interest (pulse width or
period of signal), the open circuit load behaves like a capacitor of
capacitance equal to that of the entire line $C_{line}=lC$.

Under the same step input $V_0(s)=V/s$, or $v_0(t)=Vu(t)$, we have
\[	V_2(s)=T_2(s)V_0(s)=\frac{V}{s}e^{-sT}	\]
or
\[	v_2(t)={\cal L}^{-1}[V_2(s)]=Vu(t-T)	\]
i.e., the output is just a delayed version of the input. But at the
front
\[	V_1(s)=T_1(s)V_0(s)=\frac{V}{2s}(1+e^{-2sT})	\]
or
\[	v_1(t)={\cal L}^{-1}[V_1(s)]=\frac{V}{2}[u(t)+u(t-2T)]	\]
The second term $u(t-2T)$ is the reflection at the end of the line
with coefficient $\eta_2(s)=1$ which arrives at the front after
the signal traveling along the line forward and backward in $2T$ time.
But as $\eta_1(s)=0$, it is no longer reflected at the front.

\item Assume $Z_1(s)=Z_0(s)$ but the load is a short circuit, i.e., 
$Z_2(s)=0$, we have
\[	A(s)=\frac{1}{2} \]
\[	\eta_1(s)=0	\]
\[	\eta_2(s)=-1	\]
\[	T_1(s)=\frac{1}{2}(1-e^{-2sT})	\]
\[	T_2(s)=0	\]
\[ Z_{in}(s)=\frac{1-e^{-2sT}}{1+e^{-2sT}}Z_0(s)=Z_0(s)\;tanh(sT) \]
In particular, if $|sT| << 1$, we have
\[ Z_{in}(s)\approx Z_0(s) \;sT=\sqrt{L/C}sl\sqrt{LC}
	=slL=sL_{line}	\]
i.e., if $T$ is short compared to times of interest, the short
circuit load behaves like an inductor of inductance equal to that of 
the entire line $L_{line}=lL$.

Under the same step input $V_0(s)=V/s$, or $v_0(t)=Vu(t)$, we have
\[	V_2(s)=T_2(s)V_0(s)=0	\]
i.e., the output is zero due to the short circuit load. But at the
front
\[	V_1(s)=T_1(s)V_0(s)=\frac{V}{2s}(1-e^{-2sT})	\]
or
\[	v_1(t)={\cal L}^{-1}[V_1(s)]=\frac{V}{2}[u(t)-u(t-2T)]	\]
The second term $-u(t-2T)$ is the reflection at the end of the line
with coefficient $\eta_2(s)=-1$ which arrives at the front after
the signal traveling along the line forward and backward in $2T$ time.
As the reflected signal $-u(t-2T)$ cancels the input $u(t)$ after $2T$
time, the step input becomes a pulse of width $2T$ at the front of
the line.

\item Assume $Z_1(s)=Z_0(s)$ but the load is a pure inductance, i.e.,
$Z_2(s)=sL_2$, we have
\[	A(s)=\frac{1}{2} \]
\[	\eta_1(s)=0	\]
\[	\eta_2(s)=\frac{sL_2-Z_0(s)}{sL_2+Z_0(s)}
	=\frac{s-1/\tau}{s+1/\tau} \]
where $\tau\stackrel{\triangle}{=}sL_2/Z_0(s)$ is the time constant of the
circuit. 
\[	T_1(s)=\frac{1}{2}[1+\frac{s-1/\tau}{s+1/\tau} e^{-2sT}] \]
\[	T_2(s)=\frac{s}{s+1/\tau}e^{-sT} \]

The response at the front of the line to a step input 
$v_0(t)=Vu(t)$ or $V_0(s)=V/s$ is
\begin{eqnarray}
V_1(s) & = & T_1(s)V_0(s)
	=\frac{1}{2}[1+\frac{s-1/\tau}{s+1/\tau}e^{-2sT}]\frac{V}{s}
	\nonumber \\
 & = & \frac{V}{2s}[1+(\frac{2s}{s+1/\tau}-1)e^{-2sT}]
  =\frac{V}{2s}(1-e^{-2sT})+\frac{V}{s+1/\tau}e^{-2sT}
	\nonumber 
\end{eqnarray}
or in time domain
\[	v_1(t)={\cal L}^{-1}[V_1(s)]=\frac{V}{2}[u(t)-u(t-2T)]+Ve^{-(t-2T)/\tau}u(t-2T) \]
The response at the end of the line is
\[ 	V_2(s)=T_2(s)V_0(s)=\frac{s}{s+\tau}e^{-sT} \frac{V}{s}
	=\frac{V}{s+\tau}e^{-sT} \]
or in time domain
\[	v_2(t)={\cal L}^{-1}[V_2(s)]=Ve^{-(t-T)/\tau}u(t-T)	\]
We see that this is a {\em delayed differentiator} with delay time $T$. 

%The input impedance is
% \[ Z_{in}(s)=\frac{sL_2(1+e^{-2sT})+Z_0(s)(1-e^{-2sT})}
%	{sL_2(1-e^{-2sT})+Z_0(s)(1+e^{-2sT})}Z_0(s)	\]

\item So far we have only considered some special cases where either
$\eta_1(s)=0$ or $\eta_2(s)=0$ (reflection at either or both ends is
zero). As the result, the denominator of $T_1(s)$ and $T_2(s)$ is always 1.
Now we consider the general case where $\eta_1(s)\ne 0$ and $\eta_2(s) 
\ne 0$. Assume $Z_1=Z_0/2$ and $Z_2=2Z_0$, and an impulse input
\[	v_0(t)=u(t)-u(t-T/3)	\]
then we have
\[ A=\frac{Z_0}{Z_1+Z_0}=\frac{2Z_1}{Z_1+2Z_1}=\frac{2}{3} \]
\[ \eta_1=\frac{Z_1-Z_0}{Z_1+Z_0}=\frac{Z_1-2Z_1}{Z_1+2Z_1}=-\frac{1}{3} \]
\[ \eta_2=\frac{Z_2-Z_0}{Z_2+Z_0}=\frac{2Z_0-Z_0}{2Z_0+Z_0}=\frac{1}{3}  \]
The output 
\[	V_2=V_0A(1+\eta_2)e^{-sT}[1+\eta_1\eta_2 e^{-2sT}
	+(\eta_1\eta_2)^2 e^{-4sT}+\cdots ]	\]
where
\[	1+\eta_2=1+\frac{1}{3}=\frac{4}{3}	\]
\[ 	\eta_1 \eta_2=-\frac{1}{9}	\]
\begin{itemize}
\item $t=T:$ 
\[	v_2=\frac{4}{3}Av_0=\frac{8}{9}v_0	\]
\item $t=3T:$ 
\[ v_2=\frac{4}{3}A(-\frac{1}{9})v_0=-\frac{4}{27}Av_0=-\frac{8}{81}v_0 \]
\item $t=5T:$ 
\[ v_2=\frac{4}{3}A(\frac{1}{81})v_0=\frac{4}{243}Av_0=\frac{8}{729}v_0	\]
\item
\[	\cdots  \cdots  \cdots  \cdots  \]
\end{itemize}
Note that as the signal only has a finite $T/3$ duration, at any time $t=2kT$
only the term $(\eta_1\eta_2)^k e^{-2ksT}$ representing the wave arriving at
the end most recently is non-zero.

\end{itemize}

\newpage
\section*{Attenuation}

The above LC model of the transmission line above is based on the 
assumption that the line is loss-less, i.e., $R=G=0$. In reality the 
signal is always attenuated due to the resistance $R$ in series with 
the inductance $L$ and the conductance $G$ in parallel with the 
capacitance $C$.($G=1/R_l$, where $R_l$ is the leakage resistance of
the capacitor.) The two partial differential equations describing the
line become
\[
\frac{\partial v}{\partial x}+L \frac{\partial i}{\partial t}+Ri=0
\]
\[
\frac{\partial i}{\partial x}+C \frac{\partial v}{\partial t}+Gv=0
\]
In Laplace domain, these equations are
\[
\frac{dV(x,s)}{dx}=-(R+sL)I(x,s)=-sL'I(x,s)
\]
\[
\frac{dI(x,s)}{dx}=-(G+sC)V(x,s)=-sC'V(x,s)
\]
where
\[	L'\stackrel{\triangle}{=}L(1+\frac{R}{sL})	\]
\[	C'\stackrel{\triangle}{=}C(1+\frac{G}{sC})	\]
Based on these definitions, the above equations take the same form
as that of the loss-less model (with $L$ and $C$ replaced by $L'$ and
$C'$ respectively) and can be solved in the same way. Specifically,
we have
\[
T=l\sqrt{L'C'}=l\sqrt{LC}\sqrt{(1+\frac{R}{sL})(1+\frac{G}{sC})}
\]
and
\[ Z_0(s)=\sqrt{\frac{L'}{C'}}=\sqrt{\frac{L(1+R/sL)}{C(1+G/sC)}}
\]
If $R<<sL$, $G<<sC$, we have
\[ T\approx l\sqrt{LC}[1+(\frac{R}{L}+\frac{G}{C})\frac{1}{2s}]	\]
As before, the solution of the above equations is
\[	V(x,s)=V_b(s)e^{sxT/l}+V_f(s)e^{-sxT/l}
	\approx V_b(s)e^{\alpha x}e^{s\sqrt{LC}x}
	+V_f(s)e^{-\alpha x}e^{-s\sqrt{LC}x}
\]
where $\alpha$ is the {\em attenuation constant} defined as:
\[
\alpha\stackrel{\triangle}{=}\frac{R}{2}\sqrt{\frac{C}{L}}+
	\frac{G}{2}\sqrt{\frac{L}{C}}
\]
Note that the forward wave $V_f(s)$ attenuates exponentially as
$x$ increases from 0 to $l$, while the backward wave $V_b(s)$
attenuates exponentially as $x$ decreases from $l$ to 0.


\end{document}




The inductance and capacitance 
associated with an infinitesimal section $dx$ of the transmission 
line are $Ldx$ and $Cdx$, respectively, i.e., the total inductance
and capacitance of the line are
\[	L_{line}=\int_l Ldx=lL	\]
and
\[	C_{line}=\int_l Cdx=lC	\]
