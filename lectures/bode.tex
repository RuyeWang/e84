\documentstyle[12pt]{article}
\usepackage{html}
\begin{document}

\section*{Logarithmic and Complex operations}

Assume $w=u+jv=|w|\angle w=|w|e^{j\angle w}$, and $z=x+jy=|z|\angle z=|z|e^{j\angle z}$. 
\begin{itemize}
  \item 
    \[  \left\{ \begin{array}{l}
      |w|=\sqrt{u^2+v^2},\;\;\;\;\angle w=\tan^{-1}\left( v/v \right) \\
      |z|=\sqrt{x^2+y^2},\;\;\;\;\angle z=\tan^{-1}\left( y/x \right) \end{array} \right. \]
  \item 
    \[ w z=(u+jv)(x+jy)=|w|e^{j\angle w}\;|z|e^{j\angle z} \]
    \[ \left| wz \right| =|w||z|,\;\;\;\;\;
    e^{j\angle w} e^{j\angle z}=e^{j(\angle w+\angle z)},
    \;\;\;\mbox{or}\;\;\;\;
    \angle \left( wz \right) =\angle w+\angle z \]
  \item 
    \[ \frac{w}{z}=\frac{u+jv}{x+jy}=\frac{|w|e^{j\angle w}}{|z|e^{j\angle z}} \]
    \[ \left| \frac{w}{z} \right| =\frac{|w|}{|z|},\;\;\;\;\;
    \frac{e^{j\angle w}}{e^{j\angle z}}=e^{j(\angle w-\angle z)},\;\;\;\mbox{or}\;\;\;
    \angle \left( \frac{w}{z} \right)=\angle w-\angle z  \]
\end{itemize}

\begin{itemize}
  \item \[ \log \;(ab)=\log a+\log b \]
  \item \[ \log \;(a/b)=\log a-\log b \]
  \item \[ \log \;(a^n)=n\;\log a \]
  \item \[ \log \;(a^{-n})=-n\;\log a \]
\end{itemize}

\section*{Impedance}

In DC circuits the relationships of various voltages and currents are described 
by a set of linear algebraic equations, while in AC circuits they are described
by a set of linear differential equations. 

The resistance $R$ of a resistor is defined by Ohm's law as the ratio of voltage 
$v$ across a resistor and current $i$ through the resistor: $ R=v/i$.
This concept is generalized to that of impedance $Z$ of any element (L, C, as
well R) in AC circuits, defined as the frequency response function of the element
with the current through the element as the input and the voltage across the element
as the output:
\[
Z=\frac{V}{I}=\frac{v_m e^{j\phi}}{i_m e^{j\psi}}=\frac{v_m}{i_m} e^{j(\phi-\psi)} 
\]
This is the generalized Ohm's law, which represents impedance $Z$ as the ration 
of the phasor voltage $V$ across an element and the phasor current $I$ through it.
In other words, impedance $Z$ represents (a) the phase difference between $V$ and 
$I$ as well as (b) the ratio of their amplitudes.


\begin{itemize}
\item Inductor $L$: Assume the current through $L$ is $I e^{j\omega t}$, 
  then the voltage across L is:
  \[ V e^{j\omega t}=L\frac{d}{dt} [I e^{j\omega t}]=j\omega L I e^{j\omega t}
  \;\;\;\;\mbox{i.e.}\;\;\;\;\;\;Z_L=\frac{V}{I}=\frac{j\omega LI}{I}=j\omega L \]
\item Capacitor $C$: Assume the voltage across $C$ is $V e^{j\omega t}$, 
  then the current through C is
  \[ I e^{j\omega t}=C\frac{d}{dt} [V e^{j\omega t}]=j\omega C V e^{j\omega t}
  \;\;\;\;\mbox{i.e.}\;\;\;\;\;\;Z_C=\frac{V}{I}=\frac{V}{j\omega CV}=\frac{1}{j\omega C} \]
\item Resistor $R$: the voltage across R and current through R are in phase, therefore:
  \[ Z_R=\frac{V}{I}=\frac{v_me^{j\phi}}{i_me^{j\phi}}=\frac{v_m}{i_m}=R \]
\end{itemize}

All familiar laws such as Ohm's law, KCL and KCL, current divider and voltage
divider, can be generalized and applied to the analysis of AC circuit containing
elements such as L, C as well as R.

\section*{Decibel (dB)}

The bel (B) is a unit of measurement for the ratio of a physical quantity 
(power, intensity, magnitude, etc.) and a specified or implied reference 
level in base-10 logarithm. As it is a ratio of two quantities with the same 
unit, it is dimensionless. 

For example, consider a power amplifier with input signal power $P_{in}=100\; mW$ 
and output signal power $P_{out}=10\;W$, then the power gain of the amplifier is 
$P_{out}/P_{in}=100$, which can be more concisely expressed in base-10 log scale:
\[
L_B=\log_{10} \frac{P_{out}}{P_{in}}=\log_{10} \frac{10}{0.1} =\log_{10} 100 =2\; bel(B)
\]
The unit bel (B) was first used in early 1920's in honor of 
\htmladdnormallink{Alexander Bell (1847 -- 1922)}{http://en.wikipedia.org/wiki/Alexander_Graham_Bell},
a telecommunication pioneer and founder of the Bell System (the Bell Labs).

As bel (B) is often too big a unit (a gain of 100 is only 2 B), a smaller
unit of decibel (dB), 1/10 of the unit bel (B), is more widely used instead. 
Now the power gain above can be expressed as:
\[ L_B=\log_{10}\frac{P_{out}}{P_{in}}=\log_{10} 100 =2\;B=20\;dB,
\;\;\;\;\mbox{or}\;\;\;\;\;
L_{dB}=10 \log_{10}\frac{P_{out}}{P_{in}}=20\;dB \]
Similarly a 1,000 fold power gain is expressed as:
\[ L_B=\log_{10}\frac{P_{out}}{P_{in}}=\log_{10} 1000 =3\;B=30\;dB,
\;\;\;\;\mbox{or}\;\;\;\;
L_{dB}=10 \log_{10}\frac{P_{out}}{P_{in}}=30\;dB \]
Given the input power $P_{in}$ and the power gain in decibel, e.g., $L_}dB}=30\;dB$, 
the output power can be obtained as:
\[ \frac{P_{out}}{P_{in}}=10^{L_{dB}/10}=10^{30/10}=10^3,
\;\;\;\;\mbox{i.e.}\;\;\;\;P_{out} =10^3\;P_{in}=1,000\;P_{in} \]

As another example, the sound level is measured in decibel, in terms of the ratio
of the sound intensity (power per area, e.g., $W/m^2$) and the threshold of human 
hearing ($10^{-12}\; W/m^2$) as the reference. The human hearing has a large range 
from 0 dB (threshold) to 140 dB (military jet takeoff, $10^{14}$ times the threshold, 
i.e., $10^2\; W/m^2$). 160 dB sound level will cause instant membrane/eardrum 
perforation.

In general, power (and energy) is always proportional to the amplitude of certain
quantity squared (e.g., $P=V^2/R=I^2 R$, $E=mv^2/2$, $E=kx^2/2$). Therefore a 
different definition is used for ratios between two amplitudes, for example, the
output and input voltages $V_{out}$ and $V_{in}$ of a voltage amplifier, we have:
\[ L_{dB}=10 \log_{10} \frac{V^2_{out}}{V^2_{in}}
=20 \log_{10} \frac{V_{out}}{V_{in}}\;dB \]
If the input to a voltage amplifier is 10 mV and the output voltage is 1 V, then the
voltage gain in terms of decibel is:
\[ 20 \log_{10} \frac{V_{out}}{V_{in}}=20 \log_{10} \frac{1,000}{10}=40\; dB \]
If the output voltage is 10 V, then
\[ 20 \log_{10} \frac{V_{out}}{V_{in}}=20 \log_{10} \frac{10,000}{10}=60\; dB \]
We see that the difference of one order of magnitude in the gain corresponds to 20 dB.

Given the input voltage $V_{in}$ and the voltage gain in decibel, e.g., $L_{dB}=60\;dB$, 
the output voltage can be obtained as:
\[ \frac{V_{out}}{V_{in}}=10^{L_{dB}/20}=10^{60/20}=10^3,
\;\;\;\;\mbox{i.e.,}\;\;\;\;\; V_{out}=10^3 V_{in}=1,000 V_{in} \]

A related issue is the half power point. Recall that for a second order system, when
$\zeta$ is small (e.g., $\zeta<0.2$), the magnitude $|H(j\omega)|$ of the frequency 
response function has a peak at $\omega=\omega_p\approx \omega_n$. The bandwidth of 
the peak is defined as the difference between two cut-off frequencies $\omega_1$ 
and $\omega_2$ ($\omega_1<\omega_n < \omega_2$) at which 
\[ |H(j\omega_1)|^2=|H(j\omega_2)|^2=\frac{1}{2} |H(j\omega_p)|^2\;\;\;\mbox{i.e.,}\;\;\;\;
   | H(j\omega_{1,2}) |=0.707\; | H(j\omega_p) | \]
The ratio between the half-power point and the peak in decibel is
\[ 20 \log_{10} \left( \frac{ |H(j\omega_{1,2})|}{| H(j\omega_p) |} \right)
=20 \log_{10} 0.707=-3.01\;dB \approx -3\;dB \]

\section*{Bode Plots}

The Bode plot is named after 
\htmladdnormallink{Hendrik Wade Bode (1905 -- 1982)}{http://en.wikipedia.org/wiki/Hendrik_Wade_Bode}, an American engineer and scientist, of Dutch ancestry,
a pioneer of modern control theory and electronic telecommunications. 

The {\bf frequency response function (FRF)} is a complex function of the 
frequency $\omega=2\pi f$ that describes the response of a system to input of
different frequencies:
\[ H(j\omega)=|H(j\omega)|e^{j\angle H(j\omega)}=|H(j\omega)| \angle H(j\omega) \]

The {\bf Bode plot} presents both the magnitude $|H(j\omega)|$ and phase 
angle $\angle H(j\omega)$ of $H(j\omega)$ as functions of frequency in 
logarithmic scale. (Zero frequency is at $-\infty$ as $10^{-\infty}=0$.)
Moreover, the magnitude $| H(j\omega) |$ is also represented in logarithmic
scale in decibel (dB), and is called log magnitude.
A Bode plot is composed of two parts:
\begin{itemize}
  \item The log magnitude (Lm) of $|H(j\omega)|$ with unit decibel (dB):
  \[ Lm\; H(j\omega)=20 \;log_{10} |H(j\omega)|\;dB \]
  \item The phase plot $\angle H(j\omega)$ with either in radian or degree.
\end{itemize}
The logarithmic scale of the frequency is composed of several ``decades'' each 
for a range of frequencies from $\omega$ to $10 \omega$, independent of the 
specific frequency $\omega$.

Bede plots have the following advantages:
\begin{itemize}
  \item Due to the logarithmic scale in frequency, large frequency range of 
    several orders of magnitude can be represented;

  \item Convenient straight line asymptotes can be used to approximate the plots;

  \item The behavior of the system in terms of the magnitude, even
    approaching zero, can be clearly described.
    
  \item Due to the logarithmic scale of the magnitude of the FRF, multiplications 
    and divisions of FRFs can be represented as addition and subtractions in the 
    plot (while the phases are always added/subtracted):
    \[ \left\{ \begin{array}{ll}
      Lm(H_1H_2)=Lm\;H_1+Lm\;H_2,&\;\;\;\;\angle (H_1H_2)=\angle H_1+\angle H_2\\
      Lm(H_1/H_2)=Lm\;H_1-Lm\;H_2,&\;\;\;\;\angle (H_1/H_2)=\angle H_1-\angle H_2\\
      Lm \;H^n=n\;Lm \;H,&\;\;\;\;\angle H^n=n\;\angle H \\
      Lm (1/H)=-Lm\;H,&\;\;\;\;\angle (1/H)=-\angle H \end{array} \right. \]
\end{itemize}

All FRFs of interest in this course can be considered as a combination of 
some components or building blocks, including:
\begin{itemize}
  \item Constant gain $k$;
  \item Integral/derivative factors $(j\omega)$
  \item Delay factor: $e^{\pm j\omega \tau}$;
  \item First-order factor $(1+j\omega\tau)$;
  \item Second-order factor 
    $(j\omega)^2+2\zeta\omega_n\omage j+\omega_n^2
    =(\omega_n^2-\omega^2)+j\,2\zeta\omega_n$
\end{itemize}
Given the Bode plot of any building block $H(j\omega)$, we can obtain the plots
of any combination of them.

We will first consider each of such components at a time, and then consider 
their combinations. In particular, we will study the first order system:
\[ H(j\omega)=\frac{N(j\omega)}{1+j\omega \tau} \]
and the second order system:
\[ H(j\omega)=\frac{N(j\omega)}{(j\omega)^2+2\zeta\omega_n j\omega +\omega_n^2}
=\frac{N(j\omega)}{(\omega_n^2-\omega^2)+j\;(2\zeta\omega_n \omega) }\]

\section*{Bode Plots of Components}

\begin{enumerate}
\item {\bf Constant gain $k$}

  \[ \left\{ \begin{array}{l}
    \mbox{If $k>0, \;\;\;\;k=|k|e^{j0}, \;\;Lm\;k=20 \;\log_{10}|k|,\;\;\angle k=0$} \\
    \mbox{If $k<0, \;\;\;\;k=-|k|=|k|e^{j\pi}, \;\;Lm\;k=20\;\log_{10}|k|,\;\;\angle k=\pi$} 
    \end{array} \right. \]

\item Delay factor: $e^{\pm j\omega \tau}$
  \[ Lm \;e^{ j\omega \tau}=20\;\log_{10} |e^{ j\omega \tau}|=20\;\log_{10} 1=0,\;\;\;\;
  \angle e^{ j\omega \tau} =\pm \omega \tau \]

\item {\bf Derivative factor $j\omega=\omega\; e^{j\pi/2}$:}
  \[ Lm\; (j\omega)=20\; log_{10} \omega\;dB,\;\;\;\;\;\angle(j\omega)=\frac{\pi}{2} \]
  In particular:
  \begin{itemize}
  \item When $\omega=1$, $Lm \;(1) =20\; \log_{10} 1=0\;dB$
  \item If a frequency $\omega$ becomes ten times higher, then
    \[ Lm\; (j10\omega)=20\; \log_{10} 10\omega=20 \;\log_{10} 10+20\;\log_{10}\omega 
    =20+Lm(j\omega) \]
    The Lm plot of $j\omega$ is a straight line with a slop of 20 dB/dec that goes
    through a zero-crossing at $\omega=1$.

  \end{itemize}

  Also consider two additional cases related to $j\omega$. First, 
  $(j\omega)^{\pm m}=\omega^{\pm m} e^{\pm j m\pi/2}$
  \[ Lm(j\omega)^{\pm m}=\pm m\;Lm(j\omega),\;\;\;\;\;\angle(j\omega)^{\pm m}=\pm m\pi/2\]
  The slop of the Lm plot is $\pm 20m/dec$. For example, when $m=2$, we have:
  \[ Lm\; (j\omega)^2=40\log_{10}\omega,\;\;\;\;\;\angle\;(j\omega)^2=\pi \]

  Second, the plots of $j\omega\tau$ are similar to those of $j\omega$, except the
  zero-crossing occurs at $\omega\tau=1$, i.e., $\omega=1/\tau$.

\item {\bf Integral factor $1/j\omega=(j\omega)^{-1}$:}

  \[ Lm \;(j\omega)^{-1}=-Lm\;(j\omega)=-20\;log_{10} \omega\;dB,
  \;\;\;\;\angle\; (j\omega)^{-1}=-\angle(j\omega)=-\frac{\pi}{2} \]
  The Lm plot of $1/j\omega$ is a straight line with a slop of -20 dB/dec that goes
  through a zero-crossing at $\omega=1$.

\item {\bf First order factor in numerator $1+j\omega\tau$}
  \[ 1+j\omega \tau=\sqrt{1+(\omega \tau)^2}\;e^{j\tan^{-1}(\omega \tau)}
  =\sqrt{1+(\omega \tau)^2}\;\angle \tan^{-1}(\omega \tau) \]
  \[ Lm(1+j\omega \tau)=20\;\log_{10}\sqrt{1+(\omega \tau)^2}
  =20\;\log_{10}(1+(\omega \tau)^2)^{1/2}=10\;\log_{10}(1+(\omega \tau)^2) \]
  \[ \angle(1+j\omega \tau)=\tan^{-1}(\omega\tau) \]
  Consider the following three cases:
  \begin{itemize}
  \item $\omega\tau=1$, i.e., $\omega_c=1/\tau$ is the corner frequency, we have
    \[ Lm(1+j)=20\;\log_{10} \sqrt{1^2+1^2}=20\;\log_{10} 0.707\approx 3.01\;dB,\;\;\;\;\;
    \angle(1+j)=\frac{\pi}{4} \]
  \item $\omega\tau \ll 1$ (e.g., $\omega\tau\le 10$):
    \[ Lm(1+j\omega \tau)\approx10\;\log_{10}(1)=0,\;\;\;\;\;
    \angle(1+j\omega \tau)\approx \angle(1)=0 \]
  \item $\omega\tau \gg 1$ (e.g., $\omega\tau\ge 10$):
    \[ Lm(1+j\omega \tau)\approx 20\;\log_{10}(\omega \tau),\;\;\;\;
    \angle(1+j\omega \tau)\approx \angle(j\omega \tau)=\frac{\pi}{2} \]
  \end{itemize}
  The straight-line asymptote of $Lm(1+j\omega\tau)$ has zero slope when $\omega\tau<1$
  but a slope 20 dB/dec when $\omage\tau>1$. The straight-line asymptote of 
  $\angle(1+j\omega\tau)$ is zero when $\omega\tay<0.1$, $\pi/2$ when $\omega\tau>10$, 
  but with a slope $45^\circ/dec$ in between.

\item {\bf First order factor in denominator $1/(1+j\omega\tau)=(1+j\omega\tau)^{-1}$}
  \[ Lm\;(1+j\omega\tau)^{-1}=-Lm(1+j\omega\tau)
  =-10\;\log_{10}(1+(\omega \tau)^2) \]
  \[ \angle\;(1+j\omega \tau)^{-1}=-\angle(1+j\omega \tau)
  =-\tan^{-1}(\omega\tau) \]
  Both the Lm and phase plots of $1/(1+j\omega\tau)$ is simply the negative 
  version of $(1+j\omega\tau)$. 

  The figure below shows the plots of two first order systems corner frequencies 
  $\omega_1=100$ and $\omega_2=1000$, together with the plots of their product, a 
  second order system.

\htmladdimg{../figures/bodeplot1storder.gif}

\item {\bf Second-order factor}

  \[ H(j\omega)=\frac{1}{(j\omega)^2+2\zeta\omega_n j\omega+\omega_n^2}
  =\frac{1}{(\omega^2_n-\omega^2)+2\zeta\omega_n j\omega}
  =\frac{\frac{1}{\omega_n^2}}{1-(\frac{\omega}{\omega_n})^2+j\,2\zeta\frac{\omega}{\omega_n}} \]
  The denominator is a 2nd order polynomial for variable $j\omega$. Consider the
  following two cases:

  First, if $\Delta=b^2-4ac=(2\zeta\omega_n)^2-4\omega_n^2=4\omega^2_n(\zeta^2-1)\ge 0$
  i.e., if $\zeta\ge 1$, the denominator has two real and negative roots:
  \[ p_{1,2}=(-\zeta\pm\sqrt{\zeta^2-1})\omega_n < 0 \]
  and $H(j\omega)$ can be written as a product of two first order FRFs:
  \[ H(j\omega)=\frac{1}{(j\omega-p_1)(j\omega-p_2)} 
  =\frac{1/p_1p_2}{(j\omega/p_1-1)(j\omega/p_2-1)} 
  =\frac{\tau_1}{1+j\omega\tau_1}\;\frac{\tau_2}{1+j\omega\tau_2}
  =H_1(j\omega)H_2(j\omega) \]
  where $\tau_1=-1/p_1>0$ and $\tau_2=-1/p_2>0$ are the two time constant of the two
  first order systems. Now the second order factor is the product of two first order 
  factors and
  \[ Lm\;(H_1 H_2)=Lm\; H_1+Lm\; H_2,\;\;\;\;\angle (H_1 H_2)=\angle H_1+\angle H_2 \]
  with corner frequencies at $\omega_{c1}=1/\tau_1=p_1$ and $\omega_{c1}=1/\tau_2=p_2$.

  Second, if $0<\zeta<1$, i.e., the two roots are complex. We consider the numerator 
  and the denominator separately. The numerator is just a constant with zero phase and
  log-magnitude of $20\log_{10} \omega_n^{-}2=-40\log_{10} \omega_n$. Next consider the
  rest of the function:
  \[ |H(j\omega)|=[(1-(\frac{\omega}{\omega_n})^2)^2+(2\zeta\frac{\omega}{\omega_n})^2]^{-1/2}\]
  We have
  \begin{eqnarray}
    && Lm\;H(j\omega)=20\log_{10} |H(j\omega)|
    =-10\;\log_{10}[\; (1-(\frac{\omega}{\omega_n})^2)^2+(2\zeta\frac{\omega}{\omega_n})^2\;]
    \nonumber 
  \end{eqnarray}
  \[ \angle H(j\omega)=-\tan^{-1}\frac{2\zeta\omega/\omega_n}{1-(\omega/\omega_n)^2} \]
  Consider three cases:
  \begin{itemize}
  \item $\omega/\omega_n=1$:
    Now $H(j\omega)=1/j2\zeta=-j/2\zeta$ and 
    \[ Lm\;H(j\omega)=-20\;\log_{10} 2\zeta,\;\;\;\;\;\angle H(j\omega)=-\frac{\pi}{2} \]
  \item $\omega/\omega_n\ll 1$, i.e., $\omega \ll \omega_n$:
    \[ Lm\;H(j\omega) \approx -10\;\log_{10} (1)=0,\;\;\;\;\;\angle H(j\omega)=0^\circ \]
  \item $\omega/\omega_n\gg 1$, i.e., $\omega \ll \omega_n$:
    \[ Lm\;H(j\omega)\approx-10\;\log_{10}[\; (\frac{\omega}{\omega_n})^4 ]
    =-40 \;\log_{10} \frac{\omega}{\omega_n}    \]
    This is a straight line with slop of -40 dB per decade.
    \[ \angle H(j\omega) \approx -\tan^{-1} (-2\zeta \omega_n/\omega)
    \approx -\tan^{-1} (-0)=-\pi=-180^\circ \]
  \end{itemize}

\end{enumerate}

\htmladdimg{../figures/bodeplotzeta.gif}

  The magnitude of the second-order factor is
  \[ |H(j\omega)|
  =\frac{1}{\sqrt{(1-\frac{\omega^2}{\omega_n^2})^2+4\zeta^2 \frac{\omega^2}{\omega_n^2}}}
  =\frac{1}{\sqrt{(1-u)^2+4\zeta^2 u}} \]
  where $u=(\omega/\omega_n)^2$. When $u=1$ i.e., $\omega=\omega_n$, we have
  \[ | H(j\omega_n) |=\frac{1}{2\zeta}=Q \]
  However, the peak of $|H(j\omega)|$ is not at $\omega_n$, but at the resonant frequency 
  $\omega_p$, which can be found by taking derivative of the magnitude of the denominator 
  with respect to $u$ and setting it to zero:
  \[ \frac{d}{du}[u^2+(4\zeta^2-2)u+1]=2u+4\zeta^2-2=0 \]
  Solving it, we get:
  \[ u=\frac{\omega^2}{\omega_n^2}=1-2\zeta^2,\;\;\;\mbox{i.e.,}
  \;\;\;\;\omega=\omega_n\sqrt{1-2\zeta^2} < \omega_n \]
  At this peak frequency $\omega_p=\omega_n\sqrt{1-2\zeta^2}$, the peak is:
  \[ | H(j\omega_p) |=\frac{1}{2\zeta\sqrt{1-\zeta^2}} > \frac{1}{2\zeta}=| H(j\omega_n) | \]
  Note that if $\zeta^2>1/2$, i.e., $\zeta>0.707$, the result is complex indicating there 
  is no peak.


\section*{Bode Plots of first and Second Order Systems}
{\bf First order circuits}

\htmladdimg{../figures/RC.gif}

\begin{itemize}
\item Voltage $V_C$ across C is treated as output. According to voltage
  divider rule, we have:
\[  H_C(j\omega)=\frac{V_C}{V_{in}}=\frac{Z_C}{Z_R+Z_C}
    =\frac{1/j\omega C}{R+1/j\omega C} 
    =\frac{1}{j\omega RC+1}=\frac{1}{j\omega \tau+1} \]
    where $\tau=RC$.
\item Voltage $V_R$ across R is treated as output:
  \[ H_R(j\omega)=\frac{V_R}{V_{in}}=\frac{Z_R}{Z_R+Z_C}
    =\frac{R}{R+1/j\omega C}=\frac{j\omega RC}{j\omega RC+1} 
    =\frac{j\omega \tau}{j\omega \tau+1} \]
\end{itemize}
As $H_R(j\omega)$ can be written as:
\[ H_R(j\omega)= \frac{1}{j\omega \tau+1} j\omega \tau  \]
The first term is just $H_C(j\omega)$. Now the log-magnitude is:
\[ Lm\;H_R(j\omega)
=20\log_{10} \left| \frac{1}{j\omega \tau+1}\right|+20\log_{10} \left| j\omega \tau \right| 
=Lm\; H_C(j\omega) +20\log_{10} (\omega\tau) \]
The first term is the same as $H_C(j\omega)$ and the second plot is a straight line
with slope of 20 dB/dec. at $\omega=\omega_c=1/\tau$, the first term is -3 dB and the  
second is 0 dB.  The phase plot is:
\[ \angle H_R(j\omega)=\angle \left(\frac{1}{j\omega \tau+1}\right)+\angle j\omega \tau  
=\angle H_C(j\omega)+\frac{\pi}{2} \]
In the plots below, $\tau=0.01$, $\omega_C=100$ rad/sec.

\htmladdimg{../figures/bodeplot1storder1.gif}

Define $\omega_c=1/\tau=1/RC$ as the cut-off frequency, then when $\omega=\omega_c$, 
we have $\omega\tau=1$, and $|H_R(j\omega)|=|H_C(j\omega)|=1/\sqrt{2}$, i.e., $\omega_c$
is the half-power point, where $|H(j\omega)|$ is -3 dB.

{\bf Second order circuits}

\htmladdimg{../figures/RCL.gif}

\begin{itemize}
\item Voltage $V_C$ across C is treated as output:
  \begin{eqnarray}
    H_C(j\omega)&=&\frac{V_C}{V_{in}}=\frac{Z_C}{Z_L+Z_R+Z_C}
    =\frac{1/j\omega C}{j\omega L+R+1/j\omega C}
    =\frac{1}{(j\omega)^2 LC+j\omega RC+1}
    \nonumber \\
    &=&\frac{1/LC}{(j\omega)^2 +j\omega R/L+1/LC}
    =\frac{\omega_n^2}{(j\omega)^2 +2\zeta\omega_n j\omega+\omega^2_n} 
    =\frac{1}{(1-\frac{\omega^2}{\omega_n^2})+j2\zeta\frac{\omega}{\omega_n}}
    \nonumber 
  \end{eqnarray}
  where 
  \[ \omega_n=\frac{1}{\sqrt{LC}},\;\;\;\;\zeta=\frac{R}{2}\sqrt{\frac{C}{L}} \]
  The magnitude is
  \[ |H_C(j\omega)|
  =\frac{1}{\sqrt{(1-\frac{\omega^2}{\omega_n^2})^2+4\zeta^2 \frac{\omega^2}{\omega_n^2}}}
  =\frac{1}{\sqrt{(1-u)^2+4\zeta^2 u}} \]
  where $u=(\omega/\omega_n)^2$. When $u=\omega/\omega_n=1$ or $\omega=\omega_n$, we have
  \[ | H(j\omega_n) |=\frac{1}{2\zeta}=Q \]
\item Voltage $V_R$ across R is treated as output:
  \begin{eqnarray}
    H_R(j\omega)&=&\frac{V_R}{V_{in}}=\frac{Z_R}{Z_L+Z_R+Z_C}
    =\frac{R}{j\omega L+R+1/j\omega C}
    =\frac{j\omega RC}{(j\omega)^2 LC+j\omega RC+1}
    \nonumber \\
    &=&\frac{j\omega R/L}{(j\omega)^2 +j\omega R/L+1/LC}
    =\frac{2\zeta\omega_nj\omega}{(j\omega)^2 +2\zeta\omega_n j\omega+\omega^2_n} 
    =H_C(j\omega) \;2\zeta \omega_n\;j \omega 
    \nonumber 
  \end{eqnarray}
  Now we have:
  \[ Lm\;H_R(j\omega)=Lm\; H_C(j\omega)+Lm\;(2\zeta\omega_n\;j\omega),\;\;\;\;
  \angle H_R(j\omega)=\angle H_C(j\omega)+\angle(2\zeta\omega_n\;j\omega) \]
  The log-magnitude of the second factor is a straight line with slope 20 dB/dec,
  and at $\omega=\omega_n$, its value is $20\log_{10} 2\zeta\omega_n^2$. The phase is 
  $90^\circ$ for all $\omega$.
  
  The denominator can be written as $R+j(\omega L-1/\omega C)$, which is minimized
  when the imaginary part is zero, i.e, $j\omega L=1/j\omega C$. In other words, when 
  $\omega=\omega_n=1/\sqrt{LC}$, $|H_R(j\omega)|$ reaches its peak value.
\item Voltage $V_L$ across L is treated as output:
  \begin{eqnarray}
    H_L(j\omega)&=&\frac{V_L}{V_{in}}=\frac{Z_L}{Z_L+Z_R+Z_C}
    =\frac{j\omega L}{j\omega L+R+1/j\omega C}
    =\frac{(j\omega)^2 LC}{(j\omega)^2 LC+j\omega RC+1}
    \nonumber \\
    &=&\frac{(j\omega)^2}{(j\omega)^2 +j\omega R/L+1/LC}
    =\frac{(j\omega)^2}{(j\omega)^2 +2\zeta\omega_n j\omega+\omega^2_n} 
    =H_C(j\omega) \;(j\omega_n)^2
    \nonumber 
  \end{eqnarray}
  Now we have:
  \[ Lm\;H_L(j\omega)=Lm\; H_C(j\omega)+Lm\;(j\omega)^2,\;\;\;\;
  \angle H_L(j\omega)=\angle H_C(j\omega)+\angle (j\omega)^2 \]
  The log-magnitude of the second factor is a straight line with slope 40 dB/dec,
  and at $\omega=\omega_n$, it's value is $20\log_{10} \omega_n^2$. The phase is
  $180^\circ$ for all $\omega$.
\end{itemize}

\htmladdimg{../figures/bodeplot2ndorderline.gif}

In the following plots, $\omega_n=100$ rad/sec and $\zeta=0.05$.
At $\omega=\omega_n$, $20\log_{10} (2\zeta\omega_n^2)=20\log_{10} 1000=60$ dB, and
$20\log_{10}(\omega_n^2)=20\log_{10} 10,000=80$ dB.

\htmladdimg{../figures/bodeplot2ndorder.gif}

{\bf Example, a Band-pass filter:}

\htmladdimg{../figures/opamp4b.gif}

\[ H(j\omega)=-\frac{Z_2(j\omega)}{Z_1(j\omega)}
=-\frac{R_2||1/j\omega C_2}{R_1+1/j\omega C_1}
=-\frac{R_2/(1+j\omega R_2C_2)}{(1+j\omega R_1C_1)/j\omega C_1}
=-\frac{j\omega \tau_3}{(1+j\omega \tau_1)(1+j\omega \tau_2)} \]
where $\tau_1=R_1C_1$, $\tau_2=R_2C_2$, $\tau_3=R_2C_1$.


\end{document}

