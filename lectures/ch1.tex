%\documentstyle[12pt]{article}
\documentclass{article}
\usepackage{amsmath}
\usepackage{amssymb}
\usepackage{graphics}
\usepackage{comment}
\usepackage{html,makeidx}

\begin{document}

\section*{Chapter 1: Basic Quantities and Laws}

\subsection*{Basic Quantities}

\begin{itemize}

\item {\bf Charge:} The charge of $6.24 \times 10^{18}$ electrons is a 
  {\em Coulomb}, i.e., the charge each electron carries is 
  $1/ (6.24 \times 10^{18})=1.602 \times 10^{-19}$ Coulomb. Charge is 
  conservative (can be neither created nor destroyed).

\item {\bf Force $f$:} Force can be defined by Newton's second law: 
  \begin{equation} 
    f=ma,\;\;\;\;\;\;\;[Newton]=\frac{[kilogram][meter]}{[second]^2}	
  \end{equation}
  The force of 1 Newton will cause a mass of 1 kilogram to accelerate
  at 1 meter per second per second. 

  \begin{itemize}
  \item {\bf Gravitational force:}
    \begin{equation}	
      f=G\frac{m_1m_2}{r^2}	
    \end{equation}
    where $G=6.67384\times 10^{-11}$ $[Newton\,(meter/kilogram)^2]$ is
    the gravitational constant.
    In particular, on surface of Earth, the ``weight'' of a mass $m$ is
    \begin{equation}	
      f=G\frac{Mm}{r^2}=gm	
    \end{equation}
    where $g=GM/r^2=9.8\;\mbox{meter}/\mbox{second}^2$ is the gravitational 
    acceleration that measures the intensity of Earth's gravitational field,
    and $f$ is the force this field asserts on mass $m$.
  \item{\bf Electric force:}
    \begin{equation}	
      f=k\frac{q_1q_2}{r^2}=Eq_2	
    \end{equation}
    where $k=8.98755\times 10^9$ $[Newton\,(meter/Coulumb)^2]$ is the
    Coulomb's constant, $E=kq_1/r^2$ is the intensity of the electric 
    field caused by charge $q_1$, and $f$ is the force this field asserts
    on charge $q_2$.
  \end{itemize}

\item {\bf Potential Energy:} 

  \begin{itemize}
  \item {\bf Gravitational Potential Energy:} In the gravitational field 
    of $g$, if a mass $m$ (with weight $f=mg$) is raised up from height 
    $h_1$ to $h_2$, it receives a potential energy
    \begin{equation}
      w=fl=mg(h_2-h_1),\;\;\;\;\;\;g=\frac{w}{m(h_2-h_1)},\;\;\;\;
      \frac{[Joule]}{[kilogram][meter]}=\frac{[Newton]}{[kilogram]}
      =\frac{[meter]}{[s]^2}
    \end{equation}

  \item {\bf Electric Potential Energy:} In an electric field of $E$, if 
    a charge $q$ is moved along the direction of the field from point $l_1$
    to point $l_2$, it receives a potential energy 
    \begin{equation}
      w=fl=qE(l_2-l_1),\;\;\;\;\;\;E=\frac{w}{q(l_2-l_1)},\;\;\;\;
      \frac{[Joule]}{[Coulomb][meter]}=\frac{[Newton]}{[Coulomb]}
    \end{equation}
  \end{itemize}

\item {\bf Current:} The rate of flow of positive charge 
  \begin{equation} 
    i=\frac{dq}{dt},\;\;\;\;\; q=\int i \; dt,	\;\;\;\;\;
    [\mbox{Ampere}]=\frac{[Coulomb]}{[second]}	
  \end{equation}
  which is one of the
  \htmladdnormallink{seven base units}{http://en.wikipedia.org/wiki/SI_base_unit}
  in the \htmladdnormallink{International System of Units (IS)}{http://en.wikipedia.org/wiki/International_System_of_Units}.

  The current is the same through out an electricity conducting component,
  it is a {\em through variable}.

  Current density $j=\lim_{A\rightarrow 0} i(A)/A$, 
  \begin{equation}
    q=\int i(t) \,dt =\int {\bf j} \cdot d{\bf A}\; dt
  \end{equation}

\item {\bf Voltage:} The potential energy per unit charge is measured
  by volt:
  \begin{equation} 
    v=v-v_0=\frac{w}{q}=\frac{Eq(l-l_0)}{q}=E(l-l_0),
    \;\;\;\;[Volt]=\frac{[Joule]}{[Coulomb]} 
  \end{equation}
  where the voltege at $l_0$ (a reference point or ``ground'') is $v_0=0$.
  The voltage is the difference between two potential energy levels $v_1$
  and $v_2$ per unit charge in an electric field. A unit charge of $q=1$
  Coulomb moved from one electric potential $v_1=El_1$ to another $v_2=El_2$
  gains a potential energy
  \begin{equation}
    v=v_2-v_1=E(l_2-l_1)
  \end{equation}
  The electric field can therefore also defined as:
  \begin{equation}
    E=\frac{v}{l}=\frac{qv}{ql}=\frac{w}{ql}=\frac{fl}{ql}=\frac{f}{q}
  \end{equation}
  If a unit charge of $q=1$ Coulomb is moved through an electric field 
  $E(l_2-l_2)$ of 1 volt, it receives or delivers 1 Joule of energy.

  Voltage $v=w/q$ is energy per charge, while electric field $E=f/q$ is 
  force per charge.

  The voltage is measured as the difference across two points in an 
  electric field or circuit (or a point with respect to a reference 
  point called {\em ground}), i.e., it is an {\em across variable}.

\item {\bf Power and Energy:}

  Power $p$ is the rate of energy transformation. The transformation 
  of 1 Joule of energy in 1 second represents a power of 1 Watt:
  \begin{equation}	
    p=\frac{dw}{dt}=\frac{dw}{dq} \; \frac{dq}{dt}=v\,i,
    \;\;\;\;\;\;\;[Watt]=\frac{[Joule]}{[second]}	
    =\frac{[Joule]}{[Coulomb]}\frac{[Coulomb]}{[second]}
    =[Volt][Ampere]
  \end{equation}
  We also have:
  \begin{equation}	
    w=\int p\;dt=\int vi\;dt,
    \;\;\;\;[Joule]=[Watt][second]=[Volt][Ampere][second]
  \end{equation}
  Another unit for power is {\em horse power}: 
  $1 \; \mbox{hp}=746 \; \mbox{Watts} $
  
  Energy can also be measured by {\em kilowatt-hours (kWh)} 
  $ 1000 \times 3600=3.6 \times 10^6 $ Joules.

  \begin{itemize}
  \item {\bf Electrical potential energy:}
    
    The energy needed to move a charge $q$ from point A with  potential 
    $v_1$ to point B with potential $v_2$ is:
    \begin{equation}
      w=qv=q(v_2-v_1)=qE(l_2-l_1)
    \end{equation}
    where the voltage $v=v_1-v_2=E(l_2-l_1)$ is the potential difference
    between the two points.
  
    An electric field $E=(v_2-v_1)/l$ is the electric potential difference
    per unit distance.

  \item {\bf Gravitational potential energy:}

    The energy needed to move a mass $m$ from height $h_1$ with 
    potential $gh_1$ to height $h_2$ with potential $gh_2$ is 
    \begin{equation} 
      w=fl=mg(h_2-h_1)
    \end{equation}
    where $gl=g(h_2-h_1)=gh_2-gh_1$ is the potential difference between the 
    two heights.
    The Gravitational field $g=(gh_2-gh_1)/l$ is the gravitational potential
    difference per unit distance.

  \end{itemize}

  \htmladdimg{../figures/basicquantities.gif}

  \begin{itemize}
  \item Electric force per unit charge is electric field $E=f/q$;
    Energy per unit charge is voltage $v=w/q=Eql/q=El$ is voltage.
  \item Gravitational force per unit mass is gravitational field $g=f/m$;
    Energy per unit mass is $w/m=mgh/m=gh$. Although this quantity is not
    explicitly defined, it is analogous equivalent to voltage in electric
    field.
  \end{itemize}

  \begin{tabular}{c||c|c}\hline
    Force & $f=GMm/r^2=gm$ $(g=GM/r^2)$ & $f=kQq/r^2=Eq$ $(E=kQ/r^2)$ \\
    Field intensity & $g=(gh_1-gh_2)/L=f/m$ & $E=(v_2-v_1)/L=f/q$ \\
    Potential difference & $gh_2-gh_1$ & $v_2-v_1$ \\
    Potential energy & $w=mg(h_2-h_1)=mgL$ & $w=q(v_2-v_1)$ \\
  \end{tabular}

\item {\bf Magnetic Flux:} 

  The intensity of magnetic effect (lines per unit area in a magnetic
  field or flux) is measured by magnetic flux density ${\bf B}$ in Tesla. 
  The Earth's magnetic field is about 25 to 65 micro Tesla, the MRI machine 
  is either 1.5 or 3 Tesla.

  The {\em magnetic flux} $\Phi$ through an area ${\bf A}$ (in the normal
  direction of the area) is
  \begin{equation}
    \Phi=\int {\bf B} \cdot d{\bf A}=|B|\;|A| \cos\alpha,\;\;\;\;
              [Weber]=[Tesla] [meter]^2	
  \end{equation}   
  where $\alpha$ is the angle between the two directions. 
	
  When ${\bf B}$ and ${\bf A}$ are in the same direction ($\alpha=0$), 
  and if ${\bf B}$ is 1 Tesla and ${\bf A}$ is 1 square meter, then the
  flux $\Phi$ is 1 {\em Weber}.

\item {\bf Electric Magnetic Interaction:}

  In a magnetic field ${\bf B}$, a force ${\bf f}$ is exerted on a charge $q$ 
  moving with velocity ${\bf u}$:
  \begin{equation}
    {\bf f}=q{\bf u} \times {\bf B},\;\;\;\;\;
    [Newton]=[Coulomb]\frac{[meter]}{[second]} [Tesla]	
  \end{equation}
  where the force vector ${\bf f}$ is the cross product of velocity vector 
  ${\bf u}$ and magnetic flux vector ${\bf B}$ (right-hand rule).

  A force of 1 Newton is experienced by a charge of 1 Coulomb moving with a 
  velocity of 1 meter per second normal to a magnetic flux density of 1 Tesla.

  \htmladdimg{../figures/magneticforce.gif}

  The \htmladdnormallink{Lorentz force}{http://en.wikipedia.org/wiki/Lorentz_force_law}
  on a charge $q$ in electromagnetic field is 
  \begin{equation}
    {\bf f}=q({\bf E}+{\bf u}\times {\bf B}) 
  \end{equation}

\end{itemize}


\subsection*{Resistor, Capacitor, and Inductor}

In the following, we adopt the convention that a constant or 
{\em direct current (DC)} or voltage is represented by an upper-case letter 
$I$ or $V$, while a time-varying or {\em alternating current (AC)} current 
or voltage is represented by a lower-case letter $i(t)$ or $v(t)$, sometimes 
simply $i$ and $v$. 

Each of the three basic components resistor R, capacitor C, and inductor
L can be described in terms of the relationship between the voltage 
across and the current through the component:

\begin{itemize}
\item {\bf Resistor}

  The voltage across and the current through a resistor are related 
  by Ohm's law:
  \begin{equation}
    R=\frac{V}{I}=\frac{v}{i},\;\;\;\;\;\;\;[Ohm]=\frac{[Volt]}{[Ampere]},
    \;\;\;\;\;\;\;
    \Omega=\frac{V}{A}
  \end{equation}
  Here $R$ is the {\em resistance} of the conductor measured by Ohm 
  $\Omega$
  (\htmladdnormallink{George Ohm (1789-1854)}{http://www-gap.dcs.st-and.ac.uk/~history/Mathematicians/Ohm.html}). 

  The reciprocal of the resistance is the {\em conductance}:
  \begin{equation} 
    G=\frac{1}{R}=\frac{I}{V}=\frac{i}{v},\;\;\;\;\;\;\;
    [Siemens]=\frac{1}{[Ohm]}=\frac{[Ampere]}{[Volt]},
    \;\;\;\;\;\;S=\frac{1}{\Omega}=\frac{A}{V}
  \end{equation}
  Conductance is measured by $[Siemens]=1/[Ohm]$ or $S=1/\Omega$
  (\htmladdnormallink{Werner von Siemens (1816-1892)}{https://en.wikipedia.org/wiki/Werner_von_Siemens})


%{http://www.britannica.com/eb/article?eu=69422&tocid=0&query=werner%20bischof}

\item {\bf Capacitor}

  A capacitor is composed of a pair of conductor plates separated by some 
  insulation material. The same amount of charge $Q$ (of opposite polarity) 
  is stored on each of the two plates. 

  The voltage $V$ between the two plates is proportional to the charge $Q$, 
  but inversely proportional to the {\em capacitance} $C$ of the capacitor:
  \begin{equation}
    V=\frac{Q}{C},\;\;\;\;\;\;\;Q=VC,\;\;\;\;\;\;\; C=\frac{Q}{V}
  \end{equation}

  \htmladdimg{../figures/capacitor.gif}
   
  \htmladdimg{../figures/capacitoranalogy1.gif}
  
  This relationship can be understood by considering the water tank
  analogy of the capacitor. The capacity $C_1$ (analogous to capacitance
  of a capacitor) of the tank on the left is smaller than that of $C_2$
  on the right, for the same amount of water $Q$ (analogous to charge), 
  the water surface $h_1$ is higher than that of $h_2$, indicating the 
  surface height $h_1$ (analogous to voltage $V$) is proportional to 
  water volume $Q$ but inversely proportional to the tank capacity $C$,
  i.e., $V=Q/C$.
   
  \htmladdimg{../figures/capacitoranalogy2.gif}
   
  Why can an AC current ``flow through'' a capacitor composed of two 
  insulated plates? Again consider the water tank analogy of the 
  capacitor. If the pipeline is disconnected (an open circuit), no 
  water flow (current) can go through. If two tanks are connected to 
  the ends of the pipeline (a capacitor), and the pump drives the 
  water in one direction (analogous to a DC voltage source), one of 
  the tanks will fill up while the other one is empty (due to some
  initial current), there is still no continuous current. However, 
  if the pump drives the water in alternative directions (analogous 
  to AC voltage source), the water can flow through the pipeline, 
  analogous to an AC current going through a capacitor (not through 
  the insulation between its two plates).

  The current through a capacitor can be found as:
  \begin{equation} 
    i(t)=\frac{dq(t)}{dt}=\frac{dq}{dv}\,\frac{dv}{dt}=C\frac{dv(t)}{dt},
    \;\;\;\;\;\;\;
    v(t)=\frac{q(t)}{C}=\frac{1}{C}\int_{-\infty}^t i(\tau) d\tau,
  \end{equation}
  where $q=\int_{-\infty}^t i(\tau)\,d\tau$, and the capacitance
  $C=dq/dv$ represents the capacitor's capability to store charge 
  per unit voltage.
  Capacitance $C$ is determined by the parameters of the capacitor:
  \begin{equation}
    C=\frac{\epsilon A}{d}=\frac{\epsilon_0\epsilon_r A}{d}
  \end{equation}
  where $A$ is the overlapping area of the plates and $d$ is the distance 
  between them, while $\epsilon$ is the 
  \htmladdnormallink{\em permittivity (dielectric constant)}
  {https://en.wikipedia.org/wiki/Permittivity} (the amount of charge needed 
  to generate one unit of electric flux) of the medium between the plates,
  $\epsilon_0=8.854\times 10^{-12} F/m$ is the 
  \htmladdnormallink{vacuum permittivity}{https://en.wikipedia.org/wiki/Vacuum_permittivity}, and $\epsilon_r=\epsilon/\epsilon_0$ is the relative permitivity.
  In an \htmladdnormallink{electrolytic capacitor}
  {https://en.wikipedia.org/wiki/Electrolytic_capacitor}, the gap between
  the two plates is filled with dielectric medium of higher permittivity
  so that the capacitance $C$ is increased.  

  $C$ is measured in {\em Farads (F)} 
  (\htmladdnormallink{Michael Faraday (1791-1867)}{http://www-gap.dcs.st-and.ac.uk/~history/Mathematicians/Faraday.html}):
  \begin{equation} 
    [Farad]=\frac{[Ampere][second]}{[Volt]} =\frac{[Coulomb]}{[Volt]},
    \;\;\;\;\;\;\;
    F=\frac{A\;s}{V}=\frac{C}{V}
  \end{equation}
  Other units also used for capacitance include $\mu F=10^{-6} F$, 
  $nF=10^{-9}F$, and $pF=10^{-12}F$.

  Specially, when the voltage is sinusoidal $v(t)=sin(\omega t)$, the 
  current is
  \begin{equation}
    i(t)=C \frac{d\,v(t)}{dt}=C \frac{d\sin(\omega t)}{dt}
    =\omega C\;\cos(\omega t)
  \end{equation}
  
  \htmladdimg{../figures/capacitorIV.gif}

  \begin{enumerate}
  \item The current (red) has a 90 degree phase lead compared to 
    the voltage (green), as it takes time for the voltage across the 
    capacitor to build up;
  \item The amplitude of the current is proportional to the frequency 
    $\omega=2\pi f$ of the voltage. In particular, for DC ($\omega=0$). 
    The current $i(t)=\omega C\cos(\omega t)$ is 0 (open circuit), and
    when the frequency is very high ($\omega \rightarrow \infty$), the 
    current $i(t)=\omega C\cos(\omega t) \rightarrow \infty$ (short circuit).
  \end{enumerate}


\item {\bf Inductor}

  \begin{itemize}
  \item {\bf Electromagnetic Interaction: Electricity to Magnetism}

    Magnetic field (flux) is generated in the space around a current 
    flowing through a piece of conductor:

    \htmladdimg{../figures/coil0.gif}

    The magnetic field around a coil is the superposition of the magnetic 
    flux generated by each section of the coil:

    \htmladdimg{../figures/coil1.gif}

  \item {\bf Electromagnetic Interaction: Magnetism to Electricity}

    Electric current is induced in a conductor when there is changing
    magnetic flux in the surrounding space.

    \htmladdimg{../figures/electromagnetic.gif}
    
    \htmladdimg{../figures/electromagnetic1.gif}

  \item {\bf Self and Mutual Induction}

    A time-varying electric current in a coil will cause a time-varying 
    magnetic field in the surrounding space, which in turn will induce 
    electric voltage and then current in the same coil (self-induction) 
    or a different coil in the neighborhood (mutual-induction).
    
  \item {\bf Faraday's Law:} 
    
    The self-induced voltage, the {\em electromotive force (emf)}, across 
    the inductor coil due to a current $i(t)$ is proportional to the rate 
    of change of the total magnetic flux $\Psi=N\Phi$ ($\Phi$ being the 
    flux in one of the $N$ turns of the coil) caused by the current $i(t)$: 
    \begin{equation} 
      v(t)=\frac{d\Psi(t)}{dt}=\frac{d\Psi}{di}\frac{di}{dt}=L\frac{di}{dt},
      \;\;\;\;\;\;\;\;
      i(t)=\frac{1}{L}\int_{-\infty}^t v(\tau) d\tau=\frac{\Psi}{L}
    \end{equation}
    where $\Psi=\int_{-\infty}^t v(\tau)\,d\tau$, and $L=d\Psi/di$ is 
    the {\em inductance} of the inductor representing its capability to 
    produce magnetic flux per unit current.
    Inductance $L$ is determined by the parameters of the inductor: 
    \begin{equation}
      L=\frac{\mu N^2 A}{l}
    \end{equation}
    where $A$ and $l$ are respectively the cross section area and 
    length of the coil, $N$ is the number of turns, and $\mu$ is the
    \htmladdnormallink{\em magnetic permeability}
    {https://en.wikipedia.org/wiki/Permeability_(electromagnetism)}
    (the ability of a material to support the 
    formation of a magnetic field within itself) of the medium inside 
    the coil. When a medium of high permeability, such as an iron core, 
    is inserted into the coil, its inductance $L$ is increased.

    The unit of $L$ is {\em henrys (H)}
    (\htmladdnormallink{Joseph Henry (1797-1878)}
    {https://en.wikipedia.org/wiki/Joseph_Henry}):
    \begin{equation} 
      [Henry]=\frac{[Volt][second]}{[Ampere]},
      \;\;\;\;\;\;\; H=\frac{Vs}{A}
    \end{equation}
    Other units also used for $L$ include $mH=10^{-3}H$ and $\mu H=10^{-6}H$. 

  \item {\bf Lenz's Law:} 

    The polarity of the self-induced voltage $v(t)$ in a coil is such 
    that it tends to produce a current which induces a magnetic flux to 
    oppose the change of the magnetic field that induced the voltage, 
    thereby opposing any change in current $i(t)$ that is causing the 
    magnetic flux. 

    When current $i(t)$ increases, the induced voltage $v(t)$ tends to
    resist it, when current $i(t)$ decreases, the induced voltage
    $v(t)$ tends to sustain it.

  \end{itemize}

  \begin{comment}
    The ratio between the primary voltage $V_1$ and the secondary voltage 
    $V_2$ of a transformer is proportional to the ratio between the numbers 
    of turns:
    \begin{equation}
      \frac{V_2}{V_1}=\frac{n_2}{n_1}	
    \end{equation}
    If there is no power loss by the transformer, then the transformer is 
    ideal and we have
    \begin{equation}
      P_1=V_1I_1=P_2=V_2I_2,\;\;\;\;\mbox{i.e.}\;\;\;\;
    \frac{I_2}{I_1}=\frac{V_1}{V_2}=\frac{n_1}{n_2}	
    \end{equation}
    The ratio between the primary current $I_1$ and the secondary current $I_2$ 
    of a transformer is inversely proportional to the ratio between the numbers
    of turns. Also note the reference directions of the currents $I_1$ and $I_2$
    and the reference polarities of the voltages $V_1$ and $V_2$, reflecting the
    fact that the secondary current $I_2$ is caused by the induced voltage $V_2$
    (consistent polarity), while $V_1$ is the induced voltage opposing the current
    $I_1$.

    We can also find the ratio between the primary and secondary impedances based 
    on the assumption that there is no power loss in the transformer, i.e.,
    \begin{equation}	
      P_1=\frac{V_1^2}{R_1}=P_2=\frac{V_2^2}{R_2}, \;\;\;\;\;\;\;\;\;\;
    \frac{R_1}{R_2}=\left(\frac{V_1}{V_2}\right)^2=\left(\frac{n_1}{n_2}\right)^2  
    \end{equation}
  \end{comment}

  Specially, when the current is sinusoidal $i(t)=sin(\omega t)$, the 
  voltage is
  \begin{equation}
    v(t)=L \frac{d\,i(t)}{dt} =L \frac{d\,\sin(\omega t)}{dt} 
    = \omega L\;\cos(\omega t)
  \end{equation}

  \htmladdimg{../figures/inductorIV.gif}

  \begin{enumerate}
  \item The current (red) has a 90 degree phase lag compared to the 
    voltage (green) as it takes time to overcome the counter potential 
    of the inductor for the current to flow;
  \item The amplitude of the voltage is proportional to the frequency
    $\omega=2\pi f$ of the current. In particular, for DC ($\omega=0$), 
    the voltage is 0 (short circuit), and when the frequency is very 
    high ($\omega \rightarrow \infty$), the voltage $i(t) \rightarrow 0$
    (open circuit) for a finite $v(t)$.
  \end{enumerate}

  From the above discussion, we see that for sinusoidal voltage and
  current, the voltage across a capacitor is lagging behind the current
  by 90 degrees, as it takes time to build up the charge $Q=\int i\;dt$ 
  and thereby the voltage $V_C=Q/C$; while the current through an inductor
  is lagging behind the voltage across it, as it takes time to build up 
  the magnective flux $\Psi=\int v\;dt$ and thereby the current $i=\Psi/L$.
  This fact can be easily memorized by 
  \htmladdnormallink{``ELI the ICE man''}{../ELI/ELI.html}
  (with E for voltage and I for current).

  Note the following dimensionalities:
  \begin{equation} 
    \left[\sqrt{\frac{L}{C}}\right]=\sqrt{\frac{[Henry]}{[Farad]}}
    =\sqrt{\frac{[Volt]\;[second]}{[Ampere]}\frac{[Volt]}{[second]\;[Ampere]}}
    =\frac{[Volt]}{[Ampere]}=[Ohm] 
  \end{equation}

  \begin{comment}
  \begin{equation} 
    \left[\sqrt{\frac{L}{C}}\right]=\sqrt{\frac{H}{F}}
    =\sqrt{\frac{V\;s}{A}\;\;\frac{V}{s\;A}}=\frac{V}{A}=\Omega
  \end{equation}
  \end{comment}

  \begin{equation} 
    \left[ \sqrt{LC} \right]=\sqrt{[Henry]\;[Farad]}
    =\sqrt{\frac{[Volt]\;[second]}{[Ampere]}\frac{[Ampere]\;[second]}{[Volt]}}
    =[second]
  \end{equation}

  \begin{comment}
  \begin{equation} 
    \left[\sqrt{LC}\right]=\sqrt{H\; F}=\sqrt{ \frac{V\;s}{A} \frac{A\;s}{V} }
    =s
  \end{equation}
  \end{comment}

  Comparing the relationships between the current through and voltage
  across the three components below, we see that capacitance $C$ is a 
  conductive variable similar to $G$, while inductance $L$ is a resistive 
  variable similar to $R$.
  \begin{equation}
    \begin{array}{c||l|l}\hline
      \mbox{Resistor}  & i=v/R=Gv & v=Ri=i/G \\ \hline
      \mbox{Inductor}  & i=\int v\,dt/L & v=L\;di/dt \\ \hline
      \mbox{Capacitor} & i=C\,dv/dt & v=\int i\,dt/C \\ \hline
    \end{array}
  \end{equation}


\item {\bf Ideal Transformer}

  \htmladdimg{../figures/idealtransformer.gif}

  Two coils around a common iron core form a transformer. Assume the
  primary coil has $N_1$ turns of wire and the secondary coil has $N_2$ 
  turns. The total magnetic flux $\Psi=N\Phi$ is proportional to the 
  number of turns $N$, where $\Phi$ is the flux with one turn of wire
  in both primary and secondary coils. We assume the transformer is 
  ideal or lossless, in the sense that 
  \begin{itemize}
  \item There is no flux loss. The same flux $\Phi$ goes through the 
    iron core of both primary and secondary coils; 
  \item There is no power loss, the power $p$ received by the primary 
    coil is is completely delivered to the secondary coil.
  \end{itemize}

  {\bf Faraday's Law:} The voltage across a coil is proportional to 
  the rate of change of the total magnetic flux: 
  \begin{equation} 
    v_1(t)=\frac{d\Psi_1}{dt}=N_1\frac{d\Phi}{dt},\;\;\;\;\;\;
    v_2(t)=\frac{d\Psi_2}{dt}=N_2\frac{d\Phi}{dt} 
  \end{equation}
  i.e.,
  \begin{equation}
    \frac{v_2}{v_1}=\frac{N_2}{N_1}	
  \end{equation}
  Also, as there is no power loss in an ideal transformer, the powers on
  both the primary and secondary sides are the same:
  \begin{equation}
    p_1=v_1 i_1=p_2=v_2 i_2,\;\;\;\;\;\mbox{i.e.,}\;\;\;\;\;\;\;
    \frac{i_1}{i_2}=\frac{v_2}{v_1}=\frac{N_2}{N_1}
  \end{equation}
  from which we get:
  \begin{equation}
    v_2=\frac{N_2}{N_1} v_1,\;\;\;\;\;\;\;i_2=\frac{N_1}{N_2} i_1
  \end{equation}
  If we assume the secondary side is connected to a load resistance $R_L$,
  then according Ohm's law, we have $v_2/i_2=R_2$, and
  \begin{equation}
    \frac{v_1}{i_1}=\frac{(N_1/N_2)v_2}{(N_2/N_1)i_2}
    =\left(\frac{N_1}{N_2}\right)^2 \frac{v_2}{i_2}
    =\left(\frac{N_1}{N_2}\right)^2 R_L=R'_L 
  \end{equation}
  which is the equivalent resistance that appears on the primary side.

\end{itemize}


\subsection*{Energy Dissipation/Storage in R, C, and L}

The electric power associated with an element (R, C, or L) is the 
product of the voltage $v(t)$ across and current $i(t)$ through the
element:
\begin{equation} 
  p(t)=\frac{dw}{dt}=\frac{dw}{dq}\;\frac{dq}{dt}=v(t)\,i(t)
\end{equation}
Integrating $p(t)$ over a time period $T$, we get the energy 
\begin{equation}
  w=\int_0^T p(t)\; dt=\int_0^T v(t)\; i(t)\; dt	
\end{equation}
Depending on its sign, the energy can be either consumed (dissipated,
converted to heat) if $w>0$, or stored in the element if $w<0$. We 
consider specifically the energy dissipation/storage in each of 
the three types of elements $R$, $C$, and $L$.

\begin{itemize}

\item {\bf Energy dissipated by resistor $R$}

  When a voltage $v(t)$ is applied across $R$, the current through it is
  $i(t)=v(t)/R$, power consumption is 
  \begin{equation} 
    p(t)=v(t)\;i(t)=\frac{v^2(t)}{R}=i^2(t)R 
  \end{equation}
  The energy dissipated during time period $0 \sim T$ is
  \begin{equation}
    w=\int_0^T p(t) dt=\int_0^T v(t)\; i(t) dt
    =\frac{1}{R}\;\int_0^T v^2(t)\; dt =R\;\int_0^T i^2(t)\; dt 
  \end{equation}
  This energy is converted irreversibly from electrical energy to heat. The 
  rate of dissipation is $p(t)=d\,w/dt$.

  \begin{itemize}
    \item When a DC voltage $v(t)=V$ is applied across $R$, the current 
      through it is $I=V/R$, and the power consumed by the resistor is 
      \begin{equation}
        P=VI=\frac{V^2}{R}=I^2R 
      \end{equation}
      The energy dissipation during time period $0\sim T$ is
      \begin{equation} 
        w=\int_0^T P\; dt=PT=VIT=\frac{V^2}{R}\,T=I^2RT 
      \end{equation}
    \item When a sinusoidal voltage $v(t)=V_p \sin(\omega t)$ is applied 
      across $R$, the current through it is $i(t)=v(t)/R=V_p\sin(\omega t)/R$, 
      and the energy dissipated in time period $T=1/f=2\pi/\omega$ is:
      \begin{equation}
        w=\frac{1}{R}\int_0^T v^2(t) dt
        =\frac{V_p^2}{R}\int_0^T \sin^2(\omega t) dt
        =\frac{V_p^2}{2R}\int_0^T [1-\cos(2\omega t)] dt=\frac{V_p^2}{2R}\,T
      \end{equation}
      We define the {\em effective} or {\em root-mean-square (RMS)} $V_{rms}$ 
      as the equivalent DC voltage needed for $R$ to dissipate the same amount 
      of energy as $v(t)=V_p \sin(\omega t)$:
      \begin{equation} 
        \frac{{\bf V}^2_{rms}}{R}T=\frac{1}{R}\int_0^T v^2(t)dt=\frac{V_p^2}{2R}T
      \end{equation}
      Solving for $V_{rms}$ we get
      \begin{equation} 
        V_{rms}=\sqrt{\frac{1}{T}\int_0^T v^2(t) dt }=\frac{V_p}{\sqrt{2}}=0.707\,V_p
      \end{equation}
      i.e., the same amount of energy is dissipated by resistor $R$ when 
      either a sinusoidal voltage with peak amplitude $V_p$ or a DC voltage 
      $V_{rms}=V_p/\sqrt{2}$ is applied across it.
   
      \htmladdimg{../figures/dissipation.gif}

      Some useful trigonometric identities:
      \begin{equation} 
        \sin\alpha \cos\alpha=\frac{1}{2} \sin{2\alpha},\;\;\;
        \sin^2\alpha =\frac{1}{2}[1-\cos(2\alpha)],\;\;\;
        \cos^2\alpha =\frac{1}{2}[1+\cos(2\alpha)] 
      \end{equation}

      The average of a time varying current $i(t)$ is the value of
      a DC (direct current) current $I_{av}$ that in period $T$ would 
      transfer the same charge $Q$:
      \begin{equation}	
        I_{av}T=Q=\int_0^T i(t) dt,\;\;\;\;\mbox{i.e.}\;\;\;\;
        I_{av}=\frac{1}{T}\int_0^T i(t) dt	
      \end{equation}
      Similarly, the average voltage is defined as:
      \begin{equation}
        V_{av}=\frac{1}{T}\int_0^T v(t) dt	
      \end{equation}
      If the current/voltage is sinusoidal
      \begin{equation}
        i(t)=I_p\,\sin(\omega t)=I_p\,\sin(2\pi ft)=I_p\,\sin(2\pi t/T)
      \end{equation}
      The average over the complete cycle $T=1/f$ is always zero (the 
      charge transferred during the first half is the opposite to that 
      transferred in the second). However, if we consider the half-cycle 
      over $T/2$, the average is:
      \begin{eqnarray} 
        I_{av}&=&\frac{1}{T/2}\int_0^{T/2} i(t)\; dt
        =\frac{2}{T}\int_0^{T/2} \;I_p\,\sin(2\pi t/T)\;dt	
        =-\frac{2}{T}\frac{T}{2\pi} I_p\,\cos(2\pi t/T)\bigg|_0^{T/2}
        \nonumber \\
        &=& \frac{I_p}{\pi}\left[\cos(0)-\cos(\pi)\right]
        =\frac{2}{\pi}\,I_p=0.637\,I_p
      \end{eqnarray}

  \end{itemize}

\item {\bf Energy storage in capacitor $C$}

  Given voltage $v(t)$ across and current $i(t)$ through a capacitor $C$,
  the associated energy is:
  \begin{equation}	
    w=\int_0^T p(t)dt=\int_0^T v(t)\; i(t) dt
    =\int_0^T v(t) \;C\frac{dv(t)}{dt} dt
    =C \int_0^V v\;dv=\frac{1}{2}CV^2
  \end{equation}
  where we have assumed $v(0)=0$ and $v(T)=V$. 

  If $v(t)=V_p \sin(\omega t)$, then
  $i(t)=C\;dv(t)/dt=\omega C V_p\;\cos(\omega t)$, and the energy dissipated
  in period $T=2\pi/\omega$ is
  \begin{equation} 
    w= \int_0^T v(t)\; i(t) dt
    =V_p^2\omega C\int_0^T \sin(\omega t)\;\cos(\omega t)\,dt
    =\frac{V_p^2\omega C}{2} \int_0^T \sin(2\omega t) dt=0 
  \end{equation}
  This results indicates that there is no energy dissipated over the complete 
  period $T$. In the first and third quarters of the period $T$, the energy
  is stored in the electric field of the capacitor (equivalent to a battery 
  being charged), but in the 2nd and 4th quarters of the period $T$, the energy
  is released from the capacitor to the rest of the circuit (equivalent to a 
  battery delivering power).
  
\item {\bf Energy storage in inductor $L$}

  Given voltage $v(t)$ across and current $i(t)$ through an inductor $L$,
  the associated energy is
  \begin{equation}	
    w=\int_0^T p(t)dt=\int_0^T i(t)\; v(t) dt
    =\int_0^T i(t) \;L \frac{di(t)}{dt} dt
    =L \int_0^I i\;di=\frac{1}{2}LI^2
  \end{equation}
  where we have assumed $i(0)=0$ and $i(T)=I$. 

  If $i(t)=I_p \sin(\omega t)$, then
  $v(t)= L\;di(t)/dt=L I_p\omega\;\cos(\omega t)$, and the energy dissipated
  in time period $T=2\pi/\omega$ is
  \begin{equation}
    w=\int_0^T v(t)\;i(t) dt=I_p^2\omega L\int_0^T \sin(\omega t)\;\cos(\omega t)\,dt
    =\frac{I_p^2\omega L}{2} \int_0^T \sin(2\omega t) dt=0 
  \end{equation}
  Again, no energy is dissipated by the inductor during the complete period
  $T$ of a sinusoidal voltage. In the first and third quarter of the period 
  $T$, the energy is stored in the magnetic field of the inductor, but in 
  the 2nd and 4th quarter of the period $T$, the energy is released from the 
  inductor to the rest of the circuit.

\end{itemize}

The figure below shows the plots of the voltage across and current through the 
capacitor and inductor. We note that the voltage and current are $\pi/2$ or
$90^\circ$ out of phase. Specifically,
\begin{itemize}
\item Capacitor: the voltage $v_C(t)=\sin(\omega t)$ (red) lags the current
  $i_C=\omega C\cos(\omega t)$ (green) by $\pi/2$ (or $90^\circ$).
\item Inductor: the voltage $v_L(t)=\omega L \cos(\omega t)$ (green) leads 
  the current $i_L=\sin(\omega t)$ (red) by $\pi/2$ (or $90^\circ$).
\end{itemize}

The figure below illustrates the energy flow in a circuit involving capacitor
and inductor, as energy storing components:

\htmladdimg{../figures/storage.gif}




\htmladdimg{../figures/energyexchange.gif}



Comparison of Energy storage in mechanical and electromagnetic systems:
\begin{itemize}
\item {\bf Electromagnetic energy:}
  \begin{itemize}
  \item Energy stored in a capacitor with capacity $C$ and voltage $V$:
    \begin{equation}	
      w=\frac{1}{2}CV^2
    \end{equation}
    \begin{eqnarray}
      &&[Farad][volt]^2=\frac{[Ampere][second]}{[Volt]}[Volt]^2
      \nonumber\\
      &=&[Ampere][Volt][second]=[Watt][second]=[Joule]
    \end{eqnarray}
  \item Energy stored in an inductor with inductance $L$ and current $I$:
    \begin{equation}	
      w=\frac{1}{2}LI^2
    \end{equation}
    \begin{eqnarray}
      &&[Henry][Ampere]^2=\frac{[Volt][second]}{[Ampere]}[Ampere]^2
      \nonumber\\
      &=&[Volt][Ampere][second]=[Watt][second]=[Joule]
    \end{eqnarray}
  \end{itemize}
  
\item {\bf Mechanical energy:}
  \begin{itemize}
  \item {\bf Kinetic Energy:} Energy stored in a mass of 1 kilogram
    moving with a velocity $u$ of 1 meter per second possesses 1/2 Joule 
    of kinetic energy.
    \begin{equation}	
      w=\frac{1}{2}mu^2,\;\;\;\;\;\frac{[kilogram][meter]^2}{[second]^2}=[Joule]
    \end{equation}
    Another unit for energy is {\em calorie}:
    $ 1 \; \mbox{cal}=4.2\;\mbox{Joules}	$
    
  \item {\bf Potential energy:} Energy stored in a spring ($x=Cf=f/K$) 
    of {\em stiffness} $K$ or {\em compliance} $C=1/K$ is 
    \begin{equation}
      w=\frac{1}{2}Cf^2,\;\;\;\;\;	
      \frac{[meter]}{[Newton]}[Newton]^2=[meter][Newton]=[Joule]
    \end{equation}	
  \end{itemize}
  
\end{itemize}



{\bf Comparison with mechanical systems:}

The work of a mechanical system does is $w=\int_0^X f(x)\; dx$ where $f(x)$ 
is force and $X$ is displacement.
\begin{itemize}		
\item {\bf Potential energy:} A spring can be described by Hooke's law $f=Kx$
  where $K$ is the {\em stiffness}, or $x=Cf$ where $C=1/K$ is the 
  {\em compliance}. The potential energy stored in the spring is
  \begin{equation}
    w=\int_0^X f(x)\;dx=\int_0^F f\;C\;df=\frac{1}{2}CF^2	
  \end{equation}	
  i.e., the compliance $C$ is a measure of the spring's ability to 
  store potential energy (the less stiff, the more potential energy can
  be stored in the spring with the same force).
\item {\bf Kinetic energy:} A mass moving at a velocity $v$ has kinetic
  energy
  \begin{equation}
    w=\int_0^T f\;v(t)\;dt=\int_0^T m \frac{dv}{dt}\;v(t)\;dt
    =m \int_0^V v\;dv=\frac{1}{2}mV^2 
  \end{equation}
  i.e., the mass $m$ of a body is a measure of the body's ability to 
  store kinetic energy (the more mass, the more kinetic energy can be 
  stored in the body with the same velocity).
\end{itemize}		

\subsection*{Examples: Mechanical and Electrical Systems}

\begin{itemize}
\item A mechanical system composed of a spring $k$ and a mass $m$ can be 
  described by the following equation (Newton's second law $f=ma$):
  \begin{equation} 
    m\frac{d^2}{dt^2}x(t)+kx(t)=0,\;\;\;\;\mbox{i.e.}\;\;\;\;
    \frac{d^2}{dt^2}x(t)+\frac{k}{m}x(t)=\ddot{x}+\omega_n^2 x(t)=0	
  \end{equation}
  where $x(t)$ is the displacement of the mass (horizontal) and 
  $\omega_n=\sqrt{k/m}=1/\sqrt{cm}$ ($c=1/k$ is the compliance of the spring).

\item An electrical system composed of a capacitor $C$ and an inductor $L$
  in parallel can be described by:
  \begin{equation}	
    i_C(t)=C\frac{dv(t)}{dt},\;\;\;v(t)=L\frac{di_L(t)}{dt}	
  \end{equation}
  where $v(t)$ is the voltage across both components, and $i_C(t)$ and $i_L(t)$
  are currents through $C$ and $L$, respectively. 

  \htmladdimg{../LCExample.gif}

  As $i_C(t)+i_L(t)=0$, i.e. $i_C(t)=-i_L(t)$, we have
  \begin{equation}
    v(t)=L\frac{d}{dt}i_L(t)=L\frac{d}{dt}\;(-C\frac{dv}{dt}),\;\;\;
    \mbox{i.e.}\;\;\;\;\frac{d^2v(t)}{dt^2}+\frac{1}{LC}v(t)
    =\ddot{v}(t)+\omega_n^2 v(t)=0	
  \end{equation}

\end{itemize}

To find the homogeneous solution of the differential equation above 
(for either the mechanical or electrical system), we assume $v(t)=e^{st}$ 
and get $\ddot{v}(t)=s^2 e^{st}$, and the DE becomes:
\begin{equation}
  (s^2+\omega_n^2) e^{st}=0,\;\;\;\;\;\;\mbox{i.e.,}\;\;\;\;\;\;
  s^2+\omega_n^2=0 
\end{equation}
Solving this we get $s=\pm j\omega_n$ and 
\begin{equation}
  v(t)=e^{j\omega_n t}=\cos \omega t\pm j\sin \omega_n t 
\end{equation}

\begin{itemize}
\item In the mechanical system, when the displacement is zero $x(t)
  =\sin(\omega_n t)=0$, i.e., the potential energy stored in the spring 
  is zero, the velocity $\dot{x}(t)=\omega_n\;\cos(\omega_n t)=\pm \omega_n$,
  i.e., the kinetic energy is maximal. On the other hand, when the 
  displacement is $x(t)=\sin(\omega_n t)=\pm 1$, i.e., the potential 
  energy stored in the spring is maximal, the velocity $\dot{x}(t)=\omega_n
  \;\cos(\omega_n t)=0$, i.e., the kinetic energy is zero.
\item In the electrical system, when $v(t)=\sin(\omega_n t)=0$, i.e., the
  energy stored in the capacitor is zero $W_C=Cv^2/2=0$,
  $i(t)=C\dot{v}(t)=\omega_n C\;\cos(\omega_n t)=\pm \omega_n C$, i.e., 
  the energy stored in the inductor $W_L=Li^2/2$ is maximal. On the other
  hand, when $v(t)=\sin(\omega_n t)=\pm 1$, i.e., the energy stored in the 
  capacitor is maximal, $i(t)=C\dot{v}(t)=\omega_n C\;\cos(\omega_n t)=0$,
  i.e., the energy stored in the inductor is zero. 
\end{itemize}
In both cases, the energy in the system is converted back and forth 
between different forms (potential vs. kinetic in the mechanical system,
and electrical vs. magnetic in the electrical system), while the total
amount is always reserved. 

However, when a dash-pot (causing friction proportional to speed $\dot{x}$)
is added (in parallel to the spring) in the mechanical system, and a resistor 
is added in series with the electrical circuit, the energy is dissipated 
(converted to heat) in both systems:

\begin{equation}
  \frac{d^2x}{dt^2}+\frac{b}{m}\frac{dx}{dt}+\frac{k}{m}x=0	
\end{equation}
\begin{equation}
  \frac{d^2v}{dt^2}+\frac{R}{L}\frac{dv}{dt}+\frac{1}{LC}v=0	
\end{equation}

\htmladdimg{../figures/RCLExample.gif}

The corresponding solution of the DEs will be decaying sinusoidal,
indicating the dissipation of the energy in the system.

Consider the power in the RCL electrical system:
\begin{equation} 
  p(t)=v(t) i(t) =v(t) \frac{d}{dt} Q(t) 
\end{equation}
The energy consumed in the system is:
\begin{equation} 
  w=\int_0^t P(\tau) d\tau=v(t) [Q(t)-Q(0)]=Cv^2(t) 
\end{equation}
where we have assumed $Q(0)=0$ and $Q(t)=Cv(t)$. On the other hand,
as we also know the energy stored in $C$ is $W_C=Cv^2/2$, we see that
half of the energy is consumed (dissipated/stored) in the rest of 
the circuit ($R$ and $L$). This is always the case independent of 
the system parameters. When the input voltage is DC $v(t)=V$, the 
current at the steady-state is zero and the energy stored in $L$ is 
$Li^2/2=0$, i.e., half of the energy from the source is dissipated
by $R$. However, when $R=0$, the energy is converted back and forth
between $C$ and $L$ as described above.


\subsection*{Kirchhoff's Laws}

Here are some terminologies for electric circuits:
\begin{itemize}
\item {\em Node:} a point where three or more current-carrying elements 
  (branches) are connected;
\item {\em Branch:} a path connecting two nodes along which there is 
  one or more elements in series; 
\item {\em Loop:} a sequence of multiple paths that form a closed loop.
\end{itemize}

In a circuit diagram, both the direction of a current through, and the
polarity of a voltage across an element can be arbitrarily labeled. 
The actual direction and polarity will be determined by the sign of 
the specific values obtained after the circuit is solved. For example, 
a current labeled in the left-to-right direction with a negative value
is actually flowing in the right-to-left direction.

{\bf Kirchhoff's Laws}
\begin{itemize}
\item {\bf Kirchhoff current Law (KCL)} 

  The algebraic sum of the currents into a node is zero:
  \begin{equation}
    \sum_k I_k=0	
  \end{equation}
  due to the principle of conservation of electric charge (electric
  charge can not be created or destroyed in the circuit).  

  Here we can assume the directions of all currents through the elements
  are either into or out of the node.

\item {\bf Kirchhoff voltage Law (KVL)} 

  The algebraic sum of all voltage drops around a loop is zero:
  \begin{equation} 
    \sum_k w_k=\sum_k V_k Q=0,\;\;\;\;\;  \sum_k V_k=0	
  \end{equation}
  due to the principle of conservation of energy (energy can not be
  created or destroyed in the circuit).
  
  Here we can assume the polarities of all voltages across the elements
  are from high (+) to low (-), while going around the loop in either 
  clockwise or counter-clockwise direction.
\end{itemize}

{\bf Example 1:}

\htmladdimg{../figures/Kirchhoff.png}

Assume currents flowing into the node are positive and those leaving the 
node negative, the KCL states: $4+5-3-4-2=0$. 

Assume the current flows around the loop in clockwise direction, the KVL
states: $-12+3+4+5=0$. 

{\bf Example 2:}  Given the circuit below, find $V_2$, $V_0$, $I_2$, $R_1$
and $R_2$.

\htmladdimg{../figures/Kirchhoff1.gif}

According to Ohm's law, we have $I_2=3/2=1.5A$.  

Apply KVL to the loop on the right to get:
\begin{equation} 
  V_2-5+3=0,\;\;\;\;\;\; V_2=2V 
\end{equation}

According to Ohm's law, we have $R_2=V_2/I_2=2V/1.5A=1.33\Omega$.

Apply KCL to the middle node on top to get:
\begin{equation}
  2-I_1-I_2=2-I_1-1.5=0,\;\;\;\;\;I_1=0.5A 
\end{equation}

Again by Ohm's law, we get $R_1=5V/0.5A=10\Omega$. 

Apply KVL to the loop on the left to get:
\begin{equation} 
  3\times 2 +5-V_0=0,\;\;\;\;\;\;V_0=11V 
\end{equation}

{\bf The series and parallel combinations of circuit components}

\begin{itemize}

\item {\bf Resistors in series:} 

  \htmladdimg{../figures/resistorseries.gif}

  According to KVL, the sum of voltages across 
  the resistors is equal to the input voltage:
  \begin{eqnarray}	
    V=V_1+\cdots+V_n&=&IR_1+\cdots+IR_n=I(R_1+\cdots+R_n)=IR_s
    \nonumber \\
    &=&\frac{I}{G_1}+\cdots+\frac{I}{G_n}
    =I\left(\frac{1}{G_1}+\cdots+\frac{1}{G_n}\right)=\frac{I}{G_s}
  \end{eqnarray}
  where
  \begin{equation} 
    R_s=\frac{V}{I}=R_1+\cdots+R_n	
  \end{equation}
  and
  \begin{equation} 
    G_s=\frac{1}{R_s}=\frac{I}{V}=\frac{1}{1/G_1+\cdots+1/G_n}
  \end{equation}

\item {\bf Voltage divider:}

  According to Ohm's law, the voltage $V_k$ across the kth resistor $R_k$ can be 
  found to be:
  \begin{equation}
    V_k=IR_k=\frac{V}{R_s}R_k=V\frac{R_k}{R_1+R_2+\cdots+R_n} 
  \end{equation}
  In particular if $n=2$, we have
  \begin{equation} 
    V_1=V\frac{R_1}{R_1+R_2},\;\;\;\;\;\;\;\;V_2=V\frac{R_2}{R_1+R_2} 
  \end{equation}

\item {\bf Resistors in parallel:} 

  \htmladdimg{../figures/resistorparallel.gif}

  According to KCL, the sum of currents through the resistors is equal to 
  the input current:
  \begin{eqnarray}	
    I=I_1+\cdots+I_n&=&\frac{V}{R_1}+\cdots+\frac{V}{R_n}
    =V\left(\frac{1}{R_1}+\cdots+\frac{1}{R_n}\right)=\frac{V}{R_p}	
    \nonumber \\
    &=&VG_1+\cdots+VG_n=V(G_1+\cdots+G_n)=VG_p
  \end{eqnarray}
  where
  \begin{equation} 
    R_p=\frac{V}{I}=\frac{1}{1/R_1+\cdots+1/R_n}=R_1||R_2||\cdots||R_n 
  \end{equation}
  and
  \begin{equation} 
    G_p=\frac{1}{R_p}=\frac{I}{V}=G_1+\cdots+G_n=\frac{1}{R_p}
    =\frac{1}{R_1}+\cdots+\frac{1}{R_n}
  \end{equation}

  In particular, when $n=2$, 
  \begin{equation}
    R_p=R_1||R_2=\frac{1}{1/R_1+1/R_2}=\frac{R_1\;R_2}{R_1+R_2}	
  \end{equation}

\item {\bf Current divider:}

  According to Ohm's law, the current $I_k$ through the kth resistor $R_k$ can be 
  found to be:
  \begin{equation} 
    I_k=\frac{V}{R_k}=\frac{IR_p}{R_k}=I\frac{1/R_k}{1/R_1+1/R_2+\cdots+1/R_n}
    =I\frac{G_k}{G_1+G_2+\cdots+G_n} 
  \end{equation}
  In particular if $n=2$, we have
  \begin{equation} 
    I_1=I\frac{G_1}{G_1+G_2}=I\frac{1/R_1}{1/R_1+1/R_2}
    =I\frac{R_2}{R_1+R_2}
  \end{equation}
  \begin{equation}
    I_2=I\frac{G_2}{G_1+G_2}=I\frac{1/R_2}{1/R_1+1/R_2}
    =I\frac{R_1}{R_1+R_2} 
  \end{equation}

\item {\bf Inductors in series:} According to KVL, the sum of voltages across 
  the inductors is equal to the input voltage:
  \begin{equation}	
    v=v_1+v_2+\cdots+v_n=L_1\frac{di}{dt}+\cdots+L_n\frac{di}{dt}
    =(L_1+L_2+\cdots+L_n)\frac{di}{dt}=L_s\frac{di}{dt}	
  \end{equation}
  i.e.,
  \begin{equation}
    L_s=L_1+L_2+\cdots+L_n	
  \end{equation}

\item {\bf Inductors in parallel:} According to KCL, the sum of currents through 
  the inductors is equal to the input current:
  \begin{equation}	
    i=i_1+i_2+\cdots+i_n
    =\frac{1}{L_1}\int v\; dt+\cdots+\frac{1}{L_n}\int v\; dt 
    =\left(\frac{1}{L_1}+\cdots+\frac{1}{L_n}\right)\int v\; dt
    =\frac{1}{L_p}\int v\; dt	 
  \end{equation}
  we get
  \begin{equation}
    \frac{1}{L_p}=\frac{1}{L_1}+\cdots+\frac{1}{L_n} 
  \end{equation}

\item {\bf Capacitors in parallel:} According to KCL, the sum of currents through 
  the resistors is equal to the input current:
  \begin{equation}	
    i=i_1+i_2+\cdots+i_n=C_1\frac{dv}{dt}+\cdots+C_n\frac{dv}{dt}
    =(C_1+C_2+\cdots+C_n)\frac{dv}{dt} =C_p\frac{dv}{dt}		
  \end{equation}
  i.e.,
  \begin{equation}	
    C_p=C_1+C_2+\cdots+C_n	
  \end{equation}

\item {\bf Capacitors in series:} According to KVL, the sum of voltages across 
  the capacitors is equal to the input voltage:
  \begin{equation}	
    v=v_1+v_2+\cdots+v_n
    =\frac{1}{C_1}\int i\;dt+\cdots+\frac{1}{C_n}\int i\;dt 
    =\left(\frac{1}{C_1}+\cdots+\frac{1}{C_n}\right)\int i\;dt
    =\frac{1}{C_s}\int i\;dt 	
  \end{equation}
  i.e.,
  \begin{equation}
    \frac{1}{C_s}=\frac{1}{C_1}+\cdots+\frac{1}{C_n} 
  \end{equation}

\end{itemize}

\begin{tabular}{c||c|c|c}\hline
  & \mbox{Resistor} $R$ & \mbox{Inductor} $L$ & \mbox{Capacitor} $C$ \\ \hline \hline
  \mbox{Governing Equation} & $v=Ri=i/G$, $i=v/R=Gv$ & $v=L\;di/dt$, $i=\int v\; dt/L$ & $v=\int i\;dt/C$, $i=C\;dv/dt$ \\ \hline
  \mbox{Series connection}   & $R_s=R_1+R_2$, $1/G_s=1/G_1+1/G_2$ 
  & $L_s=L_1+L_2$ & $1/C_s=1/C_1+1/C_2$ \\ \hline
  \mbox{Parallel connection} & $1/R_p=1/R_1+1/R_2$, $G_p=G_1+G_2$
  & $1/L_p=1/L_1+1/L_2$ & $C_p=C_1+C_2$ \\ \hline
\end{tabular}

{\bf Example 3} Consider the following six circuits as either current
or voltage dividers.

\htmladdimg{../figures/RCLPS.png}

\begin{itemize}
\item For each of the three parallel circuits, find $i_1$ and $i_2$ in
  terms of the given current $i$ and the resistances, capacitances, or
  inductances.
%  \begin{comment}
  \begin{itemize}
  \item Resistor circuit:
    \begin{equation}
      i_1=\frac{R_2}{R_1+R_2}i,\;\;\;i_2=\frac{R_1}{R_1+R_2}i
    \end{equation}
  \item Capacitor circuit:
    \begin{equation}
      i_1=\frac{C_1}{C_1+C_2}i,\;\;\;i_2=\frac{C_2}{C_1+C_2}i
    \end{equation}
  \item Inductor circuit:
    \begin{equation}
      i_1=\frac{L_2}{L_1+L_2}i,\;\;\;i_2=\frac{L_1}{L_1+L_2}i
    \end{equation}
  \end{itemize}
%  \end{comment}
\item For each of the three series circuits, find $v_1$ and $v_2$ in
  terms of the given voltage $v$ and the resistances, capacitances, or 
  inductances.
  \begin{itemize}
  \item Resistor circuit:
    \begin{equation}
      v_1=\frac{R_1}{R_1+R_2}v,\;\;\;v_2=\frac{R_2}{R_1+R_2}v
    \end{equation}
  \item Capacitor circuit:
    \begin{equation}
      v_1=\frac{C_2}{C_1+C_2}v,\;\;\;v_2=\frac{C_1}{C_1+C_2}v
    \end{equation}
  \item Inductor circuit:
    \begin{equation}
      v_1=\frac{L_1}{L_1+L_2}v,\;\;\;v_2=\frac{L_2}{L_1+L_2}v
    \end{equation}
  \end{itemize}

\end{itemize}



\subsection*{Energy Sources}

{\bf Ideal Energy Sources:}

Consider the following ideal voltage source $V_0$ and ideal current 
source $I_0$, both directly connected to a load resistor $R_L$. We want
to find both the load voltage $V$ across $R_L$ and the load current $I$
through $R_L$:

\begin{itemize}
\item An ideal voltage source provides constant voltage $V_0$ independent 
  of the current through it $I=V_0/R_L$. The ideal voltage source can
  provide constant voltage to any number of resistors in parallel with 
  the source, independent of how much current they each draw.

\item An ideal current source provides constant current $I_0$ independent 
  of the voltage across it $V=I_0 R_L$. The ideal current source can 
  provide constant current to any number of resistors in series with 
  the source.
\end{itemize}
\htmladdimg{../figures/VIsources.gif}

However, such ideal sources do not exist in reality, due to the following
dilemmas:
\begin{itemize}
\item If $R_L=0$ (short circuit), the load voltage $V=R_L I=0\ne V_0$,
\item If $R_L=\infty$ (open circuit), the load current $I=V/R_L=V/\infty=0\ne I_0$,
\end{itemize}

{\bf Realistic voltage source:} 

In reality, all voltage sources (e.g., a battery or a voltage amplifier 
circuit) can be more realistically modeled by an ideal voltage source $V_0$ 
in series with a nonzero {\em internal resistance} $R_0$, which causes an
internal voltage drop $IR_0$ due to the current $I$ drawn by the load $R_L$, 
so that the actual output voltage across the load $V=V_0-IR_0< V_0$ is lower 
than $V_0$. The load voltage $V$ and current $I$ are constrained by the 
following two relationships imposed by he voltage source $(V_0, R_0)$ and 
the load $R_L$:
\begin{equation} 
  \left\{ \begin{array}{lll}
    \mbox{Source:} &  V=V_0-IR_0\;\;\;\;\;\;\;\;
    (I=0,\;\;V=V_0),\;\;\;\;\;\;\;(V=0,\;\;I=V_0/R_0)\\ 
    \mbox{load:} & V=R_LI \end{array} \right.
\end{equation}
Solving for $I$ and $V$, we get:
\begin{equation} 
  \left\{ \begin{array}{l}
    I=V_0/(R_0+R_L) \\ V=R_L I=V_0\,R_L/(R_0+R_L) 
  \end{array} \right. 
\end{equation}
For the output (load) voltage $V$ to be as close to the voltage source $V_0$
as possible, the internal resistance $R_0$ of a voltage source needs to be as 
small as possible, ideally $R_0=0$.

\htmladdimg{../figures/VoltageSource.png}

The slope of the first curve is the internal resistance $R_0$ and the slope 
of the second curve is $R_L$. Solving these two equations we get load voltage
$V$ and current $I$.

Only in the case of an ideal voltage source with $R_0=0$ will $V=V_0$. 
For $R_0>0$, the heavier the load, i.e., the smaller $R_L$, the larger the 
load current $I$, and the lower the load voltage $V$:
\begin{equation}
  R_L \downarrow \Longrightarrow I \uparrow \Longrightarrow R_0I \uparrow
  \Longrightarrow V \downarrow 
\end{equation}
In particular, when the load is a short circuit ($R_L=0$), the output 
voltage of a battery is zero (instead of the specified $V_0=1.5V$), as 
the voltage drop across the nonzero internal resistance $R_0$ is the same 
as $V_0$, i.e., the electric energy of the battery is consumed internally,
with no energy delivered to external circuit.


{\bf Realistic current source:} 

In reality, all current sources (e.g., a solar cell or a current amplifier
circuit) can be modeled by an ideal current source $I_0$ in parallel with a 
nonzero {\em internal resistance} $R_0$, which causes an internal current 
$V/R_0$ so that the actual output current through the load $I=I_0-V/R_0<I_0$ 
is lower than $I_0$. The load voltage $V$ and current $I$ are constrained by 
the following two relationships imposed by the current source $(I_0, R_0)$ and 
the load $R_L$:
\begin{equation} 
  \left\{ \begin{array}{ll} 
    \mbox{Source:} & I=I_0-V/R_0\;\;\;\;\;\;\;\;\;
    (I=0,\;\;V=R_0I_0),\;\;\;\;\;\;\;(V=0,\;\;I=I_0) \\ 
    \mbox{Load:} & I=V/R_L 
  \end{array} \right.
\end{equation}
Solving for $V$ and $I$, we get
\begin{equation}
  \left\{ \begin{array}{l} V=I_0R_0R_L/(R_0+R_L)=I_0\;R_0||R_L \\
    I=V/R_L=I_0R_0/(R_0+R_L)\end{array} \right.
\end{equation}

\htmladdimg{../figures/CurrentSource.png}

For the output (load) current $I$ to be as close to the current source $I_0$
as possible, the internal resistance $R_0$ of a current source should be as
large as possible, ideally $R_0=\infty$.

%\htmladdimg{../figures/currentsource1.gif}

The slope of the first curve is the internal resistance $R_0$ and the slope 
of the second curve is $R_L$. Solving these two equations we get load voltage
$V$ and current $I$.

Only in the case of an ideal current source will $R_0=\infty$ and $I=I_0$. 
For $R_0<\infty$, the larger the load resistance $R_L$, the smaller the 
current $I_L$.
\begin{equation}
  R_L \uparrow \Longrightarrow V \uparrow \Longrightarrow V/R_0 \uparrow
  \Longrightarrow I \downarrow 
\end{equation}

{\bf Energy Source Conversion}

Any two circuits with the same voltage-current relation 
({\em V-I characteristics}) at the output port with $R_L$ are 
equivalent to each other, as they have the same external behavior, 
although they may be different internally.

\htmladdimg{../figures/sourcetransform.gif}

Comparing the voltage-current relations of the two energy sources:
\begin{itemize} 
\item The voltage source ($V_0, R_0$):
  \begin{equation}
    V=V_0-R_0I=R_0(V_0/R_0-I) 
\end{equation}
\item The current source $(I_0, R'_0)$:
  \begin{equation}
    V=R'_0(I_0-I)=R'_0I_0-R'_0I 
  \end{equation}
\end{itemize} 
If $R_0=R'_0$ and $V_0=I_0 R_0$, then the two sources have the same
V-I characteristics and are therefore equivalent to each other. We
also see that
\begin{itemize}
\item a non-ideal voltage source $(V_0, R_0)$ can be converted into a 
  non-ideal current source $(I_0=V_0/R_0,\,R_0)$,
\item a non-ideal current source $(I_0, R'_0)$ can be converted into a 
  non-ideal voltage source $(V_0=I_0R'_0,\,R'_0)$.
\end{itemize}

\htmladdimg{../figures/VoltageCurrentSource.png}

Both of the two energy sources above can be treated as either a voltage 
or a current source. 
\begin{itemize}
\item The source on the left has a small $R_0$ and is therefore a good 
  voltage source as the voltage $V$ received by the load is close to 
  the source voltage $V_0$, but a bad current source as the current $I$ 
  received by the load is much lower than the source current $I_0=V_0/R_0$. 
\item The source on the right has a large $R_0$ and is therefore a bad 
  voltage source as the voltage $V$ received by the load is much lower 
  than the source voltage $V_0=I_0R_0$, but a good current source as the 
  current $I$ received by the load is close to the source current.
\end{itemize}


{\bf The Internal Resistance $R_0$:}

The internal resistance $R_0$ can be found as the absolute value of the 
slope of the straight line of the V-I characteristic plot:
\begin{equation}
  R_0=\frac{|\Delta V|}{|\Delta I|}
\end{equation}
where $\Delta V$ and $\Delta I$ are respectively the difference in voltage 
and current between any two points along the line, which can be chosen 
arbitrarily. The most convenient choice would be the intersections of the 
straight line with the $V$ and $I$ axes:
\begin{equation}
  \begin{array}{c||c|c}\hline
    & \mbox{Open circuit ($I=0$)} & \mbox{Short circuit ($V=0$)} \\ \hline\hline
    \mbox{Voltage Source} & V_{oc}=V_0 & I_{sc}=V_0/R_0 \\ \hline
    \mbox{Current Source} & V_{oc}=I_0R_0 & I_{sc}=I_0 \\ \hline
  \end{array}
\end{equation}
In either case, we have:
\begin{equation}
  \frac{\mbox{open-circuit voltage}}{\mbox{short-circuit current}}
  =\frac{V_{oc}}{I_{sc}}=\left\{\begin{array}{cc}V_0/(V_0/R_0)&\mbox{(voltage source)}\\
  (I_0R_0)/I_0&\mbox{(current source)}\end{array}\right\}=R_0
\end{equation}
\begin{comment}
\begin{itemize}
\item For voltage source $(V_0, R_0)$, the open-circuit voltage is 
  $V_{oc}=V_0$, the short-circuit current is $I_{sc}=V_0/R_0$, and their 
  ratio is $R_0$. 
\item For current source with $(I_0, R_0)$, the open-circuit voltage is 
  $V_{oc}=I_0R_0$, the short-circuit current is $I_{sc}=I_0$, and their 
  ratio is $R_0$.
\end{itemize}
\end{comment}

\htmladdimg{../figures/InternalResistance.gif}
%\htmladdimg{../figures/InternalResistance1.png}

While this method can be used to find the internal resistance $R_0$ 
without knowing either $I_0$ or $V_0$ in theory, it may not be practical, 
as the short circuit current is difficult to get (the voltage source may 
be damaged). Instead, we can find some other two voltages and currents 
$V_i$ and $I_i$ ($i=1,2$) corresponding to two load resistors $R_1$ and 
$R_2$. Then $R_0$ can be found as the slope of the straight line determined 
by the two points $(V_1, I_1)$ and $(V_2, I_2)$. We see that the previous 
method can be considered as a special case when $R_1=\infty$ (open circuit) 
and $R_2=0$ (short circuit).

{\bf Example 1:} 

A given voltage source of $V_0=10V$ and $R_0=10\Omega$ can be converted to a 
current source of $I_0=V_0/R_0=1A$ with the same $R_0$ (and vice versa). A load 
of $R_L=40\Omega$ receives from this energy source a voltage $V_L=8V$ (80\% of 
the voltage source) and a current $I_L=0.2A$ (20\% of the current source). As 
the energy source has a low internal resistance $R_0$, it is a good voltage 
source but a poor current source.

\htmladdimg{../figures/sourceex1.gif}
\htmladdimg{../figures/PowerSources1.gif}

{\bf Example 2:} 

A given current source of $I_0=1 mA$ and $R_0=8 K\Omega$ can be converted to a 
voltage source of $V_0=I_0 R_0=8V$ with the same $R_0$ (and vice versa). A load 
of $R_L=2 K\Omega$ receives from this energy source a voltage $V_L=1.6V$ (20\%
of the voltage source) and a current $I_L=0.8 mA$ (80\% of the current source). 
As energy source has a high internal resistance $R_0$, it is a good current 
source but a poor voltage source.
				   
\htmladdimg{../figures/sourceex2.gif}
\htmladdimg{../figures/PowerSources2.gif}

{\bf Power Delivery/Absorption}

\begin{itemize}
\item {\bf Voltage source}
  \begin{itemize}
  \item If the direction of the current $I$ in a circuit is such that it 
    goes internally through a voltage source $V$ from its low potential 
    (-) to high (+), and externally through the rest of the circuit from
    high to low, then the polarities of the voltage and the current are
    consistent (both positive or negative depending on the assumed 
    polarity), and the source is delivering power $W=VI$.
  \item If the direction of the current is reversed, then the power 
    delivered $W=-VI$ will be negative, i.e., the voltage source is actually 
    receiving power. A typical example is the rechargeable battery in your 
    car that works in either of the two states.
  \end{itemize}
\item {\bf Current source}
  \begin{itemize}
  \item If the polarity of the voltage $V$ across a current 
    source is such that the head of the arrow of the current source is
    at high potential (+) and the tail of the arrow is at low potential (-),
    then the polarity of the voltage and the current is consistent, and 
    the current source is delivering power $W=VI$.
  \item If the direction of the current is reversed, then the power 
    delivered $W=-VI$ will be negative, i.e., the current source is actually 
    receiving power. 
  \end{itemize}
\end{itemize}

{\bf Example 3:}

The current in a circuit composed of an ideal voltage source $V_0=5\,V$ 
and a resistor $R=5\Omega$ is $I=V/R=1\,A$. The power consumption of the
resistor and voltage source are $W_R=I^2R=V^2/R=IV=5$ and $W_V=-IV=-5$,
respectively. The negative value of $W_V$ indicates the power is actually
not consumed but generated by the voltage source (converted from other 
forms of energy, e.g., chemical, mechanical, etc.)

{\bf Example 4:} 
    
In the circuit shown below, the ideal current source is $I_0=1A$, and the 
ideal voltage source is $V_0=2V$, the resistor is $R=3\Omega$. Find the
current through and voltage across each of the three components. Find the 
power delivered, absorbed, or dissipated by each of the three components.

\htmladdimg{../figures/powerdeliveryexample.gif}

The current source provides $I_0=1A$ current through the left branch 
(upward), while the voltage source provides $V_0=2V$ across all three
components. The current through $R$ is $2V/3\Omega=2/3A$ (downward), 
by KCL, the current through the voltage source is $I_v=1-2/3=1/3A$. 
We therefore have:

\begin{itemize}
\item Power dissipated by $R$ is $P_R=V_0^2/R=4/3\;W$,
\item Power received by voltage source is $P=V_0I_V=2/3\;W$.
\item Power delivered by current source is $P=V_0I_0=2\;W$ $(2/3+4/3=2)$.
\end{itemize}

(Homework) Redo the above with the polarity of $V_0$ reversed. Find:
\begin{itemize}
\item Power dissipated by $R$: 
\item Power received/delivered by voltage source:
\item Power delivered/received by current source:
\end{itemize}
Verify your results so that total power delivered is equal to total
power received and dissipated.

\htmladdnormallink{Answer}{../ch1_sub2/index.html}

{\bf Comment:} While various voltage sources such as batteries are common 
in everyday life, current sources do not seem to be widely available. One
type of current source is solar-cell, which generates current proportional
to the intensity of the incoming light. Also, certain transistor circuits
are designed to output constant current. Moreover, as discussed above, any
voltage source can be converted into a current source. For example, a 
current source with $I_0=1$ mA and $R_0=1\;M\Omega$ can be implemented
by a voltage source of $V_0=1000\; V$ in series with $R_0=1\;M\Omega$.


{\bf Example 5} (Homework)

\htmladdimg{../figures/sourcemeter.gif}

A realistic voltage source (e.g., a battery) can be modeled as an ideal 
voltage source $V_0$ in series with an internal resistance $R_0$. Ideally,
the voltage $V_0$ can be obtained by measuring the open-circuit voltage 
$V_{oc}$ with a voltmeter
\begin{equation} 
  V_0=V_{oc} 
\end{equation}
while the internal resistance $R_0$ can be obtained as the ratio of the 
open-circuit voltage $V_{oc}$ to the shirt-circuit current $I_{sc}$, which 
can be measured by an ammeter:
\begin{equation}
  R_0=\frac{V_{oc}}{I_{sc}} 
\end{equation}

However, in reality, any voltmeter has an internal resistance $R_v$ in 
parallel with the meter, and any ammeter has an internal resistance $R_a$ 
in series with the meter. For better measurement accuracy, should $R_v$ be
small or large, how about $R_a$? Why? Give the expression of the measured
open-circuit voltage $V$ and short-circuit current $I$ in terms of the
true $R_0$ and $V_0$, as well as $R_v$ and $R_a$.

Assume $V_0=6V, R_0=100\Omega, R_v=10,000 \Omega, R_a=200 \Omega$. What
are the measured open-circuit voltage $V$, and the short-circuit current 
$I$? Given $V$, $I$, and the known $R_v$ and $R_a$, how do your get the 
true $V_0$ and $R_0$ using your method above? Show your numerical computations.

\htmladdnormallink{Answer}{../ch1_sub1/index.html}

Design a method to obtain the true source voltage $V_0$ and internal
resistance $R_0$ by a voltmeter and an ammeter with known $R_v$ and $R_a$.
Give the expression of $V_0$ and $R_0$ in terms of the measured open-circuit
voltage $V$, short-circuit $I$, and $R_v$ and $R_a$. 

\htmladdnormallink{Answer}{../ch1_sub3/index.html}


{\bf Example 6} (Homework)

Usually the internal resistances of the voltmeter and ammeter are not 
readily known (and the values may change depending on the scale used). 
As another method to find $R_0$ and $V_0$ of a voltage source, we can 
measure the voltage $V_L$ across two different load resistors $R_L$ 
connected to the voltage source. If the values of $R_L$ are significantly
smaller than that of the internal resistance $R_v$ of the voltmeter, the 
voltmeter can be considered to be ideal with $R_v\rightarrow \infty$. 

Assume when the load resistor is $R_L=1 k\Omega$, the voltage across it 
is found to be $V_L=9.09V$, when a different load $R_L=2 k\Omega$ is used 
and the voltage across it is $V_L=9.52V$. Find $V_0$ and $R_0$ of the 
voltage source.

%\begin{comment}
{\bf Answer:}
\begin{equation}
  V_0\frac{1}{R_0+1}=9.09,\;\;\;\;\;\;V_0\frac{2}{R_0+2}=9.52
\end{equation}
i.e.,
\begin{equation}
  V_0=9.09\,R_0+9.09,\;\;\;\;\;\;\;V_0=4.76\,R_0+9.52
\end{equation}
Subtracting we get 
\begin{equation}
  R_0=\frac{9.52-9.09}{9.09-4.76}=\frac{0.43}{4.33}\approx 0.1\;k\Omega
\end{equation}
and 
\begin{equation}
  V_0=9.09\,R_0+9.09=9.09\times 0.1+9.09=10\,V
\end{equation}

%\htmladdnormallink{Answer}{../ch1_sub1/index.html}
%\end{comment}


\subsection*{Source and Load}

It is often needed to concatenate two circuits in series (cascade),
by connecting the output port of the first circuit, considered as 
the {\em source}, to the input port of the second circuit, considered 
as the {\em load}. Here are some examples: 
\begin{itemize}
\item The source is a voltage or current source, and the load is 
  a resistor.
\item The source is a sensor and the load is an amplifier.
\item The source is signal (waveform) generator that produces various
  waveforms (sinusoids, saw tooth, square wave, etc.), and the load is 
  an oscilloscope to display such waveforms. 
\item Two {\em voltage amplifiers} can be cascaded to achieve greater
  amplification gain to amplify some weak signals.
\item One or more filters can be cascaded to filter signals of 
  different frequencies.
\end{itemize}

In general, an {\em active circuit} containing {\em active components}
(e.g., transistors and operational amplifiers) as well as 
{\em passive components} (e.g., $R$, $C$ and $L$) can be modeled by an 
{\em input resistance} $R_{in}$ and a voltage source $v_{out}$ in series 
with an {\em output resistance} $R_{out}$, where $v_{out}=Av_{in}$ is 
proportional to the input voltage $v_{in}$ and $A$ is the voltage gain. 
When two such circuits are cascaded, $R_{out}$ of the first circuit is 
the internal resistance of the source, and $R_{in}$ of the second circuit 
is the resistance of the load.

\htmladdimg{../figures/cascade.gif}

Consider the following three examples:

\begin{itemize}
\item {\bf Maximize voltage delivery}

  The output resistance $R_{out}$ of the course and the input resistance
  $R_{in}$ of the load form a voltage divider:

  \htmladdimg{../figures/powermatch.gif}

  For the load to receive maximize voltage, we want 
  \begin{itemize}
  \item the output (internal) resistance $R_0=R_{out}$ of the source 
    circuit to be as small as possible, so that little voltage drop 
    across it will be caused when the load circuit draws a large curren 
    from the source even if its input (load) resistance $R_L=R_{in}$ is 
    small.

  \item the input (load) resistance $R_L=R_{in}$ of the load circuit to 
    be as large as possible, so that it draws little current from the 
    source, causing small internal voltage drop across its output 
    resistance $R_0=R_{out}$, even if it is large.
  \end{itemize}
  
\item {\bf Maximize power delivery}

  Some times we need to maximize the power (instead of the voltage)
  delivered from the source to the load. For example, the {\em power 
  amplifier} of a stereo system needs to deliver high power to the 
  speakers as the load. Given the internal resistance $R_0=R_{out}$ of 
  the source and the load resistance $R_L=R_{in}$, the power received 
  by the load is:
  \begin{equation}
    P_L=I^2 R_L=\left(\frac{V_0}{R_0+R_L}\right)^2 R_L	
    =V_0^2\frac{R_L}{(R_0+R_L)^2} 
  \end{equation}
  \begin{itemize}
  \item Given $R_L$, we can minimize $R_0$ to maximize the delivered
    power $P_L$:  
    \begin{equation}
      P_L=V_0^2\frac{R_L}{(R_0+R_L)^2}\stackrel{R_0\rightarrow 0}{\Longrightarrow}
      \frac{V_0^2}{R_L}
    \end{equation}
  \item On the other hand, given $R_0$, we need to find the optimal value 
    for $R_L$ to maximize $P_L$, by solving the following equation:
    \begin{equation}
      \frac{d}{dR_L} P_L(R_L)
      =V_0^2\frac{(R_0+R_L)^2-2R_L(R_0+R_L)}{(R_0+R_L)^4}	
      =V_0^2\frac{R_0-R_L}{(R_0+R_L)^3}=0
    \end{equation}
    We see that only when $R_L$ is equal to the internal resistance $R_0$,
    will it receive maximum power.
  \end{itemize}

  The maximum power received by $R_L=R_0$ is:
  \begin{equation}
    P_L=\frac{V_0^2}{(R_0+R_L)^2} R_L\bigg|_{R_L=R_0}=\frac{V_0^2}{4R_0} 
  \end{equation}
  the load current is 
  \begin{equation}
    I=\frac{V_0}{R_0+R_L}=\frac{V_0}{2R_0}
  \end{equation}
  and the total power delivered by the voltage source $V_0$ is 
  \begin{equation}
    P_0=V_0 I=\frac{V_0^2}{2R_0}=2P_L 
  \end{equation}
  The internal resistance $R_0=R_L$ consumes the same amount of power 
  as the load $R_L=R_0$. The figure below shows the power $P_L$ consumed 
  by the load resistor $R_L=x$ when $R_0=2$. We see that if $R_L$ is 
  either smaller than or greater than $R_0$, $P_L$ is reduced.

  \htmladdimg{../figures/maxpower.gif}

  The efficiency of the circuit is defined as the ratio of the power 
  delivered to the load $R_L$ and the power generated by the source $P_0$:
  \begin{equation}
    \eta=\frac{P_L}{P_0}=\frac{I^2 R_L}{I^2 (R_0+R_L)}=\frac{R_L}{R_0+R_L}
  \end{equation}
  When $R_L=R_0$ and the load receives maximal power, but the efficiency 
  is only $50\%$. We can improve the efficiency by increasing $R_L$ so 
  that $\eta$ approaches 1 when $R_L \gg R_0$. But in this case the 
  power received by the load is reduced. 

  For example, if $R_L=2R_0$, the efficiency becomes:
  \begin{equation}
    \eta=\frac{R_L}{R_0+R_L}\bigg|_{R_L=2R_0}=\frac{2}{3}>\frac{1}{2}	
  \end{equation}
  but the power received by the load is less than the maximum power:
  \begin{equation}
    P_L=I^2R_L=\frac{V_0^2}{(R_0+R_L)^2}R_L=V_0^2\frac{2R_0}{(R_0+2R_0)^2}
    =\frac{2V_0^2}{9R_0} < \frac{V_0^2}{4R_0} 
  \end{equation}
  In some applications with small power, efficiency can be sacrificed to 
  maximize the load power. 

  \begin{comment}
  For example, in an audio system, it is important for the speaker's 
  impedance to match the output impedance of the power amplification 
  circuit, so that the speaker can receive maximum power. As the 
  resistance of many speakers is $8\Omega$, all audio amplifiers need
  to be designed to have an $8\Omega$ output resistance in order to 
  deliver maximum power to the speaker. 

  Also, the {\em characteristic resistance} of most transmission 
  lines (e.g., coaxial cables to conduct high frequency signals) is 
  $50\Omega$, all signal generators need to be designed to have a 
  $50\Omega$ output resistance to avoid reflection. 
  \end{comment}

\item {\bf Minimize loss in power transmission line:} 

  \htmladdimg{../figures/powerdelivery.gif}

  A power transmission line delivers power from power generation (a 
  power plant) to power consumption (e.g., a city). The concern is no
  longer delivering maximum power (as long as needed power is delivered),
  but to achieve maximum efficiency in the sense that the power loss 
  along the power transmission line is minimized.

  We denote the resistance of the transmission line by $R_T$ and the total 
  load resistance of the power consumption by $R_L$. Also denote the voltage
  on the consumer's side by $V_L$. We have
  \begin{itemize}
  \item Power consumption by the load $R_L$: 
    \begin{equation} 
      P_L=V_L I\;\;\;\;\;\; \mbox{i.e.}\;\;\;\;\;\;\; I=\frac{P_L}{V_L}
    \end{equation}
  \item Power dissipation along the transmission line $R_T$: 
    \begin{equation}
      P_T=R_T I^2=R_T \;\left(\frac{P_L}{V_L}\right)^2=\frac{R_T\,P_L^2}{V_L^2}
    \end{equation}
  \end{itemize}
  As the transmission resistance $R_T$ is fixed (already minimized) and 
  the power consumption $P_L$ is to be guaranteed, to minimize the power
  loss $P_T$ along the transmission line we can only increase the voltage
  $V_L$. For example, when the voltage is increased 10 times, the power 
  loss will be reduced to 1/100. In practice, the voltage $V_L$ can be 
  as high as a few hundred $kV$ (69 kV, 115 kV, 230 kV, 500 kV, 765 kV).

\end{itemize}

{\bf Summary: } The circuits in the three examples above are essentially
the same, i.e., they all have a voltage source $V_0$ with an internal 
resistance $R_0$ (or $R_T$), and a load resistance $R_L$. However, the 
circuit will be optimized differently according to different requirements:

\htmladdimg{../figures/source_load.gif}

\begin{itemize}
\item {\bf Maximize voltage delivery (or minimize loading effect)}:
  minimize output resistance $R_0$, maximize input resistance $R_L$.
\item {\bf Maximize power delivery ($50\%$ of total power): } 
  match the output and input resistances $R_L =R_0$ 
\item {\bf Minimize power dissipation in transmission: } 
  increase output voltage $V_L$.
\end{itemize}

\subsection*{Review and Summary}

\htmladdnormallink{Review and Summary (A Handout)}{../handout1/index.html}

\end{document}


	

	

