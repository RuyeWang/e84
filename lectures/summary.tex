\documentstyle[12pt]{article}
\usepackage{html}
\textwidth 6.0in
\topmargin -0.5in
\oddsidemargin -0in
\evensidemargin -0.5in
% \usepackage{graphics}  
\begin{document}

\section*{Summary of First Half}

\section*{Basic Laws}
\begin{itemize}
\item Ohm's law, KVL, KCL, voltage/current divider
\item Non-ideal voltage source: $V_0$ in series with $R_0$
\item Non-ideal current source: $I_0$ in parallel with $R_0$
\item Conversion between current and voltage sources:
  \[ 
  R_V=R_I=R_0,\;\;\;\;\;\;V_0=I_0 R_0,\;\;\;\;\;I_0=V_0/R_0
  \]
\item The source-load view of circuit: The source $V_s$ in series with 
  $R_s$; the load $R_L$. 

  The voltage across $R_L$ and the current through $R_L$ are related by
  \[
  \left\{\begin{array}{ll}  V=V_s-IR_s & \mbox{dictated by source}\\
  V=IR_L & \mbox{dictated by load}\end{array}\right.
  \]
  The load $R_L$ can be replaced by any other component, such as 
  non-linear diode, transistor, etc.
  
\end{itemize}

\section*{DC circuit analysis}

\begin{itemize}
\item Loop-current method based on KVL
\item Node-voltage method based on KCL
\item Superposition principle
\item Thevenin's theorem: convert any circuit to a voltage source
  with $V_T$ and $R_T$ in series.
\item Norton's theorem: convert any circuit to a current source
  with $I_N$ and $R_N$ in parallel.
\item Conversion between the voltage and current sources.
  \[
  R_T=R_N,\;\;\;\;V_T=I_N R_N,\;\;\;\;\;I_N=V_T/R_T
  \]
\item Delta-Y conversion
\item A two-port network can be modeled in terms of three parameters: 
  \begin{itemize}
  \item Input resistance $R_{in}=v_{in}/i_{in}$ (Ohm's law) 
  \item Output resistance $R_{out}=v_{oc}/i_{sc}$ (Thevinin's theorem)
  \item Voltage source: $A_{oc} v_{in}$ in series with $R_{out}$, or a 
    current source: $I=A_{oc} v_{in}/R_{out}$ in parallel with $R_{out}$.
    where $A_{oc}$ is the open-circuit gain.
  \end{itemize}
  Transistors, op-amps, and many other circuits that takes an input 
  and generates an output can all be examples of such a two-port network.
\item The loading effact, the voltage $V=V_S-IR_S$ across load $R_L$ is
  lower than the open-voltage voltage of the source.
\end{itemize}

\section*{AC circuit analysis}

\begin{itemize}
\item Phasor representation of sinusoidal voltage and current:
  $\dot{V}=V_{rms} \angle \phi$, $\dot{I}=I_{rms} \angle \phi$.
\item Impedance $Z_R=R$, $Z_L=j\omega L$ and $Z_C=1/j\omega C$,
  $\dot{V}=Z\;\dot{I}$,   ``ELI the ICE man''
\item Based on impedances and phasors, all AC circuits in steady 
  state can be solved by the same methods used for DC circuit
  analysis.
\item First-order systems: RL with $\tau=L/R$ and RC with $\tau=RC$.
  \begin{itemize}
  \item Particular solution $x_p(t)$ (by phasor/impedance method)
  \item Homogeneous solution $x_h(t)$ (by solving DE)
  \item Complete solution, the ``short-cut method'' based on 
    (1) initial value $x(0^+)$, (2) steady-state value $x_\infty(t)$,
    and (3) time constant $\tau=RC$ or $\tau=L/R$.
  \item Used as LP or HP filters.
  \end{itemize}
\item Second-order systems: RLC in series, parallel, etc., with
  \[
  \omega_n=\frac{1}{\sqrt{LC}},\;\;\;\;\;\;
  \zeta_S=\frac{R}{2}\sqrt{\frac{C}{L}},\;\;\;\;\;\;
  \zeta_p=\frac{1}{2R}\sqrt{\frac{L}{C}}
  \]
  \begin{itemize}
  \item Particular solution $x_p(t)$ (by phasor/impedance method)
  \item Complete solution, solving DE
  \item Used filters:
    \[
    H(j\omega)=\frac{N(j\omega)}{D(j\omega)}
    =\frac{N(j\omega)}{(j\omega)^2+2\zeta\omega_n j\omega+\omega_n^2}
    =\frac{N(j\omega)}{(j\omega)^2+\Delta\omega j\omega+\omega_n^2}
    \]
    \[
    Q=\frac{1}{2\zeta},\;\;\;\;\;
    \Delta\omega=2\zeta\omega_n=\frac{\omega_n}{Q}
    \]
  \item 
    \[
    \left\{\begin{array}{cc}
    H(j\omega)=\omega_n^2/D(j\omega) & \mbox{LP}\\
    H(j\omega)=2\zeta\omega_n j\omega/{D(j\omega) & \mbox{BP}\\
    H(j\omega)=(j\omega)^2/D(j\omega) & \mbox{HP}
    \end{array}\right.
    \]
  \item Power factor: $\lambda=\cos\phi$

\end{itemize}

\section*{Semiconductor devices}

\begin{itemize}
\item PN junction and Diode:
\item transistor, common-emitter transistor circuit:
  input/output characteristic plots, load line, 
  cut-off, linear and saturation regions
\item DC operating point, biasing
\item AC small signal linear model,
  input/output resistance, AC gain
\end{itemize}

\section*{Op-Amp circuits}

\begin{itemize}
\item $v_{out}=A(v^+-v^-)$
\item large $r_{in}$, small $r_{out}$, large gain $A$
\item Based on ``virtual ground'' assumption, analysis of
  op-amp circuits can be simplified.


\end{document}





\item loop-current method based on KVL
\end{itemize}


\begin{itemize}

\item \bf{Governing equations of R,C and L}
  \[ v_R(t)=R\;i_R(t),\;\;\;\;i_R(t)=\frac{v_R(t)}{R} \]
  \[ v_L(t)=L\frac{d}{dt} i_L(t) dt,\;\;\;\;i_L(t)=\frac{1}{L}\int i_L(t) dt\]
  \[ v_C(t)=\frac{1}{C}\int i_C(t) dt,\;\;\;\;i_C(t)=C\frac{d}{dt} v_C(t) \]

\item \bf{Series and parallel connections of basic components R, C, and L}
\begin{itemize}
\item Series:
  \[ R=R_1+R_2,\;\;\;L=L_1+L_2,\;\;\;\frac{1}{C}=\frac{1}{C_1}+\frac{1}{C_2} \]
\item Parallel: 
  \[ \frac{1}{R}=\frac{1}{R_1}+\frac{1}{R_2},\;\;\;\frac{1}{L}=\frac{1}{L_1}+\frac{1}{L_2}\;\;\;\;C=C_1+C_2 \]

\end{itemize}

\item \bf{Basic circuit laws: Kirchhoff's laws (KVL, KCL), current and voltage 
  dividers}
  \begin{itemize}
    \item KVL: around a loop: $\sum_k v_k=0$
    \item KCL: at a node: $\sum_k i_k=0$
    \item Voltage divider: two resistors in series:
      \[V_1=V\; \frac{R_1}{R_1+R_2},\;\;\; V_2=V\; \frac{R_2}{R_1+R_2}\]
    \item Current divider: two resistors in parallel:
      \[I_1=I\; \frac{R_2}{R_1+R_2},\;\;\; I_2=V\; \frac{R_1}{R_1+R_2}\]
  \end{itemize}



    \item Internal resistance:
\[ R_0=\frac{\mbox{open-circuit voltage}}{\mbox{short-circuit current}}
  =\frac{V_{oc}}{I_{sc}} \]      
  \end{itemize}
  
\item \bf{Methods for solving circuits}
  \begin{itemize}
    \item loop current
    \item node voltage 
  \end{itemize}

\item \bf{Superposition principle}

  Find network's response to each of the multiple energy sources, the 
  overall response is their algebraic sum.

\item \bf{Thevenin's and Norton's theorems}
  \begin{itemize}
    \item Thevenin's theorem: treat the component in question as the load
      of a 1-port network with open-circuit voltage $V_T$ in series with
      an equivalent internal resistance $R_T$.
    \item Norton's theorem: treat the component in question as the load
      of a 1-port network with short-circuit current $I_N$ in parallel with
      an equivalent internal resistance $R_N$.
    \item Conversion between the two models:
      \[
      R_T=R_N,\;\;\;\;\;V_T=I_NR_N,\;\;\;\;\;I_N=V_T/R_T
      \]
  \end{itemize}

  

\item \bf{Delta-Y conversions}
\begin{itemize}
\item $\Delta$ to $Y$ conversion:
\[ \left\{ \begin{array}{rr}
	R_a=R_{ab}R_{ac}/(R_{ab}+R_{ac}+R_{bc}) \\
	R_b=R_{ab}R_{bc}/(R_{ab}+R_{ac}+R_{bc}) \\
	R_c=R_{ac}R_{bc}/(R_{ab}+R_{ac}+R_{bc}) \end{array} \right. \]
\item $Y$ to $\Delta$ conversion:
\[ \left\{ \begin{array}{rr}
	R_{ab}=R_a+R_b+R_aR_b/R_c	\\
	R_{ac}=R_a+R_c+R_aR_c/R_b	\\
	R_{bc}=R_b+R_c+R_bR_c/R_a	\end{array} \right. \]
\end{itemize}

\item \bf{Phasor representation of sinusoidal variables}
\[ x(t)=X_p cos(\omega t+\phi)=Re[X_p e^{j(\omega t+\phi)}]
   =Re[\dot{X} \sqrt{2}e^{j\omega t}] \]
     with phasor $\dot{X}=X_{rms}\;e^{j\phi}=X_p/\sqrt{2}\;e^{j\phi}$.

\item \bf{Impedance (resistance, reactance) and admittance (conductance, 
  susceptance)}
  \[
  Z=R+jX,\;\;\;\;\;\mbox{impedance}=\mbox{resistance}+j\;\mbox{reactance} 
  \]
  \[
  Y=G+jB,\;\;\;\;\;\mbox{admittance}=\mbox{conductance}+j\;\mbox{susceptance} 
  \]
  \[
  Y=\frac{1}{Z}=\frac{1}{R+jX}=\frac{R}{R^2+X^2}+j\;\frac{-X}{R^2+X^2}=G+jB 
  \]

\item \bf{Generalized Ohm's law, KCL and KVL}
  \[
  \dot{V}=\dot{I} Z,\;\;\;\sum_k \dot{V}_k=0,\;\;\;\;\sum_k \dot{I}_k=0 
  \]

\item \bf{Use phasor to solve AC circuits (complex calculations)}

\item \bf{Complete responses}
  \begin{itemize}
  \item Complete response to DC input (3 elements $f(\infty)$, $f(0_+)$ and 
    $\tau$) 
    \[ 
    f(t)=f(\infty)+[f(0_+)-f(\infty)]\;e^{-t/\tau} 
    \]

\item Complete response to AC input (3 elements $f_\infty(t)$, $f(0_+)$ and 
  $\tau$).
  \[ 
  f(t)=f_\infty(t)+[f(0_+)-f_\infty(0)]\;e^{-t/\tau}	
  \]
\item Superposition applicable when multi-source of different frequencies.
  \end{itemize}

\end{itemize}

\end{document}

\item Series and parallel resonance. Resonant frequency, peak frequency,
  bandwidth. Quality factor.

\item Power factor 

