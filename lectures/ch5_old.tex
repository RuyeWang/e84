\documentstyle[12pt]{article}
\usepackage{html}
\textwidth 6.0in
\topmargin -0.5in
\oddsidemargin -0in
\evensidemargin -0.5in
% \usepackage{graphics}  
\begin{document}

\section*{Chapter 5: Operational Amplifiers (Op-amps)}

{\bf Circuit Schematic}

\htmladdimg{../figures/opamp741.gif}

{\bf Operation:}	
\[	v_{out}=A v_{in}=A (v^+ - v^-)	\]

Although the schematic of an Op-amp looks complicated, its 
analysis can be simplified by the following approximation:
\begin{itemize}
\item The voltage gain $A$ is huge ($10^5 \sim 10^9$) and could be considered
	infinity, i.e., for finite output $v_{out}$, the input is
	close to zero $v_{in}=v^+ - v^-=0$ or $v^+=v^-$;
\item The input impedance is huge ($10^6\sim 10^{12} \;\Omega$ depending
	on whether BJT or FET is used), approximated as infinity 
	$z_{in}\rightarrow \infty$;
\item The op-amp draws little current from its source 
	($10^{-9}\sim10^{-12}\;A$), approximated as $i^+=i^-=0$;
\item The output impedance is small (50 $\Omega$), approximated as 
	zero $z_{out}=0$, i.e., the output $v_{out}$ is not affected
	by the load $R_L$ (so long as it is much greater than 50 $\Omega$).
\item The bandwidth is large ($1 \sim 20MHz$), approximated as infinity.
\end{itemize}

\htmladdimg{../figures/opam1.gif}

\begin{itemize}
\item {\bf Inverter}

\htmladdimg{../figures/opam2.gif}

As the input current is negligible, i.e., $i=i_1+i_2 \approx 0$, we have
\[
	i_1+i_2= \frac{v^- -v_1}{R_1}+\frac{v^- -v_o}{R_f}=0
\]
but also since
\[	v^- \approx v^+ = 0	\]
we have
\[	\frac{v_o}{v_1}=-\frac{R_f}{R_1}	\]
In particular, if $R_f=R_1$, we have
\[	v_o=-v_1	\]

In general, $R_1$ and $R_2$ in the inverter can be replaced by two networks
$Z_1$ and $Z_2$ containing resistors and capacitors and the analysis of the
circuit can be carried out easily in Laplace domain:
\[	H(s)=\frac{V_o}{V_i}=-\frac{Z_2(s)}{Z_1(s)}	\]
This is a convenient way to design filters of various frequency 
characteristics.
\htmladdimg{../figures/opam11.gif}

\item {\bf Summer-inverter}

\htmladdimg{../figures/opam3.gif}

As the input current $i$ is negligible, we have
\[ i=\sum_{j=1}^n \frac{v^--v_j}{R_j}+\frac{v^--v_o}{R_f} \approx 0	\]
and we also know
\[	v^- \approx v^+= 0 \]
we have
\[
	v_o=-R_f \sum_{j=1}^n \frac{v_j}{R_j} = - \sum_{j=1}^n k_j v_j \]
where $k_j \stackrel{\triangle}{=}R_f/R_j$ ($j=1,2,\cdots,n$) are the $n$ 
coefficients.

\item {\bf Summer with different signs}

\htmladdimg{../figures/opam5.gif}

Define $v$ as
\[	v\stackrel{\triangle}{=}v^+ \approx v^- \]
and note
\[	i_1 \approx 0 \;\;\;\;\mbox{   }\;\;\; i_2 \approx 0	\]
we have these simultaneous equations
\[
\left\{ \begin{array}{ll} 
	(v-v_1)/R_1+(v-v_o)/R_f=0 & (a) \\
	(v-v_2)/R_2+(v-v_3)/R_3=0 & (b) 
	\end{array} \right.
\]
We solve (b) for $v$ to get
\[	v=\frac{R_2}{R_2+R_3} v_3 + \frac{R_3}{R_2+R_3}v_2	\]
and substitute it into (a) to get
\[
v_o=(\frac{R_f}{R_1}+1)[\frac{R_2}{R_2+R_3} v_3+\frac{R_3}{R_2+R_3} v_2]-\frac{R_f}{R_1}v_1
=-k_1v_1+k_2v_2+k_3v_3	
\]
where
\[ \left\{ \begin{array}{l} k_1=\frac{R_f}{R_1}	\\ 
	k_2=(\frac{R_f}{R_1}+1)\frac{R_3}{R_2+R_3} \\
	k_3=(\frac{R_f}{R_1}+1)\frac{R_2}{R_2+R_3} \end{array} \right.
\]
\htmladdimg{../figures/opam6.gif}

\item {\bf Difference amplifier}

\htmladdimg{../figures/opam10.gif}
As the input currents are zero, currents flowing through $R_1$ and $R_2$
are equal for both sides, i.e.,
\[	\frac{V_1-V^-}{R_1}=\frac{V^--V_o}{R_2},\;\;\;\;\;\;
	\frac{V_2-V^+}{R_1}=\frac{V^+-0}{R_2}		\]
Assumeing $V_1=V_2$ and subtracting the second equation from the first, 
we get:
\[	\frac{V_1-V_2}{R_1}=\frac{-V_o}{R_2},\;\;\;\;
	V_o=-\frac{R_2}{R_1}\;(V_1-V_2)=\frac{R_2}{R_1}\;(V_2-V_1)	\]

\item {\bf A/D converter}

Without feedback, the output of an op-amp is $V_o=A(V^--^+)$. As $A$ is
large, $V_o$ is usually saturated, equal to approximately either the 
positive or the negative power supply, depending on whether or not $V^+$ 
is greater than $V^-$. These two possible outputs, positive and negative,
can be treated as ``1'' and ``0'' of the binary system. The figure shows
an A/D converter built by three op-amps to measure voltage $V_i$ from 0 to 
3 volts with resolution 1 V.

\htmladdimg{../figures/opam12.gif}

Due to the voltage dividor, the input voltages to the three opamps are, 
respectively, 2.5V, 1.5V and 0.5V. The output of these opamps are listed
below for each of the voltages:

\begin{tabular}{c|cccc}\hline \\
Voltage (volts) & 0 	& 1	& 2	& 3	\\
Opamps Outputs	& 000	& 001	& 011	& 111	\\
Binary Representation	& 00	& 01	& 10	& 11	\\ \hline
\end{tabular}
A digital logic circuit is then needed to convert the 3-bit output
of the op-amps to the two-bit binary representation.

\item {\bf First order system --- integrator and differentiator}

\htmladdimg{../figures/opam4.gif}

In time domain, as $v^-=v^+=0$ and $i_R+i_C=0$, we have
\[	i_R=-i_C \Longrightarrow \frac{v_i}{R}=-C\frac{d v_C}{dt}
	=-C\frac{d v_o}{dt}	\]
i.e.,
\[	v_o=v_C=-\frac{1}{\tau} \int v_i dt+v_C(0)	\]
where $v_C(0)$ is the initial voltage across $C$ and 
$\tau\stackrel{\Triangle}{=}RC$. In Laplas domain, following the result
similar for the inverter, we have
\[
H(s)=\frac{V_o(s)}{V_i(s)}=-\frac{Z_C(s)}{Z_R(s)}=-\frac{1}{RCs}
	=-\frac{1}{\tau s}	\]
If we swap the resistor and the capacitor, we get
\[	V_C=V_R \Longrightarrow C\frac{d V_i}{dt}=\frac{-V_o}{R}	\]
i.e.,
\[	V_o=-RC \frac{d V_i}{dt}=-\tau \frac{d V_i}{dt}	\]
in Laplas domain,
\[
	H(s)=\frac{V_o(s)}{V_i(s)}=-\frac{Z_R(s)}{Z_C(s)}=-RCs=-\tau s	\]

\item {\bf Higher order systems}

Higher than first order systems can be built with multiple integrators, as shown here for
a third order system:

\htmladdimg{../figures/opam7.gif}

From the diagram, we can get
\[
\left\{ \begin{array}{l}
	y_3=y_2/s \Longrightarrow y_2=y_3s	\\
	y_2=y_1/s \Longrightarrow y_1=y_2s=y_3s^2	\\
	y_1=y_0/s \Longrightarrow y_0=y_1s=y_3s^3	
\end{array} \right.
\]
But we also have
\[	y_0=x-(k_1y_1+k_2y_2+k_3y_3)	\]
i.e., 
\[	x=y_0+k_1y_1+k_2y_2+k_3y_3=(s^3+k_1s^2+k_2s+k_3) y_3	\]
we get the transfer function
\[
	H(s)=\frac{y_3}{x}=\frac{1}{s^3+k_1s^2+k_2s+k_3}
\]


\item {\bf Second order system by 2 integrators}

\htmladdimg{../figures/opam8.gif}

From the diagram, we can get
\[
\left\{ \begin{array}{ll}
	y_2=-c_2y_1/s  \Longrightarrow  y_1=-sy_2/c_2 \\
	y_1=-c_1y_0/s  \Longrightarrow y_0=-sy_1/c_1=s^2y_2/c_1c_2 \\
	y_0=k_0 x+k_1y_1+k_2y_2 
	\end{array} \right.
\]
substituting the first two equations into the last one, we get
\[	\frac{s^2}{c_1c_2} y_2=k_0x+k_1(-\frac{s}{c_2})y_2+k_2y_2 \]
from which we obtain the transfer function as
\[
H(s)=\frac{y_2}{x}=\frac{k_o}{\frac{s^2}{c_1c_2}+\frac{s}{c_2}s-k_2}
	=\frac{k_oc_1c_2}{s^2+k_1c_1s-c_1c_2k_2}
\]
which is a second order system. In particular, if $c_1=c_2=c$, we have
\[
	H(s)=k_0\frac{c^2}{s^2+c k_1s-k_2c^2}
\]
Comparing this with the canonical 2nd order system transfer function
\[
	H(s)=\frac{\omega_n^2}{s^2+2\zeta \omega_n s+\omega_n^2}
\]
we see that we can let $c=\omega_n$ and $k_1=2\zeta$. Moreover, $k_2<0$, 
i.e., the feedback from the output should be negative. $k_0$ is a constant
scaler which can take any value.
	
\htmladdimg{../figures/opam9.gif}
\end{itemize}

\end{document}


	

	












