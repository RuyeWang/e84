\documentstyle[12pt]{article}
\usepackage{html}
% \usepackage{graphics}  
\begin{document}

\subsection*{Twin-T filter}

The twin-T network is composed of two T-networks: 
\begin{itemize}
\item The CRC network is formed by two capacitors $C$ and one
  resistor $R/2$. This T (or Y) network can be converted to a
  $\pi$ (or $\Delta$) network:
  \[
  Z'_1=Z'_2=\frac{R}{2}+\frac{1}{sC}+\frac{R/2sC}{1/sC}=R+\frac{1}{sC}
  \]
  \[
  Z'_3=\frac{1}{sC}+\frac{1}{sC}+\frac{1/s^2C^2}{R/2}
  =\frac{2}{sC}+\frac{2}{Rs^2C^2}=\frac{2(1+sRC)}{Rs^2C^2}
  \]
\item The RCR network is formed by two resistors $R$ and one
  capacitor $2C$. This T (or Y) network can be converted to a
  $\pi$ (or $\Delta$) network:
  \[
  Z''_1=Z''_2=R+\frac{1}{2sC}+\frac{R/2sC}{R}=R+\frac{1}{sC}
  \]
  \[
  Z''_3=R+R+\frac{R^2}{1/2sC}=2R+2R^2sC=2R(1+sRC)
  \]
\end{itemize}
The two parallel $\pi$-networks can be combined to become one
$\pi$-network with three branches $Z_1=Z'_1||Z''_1$, $Z_2=Z'_2||Z''_2$, 
and $Z_3=Z'_3||Z''_3$:
\[
Z_1=Z'_1||Z''_1=Z_2=Z'_2||Z''_2=\frac{1}{2}\left(R+\frac{1}{sC}\right)
\]
\[
  Z_3=Z'_3||Z''_3=\frac{Z'_3 Z''_3}{Z'_3+Z''_3}
  =\frac{4(1+sRC)^2/s^2C^2}{2R(1+sRC)(1/Rs^2C^2+1)}
  =\frac{2R(1+sRC)}{1+(sRC)^2}
\]

The transfer function of this $\pi$-network (a voltage divider) is:
\begin{eqnarray}
H(s)&=&\frac{Z_2}{Z_2+Z_3}=\frac{R+1/sC}{R+1/sC+4R(1+sRC)/(1+(sRC)^2)}
\nonumber \\
&=&\frac{(1+sRC)/sC}{(1+sRC)/sC+4R(1+sRC)/(1+(sRC)^2)}
\nonumber \\
&=&\frac{1/sC}{1/sC+4R/(1+(sRC)^2)}
=\frac{1}{1+4sRC/(1+(sRC)^2)}
\nonumber \\
&=&\frac{1+(sRC)^2}{1+(sRC)^2+4sRC}
=\frac{s^2+(1/RC)^2}{s^2+4s/RC+(1/RC)^2}
\nonumber
\end{eqnarray}
We define
\[
\omega_0=\frac{1}{RC},\;\;\;\;\;\mbox{i.e.,}\;\;\;\;\;
f_0=\frac{1}{2\pi RC}
\]
and express the second order denominator in the canonical form as
\[
H(s)=\frac{V_{out}(s)}{V_{in}(s)}=\frac{s^2+\omega_0^2}{s^2+4\omega_0s+\omega_0^2}
=\frac{s^2+\omega_0^2}{s^2+\omega_0s/Q+\omega_0^2}
=\frac{s^2+\omega_0^2}{s^2+\Delta\omega\;s+\omega_0^2}
\]
where $Q=1/4=0.25$ is the quality factor, and 
$\Delta\omega=\omega_0/Q=4\omega_0$ is the bandwidth of the filter.
We let $s=j\omega$ and get the frequency response function:
\[
H(j\omega)=\frac{\omega_0^2-\omega^2}{-\omega^2+j\omega_0\omega/Q+\omega_0^2}
\]

\begin{itemize}
\item When $\omega=0$, 
  \[
  H(j\omega)\big|_{\omega=0}=H(0)=\frac{\omega_0}{\omega_0}=1
  \]
\item When $\omega\rightarrow \infty$, 
  \[
  H(j\omega)\big|_{\omega\rightarrow\infty}=\lim\limits_{\omega\rightarrow\infty}
  \left( \frac{\omega_0^2-\omega^2}{-\omega^2+j\omega_0\omega/Q+\omega_0^2} \right)
  =\lim\limits_{\omega\rightarrow\infty}\left(\frac{\omega^2}{\omega^2}\right)=1
  \]
\item When $\omega=\omega_0=1/RC$, $H(j\omega_0)=0$
\end{itemize}
We see that this is a band-stop filter. In particular, when
$\omega=\omega_0=1/RC$, $H(j\omega)=0$, and the output is zero, in other
words, this twin-T network is a band-stop filter (notch filter) which 
attenuates frequncy $\omega_0=1/RC$ to zero.

The bandwidth $\Delta\omega=\omega_0/Q=4\omega_0$ is likely to be too large 
for certain applications due to the small quality factor $Q=1/4$. To overcome 
this problem, an active filter circuit containing two opamp followers (with
unity gain $A=1$ can be used to intruce a positive feedback loop as shown below:



Now the common terminal of the twin-T filter is no longer grounded, instead
it is connected a potentiameter, to the voltage dividor composed of $R_4$ and
$R_5$, to form a feedback loop by which a fraction of the output $V_{out}$ is 
fed back:
\[
V_1=\frac{R_5}{R_4+R_5}\;V_{out}
\]
The input and output of the twin-T network are now related by
\[
V_{out}-V_1=H(V_{in}-V_1)
\]
Substituting $V_1=V_{out}\;R_5/(R_4+R_5)$ and rearranging, we get
\begin{eqnarray}
  HV_{in}&=&V_{out}+(H-1) V_1=V_{out}+(H-1)\frac{R_5}{R_4+R_5}V_{out}
  \nonumber \\
  &=&V_{out}\;\left(1+(H-1)\;\frac{R_5}{R_4+R_5}\right)
  \nonumber
\end{eqnarray}
The transfer function of this active filter with feedback can be found to be
\[
H'(s)=\frac{V_{out}}{V_{in}}=\frac{H}{1+(H-1)R_5/(R_4+R_5)}
=\frac{H(R_4+R_5)}{(R_4+R_5)+(H-1)R_5}
\]
Substituting $H(s)=(s^2+\omega_0^2)/(s^2+4\omega_0 s+\omega_0^2)$ and
$H(s)-1=-4\omega_0s/(s^2+4\omega_0 s+\omega_0^2)$, we
get
\begin{eqnarray}
  H'(s)&=&\frac{(S^2+\omega_0^2)(R_4+R_5)}{(R_4+R_5)(s^2+4\omega_0 s+\omega_0^2)- 4\omega_0 s R_5}
  \nonumber \\
  &=&\frac{S^2+\omega_0^2}{(s^2+4\omega_0s+\omega_0^2)-4\omega_0 s\;R_5/(R_4+R_5)}
  \nonumber \\
  &=&\frac{S^2+\omega_0^2}{s^2+\omega_0 4(1-R_5/(R_4+R_5)) s+\omega_0^2}
  \nonumber \\
  &=&\frac{S^2+\omega_0^2}{s^2+\omega_0 4R_4/(R_4+R_5) s+\omega_0^2}
  \nonumber \\
  &=&\frac{S^2+\omega_0^2}{s^2+\omega_0/Q' s+\omega_0^2}
  =\frac{S^2+\omega_0^2}{s^2+\Delta\omega' s+\omega_0^2}
  \nonumber
\end{eqnarray}
where $Q'$ and $\Delta\omega'$ are respectively the quality factor and the bandwidth of the 
active filter with feedback:
\[
Q'=\frac{R_4+R_5}{4R_4},\;\;\;\;\;\;\Delta\omega'=\frac{\omega_0}{Q'}
\]
Now by varying $R_4$ and $R_5$, the quality factor $Q'$ and bandwidth $\Delta\omega'$ can 
be adjusted. In particular, 
\begin{itemize}
\item when $R_5=0$, $V_1=0$ (no feedback), $Q'=1/4=Q$, $\Delta\omega=\omega_0/Q=4\omega_0$; 
\item when $R_4=0$, $V_1=V_{out}$ (one hundred percent feedback), $Q'=\infty$, $\Delta\omega=\omega_0/Q'=0$.
\end{itemize}


If in the CRC T-network the vertical capacitor branch is dropped, i.e.,
$C=0$, while the RCR T-network is still the same, we get a bridged T 
network. Now $Z''_3=2R$, and we have
\[
  Z_3=Z'_3||Z''_3=\frac{Z'_3 Z''_3}{Z'_3+Z''_3}
  =\frac{2R(1+sRC)}{1+sRC+(sRC)^2}
\]
The transfer function of this bridged T network (a voltage divider) is:
\begin{eqnarray}
H(s)&=&\frac{Z_2}{Z_2+Z_3}=\frac{R+1/sC}{R+1/sC+2R(1+sRC)/(1+sRC+(sRC)^2)}
\nonumber \\
&=&\frac{1/sC}{1/sC+2R/(1+sRC+(sRC)^2)}
=\frac{1}{1+2sRC/(1+sRC+(sRC)^2)}
\nonumber \\
&=&\frac{1+sRC+(sRC)^2}{1+3sRC+(sRC)^2}
=\frac{s^2+s/RC+1/(RC)^2}{s^2+3s/RC+1/(RC)^2}
\nonumber
\end{eqnarray}
We let $\omega_0=1/RC$, and express the second order denominator in the
canonical form as
\[
H(s)=\frac{s^2+\omega_0 s+\omega_0^2}{s^2+3\omega_0s+\omega_0^2}
=\frac{s^2+\Delta\omega_n s+\omega_0^2}{s^2+\Delta\omega_ds+\omega_0^2}
=\frac{s^2+\Delta\omega_n s+\omega_0^2}{s^2+\Delta\omega_ds+\omega_0^2}
\]
where 
\[
\Delta\omega_n=\omega_0,\;\;\;\;\;\;\;\Delta\omega_d=3\omega_0
\]
are the bandwidth of the 2nd-order systems of the numerator and the
denominator, respectively.
We let $s=j\omega$ and get the frequency response function:
\[
H(j\omega)=\frac{-\omega^2+\Delta\omega_n j\omega+\omega_0^2}{-\omega^2+\Delta\omega_dj\omega+\omega_0^2}
\]
\begin{itemize}
\item When $\omega=0$, $H(j\omega)=H(0)=1$
\item When $\omega\rightarrow \infty$, $H(j\omega)=1$
\item When $\omega=\omega_0=1/RC$, $H(j\omega_0)=1/3$
\end{itemize}
We see that this is a band-stop filter.


\subsection*{Wien bridge}


The Wien bridge is a particular type of the Wheatstone bridge of which
two of the four arms are composed of a capacitor as well as a resistor
in parallel and series:


For this bridge to balance, the ratios of the left and right branches
should be the same:
\[
\frac{R_3}{R_4}=\frac{R_2+1/j\omega C_2}{R_1||1/j\omega C_1}
=\frac{j\omega R_1C_1+1}{R_1}\;
\frac{j\omega R_2C_2+1}{j\omega C_2}
=\frac{1-\omega^2R_1R_2C_1C_2+j\omega(R_1C_1+R_2C_2)}{j\omega R_1C_2}
\]
For the ratio on the right-hand side to be the same as that on the
left-hand side, it needs to be real, i.e.,
\[
1-\omega^2R_1R_2C_1C_2=0,\;\;\;\;\;\mbox{i.e.,}\;\;\;\;\;
\omega=\frac{1}{\sqrt{R_1R_2C_1C_2}}
\]
and the equation above becomes
\[
\frac{R_3}{R_4}=\frac{R_1C_1+R_2C_2}{R_1C_2}=\frac{C_1}{C_2}+\frac{R_2}{R_1}
\]
In particular, if $R_1=R_2=R$ and $C_1=C_2=C$, we have:
\[
\omega=\frac{1}{\sqrt{R_1R_2C_1C_2}}=\frac{1}{RC}
\]
and
\[
\frac{R_3}{R_4}=\frac{C_1}{C_2}+\frac{R_2}{R_1}=1+1=2,
\;\;\;\;\;\;\mbox{i.e.}\;\;\;\;\;R_4=2R_3
\]

\subsection{Wien bridge oscillator}


An osillator is typically formed by an amplifier with gain $A(s)$ and
a feedback branch of gain $B(s)$. The open-loop gain is 
\[
\frac{V_o}{V_i}=A(j\omega)\;B(j\omega)
\]
If the input $V_i=\cos(\omega_0 t)$ is a sinusoid, and 
\[
|A(j\omega_0)\;B(j\omega_0)|=1
\;\;\;\;\;\;\mbox{and}\;\;\;\;\;\;
\angle A(j\omega_0)B(j\omega_0)=0
\]
then the output $V_o=A(j\omega_0)B(j\omega_0)V_i=\cos(\omega_0 t)$. In 
other words, the closed loop is an oscillator.


The op-amp circuit shown in the figure has both positive and negative
feedback branches. If the voltage to non-inverting input $V_+$ is 
considered as the input, the circuit is a non-inverting amplifier 
with gain:
\[
A=\frac{V_o}{V_+}=1+\frac{R_f}{R_3}=K
\]
On the other hand, the positive feedback gain is
\[
B(j\omega)=\frac{V_+}{V'_o}=\frac{Z_1}{Z_1+Z_2}
=\frac{R_1||1/j\omega C_1}{R_1||1/j\omega C_1+R_2+1/j\omega C_2}
\]



\subsection{Wien osillator}



\end{document}


http://users.ece.gatech.edu/mleach/ece3050/notes/osc/wienbr.pdf

http://www.ece.msstate.edu/~winton/classes/ece3144/labs/Exp10.pdf

http://www.drp.fmph.uniba.sk/ESM/twin.pdf
