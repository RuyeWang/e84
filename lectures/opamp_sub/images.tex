\batchmode

\documentclass{article}
\RequirePackage{ifthen}


\usepackage{amsmath}
\usepackage{amssymb}
\usepackage{graphics}
\usepackage{comment}
\usepackage{html,makeidx}




\usepackage{xcolor}

\usepackage[]{inputenc}



\makeatletter

\makeatletter
\count@=\the\catcode`\_ \catcode`\_=8 
\newenvironment{tex2html_wrap}{}{}%
\catcode`\<=12\catcode`\_=\count@
\newcommand{\providedcommand}[1]{\expandafter\providecommand\csname #1\endcsname}%
\newcommand{\renewedcommand}[1]{\expandafter\providecommand\csname #1\endcsname{}%
  \expandafter\renewcommand\csname #1\endcsname}%
\newcommand{\newedenvironment}[1]{\newenvironment{#1}{}{}\renewenvironment{#1}}%
\let\newedcommand\renewedcommand
\let\renewedenvironment\newedenvironment
\makeatother
\let\mathon=$
\let\mathoff=$
\ifx\AtBeginDocument\undefined \newcommand{\AtBeginDocument}[1]{}\fi
\newbox\sizebox
\setlength{\hoffset}{0pt}\setlength{\voffset}{0pt}
\addtolength{\textheight}{\footskip}\setlength{\footskip}{0pt}
\addtolength{\textheight}{\topmargin}\setlength{\topmargin}{0pt}
\addtolength{\textheight}{\headheight}\setlength{\headheight}{0pt}
\addtolength{\textheight}{\headsep}\setlength{\headsep}{0pt}
\setlength{\textwidth}{349pt}
\newwrite\lthtmlwrite
\makeatletter
\let\realnormalsize=\normalsize
\global\topskip=2sp
\def\preveqno{}\let\real@float=\@float \let\realend@float=\end@float
\def\@float{\let\@savefreelist\@freelist\real@float}
\def\liih@math{\ifmmode$\else\bad@math\fi}
\def\end@float{\realend@float\global\let\@freelist\@savefreelist}
\let\real@dbflt=\@dbflt \let\end@dblfloat=\end@float
\let\@largefloatcheck=\relax
\let\if@boxedmulticols=\iftrue
\def\@dbflt{\let\@savefreelist\@freelist\real@dbflt}
\def\adjustnormalsize{\def\normalsize{\mathsurround=0pt \realnormalsize
 \parindent=0pt\abovedisplayskip=0pt\belowdisplayskip=0pt}%
 \def\phantompar{\csname par\endcsname}\normalsize}%
\def\lthtmltypeout#1{{\let\protect\string \immediate\write\lthtmlwrite{#1}}}%
\usepackage[tightpage,active]{preview}
\newbox\lthtmlPageBox
\newdimen\lthtmlCropMarkHeight
\newdimen\lthtmlCropMarkDepth
\long\def\lthtmlTightVBox#1#2{%
    \setbox\lthtmlPageBox\vbox{\hbox{\catcode`\_=8 #2}}%
    \lthtmlCropMarkHeight=\ht\lthtmlPageBox \advance \lthtmlCropMarkHeight 6pt
    \lthtmlCropMarkDepth=\dp\lthtmlPageBox
    \lthtmltypeout{^^J:#1:lthtmlCropMarkHeight:=\the\lthtmlCropMarkHeight}%
    \lthtmltypeout{^^J:#1:lthtmlCropMarkDepth:=\the\lthtmlCropMarkDepth:1ex:=\the \dimexpr 1ex}%
    \begin{preview}\copy\lthtmlPageBox\end{preview}}}%
\long\def\lthtmlTightFBox#1#2{%
    \adjustnormalsize\setbox\lthtmlPageBox=\vbox\bgroup %
    \let\ifinner=\iffalse \let\)\liih@math %
    {\catcode`\_=8 #2}%
    \@next\next\@currlist{}{\def\next{\voidb@x}}%
    \expandafter\box\next\egroup %
    \lthtmlCropMarkHeight=\ht\lthtmlPageBox \advance \lthtmlCropMarkHeight 6pt
    \lthtmlCropMarkDepth=\dp\lthtmlPageBox
    \lthtmltypeout{^^J:#1:lthtmlCropMarkHeight:=\the\lthtmlCropMarkHeight}%
    \lthtmltypeout{^^J:#1:lthtmlCropMarkDepth:=\the\lthtmlCropMarkDepth:1ex:=\the \dimexpr 1ex}%
    \begin{preview}\copy\lthtmlPageBox\end{preview}}%
    \long\def\lthtmlinlinemathA#1#2\lthtmlindisplaymathZ{\lthtmlTightVBox{#1}{#2}}
    \def\lthtmlinlineA#1#2\lthtmlinlineZ{\lthtmlTightVBox{#1}{#2}}
    \long\def\lthtmldisplayA#1#2\lthtmldisplayZ{\lthtmlTightVBox{#1}{#2}}
    \long\def\lthtmlinlinemathA#1#2\lthtmlindisplaymathZ{\lthtmlTightVBox{#1}{#2}}
    \def\lthtmlinlineA#1#2\lthtmlinlineZ{\lthtmlTightVBox{#1}{#2}}
    \long\def\lthtmldisplayA#1#2\lthtmldisplayZ{\lthtmlTightVBox{#1}{#2}}
    \long\def\lthtmldisplayB#1#2\lthtmldisplayZ{\\edef\preveqno{(\theequation)}%
        \lthtmlTightVBox{#1}{\let\@eqnnum\relax#2}}
    \long\def\lthtmlfigureA#1#2\lthtmlfigureZ{\let\@savefreelist\@freelist
        \lthtmlTightFBox{#1}{#2}\global\let\@freelist\@savefreelist}
    \long\def\lthtmlpictureA#1#2\lthtmlpictureZ{\let\@savefreelist\@freelist
        \lthtmlTightVBox{#1}{#2}\global\let\@freelist\@savefreelist}
\def\lthtmlcheckvsize{\ifdim\ht\sizebox<\vsize 
  \ifdim\wd\sizebox<\hsize\expandafter\hfill\fi \expandafter\vfill
  \else\expandafter\vss\fi}%
\providecommand{\selectlanguage}[1]{}%
\makeatletter \tracingstats = 1 
\providecommand{\Epsilon}{\textrm{E}}
\providecommand{\Beta}{\textrm{B}}
\providecommand{\Kappa}{\textrm{K}}
\providecommand{\Iota}{\textrm{J}}
\providecommand{\Zeta}{\textrm{Z}}
\providecommand{\Nu}{\textrm{N}}
\providecommand{\Alpha}{\textrm{A}}
\providecommand{\Tau}{\textrm{T}}
\providecommand{\Rho}{\textrm{R}}
\providecommand{\omicron}{\textrm{o}}
\providecommand{\Chi}{\textrm{X}}
\providecommand{\Eta}{\textrm{H}}
\providecommand{\Mu}{\textrm{M}}
\providecommand{\Omicron}{\textrm{O}}


\begin{document}
\pagestyle{empty}\thispagestyle{empty}\lthtmltypeout{}%
\lthtmltypeout{latex2htmlLength hsize=\the\hsize}\lthtmltypeout{}%
\lthtmltypeout{latex2htmlLength vsize=\the\vsize}\lthtmltypeout{}%
\lthtmltypeout{latex2htmlLength hoffset=\the\hoffset}\lthtmltypeout{}%
\lthtmltypeout{latex2htmlLength voffset=\the\voffset}\lthtmltypeout{}%
\lthtmltypeout{latex2htmlLength topmargin=\the\topmargin}\lthtmltypeout{}%
\lthtmltypeout{latex2htmlLength topskip=\the\topskip}\lthtmltypeout{}%
\lthtmltypeout{latex2htmlLength headheight=\the\headheight}\lthtmltypeout{}%
\lthtmltypeout{latex2htmlLength headsep=\the\headsep}\lthtmltypeout{}%
\lthtmltypeout{latex2htmlLength parskip=\the\parskip}\lthtmltypeout{}%
\lthtmltypeout{latex2htmlLength oddsidemargin=\the\oddsidemargin}\lthtmltypeout{}%
\makeatletter
\if@twoside\lthtmltypeout{latex2htmlLength evensidemargin=\the\evensidemargin}%
\else\lthtmltypeout{latex2htmlLength evensidemargin=\the\oddsidemargin}\fi%
\lthtmltypeout{}%
\makeatother
\setcounter{page}{1}
\onecolumn

% !!! IMAGES START HERE !!!

{\newpage\clearpage
\lthtmlinlinemathA{tex2html_wrap_inline157}%
$ V\stackrel{\triangle}{=}V^+ \approx V^- $%
\lthtmlindisplaymathZ
\lthtmlcheckvsize\clearpage}

{\newpage\clearpage
\lthtmlinlinemathA{tex2html_wrap_inline159}%
$ V^-$%
\lthtmlindisplaymathZ
\lthtmlcheckvsize\clearpage}

{\newpage\clearpage
\lthtmlinlinemathA{tex2html_wrap_inline161}%
$ V^+$%
\lthtmlindisplaymathZ
\lthtmlcheckvsize\clearpage}

{\newpage\clearpage
\lthtmlinlinemathA{tex2html_wrap_indisplay163}%
$\displaystyle \frac{V_1-V}{R_1}+\frac{V_2-V}{R_2}+\frac{V_{out}-V}{R_f}=0,\;\;\;\;\;\;$%
\lthtmlindisplaymathZ
\lthtmlcheckvsize\clearpage}

{\newpage\clearpage
\lthtmlinlinemathA{tex2html_wrap_indisplay164}%
$\displaystyle \;\;\;\;\;\;\;\;
  \frac{V_3-V}{R_3}+\frac{V_4-V}{R_4}=0 
  $%
\lthtmlindisplaymathZ
\lthtmlcheckvsize\clearpage}

{\newpage\clearpage
\lthtmlinlinemathA{tex2html_wrap_inline166}%
$ V$%
\lthtmlindisplaymathZ
\lthtmlcheckvsize\clearpage}

{\newpage\clearpage
\lthtmlinlinemathA{tex2html_wrap_indisplay168}%
$\displaystyle V=\frac{R_4}{R_3+R_4} V_3 + \frac{R_3}{R_3+R_4} V_4	
  $%
\lthtmlindisplaymathZ
\lthtmlcheckvsize\clearpage}

{\newpage\clearpage
\lthtmlinlinemathA{tex2html_wrap_indisplay170}%
$\displaystyle V_{out}=-\frac{R_f}{R_1}V_1-\frac{R_f}{R_2}V_2
  +\left(\frac{R_f}{R_1}+\frac{R_f}{R_2}+1\right)
  \left(\frac{R_4}{R_3+R_4} V_3+\frac{R_3}{R_3+R_4} v_4\right) 
  $%
\lthtmlindisplaymathZ
\lthtmlcheckvsize\clearpage}

{\newpage\clearpage
\lthtmlinlinemathA{tex2html_wrap_indisplay172}%
$\displaystyle V_{out}=\sum_i k_iV_i
  $%
\lthtmlindisplaymathZ
\lthtmlcheckvsize\clearpage}

{\newpage\clearpage
\lthtmlinlinemathA{tex2html_wrap_inline175}%
$ R_2/R_1=R_4/R_3$%
\lthtmlindisplaymathZ
\lthtmlcheckvsize\clearpage}

{\newpage\clearpage
\lthtmlinlinemathA{tex2html_wrap_indisplay177}%
$\displaystyle \frac{V^--V}{R_1}+\frac{V_0-V}{R_2}=0,\;\;\;\;\;\;
  \frac{V^+-V}{R_3}+\frac{V_0-V}{R_4}=\frac{V}{R_L};
  $%
\lthtmlindisplaymathZ
\lthtmlcheckvsize\clearpage}

{\newpage\clearpage
\lthtmlinlinemathA{tex2html_wrap_inline179}%
$ V_0-V$%
\lthtmlindisplaymathZ
\lthtmlcheckvsize\clearpage}

{\newpage\clearpage
\lthtmlinlinemathA{tex2html_wrap_indisplay181}%
$\displaystyle V_0-V=(V-V^-)\frac{R_2}{R_1}=(V-V^-)\frac{R_4}{R_3}
  $%
\lthtmlindisplaymathZ
\lthtmlcheckvsize\clearpage}

{\newpage\clearpage
\lthtmlinlinemathA{tex2html_wrap_indisplay183}%
$\displaystyle \frac{V^+-V}{R_3}+\frac{V-V^-}{R_3}
  =\frac{V^+-V^-}{R_3}=\frac{V}{R_L}=I_L
  $%
\lthtmlindisplaymathZ
\lthtmlcheckvsize\clearpage}

{\newpage\clearpage
\lthtmlinlinemathA{tex2html_wrap_inline185}%
$ R_L$%
\lthtmlindisplaymathZ
\lthtmlcheckvsize\clearpage}

{\newpage\clearpage
\lthtmlinlinemathA{tex2html_wrap_indisplay190}%
$\displaystyle I_D=I_0 \left( e^{V_D/V_T}-1 \right)
  $%
\lthtmlindisplaymathZ
\lthtmlcheckvsize\clearpage}

{\newpage\clearpage
\lthtmlinlinemathA{tex2html_wrap_inline192}%
$ I_0$%
\lthtmlindisplaymathZ
\lthtmlcheckvsize\clearpage}

{\newpage\clearpage
\lthtmlinlinemathA{tex2html_wrap_inline194}%
$ V_T$%
\lthtmlindisplaymathZ
\lthtmlcheckvsize\clearpage}

{\newpage\clearpage
\lthtmlinlinemathA{tex2html_wrap_indisplay196}%
$\displaystyle \frac{v_{out}}{R}=-I_0(e^{v_{in}/V_T}-1)\approx -I_0\;e^{v_{in}/V_T},
  \;\;\;\;\;\;$%
\lthtmlindisplaymathZ
\lthtmlcheckvsize\clearpage}

{\newpage\clearpage
\lthtmlinlinemathA{tex2html_wrap_indisplay197}%
$\displaystyle \;\;\;\;\;
  \frac{v_{in}}{R}=I_0(e^{-v_{out}/V_T}-1)\approx I_0\,e^{-v_{out}/V_T}
  $%
\lthtmlindisplaymathZ
\lthtmlcheckvsize\clearpage}

{\newpage\clearpage
\lthtmlinlinemathA{tex2html_wrap_indisplay199}%
$\displaystyle v_{out}\approx -R I_0 e^{v_{in}/V_T},\;\;\;\;\;\;$%
\lthtmlindisplaymathZ
\lthtmlcheckvsize\clearpage}

{\newpage\clearpage
\lthtmlinlinemathA{tex2html_wrap_indisplay200}%
$\displaystyle \;\;\;\;\;
  v_{out}\approx -V_T\ln(v_{in}/I_0R)
  $%
\lthtmlindisplaymathZ
\lthtmlcheckvsize\clearpage}


\end{document}
