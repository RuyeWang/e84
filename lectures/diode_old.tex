\documentstyle[12pt]{article}
\usepackage{html}
%\usepackage{graphics}  
\begin{document}

{\bf Chapter 4: Semiconductor Devices}

\section*{Semiconductor materials}

\begin{itemize}
\item {\bf Conductors and Insulators:} 

Good conductors, such as copper (Cu $2+8+18+1$), silver (Ag $2+8+18+18+1=47$),
gold (Au $2+8+18+32+18+1=79$), aluminum (Al $2+8+3=13$) can conduct electricity 
with little resistance because they only have a small number (no more than 
three) of valence electrons (electrons on the out-most layer of the atom) which
are only loosely bound to the atom and can easily become freely movable (free 
electrons) to conduct electricity.

On the other hand, insulators do not conduct electricity as no free electrons
exist in the material.

\item {\bf Semiconductors:} 

The conductivity of those elements with four valence electrons is not as
good as the conductors but still better than the insulators, and they are
given the name semiconductors. The two semiconductors of great importance 
are silicon (Si $2+8+4=14$) and germanium (Ge $2+8+18+4=32$), both of which 
have four valence electrons. Their crystal structure (lattice) has a tetrahedral
pattern with each atom sharing one valence electron with each of its four 
neighbors (covalent bonds). 

If an electron gains enough thermal energy (1.1 eV for Si or 0.7 eV for
Ge), it may break the covalent bond and becomes a free electron of negative
charge, while leaving a vacancy or a hole of positive charge. In an electric
field, a free electron may move to a new location to fill a hole there, i.e.,
both such electrons and holes contribute to electrical conduction. Such 
crystal is called intrinsic semiconductor.

At room temperature, relatively few electrons gain enough energy to become
free electrons, the over all conductivity of such materials is low, thereby
their name semiconductors. 

\htmladdimg{../figures/elements.gif}

\item {\bf Doped Semiconductors:}

The conductivity of semiconductor material can be improved by doping, i.e.,
by adding an impurity element with either three or five valence electrons,
called, respectively, trivalent and pentavalent elements. A semiconductor
is called either intrinsic or extrinsic, depending on whether it contains
any doped impurity.

\begin{itemize}
\item {\bf n-type semiconductor:}

When a small amount of pentavalent donor atoms (e.g., phosphorus (P) and
Arsenic (As)) is added, a silicon atom in the lattice may be replaced by
a donor atom with four of its valence electrons forming the covalent bounds 
and one extra free electron. This is an {\bf n-type} semiconductor whose 
conductivity is much improved compared to the intrinsic semiconductors, due 
to the extra free electrons in the lattice, which are called {\bf predominant
or majority current carriers}. There also exist some tiny number of holes 
called {\bf minority carriers}.

\item {\bf p-type semiconductor:}

When a small amount of trivalent acceptor atoms (e.g., boron (B) and aluminum
(Al)) is added, a silicon atom in the lattice may be replaced by a acceptor
atom with only three valence electrons forming three covalent bounds and a 
hole in the lattice. This is a {\bf p-type} semiconductor whose conductivity 
is also much improved compared to the intrinsic semiconductors, due to the 
holes in the lattice, which are called {\bf predominant or majority current 
carriers}. There also exist some tiny number of free electrons called 
{\bf minority carriers}.
\end{itemize}

\htmladdimg{../figures/lattice.gif}

\item {\bf PN Junction}

\htmladdimg{../figures/pn_junction0.gif}

When p-type and n-type materials in contact with each other, a p-n junction is
formed due to two effects:
\begin{itemize}
\item {\bf Diffusion:}

Although both sides are electrically neutral, but they have different
concentration of electrons (the n-type) and holes (the p-type), and the 
free electrons in the n-type material begin to diffuse across the p-n junction
between the two materials, due to their thermal motion, and to fill some of the
holes in the p-type material. Equivalently, the holes are also drifting from
the p-type side to the n-type side.

\item {\bf Electric Field}

If no other forces were involved, the diffusion would carry out 
continuously until the free electrons and holes are uniformly distributed
across both materials. However, as the result of the diffusion process,
electrical field is gradually established, negative on the side of p-type 
material due to the extra electrons, positive on the side of n-type 
material due to the loss of free electrons. This electrical field prevents
further diffusion as the electrons on the n-type side are expelled from
the p-type side by the electrical field.

\end{itemize}

The effects of both diffusion and electric field eventually lead to an 
equilibrium where the two effects balance each other so that there are
no more charge carriers (free electrons or holes) crossing the p-n junction.
This region around the p-n junction, called the {\bf depletion region} as 
there no longer exist freely movable charge carriers, becomes a barrier 
between the two ends of the material that prevent current to flow through.

\htmladdimg{../figures/pn_junction1.gif}

{\bf Solar Cell}

A solar cell converts light energy to electrical energy and is a current source.
When a photon of light hits a piece of silicon, it either goes straight through
material if its energy is lower than the band-gap energy of the silicon semiconductor,
or is absorbed by the silicon if its energy is greater than the band-gap energy.
In the latter case, an electron-hole pair is produced, and the electron and hole
are separated by the internal electric filed near the PN-junction formed by the 
n-type and p-type materials, and a current is formed through the external circuit.


\htmladdimg{../figures/solarcell.gif}

\end{itemize}

\section*{Diodes}

Due to the fact that there exist few freely movable charge carriers in the
depletion region around the p-n junction, the conductivity is very poor.
However, if certain voltage is applied to the two ends of the material, 
the conductivity may change, depending one the polarity of the applied
voltage:

\htmladdimg{../figures/diode0.gif}

\begin{itemize}
\item {\bf Reverse bias} (negative to p-type, positive to n-type)

  The negative voltage applied to the p-type will repel electrons in n-type
  and attract holes in p-type so that both carriers are moving away from 
  the p-n junction. As the depletion region becomes thicker than before, 
  there is no current through the p-n junction and the conductivity is zero.

\item {\bf Forward bias} (positive to p-type, negative to n-type)

  The positive voltage applied to the p-type will attract electrons in n-type
  and repel holes in p-type so that both carriers are moving towards the p-n
  junction. As the depletion region becomes thinner, the conductivity is 
  improved and there is current through the p-n junction. The conductivity
  increases as the voltage becomes higher.
  
\end{itemize}
The voltage-current behavior of a p-n junction is described by
\[ I_D=I_0 ( e^{V_D/\eta V_T}-1 ), \;\;\;\;\mbox{or}\;\;\;\;
	V_D=\eta V_T\;ln (\frac{I_D}{I_0}+1)=\eta V_T\;ln (\frac{I_D+I_0}{I_0})	\]

where 
\begin{itemize}
\item $I_0$ is the {\em reverse saturation current}, a tiny current that 
  flows in the reverse direction when $V_D \ll 0$, due to the minority 
  carriers. As this current is limited by the minority carriers available,
  it is called saturation current. $I_0$ is about $10^{-10} \sim 10^{-12}$ 
  A for Si and $10^{-4}$ A for Ge.
\item $ V_T=kT/e $ is the thermal voltage, where 
  $k=1.38\times 10^{-23}$ Joules/Kelvin is Boltzmann's constant, 
  $e=1.602\times 10^{-19}$ coulomb is the charge of an electron, and
  $T$ is the temperature in degree K. For room temperature $T=300K$, 
  $V_T=kT/e=26\; mV$.
\item $\eta$ is the ideality factor which varies between 1 and 2, depending
  on the fabrication process and semiconductor material. In many cases $\eta$
  can be assumed to be approximately equal to 1.
\end{itemize}
In particular, when $V_D=0$, $I_D=0$, when $V_D\ll 0$, $I_D=-I_0$, when
$V_D\gg 0$, $I_D=I_0 e^{V_D/V_T}$.

\htmladdimg{../figures/diode1.gif}

The resistance of an electrical device is defined as $r=\Delta V/\Delta I$.
For a diode, as $V_D(I_D)$ is not a linear function, the resistance 
$R_0=dV_D/dI_D$ is not a constant, but a function of $I_D$:
\[
 R_0=\frac{d}{dI_D}V_D=\frac{d}{dI_D} [\eta V_T\;ln (\frac{I_D+I_0}{I_0})]
=\eta V_T \frac{I_0}{I_D+I_0}\frac{1}{I_0}=\eta \; \frac{V_T}{I_D+I_0}
\approx \eta \; \frac{V_T}{I_D}	\]
The last approximation is due to the fact that $I_D \gg I_0$, i.e., 
$I_D+I_0\approx I_D$. We assume $V_T=26\;mV$, $\eta=1$, then if $I_D=1\;mA$, 
$R_0=26\;\Omega$, but if $I_D=2\;mA$, $R_0=13\;\Omega$. In other words, the 
resistance $R_0$ of a diode is not a constant, but a function of the current 
$I_D$, i.e.,  a diode is not a linear element.

{\bf Models of diodes:}

\htmladdimg{../figures/diode3.gif}

\begin{itemize}
\item Ideal model: if $V_D<0$, then $I_D=0$, else $I_D=V/R$
\item Ideal model with a voltage threshold $V_0=0.7V$:
	if $V_D<V_0$, then $I_D=0$, else $I_D=(V-V_0)/R$
\item The model above in series with a resistance $R_0=20\Omega$:
	if $V_D<V_0$, then $I_D=0$, else $I_D=(V-V_0)/(R+R_0)$
\item The model above in parallel with a current source that simulates
  the reverse saturation current.
\end{itemize}

\htmladdimg{../figures/diodemodel.gif}
\htmladdimg{../figures/diodemodels.gif}

\begin{tabular}{c||c|c|c}\\ \hline
 $I_0$	& 1 mA & 10 mA & 100 mA	\\ \hline
$V_D$ for Si ($I_0=10^{-10}$, $\eta=1.4$) & 0.58 V & 0.67 V & 0.75 V \\
$V_D$ for Ge ($I_0=10^{-4}$, $\eta=1.0$) & 0.06 V & 0.12 V & 0.18 V \\
\end{tabular}

In general, when the forward voltage applied to a diode exceeds 0.6 to 
0.7V for silicon (or 0.1 to 0.2 V for germanium) material, the diode is 
assumed to be conducting with very little resistance.

{\bf Photodiode}

Photodiodes are similar to regular semiconductor diodes except that they 
may be either exposed or packaged with a window or optical fiber connection
to allow light to reach the sensitive part of the device. 

When a photon of sufficient energy strikes the diode, it excites an electron,
thereby creating a free electron paired with a hole. If the absorption occurs 
in the depletion region of the PN junction, these carriers go through the 
junction by the built-in field of the depletion region, i.e., the hole moves
toward the P-side (anode), and the electron toward the N-side (cathode), and
a photocurrent is produced.

\begin{itemize}
\item {\bf Photovoltaic mode}
  
  The PN-junction is zero-biased, the P side is negative and and N side positive 
  due to diffusion of the electrons and holes. The photocurrent going from N to P
  will build up an external voltage, positive on the P side, negative on the N
  side.
\item {\bf Photoconductive mode}

  The PN-junction is reverse-biased and the depletion region is widened. The 
  photocurrent is more linearly related to the illuminance.

\end{itemize} 

\htmladdimg{../figures/LED.gif}

{\bf Light-Emitting Diode (LED)}

When a light-emitting diode is forward biased (turned on), electrons on the N side
can pass through the PN-junction to recombine with holes on the P side. These 
electrons fall to a lower level and release energy in the form of photons. This 
effect is called {\em electroluminescence} and the wavelength (color) of the light,
corresponding to the energy of the photon, is determined by the energy gap of the 
semiconductor.


In silicon or germanium diodes, the electrons and holes recombine by a non-radiative 
transition which produces no optical emission, because these are indirect band gap 
materials. The materials used for the LED, such as gallium arsenide (GaAs) have a 
direct band gap with energy corresponding to near-infrared, visible or even 
near-ultraviolet light.

\end{document}
