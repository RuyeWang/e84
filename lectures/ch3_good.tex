\documentstyle[12pt]{article}
\usepackage{html}
\textwidth 6.0in
\topmargin -0.5in
\oddsidemargin -0in
\evensidemargin -0.5in
% \usepackage{graphics}  
\begin{document}

\section*{Chapter 3: AC Circuit Analysis}

\subsection*{Sinusoidal Variables}

Sinusoidal variables are of special importance in electrical and 
electronic systems, not only because they occur frequently in such 
systems, but also because any periodical signal can be represented 
as a linear combination of a set of sinusoidal signals of different 
frequencies, amplitudes, and phase angles (Fourier transform theory).

A sinusoidal variable (voltage or current) can be written as
\[	x(t)=\cos(\omega t + \phi),\;\;\;\mbox{or}\;\;\;
	x(t)=A\;\sin(\omega t+\phi+\pi/2) \]
The three parameters $A$, $\omega$ and $\phi$ represent three 
important elements:
\begin{itemize}
\item {\bf $A$: amplitude or peak value}
\item {\bf $\phi$: phase angle $0 \le \phi \le 2\pi$ in radians or
	$0^\circ \le \phi \le 360^\circ$ in degrees.}
\item {\bf $\omega$: angular frequency in radians per second.}
\end{itemize}
Frequency can also be measured by cycles per second. i.e., $f=1/T$ 
where $T$ is cycle time or period (in seconds). Since one cycle is
$2\pi$ radians, we have 
\[ \omega=2\pi f=2\pi/T,\;\;\;\;f=1/T=\omega/2\pi,\;\;\;T=1/f=2\pi/\omega \]

{\bf Example:} A sinusoidal current with a frequency of 60 Hz reaches
a positive maximum of 20A at $t=2 \; ms$. Give the expression of this
current as a function of time $i(t)$.

We have $A=20$, $\omega=2\pi f=6.28\times 60=377\;rad/s$. As cosine
function $\cos(\alpha)$ reaches maximum when $\alpha=0$ (or $\pm 2\pi,
\pm 4\pi, \cdots$), the phase angle $\phi$ should satisfy 
$\omega t+\phi=0$ where $\omega=377$ and $t=2\times 10^{-3}$, i.e.,
\[	\phi=-\omega t=-377 \times 2 \times 10^{-3}=-0.754\; rad 
	=-0.754\times 360^\circ /2\pi=-43.2^\circ	\]
The current is
\[	i(t)=20\;\cos(377 t-43.2^\circ)	\]

{\bf Average Value} 

The average of a varying current $i(t)$ is the steady value of current 
$I_{av}$ that in period $T$ would transfer the same charge $Q$:
\[	I_{av}T=Q=\int_0^T i(t) dt,\;\;\;\;\mbox{i.e.}\;\;\;\;
	I_{av}=\frac{1}{T}\int_0^T i(t) dt	\]
Similarly, the average voltage is defined as:
\[	V_{av}=\frac{1}{T}\int_0^T v(t) dt	\]
In general, when $i(t)$ and $v(t)$ are periodic, the time period $T$ is
one complete cycle. For a sinusoidal variable $i(t)=I_m \cos(\omega t)$, 
the average over the complete cycle is always zero (the charge transferred 
during the first half is the opposite to that transferred in the second).
We can consider the half-cycle average:
\begin{eqnarray} 
I_{av}&=&\frac{1}{T/2}\int_{-T/4}^{T/4} i(t) dt
	=\frac{2}{T}\int_{-T/4}^{T/4} I_m\;\cos(2\pi t/T)dt	
	=\frac{2I_m}{T}\frac{T}{2\pi} \sin(2\pi t/T)|_{-T/4}^{T/4}
	\nonumber \\
	&=& \frac{I_m}{\pi}[\sin(\pi/2)-\sin(-\pi/2)]
	=2I_m/\pi=0.637\;I_m
	\nonumber
\end{eqnarray}

{\bf Effective Value}

The effective value of a varying current $i(t)$ is the steady value of 
current $I_{eff}$ that in period $T$ would transfer the same amount of 
energy $W$:
\[	I^2_{eff}RT=W=\int_0^T i^2(t)R dt,\;\;\;\;\mbox{i.e.}\;\;\;\;
	I_{eff}=\sqrt{\frac{1}{T}\int_0^T i^2(t) dt}	\]
As $I_{eff}$ is the ``square root of the mean squared value'', it is
also called the {\em root-mean-square (rms)} current $I_{rms}$.
Similarly, the effective voltage is defined as:
\[	V_{eff}=\sqrt{\frac{1}{T}\int_0^T v^2(t) dt}	\]
For a sinusoidal variable $i(t)=I_m \cos(\omega t)$, we have
\[
I^2_{eff} = \frac{1}{T}\int_0^T i^2(t) dt
	= \frac{I^2_m}{T}\int_0^T \cos(2\pi t/T) dt
	= \frac{I^2_m}{2T}\int_0^T (1+\cos(4\pi t/T) dt
	=\frac{I^2_m}{2}
\]
i.e.,
\[	I_{eff}=\frac{I_m}{\sqrt{2}}=0.707 I_m	\]

\subsection*{Complex and Vector Representations}

{\bf Phasor Representation}

To simplify the computation of sinusoidal variables, they are often
represented as the real or imaginary part of the corresponding complex 
variables (vectors in complex plane) which can be more conveniently 
operated.

A sinusoidal variable is either the real or imaginary part of a complex
variable, e.g.,:
\[	A\;\cos(\omega t+\phi)=Re[A\;e^{j\omega t+\phi}],\;\;\;\;
	A\;\sin(\omega t+\phi)=Im[A\;e^{j\omega t+\phi}]	\]
A sinusoidal voltage can be represented as:
\[	v(t)=V_m\;cos(\omega t+\phi)=Re[V e^{j\phi} \sqrt{2}e^{j\omega t}] 
	=Re[\dot{V} \sqrt{2} e^{j\omega t}] \]
where $V_m$ is the peak magnitude of the voltage and $V=V_m/\sqrt{2}$ is the
effective value (RMS), and the complex component of the voltage is defined 
as the {\bf phasor} representation of the voltage:
\[	\dot{V}\stackrel{\triangle}{=}V e^{j\phi} =V \angle \phi	\]
Phasor represents the magnitude, the rms (effective) value $V$, and the 
phase $\phi$ of the voltate. The frequency $\omega=2\pi f$ is not explicitly 
represented by the phasor, as all currents and voltages in the circuit 
considered here have the same frequency.

{\bf Example} A 120V 60Hz AC voltage 
\[	v(t)=120 \sqrt{2}\; \cos(2\pi f t+60^\circ)
	=120\sqrt{2}\; \cos(6.28\times 60 t+60^\circ)
	=170\;\cos(377\;t+60^\circ)	\]
is expressed as $\dot{V}=V \angle \phi= 120 \angle 60^\circ$ with rms
value $V=120$ and $\phi=60^\circ$, and the implied frequency $f=60Hz$. 
All sinusoidal signals, currents as well as voltages can be represented
by phasors.

{\bf Complex Arithmetic} 

A complex number can be represented in four different formats:
\begin{itemize}
\item {\bf Algebraic expression:} $z=x+jy$ where $x$ and $y$ are the
	real and imaginary part of complex variable $z$.
\item {\bf Complex exponential:} $z=|z|e^{j\angle z}=|z|\;e^{j\phi}$ where 
\[ \left\{ \begin{array}{ll} |z|=\sqrt{x^2+y^2} & \mbox{amplitude}\\
			\phi=\angle z=tan^{-1} (y/x) & \mbox{phase angle}
	\end{array} \right. \]
\item {\bf Trigonometric:} $z=|z|\;e^{j\phi}=|z|(\cos\phi+j\;\sin\phi)=x+jy$
	due to Euler identity, i.e.,
\[ \left\{ \begin{array}{ll} x=|z|\;\cos\phi & \mbox{real part}\\
			      y=|z|\;\sin\phi & \mbox{imaginary part}
	\end{array} \right. \]
\item {\bf Polar coordinates:} $z=|z|\angle z=|z|\angle \phi=|z| e^{j\phi}$
\end{itemize}

The arithmetic operations of two complex numbers $z=x+jy=|z|\;e^{j\phi}$ 
and $w=u+jv=|w|\;e^{j\psi}$ are listed below:
\begin{itemize}
	\item {\bf Add/Subtract:} 
\[	z+w=(x+u)+j(y+v),\;\;\;\;z-w=(x-u)+j(y-v)	\]
	\item {\bf Multiply:}
\[	z\;w=(x+jy)(u+jv)=(xu-yv)+j(xv+yu)=|z|\;|w|\;e^{j(\phi+\psi)}	\]
	\item {\bf Divide:}
\[	\frac{z}{w}=\frac{x+jy}{u+jv}=\frac{(x+jy)(u-jv)}{(u+jv)(u-jv)}
	=\frac{(xu+yv)+j(yu-xv)}{u^2+v^2}
	=\frac{|z|}{|w|}\;e^{j(\phi-\psi)}	\]
	\item {\bf Rotation:}

A complex number (vector) $z=|z|\;e^{j\phi}$ multiplied by $e^{j\alpha}$ 
will become $z\;e^{j\alpha}=|z|\;e^{j(\phi+\alpha)}$, i.e., rotated by
an angle of $\alpha$. In particular, As $e^{j\pi/2}=j$ and $e^{-j\pi/2}=-j$,
they can be considered as $90^\circ$ rotation factors. Any complex number
multiplied by $j$ or $-j$ will be rotated counter clockwise or clockwise 
by 90 degrees.

	\item {\bf Complex Conjugate:}

The complex conjugate of $z=x+jy=|z|e^{j\angle z}$ is 
$z^*=x-jy=|z|e^{-j\angle z} $. In general, $z^*$ can be obtained by
negating every $j$ in the expression of $z$ (replacing $j$ by $-j$). 
The magnitude of a complex number $z=x+jy$ can be found by:
\[	\sqrt{zz^*}=\sqrt{(x+jy)(x-jy)}=\sqrt{x^2+y^2}	\]

	\item {\bf Reciprocal:}
\[	|z^{-1}|=\frac{1}{|z|}=\frac{1}{\sqrt{x^2+y^2}},\;\;\;\;
	\angle(z^{-1})=\angle(\frac{1}{z})=0-\angle{z}=-\angle z	\]

\end{itemize}

\subsection*{Generalized Kirchhoff's Laws and Ohm's Law}

If we assume all voltages and currents in a circuit are sinusoidals of
the same frequency $\omega$, they can be represented as complex phasors 
(vectors), e.g.,
\[
v(t)=V_m cos(\omega t+\phi)=Re[V e^{j\phi}\;\sqrt{2}\;e^{j\omega t}]
=Re[\dot{V}\;\sqrt{2}\;e^{j\omega t}]	\]
and the KVL $\sum_k v_k(t)=0$ can be expressed as $\sum_k \dot{V}_k=0$,
and the same is true for KCL. In general, the Kirchhoff's laws can now 
be stated as
\begin{itemize}
\item {\bf Current Law (KCL):} The vector sum of the currents into a node
	at any instant is zero	$\sum_k \dot{I}_k=0$.
\item {\bf Voltage Law (KVL):} The vector sum of the voltages around a
	loop at any instant is zero $\sum_k \dot{V}_k=0$.
\end{itemize}

{\bf Example 1:} Consider three sinusoidal voltage sources 
$v_1(t)=6\sqrt{2}\;\sin(\omega t)$, $v_2(t)=12\sqrt{2}\;\sin(\omega t+\pi/2)$
and $v_3(t)=4\sqrt{2}\;\sin(\omega t-\pi/2)$ in series. According to KVL,
the overall voltage will be the algebra sum of the three:
\[ 	v(t)=v_1(t)+v_2(t)+v_3(t)
	=6\sqrt{2}\;\sin(\omega t)+12\sqrt{2}\;\sin(\omega t+\pi/2)
	+4\sqrt{2}\;\sin(\omega t-\pi/2)	\]

\htmladdimg{../figures/sinusoidalsum.gif}

While the addition of these sinusoidal functions is not easy to carry out, 
it is quite straight forward to find the vector sum when the voltages are
represented as phasors:
\begin{eqnarray}
\dot{V}&=&\dot{V}_1+\dot{V}_2+\dot{V}_3=6\angle 0^\circ+12\angle 90^\circ+
	4\angle -90^\circ	
	\nonumber \\
	&=&6+j12-j4=6+j8=10 \angle tan^{-1}(8/6)
	=10\angle 53.1^\circ	\nonumber \end{eqnarray}
The resulting voltage is therefore $v(t)=10\sqrt{2}\;sin(\omega t+53.1^\circ)$

\htmladdimg{../figures/vectorsum.gif}

\htmladdimg{../figures/sinusoidalsum1.gif}

{\bf Impedance}

The complex impedance of an element can be obtained according the physics
of the element. We represent sinusoidal voltages and currents as complex
variables:
\begin{itemize}
\item voltage $u(t)=V_m cos(\omega t)=Re[V_m e^{j\omega t}]=Re[{\bf V}(t)]$
	is represented by ${\bf V}(t)=V_m e^{j\omega t}$,
\item current $i(t)=I_m cos(\omega t)=Re[I_m e^{j\omega t}]=Re[{\bf I}(t)]$
	is represented by ${\bf I}(t)=I_m e^{j\omega t}$.
\end{itemize}
The {\bf impedance} of the element is defined as the ratio between complex 
voltage across an element and the complex current through the element.
The complex current and complex voltage are related differently for
different components:
\begin{itemize}
\item {\bf Resistor:} 
\[	{\bf I}(t)=\frac{{\bf V}(t)}{R},\;\;\;\;\;Z_R=\frac{{\bf V}(t)}{{\bf I}(t)}=R	\]
\item {\bf Capacitor:} 
\[	{\bf I}(t)=C \frac{d}{dt}{\bf V}(t)=C\frac{d}{dt}[V_m e^{j\omega t}]
	=j\omega C V_m e^{j\omega t}=j\omega C {\bf V}(t),\;\;\;\;
	Z_C=\frac{{\bf V}(t)}{{\bf I}(t)}=\frac{1}{j\omega C}=\frac{-j}{\omega C}
\]
\item {\bf Inductor:}
\[	{\bf V}(t)=L \frac{d}{dt}{\bf I}(t)=L\frac{d}{dt}[I_m e^{j\omega t}]
	=j\omega L I_m e^{j\omega t}=j\omega L {\bf I}(t),\;\;\;
	Z_L=\frac{{\bf V}(t)}{{\bf I}(t)}=j\omega L	\]
\end{itemize}
When $\omega=0$, $Z_C\rightarrow \infty$ and the capacitor has zero 
conductivity due to the insulation between its two plates (open circuit),
and $Z_L=0$ as there is no flux change in the inductor and the resistance 
of the coil is ideally zero.

When $\omega\rightarrow \infty$, $Z_C\rightarrow 0$ and the capacitor 
becomes highly conductive, and $Z_L\rightarrow \infty$ as the self-induced
voltage in the coil always acts against any change in the input (Lenz's Law).

{\bf Generalized Ohm's law}

In general, all sinusoidal voltages and currents in a circuit can be 
represented as a complex variable 
\begin{eqnarray}
&&v(t)=V_m cos(\omega t+\phi)=Re[{\bf V}]=Re[V_m e^{j\omega t+\phi}]
	=Re[\dot{V} \sqrt{2} e^{j\omega t}] 
	\nonumber \\
&&i(t)=I_m cos(\omega t+\psi)=Re[{\bf I}]=Re[I_m e^{j\omega t+\psi}]
	=Re[\dot{I} \sqrt{2} e^{j\omega t}] 
	\nonumber
\end{eqnarray}
and the Ohm's law can be further generalized in terms of phasors:
\[	Z=\frac{{\bf V}}{{\bf I}}=\frac{V_m e^{j\omega t+\phi}}
{I_m e^{j\omega t+\psi}}=\frac{\dot{V}e^{j\omega t}}{\dot{I}e^{j\omega t}}
=\frac{\dot{V}}{\dot{I}},\;\;\;\;\;\;\;
	\dot{I}=\frac{\dot{V}}{Z},\;\;\;\;\;Z\dot{I}=\dot{V}	\]
Here $Z$ is the complex {\bf impedance}:
\[	Z=R+jX=|Z|e^{j\angle Z}=|Z|\angle Z	\]
The magnitude and phase angle of $Z$ are:
\[	|Z|=\sqrt{R^2+X^2},\;\;\;\;\angle Z=tan^{-1}\frac{X}{R}	\]
\begin{itemize}
\item The real part of impedance $Re[Z]=R$ is called {\bf resistance}. 
\item The imaginary part of impedance  $Im[Z]=X$ is called {\bf reactance}. 
\end{itemize}
Impedance, resistance and reactance are all measured by the same unit 
Ohm ($\Omega$). 

The reciprocal of the impedance $Z$ is called {\bf admittance}:
\[	Y=\frac{1}{Z}=\frac{1}{R+jX}=\frac{R-jX}{R^2+X^2}=G+jB	\]
which contains real and imaginary parts:
\begin{itemize}
\item The real part of admittance is called {\bf conductance}:
\[	G=Re[Y]=\frac{R}{R^2+X^2}	\]
\item The imaginary part of admittance is called {\bf susceptance}:
\[	B=Im[Y]=\frac{-X}{R^2+X^2}	\]
\end{itemize}
Unlike $R$ and $X$, $G$ and $B$ do not correspond to any particular
circuit elements. The magnitude and phase of complex admittance are
\[	|Y|=\sqrt{G^2+B^2}=\frac{1}{\sqrt{R^2+X^2}},\;\;\;\;\;
	\angle Y=\tan^{-1} \frac{B}{G}=\tan^{-1} \frac{-X}{R}
	=-\angle Z	\]
Admittance, conductance and susceptance are all measured by the same 
unit Siemen ($S$). 

Impedance $Z$ and admittance $Y=1/Z$ are both complex variables. The 
real parts $Re[Z]=R$ and $Re[Y]=G$ are always positive, while the
imaginary parts $Im[Z]=X$ and $Im[Y]=B$ can be either positive or
negative. Therefore $Z$ and $Y$ can only be in the first or the 
fourth quadrants of the complex plane.

In particular, the admittances of the three types of elements R, L
and C are
\[ Y_R=\frac{1}{R},\;\;\;\;
Y_L=\frac{1}{Z_L}=\frac{1}{j\omega L}=\frac{-j}{\omega L},\;\;\;\;
Y_C=\frac{1}{Z_C}=\frac{1}{1/j\omega C}=j\omega C	\]

{\bf Circuit analysis using admittance}

The Ohm's law can also be expressed in terms of admittance as:
\[	\dot{I}=\dot{V}/Z=\dot{V}Y	\]
Sometimes it is more convenient in circuit analysis to use admittance
instead of impedance.
\begin{itemize}
\item Components Parallel:
\[ Z_{total}=\frac{Z_1\;Z_2}{Z_1+Z_2},\;\;\;\;Y_{total}=Y_1+Y_2 \]
\item Components in series:
\[ Y_{total}=\frac{Y_1\;Y_2}{Y_1+Y_2},\;\;\;\;Z_{total}=Z_1+Z_2 \]
\end{itemize}

\subsection*{Sinusoidal AC Circuit Analysis}

{\bf Example 1:} Given a resistor $R$, and an inductor $L$ connected in 
series to an AC voltage source $v(t)=\cos(\omega t)=Re[e^{j\omega t}]$, 
find the current $i(t)$.

{\bf Method 1:} The DE describing the circuit is:
\[	v_L(t)+v_R(t)=L\frac{d}{dt}i(t)+Ri=v(t)=\cos(\omega t)	\]
We want to find its particular or steady state solution. (The homogeneous
or transient solution will be discussed later.) Due to Eular's formula,
the input can be written as:
\[	\cos(\omega t)=(e^{j\omega t}+e^{-j\omega t})/2	\]
Since the DE describes a linear system, the superposition principle applies
$O(ax+by)=aO(x)+bO(y)$. If we find the solutions for
\[	L\frac{d}{dt}i_1(t)+Ri_1(t)=v_1(t)=e^{j\omega t},\;\;\;\;\;
	\mbox{and}\;\;\;\;\;
	L\frac{d}{dt}i_2(t)+Ri_2(t)=v_2(t)=e^{-j\omega t}
\]
we can then find the solution for the original input as $i(t)=[i_1(t)+i_2(t)]/2$.
To solve the DE with input $v_1(t)=e^{j\omega t}$, we assume
\[ 
i(t)=A\;e^{j\omega t},\;\;\;\;\;\frac{d}{dt}i(t)=j\omega A\;e^{j\omega t} 
\]
and substitute them back into the DE to get
\[	j\omega L A e^{j\omega t}+RA e^{j\omega t}
	=(R+j\omega L)A\; e^{j\omega t}=e^{j\omega t} \]
i.e.
\[	A=\frac{1}{R+j\omega L}=|A|e^{-j\phi},\;\;\;\;\mbox{where}
	\;\;\;\;\;\;|A|=\frac{1}{\sqrt{R^2+\omega^2 L^2}},\;\;\;\;\;
	\phi=\angle A=tan^{-1}(\frac{\omega L}{R})	\]
and the solution is:
\[ i_1(t)=A\;e^{j\omega t}=\frac{1}{\sqrt{R^2+\omega^2 L^2}} 
	e^{j(\omega t-\phi)} \]
Similarly, when the input is $v_2(t)=e^{-j\omega t}$ we assume
\[ 
i(t)=A\;e^{-j\omega t},\;\;\;\;\;\frac{d}{dt}i(t)=-j\omega A\;e^{-j\omega t} 
\]
and substitute them back into the DE to get
\[	-j\omega L A e^{-j\omega t}+RA e^{-j\omega t}
	=(R-j\omega L)A\; e^{-j\omega t}=e^{-j\omega t} \]
i.e.
\[	A=\frac{1}{R-j\omega L}=|A|e^{j\phi}	\]
with $|A|$ and $\phi$ the same as defined above. The solution is:
\[ i_2(t)=A\;e^{-j\omega t}=\frac{1}{\sqrt{R^2+\omega^2 L^2}} 
	e^{-j(\omega t-\phi)} \]
The solution (the response or output) for the DE with input 
$v(t)=[v_1(t)+v_2(t)]/2$ is 
\[ 
i(t)=\frac{v_1(t)+v_2(t)}{2}=A\;\frac{e^{j(\omega t-\phi)}+e^{-j(\omega t-\phi)}}{2}
=\frac{1}{\sqrt{R^2+\omega^2 L^2}}\cos(\omega t-\phi)	\]

{\bf Method 2:}
The overall complex impedance of the two elements in series is:
\[	Z=Z_R+Z_L=R+j\omega L=\sqrt{R^2+\omega^2L^2}e^{j\phi}	\]
Representing voltage $v(t)$ also as a complex variable $e^{j\omega t}$, 
we get the current by generalized Ohm's law:
\[	\frac{e^{j\omega t}}{Z}
	=\frac{e^{j\omega t}}{\sqrt{R^2+\omega^2L^2}\;e^{j\phi}}
	=\frac{1}{\sqrt{R^2+\omega^2L^2}} e^{j(\omega t-\phi)}
\]
where $\phi=tan^{-1}(\frac{\omega L}{R})$
and the real current is
\[	i(t)=Re[\frac{1}{\sqrt{R^2+\omega^2 L^2}}e^{j(\omega t-\phi)}]
	=\frac{1}{\sqrt{R^2+\omega^2 L^2}}\cos(\omega t-\phi)	\]
The second method, much easier than the first one, is actually a short 
cut representation of the first DE method. This is the justification of
the complex variable (vector) method for analysis of AC circuits. However,
note that if the transient solution can only be obtained by the DE method,
as the phasor method can only find the steady state solution.

\htmladdimg{../figures/phasorfigure.gif}

{\bf Example 2:} A current $i(t)=17\;cos(1000t+90^\circ)=
12\sqrt{2}cos(1000t+90^\circ)$ flows
through a circuit composed of a resistor $R=18\Omega$, and capacitor
$C=83.3\mu F=83.3\times 10^{-6}F$ and an inductor $L=30 mH=30\times 
10^{-3}H$ connected in series. Find the resulting voltage across all 
three elements.

\begin{itemize}
\item Express $i(t)$ in phasor: $\dot{I}=12\angle{90^\circ}$
\item Find impedance for each element ($\omega=1000$):
\[	Z_R=R=18,\;\;\;\;Z_C=1/j\omega C=12\angle{-90^\circ}=0-j12,\;\;\;\;
	Z_L=j\omega L=30\angle{90^\circ}=0+j30	\]
\item Find overall impedance:
\[ Z_{total}=Z_R+Z_C+Z_L=18+j(-12+30)=18+j18=18\sqrt{2}\angle{45^\circ} \]
\item Find voltage across all three elements:
\[ \dot{V}_{total}=\dot{I}Z_{total}=(12\angle{90^\circ})\;(18\sqrt{2}\angle{45^\circ})=216\sqrt{2}\angle{135^\circ}	\]
\[	v(t)=(216\sqrt{2})\sqrt{2}cos(1000t+135^\circ)=432\;cos(1000t+135^\circ)	\]
\end{itemize}
Alternatively, we can find voltage across each element:
\[	V_R=IZ_R=216\angle{90^\circ},\;\;\;\;
	V_C=IZ_C=144\angle{0^\circ},\;\;\;\;
	V_L=IZ_L=360\angle{180^\circ}	\]
and the total voltage is
\[	\dot{V}_{total}=V_R+V_C+V_L=216\angle{90^\circ}-216\angle{180^\circ}
	=216\sqrt{2}\angle{135^\circ}	\]

{\bf Example 3:} The circuit below represents a load consisting of $C$, $R$
and $L$ supplied by a generator over a transmission line. The voltage from
the generator is $v(t)=28.3\;cos(5000t+45^\circ)\;V$.

\htmladdimg{../figures/phasorexample.gif}

First find the impedances and admittances of the components and the two
branches. As $\omega=5000$, we get
\begin{itemize}
\item $Y_C=j\omega C=j\;5000\times 10^{-5}=j0.05=0.05\angle{90^\circ}$,

	$Z_C=1/Y_C=1/j0.05=-j20=20\angle{-90^\circ}$
\item $Z_L=j\omega L=j5000\times 4\times 10^{-3}=0+j20=20\angle{90^\circ}$,

	$Y_L=1/Z_L=-j0.05=0.05\angle{-90^\circ}$
\item $Z_R=R=20+j0=20\angle{0^\circ}$

\item $Z_{RL}=Z_R+Z_L=20+j20=20\sqrt{2}\angle{45^\circ}$,
	$Y_{RL}=1/Z_{RL}=1/20\sqrt{2}\angle{45^\circ}
	=0.025\sqrt{2}\angle{-45^\circ}=0.025-j0.025$

\item $Y_{load}=Y_C+Y_{RL}=j0.05+0.025-j0.025=0.025+j0.025
	=0.025\sqrt{2}\angle{45^\circ}$
\end{itemize}
Next express voltage $v(t)=28.3\;cos(5000t+45^\circ)\;V$ in phasor form
$\dot{V}=20\angle{45^\circ}$, and find currents in phasor form:
\begin{itemize}
\item $\dot{I}_{load}=\dot{V}Y_{load}=(20\angle{45^\circ})\;
	(0.025\sqrt{2}\angle{45^\circ})=0.5\sqrt{2}\angle{90^\circ}$,

	$i_{load}(t)=Re[\dot{I}\sqrt{2}e^{j5000t}]=cos(5000t+90^\circ)$
\item $\dot{I}_C=Y_C\dot{V}=(0.05\angle{90^\circ})(20\angle{45^\circ})
	=1\angle{135^\circ}=-0.5\sqrt{2}+j0.5\sqrt{2}$,

	$i_C(t)=\sqrt{2}cos(5000t+135^\circ)$
\item $\dot{I}_{RL}=Y_{RL}\dot{V}=(0.025\sqrt{2}\angle{-45^\circ})(20\angle{45^\circ})
	=0.5\sqrt{2}\angle{0^\circ}=0.5\sqrt{2}$,

	$i_{RL}(t)=cos(5000t)$
\item $\dot{V}_R=Z_R\dot{I}_{RL}=(20\angle{0^\circ})(0.5\sqrt{2}\angle{0^\circ})
	=10\sqrt{2}\angle{0^\circ}=10\sqrt{2}+j0$,
	$i_R(t)=20\;cos(5000t)$
\item $\dot{V}_L=Z_L\dot{I}_{RL}=(20\angle{90^\circ})(0.5\sqrt{2}\angle{0^\circ})
	=10\sqrt{2}\angle{90^\circ}=0+j10\sqrt{2}$,
	$v_L(t)=20\;cos(5000t+90^\circ)$.
\end{itemize}
Verify: 
\begin{itemize}
\item $\dot{V}_R+\dot{V}_L=10\sqrt{2}+j10\sqrt{2}=20\angle{45^\circ}
	=\dot{V}$
\item $\dot{I}_C+\dot{I}_{RL}=0.5\sqrt{2}-0.5\sqrt{2}+j0.5\sqrt{2}
	=0.5\sqrt{2}\angle{90^\circ}=\dot{I}_{load}$
\end{itemize}

\subsection*{Complete Response}

The behavior of a circuit composed of resistors, capacitors and inductors can
be described by differential equations. The solution of DE represents the 
response (or output) of the circuit to both the external stimulus (or input) 
and the initial state, and is composed of two parts:
\begin{itemize}
\item homogeneous solution representing the transient or natural response 
	caused by the non-zero initial condition,
\item particular solutions representing the steady-state or forced response 
	caused by the external stimulus. 
\end{itemize} 

For example, the RC and RL circuits shown in the figure are composed of a 
resistor $R$ and capacitor $C$, or an inductor $L$ in series with a switch
which is closed at $t=0$. The current $i(t),\;\;(t>0)$ through the elements
and the voltages $v_R(t)$, $v_C(t)$ and $v_L(t)$ across the components are
considered as the responses of the circuit.

\htmladdimg{../figures/RC_RLcircuit.gif}

{\bf Homogeneous response}

The RC circuit can be described by KVL:
\[	v_R(t)+v_C(t)=0	\]
But since 
\[	v_R(t)=R i(t),\;\;\;\;\;i(t)=C\frac{d}{dt}v_C(t)	\]
we get a first order linear homogeneous ordinary DE:
\[	RC\frac{d}{dt} v_c(t)+v_c(t)=\tau \frac{d}{dt} v_C(t)+v_C(t)=0	\]
where $\tau=RC$ is the time constant of the system. The dimension of $\tau$ 
can be found to be time:
\[	[RC]=\frac{[V]}{[I]}\frac{[Q]}{[V]}=[T]	\]

Similarly, for the RL circuit we have:
\[	v_R(t)+v_L(t)=0	\]
But since 
\[	v_R(t)=R i(t),\;\;\;\;\;v_L(t)=L\frac{d}{dt}i(t)	\]
we get a first order linear homogeneous ordinary DE:
\[	L\frac{d}{dt} i(t)+R\;i(t)=\tau \frac{d}{dt} i(t)+i(t)=0	\]
where $\tau=L/R$ is the time constant of the system. The dimension of 
$L$ is $[L]=[V][T]/[I]=[R][T]$ and the dimension of $\tau$ is: 
\[	[L/R]=\frac{[R][T]}{[R]}=[T]	\]

To solve this homogeneous DE, we substitute the general solution 
$v_C=A e^{pt}$ and its derivative $dv_C(t)/dt=p A e^{pt}$ into the DE,
and get
\[	\tau p+1=0,\;\;\;\;\;p=-\frac{1}{\tau}	\]
i.e., $v_C(t)=Ae^{-t/\tau}$ for ($t\le 0$). Assume the initial charge on $C$
is $v_C(0)=V_0$ (initial condition), we can find the unknown constant $A$:
\[	v_C(0)=A e^{0}=V_0,\;\;\;\;\;\mbox{i.e.}\;\;\;\;\;A=V_0	\]
and the solution is 
\[	v_C(t)=V_0 e^{-t/\tau}		\]
The current through $R$ and $C$ is
\[	i(t)=C\frac{d}{dt}v_C(t)=C\frac{d}{dt}[V_0 e^{-t/\tau}]
	=-V_0 \frac{C}{\tau} e^{-t/\tau}=-\frac{V_0}{R} e^{-t/\tau}	\]
The voltage across $R$ is 
\[	v_R(t)=i(t)R=-V_0 e^{-t/\tau}=-v_C(t)	\]
This result can be verified: $v_C(t)+v_R(t)=0$.

{\bf DC-input response}

For an non-zero input of a DC voltage $v(t)=V_s$, the DE becomes
\[	v_R(t)+v_C(t)=\tau\frac{d}{dt} v_c(t)+v_c(t)=V_s	\]
The solution of this inhomogeneous DE is composed of two parts, 
\begin{itemize}
\item {\bf homogeneous (natural, transient) solution:} $v'_C(t)=V_s$ as
	the steady-state response to the constant input $V_s$,
\item {\bf particular (forced, steady-state) solution:} 
	$v''_C(t)=A e^{-t/\tau}$ same as before.
\end{itemize}
Therefore the overall solution is:
\[	v_C(t)=v'_C(t)+v''_C(t)=V_s+A e^{-t/\tau}	\]
From the intital condition $v(0)=V_0$, we have
\[	v_C(0)=V_0=V_s+A,\;\;\;\;\;\mbox{i.e.,}\;\;\;\;A=V_0-V_s	\]
and the solution becomes
\[	v_C(t)=V_s+(V_0-V_s) e^{-t/\tau}	\]
In particular, for zero initial condition $v_C(0)=0$, the solution is
\[	v_C(t)=V_s+(-V_s) e^{-t/\tau}=V_s(1-e^{-t/\tau})		\]
The voltage across $R$ is
\[
v_R(t)=V_s-v_C(t)=V_s-[V_s+(V_0-V_s) e^{-t/\tau}]=(V_s-V_0) e^{-t/\tau}
\]
and the current through the circuit is
\[	i(t)=\frac{V_R(t)}{R}=\frac{V_s-V_0}{R} e^{-t/\tau}	\]

The two plots below show $v_C(t)$ (red) and $v_R(t)$ (green) under
different initial conditions $V_0$ (purple) and inputs $V_s$ (blue):

\htmladdimg{../figures/first_order_response1.gif}
\htmladdimg{../figures/first_order_response2.gif}

In general, A first order system's response $f(t)$ (whether a current 
$i(t)$ or a voltage $v(t)$) to a DC input with non-zero initial condition 
can be expressed as:
\[	f(t)=f(\infty)+[f(0)-f(\infty] e^{-t/\tau}	\]
where 
\begin{itemize}
\item $f(0)$ is the initial value, 
\item $f(\infty)$ is the steady-state response, 
\item $\tau$ is the time constant of the system.
\end{itemize}
These are the three essential components of the system's response.
Where there is only one resistor in the circuit, the time constant
is $\tau=RC$ or $\tau=L/R$. When there are multiple resistors, the
time constant can be found by:
\begin{itemize}
\item Remove $C$ or $L$ so that the rest of the circuit ($t>0$) is 
	a one port network.
\item Find the equivalent resistance $R$ of the network by turning 
	off all energy sources (short-circuit for voltage source,
	open-circuit for current source).
\item Find time constant $\tau=RC$ or $\tau=L/R$.
\end{itemize}

{\bf Example 1:}

Find $v_C(t)$ after the switch is closed at $t=0$. Assume $V_0=10V$,
$R_1=R_2=2K\Omega$, $R_3=5K\Omega$, $C=0.5\mu F$, and the circuit was 
in steady-state at $t=0$. 

\htmladdimg{../figures/bridgecapacitor.gif}

\begin{itemize}
\item Find initial value $v_C(0)$. Current through and voltage across
	$R_3$ are both zero as the circuit was stable when $t=0$.
\[	v_C(0)=-V_{R_1}=-V_0\frac{R_1}{R_1+R_2}=-V_0/2=-5V	\]
\item Find steady-state value $v_C(\infty)$:
\[	v_C(\infty)=V_{R_2}=V_0\frac{R_2}{R_1+R_2}=V_0/2=5V	\]
\item Find equivalent resistance $R$:
\[	R=R_1 || R_2=\frac{R_1 R_2}{R_1+R_2}=1	\]
\item Find time constant $\tau=RC=1000\times 0.5\times 10^{-6}=5\times 10^{-4}$
\item Find the complete response
\[	v_C(t)=5+(-5-5) e^{-t/0.0005}=5-10 e^{-2000t}\;(V)	\]
\item Find currents through $R_2$ and $R_1$:
\[	i_2=\frac{v_C(t)}{R_2}=\frac{5-10 e^{-2000t}}{2\times 10^3}
	=2.5-5 e^{-2000t}\; (mA)	\]
\[	i_1=\frac{V_0-v_C}{R_1}=\frac{10-(5-10 e^{-2000t})}{2\times 10^3}
	=2.5+5 e^{-2000t} \; (mA)	\]
\[	i_C=i_1-i_2=10 e^{-2000t} \; (mA)	\]
\end{itemize}

{\bf Sinudoidal input response}

If the input voltage is sinusoidal $v(t)=V_s cos(\omega t+\psi)$, the 
DE becomes
\[ v_R(t)+v_C(t)=\tau\frac{d}{dt} v_c(t)+v_c(t)=V_s cos(\omega t+\psi)	\]
The homogeneous solution is the same as before $v'_C(t)=A e^{-t/\tau}$.
The particular solution as the steady-state response to $v(t)$ can be
found by phasor method. The input is 
\[	v(t)=V_s cos(\omega t)=Re[\dot{V}\sqrt{2}e^{j\omega t}]	\]
where $\dot{V}=V_s/\sqrt{2}\;e^{j\psi}$. The steady-state voltage across
$C$ is (voltage divider)
\[	\dot{V}_C=\dot{V} \frac{1/j\omega C}{R+1/j\omega C}
	=\frac{\dot{V}}{j\omega \tau+1}	
	=\frac{\dot{V}\;e^{-j\phi}}{\sqrt{(\omega \tau)^2+1}} \]
where $\phi=\tan^{-1} \omega \tau$, and 
\[
v'_C(t)=Re[\frac{V_s e^{j\omega t+\psi}\;e^{-j\phi}}{\sqrt{(\omega \tau)^2+1}} ]
= \frac{V_s}{\sqrt{(\omega \tau)^2+1}} cos(\omega t+\psi-\phi)	\]
The complete solution is then the sum of both homogeneous and particular
solutions:
\[	v_C(t)=v'_C(t)+v''_C(t)=
\frac{V_s}{\sqrt{(\omega \tau)^2+1}}cos(\omega t+\psi-\phi)+A e^{-t/\tau}	\]
From the inital condition $v_C(0)=V_0$, we have
\[ v_C(0)=V_0=\frac{V_s}{\sqrt{(\omega \tau)^2+1}}cos(\psi-\phi)+A \]
Solving for $A$ we get
\[	A=V_0-\frac{V_s}{\sqrt{(\omega \tau)^2+1}}cos(\psi-\phi) \]
Substituting $A$ back to $v_C(t)$, we get 
\[  v_C(t)=\frac{V_s}{\sqrt{(\omega \tau)^2+1}}cos(\omega t+\psi-\phi)+ 
[V_0-\frac{V_s}{\sqrt{(\omega \tau)^2+1}}cos(\psi-\phi)] e^{-t/\tau}	\]
If the initial voltage on $C$ is zero $V_0=0$, then
\[  v_C(t)=\frac{V_s}{\sqrt{(\omega \tau)^2+1}}[cos(\omega t+\psi-\phi) 
-cos(\psi-\phi) e^{-t/\tau}]	\]
In particular, when $\psi-\phi=\pm 90^\circ$, $cos(\psi-\phi)=\pm 1$,
the magnitude of the initial voltage $v_C(0)$ could be twice as much as
the steady-state value.

\htmladdimg{../figures/first_order_response3.gif}
\htmladdimg{../figures/first_order_response4.gif}

In general, a first order system's response $f(t)$ (whether a current 
$i(t)$ or a voltage $v(t)$) to a sinusoidal input with non-zero initial 
condition can be expressed as:
\[	f(t)=f_\infty(t)+[f(0)-f_\infty(0)] e^{-t/\tau}	\]
where 
\begin{itemize}
\item $f(0)$ is the initial value, 
\item $f_\infty(t)$ is the steady-state response, 
\item $\tau$ is the time constant of the system.
\end{itemize}
These are the three essential components of the system's response.

{\bf Example 2:}

An electromagnet, modeled by a resistor $R=20\Omega$ and $L=0.3H$,
is powered by sinusoidal voltage of $120V$ and $60Hz$. Find the current
through the circuit when the switch is closed at $t=0$ when the phase
angle happens to be $\psi=10^\circ$, i.e., $v(t)=120\sqrt{2}
cos(6.28\times 60 t+10^\circ)$.

\begin{itemize}
\item Find initial value $i(0)=0$. 
\item Find impedance of circuit:
\[ Z=R+j\omega L=20+j6.28\times 60 \times 0.3=20+j113=114.8\angle{80^\circ}	\]
\item Find steady-state value $i_\infty(t)$ by phasor method:
\[
\dot{V}=120 \angle{10^\circ},\;\;\;\;\dot{I}=\frac{\dot{V}}{Z}
	=\frac{120\angle{10^\circ}}{114.8\angle{80^\circ}}
	=1.05\angle{-70^\circ}	
\]
\[	i_\infty(t)=1.05\sqrt{2}cos(6.28\times 60 t -70^\circ)
	,\;\;\;\;i_\infty(0)=1.05\sqrt{2}cos(-70^\circ)
	=1.05\sqrt{2}\times 0.342=-0.52	\]
\item Find time constant $\tau=L/R=0.3/20=0.015$.
\item Find current
\[	i(t)=1.05\sqrt{2}cos(6.28\times 60 t -70^\circ)+0.52 e^{-t/0.015}	\]

\end{itemize}

\subsection*{Series and Parallel Resonance}

{\bf Series Resonance:} An RCL series circuit consists of a resistor $R$, 
an inductor $L$ and a capacitor connected in series to a voltage source.

The overall impedance of the three elements is
\[ Z=R+j\omega L+\frac{1}{j\omega C}=R+j(\omega L-\frac{1}{\omega C})
	=|Z|e^{j\phi}	\]
where
\[	|Z|=\sqrt{R^2+(\omega L-\frac{1}{\omega C})^2},\;\;\;\;
	\phi=\angle Z=tan^{-1} \frac{\omega L-1/\omega C}{R} \]

\htmladdimg{../figures/impedanceRCL1.gif}

In particular we define the {\bf resonant frequency} as
\[	\omega_0\stackrel{\triangle}{=}\frac{1}{\sqrt{LC}}	\]
When $\omega=\omega_0$, the circuit is at resonance with the following
properties:
\begin{itemize}
\item The effects of $L$ and $C$ cancel each other
\[	j\omega L+\frac{1}{j\omega C}=j(\omega L-\frac{1}{\omega C})=0 \]
\item The impedances of the capacitor and the inductor have the same 
	magnitude but opposite phase:
\[	Z_L=j\omega_0L=j\sqrt{\frac{L}{C}},\;\;\;\;
	Z_C=1/j\omega_0L=-j\sqrt{\frac{L}{C}}	\]
\item The complex impedance $Z$ is real and reaches a minimum
\[	|Z|=\sqrt{R^2+(\omega L-\frac{1}{\omega C})^2}=R,\;\;\;\;
	\angle Z=0	\]
\item The current $\dot{I}$ and voltage $\dot{V}$ are in phase, and given
	$\dot{V}$, the current $\dot{I}=\dot{V}/Z=\dot{V}/R$ reaches 
	maximum value.
\end{itemize}

The ratio of the magnitude of the inductor/capacitor impedance and the
resistance is defined as the {\bf quality factor}
\[	Q\stackrel{\triangle}{=}\frac{\omega_0L}{R}=\frac{1}{\omega_0CR}
	=\frac{1}{R}\sqrt{\frac{L}{C}}	\]
$Q$ is inversely proportional to the resistance $R$ in the circuit.

According to the phasor version of Ohm's law, the voltages across each 
of the three components are:
\begin{itemize}
\item 
\[ \dot{V}_R=\dot{I} R=\dot{V}	\]
\item 
\[ \dot{V}_L=\dot{I} j\omega_0 L|_{\omega_0=1/\sqrt{LC}}
	=j \sqrt{\frac{L}{C}} \frac{\dot{V}}{R}	=jQ\dot{V}	\]
\item 
\[ \dot{V}_C=\dot{I} \frac{1}{j\omega_0 C}|_{\omega_0=1/\sqrt{LC}}
	=-j \sqrt{\frac{L}{C}} \frac{\dot{V}}{R}=-jQ\dot{V}	\]
\end{itemize}
The magnitude of the voltage $V_L$ across $L$ and $V_C$ across $C$ are
$Q$ times larger than the voltage $\dot{V}_R$ across $R$ which is the 
same as the voltage source $\dot{V}$). But as $V_L$ and $V_C$ are in
opposite polarity, they cancel each other.

The RCL series circuit is a band-pass filter with the passing band centered
around the resonant frequency $\omega_0=1/\sqrt{CL}$, which can be adjusted
by changing either $C$ or $L$. The bandwidth or the sharpness is determined
by the quality factor $Q$ (or the resistance $R$), the larger $Q$ (smller
$R$), the narrower the bandwidth. The impedance $Z=Z_R+R_L+Z_C$ as a function 
of $\omega$ is shown below

\htmladdimg{../figures/omega0c.gif}

and the admittance $Y=1/Z$ for different $Q$ ($R$) and $C$ is shown below:

\htmladdimg{../figures/omega0a.gif}
\htmladdimg{../figures/omega0b.gif}

This bandpass effect can be intuitively explained. When $\omega$ is high,
the inductor's impedance $\omega L$ is high, and when $\omega$ is low,
the capacitor's impedance $1/\omega C$ is high. When $\omega=\omega_0$
the overall impedance is the smallest. If the input is a voltage source 
$v(t)$, the current through the circuit $i(t)=v(t)/Z=v(t)Y$ will reach a
maximum value when $\omega=\omega_0$.

{\bf Example: } In a series RLC circuit, $R=5\Omega$, $L=4\;mH$ and
$C=0.1\;\mu F$. The resonant frequency $\omega_0$ can be found to be
$\omega_0=1/\sqrt{LC}=1/\sqrt{4\times 10^{-3}\times 10^{-7}}=5\times 10^4$.
The quality factor is
\[	Q=\frac{\omega_0L}{R}=\frac{(5\times 10^4)\times (4\times 10^{-3})}{5}
	=40	\]
or
\[	Q=\frac{1}{\omega_0CR}=\frac{1}{(5\times 10^4)\times 10^{-7}\times 5}
	=40	\]
If the input voltage is $V_{rms}=10V$ at the resonant frequency, the current
is $I=V/R=10/5=2 A$, and the voltages across each of the elements are:
\begin{itemize}
\item $V_R=RI=V=10V$
\item $V_L=\omega L I=5\times 10^4\times 4\times 10^{-3} \times 2=400V$
\item $V_C=I/\omega_0C=2/(5\times 10^4\times 10^{-7})=400V$
\end{itemize}
Or more conveniently, the amplitudes of $V_L$ and $V_C$ can be found by
$ V_L=V_C=QV=40\times 10V=400V$. Note that although input voltage is $10V$,
the voltage across L and C ($Q$ times the input) could be very high (but
they are in opposite phase and therefore cancel).

{\bf Parallel Resonance:} A GCL parallel circuit consists of a resistor 
$R=1/G$, an inductor $L$ and a capacitor connected in parallel to 
input voltage. 

\htmladdimg{../figures/impedanceRCL2.gif}

In this case, it is much easier to consider the conductance of the 
admittance $Y=1/Z$ of each of the element. The overall admittance of
the three elements in parallel is
\[ Y=G+j\omega C+\frac{1}{j\omega L}=G+j(\omega C-\frac{1}{\omega L})
	=|Y|e^{j\phi}	\]
where
\[	|Y|=\sqrt{G^2+(\omega C-\frac{1}{\omega L})^2},\;\;\;\;
	\phi=\angle Y=tan^{-1} \frac{\omega C-1/\omega L}{G} \]
In particular when
\[	j\omega L+\frac{1}{j\omega C}=j(\omega L-\frac{1}{\omega})=0 \]
the effects of $L$ and $C$ cancel each other, i.e., at the {\bf resonant
frequency}
\[	\omega_0\stackrel{\triangle}{=}\frac{1}{\sqrt{LC}}	\]
The complex admittance $Y$ is real and reaches a minimum
\[	|Y|=\sqrt{G^2+(\omega L-\frac{1}{\omega c})^2}=G,\;\;\;\;\;	
	\angle{Y}=0	\]
and the current $\dot{I}$ reaches a minimum value $\dot{I}=\dot{V}G$.

The {\bf Quality Factor} $Q$ of a parallel resonance circuit is defined
as the ratio of the magnitude of the inductor/capacitor susceptance and 
the conductance:
\[	Q\stackrel{\triangle}{=}\frac{\omega_0C}{G}=\frac{1}{\omega_0LG}
	=\frac{1}{G}\sqrt{\frac{C}{L}}	\]
$Q$ is inversely proportional to the conductance $G$ in the circuit.

The currents through each of the three components are:
\begin{itemize}
\item 
\[ \dot{I}_R=\dot{V} G=\dot{I}	\]
\item 
\[ \dot{I}_C=\dot{V} j\omega_0 C|_{\omega_0=1/\sqrt{LC}}
	=j \frac{\dot{I}}{G}\sqrt{\frac{C}{L}}=jQ\dot{I}	\]
\item 
\[ \dot{I}_L=\dot{V}\frac{1}{j\omega_0 L}|_{\omega_0=1/\sqrt{LC}}
	=-j \frac{\dot{I}}{G}\sqrt{\frac{C}{L}}=-jQ\dot{I}	\]
\end{itemize}
The magnitude of the current $I_L$ through $L$ and $I_C$ through $C$ are
$Q$ times larger than the current $\dot{V}_R$ through $R$ which is the 
same as the current source $\dot{I}$). But as $I_L$ and $I_C$ are in
opposite direction, they form a loop current through $L$ and $C$ with
no effect to the rest of the circuit.

The parallel RCL circuit behaves like a bandstop filter which can be 
intuitively understood. When $\omega$ is high, the capacitor's impedance 
$1/\omega C$ is low, and when $\omega$ is low, the inductor's impedance 
$\omega L$ is low. When $\omega=\omega_0$ the overall impedance is the
largest. However, if the input is a current source, the voltage across 
the elements $\dot{V}=\dot{I}Z=\dot{I}/Y$ will reach a maximum value
when $\omega=\omega_0$.


\subsection*{Quality Factor}

The general meaning of the quality factor $Q$ of a circuit is the ratio 
between the energy stored in the circuit (in $C$ and $L$) and the energy 
dissipated (by $R$):
\[
Q=2\pi\frac{\mbox{maximum energy stored}}{\mbox{energy dissipated per cycle}}
\]
We can show that this definition is the same as the $Q$ for the series 
RCL circuit. The energy is stored in the inductor and the capacitor 
alternatively, and the maximum energy stored in $L$ is the same as that 
in $C$. 

The energy store in $L$ is:
\[	W_L=\int_0^T v(t)\; i(t) dt=\int_0^T i(t) L \frac{di(t)}{dt} dt
	=L \int_0^{I_p} i di=\frac{1}{2}LI_p^2=LI^2_{rms}	\]
where $i(T)=I_p=\sqrt{2}I_{rms}$ is the maximum or peak current through
$L$. 

The energy dissipated in $R$ per cycle $T_0=2\pi/\omega_0=1/f_0$ is:
\[	W_R=P_R T_0=I^2_{rms} R T_0	\]
Then
\[ 2\pi\frac{LI^2_{rms}}{I^2_{rmw}RT_0 }
	=2\pi f_0\frac{L}{R}=\frac{\omega_0L}{R}=Q	\]
We can also use the energy store in $C$:
\[	W_C=\int_0^T v(t)\; i(t) dt=\int_0^T v(t) C \frac{dv(t)}{dt} dt
	=C \int_0^{V_p} v dv=\frac{1}{2}CV_p^2=CV^2_{rms}	\]
where $v(T)=V_p=\sqrt{2}V_{rms}$ is the maximum or peak voltage across
$C$, and $V_{rms}=I_{rms}/\omega_0C$. Then
\[ 2\pi\frac{CV^2_{rms}}{I^2_{rmw}RT_0 }
	=2\pi f_0\frac{CI^2_{rms}/\omega_0^2C^2}{I^2_{rms} R}
	=\frac{1}{\omega_0CR}=Q	\]

{\bf Bandwidth}

For a series RCL circuit with voltage input, the impedance is
\[	Z=R+j(\omega L-\frac{1}{\omega C})=\frac{1}{Y} \]
and at resonance when $\omega=\omega_0=1/\sqrt{LC}$, we have
\[	Z_0=R=\frac{1}{Y_0}	\]
Consider the ratio 
\[	\frac{Y}{Y_0}=\frac{Z_0}{Z}=\frac{R}{R+j(\omega L-1/\omega C)}
	=[1+j(\frac{\omega L}{R}-\frac{1}{\omega CR})]^{-1}	\]
As $Q=\omega_0L/R=1/\omega_0CR$, we substitute $L/R=Q/\omega_0$ and
$1/RC=Q\omega_0$ into the equation above and get
\[
\frac{Y}{Y_0}=[1+jQ(\frac{\omega}{\omega_0}-\frac{\omega_0}{\omega})]^{-1}
\]

\htmladdimg{../figures/bandwidth1.gif}

%Similarly, for a parallel GCL circuit, the .... Q of GCL....

At the resonant frequency $\omega=\omega_0$, the ratio $Y/Y_0=1$
reaches the peak, and when $\omega$ is either lower or higher than 
$\omega_0$, the magnitude ratio $Y/Y_0$ is smaller. The bandwidth 
is defined as 
\[ \triangle \omega\stackrel{\triangle}{=}\omega_2-\omega_1 \]
where $\omega_2>\omega_1$ are the two cut-off frequencies (or 
half-power frequency) at which $|Y/Y_0|=1/\sqrt{2}=0.707$ (the
power is halved). This requires
\[	\frac{Y}{Y_0}=\frac{1}{1\pm j1},\;\;\;\;\;\mbox{i.e.}\;\;\;\;
\frac{\omega}{\omega_0}-\frac{\omega_0}{\omega}=\pm \frac{1}{Q} \]
The two cuf-off frequencies should satisfy
\[ \frac{\omega_1}{\omega_0}-\frac{\omega_0}{\omega_1}=-\frac{1}{Q},\;\;\;\;
   \frac{\omega_2}{\omega_0}-\frac{\omega_0}{\omega_2}=\frac{1}{Q}	\]
Solving these two equations, we get:
\[ \omega_1=\omega_0[\sqrt{1+(\frac{1}{2Q})^2}-\frac{1}{2Q}],\;\;\;\;
   \omega_2=\omega_0[\sqrt{1+(\frac{1}{2Q})^2}+\frac{1}{2Q}]	\]
from which we can see that:
\[	\omega_2-\omega_0 > \omega_0-\omega_1	\]
i.e., the lower half-power point $\omega_1$ is closer to resonant frequency
$\omega_0$ than the higher half-power point $\omega_2$:

The bandwidth is 
\[ \triangle \omega=\omega_2-\omega_1=\frac{\omega_0}{Q},
\;\;\;\;\mbox{or}\;\;\;\;\triangle f=f_2-f_1=\frac{f_0}{Q}	\]
i.e., the bandwidth is inversely proportional to the quality factor $Q$.

\htmladdimg{../figures/bandwidth.gif}

In practice, $Q$ is usually much greater than 1 (typically $Q>10$), we have
$\sqrt{1+(1/2Q)^2} \approx 1$ and
\[	\omega_1=\omega_0-\frac{\omega_0}{2Q},\;\;\;\;\;\;
	\omega_2=\omega_0+\frac{\omega_0}{2Q}			\]
we therefore get these simple relations:
\[	\omega_2-\omega_0 = \omega_0-\omega_1,\;\;\;\;\;
	\omega_2-\omega_1=\frac{\omega_0}{Q}	\]

For a parallel RCL circuit with current input, due to the duality between
current and voltage, parallel and series configuration, the exactly same 
derivation of bandwidth can be carried out as for the RCL series circuit
with voltage input, and the same conclusions can be obtained. 


{\bf Summary:}
\begin{itemize}
\item In both series and parallel RCL circuits, an extremum is reached
when the frequency of the signal is the same as the resonant frequency
completely determined by $L$ and $C$: $\omega_0=1/\sqrt{LC}$, independent
of the resistance $R$ in the circuit.

\item In series RCL with voltage input and parallel RCL with current input, 
the quality factor $Q$ is proportional to the ratio between $L$ and $C$:
\[ Q_s=\frac{1}{R}\sqrt{\frac{L}{C}},\;\;\;\;Q_p=R\sqrt{\frac{C}{L}} \]
The resistance $R$ plays a different role: in series RCL, $Q_s$ is inversely
proportional $R$ (the larger $R$, the smaller $Q_s$, the more energy lost 
and the wider bandwidth), while in parallel RCL, $Q_p$ is proportional to
$R$ (the larger $R$, the larger $Q_p$, the less energy lost and the narrower
bandwidth).

\item At resonant frequency, the impedance of a series RCL circuit reaches
minimum and while impedance of a parallel RCL circuit reaches maximum. For
a voltage input, the series RCL is a band-pass filter and the parallel RCL 
is a band-stop filter, while for a current input, they behave in an opposite
way.

\item The quality factor $Q_s$ and $Q_p$ are derived under the assumption 
of either a voltage or a current input. Therefore when the input to a 
parallel (series) RCL circuit is a voltage (current) source, we should 
still use $Q_s$ ($Q_p$) in the calculation bandwidth. 

\end{itemize}

{\bf Example:}

A series CL circuit composed of an inductor $L=80\mu H$ and $R=8\Omega$
and a capacitor $C$ is connected to a voltage source. Find the value of 
$C$ for this circuit to resonate at $f=400\;kHz$, also find the bandwidth.

\[	\omega_0=\sqrt{\frac{1}{LC}},\;\;\;\;
C=\frac{1}{\omega_0^2 L}=\frac{1}{(2\pi 400\times 10^3)^2\times 80\times 10^{-6}}
=20nF	\]
The quality factor is
\[
Q=\frac{\omega_0 L}{R}=\frac{2\pi 400\times 10^3\times 80\times 10^{-6}}{8}=25
\]
The bandwidth is
\[	f_2-f_1=\frac{f_0}{Q}=\frac{400\times 10^3}{25}=16\;kHz	\]

%In reality, all inductors have a non-zero resistance, therefore a 
%parallel resonance circuit should be modeled as shown in the figure.
%\htmladdimg{../figures/parallelRCL.gif}
%The admittance is:
%\[	Y=\frac{1}{R+j\omega L}+j\omega C
%	=\frac{R-j\omega L+j\omega C(R^2+\omega^2L^2)}{R^2+\omega^2L^2}
%\]
%When the imaginary part of $Y$ is zoro, the circuit is at resonance and 
%the resonant frequency $\omega_0$ can be found by solving $Im[Y]=0$:
%\[	\omega_0 L=\omega_0 C(R^2+\omega_0^2L^2),\;\;\;\;\Longrightarrow
%	\;\;\;\;\;\omega_0=\sqrt{\frac{1}{LC}-(\frac{R}{L})^2}	\]
%For $\omega_0$ to be real, we must have
%\[	\frac{1}{LC} > (\frac{R}{L})^2, \;\;\;\;\;\;\mbox{i,e,}
%	\;\;\;\;\;\;R<\sqrt{\frac{L}{C}}	\]
%Typically, we have $R \ll \sqrt{L/C}$, and the resonant frequency is
%\[
%\omega_0=\sqrt{\frac{1}{LC}-(\frac{R}{L})^2}\approx \frac{1}{\sqrt{LC}}	
%\]


\subsection*{Sinusoidal AC Power}

A two-terminal network of passive elements (resistors, inductors, 
capacitors, but not energy sources) can be represented by its complex 
impedance 
\[	Z=R+jX=|Z|e^{j\phi}=|Z|\angle \phi \]
where 
\[	|Z|=\sqrt{R^2+X^2},\;\;\;\phi=\angle Z=\tan^{-1}(X/R) \]
If the input voltage to the network is:
\[ \dot{V}=V_{rms}\angle 0,\;\;\;\;v(t)=\sqrt{2}V_{rms} \cos(\omega t) \]
the output current through the network is:
\[ \dot{I}=\frac{\dot{V}}{|Z|\angle \phi}=\frac{V_{rms}}{|Z|\angle \phi}=
I_{rms}\angle(-\phi),\;\;\;\;i(t)=\sqrt{2}I_{rms} \cos(\omega t-\phi)  \]
where $V_{rms}$ and $I_{rms}=V_{rms}/|Z|$ are the effective voltage and 
current, respectively.

The {\bf instantaneous power} in the network is
\begin{eqnarray}
p_{in}(t) &=& v(t)\;i(t)=2V_{rms}I_{rms}\;\cos(\omega t)\;\cos(\omega t-\phi)
	\nonumber \\
&\stackrel{*}{=}&2V_{rms}I_{rms}\;\cos(\omega t)\;[\cos(\omega t)\cos\phi
	+\sin(\omega t)\sin\phi ]	\nonumber \\
&=&V_{rms}I_{rms}\cos\phi [2\cos^2(\omega t)]
	+V_{rms}I_{rms}\sin\phi [2\cos(\omega t)\sin(\omega t)] \nonumber \\
&=&V_{rms}I_{rms}\cos\phi\; p(t)+V_{rms}I_{rms}\sin\phi\; q(t)	\nonumber 
\end{eqnarray}
Here
\[ 	p(t)\stackrel{\triangle}{=}2\cos^2(\omega t),\;\;\;\;\;\;
	q(t)\stackrel{\triangle}{=}2\cos(\omega t)\sin(\omega t) \]

(* $\cos(u-v)=\cos u\;\cos v+\sin u\;\sin v$)

\htmladdimg{../figures/instantaneouspower1.gif}

In particular, consider the average of $p(t)$ and $q(t)$ over a period $T$:
\[ \frac{1}{T}\int_T p(t)dt=\int_T 2\cos^2(\omega t) dt
	=\frac{1}{T}\int_T (1+cos(2\omega t) dt=1	\]
\[ \frac{1}{T}\int_T q(t)dt=\int_T 2\cos(\omega t)\sin(\omega t) dt
	=\frac{1}{T}\int_T \sin(2\omega t) dt=0	\]
It is seen that the instantaneous power is composed of two components: the 
first term $p(t)=2\cos^2(\omega t)$ representing energy dissipation, and the 
second term $q(t)=2\cos(\omega t)\sin(\omega t)$ representing energy not dissipated
but store in the system. Instantaneous $p_{in}(t)$ can be both positive (for energy 
consumed by the load) and negative (for energy released by the load).

\htmladdimg{../figures/instantaneouspower2.gif}

The {\bf average power} over period $T=2\pi /\omega$ is:
\begin{eqnarray}
P_{av}&\stackrel{\triangle}{=}&\frac{1}{T}\int_0^T p_{in}(t) dt
=V_{rms}I_{rms}\cos\phi \frac{1}{T}\int_0^T p(t) dt
+V_{rms}I_{rms}\sin\phi \frac{1}{T}\int_0^T q(t) dt	\nonumber \\
&\stackrel{*}{=}&
V_{rms}I_{rms}\cos\phi \; 1+V_{rms}I_{rms}\sin\phi \;0
	\nonumber \\
&\stackrel{**}{=}&V_{rms}I_{rms}[\cos\phi\;1 +\sin\phi\; 0]
	=V_{rms}I_{rms}\;\cos\phi
	\nonumber 
\end{eqnarray}

\begin{itemize}
\item {\bf Apparent Power:} $S\stackrel{\triangle}{=}V_{rms}I_{rms}$;
\item {\bf Real power:} $P\stackrel{\triangle}{=}V_{rms}I_{rms}\;\cos\phi
	=V_{rms}I_{rms}\;\lambda=P_{av}$

	This is the first term (non-zero) of the average power expression, 
	which is consumed by the load (dissipated, converted to heat);
\item {\bf  Reactive Power:} $Q\stackrel{\triangle}{=}V_{rms}I_{rms}\;\sin\phi$

	This is the second term (zero) of the average power expression,	
	which is not consumed but converted back and forth between the
	energy source and the energy storing elements (inductors and capacitors).
\end{itemize}
The definitions of real power $P=V_{rms}I_{rms} \cos\phi$ and reactive 
power $Q=V_{rms}I_{rms} \sin\phi$ above suggest that the power $S$ can 
be treated a complex variable:
\[ S\stackrel{\triangle}{=}VI^*=V_{rms}\angle 0\; I_{rms}\angle \phi
	=V_{rms}I_{rms}\angle \phi=V_{rms}I_{rms}(\cos\phi+j\;\sin\phi)
	=P+jQ	\]
Substituting $V=IZ$ into the equation, we have
\[ S\stackrel{\triangle}{=}VI^*=IZI^*=I^2_{rms}(R+jX)
	=I^2_{rms}R+j I^2_{rms}X	\]
i.e.,
\[	P=I^2_{rms}R=V_{rms}I_{rms}\cos\phi,\;\;\;
	Q=I^2_{rms}X=V_{rms}I_{rms}\sin\phi,\;\;\;
	|S|=\sqrt{P^2+Q^2},\;\;\angle S=\tan^{-1}(Q/P)	\]

\htmladdimg{../figures/complexpower.gif}

{\bf Improvement of Power Factor}

The {\bf Power factor} is defined as
\[	\lambda\stackrel{\triangle}{=}\cos \phi < 1	\]
which measures the phase difference between the voltage and current in
the system. In a power system, we want to deliver as much real power 
($P=V_{rms}I_{rms} \cos\phi$) to the consumer as possible while still 
keeping the voltage $V$ and current $I$ under certain limits (determined 
by the capacity of the power system). To efficiently use the system, it
is desirable to maximize the power factor $\lambda=\cos\phi$ by reducing
$\phi$. As most large loads are inductive (e.g., the coils in electric
motors), the power factor can be reduced by using the shunt capacitor 
to match the inductance in the system so that their effects can be 
canceled.

\htmladdimg{../figures/shuntcapacitor.gif}

{\bf Example: } Find $C$ of the shunt capacitor so that the power
factor can be improve to 1.

Assume before $C$ is used, the impedance of the inductive load is
$Z=R+j\omega L$ with phase $\phi=tan^{-1} (\omega L/R)$. After the
shunt capacitor is added, the load becomes 
\[	Z'=(R+j\omega L)\; || \;(1/j\omega C)
	=\frac{(R+j\omega L)/j\omega C}{(R+j\omega L)+1/j\omega C}
	=\frac{R+j\omega L}{j\omega CR-\omega^2 LC+1}	\]
For the new phase angle to be zero, we need to have
\[ \tan^{-1}\frac{\omega L}{R}=\tan^{-1}\frac{\omega RC}{1-\omega^2 LC},
	\;\;\;\mbox{i.e.}\;\;\;	
	\frac{\omega L}{R}=\frac{\omega RC}{1-\omega^2 LC}	\]
which can be solved for $C$ to get
\[	C=\frac{L}{R^2+\omega^2 L^2}	\]

The shunt capacitor could also be added in series with the inductive 
load. If we choose $C=1/\omega^2L$ so that $1/j\omega C+j\omega L=0$, 
the imaginary part of the impedance is zero and the phase angle is zero
(power factor is 1). In other words, the load is an RCL series circuit
with resonant frequency and quality factor:
\[	\omega_0=1/\sqrt{LC},\;\;\;\;\;\;Q=\frac{1}{R}\sqrt{\frac{L}{C}} \]
The magnitude of the voltage across the inductor and the capacitor is $Q$
times higher than that across the resistor:
\[	\dot{V}_L=jQ\dot{V},\;\;\;\;\;\dot{V}_C=-jQ\dot{V}	\]
When $X_L=\omega L \gg R$ (i.e., when $\phi$ is close to 90 degrees), $Q$
is very large and the high voltages across the inductor and capacitor may
cause damage to the load system (e.g., a motor). 

\subsection*{Two-Port Networks}

{\bf Models of two-port networks}

Many complex circuits can be modeled by a two-port passive and linear model
as shown below. A two-port network has an input port with volage $V_1$ and
current $I_2$, and an output with voltage $V_2$ and current $I_2$. The
relationships between these four variables can be described in different 
ways, depending on which two of the four variables $V_1$, $V_2$, $I_1$ and 
$I_2$ are given, while the other two can always be derived.

{\bf Note: } All voltages and currents below are complex variables and
represented by phasors containing both magnitude and phase angle. However, 
for convenience the phasor notation $\dot{V}$ and $\dot{I}$ are replaced by 
$V$ and $I$ respectively.

\htmladdimg{../figures/twoportmodel.gif}

{\bf Z-model:} In the Z-model or impedance model, the two currents $I_1$ 
and $I_2$ are assumed to be known, and the voltages $V_1$ and $V_2$ can be 
found by:

\[ \left\{ \begin{array}{l} V_1=Z_{11}I_1+Z_{12}I_2 \\
	V_2=Z_{21}I_1+Z_{22}I_2 \end{array} \right.\;\;\;\;\;
	\left[ \begin{array}{l} V_1 \\ V_2\end{array} \right]=
	\left[ \begin{array}{ll} Z_{11} & Z_{12}\\Z_{21} & Z_{22}\end{array}\right]
	\left[ \begin{array}{l} I_1 \\ I_2\end{array} \right]
	={\bf Z}\left[ \begin{array}{l} I_1 \\ I_2\end{array} \right]
\]
Here all four parameters $Z_{11}$, $Z_{12}$, $Z_{21}$, and $Z_{22}$ represent
impedance. In particular, $Z_{21}$ and $Z_{12}$ are {\bf transfer impedances}, 
defined as the ratio of a voltage $V_1$ (or $V_2$) in one part of a network to 
a current $I_2$ (or $I_1$) in another part $Z_{12}=V_1/I_2$. ${\bf Z}$ is a 2 
by 2 matrix containing all four parameters.

{\bf Y-model:} In the Y-model or admitance model, the two voltages $V_1$ and $V_2$
are assumed to be known, and the currents $I_1$ and $I_2$ can be found by:
\[ \left\{ \begin{array}{l} I_1=Y_{11}V_1+Y_{12}V_2 \\
	I_2=Y_{21}V_1+Y_{22}V_2 \end{array} \right.\;\;\;\;\;
	\left[ \begin{array}{l} I_1 \\ I_2\end{array} \right]=
	\left[ \begin{array}{ll} Y_{11} & Y_{12}\\Y_{21} & Y_{22}\end{array}\right]
	\left[ \begin{array}{l} V_1 \\ V_2\end{array} \right]
	={\bf Y}\left[ \begin{array}{l} V_1 \\ V_2\end{array} \right]
\]
Here all four parameters $Y_{11}$, $Y_{12}$, $Y_{21}$, and $Y_{22}$ represent
admitance. In particular, $Y_{21}$ and $Y_{12}$ are {\bf transfer admitances}. 
${\bf Y}$ is the corresponding parameter matrix.

{\bf A-model:} In the A-model or transmission model, we assume $V_2$ and $I_2$
are known, and find $V_1$ and $I_1$ by:
\[\left\{ \begin{array}{l} 
	V_1=A_{11}V_2+A_{12}(-I_2) \\
	I_1=A_{21}V_2+A_{22}(-I_2) \end{array} \right.\;\;\;\;\;
	\left[ \begin{array}{l} V_1 \\ I_1\end{array} \right]=
	\left[ \begin{array}{ll} A_{11} & A_{12}\\A_{21} & A_{22}\end{array}\right]
	\left[ \begin{array}{r} V_2 \\ -I_2\end{array} \right]
	={\bf A}\left[ \begin{array}{r} V_2 \\ -I_2\end{array} \right]
\]
Here $A_{11}$ and $A_{22}$ are dimensionless coefficients, $A_{12}$ is impedance 
and $A_{21}$ is admitance. A negative sign is added to the output current $I_2$ 
in the model, so that the direction of the current is out-ward, for easy
analysis of a cascade of multiple network models.

{\bf H-model:} In the H-model or hybrid model, we assume $V_2$ and $I_1$ are 
known, and find $V_1$ and $I_2$ by:
\[\left\{ \begin{array}{l} 
	V_1=H_{11}I_1+H_{12}V_2 \\
	I_2=H_{21}I_1+H_{22}V_2 \end{array} \right.\;\;\;\;\;
	\left[ \begin{array}{l} V_1 \\ I_2\end{array} \right]=
	\left[ \begin{array}{ll} H_{11} & H_{12}\\H_{21} & H_{22}\end{array}\right]
	\left[ \begin{array}{l} I_1 \\ V_2\end{array} \right]
	={\bf H}\left[ \begin{array}{l} I_1 \\ V_2\end{array} \right]
\]
Here $H_{12}$ and $H_{21}$ are dimensionless coefficients, $H_{11}$ is impedance
and $H_{22}$ is admitance. 

For each of the four types of models, the four parameters can be found
from variables $V_1$, $V_2$, $I_1$ and $I_2$ of a network by the following.
\begin{itemize}
\item For Z-model:
\[	Z_{11}=\frac{V_1}{I_1}|_{I_2=0},\;\;\;\;
	Z_{12}=\frac{V_1}{I_2}|_{I_1=0},\;\;\;\;
	Z_{21}=\frac{V_2}{I_1}|_{I_2=0},\;\;\;\;
	Z_{22}=\frac{V_2}{I_2}|_{I_1=0}
\]
\item For Y-model:
\[	Y_{11}=\frac{I_1}{V_1}|_{V_2=0},\;\;\;\;
	Y_{12}=\frac{I_1}{I_2}|_{V_1=0},\;\;\;\;
	Y_{21}=\frac{I_2}{I_1}|_{V_2=0},\;\;\;\;
	Y_{22}=\frac{I_2}{I_2}|_{Y_1=0},\;\;\;\;
\]
\item For A-model:
\[	A_{11}=\frac{V_1}{V_2}|_{I_2=0},\;\;\;\;
	A_{12}=\frac{V_1}{I_2}|_{V_2=0},\;\;\;\;
	A_{21}=\frac{I_1}{V_2}|_{I_2=0},\;\;\;\;
	A_{22}=\frac{I_1}{I_2}|_{V_2=0}
\]
\item For H-model:
\[	H_{11}=\frac{V_1}{I_1}|_{V_2=0},\;\;\;\;
	H_{12}=\frac{V_1}{V_2}|_{I_1=0},\;\;\;\;
	H_{21}=\frac{I_2}{I_1}|_{V_2=0},\;\;\;\;
	H_{22}=\frac{I_2}{V_2}|_{I_1=0}
\]
\end{itemize}

If we further define
\[	{\bf V}=[V_1, V_2]^T,\;\;\;\;\;\;{\bf I}=[I_1, I_2]^T	\]
then the Z-model and Y-model above can be written in matrix form:
\[ {\bf V}={\bf Z} {\bf I},\;\;\;\;\;\;{\bf I}={\bf Y} {\bf V},\;\;\;\;
	{\bf Y}={\bf Z}^{-1}		\]

{\bf Combinations of two-port models}

\htmladdimg{../figures/networkmodels1.gif}

{\bf Example:} The circuit shown below contains a two-port network with
$Z_{11}=4\Omega$, $Z_{12}=j3\Omega$, $Z_{21}=j3\Omega$ and $Z_{22}=2\Omega$.
The input voltage is $V_0=3\angle 0^\circ$ with an internal impedance 
$Z_0=5\Omega$ and the load impedance is $R_L=4\Omega$. Find the two voltages 
$V_1$, $V_2$ and two currents $I_1$, $I_2$.

\htmladdimg{../figures/networkexample1.gif}

{\bf Method 1:} 
\begin{itemize}
\item First, according the Z-model, we have
\[ \left\{ \begin{array}{l} V_1=Z_{11}I_1+Z_{12}I_2= 4I_1+j3I_2 \\
	V_2=Z_{21}I_1+Z_{22}I_2=j3I_1+ 2I_2 \end{array} \right.	\]
\item Second, two more equations can be obtained from the circuit:
\[ \left\{ \begin{array}{l} V_1=V_0-Z_0 I_1=3-5I_1 \\
	V_2=-Z_L I_2=-4 I_2 \end{array} \right.	\]
\item Substituting the last two equations for $V_1$ and $V_2$ into the 
	first two, we get
\[ \left\{ \begin{array}{l} 9I_1+j3I_2=3 \\ j3I_1+6I_2=0 \end{array} \right. \]
\item Solving these we get 
\[	I_1=\frac{2}{7},\;\;\;\;\;I_2=-\frac{j}{7} \]
\item Then we can get the voltages
\[	V_1=\frac{11}{7},\;\;\;\;\;V_2=\frac{j4}{7} \]
\end{itemize}

{\bf Method 2:} We can also use Thevenin's theorem to treat everything before
the load impedance as an equivalent voltage source with Thevenin's voltage
$V_T$ and resistance $R_T$, and the output voltage $_2$ and current $I_2$ can
be found.

\begin{itemize}
\item Find $Z_T=V_2/I_2$ with voltage $V_0$ short-circuit:
  \begin{itemize}
  \item The Z-model:
    \[ \left\{ \begin{array}{l} V_1=Z_{11}I_1+Z_{12}I_2=4I_1+j3I_2	\\
      V_2=Z_{21}I_1+Z_{22}I_2=j3I_1+2I_2 \end{array} \right. \]
  \item Also due to the voltage source $V_0$, we have
    \[ V_1=V_0-I_1 Z_0=0-5I_1	\]
  \item equating the two expressions for $V_1$, we get
    \[ 4I_1+j3I_2=-5I_1,\;\;\;\;\mbox{i.e.}\;\;\;\;I_1=-\frac{j}{3} I_2	\]
  \item Substituting this $I_1$ into the equation for $V_2$ above, we get
    \[ V_2=j3I_1+2I_2=I_2+2I_2=3I_2 \]
  \item Find $Z_2=V_22/I_2$:
    \[ Z_T=\frac{V_2}{I_2}=3	\]
  \end{itemize}
\item Find open-circuit voltage $V_T$ with $I_2=0$:
  \begin{itemize}
  \item Since the load is an open-circuit, $I_2=0$, we have
    \[ \left\{ \begin{array}{l} V_1=Z_{11}I_1=4I_1 \\
      V_2=Z_{21}I_1=j3 I_1 \end{array} \right. \]
    \item Find $I_1$:
      \[	I_1=\frac{V_0-V_1}{Z_0}=\frac{3-4I_1}{5}	\]
      Solving this to get $I_1=1/3$
    \item Find open-circuit voltage $V_T=V_2$:
      \[	V_2=Z_{21}I_1=j3 \frac{1}{3}=j	\]
  \end{itemize}
\item Find load voltage $V_2$:
  \[ V_2=I_2 Z_L=\frac{V_T}{Z_T+Z_L}\;Z_L=\frac{j}{3+4}\;4=\frac{j4}{7} \]
\item Find load voltage $I_2$:
  \[ I_2=-\frac{V_2}{Z_L}=-\frac{j4}{7}\;\frac{1}{4}=-\frac{j}{7} \]
\end{itemize}


{\bf principle of reciprocity}:
 
\htmladdimg{../figures/reciprocal.gif}

Consider the example circuit on the left above, which can be simplified 
as the network in the middle. The voltage source is in the branch on the
left, while the current $\dot{I}_2$ is in the branch on the right, which
can be found to be (current divider):
\[ \dot{I}_2=\frac{V}{Z_1+Z_2 Z_3/(Z_2+Z_3)}\;\frac{Z_3}{Z_2+Z_3}
=V \frac{Z_3}{Z_1Z_2+Z_1Z_3+Z_2Z_3} \]
We next interchange the positions of the voltage source and the current, 
so that the voltage source is in the branch on the right and the current 
to be found is in the branch on the left, as shown on the right of the 
figure above. The current can be found to be
\[ \dot{I}_1=\frac{V}{Z_2+Z_1 Z_3/(Z_1+Z_3)}\frac{Z_3}{Z_1+Z_3}
=V \frac{Z_3}{Z_1Z_2+Z_1Z_3+Z_2Z_3} \]
The two currents $\dot{I}_1$ and $\dot{I}_2$ are exactly the same! This
result illustrates the following priciple, which can be proven in general:

{\em 
In any passive (without energy sources), 
linear network, if a voltage $V$ applied in branch 1 causes a current $I$ in 
branch 2, then this voltage $V$ applied in branch 2 will cause the same current
$I$ in branch 1.}

{\em In any passive, inear network, the transfer impedance $Z_{12}$ is equal 
to the reciprocal transfer impedance $Z_{21}$.}

Based on this reciprocity principle, any complex passive linear network can
be modeled by either a T-network or a $\Pi$-network:

{\bf T-Network Model:}

\htmladdimg{../figures/Tmodel.gif}

From this T-model, we get
\[	\left\{ \begin{array}{l} V_1=(Z_1+Z_3)I_1+Z_3I_2=Z_{11}I_1+Z_{12}I_2 \\
		V_2=Z_3I_1+(Z_2+Z_3)I_2=Z_{21}I_1+Z_{22}I_2  \end{array} \right. \]
Comparing this with the Z-model, we get
\[ Z_{11}=Z_1+Z_3,\;\;\;\;\;Z_{22}=Z_2+Z_3,\;\;\;\;\;Z_{12}=Z_{21}=Z_3	\]
Solving these equations for $Z_1$, $Z_2$ and $Z_3$, we get
\[ Z_1=Z_{11}-Z_{12},\;\;\;\;Z_2=Z_{22}-Z_{21},\;\;\;\;Z_3=Z_{12}=Z_{21} \]

{\bf $\Pi$-Network Model:}
\htmladdimg{../figures/Pmodel.gif}

From this $\Pi$-model, we get:
\[ \left\{ \begin{array}{l} 
I_1=Y_1V_1+Y_3(V_1-V_2)=(Y_1+Y_3)V_1-Y_3V_2=Y_{11}V_1+Y_{12}V_2 \\
I_2=Y_3(V_2-V_1)+Y_2V_2=-Y_3V_1+(Y_2+Y_3)V_2=Y_{21}V_1+Y_{22}V_2 \end{array} \right. \]
Comparing this with the Z-model, we get
\[ Y_{11}=Y_1+Y_3,\;\;\;\;\;Y_{22}=Y_2+Y_3,\;\;\;\;\;Y_{12}=Y_{21}=-Y_3	\]
Solving these equations for $Y_1$, $Y_2$ and $Y_3$, we get
\[ Y_1=Y_{11}+Y_{12},\;\;\;\;Y_2=Y_{22}+Y_{21},\;\;\;\;Y_3=-Y_{12}=-Y_{21} \]

{\bf Example 1:} Convert the given T-network to a $\Pi$ network.

\htmladdimg{../figures/TPmodelexample.gif}

{\bf Solution:} Given $Z_1=j5$, $Z_2=-j5$, $Z_3=1$, we get its Z-model:
\[  Z_{11}=Z_1+Z_3=1+j5,\;\;\;\;Z_{22}=Z_2+Z_3=1-j5,\;\;\;\;Z_{12}=Z_{21}=Z_3=1 \]
The Z-model can be expressed in matrix form:
\[	\left[ \begin{array}{l} V_1 \\ V_2\end{array} \right]=
	\left[ \begin{array}{cc} Z_{11} & Z_{12} \\ Z_{21} & Z_{22} \end{array} \right]
	\left[ \begin{array}{l} I_1 \\ I_2\end{array} \right]
=	\left[ \begin{array}{cc} 1+j5 & 1 \\ 1 & 1-j5 \end{array} \right]
	\left[ \begin{array}{l} I_1 \\ I_2\end{array} \right]
\]
This Z-model can be converted into a Y-model:
\[	\left[ \begin{array}{l} I_1 \\ I_2\end{array} \right]=
	\left[ \begin{array}{cc} 1+j5 & 1 \\ 1 & 1-j5 \end{array} \right]^{-1}
	\left[ \begin{array}{l} V_1 \\ V_2\end{array} \right]
=\frac{1}{25}
	\left[ \begin{array}{cc} 1-j5 & -1 \\ -1 & 1+j5 \end{array} \right]
	\left[ \begin{array}{l} V_1 \\ V_2\end{array} \right]
=	\left[ \begin{array}{cc} Y_{11} & Y_{12} \\ Y_{21} & Y_{22} \end{array} \right]
	\left[ \begin{array}{l} V_1 \\ V_2\end{array} \right]
\]
This Y-model can be converted to a $\Pi$ network:
\[ Y_1=Y_{11}+Y_{12}=-j/5,\;\;\;\;Z_2=Y_{22}+Y_{21}=j/5,\;\;\;\;Y_3=Y_{21}=Y_{12}=1/25 \]
These admitances can be further converted into impedances:
\[ Z_1=1/Y_1=5j,\;\;\;\;Z_2=1/Y_2=-j5,\;\;\;\;Z_3=1/Y_3=25	\]

\subsection*{Ideal Transformer}

\htmladdimg{../figures/idealtransformer.gif}

The ratio between the primary voltage $V_1$ and the secendary voltage $V_2$ of 
a transformer is proportional to the ratio between the number of turns:
\[	\frac{V_2}{V_1}=\frac{n_2}{n_1}	\]
If there is no power loss by the transformer, then the transformer is ideal and
we have
\[P_1=V_1I_1=P_2=V_2I_2,\;\;\;\;\mbox{i.e.}\;\;\;\;
	\frac{I_2}{I_1}=\frac{V_1}{V_2}=\frac{n_1}{n_2}	\]
The ratio between the primary current $I_1$ and the secendary current $I_2$ of 
a transformer is inversely proportional to the ratio between the number of turns.
Also note the direction of the current.

We can also find the ratio between the primary and secondary impedances based on 
the assumption that there is no power loss in the transformer, i.e.,
\[	P_1=\frac{V_1^2}{Z_1}=P_2=\frac{V_2^2}{Z_2}, \;\;\;\;\;
	\frac{Z_1}{Z_2}=(\frac{V_1}{V_2})^2=(\frac{n_1}{n_2})^2  \]

{\bf Example 1:} Assume $R_0=1000\Omega$, $R_L=10\Omega$, and the voltage souce is
$V=12V$. Find the turn ratio of the transformer so that the load resistor will get 
maximum power from the voltage source.

\htmladdimg{../figures/transformerexample1.gif}

We know that when the load resistor will get maximum power if its resistance is 
equal to the internal resistance of the voltage source. 
\begin{itemize}
\item Convert load resistance on the secondary side $R_L=10\Omega$ to 
	$R'_L=n^2\; R_L=10\; n^2$ on the primary side.
\item Match $R'_L=10\;n^2$ to $R_0$, i.e., $10\; n^2=1000\Omega$
\item Find turn ratio $n=N_1/N_2=10$.
\item Find power consumption on $R'_L$: 
	\[ P=(\frac{V}{2})^2/R'_L=36/1000=0.036W	\]
\item The power consumed by real load $R_L$ is the same, due to ideal transformer.
\end{itemize}

{\bf Example 2:} Assume $R_0=10\Omega$, $R_L=5\Omega$, and the turn ratio is
$n=n_1/n_2=2$. Describe this circuit as a two-port network.

\htmladdimg{../figures/transformerexample2.gif}

\begin{itemize}
\item Set up basic equations:
\[	\left\{ \begin{array}{l} 
	V_1=10\;I_1+2\;V_2 \\ I_2-V_2/5=-2\;I_1 \end{array} \right. \]
\item Rearrange the equations in the form of a Z-model. The second equation is
\[	V_2=10\;I_1+5\;I_2	\]
Substituting into the first equation, we get
\[	V_1=30\;I_1+10\I_2	\]
The Z-model is:
\[	\left[ \begin{array}{l} V_1 \\ V_2 \end{array} \right]=
	\left[ \begin{array}{rr} Z_{11} & Z_{12} \\ Z_{21} & Z_{22} \end{array} \right]
	\left[ \begin{array}{l} I_1 \\ I_2 \end{array} \right]
=	\left[ \begin{array}{rr} 30 & 10 \\ 10 & 5 \end{array} \right]
	\left[ \begin{array}{l} I_1 \\ I_2 \end{array} \right]
\]
As $Z_{12}=Z_{21}=10$, this is a reciprocal network.
\end{itemize}
Alternatively, we can set up the equations in terms of the currents:
\begin{itemize}
\[	\left\{ \begin{array}{l} 
	I_1=(V_1-2\;V_2)/10 \\ I_2=-2\;I_1+V_2/5 \end{array} \right. \]
\item Rearrange the equations in the form of a Y-model. The first equation is
\[	I_1=V_1/10-V_2/5	\]
Substituting into the second equation, we get
\[	I_2=-V_1/5+2V_2/5+V_2/5=-V_1/5+3V_2/5	\]
The Y-model is:
\[	\left[ \begin{array}{l} I_1 \\ I_2 \end{array} \right]=
	\left[ \begin{array}{rr} Y_{11} & Y_{12} \\ Y_{21} & Y_{22} \end{array} \right]
	\left[ \begin{array}{l} V_1 \\ V_2 \end{array} \right]
=	\left[ \begin{array}{rr} 1/10 & -1/5 \\ -1/5 & 3/5 \end{array} \right]
	\left[ \begin{array}{l} V_1 \\ V_2 \end{array} \right]
\end{itemize}
Finally, we can verify that ${\bf Z}^{-1}={\bf Y}$
\[ {\bf Z}^{-1}=\left[ \begin{array}{rr} 30 & 10 \\ 10 & 5 \end{array} \right]^{-1}
	=\frac{1}{50}\left[ \begin{array}{rr} 5 & -10 \\ -10 & 30 \end{array} \right]
	=\left[ \begin{array}{rr} 1/10 & -1/5 \\ -1/5 & 3/5 \end{array} \right]
	={\bf Y}	\]

\end{document}


	

	










