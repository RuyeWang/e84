\documentstyle[12pt]{article}
\usepackage{html}

\begin{document}

\begin{itemize}

\item {\bf Example 0:}

  \htmladdimg{../figures/sourcemeter.gif}

  The voltage measured by a voltmeter is (voltage divider):
  \[ V=V_s \frac{R_v}{R_s+R_v} \]
  It is desireable for $R_v$ to be as large as possible (ideally infinity) so 
  that it draws little current from the source (causing little voltage drop 
  across the internal resistance $R_s$), for the measureed voltage to be 
  close to the true open-circuit voltage $V_{oc}$, i.e., the source $V_s$. 

  The current measured by an ammeter is:
  \[ I=\frac{V_s}{R_s+R_a} \]
  It is desireable for $R_a$ to be as small as possible (ideally zero)
  so that the measured current is close to true short-circuit current 
  $I_{sc}=V_s/R_s$.

  When $R_v<\infty$ and $R_a>0$, the true $V_s$ and $R_s$ can be found by
  solving these equations:
  \[ \left\{ \begin{array}{l}
     V=V_s R_v/(R_s+R_v) \\I=V_s/(R_s+R_a) \end{array} \right. 
  \;\;\;\;\;\;\mbox{i.e.}\;\;\;\;\;\;
  \left\{ \begin{array}{l}
    R_vV_s-VR_s=VR_v \\ V_s-IR_s=I R_a \end{array} \right. \]
  solving these we get:
  \[ \left\{ \begin{array}{l}
    V_s=IV(R_v-R_a)/(IR_v-V) \\
    R_s=(V-IR_a)R_v/(IR_v-V)
  \end{array} \right. \]

  Given the specific values, we get
  \[ I=\frac{R_s}{R_s+R_a}=\frac{6V}{300\Omega}=0.02A \]
  \[ V=V_s \frac{R_v}{R_s+R_v}=6V \frac{10,000}{10,000+100}=5.94 V \]
  From these measurement, the true $R_s$ and $V_s$ can be obtained as:
  \[ V_s=\frac{IV(R_v-R_a)}{IR_v-V}
  =\frac{5.94\times 0.02(10,000-200)}{0.02\times 10,000-5.94}=100 \]
  \[ R_s=\frac{(V-IR_a)R_v}{IR_v-V} 
  =\frac{(5.94-0.02\tiimes 200)\times 10,000}{0.02\times 10,000-5.94}=6 \]
  Method verified.

\begin{comment}
\item {\bf Example 1:}

  The load voltage is related to the voltage source by
  \[ V_L=V_s \frac{R_L}{R_L+R_s} \]
  i.e., 
  \[ R_L V_s-V_L R_s = V_L R_L \]
  Using the values of $R_L$ and $V_L$ of the two experiments, we get
  \[ \left\{ \begin{array}{l} 
    1000 V_s-9.09 R_s = 9,090 \\
    2000 V_s-9.52 R_s =19,040 \end{array} \right. \]
  Solving these two equations we get
  \[ \left\{ \begin{array}{l} 
    R_s=100\Omega \\ V_s=10 V \end{array} \right. \]
\end{comment}

\end{itemize}

\end{document}

