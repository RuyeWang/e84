\documentstyle[12pt]{article}
\usepackage{html}

\begin{document}

\begin{itemize}

\item {\bf Example 0:}

  \htmladdimg{../figures/sourcemeter.gif}


  \begin{itemize}

  \item The voltage measured by a voltmeter is (voltage divider):
    \[ V=V_0 \frac{R_v}{R_0+R_v} \]
    It is desireable for $R_v$ to be as large as possible (ideally infinity) so 
    that it draws little current from the source (causing little voltage drop 
    across the internal resistance $R_0$), for the measureed voltage to be 
    close to the true open-circuit voltage $V_{oc}$, i.e., the source $V_0$. 

  \item  The current measured by an ammeter is:
    \[ I=\frac{V_0}{R_0+R_a} \]
    It is desireable for $R_a$ to be as small as possible (ideally zero)
    so that the measured current is close to true short-circuit current 
    $I_{sc}=V_0/R_0$.
  \end{itemize}

  When $R_v<\infty$ and $R_a>0$, the true $V_0$ and $R_0$ can be found by
  solving these equations:
  \[ \left\{ \begin{array}{l}
     V=V_0 R_v/(R_0+R_v) \\I=V_0/(R_0+R_a) \end{array} \right. 
  \;\;\;\;\;\;\mbox{i.e.}\;\;\;\;\;\;
  \left\{ \begin{array}{l}
    R_vV_0-VR_0=VR_v \\ V_0-IR_0=I R_a \end{array} \right. \]
  Solving these we get:
  \[ \left\{ \begin{array}{l}
    V_0=IV(R_v-R_a)/(IR_v-V) \\
    R_0=(V-IR_a)R_v/(IR_v-V)
  \end{array} \right. \]

  Given the specific values: $V_0=6V, R_0=100\Omega, R_v=10,000 \Omega, R_a=200 \Omega$,
  find the measured open-circuit voltage $V$, and the short-circuit current 
  $I$. 

  Given $V$, $I$, and the known $R_v$ and $R_a$, how do you get the true 
  $V_0$ and $R_0$ using your method above? Show your numerical computations.

  we get
  \[ I=\frac{R_0}{R_0+R_a}=\frac{6V}{300\Omega}=0.02A \]
  \[ V=V_0 \frac{R_v}{R_0+R_v}=6V \frac{10,000}{10,000+100}=5.94 V \]
  From these measurement, the true $R_0$ and $V_0$ can be obtained as:
  \[ V_0=\frac{IV(R_v-R_a)}{IR_v-V}
  =\frac{5.94\times 0.02(10,000-200)}{0.02\times 10,000-5.94}=100 \]
  \[ R_0=\frac{(V-IR_a)R_v}{IR_v-V} 
  =\frac{(5.94-0.02\tiimes 200)\times 10,000}{0.02\times 10,000-5.94}=6 \]
  Method verified.

\begin{comment}
\item {\bf Example 1:}

  The load voltage is related to the voltage source by
  \[ V_L=V_0 \frac{R_L}{R_L+R_0} \]
  i.e., 
  \[ R_L V_0-V_L R_0 = V_L R_L \]
  Using the values of $R_L$ and $V_L$ of the two experiments, we get
  \[ \left\{ \begin{array}{l} 
    1000 V_0-9.09 R_0 = 9,090 \\
    2000 V_0-9.52 R_0 =19,040 \end{array} \right. \]
  Solving these two equations we get
  \[ \left\{ \begin{array}{l} 
    R_0=100\Omega \\ V_0=10 V \end{array} \right. \]
\end{comment}

\end{itemize}

\end{document}

