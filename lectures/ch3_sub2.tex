\documentstyle[12pt]{article}
\usepackage{html}

\begin{document}

\htmladdimg{../figures/BridgeCapacitor.png}
%\htmladdimg{../figures/ch3Ex0.png}

Consider node voltage method. Applying KCL to node $c$ 
we get
\[
\frac{V_0-v_c(t)}{R_1}=C\dot{v}_c(t)+\frac{v_c(t)}{R_2}
\]
i.e.,
\[
\frac{R_1R_2}{R_1+R_2}\,C\dot{v}_c(t)+v_c(t)
=\tau\dot{v}_c(t)+v_c(t)=V_0\frac{R_2}{R_1++R_2}
\]
where 
\[
\tau=C\frac{R_1R_2}{R_1+R_2}=C(R_1||R_2)
=0.5\times 10^{-6}\times 10^3=5\times 10^{-4}
\]
The initial condition for the DE above is $v_c(0)=-V_0/2$, the
homogeneous solution is $v_h(t)=Ae^{-t/\tau}$ and the particular 
(steady state) solution is $v_p(t)=V_0/2$. The complete solution 
is 
\[
v_c(t)=v_h(t)+v_p(t)=Ae^{-t/\tau}+\frac{V_0}{2}
\]
To find $A$, we evaluate $v_c(t)$ at $t=0$ and equate that to the
known initial condition $v_c(0)$ to get
\[
v_c(t)\bigg|_{t=0}=V_s+Ae^{-t/\tau}\bigg|_{t=0}=\frac{V_0}{2}+A
=v_c(0)=-\frac{V_0}{2}
\]
i.e., $A=-V_0$. Now the solution is
\[
v_c(t)=Ae^{-t/\tau}+\frac{V_0}{2}
=\frac{V_0}{2}-V_0 e^{-t/\tau}=5-10\,e^{-2000\,t}
\]

\end{document}

