\documentstyle[12pt]{article}
\usepackage{html}
\textwidth 6.0in
\topmargin -0.5in
\oddsidemargin -0in
\evensidemargin -0.5in
% \usepackage{graphics}  
\begin{document}

\section*{Chapter 5: Operational Amplifiers (Op-amps)}

\subsection*{Operational Amplifier}

{\bf Circuit Schematic}

\htmladdimg{../figures/opamp741.gif}

Although the schematic of an Op-amp looks complicated, its operation can be
simply described as
\[	v_{out}=A (v^+ - v^-)	\]
and it can be modeled by a simple voltage amplifier with three parameters
including input resistance $r_{in}$ (usually large), output resistance $r_{out}$
(usually small), and the open-circuit gain $A$ (usually very large), as show
in the figure:

\htmladdimg{../figures/opamp0.gif}

We consider the following two typical Op-Amp circuits to show how to carry out
circuit analysis.

\begin{itemize}

\item {\bf Voltage follower:} The input is connected to the positive input
  while the output is directly connected to the negative input (100\%
  negative feedback). The parameters of this circuit can be found by the
  model shown in the figure.

  \htmladdimg{../figures/voltagefollowermodel.gif}

  \begin{itemize}
    \item {\bf Open-circuit voltage gain:} Assume an ideal source voltage $v_s$ 
      ($R_s=0$) is applied to the input of the circuit and the output port is open
      circuit $R_L=\infty$. Then applying KVL to the loop, we get
      \[ v_s=(r_{in}+r_{out})i_{in}+Ar_{in}i_{in}=[(A+1)r_{in}+r_{out}]i_{in} \]
      Note that the internal voltage source is $A(v^+-v^-)=r_{in}i_{in}$. 
      The output voltage is:
      \[ v_{out}=Ar_{in}i_{in}+r_{out}i_{in} \]
      and the open-circuit voltage gain is:
      \[ A_{oc}=\frac{v_{out}}{v_s}=\frac{Ar_{in}+r_{out}}{(A+1)r_{in}+r_{out}} \]
      Since $A>>1$, $A_{oc}\approx 1$ is approximately unity.
    \item {\bf Input resistance:} 
      This time a load $R_L$ is connected to the output port, and applying KVL to
      the two loops we get:
      \[ v_s=(r_{in}+r_{out})i_{in}-r_{out}i_{out}+A r_{in}i_{in} \]
      \[ A r_{in}i_{in}=(r_{out}+R_L)i_{out}-r_{out}i_{in} \]
      Solving these two equations for $i_{in}$ and $i_{out}$ we get
      \[ i_{in}=v_s/[(A+1)r_{in}+r_{out}-r_{out}\frac{Ar_{in}+r_{out}}{R_L+r_{out}}] \]
      and the input resistance is
      \[ R_{in}=\frac{v_s}{i_{in}}=
	 (1+A)r_{in}+r_{out}-r_{out}\frac{Ar_{in}+r_{out}}{R_L+r_{out}}
      =r_{in}\frac{r_{out}+(A+1)R_L}{R_L+r_{out}}+\frac{R_L r_{out}}{R_L+r_{out}}
      =r_{in}\frac{r_{out}+(A+1)R_L}{R_L+r_{out}}+R_L|| r_{out}      \]
      Note that $R_{in}$ is affected by the load $R_L$. As usually $R_L >> r_{out}$
      and $r_{in}>>r_{out}$, the above is approximately
      \[ R_{in}\approx Ar_{in} \]
    \item {\bf Output resistance:} 
      This can be obtained as the ratio between the open-circuit voltage $v_{oc}$ 
      and the short-circuit current $i_{sc}$
      \[ R_{out}=v_{oc}/i_{sc} \]
      First, find open-circuit voltage $v_{oc}$ (with $R_L=\infty$ and $i_{out}=0$):
      \[ v_s=(R_s+r_{in}+r_{out})i_{in}+Ar_{in}i_{in},\;\;\;\mbox{i.e.,}\;\;\;\;
      i_{in}=\frac{v_s}{R_s+(A+1)r_{in}+r_{out}}      \]
      The voltage appearing at the output port is
      \[ v_{oc}=Ar_{in}i_{in}+r_{out}i_{in}
      =v_s\frac{Ar_{in}+r_{out}}{R_s+(A+1)r_{in}+r_{out}} \]
      Second, find short-circuit current $i_{sc}=i_{out}$ (with $R_L=0$ and 
      $v_{oc}=0$) by applying KVL to the two loops:
      \[ v_s=(R_s+r_{out}+(A+1)r_{in})i_{in}-r_{out} i_{out} \]
      \[ Ar_{in}i_{in}=r_{out} i_{out}-r_{out} i_{in} \]
      Solving the 2nd equation we get:
      \[ i_{out}=\frac{Ar_{in}+r_{out}}{r_{out}} i_{in} \]
      Substituting this into the 1st equation we can solve for $i_{in}$:
      \[ i_{in}=\frac{v_s}{R_s+r_{in}} \]
      and the output current is:
      \[ i_{sc}=i_{out}=v_s\frac{Ar_{in}+r_{out}}{r_{out}(R_s+r_{in})} \]
      Now the output resistance can be found:
      \[ R_{out}=\frac{v_{oc}}{i_{sc}}
      =\frac{r_{out}(R_s+r_{in})}{R_s+(A+1)r_{in}+r_{out}} \]
      Note that $R_{out}$ is affected by the internal resistance $R_s$ of the
      sourc. As usually $(A+1)r_{in}>>R_s+r_{out}$ and $r_{in}>>R_s$, we have
      \[ R_{out}\approx \frac{r_{out}}{A} \]
  \end{itemize}
  To summarize, we see that the voltage follower has unit voltage gain, but
  much increased input resistance $R_{in}\approx A r_{in}$ and much reduced
  output resistance $R_{out}\approx r_{out}/A$. Typically, $R_{in}$ is $10^{10}
  \Omega$ and $R_{out}$ is $10^{-3} \Omega$.

  {\bf Example:} 

  \htmladdimg{../figures/opampbuffer.gif}

  The figure on the left shows a circuit represented by an ideal voltate
  source $V_s$ in series with an internal resistance $R_s$ (Thevenin's
  theorem), with a load $R_L$. The voltage delivered to the load by this
  non-ideal source is
  \[ v_{out}=v_s \frac{R_L}{R_L+R_s} \]
  The output voltage across the load is only a fraction of the voltage due to 
  the voltage drop across the internal resistance $R_s$. If it is desired for
  the output voltage to be as close to the source as possible, the internal 
  resistance $R_s$ has to be small while the load resistance $R_L$ has to be
  large (lighter load). However, given $R_s$ and $R_L$, it is still possible
  for the output voltage to be very close to the source if a voltage follower
  is used as a buffer between the source and the load, as shown in the figure
  on the right. The voltage follower is modeled by its input and output
  resistances $R_{in}$ and $R_{out}$, as well as its voltage gain $A_{oc}$,
  and the output voltage can be obtained after two levels of voltage dividers:
  \[ v_{out}=A_{oc} v_{in} \frac{R_L}{R_{out}+R_L}=
  A_{oc} v_s \frac{R_{in}}{R_s+R_{in}}   \frac{R_L}{R_{out}+R_L} \]
  As $R_{in}$ is huge, $R_{in}/(R_s+R_{in})\approx 1$, also, as $R_{out}$
  is very small, $R_L/(R_{out}+R_L) \approx 1$, and $A_{oc} \approx 1$,
  therefore $v_{out} \approx v_s$, i.e., one hundred percent of the source
  voltage is delivered to the load.  

\item {\bf Inverting Amilifier}

  \htmladdimg{../figures/inverteramplifier.gif}

  The analysis of the circut using full model of the op-amp may be very involved.
  To simplify the analysis below we assume $r_{in}=\infty$ and $A>>1$.

  \begin{itemize}
    \item {\bf Open-circuit voltage gain:} We apply an ideal voltage source $v_s$ 
      ($R_s=0$) to the negative input node and find the input current to be:
      \[ i_{in}=\frac{v_s+Av_{in}}{R_1+R_f+r_{out}}
      =\frac{v_s+A(v_s-R_1 i_{in})}{R_1+R_f+r_{out}} \]
      where $v_{in}=v_s-R_1 i_{in}$. Solving this for $i_{in}$, we get:
      \[ i_{in}=v_s \frac{A+1}{(A+1)R_1+R_f+r_{out}} \]
      Now the output voltage can be found to be:
      \[ v_{out}=v_s-(R_1+R_f) i_{in}
      =v_s[1-\frac{(A+1)(R_1+R_f)}{(A+1)R_1+R_f+r_{out}}] 
      =v_s\frac{r_{out}-AR_f}{(A+1)R_1+R_f+r_{out}} \] 
      i.e., 
      \[ A_{oc}=\frac{v_{out}}{v_s}=\frac{r_{out}-AR_f}{(A+1)R_1+R_f+r_{out}} 
      \approx - \frac{R_f}{R_1} \]
      Alternatively, if we further approximate $v^-\approx v^+=0$, then we can 
      easily find $v_{out}$ given input $V_s$ by applying KCL at negative input 
      node:
      \[ \frac{v_s}{R_1}+\frac{v_{out}}{R_f} = 0 \]
      i.e., the open-circuit voltage gain is
      \[ A_{oc}=-\frac{R_f}{R_1} \]

    \item {\bf Input resistance:} Again apply an ideal voltage source $v_s$ to
      the circuit and the input resistance $R_{in}$ is found as the ratio of
      $v_s$ to the input current $i_{in}$. Also, to simplify the analysis, we 
      first find out the input resistance $R'_{in}$ of the circuit with $R_1=0$, 
      the actual input resistance can be found to be $R_{in}=R_1+R'_{in}$. 

      Applying KCL to the negative input node, we get
      \[ i_{in}=\frac{v_{in}}{r_{in}}+\frac{v_{in}-(-Av_{in})}{r_{out}+R_f}
      =v_{in}[ \frac{1}{r_{in}}+\frac{1+A}{r_{out}+R_f}] \]
      i.e., 
      \[ R'_{in}=\frac{v_{in}}{i_{in}}=r_{in} || (r_{out}+R_f)/(A+1) 
      \approx \frac{r_{out}+R_f}{A+1} \]
      The approximation is because usually $r_{in}$ is large and so is $A$.
      Adding $R_1$ back, we get the overall input resistance:
      \[ R_{in}=R_1+\frac{r_{out}+R_f}{A+1} \approx R_1 \]
      This result could also be obtained directly by inspection, due to the
      assumption that $v^-\approx v^+=0$, therefore $R_{in}=R_1$.

    \item {\bf Output resistance:} First, consider short-circuit output current:
      \[ i_{sc}=\frac{-A v_{in}}{r_{out}}+\frac{v_{in}}{R_f}
      =v_{in}\frac{r_{out}-AR_f}{r_{out}R_f}     \]
      but as
      \[ v_{in}=v_s\frac{R_f}{R_s+R_1+R_f} \]
      we have
      \[ i_{sc}=v_{in}\frac{r_{out}-AR_f}{r_{out}R_f}  
      =v_s\frac{R_f}{R_s+R_1+R_f} \frac{r_{out}-AR_f}{r_{out}R_f} 
      =v_s\frac{r_{out}-AR_f}{(R_s+R_1+R_f)r_{out}}  \]
      Next we find the open-circuit output voltage:
      \[ v_{oc}=\frac{v_{in}-(-Av_{in})}{R_f+r_{out}} r_{out}-Av_{in} 
      =v_{in} \frac{r_{out}-AR_f}{R_f+r_{out}} \]
      But $v_{in}$ is related to $v_s$ by:
      \[ \frac{v_s-v_{in}}{R_s+R_1}=\frac{v_{in}-(-Av_{in})}{R_f+r_{out}} \]
      which can be solved for $v_{in}$
      \[ v_{in}=v_s \frac{R_f+r_{out}}{(A+1)(R_s+R_1)+R_f+r_{out}} \]
      i.e.,
      \[ v_{oc}=v_{in} \frac{r_{out}-AR_f}{R_f+r_{out}}
      =v_s \frac{r_{out}-AR_f}{(A+1)(R_s+R_1)+R_f+r_{out}} \]
      Now we get
      \[ R_{out}=\frac{v_{oc}}{i_{sc}}
      =\frac{(R_s+R_1+R_f)r_{out}}{(A+1)(R_s+R_1)+R_f+r_{out}}
      \approx \frac{(R_s+R_1+R_f)r_{out}}{A(R_s+R_1)} \]
      The approximation is due to the assumption that $A>>1$. In particular, when
      $R_s=0$, we have
      \[ R_{out}\approx \frac{R_1+R_f}{R_1} \frac{r_{out}}{A} \]
  \end{itemize}

\item {\bf Non-Inverting Amilifier}

  \htmladdimg{../figures/noninverteramplifier.gif}

  {\bf Homework problem:}

  Find the three parameters of this non-inverting amplifier: open-circuit 
  voltage gain $A_{oc}$, input resistance $R_{in}$ and output resistance $R_{out}$.
  Assume $r_{in}=\infty$ and $R_L=\infty$. Note that when $R_f=0$, this 
  non-inverting amplifier will become a voltage follower, in terms of all 
  three parameters. Verify your results by checking if this is the case.

  \htmladdnormallink{Answer}{../noninvertingopamp/index.html}

\end{itemize}


\subsection*{Op-Amp Circuits }

The analysis of various op-amp circuits can be further simplified by the following 
approximation:
\begin{itemize}
\item The huge voltage gain $A$ ($10^5 \sim 10^9$) is considered to be infinity, 
  i.e., for a finite output $v_{out}$, the input is close to zero 
  $v_{in}=v^+ - v^-\approx 0$ or $v^+=v^-$; 
\item The huge input resistance $r_{in}$ ($10^6\sim 10^{12} \;\Omega$ depending
  on whether BJT or FET is used) could also be considered to be infinity 
  $r_{in}\rightarrow \infty$;
\item The small input current drawn by an op-amp ($10^{-9}\sim10^{-12}\;A$) 
  is approximately zero $i^+=i^-=0$;
\item The small output impedance $r_{out}$ (50 $\Omega$) is approximately zero 
  $r_{out}\approx 0$, i.e., the output $v_{out}$ is not affected by the load 
  $R_L$ (so long as it is much greater than 50 $\Omega$).
\item The bandwidth is large ($1 \sim 20MHz$).
\end{itemize}

\htmladdimg{../figures/opam1.gif}

Based on these approximations, the analysis of op-amp circuits can be much
simplified than before, as shown in the following examples.

\begin{itemize}
\item {\bf Inverter}

\htmladdimg{../figures/opam2.gif}

As the input current is negligible, i.e., $i^-=i_1+i_2 \approx 0$, we have
\[
	i_1+i_2= \frac{v^- -v_1}{R_1}+\frac{v^- -v_o}{R_f}=0
\]
but also since
\[	v^- \approx v^+ = 0	\]
we have
\[	\frac{v_o}{v_1}=-\frac{R_f}{R_1}	\]
In particular, if $R_f=R_1$, we have
\[	v_o=-v_1	\]

In general, $R_1$ and $R_2$ in the inverter can be replaced by two networks
(with impedances $Z_1$ and $Z_2$ respectively) containing resistors and capacitors 
and the analysis of the circuit can be carried out easily in frequency domain:
\[
H(j\omega)=\frac{V_o(j\omega)}{V_i(j\omega)}=-\frac{Z_2(j\omega)}{Z_1(j\omega)}	
\]
This is a convenient way to design filters of various frequency characteristics.

\htmladdimg{../figures/opam11.gif}

\item {\bf Summer-inverter}

\htmladdimg{../figures/opam3.gif}

As the input current $i$ is negligible, we have
\[ i=\sum_{j=1}^n \frac{v^--v_j}{R_j}+\frac{v^--v_o}{R_f} \approx 0	\]
and we also know
\[	v^- \approx v^+= 0 \]
we have
\[
	v_o=-R_f \sum_{j=1}^n \frac{v_j}{R_j} = - \sum_{j=1}^n k_j v_j \]
where $k_j \stackrel{\triangle}{=}R_f/R_j$ ($j=1,2,\cdots,n$) are the $n$ 
coefficients.

\item {\bf Summer with different signs}

\htmladdimg{../figures/opam5.gif}

Define $v$ as
\[	v\stackrel{\triangle}{=}v^+ \approx v^- \]
and note
\[	i_1 \approx 0 \;\;\;\;\mbox{   }\;\;\; i_2 \approx 0	\]
we have these simultaneous equations
\[
\left\{ \begin{array}{ll} 
	(v-v_1)/R_1+(v-v_o)/R_f=0 & (a) \\
	(v-v_2)/R_2+(v-v_3)/R_3=0 & (b) 
	\end{array} \right.
\]
We solve (b) for $v$ to get
\[	v=\frac{R_2}{R_2+R_3} v_3 + \frac{R_3}{R_2+R_3}v_2	\]
and substitute it into (a) to get
\[
v_o=(\frac{R_f}{R_1}+1)[\frac{R_2}{R_2+R_3} v_3+\frac{R_3}{R_2+R_3} v_2]-\frac{R_f}{R_1}v_1
=-k_1v_1+k_2v_2+k_3v_3	
\]
where
\[ \left\{ \begin{array}{l} k_1=\frac{R_f}{R_1}	\\ 
	k_2=(\frac{R_f}{R_1}+1)\frac{R_3}{R_2+R_3} \\
	k_3=(\frac{R_f}{R_1}+1)\frac{R_2}{R_2+R_3} \end{array} \right.
\]
\htmladdimg{../figures/opam6.gif}

\item {\bf Differential amplifier}

\htmladdimg{../figures/opamp10.gif}

As the input currents are zero, the currents flowing through $R_1$ and 
$R_2$ are equal for both sides, i.e.,
\[	\frac{V_1-V^-}{R_1}=\frac{V^--V_o}{R_2},\;\;\;\;\;\;
	\frac{V_2-V^+}{R_3}=\frac{V^+-0}{R_4}		\]
Assumeing $R_1=R_3$ and $R_2=R_4$ and $V^+=V^-$, we subtract the second 
equation from the first to get:
\[	\frac{V_1-V_2}{R_1}=\frac{-V_o}{R_2},\;\;\;\;
	V_o=-\frac{R_2}{R_1}\;(V_1-V_2)=\frac{R_2}{R_1}\;(V_2-V_1)	\]
{\bf Note 1:} It is likely that both inputs are to subjected to some
common noise $n(t)$ (such as interference of 60Hz power supply):
\[ V_1=v_1(t)+n(t), \;\;\;\;\;V_2=v_2(t)+n(t) \]
In this case the output is 
\[ V_o=\frac{R_2}{R_1}\;(V_2-V_1)=\frac{R_2}{R_1}\;(v_2(t)-v_1(t)) \]
not affected by the common noise at all, i.e., the differential amplifier 
can suppress {\em common-mode singal} (such as the noise signal $n(t)$)
while amplify the {\em differential-mode signal} (such as $v_1(t)$ and
$v_2(t)$).

{\bf Note 2:} If one of the input, e.g., $V_2$ is connected to a constant
voltage treated as a reference $V_{ref}$, then the differential amplifier
can also be used aa a level shifter. As
\[	\frac{V_1-V^-}{R_1}=\frac{V^--V_o}{R_2} \]
we get
\[ V_o=-\frac{R_2}{R_1}V_1-(1+\frac{R_2}{R_1})V^- \]
But 
\[ V^-=V^+=V_{ref}\frac{R_4}{R_3+R_4} \]
we have
\[ V_o=-\frac{R_2}{R_1}V_1-V_{ref}(1+\frac{R_2}{R_1})\frac{R_4}{R_3+R_4}
=-\frac{R_2}{R_1}V_1-V_{shift} \]
where $V_{shift}=V_{ref}\frac{(R_1+R_2)R_4}{R_1(R_3+R_4)}$. In other
words, the output is $-R_2/R_1$ times the input $V_1$, shifted by a
constant value $V_{shift}$. This level-shifter circuit can be used to
change the DC level of the signal (e.g., removal of DC component) as 
well as amplifying it.

\item {\bf Instrument Amplifier} (Homework)

\htmladdimg{../figures/instrumentamplifier.gif}

Express the output voltage $V_o$ as a function of both inputs $V_1$ and
$V_2$. Find the gain $A=V_out/(V_1-V_2)$.

{\bf Hint:} Analyze the three opamps separately. Assume the voltage at
the middle point of $R_1$ is zeor, i.e., the $v^-$ input of each of the
two opamps is grounded through $R_2/2$.

\htmladdnormallink{Answer}{../instrumentamplifier/index.html}


\item {\bf A/D converter}

Without feedback, the output of an op-amp is $V_o=A(V^--^+)$. As $A$ is
large, $V_o$ is usually saturated, equal to approximately either the 
positive or the negative power supply, depending on whether or not $V^+$ 
is greater than $V^-$. These two possible outputs, positive and negative,
can be treated as ``1'' and ``0'' of the binary system. The figure shows
an A/D converter built by three op-amps to measure voltage $V_i$ from 0 to 
3 volts with resolution 1 V.

\htmladdimg{../figures/opam12.gif}

Due to the voltage dividor, the input voltages to the three opamps are, 
respectively, 2.5V, 1.5V and 0.5V. The output of these opamps are listed
below for each of the voltages:

\begin{tabular}{c|cccc}\hline \\
Voltage (volts) & 0 	& 1	& 2	& 3	\\
Opamps Outputs	& 000	& 001	& 011	& 111	\\
Binary Representation	& 00	& 01	& 10	& 11	\\ \hline
\end{tabular}
A digital logic circuit is then needed to convert the 3-bit output
of the op-amps to the two-bit binary representation.

\item {\bf First order system --- integrator and differentiator}

\htmladdimg{../figures/opamp4.gif}

{\bf Integrator}

In time domain, as $v^-=v^+=0$ and $i_R+i_C=0$, we have
\[ i_R=-i_C \Longrightarrow \frac{v_i}{R}=-C\frac{d v_C}{dt} \]
i.e.,
\[ v_o=v_C=-\frac{1}{\tau} \int v_i dt+v_C(0)	\]
where $v_C(0)$ is the initial voltage across $C$ and 
$\tau \stackrel{\triangle}{=}RC$. In frequency domain, we have:
\[ H(j\omega)=-\frac{Z_2(j\omega)}{Z_1(j\omega)}=-\frac{1/j\omega C}{R}
=-\frac{1}{j\omega RC}=-\frac{1}{j\omega \tau}	\]

{\bf Differentiator}

If we swap the resistor and the capacitor, we get in time domain:
\[ i_R=-i_C \Longrightarrow \frac{v_o}{R}=-C\frac{d v_i}{dt} \]
i.e.,
\[ v_o=-RC \frac{d v_i}{dt}=-\tau \frac{d v_i}{dt}	\]
In frequency domain, we have:
\[ H(j\omega)=-\frac{Z_2(j\omega)}{Z_1(j\omega)}=-\frac{R}{1/j\omega C}
  =-j\omega \tau \]
where $\tau=RC$. 

\item {\bf First order systems -- low-pass and high-pass filters}

\htmladdimg{../figures/opamp4a.gif}

{\bf Low-pass filter:}
\[ H(j\omega)=-\frac{Z_2(j\omega)}{Z_1(j\omega)}=-\frac{R_2||1/j\omega C}{R_1}
=-\frac{R_2}{R_1}\frac{1}{1+j\omega R_2C}
=-H(0)\frac{1}{1+j\omega \tau} \]
where $H(0)=R_2/R_1$, $\tau=R_2C$. Intuitively, when frequency is high,
$Z_2(j\omega)$ is small and the effect of negative feedback is strong,
therefore the output is low.

{\bf High-pass filter:}
\[ H(j\omega)=-\frac{Z_2(j\omega)}{Z_1(j\omega)}=-\frac{R_2}{R_1+1/j\omega C}
=-\frac{R_2}{R_1}\frac{j\omega R_1C}{1+j\omega R_1C}
=-H(0)\frac{j\omega \tau}{1+j\omega \tau} \]
where $H(0)=R_2/R_1$, $\tau=R_1C$. Intuitively, when frequency is low
$Z_1(j\omega)$ is large and the signal is difficult to pass, therefore the 
output is low.

{\bf Band-pass filter:}

\htmladdimg{../figures/opamp4b.gif}

\[ H(j\omega)=-\frac{Z_2(j\omega)}{Z_1(j\omega)}
=-\frac{R_2||1/j\omega C_2}{R_1+1/j\omega C_1}
=-\frac{R_2/(1+j\omega R_2C_2)}{(1+j\omega R_1C_1)/j\omega C_1}
=-\frac{j\omega \tau_3}{(1+j\omega \tau_1)(1+j\omega \tau_2)} \]
where $\tau_1=R_1C_1$, $\tau_2=R_2C_2$, $\tau_3=R_2C_1$.

\item {\bf Higher order systems}

Higher than first order systems can be built with multiple integrators, as 
shown here for a third order system:

\htmladdimg{../figures/opam7.gif}

From the diagram, we can get
\[
\left\{ \begin{array}{l}
	y_3=y_2/s \Longrightarrow y_2=y_3s	\\
	y_2=y_1/s \Longrightarrow y_1=y_2s=y_3s^2	\\
	y_1=y_0/s \Longrightarrow y_0=y_1s=y_3s^3	
\end{array} \right.
\]
But we also have
\[	y_0=x-(k_1y_1+k_2y_2+k_3y_3)	\]
i.e., 
\[	x=y_0+k_1y_1+k_2y_2+k_3y_3=(s^3+k_1s^2+k_2s+k_3) y_3	\]
we get the transfer function
\[
	H(s)=\frac{y_3}{x}=\frac{1}{s^3+k_1s^2+k_2s+k_3}
\]


\item {\bf Second order system by 2 integrators}

\htmladdimg{../figures/opam8.gif}

From the diagram, we can get
\[
\left\{ \begin{array}{ll}
	y_2=-c_2y_1/s  \Longrightarrow  y_1=-sy_2/c_2 \\
	y_1=-c_1y_0/s  \Longrightarrow y_0=-sy_1/c_1=s^2y_2/c_1c_2 \\
	y_0=k_0 x+k_1y_1+k_2y_2 
	\end{array} \right.
\]
substituting the first two equations into the last one, we get
\[	\frac{s^2}{c_1c_2} y_2=k_0x+k_1(-\frac{s}{c_2})y_2+k_2y_2 \]
from which we obtain the transfer function as
\[
H(s)=\frac{y_2}{x}=\frac{k_o}{\frac{s^2}{c_1c_2}+\frac{s}{c_2}s-k_2}
	=\frac{k_oc_1c_2}{s^2+k_1c_1s-c_1c_2k_2}
\]
which is a second order system. In particular, if $c_1=c_2=c$, we have
\[
	H(s)=k_0\frac{c^2}{s^2+c k_1s-k_2c^2}
\]
Comparing this with the canonical 2nd order system transfer function
\[
	H(s)=\frac{\omega_n^2}{s^2+2\zeta \omega_n s+\omega_n^2}
\]
we see that we can let $c=\omega_n$ and $k_1=2\zeta$. Moreover, $k_2<0$, 
i.e., the feedback from the output should be negative. $k_0$ is a constant
scaler which can take any value.
	
%\htmladdimg{../figure/opam9.gif}
\end{itemize}

\end{document}


	

	












