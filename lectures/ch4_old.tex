\documentstyle[12pt]{article}
\usepackage{html}
%\usepackage{graphics}  
\begin{document}

{\bf Chapter 4: Semiconductor Devices and Circuits}

\section*{Semiconductor materials}

The \htmladdnormallink{{\em vacuum tubes}}{http://en.wikipedia.org/wiki/Vacuum_tube#Diodes} were widely used for various purposes in electronics, mostly voltage 
and power amplification, before the invention of solid state semiconductor 
devices in the 1940's. Since then semiconductor devices have gradually replaced
vacuum tubes in most of such applications, due to many of their favorable 
properties such as small size, light weight, low energy consumption, high 
frequency capability, and high reliability. However, vacuum tubes can still 
find some applications even today, such as high power radio frequency transmitters
and microwave ovens. Although the physics of vacuum tubes and semiconductor 
devices is very different, the basic functions are similar, such as controlling
current through small voltage.


\begin{itemize}
\item {\bf Conductors and Insulators:} 

Good conductors, such as copper (Cu $2+8+18+1=29$), silver (Ag $2+8+18+18+1=47$),
and gold (Au $2+8+18+32+18+1=79$) can conduct electricity with little resistance 
because the atoms have only one electron on the out-most layer or shell, called
\htmladdnormallink{\em valence electron (VE)}
{https://en.wikipedia.org/wiki/Valence_electron}, which is only loosely bound 
to the atom and can easily become a free electron freely movable under an applied
voltage to conduct electricity. 

On the other hand, insulators do not conduct electricity as no free electrons 
exist in the material.

Here is a table showing the maximum number of electrons in each shell.
In general, there are $2n^2$ electrons in the nth shell. However, each 
level is not necssarily completely filled, and the outmost layer, the 
valence shell, can only have no more than 8 valence electrons.
\[
\begin{array}{l|cccccc} \hline
\mbox{Energy Level } (n) & 1 & 2 & 3 & 4 & 5 & 6 \\ \hline
\mbox{Shell Letter} & K & L & M & N & O & P  \\ \hline
\mbox{Electron Capacity } (2n^2) & 2 & 8 & 18 & 32 & 50 & 72 \\ \hline
\end{array}
\]

\htmladdimg{../figures/PeriodicTable.png}

\item {\bf Semiconductors:} 

The conductivity of those elements with four valence electrons in the carbon 
group is not as good as the conductors but still better than the insulators, 
and they are given the name semiconductors. The two semiconductors of great 
importance are silicon (Si $2+8+4=14$) and germanium (Ge $2+8+18+4=32$), both 
of which have four valence electrons. Their crystal structure {\em lattice}
has a tetrahedral pattern with each atom sharing one valence electron with 
each of its four neighbors to form the {\em covalent bonds}. 

If an electron gains enough thermal energy (1.1 eV for Si or 0.7 eV for
Ge), it may break the covalent bond and becomes a free electron of negative
charge, while leaving a vacancy or a {\em hole} of positive charge. In an 
electric field, a free electron may move to a new location to fill a hole 
there, i.e., both such electrons and holes contribute to electrical conduction. 
Such crystal is called {\em intrinsic semiconductor}.

At room temperature, relatively few electrons gain enough energy to become
free electrons, the over all conductivity of such materials is low, thereby
their name semiconductors, and the material is neither a good conductor nor
a good insulator.

\htmladdimg{../figures/Si.png}

\item {\bf Doped Semiconductors:}

The conductivity of semiconductor material can be improved by doping, i.e.,
by adding an impurity element with either three or five valence electrons,
called, respectively, trivalent and pentavalent elements. A semiconductor
is called either intrinsic or extrinsic, depending on whether it contains
any doped impurity.

\htmladdimg{../figures/elements.gif}

\htmladdimg{../figures/Si5.png}
\htmladdimg{../figures/Si3.png}

\begin{itemize}
\item {\bf N-type semiconductor:}

When a small amount of pentavalent donor atoms (e.g., phosphorus (P) and
Arsenic (As)) is added, a silicon atom in the lattice may be replaced by
a donor atom with four of its valence electrons forming the covalent bounds 
and one extra free electron. This is an {\em N-type} semiconductor whose 
conductivity is much improved compared to the intrinsic semiconductors, due 
to the extra free electrons in the lattice, which are called {\em predominant
or majority current carriers}. There also exist some tiny number of holes 
called {\em minority carriers}.

\item {\bf P-type semiconductor:}

When a small amount of trivalent acceptor atoms (e.g., boron (B) and aluminum
(Al)) is added, a silicon atom in the lattice may be replaced by an acceptor
atom with only three valence electrons forming three covalent bounds and a 
hole in the lattice. This is a {\em P-type} semiconductor whose conductivity 
is also much improved compared to the intrinsic semiconductors, due to the 
holes in the lattice, which are called {\em predominant or majority current 
carriers}. There also exist some tiny number of free electrons called 
{\em minority carriers}.
\end{itemize}

\htmladdimg{../figures/lattice.gif}

\item {\bf PN-Junction}

When P-type and N-type materials are in contact with each other, a 
PN-junction is formed due to these two effects:
\begin{itemize}
\item {\bf Diffusion:}

Although both sides are electrically neutral, they have different
concentration of electrons (the N-type) and holes (the P-type), and 
a {\em diffusion current} is formed by the diffusion of the higher
concentration of the freely movable electrons in the N-type material 
that move across the PN-junction from the N side to the P, due to the
thermal motion. They arrive at the P side to fill some of the holes
there. Equivalently, we can also consider the holes are diffusing
from the P side to the N side.

\item {\bf Electric Field}

If no other forces were involved, the diffusion would carry out 
continuously until the free electrons and holes are uniformly distributed
across both materials. However, as the result of the diffusion process,
electrical field is gradually established, negative on the side of P-type 
material due to the extra electrons, positive on the side of N-type 
material due to the loss of free electrons. This electrical field prevents
further diffusion as the electrons on the N-type side are expelled from
the P-type side by the electrical field.

\end{itemize}

\htmladdimg{../figures/pn_junction0.gif}

\htmladdimg{../figures/pn_junction1.gif}

The effects of both diffusion and electric field eventually lead to an 
equilibrium where the two effects balance each other so that there are
no more charge carriers (free electrons or holes) crossing the PN-junction.
This region around the PN-junction, called the {\em depletion region} as 
there no longer exist freely movable charge carriers, becomes a barrier 
between the two ends of the material that prevent current to flow through.

Additional notes semiconductor materials can be found
\htmladdnormallink{here}{../SemiConductorMaterials.pdf} and
\htmladdnormallink{here}{../SemiConductorMaterials2.pdf}.

{\bf Solar Cell}

A solar cell converts light energy to electrical energy and is a current 
source. When a photon of light hits a piece of semiconductor material 
(PN-junction), it either goes straight through material if its energy is 
lower than the band-gap energy of the silicon semiconductor, or is absorbed
by the silicon if its energy is greater than the band-gap energy. In the
latter case, an electron-hole pair is produced, and the electron and hole
are separated by the internal electric filed near the PN-junction. the 
electrons are pulled by the positive potential on the N side and the 
holes are pulled by the negative potential on the P side, thereby forming
a current through the external circuit. The solar cell is a current source,
as the incoming flux of photons causes the current (not voltage).

\htmladdimg{../figures/solarcell.png}

\end{itemize}

Additional notes on photo-diodes, solar cells, and LEDs can be found
\htmladdnormallink{here}{../SemiConductorMaterials3.pdf}.

More detailed discussion about semiconductor physics can be found
\htmladdnormallink{here}{http://ecee.colorado.edu/~bart/book/book/contents.htm}

\section*{Diodes}

A diode is formed by a PN-junction with the p side called {\em anode} and the
n side called {\em cathode}. Due to the fact that there exist few freely movable 
charge carriers in the depletion region around the PN-junction, the conductivity
is very poor. However, when external voltage is applied to the two ends of the 
material, the conductivity may change, depending one the polarity of the applied
voltage.

\htmladdimg{../figures/diode6.gif}

\begin{itemize}
\item {\bf Forward bias} (positive to P-type, negative to N-type)

  The positive voltage applied to the P-type will pull electrons in N-type
  and repel holes in P-type so that both carriers are moving towards the 
  PN-junction. As the depletion region becomes thinner, the conductivity 
  increases due to the {\em drift current} through the PN-junction from 
  the P side to the N side, formed by the majority charge carriers (both 
  electrons and holes) driven by the applied voltage. The conductivity 
  increases as the applied voltage becomes higher. 

\item {\bf Reverse bias} (negative to P-type, positive to N-type)

  The negative voltage applied to the P-type will repel electrons in N-type
  and attract holes in P-type so that both carriers are moving away from 
  the PN-junction. As the depletion region becomes thicker than before, 
  there is no current through the PN-junction from the P side to the N side.
  However, there exists a very small current $I_0$, called the 
  \htmladdnormallink{{\em reverse saturation current}}{https://en.wikipedia.org/wiki/Saturation_current},
  due to the minority carriers.
  The carrier velocity increases as the applied voltage becomes higher. 
  However, as the voltage further increases, the velocity will reach a 
  maximum level called
  \htmladdnormallink{{\em saturation velocity}}{http://en.wikipedia.org/wiki/Velocity_saturation}.

%http://ecee.colorado.edu/~bart/book/book/chapter2/ch2_7.htm#2_7_3}.)
  

\end{itemize}
The non-linear voltage-current relationship of a PN-junction is described by
\[
I_D=I_0 \left( e^{V_D/\eta V_T}-1 \right), \;\;\;\;\mbox{or}\;\;\;\;
V_D=\eta V_T\;ln \left(\frac{I_D}{I_0}+1\right)
\]
where 
\begin{itemize}
\item $I_0$ is the {\em reverse saturation current}, a tiny current that 
  flows in the reverse direction when $V_D < 0$, due to the minority 
  carriers. As this current is limited by the minority carriers available,
  it is called saturation current. $I_0$ is about $10^{-10} \sim 10^{-12}$ 
  A for Si and $10^{-4}$ A for Ge.
\item $ V_T=kT/e $ is the thermal voltage, where 
  $k=1.38\times 10^{-23}$ Joules/Kelvin is the
\htmladdnormallink{Boltzmann constant}{https://en.wikipedia.org/wiki/Boltzmann_constant},
  $e=1.602\times 10^{-19}$ coulomb is the charge of an electron, and
  $T$ is the temperature in degree K. At room temperature $T=300 K$
  ($27^\circ C$), $V_T=kT/e=26\; mV$.
\item $\eta$ is the ideality factor which varies between 1 and 2, depending
  on the fabrication process and semiconductor material. In many cases $\eta$
  can be assumed to be approximately equal to 1.
\end{itemize}
In particular, 
\begin{itemize}
\item when $V_D<0$, $I_D\approx -I_0$; 
\item when $V_D=0$, $I_D=0$; 
\item when $V_D>0$, $I_D\approx I_0 e^{V_D/V_T}$.
\end{itemize}

\htmladdimg{../figures/diode1.gif}

The voltage $V_D$ across the diode is a function of the current $I_D$ through 
the diode. In the range of 5 mA to 20 mA, $V_D$ is about 0.7 V:
\[
\begin{array}{c||c|c|c|c|c} \hline
$I_D$	& 1 \;mA & 5 \;mA & 10 \;mA & 20 \; mA & 100 \;mA	\\ \hline\hline
$V_D$ for Si ($I_0=10^{-10}$, $\eta=1.4$) & 0.58 V & 0.65 & 0.67 V & 0.7 & 0.75 V \\\hline
$V_D$ for Ge ($I_0=10^{-4}$, $\eta=1.0$) & 0.06 V & 0.10 & 0.12 V & 0.14 & 0.18 V \\\hline
\end{array}
\]


The resistance of an electrical device is defined as $R=\Delta V/\Delta I$.
For a diode, as $V_D(I_D)$ is not a linear function, the resistance 
$R_D=dV_D/dI_D$ can be found as
\[
R_D=\frac{d\,V_D}{d\,I_D}=\frac{d}{d\,I_D} \left[\eta V_T\;ln \left(\frac{I_D+I_0}{I_0}\right) \right]
=\eta V_T \frac{I_0}{I_D+I_0}\frac{1}{I_0}=\eta \; \frac{V_T}{I_D+I_0}
\approx \eta \; \frac{V_T}{I_D}	
\]
The approximation is due to the fact that $I_D \gg I_0$, i.e., 
$I_D+I_0\approx I_D$. We assume $V_T=26\;mV$, $\eta=1.4$, the resistance
of the diode is not a constant, but a function of the current $I_D$, i.e., 
a diode is not a linear element:
\[
\begin{array}{c||c|c|c|c|c|c|c} \hline
$I_D$	& 0.05\; mA & 0.1\; mA & 0.2 \;mA & 0.5\; mA & 1\; mA & 2\; mA & 5\; mA \\ \hline
$R_D$ for Si ($\eta=1.4$) & $728\,\Omega$ & $364\,\Omega$ & $182\,\Omega$ & 
  $73\,\Omega$ & $36\,\Omega$ & $18\,\Omega$ & $7\,\Omega$ \\\hline
\end{array}
\]


{\bf Models of diodes:}

\htmladdimg{../figures/diode3.gif}

\begin{itemize}
\item Ideal model: 
  if $V_D<0$, $R_D=\infty$, $I_D=0$, else $R_D=0$, $I_D=V/R$.
\item Ideal model with a voltage threshold $V_0=0.7\,V$:
  if $V_D<V_0$, $I_D=0$, else $V_D=0.7$, $I_D=(V-0.7)/R$
\item The model above in series with a resistance $R_D=20\Omega$:
  if $V_D<V_0$, then $I_D=0$, else $I_D=(V-0.7)/(R+R_D)$
\item The model above in parallel with a current source that simulates
  the reverse saturation current.
\end{itemize}

\htmladdimg{../figures/diodemodel.gif}
\htmladdimg{../figures/diodemodels.gif}

In general, when the forward voltage applied to a diode exceeds 0.6 to 
0.7V for silicon (or 0.1 to 0.2 V for germanium) material, the diode is 
assumed to be conducting with very little resistance.

{\bf Example: } In the half-wave rectifier circuit shown below, 
$R=1000\Omega$, $V=3V$, and $D$ is a silicon diode. Find the current
$I_D$ through and voltage $V_D$ across $D$.

\htmladdimg{../figures/diode2.gif}

\begin{itemize}

\item {\bf Method 0:} The simplest model is to assume the diode is an ideal
  rectifier with infinite resistance when it is reverse biased but zero
  resistance when it is forward biased. As the diode is forward biased,
  the current is $I_D=V/R=3\;mA$.

\item {\bf Method 1: } Since the diode is forward biased, we can assume 
  the voltage across the diode is $0.7V$ and the current can be determined 
  by Ohm's law to be $I_D=(V-0.7)/R=2.3/1000=2.3\;mA$.

\item {\bf Method 2: } The diode is modeled as a series combination of
  $V_0=0.7\;V$, $R_D=20\Omega$, and an ideal diode. $I_D=(3-0.7)/(1000+20)=2.255\;mA$,
  $V_D=0.7+I_DR_D=0.745\;V$.

\item {\bf Method 3: } The current $I_D$ and voltage $V_D$ have to satisfy 
  two equations simultaneously:
  \[
  \left\{ \begin{array}{l} I_D=I_0(e^{V_D/V_T}-1) \\
    V_D=V-RI_D \end{array} \right. 
  \]
  The first equation relates the current $I_D$ through and voltage $V_D$ 
  across a diode, while the second is obtained by KVL. Substituting the first
  equation into the second, we get $V_D=V-RI_0(e^{V_D/V_T}-1)$. Substituting
  this into the first equation we can then find $I_D$.

\item {\bf Method 4: } The two simultaneous equations above can also be
  solved graphically. The first equation $I_D=I_0(e^{V_D/V_T}-1)$ is the
  characteristic curve of the diode, while the second equation $V=V_D+I_DR$
  is the load line, a straight line passing through $(V_D=0, I_D=V/R)$ and 
  $(I_D=V, I_D=0)$. The intersection point of the two curves is approximately
  at $(V_D=0.75\;V, I_D=2.4\;mA)$.

\end{itemize}

Diodes are typically used as rectifiers which convert an AC voltage/current
in to a DC one, such as shown in the following example.

{\bf Example 2: } Design a converter (adaptor) that converts AC power 
supply of 115V and 60 Hz to a DC voltage source of 14 V. When the load is
$R_L=10\;K\Omega$, the variation (ripple) of the output DC voltage must 
be 5\% or less.

\htmladdimg{../figures/halfwaverectifier.gif}

\htmladdimg{../figures/diode4.gif}

\begin{itemize}
\item The desired DC voltage is 14 V, but we also need to compensate for the
  voltage drop of 0.7 V due to the forward biased diode, so the  peak of the
  secondary output is $14.7\;V$ with RMS value $14.7/\sqrt{2}=10.4V$, the ratio
  of the transformer should be $115/10.4=11$. 
\item When the load is $R_L=10\;K\Omega$, the load current is approximately
  $I\approx V/R_L=14/10=1.4\;mA$.
\item During the period between two peaks $T=1/f=1/60=16.7\;ms$, the charge 
	on the capacitor is reduced by $\Delta Q\approx IT=1.4\;mA\times 
        16.7\;ms=23.4\;\mu C$. 	
\item The voltage across the capacitor is therefore dropped by
	$\Delta V=\Delta Q/C < 14\times 5/\%=0.7V $. 
\item Solve above equation for $C$, we get $C=\Delta Q/\Delta V
  =23.4\;\mu C/0.7V=33.4\;\mu F$.
\end{itemize}
This is an approximation based on the assumption that the load current is
constant, as the voltage drop is small. Otherwise the exponential decay of
the voltage across capacitor should be used, and the current is:
\[	i(t)=\frac{V}{R_L} e^{-t/\tau}	\]

More diode rectification circuits are shown below:

\htmladdimg{../figures/DiodeRectifiers.png}

\htmladdimg{../figures/Smoothers.png}

\begin{comment}
In the circuit on the left, voltages on $C_1$ and $C_2$ are both the 
same as the peak votage on the secondary side of the transformer, and
the voltage across $R_L$ is twice the voltage.

In the circuit on the right, voltage on $C_1$ is the peak voltage on 
the 2nd side of the transformer, voltegaes on both $C_2$ and $C_3$ 
are twice the voltage,  and the voltage across $R_L$ is three times
of that voltage.
\end{comment}

\section*{Bipolar Junction Transistor (BJT)} 
\htmladdnormallink{(External reference on Wikipedia)}{http://en.wikipedia.org/wiki/Bipolar_junction_transistor}

A Bipolar Junction Transistor (BJT) has three terminals connected to three
doped semiconductor regions. In an NPN transistor, a thin and lightly doped 
P-type {\em base} is sandwiched between a heavily doped N-type {\em emitter}
and another N-type {\em collector}; while in a PNP transistor, a thin and 
lightly doped N-type {\em base} is sandwiched between a heavily doped P-type
{\em emitter} and another P-type {\em collector}. In the following we will 
only consider NPN BJTs.

\htmladdimg{../figures/transistors1.gif}

\htmladdimg{../figures/transistorBJT1.gif}

In many schematics of transistor circuits (especially when there exist a
large number of transistors in the circuit), the circle in the symbol of
a transistor is omitted. The figures below show the cross section of two
NPN transistors. Note that although both the collector and emitter of a
transistor are made of N-type semiconductor material, they have totally
different geometry and therefore can not be interchanged.

\htmladdimg{../figures/transistorBJT2a.gif}

\htmladdimg{../figures/transistorBJT2b.gif}

All previously considered components (resistor, capacitor, inductor, and 
diode) have two terminals (leads) and can therefore be characterized by 
the single relationship between the current going through and the voltage 
across the two leads. Differently, a transistor is a three-terminal component,
which could be considered as a two-port network with an input-port and an
output-port, each formed by two of the three terminals, and characterized
by the relationships of both input and output currents and voltages.

Depending on which of the three terminals is used as common terminal, there
can be three possible configurations for the two-port network formed by a
transistor: 
\begin{itemize}
\item Common emitter (CE), 
\item Common base (CB), 
\item Common collector (CC).
\end{itemize}

\htmladdimg{../figures/transistors2.gif}

\begin{itemize}
\item {\bf Common-Base (CB) configuration}

  The CB configuration can be considered as a 2-port circuit. The input
  port is formed by the emitter and base, the output port is formed by 
  the collector and base. Two voltages $V_{BE}$ and $V_{CB}$ are applied 
  respectively to the emitter $E$ and collector $C$, with respect to the 
  common base $B$, so that the BE junction is forward biased while the
  CB junction is reverse biased.

  \htmladdimg{../figures/CB.png}
  \htmladdimg{../figures/CBnpn.gif}

  %\htmladdimg{../figures/transistorDC.gif}
  %\htmladdimg{../figures/transistorDC1.gif}

  The polarity of $V_{BE}$ and direction of $I_B$ associated with the
  PN-junction between E and B are the same as those associated with 
  a diode, voltage polarity: positive on P, negative on N, current 
  direction: from P to N, but $V_{CB}$ and the direction of $I_C$ 
  associated with the PN-junction between the base and collector are 
  defined oppositely. 

  The behavior of the NPN-transistor is determined by its two PN-junctions:

  \begin{itemize}
  \item The forward biased base-emitter (BE) PN-junction allows the 
    majority charge carriers, the electrons, in N-type emitter to go 
    through the PN-junction to arrive at the P-type base, forming the 
    emitter current $I_E$.

  \item As the base is thin and lightly doped, only a small number of 
    the electrons from the emitter (e.g., 1\%) are combined with the 
    majority carriers, the holes, in the P-type base to form the base 
    current $I_B$. The percentage depends on the doping and geometry 
    of the material.
    
  \item Most of the electrons from the emitter (e.g., 99\%), now the
    minority carriers in the P-type base, can go through the reverse 
    biased collector-base PN junction to arrive at the N-type collector
    forming the collector current $I_C=I_E-I_B$.

  \end{itemize}

%  \item The minority carriers originally in the base and collector also go 
%    through the reverse biased collector-base PN junction to form the 
%    {\em collector-base saturation current} $I_{CB0}$. 
%    As $I_{CB0}\ll I_C=\alpha I_E$, it can be ignored in the following.
%    \[   I_C=\alpha I_E+I_{CB0}\approx \alpha I_E    \]

The {\em current gain} or {\em current transfer ratio} of this CB circuit,
denoted by $\alpha$, is defined as the ratio between collector current 
$I_C$ treated as the output and the emitter current $I_E$ treated as the 
input:
\[ 
\alpha=\frac{I_C}{I_E}<1,\;\;\;\mbox{e.g.}\;\;\;\;\;\alpha =99\% \approx 1
\]
i.e., 
\begin{eqnarray}
  I_C& = &\alpha I_E 
  \nonumber\\
  I_B& =&I_E-I_C=I_E-\alpha I_E=(1-\alpha)I_E
  \nonumber\\
  I_E&=&\frac{I_C}{\alpha}=\frac{I_B}{1-\alpha}
  \nonumber
\end{eqnarray}

The relationships between the current and voltage of both the input 
and output ports are described by the following input and output 
characteristics.

\begin{itemize}
\item {\bf Input characteristics:} 

    The input current $I_E$ is a function of $V_{CE}$ as well as the input
    voltage $V_{BE}$, which is much more dominant:
    \[ 
    I_E=f(V_{BE}, V_{CB})\approx f(V_{BE})=\frac{I_B}{1-\alpha}
    =\frac{1}{1-\alpha} I_0 ( e^{V_{BE}/V_T}-1 )	
    \]
    Note that $V_{CB}$ has little effect on $I_E$.
    Here $I_B$ and $V_{BE}$ associated with the emitter-base PN-junction 
    satisfy the relationship for a diode:
    \[
    I_B=I_0 ( e^{V_{BE}/V_T}-1 )	
    \]
    The voltage across the forward biased PN junction can be approximated
    by $V_{BE}=0.7\,V$. 

  \item {\bf Output characteristics:} 

    The output current $I_C$ is a function of the output voltage $V_{CE}$
    as well as the input current $I_E$, which is much more dominant:
    \[ 
    I_C=f(I_E,V_{CB})\approx f(I_E)=\alpha I_E
    \]
    Here the approximation is based on the assumption that $V_{CB}>0.2V$
    (in linear region). As $V_{CB}>0$, i.e., the CB junction is reverse 
    biased, the current $I_C$ only depends on $I_E$. When $I_E=0$, 
    $I_C=I_{CB0}$ is the current caused by the minority carriers crossing 
    the PN-junction. This is similar to the diode current-voltage 
    characteristics seen before, except both axes are reversed (the 
    polarity of $V_{CB}$ and the direction $I_C$ are oppositely defined).
    When $I_E$ is increased, $I_C=\alpha I_E+I_{CB0}\approx \alpha I_E$ is 
    increased correspondingly. However, as higher $V_{CB}$ does not cause
    more electrons from the emitter, it has little effect on $I_C$.

    Note that when $V_{CB}=0$, the PN-junction between base and collector
    is not biased (short circuited), there is still a non-zero collector 
    current $I_C>0$, formed by the electrons coming from the emitter,
    through both PN-junctions to form a closed loop current.
    
    \htmladdimg{../figures/TransistorCBplots.gif}

  \end{itemize}

\item {\bf Common-Emitter (CE) configuration}

  Two voltages $V_{BE}$ and $V_{CE}$ are applied respectively to the base 
  $B$ and collector $C$ with respect to the common emitter $E$. Typically
  $V_{CE} > V_{BE}$, i.e., the BE junction is forward biased while the CB 
  junction is reverse biased, same as the CB configuration. The voltages 
  of CB and CE configurations are related by:
  \[ 
  V_{CE}=V_{CB}+V_{BE},\;\;\;\;\;\mbox{or}\;\;\;\;\;  V_{CB}=V_{CE}-V_{BE}
  \]

  \htmladdimg{../figures/CE1.png}
  \htmladdimg{../figures/CEnpn.gif}

  The CE configuration can be considered as a 2-port circuit. The input
  port is formed by the emitter and base, the output port is formed by 
  the collector and emitter. The current gain of the CE circuit, denoted 
  by $\beta$, is defined as the ratio between the collector current $I_C$ 
  treated as the output and the base current $I_B$ treated as the input:  
  \[
  \beta=\frac{I_C}{I_B}=\frac{\alpha I_E}{(1-\alpha) I_E}=\frac{\alpha}{1-\alpha},
  \]
  For example, if $\alpha=0.99$, then $\beta=0.99/(1-0.99)=99$. 

  The two parameters $\alpha$ and $\beta$ are related by any of the 
  following:
  \[
  \beta=\frac{\alpha}{1-\alpha},\;\;\;\;\;\;\alpha=\frac{\beta}{1+\beta},
  \;\;\;\;1+\beta=\frac{1}{1-\alpha},\;\;\;\;\;1-\alpha=\frac{1}{1+\beta} 
  \]


  The relationships between the current and voltage of both the input 
  and output ports are described by the following input and output 
  characteristics.

  \begin{itemize}
  \item {\bf Input characteristics:} 

    Same as in the case of common-base configuration, the EB junction of the
    common-emitter configuration can also be considered as a forward biased
    diode, the current-voltage characteristics is similar to that of a diode:
    \[
    I_B=f(V_{BE},V_{CE})\approx f(V_{BE})=I_0 ( e^{V_{BE}/V_T}-1 )	
    \]
    The voltage across the forward biased PN junction is approximately 
    $V_{BE}=0.7\,V$. $V_{CE}$ has little effect on $I_B$.
    
  \item {\bf Output characteristics:} 
    \[
    I_C=f(I_B,V_{CE})\approx f(I_B)=\beta I_B\;\;\;\;\;\;\;\;\;\;\mbox{(in linear region)} 
    \]
    The current $I_C=\beta I_B$ depends on the current $I_B$, which in 
    turn depends on $V_{BE}$. However, as higher $V_{CE}$ does not cause
    more electrons from the emitter, it has little effect on $I_C$.

    \htmladdimg{../figures/TransistorCEplots.gif}
    
  \end{itemize}
\end{itemize}

The relationship between the input and output currents of both CB and CE 
configurations is summarized below:
\[
I_E=I_C+I_B,\;\;\;\;\;\;V_{CE}=V_{CB}+V_{BE}
\]

\begin{itemize}
\item Common Base:

  \htmladdimg{../figures/CommonBase.png}
  \begin{eqnarray}
    \alpha&=&\frac{I_{out}}{I_{in}}=\frac{I_C}{I_E},
    \;\;\;\;\mbox{i.e.,}\;\;\;\;\;I_C=\alpha I_E
    \nonumber\\
    I_E&=&I_C+I_B=\alpha I_E+(1-\alpha) I_E
    \nonumber
  \end{eqnarray}

\item Common Emitter:

  \htmladdimg{../figures/CommonEmitter.png}
  \begin{eqnarray}
    \beta&=&\frac{I_{out}}{I_{in}}=\frac{I_C}{I_B},
    \;\;\;\;\mbox{i.e.,}\;\;\;\;\;I_C=\beta I_B
    \nonumber\\
    I_E&=&I_C+I_B=\beta I_B + I_B
    \nonumber
  \end{eqnarray}

\end{itemize}


The collector characteristics of the common-base (CB) and common-emitter 
(CE) configurations have the following differences:
\begin{itemize}
\item In CB circuit $I_C=\alpha I_E$ is slightly less than $I_E$, while 
  in CE circuit $I_C=\beta I_B$ is much greater than $I_B$.
\item In CB circuit, $I_C>0$ when $V_{CB}=0$; while in CE circuit $I_C=0$
  when $V_{CE}=0$ (as $V_{CB}=-V_{BE}$ has the effect of suppressing $I_C$).
\item Increased $V_{CE}$ will slightly increase $\alpha$ but more
  greatly increase $\beta=\alpha/(1-\alpha)$, thereby causing more 
  significantly increased $I_C$.
\item $I_E$ in CB is a function of two variables $V_{BE}$ and $V_{CB}$,
  but the former is much more significant then the latter.
  $I_B$ in CE is a function of two variables $V_{BE}$ and $V_{CE}$,
  but the former is much more significant then the latter.
\item $I_C$ in CB is a function of two variables $V_{CB}$ and $I_E$. 
  When $V_{CB}$ is small, its slight increase will cause significant increase 
  of $I_C$. But its further increase will not cause much change in $I_C$ due 
  to saturation (all available charge carriers travel at the saturation velocity
  to arrive at collector C), $I_C=\alpha I_E$ is mostly determined by $I_E$.
\item $I_C$ in CE is a function of two variables $V_{CE}$ and $I_B$. 
  When $V_{CE}$ is small ($\approx 0.3V$), its slight increase will cause 
  significant increase of $I_C$. But when $V_{CE}>0.3V$, its further increase 
  will not cause much change in $I_C$ due to saturation (all available charge 
  carriers travel at the saturation velocity to arrive at collector C), 
  $I_C=\beta I_B$ is mostly determined by $I_B$.
\end{itemize}

\htmladdimg{../figures/InputOutputChar.gif}

\htmladdimg{../figures/transistortemp.gif}

Various parameters of a transistor change as functions of temperature. 
For example, $\beta$ increases along with temperature.

\section*{DC operating point}


{\bf Example:} In the CE circuit shown below, $V_{CC}=12V$, $R_B=6 K\Omega$,
$R_C=2 K\Omega$, $\beta=60$. The load line can be determined by two points:
$(V_{CE}=0,\;I_C=V_{CC}/R_C=6\;mA)$ and $(I_C=0,\;V_{CE}=V_{CC}=12\;V)$. Find output
voltage $V_{out}=V_{CE}$ when $V_{in}$ takes the following values:

\htmladdimg{../figures/CEexample2.png}
\htmladdimg{../figures/CEexample2c.png}

\begin{itemize}
\item $V_{BE}=V_{in}=0<0.7V$, $I_B=0$ and $I_C=\beta I_B=0$, $V_C=V_{CC}=12\;V$,
  the transistor is the {\em cutoff region}.

\item $V_{in}=1V$, $V_{in}/R_B=1.667\;mA\approx 1.7\;mA$. We assume
  $V_{BE}\approx 0.7\;V$, and get 
  \begin{eqnarray}
    I_B&=&(V_{in}-V_{BE})/R_B=(1-0.7)/6=0.05\;mA \nonumber\\
    I_C&=&\beta I_B=60\times 0.05\;mA=3\; mA \nonumber\\
    V_{CE}&=&V_{CC}-I_C R_C=12\;V-3\;mA \times 2\;K\Omega=6\;V
    \nonumber
  \end{eqnarray}
  The transistor is in the {\em linear region}.
 
\item $V_{in}=2V$. $I_B=(2-0.7)/6=0.22mA$, $I_C=\beta I_B=13\;mA$, and 
  $V_{CE}=12\;V-13\;mA\times 2\;K\Omega=-14\;V$.

  This result is unreasonable and incorrect, because $I_B$ is so high 
  that the transistor is no longer in the linear region as in the previous
  case, but it is in the {\em saturation region}, where the linear relationship 
  $I_C=\beta I_B$ is no longer applicable, and the actual voltage $V_{CE}$ can
  be approximated to be about $V_{CE}=0.2V$, and the actual $I_C$ can be found 
  to be $(V_{CC}-V_{CE})/R_C=(12-0.2)/2=5.9\;mA$.
\end{itemize}

Summarizing the above, we see that the operation of a transistor can be in 
one of the three possible regions:
\begin{itemize}
\item {\bf Cutoff region:} 

  When $V_{BE}<0.7V$, or even negative, $I_B=0$, the output current is
  $I_C=I_{CE0} \approx 0$, $V_C=V_{CC}$, i.e., the transistor (between 
  collector and emitter) is cut off (immediate above the horizontal
  axis of the output plot).

\item {\bf Linear region:} 

  When $V_{BE}\approx 0.7V$, but $I_B>0$ is small enough so that the 
  transistor is in the linear range where the collector current 
  $I_C=\beta I_B < V_{CC}/R_C$ is proportional to base current $I_B$, 
  and $V_C=V_{CC}-R_CI_C$. The CE transistor circuit in the linear 
  region is widely used for amplification.

\item {\bf Saturation region:} 

  When $V_{BE}$ is further increased $I_B$ and $I_B$ is also significantly
  increased (due to the exponential relationship between $I_B$ and $V_{BE}$), 
  $\beta I_B > V_{CC}/R_C$, the linear relationship $I_C=\beta I_B$ no longer 
  holds as $I_C$ approaches its maximum $V_{CC}/R_C$. The transistor is 
  is saturated and $V_{CE}\approx 0.2V$, independent of $I_B$ (to the 
  immediate right of the vertical axis of the output plot).

\end{itemize}

A typical CE circuit is shown in the figure below, where $I_E=I_B+I_C$,
$V_{in}=V_{BE}=V_B$, and $V_{out}=V_{CE}=V_C$.

%\htmladdimg{../figures/OperatingPoint.gif}
\htmladdimg{../figures/transistorbiasingc.gif}

The DC steady-state operating condition of the CE transistor circuit,
in terms of the currents and voltages $I_B$ and $V_{BE}$ of the input 
port, and $I_C$ and $V_{CE}$ of the output port, is called the 
{\em DC operating point} (Q-point), which iss determined by 
\begin{itemize}
\item The external circuit including the voltage source $V_{CC}$ 
  and $R_B$ and $R_C$ (as a non-ideal voltage soource represented 
  by the linear {\em load line});

\item The nonlinear input and output characteristics of the transistor.
\end{itemize}
as the intersect of the two curves, as shown in the figures below.

\htmladdimg{../figures/fixedbias2.gif}

\htmladdimg{../figures/loadlines.gif}


\section*{AC Signal Amplification}

The common-emitter transistor circuit is commonly used for voltage
amplification, as shown in the example below. Here we assume $V_{CC}=15V$
and $R_C=1.5K\;\Omega$, and $\beta=40$. 

%\htmladdimg{../figures/transistoramplifier1.png}
\htmladdimg{../figures/transistoramplifier.gif}

The current and voltage on both input and output sides can be 
obtained either algebraically or graphically as shown below.

{\bf The input voltage and current}

\begin{itemize}

\htmladdimg{../figures/transistoramplifier2.png}
\htmladdimg{../figures/transistorinput.gif}
%\htmladdimg{../figures/transistorInputPlot.png}


\item Input voltage:

  Assuming the transistor is properly biased so that $V_{BE}=0.7\,V$, 
  we get the input voltage $v_{in}(t)=v_{be}(t)$ as the superposition 
  of DC component $V_{BE}$ and a {\em small} AC input 
  $v_{be}(t)=0.02\;\cos(\omega t)\,V$:
  \[
  V_{BE}+v_{be}(t)=0.7+0.02\;\cos(\omega t)
  \]
\item Input current:
  
  Due to the small dynamic range of the input voltage, the non-linear 
  (approximately exponential) input characteristic can be linearized
  locally as a resistance, the reciprocal of the slope of the input 
  characteristic curve around $V_{BE}=0.7$:
  \[
  r_{be}=\frac{\triangle v_{be}}{\triangle i_b}=\frac{20\,mV}{0.05\;mA}=400 \;\Omega
  \]
  The base current $i_b(t)$ can be approximated to be 
  \[ 
  I_B+i_b(t) \approx 0.1+\frac{\triangle v_{be}}{r_{be}}
  =0.1+\frac{20\;\cos(\omega t)}{400}
  =(0.1+0.05\;\cos(\omega t))\;mA 
  \]
  which is also a superposition of a DC component $I_B=0.1\;mA$ and 
  an AC component with an amplitude $0.05\;mA$. 

  Why don't we get the following base current?
  \[
  I_B+i_b(t) =\frac{V_{BE}+v_{be}(t)}{r_{be}}
  =\frac{0.7+0.02\;\cos(\omega t)}{400}
  =(1.75+0.05\;\cos(\omega t))\;mA 
  \]

\end{itemize}

{\bf The output voltage and current}

\begin{itemize}
\item The DC operating point:

  The load line is the plot of equation $V_{CE}=V_{CC}-I_C R_C$, a 
  straight line hat goes through the two points:
  \[ 
  \left\{\begin{array}{l}
  I_C=0\\ V_{CE}=V_{CC}=15V\end{array}\right.
  \;\;\;\mbox{and}\;\;\;
  \left\{\begin{array}{l}
  V_{CE}=0\\I_C=V_{CC}/R_C=15V/1.5\,k\Omega=10\, mA \end{array}\right.
  \]
\item The output current, with $\beta=40$:
  \[
  I_C+i_c(t)=\beta (I_B+i_b(t))=40\times (0.1+0.05\, cos(\omega t))
  =(4 + 2\;\cos\;\omega t) \; mA 
  \]
\item The output voltage:
  \[
  V_C+v_c(t)=V_{CC}-R_C(I_C+i_c(t))=15-1.5 (4 + 2\;\cos(\omega t )
  =(9-3\;\cos(\omega t)) \; V  
  \]
\end{itemize}

\htmladdimg{../figures/transistorCEplots1a.gif}

We note that the AC sinusoidal component of the input and output 
voltages are $0.02\,\cos(\omega t)\,V$ and $-3\,\cos(\omega t)\,V$, 
respectively, i.e., it is amplified by $-3/0.02=-150$ times. The
negative sign indicates the output voltage ($-\cos\;\omega t$) 
is $180^\circ$ out of phase compared to that of the input signal 
($\cos\;\omega t$), i.e., the CE transistor circuit is a reverse 
amplifier.

{\bf Waveform distortion}

The waveform of the output $v_c(t)$ may be distorted if the DC component 
of the input voltage $V_{BE}$ (and thereby, the base current $I_B$) is 
either too low or too high, causing either the positive or negative 
peaks of the sinusoidal component to exceed the linear range of the 
output characteristic plot, as illustrated below:

%\htmladdimg{../figures/transistorQpoint1.gif}

\htmladdimg{../figures/InputOutputPlots5.png}

We see that severe distortion in output $v_c$ will be caused if a transistor 
amplification circuit is working near either the cutoff or the saturation 
region. It is therefore desirable to properly set the DC operating point 
around the middle of the linear range along the load line, to avoid to be
too close to either the saturation or cutoff region.

\htmladdimg{../figures/OperatingRegins.gif}


{\bf Example}

\htmladdimg{../figures/CEswitch.gif}

Assume $V_{CC}=15V$, $R_C=1.5\;k\Omega$, $\beta=50$. Given the input voltage 
$V_1=V_{BE}=0.2V,\;0.7V$ or $0.8V$, find the corresponding output voltage 
$V_2=V_{CE}=V_C$.

\begin{itemize}
\item $V_{BE}=0.2V < 0.7V$, the forward bias of BE PN-junction is insufficient
  for it to conduct current, we have $I_B=0$, $I_C=\beta I_B=0$, $V_2
  =V_{CC}-I_C R_C=V_{CC}=15V$. The transistor is cutoff (the switch is 
  {\em open} or open-circuit).
\item $V_{BE}=0.7V$, the BE PN-junction is forward biased, we can find $I_B$
  from the input characteristics, here assumed to be $0.1\;mA$, and get
  $I_C=\beta I_B=5\;mA$ and 
  $V_C=V_{CC}-I_C R_C=15-5\times 10^{-3} \times 1.5\times 10^{3} =7.5\; V$. 
  The transistor is in linear region.
\item $V_{BE}=0.8V>0.7V$, the BE junction is forward biased, we can find $I_B$
  from the input characteristics, here assumed to be $0.4\;mA$. If the linear
  relationship $I_C=\beta I_B$ were to hold, we would get $I_C=\beta I_B=20\;mA$
  and $V_C=V_{CC}-I_C R_C=15-20\times 10^{-3} \times 1.5\times 10^{3}=-15 V$. 
  This result is obviously wrong, indicating that the transistor is actually 
  in the saturation region (the switch is {\em closed} or short-circuit), 
  i.e., the linear relation $I_C=\beta I_B$ does not hold. In fact, it is 
  impossible for the transistor to draw $I_C=20\;mA$ from the voltage source, 
  as the maximum current is $I_C=V_{CC}/R_C=15\;V/1.5\; k\Omega=10\;mA$ when
  $V_{CE}=0$. In this case, the actual $V_{CE}$ can be approximated on the output 
  characteristics to be about $0.2V$, the intersection of load line and the 
  curve corresponding to $I_B=0.4\;mA$), and $I_C=(15-0.2)/1.5\approx 10\;mA$. 
\end{itemize}
{\bf Conclusion: } a change in input from 0.2V to 0.8V switches the output 
current from 0 to about 10 mA, and the output voltage from 15V to 0.2V,
and the transistor is in cutoff, linear, and saturation region, respectively. 
$I_C=\beta I_B$ is only valid when the transistor is in the linear region.

\section*{Biasing}

As shown before, the DC operating point of a transistor amplification circuit
needs to be set up properly (in the middle of the linear region) to avoid
signal distortion. We now consider how the operating point is determined by 
the biasing circuit, in terms of $R_B$, $R_C$, and $V_{CC}$.

\begin{itemize}
\item {\bf Fixed Biasing} 

\htmladdimg{../figures/transistorbiasinga.gif}

By properly setting the voltage $V_{CC}$ (not too low) and $R_B$ (not too 
large), the voltage $V_{BE}$ can be approximated as a constant value of
$0.7V$, as shown in the input characteristic plot: 

\htmladdimg{../figures/fixedbias2.gif}

Then the base current can be estimated to be:
\[
I_B=\frac{V_{CC}-V_{BE}}{R_B} %\approx \frac{V_{CC}}{R_B}	
\]
%The approximation is valid only if $V_{BE}=0.7$ is much smaller than $V_{CC}$.
The collector current is $\beta$ times $I_B$ if the transistor is in linear
region:
\[
I_C=\beta I_B=\beta \frac{V_{CC}-V_{BE}}{R_B} %\approx \beta \frac{V_{CC}}{R_B}
\]
The output voltage is 
\[
V_{CE}=V_{CC}-I_C R_C	
\]
As both $I_C$ and $V_{CE}$ depend on $\beta$, which may differ for different
transistors and change depending on the temperatures the operating point may
be unstable and inconsistent. 


{\bf Example 1}

In a fixed biasing transistor circuit, $R_B=200\,k\Omega$, $\beta=100$, 
$V_{CC}=12\,V$, find $R_C$ so that the DC operating point is in the middle
of the linear region of the output characteristic plot, i.e., 
$V_{CE}=V_{CC}/2=6\,V$.

\[ 
I_B=\frac{V_{CC}-V_{BE}}{R_B}=\frac{12-0.7}{200}=0.056\,mA,\;\;\;\;\;
I_C=\beta I_B=5.6\,mA 
\]
For this DC operating point with $I_C=5.6\,mA$ to be in the middle of the
load line, we need to have 
\[
V_{CE}=V_{CC}-I_CR_C=12-5.6 R_C=6\,V,\;\;\;\;\;\mbox{i.e.}\;\;\;\;\;\;
R_C=\frac{6\,V}{5.6\,mA}\approx 1.1\,k\Omega
\]
or
\[ 
\frac{V_{CC}}{R_C}=\frac{12\,V}{R_C}=2\times I_C=11.2\;mA,\;\;\;\;\;\mbox{i.e.}
\;\;\;\;\;R_C=\frac{12\,V}{11.2\,mA}\approx 1.1\,k\Omega
\]
so that
\[
V_{CE}=V_{CC}-R_CI_C=12\,V-1.1\,k\Omega\times 5.6\,mA=5.84\,V\approx 6\,V
\]
is indeed in the middle of the load line.



{\bf Example 2:} 

In a circuit of fixed biasing, $V_{CC}=12V$, $R_B=200\,k\Omega$, 
$R_C=1\,k\Omega$. Find the operating point $(V_{CE},\;I_C)$ for 
$\beta=50,\;100,\; 200$.

\begin{itemize}
\item Find $I_B$. We assume $V_{BE}=0.6\;V$ (may not be valid if 
  $R_B$ is too large):
  \[ 
  I_B=\frac{V_{CC}-V_{BE}}{R_B}=\frac{12-0.6}{200\times 10^3}=0.057 \;mA 
  \]

\item Find $I_C=\beta\;I_B$ 
\item Find $V_C=V_{CC}-R_C I_C$
\end{itemize}

The load line corresponding to the equation 
$V_C=V_{CC}-I_C R_C$, determined by these two points:
\begin{itemize}
\item Open-circuit voltage: $I_C=0,\; V_C=12\,V$
\item Short-circuit current: $V_C=0,\; I_C=12\,V/1\,k\Omega=12\,mA$
\end{itemize}

To minimize distortion, the DC operating point needs to be in the 
middle of the load line at $(V_C=12/2=6\,V,\;I_C=12/2=6\,mA)$. 

For different $\beta$ values, we get
\[
\begin{array}{c||c|c|c}\hline
          & \beta=50 & \beta=100 & \beta=200 \\\hline\hline
I_C\;(mA) &  2.85    &  5.70     &  11.4 \\\hline
V_C\;(V)  & 9.15     &  6.3      &  0.6 \\\hline
\end{array}
\]

We see that
\begin{itemize}
\item $\beta=50$, too close to cutoff region.
\item $\beta=100$, in the middle of linear region as desired.
\item $\beta=200$, too close to the saturation region.
\end{itemize}

The DC operating point (Q-point) of this fixed biasing circuit is
not completely determined by the parameters of the circuit such 
as the resistors, as it is also directly affected by factors such 
as $\beta$ value and temperature. This situation can be improved by 
introducing negative feedback into the circuit.


\item {\bf Self-Biasing} 

To correct the problem above, self-biasing circuit shown below can be
used to decrease the effect of changing $\beta$ by negative feed back
due to the introduction of $R_E$.

\htmladdimg{../figures/transistorbiasingb.gif}

Qualitatively, an increased $I_C$ (due to increased for some reason
such as $\beta$, temperature) will cause the following to happen:
\[
I_C \uparrow \Longrightarrow V_E \uparrow \Longrightarrow V_{BE} 
\downarrow \Longrightarrow I_B \downarrow \Longrightarrow 
I_C=\beta I_B \downarrow	
\]
This is a negative feedback loop that tends to stabilize the operating
point.

Quantitatively, we can analyze the circuit by first finding the base 
voltage and base current. 
\begin{itemize}
\item If the base current is much smaller than the current through 
  $R_2$, i.e., $I_B \ll I_2$, then the basis voltage $V_B$ can be
  approximated to be (voltage divider):
  \[
  V_B = V_{CC} \;\frac{R_2}{R_1+R_2}	
  \]
  Applying KVL to the base-emitter loop of the circuit, we get
  \[
  V_B-V_{BE}-I_ER_E=V_B-V_{BE}-(I_C+I_B)R_E=V_B-V_{BE}-(\beta+1)I_BR_E=0 
  \]
  Solving for $I_B$, we get
  \[ 
  I_B=\frac{V_B-V_{BE}}{(\beta+1)R_E},\;\;\;\;\;\;\;\;\;\;\;\;\;
  I_C=\beta I_B=\frac{\beta(V_B-V_{BE})}{(\beta+1)R_E}
  \approx \frac{V_B-V_{BE}}{R_E} =\frac{V_E}{R_E} 
  \]
  Note that $I_C$ is completely determined by $R_1$, $R_2$, and $R_E$, 
  as well as $V_{CC}$, but it is independent of $\beta$.

\item If the condition $I_B \ll I_2$ is not satisfied, the method above 
  is no longer valid. In this case, we can use Thevenin's theorem to replace 
  the base circuit by an open circuit voltage $V_{Th}=V_B$ (already found 
  above), in series with the internal resistance $R_{Th}$:
  \[
  R_{Th}=R_B=R_1||R_2=\frac{R_1R_2}{R_1+R_2}	
  \]

  \htmladdimg{../figures/transistorbiasing1.gif}

  Applying KVL to the base loop we get 
  \[
  V_B-I_BR_B-V_{BE}-(I_C+I_B)R_E=V_B-V_{BE}-I_BR_B-(\beta+1)I_BR_E=0 
  \]
  Solving this equation for $I_B$, we get:
  \[
  I_B=\frac{V_B-V_{BE}}{(\beta+1) R_E+R_B},\;\;\;\;\;\mbox{and}\;\;\;\;\;\;
  I_C =\beta I_B=\frac{\beta(V_B-V_{BE})}{(\beta+1) R_E+R_B} 
  \]
  If $R_E$ is not too small (strong negative feedback effect) and 
  $R_B=R_1||R_2$ is not too large (strong voltage divider effect), 
  so that $\beta R_E \gg R_B=R_1||R_2$ (e.g., $\beta R_E \ge 10 R_B$ 
  even for a small $\beta$), then $I_C$ can be approximated as
  \[
  I_C \approx \frac{V_B-V_{BE}}{R_E} 
  \]
  which is the same as what we got previously. In this case, $I_C$, and 
  thereby $V_{CE}$ and the DC operating point, is determined only by the
  resistors of the circuit, independent of the $\beta$. Comparing this 
  with fixed biasing with $I_C \approx \beta (V_{CC}-V_{BE})/R_B$ directly 
  proportional to $\beta$, the self-biasing circuit has a much more stable 
  operating point.
\end{itemize}

{\bf Example 3:} 

In the circuit of self-biasing, $V_{CC}=12V$, $R_1=100\,k\Omega$, 
$R_2=36\,k\Omega$, $R_E=1\,k\Omega$, $R_C=2\,k\Omega$, Assume 
$V_{BE}=0.7$. The load line is determined by this equation:
\[
V_{CE}=V_{CC}-I_CR_C-I_ER_E\approx V_{CC}-I_C(R_C+R_E)
\]
corresponding to these two points at:
\begin{itemize}
\item $V_{CE}=0$ and $I_C=V_{CC}/(R_C+R_E)=12/(2+1)=4\,mA$
\item $V_{CE}=12V$ and $I_C=0\,mA$
\end{itemize}

To minimize distortion, the desired operating point should be in the 
middle of the load line at $V_{CE}=12/2=6V$ and $I_C=4/2=2\,mA$.

\begin{itemize}
\item Based on the voltage divider approximation, we get the DC
operating point independent of $\beta$:
\begin{eqnarray}
  R_B&=&R_1R_2/(R_1+R_2)=26.5\,k\Omega	
  \nonumber\\
  V_B&=&V_{CC} R_2/(R_1+R_2)=3.18\;V
  \nonumber\\
  V_E&=&V_B-V_{BE}=3.18-0.7=2.48\;V
  \nonumber\\
  I_C&\approx& I_E=V_E/R_E=2.48\;mA
  \nonumber\\
  V_C&=&V_{CC}-R_CI_C=7.05\;V
  \nonumber\\
  V_{CE}&=&V_C-V_E=4.57\;V 
  \nonumber
\end{eqnarray}

\item Based on the Thevenin theorem, we get more accurate results:
\begin{eqnarray}
  I_B&=&\frac{V_B-V_{BE}}{(\beta+1) R_E+R_B}
  \nonumber\\
  I_C&=&\beta I_B
  \nonumber\\
  V_E&=&I_E R_E=(I_C+I_B)R_E
  \nonumber\\
  V_B&=&V_E+V_{BE}
  \nonumber\\
  V_C&=&V_{CC}-I_CR_C
  \nonumber\\
  V_{CE}&=&V_C-V_E
  \nonumber
\end{eqnarray} 
The DC operating points corresponding to the three $\beta$ values 
can be found to be:
\[
\begin{array}{c||c|c|c}\hline
      & \beta=50  & \beta=100 & \beta=200 \\\hline\hline
I_C\;(mA)  & 1.60 & 1.94     & 2.18      \\\hline
V_E\;(V)   & 1.63 & 1.96     & 2.19      \\\hline
V_C\;(V)   & 8.80 & 8.11     & 7.65      \\\hline
V_B\;(V)   & 2.33 & 2.66     & 2.89      \\\hline
V_{CE}\;(V) & 7.17 & 6.15     & 5.46      \\\hline
\end{array}
\]
\end{itemize}

We see that in all three cases, $I_C\approx 2\;mA$, 
$V_{CE}\approx 6\,V$, i.e., the DC operating point is always
close to the middle of the load line.

{\bf Example 4}

In a self-biasing transistor circuit, $R_1=R_2=100\,k\Omega$, $R_E=2\,k\Omega$, 
$\beta=100$, $V_{CC}=12V$, find $R_C$ so that the DC operating point is 
in the middle of the linear region of the output characteristic plot.

We first convert the base circuit into its Thevenin's equivalent voltage 
source composed of
\[ 
R_{Th}=R_B=R_1||R_2=50\;k\Omega,\;\;\;\;\;\;\;
V_{Th}=V_B=V_{CC}\frac{R_2}{R_1+R_2}=6\,V 
\]
Then we get
\[ 
I_C=\beta I_B=\frac{\beta(V_B-V_{BE})}{(\beta+1)R_E+R_B}
=\frac{100\times(6-0.7)}{101\times 2+50}=\frac{530}{250}=2.12
\]
and
\[
V_{CE}=V_{CC}-(I_C+I_B) R_E-I_CR_C\approx V_{CC}-I_C(R_E+R_C)
=12-2.12\times (2+R_C)
\]
To set this DC operating point to be in the middle of the load line, 
we need $V_{CE}=12/2=6\,V$, and solving the equation we get $R_C=0.83\,
k\Omega$.

{\bf Example 5} The circuit below shows yet another way to introduce
feedback to stablize the DC operating point. 

\htmladdimg{../figures/Example5.gif}

\begin{itemize}
\item The resistor $R_B$ connecting the collector to the base forms
  a feedback from the output $V_C$ to $V_B$ as well as providing a
  the forward baising needed for the base-emitter PN junction:
  \[
  V_C \uparrow \Longrightarrow V_B=V_{BE}\uparrow\Longrightarrow I_B 
  \uparrow \Longrightarrow I_C \uparrow \Longrightarrow V_C=V_{CC}-R_CI_C \downarrow
  \]
\item The DC operating point can be found by KVL:
  \[
  V_{CC}=R_C(I_B+I_C)+R_BI_B+0.7=(R_C(\beta+1)+R_B)I_B+0.7
  \]
  Solving this equation we get
  \[ 
  I_B=\frac{V_{CC}-0.7}{(\beta+1)R_C+R_B},\;\;\;\;\;
  I_C=\beta I_B=\frac{\beta(V_{CC}-0.7)}{(\beta+1)R_C+R_B}  
  \]
  Note that if $(\beta+1)R_C \gg R_B$, then $I_C \approx (V_{CC}-0.7)/R_C$, 
  independent of $\beta$. (But typically $R_B$ is significantly greater
  than $R_C$.)
\item Given $V_{CC}=10V$, $\beta=100$, and the desired $I_C=2\,mA$, find
  $R_C$ and $R_B$ so that the DC operating point is in the middle of the 
  linear region of the output characteristic plot:
  
  We want the DC operating point to be at $I_C=2mA$ and $V_C=5V$, and
  get
  \[
  R_C=(V_{CC}-V_C)/I_C=5V/2mA=2.5K\Omega
  \]
  and
  \[
  I_B=I_C/\beta=0.02mA, \;\;\;\;\;R_B=(5-0.7)/0.02=4.3/0.02=215K\Omega
  \].
\item 
  \begin{itemize}
  \item When $\beta=50$:
    \[
    I_B=\frac{V_{CC}-0.7}{(\beta+1)R_C+R_B}=\frac{9.3}{51\times 2.5K+215}
    =0.027mA 
    \]
    \[
    I_C=\beta I_B=1.36mA,\;\;\;V_C=V_{CC}-(\beta+1)I_B=6.6V 
    \]
  \item When $\beta=100$:
    \[
    I_B=\frac{V_{CC}-0.7}{(\beta+1)R_C+R_B}=\frac{9.3}{101\times 2.5K+215}
    =0.02mA 
    \]
    \[
    I_C=\beta I_B=2 mA,\;\;\;V_C=V_{CC}-(\beta+1)I_B=5V 
    \]
  \item When $\beta=200$:
    \[
    I_B=\frac{V_{CC}-0.7}{(\beta+1)R_C+R_B}=\frac{9.3}{201\times 2.5K+215}
    =0.013mA 
    \]
    \[
    I_C=\beta I_B=2.6mA,\;\;\;V_C=V_{CC}-(\beta+1)I_B=3.5V 
    \]
  \end{itemize}
\end{itemize}


\section*{Small-Signal Model and H parameters}

{\bf Two-port circuit:} 

\htmladdimg{../figures/twoportmodel.gif}
\htmladdimg{../figures/transistorHmodel.gif}

A transistor circuit can be treated as a two-port circuit with input 
and output ports and four variables ($v_1=b_{be},\;i_1=i_b,\;v_2=v_{ce},
\;i_2=i_c$). In general there are $C_4^2=4!/[2!(4-2)!]=6$ ways to choose
any two of the four variables as the independent variables while the 
other two as the dependent variables (functions). For example, three 
of these six choices are:
\[
\left\{ \begin{array}{l} v_1=f_1(i_1,i_2) \\ v_2=f_2(i_1,i_2)
\end{array} \right.,
\;\;\;\;\;\;\;\;
\left\{ \begin{array}{l} i_1=f_3(v_1,v_2) \\ i_2=f_4(v_1,v_2)
\end{array} \right.,
\;\;\;\;\;\;\;\;
\left\{ \begin{array}{l} v_1=f_5(i_1,v_2) \\ i_2=f_6(i_1,v_2)
\end{array} \right.
\]
We use the third {\em hybrid} model to describe the CE transistor circuit 
with $v_1=v_{be}$, $i_1=i_b$, $v_2=v_{ce}$, and $i_2=i_{ce}$:
\[
\left\{ \begin{array}{l} v_{be}=v_{be}(i_b,v_{ce}) \\ i_c=i_c(i_b,v_{ce})
\end{array} \right.
\]
Taking the total derivative of the above, we get:
\[
dv_{be}=\frac{\partial v_{be}}{\partial i_b} d i_b
+\frac{\partial v_{be}}{\partial v_{ce}} d v_{ce} 
=h_i d i_b+h_r d v_{ce}	
\;\;\;\;\;\;
di_c=\frac{\partial i_c}{\partial i_b} d i_b
+\frac{\partial i_c}{\partial v_{ce}} d v_{ce} 
=h_f d i_b+h_o d v_{ce}
\]
where $h_i, h_f, h_r, h_o$ are the hybrid model parameters:
\begin{itemize}
\item $h_i=\partial v_{be}/\partial i_b=r_{be}$: input AC impedance
  with $v_{ce}=0$ (output AC short-circuit). This is the AC 
  resistance between base and emitter, the reciprocal of the 
  slope of the current-voltage curve of the input characteristics. 

\item $h_r=\partial v_{be}/\partial v_{ce}$: reverse transfer voltage
  ratio representing how $v_{ce}$ affects $v_{be}$ with $i_b=0$ (input 
  AC open-circuit). In general $h_i$ is small and can be ignored.

\item $h_f=\partial i_c/\partial i_b=\beta$: forward transfer current 
    ratio or current amplification factor with $v_{ce}=0$ (output AC
    short-circuit). Typically, $h_f=\beta$ is in the range of 100 to 
    200.

\item $h_o=\partial i_c/\partial v_{ce}=1/r_{ce}$: output admittance,
  with $i_b=0$ (input AC open-circuit). This is the slope of the 
  current-voltage curve in the output characteristics. In general 
  $h_o$ is small and can be ignored.
\end{itemize}
If the variations of the AC components of all these variables 
$i_b$, $v_{be}$, $i_c$ and $v_{ce}$ are small ($\Delta x\rightarrow dx$)
around the DC operating point $Q$ and far away from either the cutoff
or the saturation region, the non-linear quantities that describe the
input and output characteristics can be linearized as the following
{\em small signal model}:
\[
v_{be}=\frac{\partial v_{be}}{\partial i_b}
i_b+\frac{\partial v_{be}}{\partial v_{ce}} v_{ce}=h_i i_b+h_r v_{ce}	
\approx h_i i_b
\;\;\;\;\;\;
i_c=\frac{\partial i_c}{\partial i_b} i_b
+\frac{\partial i_c}{\partial v_{ce}} v_{ce}=h_f i_b+h_o v_{ce}
\approx h_f i_b
\]

\htmladdimg{../figures/hparameters.gif}

\htmladdimg{../figures/smallsignalmodelBJT1.gif}

In general, $h_r$ and $h_o$ are small and could be assumed zero to 
further simplify the model (right of the figure above) containing
only two components, a resistor $h_i=r_{be}$ and a current source 
$h_f I_B=\beta I_B$.

The base and emitter forms a PN-junction with a resistance $r_{be}$ 
\[	
h_i=r_{be}\approx \eta\,\frac{V_T}{I_B}	
\]
as discussed in the 
\htmladdnormallink{section of diodes}{node2.html}, which is not a
constant, but a function of current $I_B$ through the PN-junction 
between base and emitter. Typically, at room temperature $(V_T=26\,mV)$, 
if $I_B$ is approximately in the range of $0.05\sim 0.1 mA$, then 
$r_{be}$ is a few hundred ohms. 

Based on this small signal model, a transistor can be analyzed as a 
two-port circuit containing a resistor $r_{be}$ and a current source 
$\beta I_B$.

In summary, we see that there are two aspects of a transistor circuit:
\begin{itemize}
\item The DC operating point in terms of the DC currents $I_B$, $I_C$ 
  and $I_E$ and voltages $V_{BE}$ and $V_{CE}$;
\item The AC small-signal model by which all of nonlinear voltage-current 
  relationships associated with the transistor are linearized based 
  on the assumption that the signal is small and the dynamic range 
  in totally within the linear region of both the input and output
  characteristic plots. 
\end{itemize}
When analyzing the transistor circuit with an AC input signal (riding 
on top of a DC input), we need to consider both aspects. If the DC 
operating point is set up properly, i.e., in the middle of the linear
region of output characteristic plot, and if the signal is small enough 
so that the dynamic range is inside the linear region, then the linear 
small-signal model applies and the circuit can be analyzed as a linear 
system.


\section*{AC equivalent circuits}

As discussed before, the voltage a circuit receives from a source 
depends on its input impedance $R_{in}$ as well as the internal 
impedance $R_s$ of the source, while the voltage it delivers depends
on its output impedance $r_{out}$ as well as the load impedance $R_L$.
It is therefore important to consider these input and output impedances
of an amplification circuit as well as its voltage gain.

\htmladdimg{../figures/AmplifierSourceLoad.gif}

In the first figure, everything inside the red box, including the 
amplifier as well as $V_S$ and $R_s$, is treated as the source, while 
everything inside the blue box, including the amplifier as well as $R_L$,
is treated as the load. Given the amplifier as well as the source $V_S$ 
and $R_s$, and the load $R_L$, we need to find the following three
parameters so that the red and blue boxes in the first figure can be
modeled by the corresponding boxes in the second figure:

\begin{itemize}
\item Input impedance $r_{in}$
\item Output impedance $r_{out}$ 
\item voltage gain $A$
\end{itemize}

Consider the typical transistor AC amplification circuit below:

\htmladdimg{../figures/ACamplification1.gif}

If the capacitances of the coupling capacitors and the emitter by-pass 
capacitor are large enough with respect to the frequency of the AC signal 
in the circuit is high enough, these capacitors can all be approximated as 
short circuit. Moreover, note that the AC voltage of the voltage supply
$V_{CC}$ is zero, it can be treated the same as the ground. Now the AC
behavior of the transistor amplification circuit can be modeled by the 
following small signal equivalent circuit:

\htmladdimg{../figures/ACamplification2a.gif}

\htmladdimg{../figures/ACamplifierModel.gif}

As shown above, this AC small signal equivalent circuit can be modeled 
by as an active circuit containing three components:

\begin{itemize}
\item {\bf AC Input Impedance:} 

  For AC signals, the input of the amplification circuit is shown below, 
  where $R_s$ is the internal resistance of the signal source, and the 
  input impedance of the circuit is the three resistances $R_1$, $R_2$ and 
  $r_{be}$ in parallel:
  \[	
  r_{in}=R_1||R_2||r_{be}
  =\left(\frac{1}{R_1}+\frac{1}{R_2}+\frac{1}{r_{be}}\right)^{-1}
  =\frac{R_1 R_2 r_{be}}{R_1R_2+R_1r_{be}+R_2r_{be}}\approx r_{be}	
  \]
  The approximation is due to the fact that typically $R_1,\;R_2\gg r_{be}$.
\item {\bf AC Output Impedance:} 

  This is simply the resistance of the resistor $r_{out}=R_C$.

\item {\bf AC Voltage Gain:} 

  Given the AC input voltage $v_{in}$, the base voltage and current are
  \[
  v_b \approx v_{in}\;\frac{r_{be}}{r_{be}+R_s},\;\;\;\;\;\;\;\;\;\;
  i_b=\frac{v_b}{r_{be}}\approx \frac{v_{in}}{r_{be}+R_s}
  \]
  The collector current is $i_c=\beta i_b$ and collect voltage is
  \[
  v_{out}=v_c\approx -i_c\;(R_C||R_L) = -\beta\;i_b\;(R_C||R_L)	
  =-\beta \frac{v_{in}}{r_{be}+R_s} (R_C||R_L)
  \]
  Here the negative sign indicates the fact that $v_c$ is $180^\circ$ out of
  phase with $v_b$.
%  \[	
%  v_b \uparrow \Longrightarrow i_b \uparrow \Longrightarrow i_c \uparrow
%  \Longrightarrow v_c \downarrow	
%  \]
  The voltage gain is:
  \[
  A=\frac{v_{out}}{v_{in}}=\frac{v_c}{v_{in}}
  \approx -\beta \frac{R_C||R_L}{r_{be}+R_s}
  \]
  If 
  \[	
  R_s\ll r_{in}=R_1||R_2||r_{be} \approx r_{be},\;\;\;\;\;R_L\gg r_{out}=R_C 
  \]
  then the gain can be approximated as
  \[
  A=\frac{v_{out}}{v_{in}}\approx -\beta \frac{R_C}{r_{be}}	
  \]
  To achieve higher gain $A$, we want to have smaller $r_{be}$ and 
  greater $R_C$. However, this also means the input resistance
  $r_{in}\approx r_{be}$ is small and the output resistance 
  $r_{out}\approx R_C$ is large, neither is desirable.

  Note that $r_{be}$ is not constant. As shown 
  \htmladdnormallink{before}{../ch4/node2.html}, 
  $r_{be}\approx \eta\;V_T/I_B$ of the base-emitter PN-junction is 
  approximately inversely proportional to $I_B$.

  Also note that $R_C$ and $I_B$ affects the DC operating point.
  Distortion may be caused if $R_C$ or $I_B$ is set properly.
  
\end{itemize}

{\bf Example 1:} 

% \htmladdimg{../figures/transistorBJTexample1.gif}
\htmladdimg{../figures/DCACloadlineEx.gif}

$V_{CC}=12V$, $R_B=300\,k\Omega$, $R_C=R_L=4\,k\Omega$, and $\beta=40$. 
We further assume $R_s=0$, and the capacitances are large enough so 
that they can be considered as short circuit for AC signals.

% \htmladdimg{../figures/transistorBJTexample1a.gif}
% \htmladdimg{../figures/transistorBJTexample1b.gif}

\begin{itemize}
\item Find base current:
  \[
  I_B=\frac{V_{CC}-V_{BE}}{R_B} \approx \frac{V_{CC}}{R_B}=\frac{12}{300\times 10^3}
  =40\times 10^{-6}\; A=40\;\mu A	
  \]

\item Find DC load line determined by the following two points:
  \begin{eqnarray}
    I_C=0, && V_{CE}=12V\nonumber\\
    V_{CE}=0, && I_C=\frac{V_{CC}}{R_C}=\frac{12\;V}{4\times 10^3\;\Omega}
    =3\times 10^{-3}A=3 \;mA
    \nonumber
  \end{eqnarray}

\item Find DC operating point $Q$: 
  \[
  I_C=\beta I_B=40\times 40\;\mu A=1.6\;mA,\;\;\;\;\;\;\;
  V_{CE}=V_{CC}-R_CI_C=12-4\times 1.6=5.6\;V
  \]

\item Find AC load line: 

  The AC load is $R'_L=R_C||R_L=2\;k\Omega$. The AC load line is a 
  straight line passing the DC operating point with slope $1/R'_L=1/2$.
  The intersections of the AC load line with $V_{CE}$ and $I_C$ axes can 
  be found by
  \[
  \frac{1.5}{V_{AC0}-6}=\frac{1}{2}\;\;\;\;\Longrightarrow V_{AC0}=8.8\,V
  \]
  \[
  \frac{I_{AC0}-1.5}{6}=\frac{1}{2}\;\;\;\;\Longrightarrow I_{AC0}=4.4\;mA 
  \]
\item Find input voltage and current:
  
  Assume AC input voltage is $v_{in}=20\;\sin\;\omega t\;mV$ and 
  $V_{BE}=0.6V$, the overall base voltage is
  \[
  v_{be}(t)=V_{BE}+v_{in}=(0.6+0.02\;\sin\;\omega t)\;V
  \]
  and the corresponding base current can be found graphically from the 
  input characteristics to be 
  \[
  i_b(t)=I_B+i_{in}=(40+20\;\sin\;\omega t)\;\mu A
  \] 
  between 20 and 60 $\mu A$. 
\item Find $r_{be}$
  
  \[
  r_{be}=\frac{\Delta v_{be}}{\Delta i_b} =\frac{20\,mV}{20\,\mu A}=1\;k\Omega
  \]

\item Find AC output voltage: 

  The output current is
  \[
  i_c(t)=\beta\; i_b(t)=40\times(40+20\sin\omega t)\;\mu A
  =(1.6+ 0.8 \;\sin\omega t)\,mA
  \]
  The output voltage is
  \[ 
  v_c(t)=V_{AC0}-R'_L\;i_c(t)=8.8-2\times(1.6+0.8\;\sin\omega t)
  =(5.6-1.6 \;sin\;\omega t)\;V=v_{out}(t)
  \]

\item Find the AC voltage gain:
  \[
  A_v=\frac{|v_{out}|}{|v_{in}|}=\frac{1.6}{0.02}=80	
\]
\end{itemize}

\htmladdimg{../figures/DCACloadline.gif}
\htmladdimg{../figures/transistorBJTexample1c.gif}

The circuit above can also be analyzed using the small-signal model. 

% \htmladdimg{../figures/BJTexamplesmallsignal.gif}
\htmladdimg{../figures/DCACloadlineExModel.gif}

Same as before, $r_{be}=1 \,k\Omega$, and we have the following DC variables:
\begin{eqnarray}
  I_B&=&(V_{CC}-V_{BE})/R_B \approx V_{CC}/R_B=40 \;\mu A	
  \nonumber \\
  I_C&=&\beta\;I_B=40\times 40\;\mu A=1.6\;mA	
  \nonumber \\
  V_{CE}&=&V_{CC}-R_C I_C=12-1.6\times 4=5.6\;V	
  \nonumber
\end{eqnarray}
% \[	r_{be}=V_T/I_B=26\;mV/40\mu A=650 \;\Omega	\]
The AC variables:
\[
i_b=v_{in}/r_{be}, \;\;\;\;v_c=-i_c\; (R_C||R_L)
\]
The voltage gain is:
\[
A_v=\frac{v_c}{v_{in}}=-\frac{\beta\; i_b\;(R_C||R_L)}{r_{be}\,i_b}
=-\frac{\beta\,(R_C||R_L)}{r_{be}}
=-\frac{(40\times 2)\;k\Omega}{1\;k\Omega}=-80	
\]
The input resistance is $r_{in}=R_B||r_{be}\approx r_{be}=1\,k\Omega$,
the output resistance is $r_{out}=R_C=4\;k\Omega$.

%\begin{comment}
{\bf Example 2:}

Consider the circuit below with its AC small-signal model:

\htmladdimg{../figures/Example5.gif}
\htmladdimg{../figures/Example5a.png}

We can find the voltage gain, the input and output resistances when 
$R_S=0$ and $R_L=\infty$ ($v_{in}=v_b,\;\;v_{out}=v_c$).
\begin{itemize}
\item AC voltage gain:

  Apply KVL to the collector to get
  \[
  \frac{v_c-v_s}{R_B}+\beta i_b+\frac{v_c}{R_C}
  =\frac{v_c-v_s}{R_B}+\beta \,\frac{v_s}{r_{be}}+\frac{v_c}{R_C}=0  
  \]
  i.e.,
  \[
  v_C\left(\frac{1}{R_B}+\frac{1}{R_C}\right)
  =v_s\left(\frac{1}{R_B}-\beta\frac{1}{r_{be}}\right)
  \]
  Solving this we get $v_c$ and then find the AC voltage gain:
  \begin{eqnarray}
  A&=&\frac{v_{out}}{v_{in}}=\frac{v_c}{v_s}
  =\frac{1/R_B-\beta/r_{be}}{1/R_B+1/R_C}=\frac{R_C(r_{be}-\beta R_B)}{(R_B+R_C)r_{be}}
  \approx-\beta \,\frac{R_C R_B}{(R_B+R_C)r_{be}}
  \nonumber\\
  &=&-\beta \frac{R_B||R_C}{r_{be}}=-\beta\frac{R_C}{r_{be}}\frac{R_B}{R_B+R_C}
  \stackrel{R_B\rightarrow\infty}{\Longrightarrow}-\beta\frac{R_C}{r_{be}}   
  \nonumber
  \end{eqnarray}
  where the approximation is due to the fact that $r_{be}\ll \beta R_B$.
  We see that when $R_B\rightarrow\infty$, i.e., there is not negative
  feedback, the gain becomes the same as the result before without feedback.

\item AC input resistance:

  The input resistance is the parallel combination of $r_{be}$ and
  $r'_{in}$, the resistence of the circuit to the right of the base. 
  First, we realize that $i_b=v_b/r_{be}=v_s/r_{be}$, and convert the
  current source $\beta i_b$ in parallel with $R_C$ to a voltage source
  $\beta i_bR_C$ in series with $R_C$, and then get the current into the
  circuit as:
  \[
  i=\frac{v_b-(-R_C\beta i_b)}{R_B+R_C}=\frac{v_b+R_C \beta v_b/r_{be}}{R_B+R_C}
  =v_b \frac{1+R_C \beta /r_{be}}{R_B+R_C}  
  \approx v_b \frac{\beta R_C /r_{be}}{R_B+R_C}
  \]
  where the approximation is due to the fact that $r_{be}\ll \beta R_C$. 
  The resistance can then be found as
  \[
  r'_{in}=\frac{v_b}{i}
  =\frac{R_B+R_C}{\beta R_C /r_{be}}=\frac{R_B+R_C}{\beta R_C}\,r_{be}
  \]
  and the total input resistance is:
  \[
  r_{in}=r_{be}||r'_{in}=r_{be}\frac{1}{1+\beta R_C/(R_B+R_C)}
  =r_{be}\frac{R_B+R_C}{R_B+(\beta+1) R_C}
  =\left\{\begin{array}{ll} r_{be} & R_B\rightarrow\infty\\
  r_{be}/\beta & R_B\rightarrow 0\end{array}\right.
  \]
\item AC output resistance:

  \begin{itemize}
  \item Open-circuit voltage with $R_L=\infty$ is simply
    \[
    v_{oc}=A v_b=-\beta\frac{R_C}{r_{be}}\frac{1}{1+R_C/R_B}\,v_b
    \]

  \item Short-circuit current with $R_L=0$:
    Short-circuit current by $v_b$ alone is $i'=v_b/R_B$,
    Short-circuit current by current source $\beta i_b$ alone is
    $i''=-\beta i_b=-\beta v_b/r_{be}$. The total short-circuit current is
    \[
    i_{sc}=i'+i''=v_b\left(\frac{1}{R_B}-\frac{\beta}{r_{be}}\right)
    \]
  \item Output resistance:
    \begin{eqnarray}
    r_{out}&=&\frac{v_{oc}}{i_{sc}}=\frac{-\beta\frac{R_C}{r_{be}}\frac{1}{1+R_C/R_B}}
    {1/R_B-\beta/r_{be}}
    =\frac{-\beta\frac{R_C}{r_{be}}\frac{1}{1+R_C/R_B}}{\frac{r_{be}-\beta R_B}{R_B r_{be}}}
    \nonumber\\
    &\approx& \frac{-\beta\frac{R_C}{r_{be}}\frac{R_B}{R_B+R_C}}{\frac{-\beta R_B}{R_B r_{be}}}
    =\frac{R_C R_B}{R_C+R_B}=R_C||R_B\stackrel{R_B\rightarrow\infty}{\Longrightarrow}R_C
    \nonumber
    \end{eqnarray}

  \end{itemize}

\end{itemize}
%\end{comment}

{\bf Example 3:}

Consider both the DC operating point and its AC small signal model of the
circuit below:

\htmladdimg{../figures/Example6.png}

Apply KVL to get
\[
V_{CC}-V_{BE}=I_B R_B+(\beta+1)I_BR_E,\;\;\;\;\;
I_B=\frac{V_{CC}-V_{BE}}{R_B+(\beta+1)R_E}
\]
and
\[
I_C=\beta I_B=\beta\,\frac{V_{CC}-V_{BE}}{R_B+(\beta+1)R_E},\;\;\;\;\;
V_{CE}\approx V_{CC}-I_C(R_C+R_E)
\]
We assume $V_{CC}=12V,\;R_E=0.1\,k\Omega,\;\beta=100$, and consider the
following for the DC operating point to be in the middle of the linear 
region:
\begin{itemize}
\item Given $R_C=1.9\,k\Omega$, we need to find $R_B$:
  \[
  I_E=\frac{V_{CC}-V_{CE}}{R_C+R_E}=\frac{6}{0.1+1.9}=3\,mA,\;\;\;\;
  I_B=\frac{I_C}{\beta}=0.03\,mA
  \]
  But 
  \[
  R_B+(\beta+1)R_E=\frac{V_{CC}-V_{BE}}{I_B}=\frac{12-0.7}{0.03}=377\,k\Omega
  \]
  \[
  R_B=377-100\times 0.1=367\,k\Omega
  \]
\item Alternatively, given $R_B=250\,k\Omega$, we can find $R_C$:
  \[
  I_B=\frac{V_{CC}-V_{BE}}{R_B+\beta R_E}
  =\frac{12-0.7}{250+100\times 0.1}=\frac{11.3}{260}=0.044
  \]
  \[
  \frac{V_{CC}-V_{CE}}{R_C+R_E}=\frac{6}{0.1+R_C}=I_C=\beta I_B=4.4\,mA
  \]
  Solving to get $R_C=1.26\,k\Omega$.
\end{itemize}

Next consider the AC equivalent circuit based on small-signal model of the
transistor in the dashed line box:

\htmladdimg{../figures/Example6a.png}

As $R_B$ is significantly greater than all resistors in the circuit, it can
be ignored in the AC analysis. Apply KCL to the emitter to get
\[
i_b+\beta i_b=(\beta+1)i_b=\frac{v_e}{R_E}=\frac{v_{in}-r_{be}i_b}{R_E}
\]
Solving for $i_b$:
\[
i_b=\frac{v_{in}}{r_{be}+(\beta+1)R_E}
\]
\begin{itemize}
\item AC input resistance:
  \[
  r_{in}=\frac{v_{in}}{i_{in}}=\frac{v_{in}}{i_b}=r_{be}+(\beta+1)R_E
  \approx \beta R_E
  \]
\item AC voltage gain:
  \[
  A=\frac{v_{out}}{v_{in}}=-\frac{\beta i_b (R_C||R_L)}{v_{in}}
  =-\frac{\beta\,(R_C||R_L)}{v_{in}}\,\frac{v_{in}}{r_{be}+(\beta+1)R_E} 
  =-\frac{\beta\,R_C||R_L}{r_{be}+(\beta+1)R_E}
  \approx -\frac{R_C||R_L}{R_E}
  \]
\item AC output resistance:
  \begin{itemize}
  \item Open-circuit voltage ($R_L=\infty$):
    $v_{oc}=-\beta i_b R_C$
  \item Short-ciruit current ($R_L=0$):
    $i_{sc}=-\beta i_b$
  \item Output resistance:
    \[
    r_{out}=\frac{v_{oc}}{i_{sc}}=R_C
    \]
  \end{itemize}
\end{itemize}
In conclusion, the negative feedback introduced by $R_E$ increases the 
input resistance, and stabalizes the DC operating point as well as the 
AC voltage gain.


\section*{Emitter Follower}

The input and output of an emitter follower are the base and the 
emitter, respectively, and the collector is at AC zero. The circuit 
is therefore a common-collector circuit (for AC).

\htmladdimg{../figures/emitterfollower.gif}

% Assume $\beta=50$, $R_B=300K\Omega$, $R_E=4K\Omega$, $R_s=500\Omega$, 
% $R_L=5.1K\Omega$, and $V_{CC}=12$.

The negative feedback effect due to $R_E$ can be shown qualitatively:
\[
v_{out}=v_e\uparrow \Longrightarrow v_{be} \downarrow \Longrightarrow 
i_b \downarrow \Longrightarrow i_c \downarrow \Longrightarrow 
v_e\downarrow
\]
The DC operating point can be found as:
\[ 
V_{CC}=R_B I_B+V_{BE}+R_E I_E=R_B I_B+V_{BE}+R_E (\beta+1)I_B
\]
Solving this equation for $I_B$ we get:
\[
I_B=\frac{V_{CC}-V_{BE}}{(\beta+1)R_E+R_B} 
\]
\[
I_E=(\beta+1) I_B=\frac{(\beta+1)(V_{CC}-V_{BE})}{(\beta+1)R_E+R_B} 
\]
\[
V_{CE}=V_{CC}-V_E=V_{CC}-R_E I_E 
\]

{\bf Example}

Assume $R_B=100\,k\Omega$, $V_{CC}=10\,V$, $\beta=100$. Find $R_E$ so
that the DC operating point is in the middle of the load line.

For $V_{CE}$ to be in the middle of the load line, we need to have
$V_{CE}=V_{CC}-V_E=10-V_E=V_{CC}/2=5\,V$, i.e., $V_E=5\,V$:
\[
V_E&=&I_E R_E =R_E\;\frac{(\beta+1) (V_{CC}-V_{BE})}{(\beta+1)R_E+R_B}
=\frac{101\;R_E(10-0.7)}{101\;R_E+100}=5\,V
\]
Solving the equation for $R_E$ we get $R_E=1.15\,k\Omega$. Now we have
\[
I_B=\frac{V_{CC}-V_{BE}}{(\beta+1)R_E+R_B}=0.043\,mA,
\;\;\;\;\;I_C=\beta\,I_B=100\,I_B=4.3\,mA
\]
\[
I_E=(\beta+1)I_B=4.34\,mA,\;\;\;\;V_E=I_ER_E=5\,V
\]

{\bf AC small-signal equivalent circuit}

Based on this small signal model of the transistor, 

\htmladdimg{../figures/TransistorModel.png}

we can find the small signal of the emitter follower:

\htmladdimg{../figures/EmitterFollower1.png}\htmladdimg{../figures/emitterfollower2.gif}

Based on this small signal model, the three system parameters: 
voltage gain, input resistance, and output resistance can be 
obtained as shown below.

\begin{itemize}
\item {\bf AC voltage gain:} 

  As $R_B$ is significantly greater than $R_S$ and $R_{be}$, it can be 
  neglected in the analysis below.

%    With an AC input signal $v_{in}=\cos(\omega t)$, we have
%    \[
%    v_{in}=i_br_{be}+(\beta+1)i_bR_E=[r_{be}+(\beta+1)R_E]i_b
%    \]
%    The AC output is:
%    \[
%    v_{out}=(\beta+1)R_Ei_b
%    \]
%    The AC voltage gain is:
%    \[
%    A=\frac{v_{out}}{v_{in}}=\frac{(\beta+1)R_E}{r_{be}+(\beta+1)R_E}\approx 1
%    \]
%    as $r_{be}\ll(\beta+1)R_E$, $A$ is smaller than 1 but close to 1.

  \[
  \left\{ \begin{array}{l} 
    v_{out}=v_e=(R_E||R_L)i_e =(R_E||R_L)(i_c+i_b)=(R_E||R_L)(\beta+1) i_b 	\\
    v_{in}=(R_s+r_{be})i_b +v_{out}=(R_s+r_{be})i_b+(R_E||R_L)(\beta+1) i_b 
  \end{array} \right. 
  \]
  The voltage gain can be found to be:
  \[
  A=\frac{v_{out}}{v_{in}}
  =\frac{(\beta+1) (R_E||R_L)}{(R_s+r_{be})+(\beta+1) (R_E||R_L)} \approx 1 
  \]
  As $R_s+r_{be} \ll (\beta+1) (R_E||R_L)$, the voltage gain $A$ is smaller 
  than but approximately equal to 1. Note that $A$ is positive, i.e., the 
  output voltage is in phase with the input voltage.

\item {\bf Input resistance:} 

  The input resistance is $R_B$ in parallel with the resistance of the
  circuit to right of the base of the transistor, including the load 
  $R_L$, which can be found as the ratio of the voltage $v_b$ and the 
  current $i_b$. 
  \[
  v_b=i_b r_{be}+v_{out}=i_b\, [ r_{be}+(\beta+1)(R_E||R_L) ]
  \]
  therefore we get
  \[
  r'_{in}=\frac{v_b}{i_b}=r_{be}+(\beta+1) (R_E||R_L)
  \approx r_{be}+\beta (R_E||R_L)\approx \beta (R_E||R_L)
  \]
  and 
  \[
  r_{in}=R_B||r'_{in} \approx r'_{in}\approx \beta (R_E||R_L)
  \]
  Comparing this with the input resistance of the common-emitter circuit
  $r_{in}=R_1||R_2|| r_{be} \approx r_{be}$, we see that the emitter follower 
  has much higher input resistance.

%\htmladdimg{../figures/EmitterFollower2.png}

\item {\bf Output resistance:}

  The output resistance is the parallel combination of $R_E$ and the 
  resistance $r'_{out}$ of the circuit to the left of the emitter of the
  transistor (including $R_s$), which can  be found as the ratio of the 
  open-circuit voltage $v_{oc}$ (with $R_L=\infty$) and the short-circuit 
  current $i_{sc}$ (with $R_L=0$). 
  \begin{itemize}
  \item Find open-circuit voltage $v_{oc}$: 

    $v_{oc}$ is approximately the same as the source voltage $v_{oc}\approx v_{in}$,
    as the voltage gain of the emitter follower is close to unity.

  \item Find short-circuit current $i_{sc}$:
    \[
    i_{sc}=i_e=(\beta+1)i_b=(\beta+1)\frac{v_{in}}{r_{be}+R_s}
    \approx \beta \frac{v_{in}}{r_{be}+R_s}
    \]
  \end{itemize}
  We therefore get
  \[
  r'_{out}=\frac{v_{oc}}{i_{sc}} \approx\frac{r_{be}+R_s}{\beta} 
  \]
  and the overall output resistance can therefore be found to be
  \[
  r_{out}= R_E || r'_{out}=R_E || \left(\frac{r_{be}+R_s}{\beta}\right)
  \approx \frac{r_{be}+R_s}{\beta}	
  \]

\end{itemize}

{\bf Conclusion:} 

The emitter follower is a circuit with deep negative feedback, i.e., 
100\% of its output $v_{out}=v_e$ is fed back to become part of its input 
$v_{be}$. The fact that this is a negative feedback can be seen by:
\[ 
v_e \uparrow \Longrightarrow v_{be} \downarrow
<\Longrightarrow i_e \downarrow \Longrightarrow v_e \downarrow 
\]
Due to this deep negative feedback, it has the following properties:
\begin{itemize}
\item The voltage gain is smaller than but very close to unity. 
\item The input resistance is large $r_{in}=\beta \,(R_E||R_L)$
\item The output resistance is small $r_{out}=(r_{be}+R_s)/\beta$.
\end{itemize}

The emitter follower acts as an impedance transformer with a ratio 
of $\beta$, i.e., the input resistance is $\beta$ times greater than 
$R_E||R_L$ and the output resistance is $\beta$ times smaller than 
$R_s+r_{be}$. 

Although the emitter follower circuit does not amplify the signal voltage, 
it drastically improves the input and output resistances, compared with 
the input resistance $r_{in}=r_{be}$ and output resistances $r_{out}=R_C$ of 
the common-emitter transistor circuit. Specifically, due to its high input
resistance $R_{in}$, it draws very little current from the source and causes
little internal voltage drop in the source, and also due to its low output 
resistance $R_{out}$, I can drive heavy load (low $R_L$) without lowering 
the output voltage. It is therefore widely used as both the input and output
stages for a multi-stage voltage amplification circuit.


\begin{comment}
\htmladdimg{../figures/CCCE.png}

$V_{CC}=10\,V$, $\beta=100$, $R_{b1}=100\;k\Omega$, $R_{e1}=0.5\,k\Omega$,
\[
V_{e1}=\frac{V_{CC}-V_{be}}{R_{b1}+(\beta+1)R_{e1}}(\beta+1)R_{e1}
=\frac{9.3}{100+101\times 0.5}(101\times 0.5)=3.1\,V
\]
\[
V_{e2}=V_{b2}-V_{be}=3.1-0.7=2.4\,V,\;\;\;\;\;\;
I_{c2}=\frac{V_{e2}}{R_{e2}}=\frac{2.4}{0.5}=4.8\,mA
\]
For $V_{ce2}=V_{CC}/2=5\,V$, voltage across $R_{c2}$ needs to be 
\end{comment}

\section*{Multi-stage Amplification}

In order to have an amplification gain, multi-stage amplification circuits
are needed. Such a circuit is typically composed of two or more cascaded
transistor amplifiers, coupled in one of three possible ways:
\begin{itemize}
\item Capacitor coupling:
\htmladdimg{../figures/coupling_capacitor.gif}
	\begin{itemize}
	\item Independent DC operating point;
	\item AC amplification of high gain if coupling capacitors are
		larger enough;
	\item Cannot amplify DC and low frequency signals;
	\item Difficult implementation on IC.
	\end{itemize}

\item Transformer coupling:
\htmladdimg{../figures/coupling_transformer.gif}
	\begin{itemize}
	\item Independent DC operating point;
	\item Can achieve maximal power by impedance match;
	\item Cannot amplify DC and low frequency signals;
	\item Difficult implementation on IC.
	\end{itemize}

\item Direct coupling:
\htmladdimg{../figures/coupling_direct.gif}
	\begin{itemize}
	\item DC operating point not independent;
	\item Can amplify both DC and AC signals;
	\item Easy implementation on IC.
	\end{itemize}

\end{itemize}


\section*{Typical Transistor Circuits}

\begin{itemize}

\item {\bf Differential Amplifier}

Differential amplifier amplifies the difference $\triangle v=v_1-v_2$ 
between two voltages $v_1$ and $v_2$. Differential amplification has 
many applications, such as the first stage of 
\htmladdnormallink{operational amplifiers (Op-amps)}{http://fourier.eng.hmc.edu/e84/lectures/opamp/node2.html}.

The two transistors $T_1$ and $T_2$ in the circuit are identical with
the same properties, and their emitters are connected to a current source
with constant current so that $I_E=I_1+I_2$. If $I_1$ increases, $I_2$
will decrease, and vice versa. Consider these three cases:
\begin{itemize}
\item When the two input voltages are the same $v_1=v_2$, i.e., $\Delta V=v_1-v_2=0$,
  then $I_1=I_2=I_E/2$ and the output voltage is $V_{out}=V_{CC}-I_2 R_C$, which
  is treated as a reference level corresponding to $\Delta V=0$.
\item If $v_1>v_2$, the following changes take place:
  \[ v_1 \uparrow \Longrightarrow I_1 \uparrow \Longrightarrow I_2 
  \downarrow \Longrightarrow v_{out}=V_{CC}-I_2 R_c \uparrow \]
\item If $v_1<v_2$, the following changes take place:
  \[ v_1 \downarrow \Longrightarrow I_1 \downarrow \Longrightarrow I_2 
  \uparrow \Longrightarrow v_{out}=V_{CC}-I_2 R_c \downarrow \]
\end{itemize}
The output voltage $V_{out}$ can be further amplified to indicate the 
difference and its polarity between the two input voltages $v_1$ and 
$v_2$.

A simple current source is also shown in the figure. The base voltage 
$V_B$ of the transistor is fixed at approximately $0.7\times 3=2.1V$, 
so that the load current $I_L=I_{ce}$ is also approximately constant, 
independent of the load, i.e., the circuit can be used as a current source 
providing a current determined by $V_B$ but independent of the load. A 
better way to hold $V_B$ constant is to replace the diodes by a reverse biased
\htmladdnormallink{Zener diode}{https://en.wikipedia.org/wiki/Zener_diode}.
When a zener diode is reversely biased by a voltage exceeding its
{\em breakdown voltage}, the voltage drop across it, $V_B$ in the circuit,
is held at the breakdown voltage, a constant value independent of any
other variables in the circuit. Consequently $I_L$ is also constant.


%\htmladdimg{../figures/differential_amplify.gif}
\htmladdimg{../figures/DifferentialAmplifier.png}


\item {\bf Current Mirror circuits}

  The current mirror circuit shown below is a simple current source
  that provides a constant current $I_L$ independent of the load $R_L$. 

  \htmladdimg{../figures/CurrentMirror.png}  

  This circuit is composed of two matching transistors $Q_1$ and $Q_2$
  with identical behaviors such as the input and output characteristics
  and $\beta_1=\beta_2=\beta$. They are the input and output stages of
  the circuit, respectively. As the input, the reference current $I_{ref}$
  can be determined as
  \[
  I_{ref}=\frac{V_{CC}-V_B}{R_C}
  \]
  Applying the KCL to the collector of $Q_1$, and realizing $\beta\gg 1$,
  we also get
  \[
  I_{ref}=I_C+2I_B=I_C(1+2/\beta),\;\;\;\;\;\mbox{i.e.}\;\;\;\;
  I_C=I_{ref}/(1+2/\beta)\approx I_{ref}
  \]
  \begin{itemize}
  \item In the input stage, as $Q_1$'s collector and base are short-circuited
    ({\em diode connected}), it behaves like a diode, in terms of the 
    relationship between $V_B=V_C$ and $I_B=I_C/\beta$, the voltage across
    and current through the base-emitter 
    \htmladdnormallink{PN junction}{../node2.html}:
    \[
    V_B=V_T\ln \left(\frac{I_B}{I_0}+1\right)
    \approx V_T\;\ln \left(\frac{I_B}{I_0}\right)
    \]
    where $I_0$ is the reverse saturation current of the BE PN-junction,
    $V_T$ is the thermal voltage. Note that the current $I_C=\beta I_B\ne 0$
    when $V_{CB}=0$ (see the
    \htmladdnormallink{output characteriestics of the CB configuration}{node3.html}).

    We see that here $Q_1$ is actually used as a current-voltage converter
    that converts the collector current $I_C$ to an output voltage $V_B$, 
    which is held constant due to negative feedback loop:
    \[
    V_B\uparrow \Longrightarrow I_B\uparrow \Longrightarrow 
    (I_C=\beta I_B) \uparrow \Longrightarrow (V_C=V_B)\downarrow
    \] 
    As $V_B=V_{C1}$ is solely determined by $I_{C1}\approx I_{ref}=(V_{CC}-V_B)/R_C$, 
    $I_{B2}$ and therefore $I_{C2}=\beta I_{B2}=I_L$ will be constant independent 
    of the load $R_L$.
    
  \item In the output stage, as $Q_1$ and $Q_2$ are identical and 
    $V_{B2}=V_{B1}=V_B$, we have $I_{B2}=I_{B1}=I_B$ and $I_{C2}=\beta I_B=I_{C1}=I_C$. 
    The load current is determined by $R_C$ but independent of the load $R_L$:
    \[
    I_L=I_C=I_{ref}/(1+2/\beta) \approx I_{ref}
    =\frac{V_{CC}-V_B}{R_C}
    \]
    Note that the discussion above is valid only if $I_C=\beta I_B$ holds,
    i.e., both $Q_1$ and $Q_2$ must be working in the linear (active) region 
    away from either the cutoff or saturation region.
  \end{itemize}

  \htmladdnormallink{\bf Wilson current mirror:}{https://en.wikipedia.org/wiki/Wilson_current_mirror}

  
  \htmladdimg{../figures/WilsonCurrentMirror.png}  
  
  \[
  (I_L=I_{C3})\uparrow \Longrightarrow I_{C2}\uparrow \Longrightarrow 
  I_{C1}\uparrow\Longrightarrow (V_{C1}=V_{B3})\downarrow \Longrightarrow 
  I_{B3}\downarrow \Longrightarrow (I_L=I_{C3})\downarrow
  \]

\item \htmladdnormallink{{\bf Different types of amplification circuits}}{http://en.wikipedia.org/wiki/Amplifier}

By properly settng the DC operating point of the transistor circuit, 
it can be working in any one of the following modes:
\begin{itemize}
\item Class A: The transistor remains conducting in the entire sinusoidal
  cycle (conduction angle $\theta=360^\circ$). The DC operating point is 
  in the middle of the linear range of the transistor to minimize distortion
  (clipping). However, the DC power consumption is maximized even the AC 
  sinusoidal signal is zero. 
  
\item Class B: The transistor is conducting and amplifies the AC signal
  only in half of the sinusoidal cycle ($\theta=180^\circ$), while it is
  turned off and consumes no energy for the other half.

\item Class AB: this is intermediate between class A and B, the two 
  transistors are active and conducting current more than half of the
  time.

\item Class C: Less than half of the signal cycle is used (conduction angle
  $\theta<180^\circ$)

\end{itemize}

  The \htmladdnormallink{Push-pull circuit}{http://www.learnabout-electronics.org/Amplifiers/amplifiers54.php} can be considered as a class AB amplifier that is 
  typically used as the last stage of an amplification system, such as 
  in an op-amp circuit, for power amplification with large current and
  low output resistance to drive a heavy load (small $R_L$). A push-pull
  circuit is composed of a pair of two transistors that work in alternation
  during the two half cycles of the sinusoidal signal. The circuit can be 
  implemented in either of the following two ways:
  \begin{itemize}
  \item The push-pull pair (one NPN, the other PNP) receives the same 
    input signal from their bases. During the positive half cycle, the 
    NPN transistor is conductive and drives current through the load 
    $R_L$, while PNP transistor is cutoff; during the negative half cycle, 
    the PNP NPN transistor is conductive and draws current from the load 
    $R_L$, while NPN transistor is cutoff. In either polarity, the output 
    resistance, the conducting transistor, is very small.

  \item The push-pull pair (both NPN) receives the input signal $180^\circ$ 
    out of phase (e.g., from a transformer, or from the collector and emitter
    of the transistor in previous stage). The transistor receiving positive
    peak of the input is active and drives current through $R_L$, with small
    output resistance, while the other transistor receiving negative peak 
    is cutoff (open-circuit). During the next half cycle, the two transistor 
    switch roles with the conducting transistor drawing current from the load.
    
  \end{itemize}


  \htmladdimg{../figures/PushPull.png}
  \htmladdimg{../figures/PushPull2.png}
  \htmladdimg{../figures/PushPull1.png}



\item {\bf Oscillators}

  An oscillator is a circuit that receives no input but generates a sinusoidal 
  output at a desired frequency. A typical oscillator circuit is based on an
  active component (a transistor or an op-amp) with positive feedback and an 
  LC circuit (tank circuit). Initially trigged by switching on the circuit, 
  the LC circuit starts to resonate at frequency $\omega_0=1/\sqrt{LC}$, and 
  the active component with positive feedback compensates for the attenuation 
  due to the inevitable resistance in the circuit and keeps the oscillation 
  going.

  Specifically, the Hartley and Colpitts oscillators are two typical oscillation
  circuits. In either cases, a transistor amplifier is used to receive positive 
  feedback taken from the LC circuit as a collector impedance $Z_c$, which is 
  maximized at the resonant frequency, thereby the voltage gain of this circuit 
  is also maximized. A fraction of the sinusoidal at the collector is positively 
  fed back to the emitter to prevent attenuation.
  
  \begin{itemize}
  \item \htmladdnormallink{Hartley Oscillator}{http://www.learnabout-electronics.org/Oscillators/osc21.php}
    The feedback signal is taken from the $L$ branch between $L_1$ and $L_2$ in series.
    The resonant frequency is $\omega_0=1/\sqrt{C(L_1+L_2)}$.

    \htmladdimg{../figures/Hartley.png}
    \htmladdimg{../figures/Colpitts.png}

  \item \htmladdnormallink{Colpitts Oscillator}{http://www.learnabout-electronics.org/Oscillators/osc23.php}
    The feedback signal is taken from the $C$ branch between $C_1$ and $C_2$ in series.
    The resonant frequency is $\omega_0=1/\sqrt{LC_1C_2/(C_1+C_2)}$.

  \end{itemize}
  In both circuits, the feedback is a fraction of the output $v_c=v_{out}$
  \[
  v_{feedback}=v_c\frac{Z_{L_1}}{Z_{L_1}+{Z_{L_2}}}=v_c\frac{L_1}{L_1+L_2},\;\;\;\;\;\;
  v_{feedback}=v_c\frac{Z_{C_2}}{Z_{C_2}+{Z_{C_3}}}=v_c\frac{C_3}{C_2+C_3}
  \]
  and it is then sent to the emitter of the transistor, which is in phase with the 
  collector connected to the LC circuit, i.e., the feedback is indeed positive:
  \[
  v_c\uparrow \Longrightarrow v_{feedback} \uparrow \Longrightarrow 
  v_e\uparrow \Longrightarrow v_{be} \downarrow 
  \Longrightarrow i_b\downarrow  \Longrightarrow i_c\downarrow
  \Longrightarrow v_c\uparrow
  \]

% https://en.wikipedia.org/wiki/RC_oscillator

\item \htmladdnormallink{\bf Frequency Mixer}{https://en.wikipedia.org/wiki/Frequency_mixer}

When a transistor is used for amplification, its DC operating point of 
a type A amplifier is typically set in the middle of the load line to 
maximize the linear dynamic range. By so doing, the signal distortion 
will be minimized by avoiding the nonlinear region of the transistor 
circuit.

However, in some applications, the nonlinear behavior of the transistor 
circuit is taken advantage of, such as in a {\em frequency mixer}. As 
discussed previously, the output current $I_C$ is approximately an 
exponential function of the input voltage $V_{BE}$:
\[
i_c=\beta i_b=\beta\, I_o\, (e^{v_{be}/V_T}-1) 
\]
and in general an exponential function can be approximated by the first
few terms of the Taylor series expansion:
\[
e^x=\sum_{k=0}^\infty \frac{x^k}{k!},\;\;\;\;\;\mbox{i.e.}\;\;\;\;\;
e^x-1=x+\frac{1}{2} x^2+\frac{1}{3!} x^3+\cdots
 \approx x+\frac{1}{2} x^2 
\]
If the input voltage $v_{be}=\cos(\omega_1t)+\cos(\omega_2t)$ contains 
two frequency components, then the output current can be approximated as:
\begin{eqnarray} 
  i_c &\propto& (\cos\omega_1t+\cos\omega_2t) +\frac{1}{2}(\cos\omega_1t+\cos\omega_2t)^2 
  +\cdots 
  \nonumber \\
  &\approx &\cos\omega_1t+\cos\omega_2t+\frac{1}{2}\left(\cos^2\omega_1t+2\cos\omega_1t\;\cos\omega_2t +\cos^2\omega_2t\right) 
  \nonumber \\
  &=&\cos\omega_1t+\cos\omega_2t
  +\frac{1}{4}(1+\cos(2\omega_1t))
  +\frac{1}{2}\cos(\omega_1+\omega_2)t  +\frac{1}{2}\cos(\omega_1-\omega_2)t
  +\frac{1}{4}(1+\cos(2\omega_2t))
  \nonumber
\end{eqnarray}
where we have used the trigonometry identities:
\[ 
\cos^2\alpha=\frac{1+\cos 2\alpha}{2},\;\;\;\;\;
\cos\alpha\;\cos\beta=\frac{\cos(\alpha+\beta)+\cos(\alpha-\beta)}{2} 
\]
We see that $i_c$ contains many new frequency components in addition 
to the two original frequencies $\omega_1$ and $\omega_2$, including 
$\omega_1+\omega_2$, $\omega_1-\omega_2$, $2\omega_1$, and $2\omega_2$. 
This transistor circuit is therefore called a frequency mixer. By properly 
filtering in the circuit following this mixer, one of such frequencies, 
such as the difference frequency $\omega_1-\omega_2$ (the ``beat frequency'') 
is amplified, while all other frequency components are suppressed. 

Note that the specific nonlinear behavior of the circuit is not important, 
as the Taylor series expansion of any nonlinear function will contain constant,
first and second order terms as the exponential function assumed above, and 
the same frequency components will result. Frequency mixer is an important 
component in 
\htmladdnormallink{super-heterodyne reception}{http://en.wikipedia.org/wiki/Superheterodyne_receiver} 
which is used in all modern radio and TV broadcasting. Here the frequency 
$\omega_{OS}$ of the local oscillator is changed by a variable capacitor,
which can be adjusted jointly with the capacitor of the tuning circuit, so 
that the $\omega_{OS}$ of the local oscillator changes with the carrier 
frequency $\omega_{RF}$ (radio frequency) of the broadcast signal received
by the antenna in such a way that their difference is always a constant:
\[
\Delta\omega=\omega_{OS}-\omega_{RF}=\omega_{IF}
\]
This frequency $\omega_{IF}$ is called the 
\htmladdnormallink{\em intermediate frequency}{http://en.wikipedia.org/wiki/Intermediate_frequency}. 

In radio reception, $\omega_{IF}=455$ kHz for AM and $\omega_{IF}=10.7$ MHz for 
FM. The reason for this frequency shift from $\omega_{RF}$ to $\omega_{IF}$ is
for the amplification circuit of the receiver to be specialized for this
intermediate frequency, instead of a very wide range of all possible broadcast
frequencies.

\htmladdimg{../figures/superheterodyne1.png}

The circuit diagram of a simple super-heterodyne radio receiver is shown below.
Note that the first transistor is an oscillator that also receives signal from
the LC tuning circuit at the base, i.e., it is also a mixer that mixes two
frequencies. The next two transistors amplify frequency component the signal from the mixer, but


\htmladdimg{../figures/RadioReceiver.png}

\end{itemize}



\section{Colpitts Oscillators}

Oscillation in a circuit is undesirable if the circuit is an
\htmladdnormallink{amplifier}{https://en.wikipedia.org/wiki/Amplifier}
or part of a control system which needs to be stable without oscillation.
However, oscillation is desirable in many applications such as sinusoidal
signal generator, carrier signal generation is broadcast transmission
(radio and TV), clock signal in digital systems, etc.

An \htmladdnormallink{oscillator}{https://en.wikipedia.org/wiki/Electronic_oscillator}
is a feedback system composed of a forward path with gain $G(j\omega)$ 
and a feedback path with gain $F(j\omega)$:

\htmladdimg{../figures/OscillatorModel1.png}

For the system to oscillate at a certain frequency, the feedback 
needs to be positive for the frequency to be positively reinforced
while passing through the forward path in order to sustain the output
$V_o$ with zero input $V_i=0$. Specifically, the output $V_o$ and the 
input $V_i$ of a feedback system are related by
\[
V_o=G(V_i+FV_o)=GV_i+GFV_o,\;\;\;\;\;\;\;\;\;\frac{V_o}{V_i}=H=\frac{G}{1-GF},
\;\;\;\;\;\;\;V_o=HV_i=\frac{G}{1-GF}\,V_i
\]
where $GF$ is the open-loop gain and $H$ is the closed-loop gain.
For this system to oscillate, i.e., for it to produce an output with
zero input, its closed-loop gain needs to be infinite, i.e., its 
open-loop gain $GF$ need to be real, with zero phase $\angle(GF)=0$
and unit gain $|GF|=1$. 

%(In practice, $|GF|\ge 1$ to compensate energy attenuation in the physical system.)



\begin{comment}
The op-amp circuit shown in the figure has both positive and negative
feedback branches. If the voltage to non-inverting input $V_+$ is 
considered as the input, the circuit is a non-inverting amplifier 
with gain:
\[
A=\frac{V_o}{V_+}=1+\frac{R_f}{R_3}=K
\]
On the other hand, the positive feedback gain is
\[
B(j\omega)=\frac{V_+}{V'_o}=\frac{Z_1}{Z_1+Z_2}
=\frac{R_1||1/j\omega C_1}{R_1||1/j\omega C_1+R_2+1/j\omega C_2}
\]
\end{comment}

There exist many different configurations of oscillators based on a
single transistor. Shown below are three typical 
\htmladdnormallink{Colpitts oscillators}{https://en.wikipedia.org/wiki/Colpitts_oscillator}:
common-base (CB, left), common emitter (CE, middle), and common 
collector (CC, right). All such circuits contain a ``tank'' LC circuit
composed of an inductor $L$ in parallel with $C_1$ and $C_2$ in series, 
with a resonant frequency 
\[
\omega_0=\frac{1}{\sqrt{LC_s}},\;\;\;\;\;\;\mbox{where}\;\;\;\;\;
C_s=\left(\frac{1}{C_1}+\frac{1}{C_2}\right)^{-1}
\]
where $C_s$ is the equivalent capacitance of the series combination 
of $C_1$ and $C_2$. All other $C$s (without a subscript) are coupling
capacitors that have a large enough capacitance and can therefore be 
treated as short circuit for AC signals. 

%\htmladdimg{../figures/ColpittsOscillators.png}
\htmladdimg{../figures/Colpitts3a.png}

Here are the requirements for these circuits to oscillate:
\begin{enumerate}
\item an LC tank tuning circuit that generates sinusoidal oscillation 
  at its resonant frequency $\omega_0=1/\sqrt{LC_s}$
\item a positive feedback loop that sustains the oscillation.
\end{enumerate}
How each of these circuits works can be qualitatively understood as 
below:
\begin{itemize}
\item CB with the base AC grounded: The collector voltage $v_c$ is the 
  output, a fraction of which at the middle point between the two 
  capacitors, ``tap point'', is fed-back to the emitter to a positive 
  feedback loop:
  \[
  v_c\uparrow\Longrightarrow v_e\uparrow\Longrightarrow v_{be}\downarrow
  \Longrightarrow i_b\downarrow \Longrightarrow i_c=\beta i_b\downarrow
  \Longrightarrow v_c\uparrow
  \]
\item CE with the emitter AC grounded: The collector voltage $v_c$ is 
  the output, which is fed-back through the LC tank circuit to the base. 
  As the tap point is grounded, the sinusoidal voltage across the LC
  tank produces opposite voltage polarities at the far ends of $C_1$ 
  and $C_2$, i.e., $v_{C_1}=v_b$ and $v_{C_2}=v_c$ have opposite phases 
  and thereby form a positive feedback loop:  
  \[
  v_c\uparrow\Longrightarrow v_b\downarrow\Longrightarrow v_{be}\downarrow
  \Longrightarrow i_b\downarrow \Longrightarrow i_c=\beta i_b\downarrow
  \Longrightarrow v_c\uparrow
  \]
\item CC with the collector AC grounded: This a voltage follower circuit
  in which the emitter voltage $v_e$ is the output that follows the input
  voltage $v_b$. The feedback from the emitter through the LC tank circuit 
  to the base form a positive feedback loop:
  \[
  v_e=v_t\uparrow\Longrightarrow v_b\uparrow
  \Longrightarrow i_b\uparrow\Longrightarrow i_e=(\beta+1)i_b\uparrow
  \Longrightarrow v_e\uparrow
  \]
  where $v_t$ is the voltage at the tap point.

\end{itemize}

More specifically, we consider the common-collector circuit as an example.
To find out why the circuit oscillates and the resonant frequency, we 
disconnect the base path of the circuit and consider the open-loop gain 
of $H=V_o/V_i$ of the feedback loop. We further model the transistor 
by a Thevenin voltage source $V_i$ in series with an internal $R$, as 
shown in the figure:

\htmladdimg{../figures/ColpittsModel2.png}

As the load of the Thevenin source, the tank circuit receives an input 
$V_t$ at the tap point, and produces an output $V_o$ across the parallel
combination of $L$ and $C_1$ in series with $C_2$. Applying KCL at the tap 
point we get:
\[
\frac{V_t-V_i}{R}+\frac{V_t}{1/j\omega C_2}+\frac{V_t}{j\omega L+1/j\omega C_1}=0
\]
i.e.,
\[
V_t\left(\frac{1}{R}+j\omega C_2+\frac{j\omega C_1}{1-\omega^2LC_1}\right)
=\frac{V_i}{R}
\]
Solving for $V_t$ we get
\[
V_t=\frac{1}{R(\frac{1}{R}+j\omega C_2+\frac{j\omega C_1}
{1-\omega^2LC_1})}\;V_i
=\frac{1}{1+j\omega R(C_2+C_1/(1-\omega^2LC_1)}\;V_i
\]
which is maximized if the frequency is such that the imaginary part of 
the denominator is zero:
\[
C_2+\frac{C_1}{1-\omega_0^2LC_1}=0,\;\;\;\;\;\;\;\mbox{i.e.}\;\;\;\;\;
\omega_0=\frac{1}{\sqrt{LC_1C_2/(C_1+C_2)}}=\frac{1}{\sqrt{LC_s}}
\]
Here $\omega_0$ is the resonant frequency, at which the voltage $V_t$ 
become the same as the source voltage $V_t=V_i$, as the impedance of
the tank circuit as the load of the Thevenin source is infinity:
\begin{eqnarray}
  Z_{tank}&=&Z_{C_2}||(Z_{C_1}+Z_L)=\frac{Z_{C_2}(Z_{C_2}+Z_L)}{Z_{C_2}+Z_{C_1}+Z_L}
  =\frac{(1/j\omega C_1+j\omega L)/j\omega C_2}
  {1/j\omega C_2+1/j\omega C_1+j\omega L}
  \nonumber\\
  &=&\frac{(1/j\omega C_1+j\omega L)/C_2}{1/C_1+1/C_2-\omega^2 L}
  =\frac{(1/j\omega C_1+j\omega L)/C_2}{1/C_s-\omega^2 L}
  \;\;\stackrel{\omega=\omega_0}{\Longrightarrow}\;\;\infty
  \nonumber
\end{eqnarray}
When $\omega=\omega_0$, the denominator becomes zeros and $Z_{tank}=\infty$,
i.e., there is no current drawn from the source by the tank circuit. 
Consequently, the voltage drop across $R$ is zero and the voltage received 
by the tank circuit is $V_t=V_i$. Now the output voltage $V_o$ can be found 
by voltage divider:
\[
V_t=\frac{Z_{C_2}}{Z_{C_1}+Z_{C_2}}\;V_o=\frac{C_1}{C_1+C_2}\,V_o,
\;\;\;\;\;\mbox{i.e.}\;\;\;\;\;\;\;\;
V_o=\frac{C_1+C_2}{C_1}\;V_t=\frac{C_1+C_2}{C_1}\;V_i
\]
The open-loop gain (from $V_i$ to $V_o$) is:
\[
H=\frac{V_o}{V_i}=\frac{C_1+C_2}{C_1}
\]

\begin{comment}
\begin{eqnarray}
  V_o&=&\frac{j\omega L}{j\omega L+1/j\omega C}\;V_t
  =\frac{\omega^2LC_1}{\omega^2LC_1-1}\;V_t
  \nonumber \\
  &=&\left(\frac{1-\omega^2LC_1}{1-\omega^2LC_1+j\omega R(C_1+C_2-\omega^2LC_1C_2)}\right)\;
  \left(\frac{\omega^2LC_1}{\omega^2LC_1-1}\right)\;V_i
  \nonumber \\
  &=&\frac{-\omega^2LC_1}{1-\omega^2LC_1+j\omega R(C_1+C_2-\omega^2LC_1C_2)}\;V_i
  \nonumber 
\end{eqnarray}

The open-loop gain (from $V_i$ to $V_o$) is:
\[
H=\frac{V_o}{V_i}=\frac{-\omega^2LC_1}{1-\omega^2LC_1+j\omega R(C_1+C_2-\omega^2LC_1C_2)}
\]
At the resonant frequency $\omega_0$, the imaginary part is zero, we have
\[
H=\frac{-\omega^2LC_1}{1-\omega^2LC_1}
=\frac{\omega_0^2LC_1}{\omega_0^2LC_1-1}
=\frac{LC_1/LC_s}{LC_1/LC_s-1}
=\frac{C_1}{C_1-C_s}
=\frac{C_1+C_2}{C_1}
\]
\end{comment}

We see that when $\omega=\omega_0$, the open-loop gain $H$ is real but
greater than 1. However, the non-linearity of the transistor as the 
feedback path (from $V_o$ to $V_i$) will force $HG$ to become 1. The 
circuit is an oscillator with frequency at $\omega_0=1/\sqrt{LC_s}$.

\begin{comment}
http://seit.unsw.adfa.edu.au/staff/sites/hrp/teaching/Electronics4/docs/PLL/colpitts.pdf

http://users.ece.gatech.edu/mleach/ece3050/notes/osc/wienbr.pdf

http://www.ece.msstate.edu/~winton/classes/ece3144/labs/Exp10.pdf

http://www.drp.fmph.uniba.sk/ESM/twin.pdf
\end{comment}


\section*{Metal-Oxide-Semiconductor Field-Effect Transistors}

A metal-oxide-semiconductor field-effect transistor (MOSFET) has three terminals,
source, gate, and drain. In an n-MOSFET (or p-MOSFET), both the source S and drain
D are N-type (or P-type) and the substrate between them is P-type (or N-type). 
The gate and the P-type substrate is insulated by a thin layer of silicon dioxide 
($SiO_2$). Due to this insulation, there is no gate current to either the source or
drain.

\htmladdimg{../figures/MOS_FET_transistors.gif}
\htmladdimg{../figures/MOSFET4.gif}

\htmladdimg{../figures/MOSFET.gif}

%{\bf Different types of MOSFET}
%\htmladdimg{../figures/MOSFET2.gif}


The n-MOSFET can be considered as a voltage-controlled switch. When sufficient
voltage $V_{GS}$ is applied between gate and source, the positive potential at 
the gate will induce enough electrons from the P-type substrate (minority 
carriers) to form an electronic channel called an {\em inversion layer} between
source and drain, and a current $I_{DS}$ between source and drain is formed. 
Qualitatively the conductivity between source and drain of an n-channel field 
effect transistor can be described as:

\[ 
\left\{ \begin{array}{l}
  V_{GS} \uparrow \Longrightarrow I_{DS} \uparrow \Longrightarrow \mbox{conducting} \\
  V_{GS} \downarrow \Longrightarrow I_{DS} \downarrow \Longrightarrow \mbox{cut off} 
\end{array} \right. 	
\]

\htmladdimg{../figures/MOSFET3.gif}

More accurately, the behavior of an n-channel MOSFET can be described by the 
function $I_{DS}=f(V_{GS}, V_{DS})$ with a threshold voltage $V_T$, as plotted 
below:

\htmladdimg{../figures/MOSFETplots.gif}

%\htmladdimg{../figures/MOSFETplot1.gif}

This function can be divided into three different (piece-wise linear) regions:

\begin{itemize}
  \item {\bf Cutoff region:} When $V_{GS}<V_T$, no current flows through S 
    and D, due to the two PN-junctions, i.e. $I_{DS}=0$, independent of 
    $V_{DS}$. This is illustrated in part (b) of the figure above.

  \item {\bf Triode region:} When both $V_{GS}>V_T$ and $V_{GD}=V_{GS}-V_{DS}>V_T$,
    (i.e., $V_{DS}<V_{GS}-V_T$), some electrons in the P-type substrate 
    (minority carriers) are pulled toward the gate to form an inversion 
    layer close to the gate to form an N-type channel with certain 
    resistance between S and D. $I_{DS}$ increases linearly as $V_{DS}$ 
    increases, with a coefficient $\Delta V_{DS}/\Delta I_{DS}$ (Ohm's law), 
    and nonlinearly as $V_{GS}$ increases (to pull more electrons toward 
    the gate to enhance the conductivity of the n-channel).

  \item {\bf Saturation region:} When $V_{GS}>V_T$ but $V_{DS}$ increases 
    further, the voltage $V_{GD}=V_{GS}-V_{DS}$ between gate and e-channel 
    close to the drain becomes small and the e-channel close to the drain
    narrows. In particular, when the voltage between gate and drain is
    smaller than the threshold voltage:
    \[ 
    V_{GD}=V_{GS}-V_{DS}<V_T 
    \]
    the e-channel at the D end is nearly closed ({\em pinch-off}), and
    $I_{DS}$ is saturated and maintained at a constant value even when
    $V_{DS}$ increases further, as higher $V_{DS}$ tends to draw more 
    electrons toward the drain on the one hand but also enhance the 
    pinch-off effect on the other. Now $I_{DS}$ is only affected by $V_{GS}$, 
    \[ 
    I_{DS}=\left\{ \begin{array}{ll}
      0 & \mbox{if $V_{GS}<V_T$ (cutoff region)} \\
      K(V_{GS}-V_T)^2 & \mbox{if $V_{GS}\ge V_T,\;\;V_{GD}=V_{GS}-V_{DS}\le V_T$,
        (saturation region)}       
    \end{array} \right. 
    \]
    The transistor behaves like a voltage-controlled current source. 
 \end{itemize}
The triode region and the saturation region are separated by 
$V_{GD}=V_{GS}-V_{DS}=V_T$. In the $I_{DS}$ vs $V_{DS}$ plot, they are
saperated by $I_{DS}=K(V_{GS}-V_T)^2=KV_{DS}^2$.

{\bf Example:} Assume $V_T=1V$.
\begin{itemize}
\item when $V_{GS}<V_T=1V$, the MOSFET is in cutoff region with $I_{DS}=0$ 
  independent of $V_{DS}$.
\item when $V_{GS}=2V > V_T=1V$ and $V_{DS}<V_{GS}-V_T=1V$, the MOSFET is in 
  linear or triode region with $I_{DS}$ affected by both $V_{GS}$ and $V_{DS}$.
\item when $V_{DS}>V_{GS}-V_T=1V$, the MOSFET is in saturation region with
  $I_{DS}$ determined only by $V_{GS}$.
\end{itemize}

{\bf Example:} Assume $K=2 mA/V^2$ and $V_T=1V$, and both MOSFETs in the
following circuit are in the saturation region. Find output voltage $V$.

\htmladdimg{../figures/MOSFETexample1.gif}

Since both MOSFETs are in saturation region with the same $I_{DS}$ which is
determined only by $V_{GS}$ but independent of $V_{DS}$, their $V_{GS}$ must 
be the same. The upper MOSFET must have the same $V_{GS}$ as the lower one 
$V_{GS}=2V$, i.e., the output voltage has to be $3V$.

{\bf Comparison between BJT and FET}

\begin{itemize}
\item BJT has a low input resistance $r_{in}$. But as MOSFET's gate is
  insulated from the channel ($r_{in}>10^{11} \Omega$), it draws virtually
  no input current and therefore its input resistance is infinity in theory.
\item BJT is current ($I_B$ or $I_E$) controlled, but MOSFET is voltage 
  ($V_{GS}$) controlled. Consequently, the power consumption of MOSFETs 
  is lower than BJTs.
\item MOSFETs are easy to fabricate in large scale and have higher element
  density than BJTs.
\item MOSFETs have thin insulation layer which is more prone to statics
  and requires special protection. 
\item BJTs have higher cutoff frequency and higher maximum current than 
  MOSFETs.
\item MOSFETs are much more widely used (especially in computers and 
  digital systems) than BJTs.
\item Both majority and minority carriers are used in BJTs, but only 
  majority carriers are used in FETs. Consequently FETs have better
  temperature stability (minority carriers are more sensitive to
  temperature).
  
\end{itemize}

The BJT and FET can be compared with the old technology of 
\htmladdnormallink{vacuum tube}{http://en.wikipedia.org/wiki/Vacuum_tube}.
Although the specific physics of each of these devices is quite different 
from others, the working principles of these devices are essentially the 
same. In all three devices, a small AC input voltage (signal) is applied
to the input terminal of the device (base, gate, or grid) to control the
current that flows through the device (from collector, drain, or plate to 
emitter, source, or cathode, respectively), causing a much amplified voltage
to appear at the output terminal (collector, drain, or plate) of the device.


\section{MOSFET Amplifier}

 \htmladdimg{../figures/MOSFETamplifier.gif}
 \htmladdimg{../figures/MOSFETtransfer.gif}

Assume in the circuit above $V_{in}=V_{GS}>V_T$ and the transistor is in 
saturation region, i.e., $V_{DS}>V_{in}-V_T$, then we have 
\[ 
\left\{ \begin{array}{l}
I_{DS}=K(V_{in}-V_T)^2 \\
V_{out}=V_{DS}=V_{dd}-I_{DS} R=V_{dd}-KR(V_{in}-V_T)^2  \end{array} \right.
\]
The second equation relates the output $V_{out}$ to the input $V_{in}$, as
shown by the red segment of the curve in the plot above. As the transistor is 
in saturation region,
\[ 
V_{DS}=V_{out}=V_{dd}-KR(V_{in}-V_T)^2\ge V_{in}-V_T 
\]
which can be solved for $V_{in}-V_T$ to get:
\[
V_{in}-V_T <\frac{-1+\sqrt{1+4KRV_{dd}}}{2KR},\;\;\;\;\mbox{i.e.}\;\;\;\;
V_{in}<\frac{-1+\sqrt{1+4KRV_{dd}}}{2KR}+V_T 
\]
We see that for the transistor to be in the saturation region, $V_{in}$
needs to satisfy both $V_{in}>V_T$ and the inequality above, i.e., it 
needs to be in the following range:
\[
V_T<V_{in}<\frac{-1+\sqrt{1+4KRV_{dd}}}{2KR}+V_T 
\;\;\;\;\mbox{i.e.}\;\;\;\;
0<V_{in}-V_T <\frac{-1+\sqrt{1+4KRV_{dd}}}{2KR}
\]

When the transistor is in saturation mode the slope of the curve (red) 
indicates the ratio between input $V_{out}$ and output $V_{in}$, the voltage
gain of the circuit:
\[
g=\frac{d V_{out}}{d V_{in}}=\frac{d}{d V_{in}}(V_{dd}-KR(V_{in}-V_T)^2 )
=-2KR(V_{in}-V_T) 
\]

{\bf Example:} Assume $V_{dd}=10V$, $R=10 k\Omega$, $K=0.5\;mA/V^2$, $V_T=1V$.
For the transistor to be in saturation region, we need
\begin{itemize}
\item $V_{in}>V_T=1\;V$ 
\item $V_{in}<[-1+\sqrt{1+4KRV_{dd}}\;]/2KR+V_T=2.32\;V $
\end{itemize}
then we have
\[
V_{out}=V_{dd}-K(V_{in}-V_T)^2 R=10-5(V_{in}-1)^2 
\]
and the voltage gain is a function of the input $V_{in}$:
\[ g(V_{in})=\frac{d}{d\,V_{in}} V_{out}(V_{in})=-10(V_{in}-1) \]
This nonlinear equation can be represented by the table below:
\begin{tabular}{l | lllllllllllll} 
$V_{in} $ & 0  & 1  & 1.4 & 1.5 & 1.8 & 1.9 & 2.0 & 2.1 & 2.2 & 2.3 & 2.32 & 2.35 & 2.4 \\ \hline
$V_{out}$ & 10 & 10 & 9.2 & 8.8 & 6.8 & 6.0 & 5.0 & 4.0 & 2.8 & 1.6 & 1.3  & 0.9  & 0.0 
\end{tabular}
In particular, when the input $V_{in}$ increases from $1.8V$ to $1.9V$, the output 
$V_{out}$ decreases from $6.8V$ to $6V$, with a gain $g(V_{in})=g(1.8)=-10(1.8-1)=-8$.
Also when the input $V_{in}$ increases from $2.0V$ to $2.1V$, the output $V_{out}$ 
decreases from $5V$ to $4V$, with a gain $g(V_{in})=g(2)=-10(2-1)=-10$. 

 \htmladdimg{../figures/MOSFETexample.gif}

In summary, we see that
\begin{itemize}
\item When the transistor is in saturation mode, the circuit behaves as a voltage amplifier.
\item The voltage gain is the slope of the tangent of the curve (red) as a function of $V_{in}$.
\item The value of the gain depends on the level of input $V_{in}$. When 
  $V_T=1V<V_{in}<2.32V$, the gain $|g|>1$ is greater than one.
\item The output voltage is $180^\circ$ out of phase with the input voltage ($g<0$),
  as the slope of $V_{out}(V_{in})$ is negative.
\item When $V_{in}<V_T=1V$, the transistor is cutoff.  On the other hand, when 
  $V_{in}>2.32V$, $V_{out}$ is more than one $V_T$ below $V_{in}$, for example, 
  $V_{in}=2.32$, $V_{out}=1.3<V_{in}-V_T=2.32-1=1.32$, the transistor is in triode
  region. In either of the two cases, the transistor has no amplification capability.
\end{itemize}

Next we consider a MOSFET circuit with sinusoidal input. Assume the drain resistor
is $R=10\;k\Omega$, $K=1\;mA/V^2$, $V_T=1V$, $V_{dd}=5V$ and a sinusoidal input 
$v_{in}(t)=V_{bias}+\sin(\omega t)$. If the bias voltage is $V_{bias}=1.5V$, the
input voltage $v_{in}(t)$ varies between 1.4V and 1.6V. The output voltage can be
found to be:
\[ v_{out}=V_{dd}-RI_{DS}=V_{dd}-RK(V_{in}-V_T)^2=5-10 (V_{in}-1)^2 \]
In particular, corresponding to $V_{in}=1.4V,\;1.5V,\;1.6V$, the output voltage 
$v_{out}$ and the current $i_{DS}$ are, respectively, $v_{out}=3.4V,\;2.5V,\;1.4V$, 
and $i_{DS}=0.16mA,\;0.25mA,\;0.36mA$, as shown in the figure below:

\htmladdimg{../figures/MOSFETloadline.gif}

{\bf Biasing:} In the example above, the DC offset of the input is at 1.5V, so that
the transistor is working in the saturation region when the magnitude of the AC input
is limited. However, if this offset is either too high or too low, the gate voltage
may go beyond the saturation region to enter either the triode or the cutoff region.
In either case, the output voltage will be severely distorted, as shown below:

\htmladdimg{../figures/MOSFETregions.gif}

It is therefore clear that the DC offset or biasing gate voltage has to be properly
setup to make sure the dynamic range of the input signal is within the saturation 
region. 

{\bf Method 1:} One way to provide the desired DC offset is to use two resistors 
$R_1$ and $R_2$ that form a voltage divider, as shown in the figure below (a). As 
the input resistance of a MOSFET transistor is very high, therefore the gate of 
the transistor does not draw any current, the DC offset voltage can simply obtained
as:
\[ V_{bias}=V_{dd}\frac{R_1}{R_1+R_2} \]
The input AC signal through the input capacitor is then superimposed on this DC 
offset. 

\htmladdimg{../figures/MOSFETbiasing.gif}

{\bf Method 2: } Another way to set up the bias is the circuit shown in (b) above.
Assume $R_1=84k$, $R_2=16k$, $R_d=20k$, $V_{dd}=10V$, and $K=0.5mA/V^2$.  The bias 
voltage can be found to be $V_B=V_{bias}=1.6V$, and the voltage between gate and 
source is $V_{GS}=V_B-V_{in}$. The output voltage is
\[ V_{out}=V_{dd}-R_d K (V_{GS}-V_T)^2=V_{dd}-R_d K (V_B-V_{in}-V_T)^2=10-10\times 
(0.6-V_{in})^2  \]
When $V_{in}=0$, $V_{out}=6.4V$.

To determine the dynamic range of the input $V_{in}$, recall the conditions for the
transistor to be in saturation region:
\begin{itemize}
\item To avoid cutoff region: $V_{GS}-V_T \ge 0$. For this particular circuit, 
  \[ V_{GS}-V_T=V_B-V_{in}-V_T=0.6-V_{in} \ge 0 \]
  Solving this we get $V_{in} \le 0.6V$ with corresponding output $V_{out} < 10V$.

\item To avoid triode region: $V_{DS} \ge V_{GS}-V_T$. For this particular circuit, 
%\[  V_{DS}=V_{out}-V_{in} \ge V_B-V_{in}-V_T \]
\[  V_{DS}=V_{out}-V_{in}=(V_{dd}-R_dI_{DS})-V_{in}
    =V_{dd}-R_dK (V_B-V_{in}-V_T)^2-V_{in} \ge V_B-V_{in}-V_T  \]
that is
\[  V_{out}=V_{dd}-R_dK (V_B-V_{in}-V_T)^2 \ge V_B-V_T \]
i.e.,
\[  V_{out}=10-10\times (0.6-V_{in})^2 \ge 0.6 \]
Solving this for $V_{in}$ we get $V_{in} \ge -0.37V $, with corresponding output
$V_{out} \ge 0.6V$
\end{itemize}
Therefore the overall dynamic range for the input is 
\[ -0.37V \le V_{in} \le 0.6V \]
with the corresponding output range
\[  0.6V \le V_{out} \le 10V  \]
and the overall voltage gain is about $g=9.7$. Note that the output voltage is in
phase with the input voltage.

{\bf Source Follower: } If the output is taken from the source, instead of the 
drain of the transistor, the circuit is called a source follower. 

\htmladdimg{../figures/MOSFETfollower.gif}

Assume $R_s=10 k\Omega$, $V_T=1V$ and $K=10 mA/V^2$. To find the input and output 
voltages and the gain of the circuit, consider the current $I_{DS}=V_{out}/R_s$:

\[ I_{DS}=K(V_{GS}-V_T)^2=K(V_{in}-V_{out}-V_T)^2=V_{out}/R_s \]
Plugging in the given values, we get
\[ (V_{in}-V_{out}-1)^2=V_{out} \]
If $V_{in}=2V$, this equation becomes:
\[ (V_{out}-1)^2=V_{out} \]
which can be solved to get $V_{out}=2.6V$ or $V_{out}=0.4V$. We take the smaller 
voltage in order for the transistor to be outside the cutoff region:
\[ V_{GS}=V_{in}-V_{out}=2-V_{out}>V_T=1 \]
Similarly, if $V_{in}=3V$, the equation becomes:
\[ (V_{out}-2)^2=V_{out} \]
and we get $V_{out}=1V$. The voltage gain of the source follower is 
\[ g=\frac{\Delta V_{out}}{\Delta V_{in}}=\frac{1-0.4}{3-2}=0.6<1 \]

To maximize the dynamic range for the input AC signal, the DC operation point
in terms of the DC variables $\{V_{GS}, I_{DS}, V_{DS}\}$ needs to be set around
the middle point of the saturation region. If the AC signal around the DC 
operation point is small enough, the behavior of the circuit can be linearized
(first term of Taylor expansion of the nonlinear relationship) to simplify the
analysis. 

\htmladdimg{../figures/MOSFETsmallsignal.gif}

Specifically, the nonlinear relationship between $V_{GS}$ and $I_{DS}$ can 
be linearized around the DC operation point for small changes:
\[ g_m=\frac{\Delta I_{DS}}{\Delta V_{GS}} \]
Here $g_m$, called incremental transconductance, is the ratio between small 
change in $I_{DS}$ and the small change in $V_{GS}$.

\section{CMOS Digital Logic Circuits}

The \htmladdnormallink{Logic family}{https://en.wikipedia.org/wiki/Logic_family}

\begin{itemize}
\item \htmladdnormallink{\bf RTL}{https://en.wikipedia.org/wiki/Resistor-transistor_logic}
\item \htmladdnormallink{\bf DTL}{https://en.wikipedia.org/wiki/Diode-transistor_logic}
\item \htmladdnormallink{\bf TTL}{https://en.wikipedia.org/wiki/Transistor-transistor_logic}
\item \htmladdnormallink{\bf CMOS}{https://en.wikipedia.org/wiki/CMOS}
\end{itemize}



Either a p-channel MOSFET (pMOS or PFET) or an n-channel MOSFET (nMOS or NFET)
can be treated as a switch between its drain $D$ and source $S$ controlled by
the voltage $V_{gs}$ between gate $G$ and source $S$. When $V_{gs}>V_{Tn}$ 
(e.g.,  $V_{Tn}=1\;V$) for nMOS and $V_{gs}\le V_{Tp}$ (e.g.,  $V_{Tp}=-1\;V$)
for pMOS, the circuit is a short-circuit because of the low resistance between 
$D$ and $S$; otherwise, the circuit is an open-circuit due to the large 
resistance between $D$ and $S$. A circuit composed of both types of MOSFET 
transistors is called a {\em complementary MOS} or {\em CMOS} circuit, which
is widely used in digital systems.

When two switches are connected in series, the resulting circuit conducts 
only if both switches conduct, i.e., the circuit implements logic AND. On 
the other hand, when two switches are connected in parallel, the resulting 
circuit conducts if either of the two switches conducts, i.e., the circuit 
implements logic OR.

Due to such logic properties of the series and parallel connections of the
pMOS and nMOS transistors, various logic circuits can be constructed to 
realize a given logic function $f(a,b,\cdots,x)$, where each of the inputs
$a,b,\cdots,x$ is either a low or high a voltage, representing, respectively,
logic value 0 or 1. Corresponding to each possible combination of the inputs,
the output is either low or high in voltage for logic 0 or 1.

In general, a logic function $f(a,b,\cdots,x)$ is realized by two 
complementary circuits:
\begin{itemize}
\item pull-up circuit composed of pMOS transistors connected to the voltage 
  source $V_s$ for the function $f$;
\item pull-down circuit composed of nMOS transistors connected to ground 
  for the negation of the function, $f'$.
\end{itemize}

\htmladdimg{../figures/CMOSlogic.gif}

\begin{itemize}
\item When the input variables are such that the pull-down circuit for
  $f'$ is cutoff (open-circuit), the pull-up circuit for $f$ is conducting
  (short-circuit), the output $f$ is connected to the voltage source to 
  output a high voltage representing $f=1$.

\item When the input variables are such that the pull-down circuit for 
  $f'$ is conducting (short-circuit), the pull-up circuit for $f$ is cutoff
  (open-circuit), the output $f$ is connected to ground to output a low 
  voltage representing $f=0$.
\end{itemize}

Specifically, the implementation of the pull-up and pull-down circuits is 
based on De Morgan's Law: The negation (complement) of a logic function can
be found by negating the logical operations (turn AND to OR and OR to AND) 
as well as the variables in a function. For example:
\[
(A+B+C)'=A'\cdot B'\cdot C',\;\;\;\;\;\;\;\;(A\cdot B\cdot C)'=A'+B'+C' 
\]
Here are the CMOS implementations of some simple logic functions:
\begin{itemize}
\item {\bf Negation}

  The NOT gate is implemented by a pull-up circuit composed of only a pMOS 
  transistor and its complementary pull-down circuit composed of only a nMOS
  transistor:

  \htmladdimg{../figures/CMOSnot.gif}

\item {\bf Logic NAND} $f(V_1,v_2)=(v_1 V_2)'$:

  The pull-down function is $f'=V_1\;V_2$, the pull-up function is 
  $(V_1\;V_2)'=V'_1+V'_2$,
  The output function $V_{out}=f(V_1,V_2)=V'_1+V'_2=(V_1 V_2)'$ is the 
  same as the pull-up function, a negation of AND, or NAND.

\item {\bf Logic NOR} $f(V_1+v_2)=(v_1+V_2)'$:

  The pull-down function is $f'=V_1+V_2$, the  pull-up function is
  $(V_1+V_2)'=V'_1\;V'_2$,
  The output is the same as the pull-up function $V_{out}=(V_1+V_2)'$,
  negation of OR, or NOR.

  \[ \begin{tabular}{cc||c|c|c|c} \\ \hline 
    &       &  AND & OR & NAND         &   NOR        \\ \hline
    $V_1$ 	& $V_2$ & $V_1\,V_2$ & $V_1+V_2$ &  $(V_1\; V_2)'$ & $(V_1+V_2)'$ \\ \hline \hline
    0 	& 0 	& 0 & 0 & 1		 & 1            \\
    0 	& 1 	& 0 & 1 & 1		 & 0            \\
    1 	& 0 	& 0 & 1 & 1		 & 0            \\
    1 	& 1 	& 1 & 1 & 0		 & 0            \\ \hline
  \end{tabular} \]

  \htmladdimg{../figures/CMOSnandnor.gif}
\end{itemize}


More complicated logic functions can be similarly implemented using CMOS
circuits.

{\bf Example:} Implement logic function $f(a,b,c)=(a'+b')c$ by a CMOS circuit.

First, find the complementary function $f'(a,b,c)$:
\[
f'(a,b,c)=[(a'+b')c]'=(a'+b')'+c'=ab+c' 
\]
and then the CMOS circuit can be designed as shown:
\htmladdimg{../figures/CMOSlogicexample.gif}






\end{document}



\item {\bf JFET}

A junction-gate field-effect transistor (JFET) has three terminals, the 
source, gate and drain (corresponding to the emitter, base and collector 
of a BJT). In a JFET, the source and drain is connected by a conducting 
channel (N- or P-doped), which is insulated by reverse biased PN-junctions 
(the P- or N-type substrate and gate) . The current flowing through the
channel is due to the voltage $V_{DS}$ applied across drain and source, 
and controlled by the voltage $V_{GS}$ applied to the gate. In the following,
we only consider n-channel FETs.

%\htmladdimg{../figures/JFET.gif}

The drain-source current $I_D$ is affected by two voltages $V_{GS}$
and $V_{GS}$. 
\begin{itemize}
\item When $V_{GS}$ is fixed (e.g., $V_{GS}=0$) and $V_{DS}$ is small
	and positive ($V_D>V_S$), the conducting channel behaves like a
	resistor and $I_D$ increases linearly with $V_{DS}$.
\item As $V_{DS}$ continues to increase, the PN-junction close to the 
	drain is highly reverse biased and the depletion region widens
	and the channel narrows, until eventually the channel is pinched 
	off at $V_{GS}=V_P$ and $I_D$ no longer increases with $V_{DS}$ 
	to reach saturation.
\item After pinch-off point, $I_D$ is controlled by $V_{GS}$, which
	determines how much the PN-junction between the gate and the 
	channel is reverse biased and thereby the width of the channel.
\end{itemize}

This is the transfer characteristics:
%\htmladdimg{../figures/JFETplot1.gif}

This is the output characteristics:
%\htmladdimg{../figures/JFETplot2.gif}

As the PN-junction between G and S has to be reverse biased all the time, 
i.e., $V_{GS}<0$. 


\section{Small-Signal Model}

Similar to BJT, an FET can also represented by a small-signal model. As
$i_G=0$, the model for FET is simpler than BJT. The model for common-source
configuration is
\[	i_g=0	\]
and
\[	i_d=f(v_{gs}, v_{ds})=\frac{\partial i_d}{\partial v_{gs}} v_{gs}
	+\frac{\partial i_d}{\partial v_{ds}} v_{ds}
	=g_m v_{gs}+v_{ds}/r_{ds}
\]
where $g_m=\partial i_d/\partial v_{gs}$ is the transfer admittance
representing how input voltage $v_{gs}$ effects the current $i_{ds}$,
$r_{ds}=\partial v_{ds}/\partial i_d$ is the output resistance (reciprocal 
of the slope of the i-v curve in the output characteristics), representing
how $v_{ds}$ effects $I_D$. As $r_{ds}$ is large, it can be omitted to 
simplify the model to 
\[	i_{ds}=g_m v_{gs}	\]

\htmladdimg{../figures/smallsignalmodelFET.gif}


{\bf Biasing}

Similar to BJT, FET can also used as common-source, common-gate or
common-drain circuit. The common-source configuration is most typical
and there are two different biasing methods to set up the DC operating
point:
\begin{itemize}
\item {\bf Self-Biasing}

\htmladdimg{../figures/JFETbiasing1.gif}
\[	V_{GS}=-I_D R	\]
\item {\bf Voltage Divider Biasing}

\htmladdimg{../figures/JFETbiasing2.gif}
\[	V_{GS}=\frac{R_{g2}}{R_{g1}+R_{g2}} \;V_{DD}-I_D R \]
\end{itemize}
Also, common to both biasing methods, we have
\[	V_{DS}=V_{DD}-I_D(R+R_d)	\]
and from the transfer characteristics:
\[	I_D=I_{DSS}(1-V_{GS}/V_P)^2	\]
These three equations can be solved for $V_{GS}$, $V_D$ and $V_{DS}$
for the DC operating point. 


The equivalent AC resistance between base and emitter $r_{be}$ can be found 
as below. Assume small signal $(v_B(t), i_B(t))$ around the DC operating point 
$(V_B, I_B)$, the base voltage is $V_B+v_B(t)$ and the base current is
\begin{eqnarray}
  I_B+i_B(t)&=& I_0 [e^{(V_B+v_B(t))/V_T}-1]\approx I_0 e^{(V_B+v_B(t))/V_T}
  =I_0 e^{V_B/V_T} e^{v_B(t)/V_T}
  \nonumber \\
  &=& I_B e^{v_B(t)/V_T} \approx I_B(1+\frac{v_B(t)}{V_T})
  \nonumber	
\end{eqnarray}
The first approximation is due to the fact that the exponential term 
is much greater than 1 if $V_B$ is about 0.7V ($V_T=26\;mV$ in room 
temperature). The second approximation is due to the expansion 
$e^x=1+x+x^2/2!+x^3/3!+\cdots $.

 and 
\[	i_B(t)=I_B \frac{v_B(t)}{V_T}	\]
Then we get
\[	r_{be}=\frac{v_B(t)}{i_B(t)}=\frac{V_T}{I_B}	\]
