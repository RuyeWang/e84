\documentstyle[12pt]{article}
\usepackage{html}

\begin{document}

  {\bf Example 4:} (Homework) Find all node voltages with respect to the top-left
  corner treated as reference node:

  \htmladdimg{../figures/currentmethod1.gif}

  $V_1=12 V$, $V_2=6 V$, $R_1=3 \Omega$, $R_2=8 \Omega$, $R_3=6 \Omega$, $R_4=4\Omega$.

  {\bf Solution:} First assume the bottom node $V_0=0V$ is reference point, and we
  have node voltages $V_1=12V$ (left), $V_3$ (middle) and $V_2=-6V$ (right). Applying
  KCL to $V_3$, we get
  \[ 
  \frac{V_3-(-6)}{6}+\frac{V_3-12}{3}+\frac{V_3}{8}=0 
  \]
  Solving this equation we get $V_3=24/5$. Now treating $V_1$ as reference point
  (ground), we subtract $V_1=12$ from all node voltages to get $V_1=12-12=0$, 
  $V_3=24/5-12=-36/5$, $V_2=-6-12=-18$, $V_0=0-12=-12$.


\end{document}

