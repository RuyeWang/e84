\documentstyle[11pt]{article}
\usepackage{html}
\begin{document}
\begin{center}
{\Large \bf  E84 Midterm Exam 2 (Spring 2009)}
\end{center}

\section*{E84 Midterm Exam 2}

{\bf Instructions}
\begin{itemize}
\item In class, open notes, feel free to use a calculator, but not any software 
  package such as Multisim. 
\item PRINT your name and mark question numbers clearly on top of each page.
  Indicate the total number of pages submitted.
\item When solving a problem, list all the steps. In each step, indicate
  concisely what you are doing in English, then show the calculation 
  and the result of for the step. Box the final answer.
\item A final answer, even if correct, without evidence of the steps leading
  to it will receive ZERO credit.
\end{itemize}

\begin{itemize}

\item {\bf Problem 1. (50 pts)}

In the circuit shown in the figure below, $R_1=R_2=15\,k\Omega$, $R_3=30\,k\Omega$,
$C=50\,\mu F$, and the input voltage is $V=40$ V. The switch is closed at $t=0$.
Assume the circuit was at steady state before $t=0$. Find currents $i_1(t)$, $i_2(t)$
and $i_3(t)$, together with $i_c(t)$ as the complete responses of the circuit for $t>0$.
Make sure the signs of these currents are consistent with the assumed directions as 
shown in the figure.

\htmladdimg{../mdtm2_09sa.gif}

%{\bf Solution:} 
%\begin{itemize}
%  \item Find voltage across $C$ at $t=0^-$:
%    \[ v_c(0^-)=v_c(0^+)=40\frac{30}{15+15+30}=20\;V \]
%  \item Find initial current $i_c(0^+)$ through $C$:
%    \[ i_c(0^+)=-\frac{v_c(0^+)}{R_2 || R_3}=-\frac{20}{15 || 30}=-2\; mA \]
%  \item The steady state current through $C$ is $i_c(\infty)=0$.
%  \item Find time constant:
%    \[ \tau=R_{eq}C=(R_2||R_3)C=(15||30)\times 10^3\times 50\times 10^{-6}=0.5\;s \]
%  \item The current through $C$ as the complete response:
%    \[ i_c(t)=i_c(\infty)+[i_c(0^+)-i_c(\infty)] e^{-t/tau}=-2 e^{-2t}\;mA \]
%
%  \item Find current through $R_1$ for $t>0$: $i_1(t)=40/15=8/3=2.67\;mA$.
%
%  \item Find initial current $i_2(0^+)$ through $R_2$:
%    \[ i_2(0^+)=-\frac{v_c(0^+)}{R_2}=\frac{20}{15}=-\frac{4}{3}\;mA \]
%  \item The steady state current through $R_2$ is $i_2(\infty)=0$.
%  \item The current through $R_2$ as the complete response:
%    \[ i_2(t)=i_2(\infty)+[i_2(0^+)-i_2(\infty)] e^{-t/tau}=-4/3 \; e^{-2t}\;mA \]
%
%  \item Find initial current $i_3(0^+)$ through $R_3$:
%    \[ i_3(0^+)=\frac{v_c(0^+)}{R_3}=\frac{20}{30}=\frac{2}{3}\;mA \]
%  \item The steady state current through $R_3$ is $i_3(\infty)=0$.
%  \item The current through $R_3$ as the complete response:
%    \[ i_3(t)=i_3(\infty)+[i_3(0^+)-i_3(\infty)] e^{-t/tau}=2/3 \; e^{-2t}\;mA \]
%\end{itemize}

\item {\bf Problem 2. (50 pts)}

Find the current $i(t)$ through the $5\Omega$ resistor in the circuit shown in the 
figure below. Here the voltage source is $v_0(t)=10\sqrt{2}\cos 2t$ and the current
source is $i_0(t)=2\sqrt{2}\cos(3t+45^\circ)$. 
%(Hint: as the circuit is composed of all linear components, superposition principle applies.)

\htmladdimg{../mdtm2_09sb.gif}

%{\bf Solution:}
%\begin{itemize}
%\item Phasor representation of the voltage source: $\dot{V}=10\angle 0^\circ$ V.
%\item Find current through $R$ due to voltage source:
%  \[ \dot{I}_1=\frac{\dot{V}}{j2/3+(j2||-j0.5)+5}=\frac{10}{j2/3-j2/3+5}=2\;A \]
%\item Phasor representation of the voltage source: $\dot{I}=2\angle 45^\circ$ A.
%\item Find current through $R$ due to current source:
%  \[ \dot{I}_2=\dot{I}\frac{j1+j3||(-j1/3)}{5+j1+j3||(-j1/3)}
%  =2\angle 45^\circ \times \frac{j5/8}{5+j5/8}=0.25\angle 127.87^\circ\;A \]
%\item The total phasor current is
%  \[ I(t)=I_1(t)+I_2(t)=2+0.25\angle 127.87^\circ  \]
%\item Convert to time function:
%  \[ i(t)=2\sqrt{2}\cos 2t+0.25\sqrt{2}\cos(3t+127.87^\circ)\;A \]
%\end{itemize}

\end{itemize}
\end{document}
