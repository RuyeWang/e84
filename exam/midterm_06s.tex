\documentstyle[11pt]{article}
\usepackage{html}
\begin{document}
\begin{center}
{\Large \bf  Midterm Exam ---- E84, Spring, 2006}
\end{center}

\section*{Please fill out the feedback form}

{\bf Note: } To help me improve my teaching of the course in the future, I would 
like to collect your feedback and comments. Please fill the following form 
{\bf anonymously} and turn it in {\bf separately} from your midterm exam. Your 
help is greatly appreciated!

\begin{itemize}
\item {\bf Feedback:} (circle one of the numbers)
\begin{itemize}

\item {\bf Lecture pace:} (1 too slow, 3 just right, 5 too fast)

\framebox[0.5in]{1}\framebox[0.5in]{2}\framebox[0.5in]{3}\framebox[0.5in]{4}\framebox[0.5in]{5}

\item {\bf Lecture clearity:} (1 Too tedious, 3 just right, 5 too sketchy)

\framebox[0.5in]{1}\framebox[0.5in]{2}\framebox[0.5in]{3}\framebox[0.5in]{4}\framebox[0.5in]{5}

\item {\bf Overall homework workload:} (1 too little, 3 just right, 5 too much)

\framebox[0.5in]{1}\framebox[0.5in]{2}\framebox[0.5in]{3}\framebox[0.5in]{4}\framebox[0.5in]{5}

\item {\bf Overall difficulty of course materials:} (1 too easy, 3 just right, 5 too hard)

\framebox[0.5in]{1}\framebox[0.5in]{2}\framebox[0.5in]{3}\framebox[0.5in]{4}\framebox[0.5in]{5}
\end{itemize} 

\item {\bf Comments:}
\begin{itemize}

\item {\bf What went well for your learning in the course? Be specific.}

\begin{tabular}{l}
.  \\
.  \\
.  \\
\end{tabular}
\vskip 5cm

\item {\bf What did not go well for your learning in the course? Be specific.}
\begin{tabular}{l}
.  \\
.  \\
.  \\
\end{tabular}
\vskip 0.9in

\item {\bf Indicate one thing you want to improve and how.}
\begin{tabular}{l}
.  \\
.  \\
.  \\
\end{tabular}
\vskip 0.9in

\end{itemize}

\end{itemize}


\section*{E84 Midterm Exam 1}

{\bf Instructions}
\begin{itemize}
\item Take home, open everything except discussion. 
\item Mark your start and end times. 
\item Compare your print-out of the exam with the online version to make
	sure your hard copy is complete.
\item Mark your name and question number clearly on top of each page.
	Indicate the total number of pages submitted.
\item When solving a problem, list all the steps. In each step, describe 
	concisely what you are doing in English, then show the calculation 
	and the result of the step. A final answer, even if correct, without 
	evidence of the steps leading to the answer will not receive credit.
\end{itemize}

{\bf The Problems}
\begin{enumerate}

\item {\bf Problem 0. (20 points)} 
Use Norton's thereom to convert the circuit below (left) to an equivalent 
current source with $R_N$ and $I_N$ in parallel (right).

\htmladdimg{../midterm06sa.gif}

% {\bf Solution:} When the current source is open, the resistance between a 
% and b is $R_N=(8+12)||(16+4)=10$. To find the short-circuit current from a 
% to b, realize that the total current of 5A goes through the parallel connection 
% of $8\Omega$ and $12\Omega$ resistors and then the parallel connection of 
% $16\Omega$ and $4\Omega$ resistors. The currents through the $8\Omega$ and 
% $16\Omega$ resistors are, respectively, $5A *12/(12+8)=3A$ and $5A *4/(16+4)=1A$,
% and due to KCL at node a, the current from a to b is $I_N=3A-1A=2A$.

\item {\bf Problem 1. (20 points)} 

The values of $\omega$, $L$ and $C$ of the circuit below are such that 
$\omega L=5\Omega$, $1/\omega C=5\Omega$, and also $R=10\Omega$. Find the 
phase difference between the input voltage $v_{in}(t)$ and the output 
voltage $v_{out}(t)$. Which of the two voltages is leading? by how much?

\htmladdimg{../midterm06sb.gif}

% {\bf Solution:} 
% \[ \dot{V}_{out}=\dot{V}_{in} \frac{-j5||10}{-j5||10+j5}=-2j=2\angle -90^\circ \]
% So $v_{out}(t)$ is lagging $v_{in}S(t)$ by $90^\circ$.

\item {\bf Problem 3. (25 points)} 

The one-port network in the circuit below is a resistor network with some
unknown energy sources. When the switch is in position 1, the ideal ammeter 
reads 3A (DC), when the switch is in position 2, the ideal voltmeter reads 9V
(DC), with polarity as shown. Find the output voltage $v_{out}(t)$ across the
inductor with $L=1\,H$ after the switch is turned into positon 3 at $t=0$.

\htmladdimg{../midterm06sd.gif}

% {\bf Solution:} 
% \begin{itemize}
% \item Model the one-port network by a Thevenin equivalent circuit with 
% open-circuit voltage $V_T=9V$ and internal resistance $R_T$, which can
% be found to be $R_T=V_T/I_{sc}=9V/3A=3\Omega$.
% \item $v_{out}(0_+)=V_t=9V$, $v_{out}(\infty)=0V$, $\tau=L/R=1H/3\Omega=1/3\;s$,
%   the output voltage is therefore:
%   \[ v_{out}(t)=v_{out}(\infty)+[v_{out}(0_+)-v_{out}(\infty)]e^{-t/\tau}
%     =9 e^{-3t} \]
% \end{itemize}

\item {\bf Problem 3. (35 points)} 

In the circuit below, $R_1=2\Omega$, $R_2=3\Omega$, $R_3=6\Omega$, $L=0.5H$, 
$V_1=6V$, $V_2=3V$, and before the switch S closes at $t=0$, the circuit is 
already in steady state. Find the current $i_1(t)$ through $R_1$ for $t>0$.
(Hint: as current through an inductor cannot change instantaeously, the 
inductor can be treated as a current source right after the switch is closed.)

\htmladdimg{../midterm06sc.gif}

% {\bf Solution:} 
% \begin{itemize}
% \item First find current through $L$ at $t=0$: 
%   \[ i_L(0_+)=i_L(0_-)=V_2/R_3=3V/6\Omega=0.5A \]
% \item Then find $i_1(0_+)$ by superposition (with three energy sources $V_1=6V$, 
%   $V_2=3V$ and current source due to the inductor $I_L=0.5A$:
% \begin{itemize}
%   \item $V_1=6V$ alone:
%     current thru $V_1$ is $ V_1/(R_1|| R_3+R_2)=6/(2||6+3)=4/3\;A$, current thru $R_1$ is
%     $4/3 \times R_3/(R_1+R_3)=4/3 \times 6/8=1 A$ (downward).
%   \item $V_2=3V$ alone: 
%     current thru $R_1$ is $V_2/(R_2||R_3+R_1)=3/(3||6+2)=3/4=0.75 \;A $ (downward)
%   \item $I_L=0.5A$ alone:
%     current thru $R_1$ is $I_L \times (R_2||R_3)/(R_2||R_3+R_1)=0.5\times 2/4=0.25\;A$
%     (upward)
% \end{itemize}
% The total current thru $R_1$ is therefore: $i_1(0_+)=1+0.75-0.25=1.5\;A$ (downward).
% \item $i_i(\infty)=0$, as $R_1$ is by-passed by $L$ with zero impedance when $t=\infty$.
% \item Find time constant: resistance $R$ between two ends of $L$ is $R_1||R_2||R_3=1\Omega$,
%   therefore $\tau=L/R=0.5/1=0.5\;s$.
% \item Finally, the current thru $R_1$ is 
%   \[ i_1(t)=i_1(\infty)+[i_1(0_+)-i_1(\infty)] e^{-t/\tau}=1.5 e^{2t}\; A \]
% \end{itemize}



\end{enumerate}

\end{document}

