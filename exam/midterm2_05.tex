\documentstyle[11pt]{article}
\usepackage{html}
\begin{document}
\begin{center}
{\Large \bf E84 Midterm Exam 2}
\end{center}

\begin{itemize}
\item Take home, open everything except discussion.
%\item Mark your start and end times. % Don't spend more than 3 hours.
\item Indicate the total time you spent to complete the test.
\item Due Monday (4/11) in class.
\item Mark your name and question number clearly on top of each page.
	Indicate the total number of pages submitted.
\item Keep one digit after decimal point for all final answers. 
\item When solving a problem, list all the steps. In each step, describe 
	what you are doing in English, then show the calculation and the 
	result of the step. A final answer, even if correct, without 
	evidence of the steps leading to the answer will not receive credit.
\end{itemize}

\begin{enumerate}

\item {\bf Problem 1. (33 points)} 

\begin{itemize}
\item (a) The figure below shows a fluorescent light fixture and 
its model, a resistor and an inductor in series. Assume the real power
of the fluorescent light is 100 Watts when connected to 110V 60Hz voltage
supply, and its power factor is $\lambda=0.87$. Find the resistance 
R and inductance L of the fixture.

\item (b) Ten fluorescent lights discussed above and eleven
100-watt incandescent lights are connected in parallel to a 110V 60Hz
voltage supply. Find the power factor of this load.
\end{itemize}

\htmladdimg{../midterm2_05a.gif}

% {\bf Solution:} 
% \begin{itemize}
% \item The phase angle is
% \[ \phi=\cos^{-1} \lambda=\cos^{-1} 0.87=30^\circ	\]
% \item Relate $R$ and $\omega L$:
% \[ \frac{\omega L}{R}=\tan \phi=\tan 30^\circ=1/\sqrt{3},
% 	\;\;\;\;\mbox{i.e.}\;\;\;\omega L=R/\sqrt{3}	\]
% \item Find resistance. First find the real power:
% \[ P=S\cos\phi=VI\cos\phi=\frac{V^2}{|Z|}\cos\phi
% 	=\frac{110^2}{\sqrt{R^2+(\omega L)^2}}\frac{\sqrt{3}}{2}
% 	=\frac{110^2}{\sqrt{R^2+R^2/3}}\frac{\sqrt{3}}{2}=100	 \]
% 	Solve this for $R$ to get $R=91\Omega$
% \item Find $L$
% \[ \omega L=R/\sqrt{3}=52.5,\;\;\;\;\mbox{i.e.}\;\;\;\;
% 	L=R/\omega \sqrt{3}=0.14H	\]
% \end{itemize}

% {\bf Part b:} Ten fluorescent lights discussed above and elevent 
% 100-watt incandescent lights are connected in parallel to a 110V 60Hz
% voltage supply. Find the power factor of this load.

% {\bf Solution:} 
% \begin{itemize}
% \item Find resistance of incandescent light:
% \[ P=\frac{V^2}{R}=\frac{110^2}{R}=100,\;\;\;\mbox{i.e.}\;\;\;R=121\Omega \]
% \item Find overall impedance $Z$:
% \[ Z=\frac{11\times (9.1+j5.3)}{11+(9.1+j5.3)}=\frac{100+j58}{20.1+j5.3} \]
% \item Find $\angle Z$:
% \[ \angle Z=\tan^{-1} (58/100)-\tan^{-1} (5.3/20.1)=15.3^\circ \]
% \end{itemize}

\item {\bf Problem 2. (33 points)} 

\begin{itemize}
\item (a) Find the frequency response function of the circuit shown in 
the figure, i.e., find the ratio of the voltage $V_R$ across load resistor
$R$ and the input sinusoidal voltage $V$. 

\item (b) Similar to a simple RCL series resonant circuit, this circuit
can be used as a band-pass filter to pass signals around its resonant 
frequency. Find this resonant frequency $\omega_{max}$ at which 
$V_R(\omega)$ is maximized in terms of the component values $R$, $C$, 
$L_1$ and $L_2$. 

\item (c) Different from the simple RCL series resonant circuit, this 
circuit also has a stop band, i.e., around a certain freqnecy 
$\omega_{min}$ the output voltage $V_R$ is zero. Find this frequency 
$V_R$ also in terms of the circuit components. 

\item (d) Given $C=10\;\mu F$ and the two inductors are identical, 
determine the values of $R$ and $L_1=L_2$, so that $\omega_{max}=5000
\;rad/sec$. What is the corresponding stop frequency $\omega_{min}$?

\end{itemize}

\htmladdimg{../midterm2e.gif}

%  {\bf Solution:}
%  \begin{itemize}
%  \item Find total impedance of the circuit:
%  \begin{eqnarray}
%   Z&=&R+j\omega L_2+\frac{j\omega L_1/j\omega C}{j\omega L_1+1/j\omega C}
%  	=R+j\omega L_2+\frac{j\omega L_1}{1-\omega^2 CL_1}
%  	\nonumber \\
%  &=&R+\frac{j\omega(L_2(1-\omega^2CL_1)+L_1)}{1-\omega^2 CL_1}
%  =R+j\omega \frac{L_1+L_2-\omega^2CL_1L_2}{1-\omega^2CL_1}
%  	\nonumber 
%  \end{eqnarray}
%  \item At the resonant frequency $\omega_0=\omega_{max}$, the imaginary 
%  part of the impedance is zero (minimum), i.e.,
%  \[	L_1+L_2=\omega^2CL_1L_2,\;\;\;\;\mbox{i.e.}\;\;\;\;
%  	\omega_0=\frac{1}{\sqrt{CL}},\;\;\;\mbox{where}\;\;\;
%  	L=\frac{L_1L_2}{L_1+L_2}
%  \]
%  \item At the stop frequency $\omega_{min}$, the imaginary part of the 
%  impedance is infinity (maximum), i.e.,
%  \[	1-\omega^2CL_1=0, \;\;\;\mbox{i.e.}\;\;\;\;
%  	\omega_{min}=\frac{1}{\sqrt{CL_1}}	\]
%  
%  \item As $L_1=L_2$, $L=L_1/2$ and we have
%  \[ \omega_0=\frac{1}{\sqrt{CL_1/2}}=5000,\;\;\;\;\mbox{i.e.}\;\;\;\;
%  	L_1=0.008\;H=8\;mH	\]
%  \[ \omega_{min}=\frac{1}{\sqrt{CL_1}}=3536\;rad/sec	\]
%  This is true for any load of $R$.
%  
%  \end{itemize} 

\item {\bf Problem 3. (33 points)} In the circuit shown below, $V_1=2V$,
$V_2=10V$, $R_1=R_2=2\;M\Omega$, $C=1\mu F$. Switch $S_1$ closes at $t=0$, 
switch $S_2$ closes at $t=1$ second. Assume initially the voltage across
the capacitor is $v_c(0)=4V$. Find voltage $v_c(t)$ for the following 
two time periods:
\begin{itemize}
\item $0\le t < 1 s$
\item $t \ge 1\;s$ 
\end{itemize}
Sketch the plots of these two voltages for $t \ge 0$.

{\bf Hints:} For second period, assume $t'=t-1$, and use the solution of
the first period at $t=1$ as the initial value. In the final expression, 
replace $t'$ by $t-1$.

\htmladdimg{../midterm2d.gif}

%   {\bf Solution:}
%   \begin{itemize}
%   \item During the period $0\le t<1 s$, the initial value is 
%  	$v_c(0)=4$, the steady state value is $v_c(\infty)=2V$, the time 
%  	constant is $\tau_1=R_1C=2\times 10^6\times 10^{-6}=2\;sec.$, and 
%  	the overall voltage is:
%  \[ v_c(t)=v_c(\infty)+[v_c(0)-v_c(\infty)] e^{-t/\tau_1}
%  	=2+(4-2) e^{-t/2}=2(1+e^{-0.5t})	\]
%   In particular, when $t=1$, we have
%   \[ v_c(1)=2(1+e^{-0.5})=3.213V	\]
%   
%   \item During the period $t \ge 1\;s$, we have $v_c(1)=3.213V$, 
%   	$v_c(\infty)=2+2\times (10-2)/(2+2)=6V$,
%   	$\tau_2=(R_1||R_2)C=10^6 \times 10^{-6}=1\; sec.$
%  
%   \[	v_c(t')=v_c(\infty)+[v_c(1)-v_c(\infty)]e^{-t'/\tau_2}
%   	=6+(3.213-6)e^{-t'/\tau_2}=6-2.787e^{-(t-1)} \]
%  
% \end{itemize}

\end{enumerate}

\end{document}

\item {\bf Problem 3. (33 points)} 

In the circuit shown in the figure, $V_0=6V$, $I_0=2A$, $R_1=6\Omega$, 
$R_2=3\Omega$, and $L=0.5H$. Assume the circuit has reached steady state.
Find the voltage $v(t)$ across $R_1$ and current $i(t)$ through $L$ as 
time functions after the switch is closed at $t=0$.


{\bf Solution:} 
$i(0)=V_0/R_1=6/6=1A$, $i(\infty)=V_0/R_1+I_0=1+2=3A$. 

Find time constant: $R=R_1R_2/(R_1+R_2)=3\times 6/(3+6)=2\Omega$,
$\tau=L/R=0.5/2=0.25S$. 

The current $i(t)$ is therefore:
\[ i(t)=i(\infty)+[i(0)-i(\infty)]e^{-t/\tau}=3+(1-3)e^{-t/0.25}
	=3-2e^{-4t} \;A \]
\[ v(t)=V_0-V_L(t)=V_0-L\frac{d}{dt}i(t)=6-0.5 \frac{d}{dt}(3-2e^{-4t})
	6-4e^{-4t} \]


\item {\bf Problem 3. (33 points)} 
The circuit in the figure shows a voltage source $V_0$ and $R_0$ and an
amplification circuit modeled by as a two-port network. Assume the two-port 
is represented by a Z-model ($Z_{11}, Z_{12}, Z_{21}, Z_{22}$). Find the 
expression for the load impedance $Z_L$ in terms of $R_0$ as well as the
four Z-parameters, for it to get maximum power from the voltage source.
(Hint: use Thevenin's theorem.)

{\bf Solution:} 
\begin{itemize}
\item First set up all equations:
\[ \left\{ \begin{array}{l} V_1=Z_{11}I_1+Z_{12}I_2 \\
	V_2=Z_{21}I_1+Z_{22}I_2 \end{array} \right.	\]
\[ \left\{ \begin{array}{l} V_1=V_0-R_0I_1 \\
	V_2=-R_L I_2 \end{array} \right.	\]
\item Use Thevenin's theorem

\begin{itemize}
\item Find $Z_{Th}$: assume $V_0=0$, equate equations 1 and 3 to get:
\[ V_1=Z_{11}I_1+Z_{12}I_2, \;\;\;\;\mbox{i.e.,}\;\;\;\;\;
I_1=-\frac{Z_{12}}{Z_{11}+R_0} I_2 \]
Substitute this $I_1$ in equation 2 to get:
\[ V_2=(-\frac{Z_{12}Z_{21}}{Z_{11}+R_0}+Z_{22}) I_2,\;\;\;\;\;
\mbox{i.e.,}\;\;\;\;\;
Z_{Th}=\frac{V_2}{I_2}=-\frac{Z_{12}Z_{21}}{Z_{11}+R_0}+Z_{22}	\]
\item Find $V_{Th}$:, assume $I_2=0$, we have
\[ \left\{\begin{array}{l} V_1=Z_{11}I_1 \\V_2=Z_{21}I_1\end{array}\right. \]
Substitute $V_1=V_0-R_0I_1$ into $V_1=Z_{11}I_1$ to get
\[	V_{Th}=V_2=V_0\frac{Z_{21}}{Z_{11}+R_0}	\]
\end{itemize}

For $R_L$ to get maximum power, we need to have
\[ R_L=Z_{Th}=Z_{22}-\frac{Z_{12}Z_{21}}{Z_{11}+R_0}	\]
and the current is
\[ I_l=\frac{V_{Th}}{Z_{Th}+R_L}=\frac{V_{Th}}{Z_{Th}}
	=\frac{2V_0Z_{21}}{2(Z_{22}Z_{11}-Z_{12}Z_{21}+Z_{22}R_0)} \]


\end{itemize}


\end{enumerate}

\end{document}

An electric motor, modeled as an inductor and a resistor in series, has 
a power factor of 0.85. The nameplate current is 10 Amps at 115 Volts 
(60 Hz). 
\begin{itemize}
\item Find the apparent power, active power, and reactive power. 
\item Find the inductance and resistance of the motor.
\item Find the capacitance of a parallel shunt capacitor that can improve
	the power factor to 1.
\item Find the capacitance if the power factor can be 0.9.
\end{itemize}

% {\bf Solution:} 
% The apprarent power is $S=115V \times 10A = 1150 W$, the real power is
% $P=S\cos \phi=S*0.85=977.5 W$, the reactive power is 
% $P=S\sin \phi=S*0.527=605.8 W$. The impedance of the motor is
% \[	Z=\frac{V}{I}=\frac{115}{10}=11.5\Omega \]
% The inductance L and resistance R satisfy the following equations:
% \[ \left\{ \begin{array}{l} (\omega L)^2+R^2=Z^2=11.5^2 \\
% 	tan^{-1} \frac{\omega L}{R}=cos^{-1} 0.85 \end{array} \right. \]
% Given $\omega=2\pi f=377\;rad/sec$, the quations can be solve for R and
% L to get
% \[	R=9.8\Omega,\;\;\;\;L=16\;mH,\;\;\;\;\omega L=6\Omega	\]
% With the parallel capacitor C, the overall impedance is
% \[	Z=(R+j\omega L)\; || \;(1/j\omega C)
% 	=\frac{(R+j\omega L)/j\omega C}{(R+j\omega L)+1/j\omega C}
% 	=\frac{R+j\omega L}{j\omega CR-\omega^2 LC+1}	\]
% For $\angle Z=0$, we need to have
% \[ \tan^{-1}\frac{\omega L}{R}=\tan^{-1}\frac{\omega RC}{1-\omega^2 LC},
% 	\;\;\;\mbox{i.e.}\;\;\;	
% 	\frac{\omega L}{R}=\frac{\omega RC}{1-\omega^2 LC}	\]
% which can be solved for $C$ to get
% \[	C=\frac{L}{R^2+\omega^2 L^2}=120\;\mu F	\]
% For the power factor to be 0.9, or $\cos^{-1} 0.9=\phi=25.8$, we need
% \[ \tan^{-1}\frac{\omega L}{R}-\tan^{-1}\frac{\omega RC}{1-\omega^2 LC}
% 	=25.8,	\;\;\;\mbox{i.e.}\;\;\;	
% \tan^{-1}\frac{\omega RC}{1-\omega^2 LC}=\tan^{-1}\frac{\omega L}{R}-25.8=
% 	5.8 \]
% which can be solved to get $C=25.5\;\mu F$.

\documentstyle[11pt]{article}
\usepackage{html}
\begin{document}
\begin{center}
{\Large \bf E84 Midterm Exam 2}
\end{center}

\begin{itemize}
\item Take home, open everything except discussion.
\item Mark your start and end times. % Don't spend more than 3 hours.
\item Due Monday in class.
\item Mark your name and question number clearly on top of each page.
	Indicate the total number of pages submitted.
\item When solving a problem, list all the steps. In each step, describe 
	what you are doing in English, then show the calculation and the 
	result of the step. A final answer, even if correct, without 
	evidence of the steps leading to the answer will not receive credit.
\end{itemize}

\begin{enumerate}

\item {\bf Problem 1. (33 points)} 

An electric motor, modeled as an inductor and a resistor in series, has 
a power factor of 0.85. The nameplate current is 10 Amps at 115 Volts 
(60 Hz). 
\begin{itemize}
\item Find the apparent power, active power, and reactive power. 
\item Find the inductance and resistance of the motor.
\item Find the capacitance of a parallel shunt capacitor that can improve
	the power factor to 1.
\item Find the capacitance if the power factor can be 0.9.
\end{itemize}

{\bf Solution:} 
The apprarent power is $S=115V \times 10A = 1150 W$, the real power is
$P=S\cos \phi=S*0.85=977.5 W$, the reactive power is 
$P=S\sin \phi=S*0.527=605.8 W$. The impedance of the motor is
\[	Z=\frac{V}{I}=\frac{115}{10}=11.5\Omega \]
The inductance L and resistance R satisfy the following equations:
\[ \left\{ \begin{array}{l} (\omega L)^2+R^2=Z^2=11.5^2 \\
	tan^{-1} \frac{\omega L}{R}=cos^{-1} 0.85 \end{array} \right. \]
Given $\omega=2\pi f=377\;rad/sec$, the quations can be solve for R and
L to get
\[	R=9.8\Omega,\;\;\;\;L=16\;mH,\;\;\;\;\omega L=6\Omega	\]
With the parallel capacitor C, the overall impedance is
\[	Z=(R+j\omega L)\; || \;(1/j\omega C)
	=\frac{(R+j\omega L)/j\omega C}{(R+j\omega L)+1/j\omega C}
	=\frac{R+j\omega L}{j\omega CR-\omega^2 LC+1}	\]
For $\angle Z=0$, we need to have
\[ \tan^{-1}\frac{\omega L}{R}=\tan^{-1}\frac{\omega RC}{1-\omega^2 LC},
	\;\;\;\mbox{i.e.}\;\;\;	
	\frac{\omega L}{R}=\frac{\omega RC}{1-\omega^2 LC}	\]
which can be solved for $C$ to get
\[	C=\frac{L}{R^2+\omega^2 L^2}=120\;\mu F	\]
For the power factor to be 0.9, or $\cos^{-1} 0.9=\phi=25.8$, we need
\[ \tan^{-1}\frac{\omega L}{R}-\tan^{-1}\frac{\omega RC}{1-\omega^2 LC}
	=25.8,	\;\;\;\mbox{i.e.}\;\;\;	
\tan^{-1}\frac{\omega RC}{1-\omega^2 LC}=\tan^{-1}\frac{\omega L}{R}-25.8=
	5.8 \]
which can be solved to get $C=25.5\;\mu F$.


\item {\bf Problem 2. (33 points)} Study the frequency behavior of the
circuit shown in the figure by finding the ratio of the voltage $V_R$
across load resistor $R$ and the input voltage $V$ (frequency response
function) of the circuit. 
\begin{itemize}
\item Similar to a simple RCL series resonant circuit, this circuit
can be used as a band-pass filter to pass signals around its resonant 
frequency. Find this resonant frequency $\omega_{max}$ at which 
$V_R(\omega)$ is maximized in terms of the component values $R$, $C$, 
$L_1$ and $L_2$. 

\item Different from the simple RCL series resonant circuit, this 
circuit also has a stop band, i.e., around a certain freqnecy 
$\omega_{min}$ the output voltage $V_R$ is zero. Find this frequency 
$V_R$ also in terms of the circuit components. 

\item Given $C=10\;\mu F$ and the two inductors are identical, determine
the values of $R$ and $L_1=L_2$, so that $\omega_{max}=5000\;rad/sec$.

\end{itemize}

\htmladdimg{../midterm2e.gif}

{\bf Solution:}
\begin{itemize}
\item Find total impedance of the circuit:
\begin{eqnarray}
 Z&=&R+j\omega L_2+\frac{j\omega L_1/j\omega C}{j\omega L_1+1/j\omega C}
	=R+j\omega L_2+\frac{j\omega L_1}{1-\omega^2 CL_1}
	\nonumber \\
&=&R+\frac{j\omega(L_2(1-\omega^2CL_1)+L_1)}{1-\omega^2 CL_1}
=R+j\omega \frac{L_1+L_2-\omega^2CL_1L_2}{1-\omega^2CL_1}
	\nonumber 
\end{eqnarray}
\item At the resonant frequency $\omega_0=\omega_{max}$, the imaginary 
part of the impedance is zero (minimum), i.e.,
\[	L_1+L_2=\omega^2CL_1L_2,\;\;\;\;\mbox{i.e.}\;\;\;\;
	\omega_0=\frac{1}{\sqrt{CL}},\;\;\;\mbox{where}\;\;\;
	L=\frac{L_1L_2}{L_1+L_2}
\]
\item At the stop frequency $\omega_{min}$, the imaginary part of the 
impedance is infinity (maximum), i.e.,
\[	1-\omega^2CL_1=0, \;\;\;\mbox{i.e.}\;\;\;\;
	\omega_{min}=\frac{1}{\sqrt{CL_1}}	\]

\item As $L_1=L_2$, $L=L_1/2$, and
\[ \omega_0=\frac{1}{\sqrt{CL_1/2}}=5000,\;\;\;\;\mbox{i.e.}\;\;\;\;
	L_1=0.008\;H=8\;mH	\]
\[ \omega_{min}=\frac{1}{\sqrt{CL_1}}=3536\;rad/sec	\]
This is true for any load of $R$.

\end{itemize} 


\item {\bf Problem 3. (33 points)} In the circuit shown below, $V_1=2V$,
$V_2=10V$, $R_1=R_2=2\;M\Omega$, $C=1\mu F$. Switch $S_1$ closes at $t=0$, 
switch $S_2$ closes at $t=1$ second. Assume initially the voltage across
the capacitor is $v_c(0)=4V$. Find voltage $v_c(t)$ for the following 
two time periods:
\begin{itemize}
\item $0\le t < 1 s$
\item $t \ge 1\;s$ 
\end{itemize}
Sketch the plots of these two currents for $t \ge 0$.

{\bf Hints:} For second period, assume $t'=t-1$, and use $i(1)$ as 
the initial value. In the final expression, replace $t'$ by $t-1$.

\htmladdimg{../midterm2d.gif}

 {\bf Solution:}
 \begin{itemize}
 \item During the period $0\le t<1 s$, the initial value is 
	$v_c(0)=4$, the steady state value is $v_c(\infty)=2V$, the time 
	constant is $\tau_1=R_1C=2\times 10^6\times 10^{-6}=2\;sec.$, and 
	the overall voltage is:
\[ v_c(t)=v_c(\infty)+[v_c(0)-v_c(\infty)] e^{-t/\tau_1}
	=2+(4-2) e^{-t/2}=2(1+e^{-0.5t})	\]
 In particular, when $t=1$, we have
 \[ v_c(1)=2(1+e^{-0.5})=3.213V	\]
 
 \item During the period $t \ge 1\;s$, we have $v_c(1)=3.213V$, 
 	$v_c(\infty)=2+2\times (10-2)/(2+2)=6V$,
 	$\tau_2=(R_1||R_2)C=10^6 \times 10^{-6}=1\; sec.$

 \[	v_c(t')=v_c(\infty)+[v_c(1)-v_c(\infty)]e^{-t'/\tau_2}
 	=6+(3.213-6)e^{-t'/\tau_2}=6-2.787e^{-(t-1)} \]

 
 \end{itemize}




\end{enumerate}

\end{document}

\item {\bf Problem 3. (33 points)} 

In the circuit shown in the figure, $V_0=6V$, $I_0=2A$, $R_1=6\Omega$, 
$R_2=3\Omega$, and $L=0.5H$. Assume the circuit has reached steady state.
Find the voltage $v(t)$ across $R_1$ and current $i(t)$ through $L$ as 
time functions after the switch is closed at $t=0$.


{\bf Solution:} 
$i(0)=V_0/R_1=6/6=1A$, $i(\infty)=V_0/R_1+I_0=1+2=3A$. 

Find time constant: $R=R_1R_2/(R_1+R_2)=3\times 6/(3+6)=2\Omega$,
$\tau=L/R=0.5/2=0.25S$. 

The current $i(t)$ is therefore:
\[ i(t)=i(\infty)+[i(0)-i(\infty)]e^{-t/\tau}=3+(1-3)e^{-t/0.25}
	=3-2e^{-4t} \;A \]
\[ v(t)=V_0-V_L(t)=V_0-L\frac{d}{dt}i(t)=6-0.5 \frac{d}{dt}(3-2e^{-4t})
	6-4e^{-4t} \]


\item {\bf Problem 3. (33 points)} 
The circuit in the figure shows a voltage source $V_0$ and $R_0$ and an
amplification circuit modeled by as a two-port network. Assume the two-port 
is represented by a Z-model ($Z_{11}, Z_{12}, Z_{21}, Z_{22}$). Find the 
expression for the load impedance $Z_L$ in terms of $R_0$ as well as the
four Z-parameters, for it to get maximum power from the voltage source.
(Hint: use Thevenin's theorem.)

{\bf Solution:} 
\begin{itemize}
\item First set up all equations:
\[ \left\{ \begin{array}{l} V_1=Z_{11}I_1+Z_{12}I_2 \\
	V_2=Z_{21}I_1+Z_{22}I_2 \end{array} \right.	\]
\[ \left\{ \begin{array}{l} V_1=V_0-R_0I_1 \\
	V_2=-R_L I_2 \end{array} \right.	\]
\item Use Thevenin's theorem

\begin{itemize}
\item Find $Z_{Th}$: assume $V_0=0$, equate equations 1 and 3 to get:
\[ V_1=Z_{11}I_1+Z_{12}I_2, \;\;\;\;\mbox{i.e.,}\;\;\;\;\;
I_1=-\frac{Z_{12}}{Z_{11}+R_0} I_2 \]
Substitute this $I_1$ in equation 2 to get:
\[ V_2=(-\frac{Z_{12}Z_{21}}{Z_{11}+R_0}+Z_{22}) I_2,\;\;\;\;\;
\mbox{i.e.,}\;\;\;\;\;
Z_{Th}=\frac{V_2}{I_2}=-\frac{Z_{12}Z_{21}}{Z_{11}+R_0}+Z_{22}	\]
\item Find $V_{Th}$:, assume $I_2=0$, we have
\[ \left\{\begin{array}{l} V_1=Z_{11}I_1 \\V_2=Z_{21}I_1\end{array}\right. \]
Substitute $V_1=V_0-R_0I_1$ into $V_1=Z_{11}I_1$ to get
\[	V_{Th}=V_2=V_0\frac{Z_{21}}{Z_{11}+R_0}	\]
\end{itemize}

For $R_L$ to get maximum power, we need to have
\[ R_L=Z_{Th}=Z_{22}-\frac{Z_{12}Z_{21}}{Z_{11}+R_0}	\]
and the current is
\[ I_l=\frac{V_{Th}}{Z_{Th}+R_L}=\frac{V_{Th}}{Z_{Th}}
	=\frac{2V_0Z_{21}}{2(Z_{22}Z_{11}-Z_{12}Z_{21}+Z_{22}R_0)} \]


\end{itemize}


\end{enumerate}

\end{document}

