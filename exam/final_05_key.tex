\documentstyle[11pt]{article}
\usepackage{html}
\begin{document}
\begin{center}
{\Large \bf  Midterm Exam 3 ---- E84, Spring, 2005}
\end{center}

\begin{itemize}
\item Take home, open everything except discussion.
\item Mark your start and end times. % Don't spend more than 3 hours.
\item Due Monday in class.
\item Mark your name and question number clearly on top of each page.
	Indicate the total number of pages submitted.
\item When solving a problem, list all the steps. In each step, describe 
	what you are doing in English, then show the calculation and the 
	result of the step. A final answer, even if correct, without 
	evidence of the steps leading to the answer will not receive credit.
\end{itemize}

\begin{enumerate}

\item {\bf Problem 1. (20 points)} 

Find the output voltage $V_{out}$ in both (a) and (b) as a logic function of 
the three input voltages $V_1$, $V_2$ and $V_3$, each of which could be either 
0V or 5V, as listed in the following tables. Assume the voltage drop across 
a conducting diode is $0.7V$.

\htmladdimg{../midterm3i.gif}

If we assume positive logic, i.e., high voltage is logical 1 (True) and low 
voltage is logical 0 (False), then circuit (a) is a NOR (not OR) gate, and 
circuit (b) is a NAND (not AND) gate:
\[
\begin{tabular}{ccc|c}\\ \hline 
V1 & V2 & V3 & $V_{out}$ \\ \hline \hline
0V & 0V & 0V &     0     \\ \hline
0V & 0V & 5V &    4.3    \\ \hline
0V & 5V & 0V &    4.3    \\ \hline
0V & 5V & 5V &    4.3    \\ \hline
5V & 0V & 0V &    4.3    \\ \hline
5V & 0V & 5V &    4.3    \\ \hline
5V & 5V & 0V &    4.3    \\ \hline
5V & 5V & 5V &    4.3    \\ \hline
\end{tabular}
\]
\[
\begin{tabular}{ccc|c}\\ \hline
V1 & V2 & V3 & $V_{out}$ \\ \hline \hline
0V & 0V & 0V &     0.7   \\ \hline
0V & 0V & 5V &     0.7   \\ \hline
0V & 5V & 0V &     0.7   \\ \hline
0V & 5V & 5V &     0.7   \\ \hline
5V & 0V & 0V &     0.7   \\ \hline
5V & 0V & 5V &     0.7   \\ \hline
5V & 5V & 0V &     0.7   \\ \hline
5V & 5V & 5V &     5     \\ \hline
\end{tabular}
\]

\item {\bf Problem 2. (40 points)} 

The circuit below shows an NPN transistor circuit where $R_1=68k\Omega$,
$R_2=12k\Omega$, $R_c=10k\Omega$, $R_e=400\Omega$, $R_E=2k\Omega$, 
$\beta=100$, and $V_{CC}=20V$. As always, we assume $V_{be}=0.7V$ when
the transistor is conducting.

\htmladdimg{../midterm3f.gif}

\begin{enumerate}
\item Find the DC operating point in terms of $I_B$, $I_C$, and $I_E$, 
  together with $V_E$, $V_{CE}$ and $V_C$ (voltages at the three transistor 
  terminals with respect to ground). 
\item Find the AC small signal equivalent circuit, assuming $r_{be}=1k\Omega$
  and the frequency of the AC signal is high enough so that all capacitors 
  can be treated as AC short circuit. Find the AC voltage gain defined as the 
  ratio between $v_{out}$ and $v_{in}$. For simplicity, assume the voltage
  source $v_{in}$ has no internal resistance and no load $R_L=\infty$.
\item Find the input resistance defined as the ratio of the input voltage
  $v_{in}$ and the input current $i_b$ and the output resistance. Based on
  the expression of the input resistance, explain why $R_e$ is needed (not
  by-passed by the emitter capacitor).
\end{enumerate}

{\bf Solution:}
\begin{enumerate}
\item Use Thevenin's theorem to find 

  $V_{BB}=V_{CC}\frac{R_2}{R_1+R_2}=20\times \frac{12}{68+12}=3V$

  $R_B=\frac{R_1\times R_2}{R_1+R_2}=\frac{68\times 12}{68+12}=10.2\;k\Omega$
\item Find $I_B$ and $I_C$:

  $I_B=\frac{V_B-V_{be}}{R_B+(\beta+1)(R_E+R_e)}
  =\frac{3-0.7}{10.2+(100+1)(2+0.4)}=9.1\mu A$

  $I_C=\beta I_B=100\times 9.1\mu A=0.91 mA$, $I_E=I_C+I_B=0.92 mA$

\item Find $V_B$, $V_C$, $V_{ce}$:
  $V_E=(I_E)(R_e+R_E)=0.92 \times 2.4=2.2V $

  $V_C=V_{CC}-I_CR_C=20-0.91\times 10=10.9V$, $V_{ce}=V_C-V_E=10.9-2.2=8.7V$

\item The AC small signal equivalent circuit is shown below:

  \htmladdimg{../midterm3h.gif}
\item Find AC gain:
  
  $ v_{in}=i_b r_{be}+(\beta+1) i_b R_e=i_b[r_{be}+(\beta+1)R_e]$, 

  \[ v_{out}=-\beta i_b R_c=-\frac{\beta R_c v_{in}}{r_{be}+(\beta+1)R_e} \]
  \[ G=\frac{v_{out}}{v_{in}}=-\frac{100\times 10}{1+101\times0.4}=-24.15 \]

\item The input resistance is $r_{be}+(\beta+1)R_e$, in parallel with $R_1$ and
  $R_2$, much larger than $r_{be}$ without $R_e$. The output resistance is $R_C$.
\end{enumerate}


\item {\bf Problem 3. (40 points)} 

The circuit shown below is a simple Darlington transistor amplifier which is
composed of two transistors $T_1$ and $T_2$ with their collectors connected 
and the emitter of $T_1$ connected to the base of $T_2$. Assume $V_{CC}=20V$ 
and both transistors have $\beta=50$. 
\begin{enumerate}
\item Give the expressions of $I_{C1}$, $I_{E1}=I_{B2}$, $I_{C2}$, $I_{E2}$
  all in terms of $\beta$ and $I_{B1}$.
\item Given $R_C=1 k\Omega$, find $R_B$ so that $V_C=V_{CC}/2=10V$. For 
  simplicity, assume for both transistors $v_{be}=0.7V$.

\end{enumerate}

\htmladdimg{../midterm3g.gif}

{\bf Solution:}

\begin{enumerate}
\item $I_{C1}=\beta I_{B1}$, $I_{E1}=I_{B2}=(\beta+1) I_{B1}$, 
$I_{C2}=\beta I_{B2}=\beta (\beta+1) I_{B1}$, 
$I_{E2}=(\beta+1) I_{B2}=(\beta+1)^2 I_{B1}=(\beta^2+2\beta+1)I_{B1}$, 

\item In order for $V_C=V_{CC}-R_C (I_{C1}+I_{C2})=10V$, we need to have
$I_{C1}+I_{C2}=\beta I_{B1}+\beta (\beta+1) I_{B1}=2600 I_{B1}=10 mA$, i.e.,
$I_{B1}=10 mA/2600=3.8 \mu A$. 

$R_B=(V_{CC}-2V_{be})/I_{B1}=18.6/3.8=4.89 M\Omega$
\end{enumerate}

\end{document}

