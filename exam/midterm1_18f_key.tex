\documentstyle[11pt]{article}
\usepackage{html}

\begin{document}
\begin{center}
{\Large \bf  E84 Midterm Exam 1 --- Fall 2018}
\end{center}

\section*{\bf Instructions}

\begin{itemize}
\item Write your name on top of each page. Indicate the total number
  of pages submitted.
\item Spend no more than four hours to complete the test.
\item To get maximum partial credits, list the main steps in the 
  process of getting to the final answer. In each step, describe 
  concisely what you are doing, then show the calculation and box 
  the result. Also box the final answer.
\item A final answer, even if correct, will receive no credit
  without evidence of the steps leading to it.
\item Do not click the next part until you are ready to take the exam.
\end{itemize}

\section*{\bf Do not click until you are ready to take the exam}

{\bf The Problems}
\begin{enumerate}

\item {\bf Problem 1 (50 points)}

  All resistors in the following circuit are of $1\,\Omega$
  and all current sources are of $1\,A$. Find the following
  currents and voltages in the circuit:
  \begin{itemize}
  \item $I_0$ through $R_0$ (rightward)
  \item $I_1$ through $R_1$ (rightward)
  \item $I_2$ through $R_2$ (downward)
  \item $I_3$ through $R_3$ (upward)
  \item $I_4$ through $R_4$ (leftward)
  \item $V_0$ between top-right corner (+) and bottom-left corner (-)    
  \item $V_1$ between top-left corner (+) and bottom-right corner (-)    
  \end{itemize}

  \htmladdimg{../../lectures/figures/midterm1_18f0.png}

  {\bf Solution:}  Convert all three non-ideal current sources
  into non-ideal voltage sources of $1\;V$ in series with $1\,\Omega$
  resistors, find the total current through this loop 
  $I=(1+1-1)/(1+1+1+1)=1/4\,A$.Alternatively, by loop-current 
  method arround loop through $R_1,\;R_3,\;R_4,\;R_2$, we have
  $I+(I+1)+(I-1)+(I-1)=0$, i.e., $I=1/4$. We can further get
  \begin{itemize}
  \item $I_0=1\;A$
  \item $I_1=1/4\;A$
  \item $I_2=3/4\;A$
  \item $I_3=-5/4\;A$
  \item $I_4=-3/4\;A$
  \item $V_0=1/2$
  \item $V_1=5/4+1/4+1=5/2$
  \end{itemize}
  

\item {\bf Problem 1. (50 points)} 

  In the circuit below, $V=1\,V$, $I=3\,A$, $R_1=1\,\Omega$,
  $R_2=2\,\Omega$, $R_3=3\,\Omega$, $R_4=8\,\Omega$. 

  \htmladdimg{../../lectures/figures/midterm1_18f1.png}

  \begin{itemize}
  \item Find the Thevenin's model of the circuit to the left of 
    $R_L$, in terms of $R_T$ and $V_T$.
  \item Find the Norton's model of the circuit to the left of 
    $R_L$, in terms of $R_N$ and $I_N$.
  \end{itemize}

  Do these two parts independently.

%  \begin{comment}
  {\bf Solution:} 
  First find the equivalent resistance used in both models:
  \[
  R_T=R_N=2+1||(8+3)=2+\frac{11}{12}=\frac{35}{12}
  \]
  \begin{itemize}
  \item Thevenin's model: 

    Use superposition to find $V_T=V_{oc}$:
    Due to voltage source alone
    \[
    V_T'=1V\frac{3+8}{3+8+1}=\frac{11}{12}
    \]
    Due to current source alone, current through $R_1$ and $R_3$ is
    $3A\times 8/(8+3+1) =2A$, current through $R_4$ is
    $3A\times (1+3)/(8+3+1) =1A$, voltega drop across $R_2$ is
    $-3\times 2=-6V$, voltage drop across $R_1$ is $-2V$, therefore
    $V''_T=-8V$, and
    \[
    V_T=V'_T+V''_T=\frac{11}{12}-8=-\frac{85}{12}
    \]
  \item Nordon's model:

    Use superposition to find $I_T=I_{sc}$:
    Due to voltage source slone
    \[
    I'_N=\frac{V}{1+2||(3+8)}\times\frac{8+3}{2+8+3}=\frac{11}{35}
    \]
    Due to current source slone:
    Convert the triangle formed by the three resistors on top
    (of $1\Omega$, $3\Omega$, and $8 \Omega$) to a Y with
    $R_a=2/3$ (top-left), $R_b=1/4$ (top-right), $R_c=2$ (bottom),
    then find current through short circuit on the right by current
    divider as
    \[
    I''_N=-3A\frac{2+2/3}{2+2/3+1/4}=-\frac{96}{35}
    \]
    The total current is
    \[
    I_N=I'_N-I''_N=\frac{11}{35}-\frac{96}{35}=-\frac{17}{7}
    \]
  \end{itemize}

  Nordon's model can be converted to Thevenin's model:
  \[
  I_N\times R_N=-\frac{17}{7}\times\frac{35}{12}=-\frac{85}{12}
  \]

\end{enumerate}
\end{document}


  In the circuit below, $R_5=1\,\Omega$, all other resistors are
  $2\,\Omega$. $V_1=4\,V$, $V_2=8\,V$, $I_1=2\,A$, $I_2=8\,A$.

  (a) Find the voltage $V$ across, current $I$ through, and poswer
  $P$ consumed by $R_7=2\,\Omega$.
  

  (b) If you can change the resistance of $R_7$ for it to consume
  maximum power, what value should it be? What is this power?



  {\bf Solution:} 
  (0) User Thevenin's theorem. Disconnect $R_7$ as the load.

  (1) Convert $I_1=2$ and $R_1=2$ into a voltage source with $V'_1=4\,V$
  in series with $V_1=4\,V$, we get a voltage of $8\,V$ voltage in series 
  with $R_1=2$. 
  
  (2) Convert $V_2=8$ and $R_4=2$ into a current source $I'_2=4\,A$, in 
  parallel with $R_4||R_6=2||2=1,\Omega$. Then turn it back to a voltage 
  source $V'_2=4\,V$ in series with $1,\Omega$. 

  (3) Convert $I_2=8$ and $R_5=1$ into a voltage source of $8\,V$ in series
  with $R_5=1\,\Omega$.

  (4) Combine the two voltage sources in (2) and (3) to get a voltage of 
  $4\,V$ in series with a resistor $R_5+R_4||R_6=2\,\Omega$.

  (5) Combine $8\,V$ and $2\,\Omega$ in (1) with $4\,V$ and $1\,\Omega$ 
  in (4) to get a voltage of $4\,V$ and $1\,\Omega$ to get a voltage of 
  $4\,V$ in series with a resistance of $4\,\Omega$. Find current $I=1\,A$
  in the loop.

  (6) $R_T=1\,\Omega$, $V_T=4+1=6\,V$. The current through $R_7=2\,\Omega$
  is $V_T/(R_T+R_7)=6/3=2\,A$, the voltage across it is $4\,V$, the power
  consumption is $8\,W$

  (7) For $R_7$ as the load to get maximum power, it needs to be 
  $R_7=R_T=1\,\Omega$. The voltage across it is $3\,V$, the current 
  through is $3\,A$, the power consumption is $9\,W$.
