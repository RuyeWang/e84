\documentstyle[11pt]{article}
\usepackage{html}
\begin{document}
\begin{center}
{\Large \bf E84 Final Exam}
\end{center}
\begin{enumerate}

\item Answer the following questions:
  \begin{itemize}
  \item The three AC RMS voltages across R, L, and C are measured to be
    $V_1=4\,V$, $V_2=5\,V$, and $V_3=2\,V$, respectively. Find the
    source voltage $V_0$. If the AC voltage source is replaced by a DC
    voltage source $V_0=10\,V$, Find the three voltages $V_1$, $V_2$, and 
    $V_3$.

    \htmladdimg{../../lectures/figures/Problems10.png}

  \item In the figure above, it is know that $R=10\Omega$,
    $L=100\,mH$, and $C=500\,\mu F$, and the current through the 
    loop is also known to be $i(t)=\cos(100\,t)\,A$, and find all 
    four voltages $v_0(t)$, $v_1(t)$, $v_2(t)$, and $v_3(t)$.

    {\bf Solution:} 
    \[
    v_1(t)=10\cos(100\,t),\;\;\;\;
    v_2(t)=100\times 0.1 \cos(100\,t+\pi/2)=10\cos(100\,t+\pi/2),\;\;\;\;
    v_3(t)=1/(100\times 500\times 10^{-6})\cos(100\,t)=20\cos(100,\,t-\pi/2)
    \]
    \[
    V_0=V_1+V_2+V_3=10+10\angle\pi/2+20\angle(-\pi/2)=10+10\angle(-\pi/2)
    \]
    \[
    v_0(t)=10\sqrt{2}\cos(100\,t-\pi/4)
    \]

  \item The three AC RMS currents through R, L, and C are measured by the
    three ammeters to be $I_1=4\,A$, $I_2=5\,A$, and $I_3=2\,A$, respectively. 
    Find the source current $I_0$. If the AC voltage source is replaced by a
    DC current source $I_0=1\,A$, Find the three currents $I_1$, $I_2$, and 
    $I_3$ measured respectively by ammeters $A_1$, $A_2$, and $A_3$.

    \htmladdimg{../../lectures/figures/Problems11.png}

  \end{itemize}

\item In the circuit below, $R_1=20\Omega$, $R_2=10\Omega$, $R_3=30\Omega$,
  $L=10\,mH$, $I_0=5A$. The circuit is at steady state before the switch is 
  turned from position 1 to position to at $t=0$. Find the three voltages 
  $v_1(t)$, $v_2(t)$ and $v_L(t)$ for $t>0$ (assuming the top terminal is 
  positive).
  
  \htmladdimg{../../lectures/figures/Problems12.png}

  {\bf Solution:} $\tau=L/(R_2+R_3)=10^{-2}/40=2.5\times 10^{-4}\,s$
  \[
  i_L(0^-)=i_L(0^+)=I_0\frac{R_1}{R_1+R_3}=5A\frac{20}{50}=2A,\;\;\;\;\;
  i_L(\infty)=0,\;\;\;\;\;i_L(t)=2\,e^{-t/\tau}
  \]
  \[ 
  v_1(t)=5A,\;\;\;\; v_2(t)=R_2 i_L(t)=20\,e^{-t/\tai},\;\;\;\;\;
  v_3(t)=R_3 i_L(t)=30\,e^{-t/\tai}
  \]

\item The circuit below shows a simple means for obtaining improved 
  bias stability of the DC operating point of the transistor. As 
  always, assume $V_{BE}=0.7V$ when answering the following questions.

  \htmladdimg{../../lectures/figures/midterm3e.gif}

  \begin{itemize}
  \item Complete the event chain below to show qualitatively that $R_B$
    connected to the collector $C$ (rather than $V_{CC}$) introduces a 
    negative feedback by which the DC operating point tends to be 
    stabalized:
    \[
    \Longrightarrow I_C \uparrow \cdots\cdots\cdots 
    \Longrightarrow I_C \downarrow     
    \]
  \item Derive an expression for $I_C$ in terms of $R_B$ and $R_C$ 
    and $\beta$.
  \item Find $R_C$ and $R_B$ so that the DC operating point is $V_{CE}=5V$ 
    and $I_C=2mA$, when $\beta=100$ and $V_{CC}=10V$.
  \item Find $V_{CE}$ and $I_C$ for $\beta=50$, $\beta=100$, and $\beta=200$ 
    based on the resistances found above.
  \end{itemize}

  {\bf Solution:}

  \[
  \Longrightarrow I_C \uparrow \Longrightarrow V_C \downarrow
  \Longrightarrow I_B \downarrow \Longrightarrow I_C \downarrow 
  \]

  \[
  I_B=\frac{V_{CC}-0.7}{(\beta+1)R_C+R_B},\;\;\;\;\;
  I_C=\beta I_B=\frac{\beta(V_{CC}-0.7)}{(\beta+1)R_C+R_B} 
  \]
  $I_C=2mA$, $V_C=5V$,  $R_C=(V_{CC}-V_C)/I_C=5V/2mA=2.5K\Omega$,  
  $I_B=I_C/\beta=0.02mA$,  $R_B=(5-0.7)/0.02=4.3/0.02=215K\Omega$.

  \[
  I_B=\frac{V_{CC}-0.7}{(\beta+1)R_C+R_B}=\frac{9.3}{101\times 2.5K+215}
  =0.02mA
  \]

  \[
  I_C=\beta I_B=2 mA,\;\;\;V_C=V_{CC}-(\beta+1)I_B=5V 
  \]

  \[
  I_C=\beta I_B=2.6mA,\;\;\;V_C=V_{CC}-(\beta+1)I_B=3.5V 
  \]


\item Find the frequency response function (FRF) of the op-amp circuit 
  below. What kind of filter is this? First or second order? Low-pass, 
  high-pass, or band-pass? In any case, find the cut-off frequency 
  $\omega_c$ of the filter in terms of the given circuit parameters, so
  that $|H(j\omega_c)|=|H(j\omega_p)|/\sqrt{2}$, where $\omega_p$ is the
  peak frequency at which $|H(j\omega_p)|$ reaches maximum magnitude.  

  \htmladdimg{../../lectures/figures/Problems14.png}

  {\bf Solution:} Let $v^+\approx v^-=v$
  \[
  v=v_{out}\frac{R_1}{R_1+R_2},\;\;\;\;\;\;\;\;\;\;\frac{v_{in}-v}{R}=v\;j\omega
  \]
  Substituting $v$ in the first equation into the second, we find
  \[
  H(j\omega)=\frac{v_{out}}{v_{in}}=\frac{R_1+R_2}{R_1}\;\frac{1}{j\omega RC+1}
  \]
  This is a first-order low-pass filter with cut-off frequency $\omega_c=1/RC$.


\end{enumerate}
\end{document}

