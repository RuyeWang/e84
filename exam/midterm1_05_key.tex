\documentstyle[11pt]{article}
\usepackage{html}
\begin{document}
\begin{center}
{\Large \bf  Midterm Exam 1 ---- E84, Spring 2005}
\end{center}

\section*{E84 Midterm Exam 1}

{\bf Instructions}
\begin{itemize}
\item Take home, open everything except discussion. Due Wednesday in class.
\item Mark your start and end times. Don't spend more than 3 hours.
\item Compare your print-out of the exam with the online version to make
	sure your hard copy is complete.
\item Mark your name and question number clearly on top of each page.
	Indicate the total number of pages submitted.
\item When solving a problem, list all the steps. In each step, describe 
	concisely what you are doing in English, then show the calculation 
	and the result of the step. A final answer, even if correct, without 
	evidence of the steps leading to the answer will not receive credit.
\end{itemize}

{\bf The Problems}
\begin{enumerate}

\item {\bf Problem 1. (33 points)} 
Show the relationship between the output voltage $V_o$ and the three input
voltages $V_1$, $V_2$ and $V_3$ of the circuit shown below, where all 
resistors have the same resistance value $R$. Extrapolate your result to 
cover the general case of $n$ inputs $V_i$, $i=1,\cdots,n$.

\htmladdimg{../midterm1d.gif}

{\bf Solution:}

\[ \frac{V_1-V_0}{R}+\frac{V_2-V_0}{R}+\frac{V_3-V_0}{R}=\frac{V_0}{R}
,\;\;\;\;\; V_0=\frac{V_1+V_2+V_3}{4}	\]
In general
\[ V_0=\frac{1}{n+1}\sum_{i=1}^n V_i \]

\item {\bf Problem 2. (34 points)} 

In the figure below, $V_1=20V$, $V_2=V_3=10V$, $R_1=R_5=10\Omega$, 
$R_2=R_4=5\Omega$, $R_6=6\Omega$, $R_3=1.5\Omega$. Find voltage $V_{ab}$

\htmladdimg{../midterm1e.gif}

{\bf Solution:} Use Thevenin's theorem. 
\begin{itemize}
\item Remove $R_3$ and $V_3$ as the load
\item Find open-circuit voltage $V_T$:
\[	V_{ab}=V_a-V_b=V_1\frac{R_5}{R_1+R_5}-V_2\frac{R_4}{R_2+R_4}
	=20 \frac{10}{10+10}-10 \frac{5}{5+5}=10-5=5 \]
\item Find $R_T$:
\[	R_T=R_1//R_5+R_6+R_2//R_4
	=\frac{R_1R_5}{R_1+R_5}+R_6+\frac{R_4R_2}{R_2+R_4}=9 \]
\item Connect load of $V_3$ and $R_3$, find current
\[	I=\frac{V_T-V_3}{R_T+R_3}=\frac{5-10}{9+6}=-\frac{1}{3} \]
\item Find $V_{ab}$
\[	V_{ab}=-\frac{1}{3}\times 6 +10=8V	\]
\end{itemize}

\item {\bf Problem 3. (33 points)} 

In the circuit below, $I_0=6A$, $V_0=5V$, $R_2=R_4=4\Omega$, $R_1=2\Omega$,
$R_3=8\Omega$. Find the current through $R_5=6\Omega$.

\htmladdimg{../midterm1f.gif}

{\bf Solution:} Use superposition. First consider the current source $I_0$ 
only with $V_0=0$ (short-circuit). Convert the delta composed of the top three 
resistors ($R_1$, $R_2$ and $R_5$) to Y:
\[	R_a=\frac{2\times 4}{2+6+4}=\frac{2}{3},\;\;\;\;
	R_b=\frac{2\times 6}{2+6+4}=1,\;\;\;\;R_c=\frac{4\times 6}{2+6+4}=2 \]
where $R_a$ is in series with $I_0$, and $R_b$ and $R_c$ are in series with 
$R_4$ and $R_3$, respectively. Treating the two branches as two voltage dividers,
we find the voltage across $R_5$ is zero and therefore $I'=0$,

Next consider the voltage source $V_0$ only, with $I_0=0$ (open-circuit).
The total resistance of the loop is $15\Omega$, and the total current is 
$I_{total}=5V/15\Omega=1/3\;A$, and the current through $R_5$ can be  found
by current divider to be $I''=1/6\;A$. The current due to both $I_0$ and $V_0$
is therefore $I=I'+I''=1/6\;A$ 

\end{enumerate}

\end{document}

