\documentstyle[11pt]{article}
\usepackage{html}
\begin{document}

\begin{center}
{\Large \bf E84 Midterm 1 ---- 2011 Spring}

{\large \bf Professors Tanenbaum and Wang}

{\large \bf Student Name:}

\end{center}



\begin{itemize} 
\item Open-book, open-notes, calculator OK, no laptop.
\item Four problems, 25\% each.
\item Put your final answers in the space provided, show concisely your derivation/steps
  in space available or on separate papers.
\item Put your name on every page of your test submitted.
\end{itemize} 


\begin{itemize} 

\item {\bf Problem 1:}
  Find the voltage $V_i$, current $I_i$, and consumed power $P_i$ associated with 
  each of the four resistors $R_i$ ($i=1,2,3,4$) in the circuit below, where 
  $R_1=100\Omega$, $R_2=5\Omega$, $R_3=200\Omega$, $R_4=50\Omega$, and the 
  voltage and current sources are respectively $V=50V$ and $I=0.2A$. Also find
  the power $P_V$ provided by the voltage source and the power $P_I$ provided 
  by the current source.

  \htmladdimg{../11sMidterm1Fig0.gif}

  Show your steps concisely and put your final answers in the two tables provided
  below. Use an arrow to indicate the direction of current. For example, if the
  current flows through a resistor $R$ from left to right, then use a right arrow
  $\rightarrow$ to indicate its direction.

  \begin{tabular}{r|c|c|c|c}\hline
    $R_i$     & voltage $V_i$ & current $I_i$ & power $P_i$ & current direction \\ \hline
    $R_1=100$ &       &      &                    \\
    $R_2=  5$ &       &      &                    \\
    $R_3=200$ &       &      &                    \\
    $R_4= 50$ &       &      &                    \\ \hline
  \end{tabular}

  \begin{tabular}{c|l}\hline
    Energy source  & Power delivered \\ \hline
    Voltage source & $P_V=$    \\
    Current source & $P_I=$    \\ \hline
  \end{tabular}

  
  {\bf Solution:} 
  $V_b=50V$, 
  \[ \begin{array}{ll}
    \mbox{middle node $V_a$:} & V_a/5+(V_a-V_c)/200+(V_a-50)/100=0 \\
    \mbox{bottom node $V_c$:} & (V_c-V_a)/200+(V_c-50)/50=0.2 \end{array} 
  \]
  Solving this we get: 
  \[ V_a=3.46,\;\;\;\;\;V_c=48.69,\;\;\;\;V_b=50 \]

  \begin{tabular}{r|r|r|r|c}\hline
    $R_i$ & voltage $V_i$ & current $I_i$ & power $P_i$ & current direction \\ \hline
    $R_1=100$ & 46.54 & 0.465 &  21.64W & $\leftarrow$   \\
    $R_2=  5$ &  3.46 & 0.690 &   2.39W & $\leftarrow$   \\
    $R_3=200$ & 45.23 & 0.226 &  10.22W & $\uparrow$     \\
    $R_4= 50$ &  1.31 & 0.026 &  0.034W & $\downarrow$   \\ \hline
  \end{tabular}

  \begin{tabular}{c|c}\hline
    Energy source  & Power delivered \\ \hline
    Voltage source & $P_V=24.55W$    \\
    Current source & $P_I= 9.74W$    \\ \hline
  \end{tabular}

{\bf \large Student Name:}

\item {\bf Problem 2:}
  \begin{itemize}
  \item (a) For the resistive load $Z_L=R_L$ in the figure below to get maximum 
    power from the voltage source with an internal resistance $R_0$, a matching
    circuit composed of two inductors $L_1=L_2$ with impedance $jX$ and one 
    capacitor $C$ with impedance $-jX$ can be used. Find the impedance $X$
    in terms of $R_0$ and $R_L$. Show your derivation.

    \htmladdimg{../11sMidterm1Fig1.gif}

    {\bf Answer:}
  
      {\bf Solution:} $X=\sqrt{R_0 R_L}$. 

  \item (b) If the load $Z_L=R_L-jX_L$ becomes capacitive (e.g., a resistor $R_L$
    in series with a capacitor $C_L$), how would you modify the matching circuit
    for the load to consume maximum real power? (Your answer needs no more than
    one sentence.)

    {\bf Answer:}
      {\bf Solution:} Change $L_2$ so that its impedance is $j(X+X_L)$ 

  \item (c) In part (b) if the frequency of the voltage source is 
    $f=159,154$ Hz, $R_0=100\;\Omega$, $R_L=4\;\Omega$ and $C_L=0.05 \mu F$,
    find the values for $L_1$, $L_2$ and $C$ of the matching circuit.

    \begin{tabular}{l|l|l}\hline
      Inductance $L_1$ & Inductance $L_2$ & Capacitance $C$ \\ \hline
      $L_1=$            & $L_2=$          & $C=$          \\ \hline
    \end{tabular}

    {\bf Solution:} $X_L=1/2\pi f C_L=20\;\Omega$, 
    $X_1=\sqrt{100\times 4}=20\;\Omega$, $X_2=X_1+X_L=20+20=40\;\Omega$, 
    $L_1=\frac{20}{2\pi f}=2\times 10^{-5}\;H$, $L_2=4\times 10^{-5}\;H$, 
    $C=\frac{1}{2\pi f\;20}=0.05\times 10^{-6}=0.05\;\mu F$.

    \begin{tabular}{l|l|l}\hline
      Inductance $L_1$ & Inductance $L_2$ & Capacitance $C$ \\ \hline
      $L_1=2\times 10^{-5}$ & $L_2=4\times 10^{-5}$ & $C=0.05\;\mu F$          \\ \hline
    \end{tabular}
  \end{itemize}


\end{itemize}



\end{document}
