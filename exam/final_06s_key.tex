\documentstyle[11pt]{article}
\usepackage{html}
\begin{document}
\begin{center}
{\Large \bf  Final Exam ---- E84, Spring, 2006}
\end{center}

\begin{itemize}
\item Take home, open everything except discussion.
\item Mark your start and end times. % Don't spend more than 3 hours.
\item Mark your name and question number clearly on top of each page.
	Indicate the total number of pages submitted.
\item When solving a problem, list all the steps. In each step, describe 
	what you are doing in English, then show the calculation and the 
	result of the step, and box the final answer. A final answer, 
	even if correct, without evidence of the steps leading to the 
	answer will not receive credit.
\item Due ...
\end{itemize}

\begin{enumerate}

\item {\bf Problem 1. (25 points)} 

An AC voltage source of 60 Hz and 26V (rms) is connected to a coil
modeled by a resistor $R$ in series with an inductor $L$ as the load. 
It is found through measurement that the current through the load is 
2A (rms) and the real power consumed is 20 Watts. Find (a) $R$, (b) $L$,
(c) the power factor of the load, (d) the reactive power, and (e) the 
apparent power.

 {\bf Solution:} Since $L$ does not consume energy, we have $W=RI^2$,
 i.e., $20=2^2 R$, or $R=5\Omega$. Also, according to Ohm's law, we get
 the impedance to be
 \[ |Z|=\frac{26V}{2A}=[R^2+(\omega L)^2]^{1/2}=[5^2+(2\pi 60 L)^2]^{1/2}=13 \]
 Solving this we get $L=12/2\pi 60=0.0318\;H$. The power factor is
 $R/Z=0.345$. The apparent power is $26V \times 2A=52W$
 and the reactive power is $\sqrt{52^2-20^2}=48W$.

\item {\bf Problem 2. (25 points)} 

The parameters of the Y-model of the two-port network are $Y_{11}=-1$,
$Y_{12}=Y_{21}=2$, and $Y_{22}=-3$ (all in unit $\Omega^{-1}$). The 
voltage source is $V_{in}=12V$, and $R_1=3\Omega$, $R_2=6\Omega$, 
$R_3=2\Omega$, $R_4=1\Omega$. Find $V_{out}$.

\htmladdimg{../final06s_1.gif}

 {\bf Solution:}
 
 \begin{itemize}
 
 \item Find Z-model from the given Y-model:
 \[ {\bf Z}={\bf Y}^{-1}=
     \left[ \begin{array}{ll} -1 & 2 \\ 2 & -3 \end{array} \right]^{-1}
    =\left[ \begin{array}{ll} 3 & 2 \\ 2 & 1 \end{array} \right] \]
 \item Use Thevenin's theorem to find
 \[ V_{T}=V_{in} \frac{Z_2}{Z_1+Z_2}=12\frac{6}{3+6}=8V,\;\;\;\;
    R_T=\frac{Z_1 Z_2}{Z_1+Z_2}=\frac{3\times 6}{3+6}=2K\Omega  \]
 \item Set up the equations:
 \[ \left\{ \begin{array}{l} V_1=Z_{11}I_1+Z_{12}I_2=3I_1+2I_2 \\
 	V_2=Z_{21}I_1+Z_{22}I_2=2I_1+I_2 \end{array} \right. \]
 \[ \left\{ \begin{array}{l} V_1=V_T-I_1 Z_T=8-2I_1 \\ V_2=-(Z_3+Z_4)I_2=-3I_2 
 \end{array} \right. \]
 \item Solve 
 \[ \left\{ \begin{array}{l} 3I_1+2I_2=8-2I_1 \\ 2I_1+I_2=-3I_2 \end{array} \right. \]
 to get $I_1=2$ and $I_2=-1$. 
 \item Find $V_2=-3\times (-1)=3$, and $V_{out}=V_2 Z_4/(Z_3+Z_4)=1V$
 \end{itemize}

\item {\bf Problem 3. (25 points)} 

A DC circuit containing a diode is shown below, where $V_0=10V$, $R_1=0.1K\Omega$,
$R_2=0.6K\Omega$, $R_3=0.3K\Omega$, and the diode can be described by
\[ I_D=I_0 ( e^{V_D/\eta V_T}-1 ) \]
where $I_0=10\times 10^{-11}$, $\eta=1.4$ and $V_T=0.026V$. Find the voltage 
$V_D$ across and current $I_D$ through the diode in the following two ways:
\begin{itemize}
\item (1) graphically on the $I_D$ vs. $V_D$ plot, and
\item (2) algebraically by assuming $V_D=0.7$ when the diode is forward biased.
\end{itemize}

\htmladdimg{../final06s_2.gif}

 {\bf Solution:}
 \begin{itemize}
 \item (1) When $I_D=0$, $V_D=V_0 R_2/(R_1+R_2)=10 \times 0.6/(0.1+0.6)=8.57V$, 
 when $V_D=0$, current through diode is
 \[ I_D=\frac{V_0}{R_1+R_2R_3/(R_2+R_3)}\frac{R_2}{R_2+R_3}
    =V_0 \frac{R_2}{R_1R_2+R_2R_3+R_1R_3}=22.2 mA \]
 The load line is determined by these two points: 
 $(I_D=0, V_D=8.57)$ and $(V_D=0, I_D=0.22.2 mA)$, the intersection is about
 $I_D=20mA, V_D=0.7V$.
 \item (1) Assume $V_D=0.7V$, and the voltage across $R_2$ is x, then we have
 \[ \frac{x}{R_2}+\frac{x-0.7}{R_3}=\frac{V_0-x}{R_1} \]
 i.e.,
 \[ \frac{x}{0.6}+\frac{x-0.7}{0.3}=\frac{10-x}{0.1} \]
 which can be solved to get $x=6.82V$, current through $R_1$ is
 $(V_0-x)/R_1=(10-6.82)/0.1=31.8 mA$, current through $R_2$ is
 $x/R_2=6.82/0.6=11.4 mA$, current through diode is $31.8-11.4=20.4mA$.
 \end{itemize}

\item {\bf Problem 4. (25 points)} 

Find the DC operating point $(I_C, V_{CE})$ of the transistor circuit 
given below, where $R_1=100K\Omega$, $R_2=300K\Omega$, $R_C=R_E=2K\Omega$, 
$V_{CC}=12V$, and $\beta=100$. If you find the DC operating point is not 
in the middle of the linear region of the output characteristic plot, try 
to modify $R_2$ so that it will be in the middle of the linear region (to 
maximize the dynamic range of the AC output).

\htmladdimg{../../lectures/figures/transistorbiasingb.gif}

 {\bf Solution:} 
 
 Find the load line: 
 When $I_C=0$, $V_{CE}=V_{CC}=12V$, when $V_{CE}=0$, $I_C=V_{CC}/(R_C+R_E)=3mA$
 
 \[ R_{B}=\frac{R_1 R_2}{R_1+R_2}=75K,\;\;\;V_{BB}=12\frac{R_2}{R_1+R_2}=9V \]
 \[ I_B=\frac{V_{BB}-V_{BE}}{R_B+(\beta+1)R_E}=\frac{9-0.7}{75+202}=30\mu A \]
 \[ I_C=\beta I_B=3 mA, \;\;\;\;V_{CE}=V_{CC}-(R_C+R_E)I_C=0V \]
 The DC operating point is in saturation region.
 
 Now we modify $R_2$ to move Q-point to the middel point where $I_C=1.5mA$.
 \[ V_{BB}=12R_2/(100+R_2),\;\;\;\;R_B=100R_2/(100+R_2) \]
 We then plug these into 
 \[ I_C=\beta \frac{V_{BB}-V_{BE}}{(\beta+1) R_E+R_B}=1.5 \]
 and solve it for $R_2$ to get $R_2=55K$.

\end{itemize}

\end{document}

