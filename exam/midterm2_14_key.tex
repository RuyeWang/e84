\documentstyle[11pt]{article}
\usepackage{html}
\begin{document}
\begin{center}
{\Large \bf E84 Midterm Exam 2}
\end{center}

\begin{itemize}
\item Take home, open everything except discussion. Due Monday (4/9) in class.
\item There are in total three problems (33, 33, and 34 points).
\item Mark your start and end times. Don't spend more than 3 hours.
\item Compare your print-out of the exam with the online version to make sure 
  your hard copy is complete.
\item Mark your name and question number clearly on top of each page.
  Indicate the total number of pages submitted.
\item When solving a problem, list all the steps. In each step, concisely
  state what you are doing in English, then show the calculation and the 
  result of the step. A final answer, even if correct, without 
  evidence of the steps leading to the answer will not receive credit.
\end{itemize}

\begin{enumerate}

\item {\bf Problem 1. (33 points)} 

  In the circuit below, the voltage source is $v(t)=150\sqrt{2}\cos (314 t)\;V$,
  $R_1=R_2=R_3=R$, and the three currents $i_1(t)$, $i_2(t)$, and $i_3(t)$ have
  the same amplitudes. Moreover, it is also known that the total real power
  consumption of the circuit is $P=1500 W$. Find $R$, $L$, and $C$, also find
  $i_1(t)$, $i_2(t)$, and $i_3(t)$.

  \htmladdimg{../midterm2g.gif}

%  \begin{comment}
  {\bf Solution:}

  Represent voltage source $v(t)$ by phasor $\dot{V}=150\angle 0$, and the voltage 
  across the parallel branches (RC and RL) by $\dot{V}_1$, also represent the currents
  by $\dot{I}_1$, $\dot{I}_2=\dot{I}_{RL}$, $\dot{I}_3=\dot{I}_{RC}$. 

  As $|\dot{I}_2|=|\dot{I}_3|$, we have 
  \[ 
  |Z_{RC}|=\bigg|R-\frac{j}{\omega C}\bigg|=|Z_{RL}|=|R+j\omega L|
  \]
  But also as $R_2=R_3$, we have $\omega L=1/\omega C=X$, i.e., 
  $Z^*_{RL}=(R+jX)^*=R-jX=Z_{RC}$, and 
  \[
  Z_{RC}||Z_{RL}=\frac{Z_{RC}Z_{RL}}{Z_{RC}+Z_{RL}}=\frac{(R+jX)(R-jX)}{R+jX+R-jX}
  =\frac{R^2+X^2}{2R}
  \]
  is real, therefore $i_1(t)$ is in phase with $v(t)$. 

  Since $\dot{I}_1=\dot{I}_2+\dot{I}_3$, and $|\dot{I}_1|=|\dot{I}_2|=|\dot{I}_3|$,
  we know they form an equilateral triangle, i.e., 
  \[ 
  \left\{ \begin{array}{ll}
    \dot{I}_1=\dot{I}_R=I \angle 0^\circ & \mbox{($\dot{I}_1$ thru $R_1$ in phase with $\dot{V}=150\angle 0^\circ $)}\\
    \dot{I}_2=\dot{I}_{RL}=I \angle -60^\circ & \mbox{($\dot{I}_2$ thru $L$ lagging $\dot{V}_1$)}\\
    \dot{I}_3=\dot{I}_{RC}=I \angle  60^\circ & \mbox{($\dot{I}_3$ thru $C$ leading $\dot{V}_1$)}
  \end{array} \right. 
  \]
  Since $\dot{I}_1$ is in phase with $\dot{V}=150\angle 0^\circ$, we have
  \[
  |\dot{I}_1|=\frac{P}{V}=\frac{1500}{150}=10A =|\dot{I}_2|=|\dot{I}_3|
  \]
  and we get
  \[
  i_1(t)=10\times 1.414 \cos(314t),\;\;\;\;
  i_2(t)=10\times 1.414 \cos(314t-\pi/3),\;\;\;\;
  i_1(t)=10\times 1.414 \cos(314t+\pi/3)
  \]
  Also as only the three resistors $R_1=R_2=R_3$ consume real power, they each consume 
  $P/3=500W$ and 
  \[
  R_1=R_2=R_3=\frac{P}{I^2}=\frac{500}{10^2}=5\Omega 
  \]
  Then we also get
  \[
  V_1=V-RI=150-5\times 10=100 V 
  \]
  and
  \[
  |Z_{RL}|=\frac{|\dot{V}_1|}{|\dot{I}_2|}=|Z_{RC}|=\frac{|\dot{V}_1|}{|\dot{I}_3|}
  =\frac{100}{10}=10
  \]
  i.e.,
  \[
  |Z_{RL}|=\sqrt{R^2+(\omega L)^2},\;\;\;\;\;
  |Z_{RC}|=\sqrt{R^2+(1/\omega C)^2}
  \]
  Solving these (with $\omega=314$) we get
  \[
  L=0.0276\,H, \;\;\;\;\;C=367.7\,\mu F 
  \]
%  \end{comment}


\item {\bf Problem 2. (33 points)} 

  In the circuit below, $R_1=150\Omega$, $R_2=50\Omega$, $L=0.2H$, $C=5\mu F$.
  The input voltage is $v_s(t)=70\sin(\omega t+0.644)=70\sin(1000 t+0.644)$.
  The system is in steady state before the switch is closed at $t=0$. Find voltage 
  $v_C(t)$ across $C$ and current $i_L(t)$ through $L$ for $t>0$.

  \htmladdimg{../midterm2f.gif}

%  \begin{comment}

  {\bf Solution:} 
  The phasor form of the input voltage is:
  \[
  \dot{V}_s=70/\sqrt{2}\angle 0.644\,Rad =49.49\angle 36.9^\circ 
  \]
  Find $i_L(t)$:
  \[
  \dot{I}_L=\frac{\dot{V_s}}{R_1+R_2+j(\omega L-1/\omega C)}
  =\frac{49.49\angle 0.644}{150+50+j(200-200)}=0.247\angle 0.644
  \]
  \[
  i_L(t)=0.35\;\sin(\omega t+0.644)\;\;\;\;\;(t<0) 
  \]
  \[
  i_L(0)=0.35\;\sin(0.644)=0.21\;A 
  \]
  \[
  \dot{V}_C=Z_C \dot{I}=\dot{I}/j\omega C =49.49\angle 0.926
  \]
  \[
  v_C(0)=70\;sin(-53.1^\circ)=-56\;V 
  \]

  For $t>0$, the switch is closed, $\tau_C=R_2C=2.5\times 10^{-4}$, 
  $1/\tau_1=4000$, $\tau_L=L/R_1=1.33\times 10^{-3}$, $1/\tau_L=750$.
  As the steady state of $v_C(t)$ is zero, we can find $v_C(t)$ to be
  \[
  v_C(t)=-56\;e^{-4000  t} 
  \]
  Find steady state of $i_L(t)$:
  \[
  \dot{I}_L=\frac{\dot{V}_s}{R_1+j\omega L}=\frac{49.49\angle 0.644}{150+j200}
  =0.2\angle -0.283 = 0.2\angle -16.23^\circ
  \]
  the steady state of $i_L(t)$ is
  \[ 
  i_L(t)=0.2\sqrt{2}\sin(\omega t-0.283)=0.283\sin(\omega t-0.283)
  \]
  Evaluating $i_L(t)$ at $t=0$ we get:
  \[ 
  i_L(0)=0.283\sin(-0.283)=-0.08 
  \]
  The complete $i_L(t)$ is
  \[
  i_L(t)=0.283\sin(\omega t-16.23^\circ)+[0.21-(-0.08)]e^{-t/\tau_L}
  =0.283\sin(\omega t-0.283)+0.29\;e^{-750\;t} 
  \]

%  \end{comment}


%\item {\bf Problem 3. (34 points)} 
%
%In the circuit shown below, the voltage source $v(t)=100\;\sqrt{2}\;sin\;314t\;$ volts,
%and the effective values of the three currents $i$, $i_L$ and $i_C$ are the same. The
%total real energy consumed by the circuit is 866 W. Find the values of $R$, $L$ and $C$.
%(Hint: represent all currents $i(t)$, $i_C(t)$, $i_L(t)$ and voltage $v(t)$ as phasors 
%$\dot{I}$, $\dot{I}_C$, $\dot{I}_L$, $\dot{V}$, and draw them as vectors to figure out
%how they are related.)


%\htmladdimg{../midterm2h.gif}

%{\bf Solution:}
%Let $\dot{V}=100\angle 0$ be the phasor representation of $v(t)$ so that 
%\[ v(t)=Im[\sqrt{2} \dot{V} e^{j\omega t}] =Im[\sqrt{2} \dot{V} e^{j314t}] \]
%where $\omega=2\pi f=314$, i.e., $f=50$ Hz. We have
%\[ \dot{I}_C=j\omega C \dot{V}=100\omega C \angle 90^\circ,
%\;\;\;\;\;\dot{I}_L=\frac{\dot{V}}{R+j\omega L}=\frac{100}{\sqrt{R^2+\omega^2 L^2}}\angle -\phi \]
%where $\phi=tan^{-1}(\omega L/R)$. Due to KCL, we have
%\[ \dot{I}=\dot{I}_C+\dot{I}_L \]
%but also as given
%\[ |\dot{I}|=|\dot{I}_C|=|\dot{I}_L|=I \]
%we conclude that these three currents are equal in magnitude and $60^\circ$ apart in 
%phase angle, as shown below:
%\htmladdimg{../midterm2h1.gif}

%where $\dot{I}_C$ is $90^\circ$ ahead of $\dot{V}$, $\dot{I}$ is $\theta=30^\circ$
%ahead of $\dot{V}$, which in turn is $\phi=30^\circ$ ahead of $\dot{I}_L$. Therefore,
%\[ \dot{I}=I\angle \theta=I\angle 30^\circ,\;\;\;\;\;\dot{I}_L=I\angle \phi=I\angle -30^\circ,
%\;\;\;\;\;\dot{I}_C=I\angle 90^\circ  \]
%As the real power is 
%\[ P=866=V I \cos\phi =100 I \cos (-30^\circ)=86.6 I \]
%we get
%\[ I=10 A \]
%therefore we also get
%\[ P=866=RI^2=R 100,\;\;\;\;\;R=8.66 \Omega \]
%and
%\[ \dot{I}=10\angle 30^\circ,\;\;\;\;\dot{I}_L=10\angle -30^\circ,\;\;\;\dot{I}_C=10\angle 90^\circ  \]
%But from above
%\[ \dot{I}_C=100\omega C \angle 90^\circ=10\angle 90^\circ \]
%we get:
%\[ \omega C=0.1,\;\;\;\;C=0.1/314=3.18\times 10^{-4} \; F\]
%Since we also have:
%\[ \dot{I}_L=\frac{100}{\sqrt{R^2+\omega^2 L^2}}\angle -\phi=10\angle -30^\circ \]
%solving this we get
%\[ \omega L=5,\;\;\;\;\;\;L=5/\omega=5/314=0.0159\;H \]


\item {\bf Problem 3. (34 points)} 

  \begin{itemize}
  \item (a) Find the frequency response function $H(\omega)$ of the circuit 
    shown in the figure, with the voltage across the resistor on the right 
    is treated as the output while the sinusoidal voltage source on the left 
    is treated as input.

  \item (b) Similar to a simple RCL series resonant circuit, this circuit
    can be used as a band-pass filter to pass signals around its resonant 
    frequency. Find this resonant frequency $\omega_{max}$, at which the 
    output voltage across $R$ is maximized, in terms of the component values 
    $R$, $C$, $L_1$, and $L_2$. 

  \item (c) Different from the simple RCL series resonant circuit, this 
    circuit also has a stop band, i.e., at a certain frequency $\omega_{min}$
    the output voltage $V_R$ is minimized. Find this frequency also in terms
    of the circuit components. 

  \item (d) Given $C=10\;\mu F$ and the two inductors are identical, 
    determine the values of $R$ and $L_1=L_2$, so that $\omega_{max}=5000
    \;rad/sec$. What is the corresponding stop frequency $\omega_{min}$?
  \end{itemize}

  {\bf Hint:} While maximizing or minimizing the magnitude of a complex 
  function with a constant real part, simply find the variable that maximizes
  or minimizes the imaginary part of the function. No need to use the method 
  you learned in calculus (i.e., set the derivative of the function to zero
  and solve for the variable).

  \htmladdimg{../midterm2e.gif}

%  \begin{comment}

  {\bf Solution:}
  \begin{itemize}
  \item Find total impedance of the circuit:
  \begin{eqnarray}
   Z&=&R+j\omega L_2+\frac{j\omega L_1/j\omega C}{j\omega L_1+1/j\omega C}
  	=R+j\omega L_2+\frac{j\omega L_1}{1-\omega^2 CL_1}
  	\nonumber \\
  &=&R+\frac{j\omega(L_2(1-\omega^2CL_1)+L_1)}{1-\omega^2 CL_1}
  =R+j\omega \frac{L_1+L_2-\omega^2CL_1L_2}{1-\omega^2CL_1}
  	\nonumber 
  \end{eqnarray}
  \item At the resonant frequency $\omega_0=\omega_{max}$, the imaginary 
  part of the impedance is zero (minimum), i.e.,
  \[	L_1+L_2=\omega^2CL_1L_2,\;\;\;\;\mbox{i.e.}\;\;\;\;
  	\omega_0=\frac{1}{\sqrt{CL}},\;\;\;\mbox{where}\;\;\;
  	L=\frac{L_1L_2}{L_1+L_2}
  \]
  \item At the stop frequency $\omega_{min}$, the imaginary part of the 
  impedance is infinity (maximum), i.e.,
  \[	1-\omega^2CL_1=0, \;\;\;\mbox{i.e.}\;\;\;\;
  	\omega_{min}=\frac{1}{\sqrt{CL_1}}	\]
  
  \item As $L_1=L_2$, $L=L_1/2$ and we have
  \[ \omega_0=\frac{1}{\sqrt{CL_1/2}}=5000,\;\;\;\;\mbox{i.e.}\;\;\;\;
  	L_1=0.008\;H=8\;mH	\]
  \[ \omega_{min}=\frac{1}{\sqrt{CL_1}}=3536\;rad/sec	\]
  This is true for any load of $R$.
  
  \end{itemize} 

%  \end{comment}

\end{enumerate}
\end{document}

\item {\bf Problem 3. (33 points)} In the circuit shown below, $V_1=2V$,
$V_2=10V$, $R_1=R_2=2\;M\Omega$, $C=1\mu F$. Switch $S_1$ closes at $t=0$, 
switch $S_2$ closes at $t=1$ second. Assume initially the voltage across
the capacitor is $v_c(0)=4V$. Find voltage $v_c(t)$ for the following 
two time periods:
\begin{itemize}
\item $0\le t < 1 s$
\item $t \ge 1\;s$ 
\end{itemize}
Sketch the plots of these two voltages for $t \ge 0$.

{\bf Hints:} For second period, assume $t'=t-1$, and use the solution of
the first period at $t=1$ as the initial value. In the final expression, 
replace $t'$ by $t-1$.

\htmladdimg{../midterm2d.gif}

%   {\bf Solution:}
%   \begin{itemize}
%   \item During the period $0\le t<1 s$, the initial value is 
%  	$v_c(0)=4$, the steady state value is $v_c(\infty)=2V$, the time 
%  	constant is $\tau_1=R_1C=2\times 10^6\times 10^{-6}=2\;sec.$, and 
%  	the overall voltage is:
%  \[ v_c(t)=v_c(\infty)+[v_c(0)-v_c(\infty)] e^{-t/\tau_1}
%  	=2+(4-2) e^{-t/2}=2(1+e^{-0.5t})	\]
%   In particular, when $t=1$, we have
%   \[ v_c(1)=2(1+e^{-0.5})=3.213V	\]
%   
%   \item During the period $t \ge 1\;s$, we have $v_c(1)=3.213V$, 
%   	$v_c(\infty)=2+2\times (10-2)/(2+2)=6V$,
%   	$\tau_2=(R_1||R_2)C=10^6 \times 10^{-6}=1\; sec.$
%  
%   \[	v_c(t')=v_c(\infty)+[v_c(1)-v_c(\infty)]e^{-t'/\tau_2}
%   	=6+(3.213-6)e^{-t'/\tau_2}=6-2.787e^{-(t-1)} \]
%  
% \end{itemize}

\end{enumerate}

\end{document}

\item {\bf Problem 3. (33 points)} 

In the circuit shown in the figure, $V_0=6V$, $I_0=2A$, $R_1=6\Omega$, 
$R_2=3\Omega$, and $L=0.5H$. Assume the circuit has reached steady state.
Find the voltage $v(t)$ across $R_1$ and current $i(t)$ through $L$ as 
time functions after the switch is closed at $t=0$.


{\bf Solution:} 
$i(0)=V_0/R_1=6/6=1A$, $i(\infty)=V_0/R_1+I_0=1+2=3A$. 

Find time constant: $R=R_1R_2/(R_1+R_2)=3\times 6/(3+6)=2\Omega$,
$\tau=L/R=0.5/2=0.25S$. 

The current $i(t)$ is therefore:
\[ i(t)=i(\infty)+[i(0)-i(\infty)]e^{-t/\tau}=3+(1-3)e^{-t/0.25}
	=3-2e^{-4t} \;A \]
\[ v(t)=V_0-V_L(t)=V_0-L\frac{d}{dt}i(t)=6-0.5 \frac{d}{dt}(3-2e^{-4t})
	6-4e^{-4t} \]


\item {\bf Problem 3. (33 points)} 
The circuit in the figure shows a voltage source $V_0$ and $R_0$ and an
amplification circuit modeled by as a two-port network. Assume the two-port 
is represented by a Z-model ($Z_{11}, Z_{12}, Z_{21}, Z_{22}$). Find the 
expression for the load impedance $Z_L$ in terms of $R_0$ as well as the
four Z-parameters, for it to get maximum power from the voltage source.
(Hint: use Thevenin's theorem.)

{\bf Solution:} 
\begin{itemize}
\item First set up all equations:
\[ \left\{ \begin{array}{l} V_1=Z_{11}I_1+Z_{12}I_2 \\
	V_2=Z_{21}I_1+Z_{22}I_2 \end{array} \right.	\]
\[ \left\{ \begin{array}{l} V_1=V_0-R_0I_1 \\
	V_2=-R_L I_2 \end{array} \right.	\]
\item Use Thevenin's theorem

\begin{itemize}
\item Find $Z_{Th}$: assume $V_0=0$, equate equations 1 and 3 to get:
\[ V_1=Z_{11}I_1+Z_{12}I_2, \;\;\;\;\mbox{i.e.,}\;\;\;\;\;
I_1=-\frac{Z_{12}}{Z_{11}+R_0} I_2 \]
Substitute this $I_1$ in equation 2 to get:
\[ V_2=(-\frac{Z_{12}Z_{21}}{Z_{11}+R_0}+Z_{22}) I_2,\;\;\;\;\;
\mbox{i.e.,}\;\;\;\;\;
Z_{Th}=\frac{V_2}{I_2}=-\frac{Z_{12}Z_{21}}{Z_{11}+R_0}+Z_{22}	\]
\item Find $V_{Th}$:, assume $I_2=0$, we have
\[ \left\{\begin{array}{l} V_1=Z_{11}I_1 \\V_2=Z_{21}I_1\end{array}\right. \]
Substitute $V_1=V_0-R_0I_1$ into $V_1=Z_{11}I_1$ to get
\[	V_{Th}=V_2=V_0\frac{Z_{21}}{Z_{11}+R_0}	\]
\end{itemize}

For $R_L$ to get maximum power, we need to have
\[ R_L=Z_{Th}=Z_{22}-\frac{Z_{12}Z_{21}}{Z_{11}+R_0}	\]
and the current is
\[ I_l=\frac{V_{Th}}{Z_{Th}+R_L}=\frac{V_{Th}}{Z_{Th}}
	=\frac{2V_0Z_{21}}{2(Z_{22}Z_{11}-Z_{12}Z_{21}+Z_{22}R_0)} \]


\end{itemize}


\end{enumerate}

\end{document}

An electric motor, modeled as an inductor and a resistor in series, has 
a power factor of 0.85. The nameplate current is 10 Amps at 115 Volts 
(60 Hz). 
\begin{itemize}
\item Find the apparent power, active power, and reactive power. 
\item Find the inductance and resistance of the motor.
\item Find the capacitance of a parallel shunt capacitor that can improve
	the power factor to 1.
\item Find the capacitance if the power factor can be 0.9.
\end{itemize}

% {\bf Solution:} 
% The apparent power is $S=115V \times 10A = 1150 W$, the real power is
% $P=S\cos \phi=S*0.85=977.5 W$, the reactive power is 
% $P=S\sin \phi=S*0.527=605.8 W$. The impedance of the motor is
% \[	Z=\frac{V}{I}=\frac{115}{10}=11.5\Omega \]
% The inductance L and resistance R satisfy the following equations:
% \[ \left\{ \begin{array}{l} (\omega L)^2+R^2=Z^2=11.5^2 \\
% 	tan^{-1} \frac{\omega L}{R}=cos^{-1} 0.85 \end{array} \right. \]
% Given $\omega=2\pi f=377\;rad/sec$, the equations can be solve for R and
% L to get
% \[	R=9.8\Omega,\;\;\;\;L=16\;mH,\;\;\;\;\omega L=6\Omega	\]
% With the parallel capacitor C, the overall impedance is
% \[	Z=(R+j\omega L)\; || \;(1/j\omega C)
% 	=\frac{(R+j\omega L)/j\omega C}{(R+j\omega L)+1/j\omega C}
% 	=\frac{R+j\omega L}{j\omega CR-\omega^2 LC+1}	\]
% For $\angle Z=0$, we need to have
% \[ \tan^{-1}\frac{\omega L}{R}=\tan^{-1}\frac{\omega RC}{1-\omega^2 LC},
% 	\;\;\;\mbox{i.e.}\;\;\;	
% 	\frac{\omega L}{R}=\frac{\omega RC}{1-\omega^2 LC}	\]
% which can be solved for $C$ to get
% \[	C=\frac{L}{R^2+\omega^2 L^2}=120\;\mu F	\]
% For the power factor to be 0.9, or $\cos^{-1} 0.9=\phi=25.8$, we need
% \[ \tan^{-1}\frac{\omega L}{R}-\tan^{-1}\frac{\omega RC}{1-\omega^2 LC}
% 	=25.8,	\;\;\;\mbox{i.e.}\;\;\;	
% \tan^{-1}\frac{\omega RC}{1-\omega^2 LC}=\tan^{-1}\frac{\omega L}{R}-25.8=
% 	5.8 \]
% which can be solved to get $C=25.5\;\mu F$.

\documentstyle[11pt]{article}
\usepackage{html}
\begin{document}
\begin{center}
{\Large \bf E84 Midterm Exam 2}
\end{center}

\begin{itemize}
\item Take home, open everything except discussion.
\item Mark your start and end times. % Don't spend more than 3 hours.
\item Due Monday in class.
\item Mark your name and question number clearly on top of each page.
	Indicate the total number of pages submitted.
\item When solving a problem, list all the steps. In each step, describe 
	what you are doing in English, then show the calculation and the 
	result of the step. A final answer, even if correct, without 
	evidence of the steps leading to the answer will not receive credit.
\end{itemize}

\begin{enumerate}

\item {\bf Problem 1. (33 points)} 

An electric motor, modeled as an inductor and a resistor in series, has 
a power factor of 0.85. The nameplate current is 10 Amps at 115 Volts 
(60 Hz). 
\begin{itemize}
\item Find the apparent power, active power, and reactive power. 
\item Find the inductance and resistance of the motor.
\item Find the capacitance of a parallel shunt capacitor that can improve
	the power factor to 1.
\item Find the capacitance if the power factor can be 0.9.
\end{itemize}

{\bf Solution:} 
The apparent power is $S=115V \times 10A = 1150 W$, the real power is
$P=S\cos \phi=S*0.85=977.5 W$, the reactive power is 
$P=S\sin \phi=S*0.527=605.8 W$. The impedance of the motor is
\[	Z=\frac{V}{I}=\frac{115}{10}=11.5\Omega \]
The inductance L and resistance R satisfy the following equations:
\[ \left\{ \begin{array}{l} (\omega L)^2+R^2=Z^2=11.5^2 \\
	tan^{-1} \frac{\omega L}{R}=cos^{-1} 0.85 \end{array} \right. \]
Given $\omega=2\pi f=377\;rad/sec$, the equations can be solve for R and
L to get
\[	R=9.8\Omega,\;\;\;\;L=16\;mH,\;\;\;\;\omega L=6\Omega	\]
With the parallel capacitor C, the overall impedance is
\[	Z=(R+j\omega L)\; || \;(1/j\omega C)
	=\frac{(R+j\omega L)/j\omega C}{(R+j\omega L)+1/j\omega C}
	=\frac{R+j\omega L}{j\omega CR-\omega^2 LC+1}	\]
For $\angle Z=0$, we need to have
\[ \tan^{-1}\frac{\omega L}{R}=\tan^{-1}\frac{\omega RC}{1-\omega^2 LC},
	\;\;\;\mbox{i.e.}\;\;\;	
	\frac{\omega L}{R}=\frac{\omega RC}{1-\omega^2 LC}	\]
which can be solved for $C$ to get
\[	C=\frac{L}{R^2+\omega^2 L^2}=120\;\mu F	\]
For the power factor to be 0.9, or $\cos^{-1} 0.9=\phi=25.8$, we need
\[ \tan^{-1}\frac{\omega L}{R}-\tan^{-1}\frac{\omega RC}{1-\omega^2 LC}
	=25.8,	\;\;\;\mbox{i.e.}\;\;\;	
\tan^{-1}\frac{\omega RC}{1-\omega^2 LC}=\tan^{-1}\frac{\omega L}{R}-25.8=
	5.8 \]
which can be solved to get $C=25.5\;\mu F$.


\item {\bf Problem 2. (33 points)} Study the frequency behavior of the
  circuit shown in the figure by finding the ratio of the voltage $V_R$
  across load resistor $R$ and the input voltage $V$ (frequency response
  function) of the circuit. 

  \begin{itemize}
  \item Similar to a simple RCL series resonant circuit, this circuit
    can be used as a band-pass filter to pass signals around its resonant 
    frequency. Find this resonant frequency $\omega_{max}$ at which 
    $V_R(\omega)$ is maximized in terms of the component values $R$, 
    $C$, $L_1$ and $L_2$. 

  \item Different from the simple RCL series resonant circuit, this 
    circuit also has a stop band, i.e., at a certain frequency 
    $\omega_{min}$ the output voltage $V_R$ is zero. Find this frequency 
    $V_R$ also in terms of the circuit components. 

  \item Given $C=10\;\mu F$ and the two inductors are identical, determine
    the values of $R$ and $L_1=L_2$, so that $\omega_{max}=5000\;rad/sec$.

  \end{itemize}

  \htmladdimg{../midterm2e.gif}

  {\bf Solution:}
  \begin{itemize}
  \item Find total impedance of the circuit:
    \begin{eqnarray}
      Z&=&R+j\omega L_2+\frac{j\omega L_1/j\omega C}{j\omega L_1+1/j\omega C}
      =R+j\omega L_2+\frac{j\omega L_1}{1-\omega^2 CL_1}
      \nonumber \\
      &=&R+\frac{j\omega(L_2(1-\omega^2CL_1)+L_1)}{1-\omega^2 CL_1}
      =R+j\omega \frac{L_1+L_2-\omega^2CL_1L_2}{1-\omega^2CL_1}
      \nonumber 
    \end{eqnarray}
  \item At the resonant frequency $\omega_0=\omega_{max}$, the imaginary 
    part of the impedance is zero (minimum), i.e.,
    \[	L_1+L_2=\omega^2CL_1L_2,\;\;\;\;\mbox{i.e.}\;\;\;\;
    \omega_0=\frac{1}{\sqrt{CL}},\;\;\;\mbox{where}\;\;\;
    L=\frac{L_1L_2}{L_1+L_2}
    \]
  \item At the stop frequency $\omega_{min}$, the imaginary part of the 
    impedance is infinity (maximum), i.e.,
    \[	1-\omega^2CL_1=0, \;\;\;\mbox{i.e.}\;\;\;\;
    \omega_{min}=\frac{1}{\sqrt{CL_1}}	\]
    
  \item As $L_1=L_2$, $L=L_1/2$, and
    \[ \omega_0=\frac{1}{\sqrt{CL_1/2}}=5000,\;\;\;\;\mbox{i.e.}\;\;\;\;
    L_1=0.008\;H=8\;mH	\]
    \[ \omega_{min}=\frac{1}{\sqrt{CL_1}}=3536\;rad/sec	\]
    This is true for any load of $R$.

  \end{itemize} 

\item {\bf Problem 3. (33 points)} In the circuit shown below, $V_1=2V$,
  $V_2=10V$, $R_1=R_2=2\;M\Omega$, $C=1\mu F$. Switch $S_1$ closes at $t=0$, 
  switch $S_2$ closes at $t=1$ second. Assume initially the voltage across
  the capacitor is $v_c(0)=4V$. Find voltage $v_c(t)$ for the following 
  two time periods:
  \begin{itemize}
  \item $0\le t < 1 s$
  \item $t \ge 1\;s$ 
  \end{itemize}
  Sketch the plots of these two currents for $t \ge 0$.
  
  {\bf Hints:} For second period, assume $t'=t-1$, and use $i(1)$ as 
  the initial value. In the final expression, replace $t'$ by $t-1$.

\htmladdimg{../midterm2d.gif}

 {\bf Solution:}
 \begin{itemize}
 \item During the period $0\le t<1 s$, the initial value is 
	$v_c(0)=4$, the steady state value is $v_c(\infty)=2V$, the time 
	constant is $\tau_1=R_1C=2\times 10^6\times 10^{-6}=2\;sec.$, and 
	the overall voltage is:
\[ v_c(t)=v_c(\infty)+[v_c(0)-v_c(\infty)] e^{-t/\tau_1}
	=2+(4-2) e^{-t/2}=2(1+e^{-0.5t})	\]
 In particular, when $t=1$, we have
 \[ v_c(1)=2(1+e^{-0.5})=3.213V	\]
 
 \item During the period $t \ge 1\;s$, we have $v_c(1)=3.213V$, 
 	$v_c(\infty)=2+2\times (10-2)/(2+2)=6V$,
 	$\tau_2=(R_1||R_2)C=10^6 \times 10^{-6}=1\; sec.$

 \[	v_c(t')=v_c(\infty)+[v_c(1)-v_c(\infty)]e^{-t'/\tau_2}
 	=6+(3.213-6)e^{-t'/\tau_2}=6-2.787e^{-(t-1)} \]

 
 \end{itemize}




\end{enumerate}

\end{document}

\item {\bf Problem 3. (33 points)} 

In the circuit shown in the figure, $V_0=6V$, $I_0=2A$, $R_1=6\Omega$, 
$R_2=3\Omega$, and $L=0.5H$. Assume the circuit has reached steady state.
Find the voltage $v(t)$ across $R_1$ and current $i(t)$ through $L$ as 
time functions after the switch is closed at $t=0$.


{\bf Solution:} 
$i(0)=V_0/R_1=6/6=1A$, $i(\infty)=V_0/R_1+I_0=1+2=3A$. 

Find time constant: $R=R_1R_2/(R_1+R_2)=3\times 6/(3+6)=2\Omega$,
$\tau=L/R=0.5/2=0.25S$. 

The current $i(t)$ is therefore:
\[ i(t)=i(\infty)+[i(0)-i(\infty)]e^{-t/\tau}=3+(1-3)e^{-t/0.25}
	=3-2e^{-4t} \;A \]
\[ v(t)=V_0-V_L(t)=V_0-L\frac{d}{dt}i(t)=6-0.5 \frac{d}{dt}(3-2e^{-4t})
	6-4e^{-4t} \]


\item {\bf Problem 3. (33 points)} 
The circuit in the figure shows a voltage source $V_0$ and $R_0$ and an
amplification circuit modeled by as a two-port network. Assume the two-port 
is represented by a Z-model ($Z_{11}, Z_{12}, Z_{21}, Z_{22}$). Find the 
expression for the load impedance $Z_L$ in terms of $R_0$ as well as the
four Z-parameters, for it to get maximum power from the voltage source.
(Hint: use Thevenin's theorem.)

{\bf Solution:} 
\begin{itemize}
\item First set up all equations:
\[ \left\{ \begin{array}{l} V_1=Z_{11}I_1+Z_{12}I_2 \\
	V_2=Z_{21}I_1+Z_{22}I_2 \end{array} \right.	\]
\[ \left\{ \begin{array}{l} V_1=V_0-R_0I_1 \\
	V_2=-R_L I_2 \end{array} \right.	\]
\item Use Thevenin's theorem

\begin{itemize}
\item Find $Z_{Th}$: assume $V_0=0$, equate equations 1 and 3 to get:
\[ V_1=Z_{11}I_1+Z_{12}I_2, \;\;\;\;\mbox{i.e.,}\;\;\;\;\;
I_1=-\frac{Z_{12}}{Z_{11}+R_0} I_2 \]
Substitute this $I_1$ in equation 2 to get:
\[ V_2=(-\frac{Z_{12}Z_{21}}{Z_{11}+R_0}+Z_{22}) I_2,\;\;\;\;\;
\mbox{i.e.,}\;\;\;\;\;
Z_{Th}=\frac{V_2}{I_2}=-\frac{Z_{12}Z_{21}}{Z_{11}+R_0}+Z_{22}	\]
\item Find $V_{Th}$:, assume $I_2=0$, we have
\[ \left\{\begin{array}{l} V_1=Z_{11}I_1 \\V_2=Z_{21}I_1\end{array}\right. \]
Substitute $V_1=V_0-R_0I_1$ into $V_1=Z_{11}I_1$ to get
\[	V_{Th}=V_2=V_0\frac{Z_{21}}{Z_{11}+R_0}	\]
\end{itemize}

For $R_L$ to get maximum power, we need to have
\[ R_L=Z_{Th}=Z_{22}-\frac{Z_{12}Z_{21}}{Z_{11}+R_0}	\]
and the current is
\[ I_l=\frac{V_{Th}}{Z_{Th}+R_L}=\frac{V_{Th}}{Z_{Th}}
	=\frac{2V_0Z_{21}}{2(Z_{22}Z_{11}-Z_{12}Z_{21}+Z_{22}R_0)} \]


\end{itemize}


\end{enumerate}

\end{document}

