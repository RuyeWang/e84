\documentstyle[11pt]{article}
\usepackage{html}
\begin{document}
\begin{center}
{\Large \bf  E84 Midterm Exam 1 (Fall 2008)}
\end{center}

\section*{E84 Midterm Exam 1}

{\bf Instructions}
\begin{itemize}
\item In class, open notes, feel free to use a calculator, but not any software package 
  such as Multisim. 
\item Mark your name and question number clearly on top of each page.
  Indicate the total number of pages submitted.
\item When solving a problem, list all the steps. In each step, indicate
  concisely what you are doing in English, then show the calculation 
  and the result of for the step. {\bf Box the final answer}.
\item  A final answer, even if correct, without evidence of the steps
  leading to it will receive ZERO credit.
\end{itemize}

\begin{enumerate}

\item {\bf Problem 1. (40 pts)}

  Find the equivalent resistance $R_{ab}$ between points a and b. (The diagonal
  wires are not connected to each other.)

  \htmladdimg{../midterm1_09sb.gif}

  {\bf Solution}

  The circuit can be converted into two parallel branches between a and b
  each composed of two resistors with 20 $\Omega$ and 60 $\Omega$ in series,
  and two parallel branches between the two middle points each composed of 
  two 20 $\Omega$ resistors in series and two 60 $\Omega$ resistors in
  series, respectively. This is a balanced bridge with no current through
  the middle branches, i.e., $R_{ab}=40\Omega$

\item {\bf Problem 2. (60 pts)}

  Find all six currents labeled as $I_1$ through $I_6$ in the figure, where
  $R1=4\Omega$, $R2=4\Omega$, $R3=2\Omega$, and $R4=2\Omega$.

  \htmladdimg{../midterm1_09sa.gif}

  {\bf Solution:}
  $I_2=20A$.
  KVL around outer loop:
  \[ 110+100-4I_1-90=0,\;\;\;\;\Rightarrow I_1=30A \]
  \[ I_3=I_1-I_2=10A \]
  Also  we have
  \[ R_3I_4+R_4I_5=2(I_4+I_5)=110 V,\;\;\;\;\Rightarrow I_4+I_5=55 \]
  and
  \[ I_5-I_4=20 \]
  Solving these we get $I_5=37.5A$ and $I_4=17.5A$.
  Finally $I_6=I_5+I_3=37.5+10=47.5$.

\item {\bf Problem 3. (35 points)}

  Find voltage $V$ and current $I$ as labeled in the following figure:

  \htmladdimg{../midterm1_09sc.gif}

  where $R_1=5\Omega$, $R_2=1\Omega$, $R_3=6\Omega$, $R_4=6\Omega$.

  {\bf Solution:}

  {\bf Method 0, voltage/current source conversion:}
  \begin{itemize}
  \item Convert current source 9A and $R_2$ into a voltage source of $9\times 5=45V$
    in series with $R_1=5$. 
  \item Combine $R_1=5$ in series with $R_2=1$ to get $R'=6$ still in series with 
    $V'=45V$. 
  \item Convert this voltage source back to a current source with $I'=45/6$ in 
    parallel with $R'=6$. 
  \item Convert the other voltage source into a current source with $I''=24/6$ in
    parallel with $R''=6$.
  \item Combine the two parallel current sources to get $I'''=I'+I''=(45+24)/6=23/2$
    in parallel with $R'''=R'||R''=3$.
  \item Use current division to find load current $I=I'''/3=23/6$.
  \item Find load voltage $V=IR_4=23$.
  \end{itemize}

  {\bf Method 1, superposition:}
    \begin{itemize}
    \item Turn voltage source off:
      \[ I'=9\times\frac{5}{5+(1+6||6)}\frac{1}{2}=2.5A \]
      \[ V'=6I'=15 V \]
    \item Turn current source off:
      \[ I''=\frac{24}{6+(6||6)}\frac{1}{2}=4/3\;A \]
      \[ V''=6I''=8V \]
  \end{itemize}
  The total voltage and current are therefore $V=V''+V''=15+8=23V$,
  $I=I'+I''=23/6A$.

  {\bf Method 2, Thevenin's theorem:}
  
  \[ R_T=6||6=3 \]
  $V_T$ due to 24V:
  \[ V'_T=24-24 \frac{6}{12}=12V \]
  $V_T$ due to 9A:
  \[ V''_T=9\frac{5}{5+7}\times 6=22.5 \]
  Thererfore
  \[ V_T=V'_T+V''_T=34.5 \]
  Reconnecting load $R_4=6$, we get
  \[ V=V_T \frac{6}{6+3}=34.5\times 2/3=23 V \]
  and
  \[ I=V/R_4=23/6 \]

\end{itemize}

\end{document}




