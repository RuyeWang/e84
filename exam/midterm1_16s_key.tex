\documentstyle[11pt]{article}
\usepackage{html}

\begin{document}
\begin{center}
{\Large \bf  E84 Midterm Exam 1 --- Spring 2016}
\end{center}

{\bf Instructions}
\begin{itemize}
\item Write your name on top of each page. Indicate the total number of 
  pages submitted.
\item When solving a problem, list the main steps. In each step describe 
  concisely what you are doing, then show the calculation and the 
  result of the step. Box all final answers. 
\item A final answer, even if correct, without evidence of the steps 
  leading to the answer will receive no credit. 
\end{itemize}

\newpage

{\bf The Problems}
\begin{enumerate}


\item {\bf Problem 1. (40 points)}

  Find voltage $V$ and current $I$ as labeled in the following figure:

  \htmladdimg{../midterm1_09sc.gif}

  where $R_1=5\Omega$, $R_2=1\Omega$, $R_3=6\Omega$, $R_4=6\Omega$.

  {\bf Solution:}

  {\bf Method 0, voltage/current source conversion:}
  \begin{itemize}
  \item Convert current source 9A and $R_2$ into a voltage source of $9\times 5=45V$
    in series with $R_1=5$. 
  \item Combine $R_1=5$ in series with $R_2=1$ to get $R'=6$ still in series with 
    $V'=45V$. 
  \item Convert this voltage source back to a current source with $I'=45/6$ in 
    parallel with $R'=6$. 
  \item Convert the other voltage source into a current source with $I''=24/6$ in
    parallel with $R''=6$.
  \item Combine the two parallel current sources to get $I'''=I'+I''=(45+24)/6=23/2$
    in parallel with $R'''=R'||R''=3$.
  \item Use current division to find load current $I=I'''/3=23/6=3.8333$.
  \item Find load voltage $V=IR_4=23$.
  \end{itemize}

  {\bf Method 1, superposition:}
    \begin{itemize}
    \item Turn voltage source off:
      \[ I'=9\times\frac{5}{5+(1+6||6)}\frac{1}{2}=2.5A \]
      \[ V'=6I'=15 V \]
    \item Turn current source off:
      \[ I''=\frac{24}{6+(6||6)}\frac{1}{2}=4/3\;A \]
      \[ V''=6I''=8V \]
  \end{itemize}
  The total voltage and current are therefore $V=V''+V''=15+8=23V$,
  $I=I'+I''=23/6A$.

  {\bf Method 2, Thevenin's theorem:}
  
  \[ R_T=6||6=3 \]
  $V_T$ due to 24V:
  \[ V'_T=24-24 \frac{6}{12}=12V \]
  $V_T$ due to 9A:
  \[ V''_T=9\frac{5}{5+7}\times 6=22.5 \]
  Thererfore
  \[ V_T=V'_T+V''_T=34.5 \]
  Reconnecting load $R_4=6$, we get
  \[ V=V_T \frac{6}{6+3}=34.5\times 2/3=23 V \]
  and
  \[ I=V/R_4=23/6 \]

\item {\bf Problem 2. (60 points)} 
  
  In the circuit below, $R_5=1\,\Omega$, all other resistors are
  $2\,\Omega$. $V_1=4\,V$, $V_2=8\,V$, $I_1=2\,A$, $I_2=8\,A$.

  (a) Find the voltage $V$ across, current $I$ through, and poswer
  $P$ consumed by $R_7=2\,\Omega$.
  

  (b) If you can change the resistance of $R_7$ for it to consume
  maximum power, what value should it be? What is this power?

  \htmladdimg{../../lectures/figures/Problem10.png}

  {\bf Solution:} 
  (0) User Thevenin's theorem. Disconnect $R_7$ as the load.

  (1) Convert $I_1=2$ and $R_1=2$ into a voltage source with $V'_1=4\,V$
  in series with $V_1=4\,V$, we get a voltage of $8\,V$ voltage in series 
  with $R_1=2$. 
  
  (2) Convert $V_2=8$ and $R_4=2$ into a current source $I'_2=4\,A$, in 
  parallel with $R_4||R_6=2||2=1,\Omega$. Then turn it back to a voltage 
  source $V'_2=4\,V$ in series with $1,\Omega$. 

  (3) Convert $I_2=8$ and $R_5=1$ into a voltage source of $8\,V$ in series
  with $R_5=1\,\Omega$.

  (4) Combine the two voltage sources in (2) and (3) to get a voltage of 
  $4\,V$ in series with a resistor $R_5+R_4||R_6=2\,\Omega$.

  (5) Combine $8\,V$ and $2\,\Omega$ in (1) with $4\,V$ and $1\,\Omega$ 
  in (4) to get a voltage of $4\,V$ and $1\,\Omega$ to get a voltage of 
  $4\,V$ in series with a resistance of $4\,\Omega$. Find current $I=1\,A$
  in the loop.

  (6) $R_T=1\,\Omega$, $V_T=4+1=6\,V$. The current through $R_7=2\,\Omega$
  is $V_T/(R_T+R_7)=6/3=2\,A$, the voltage across it is $4\,V$, the power
  consumption is $8\,W$

  (7) For $R_7$ as the load to get maximum power, it needs to be 
  $R_7=R_T=1\,\Omega$. The voltage across it is $3\,V$, the current 
  through is $3\,A$, the power consumption is $9\,W$.


\end{enumerate}
\end{document}
