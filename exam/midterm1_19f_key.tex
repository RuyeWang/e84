%\documentstyle[11pt]{article}
\documentclass{article}
\usepackage{amsmath}
\usepackage{amssymb}
\usepackage{graphics}
\usepackage{comment}
\usepackage{html,makeidx}


\begin{document}
\begin{center}
{\Large \bf  Midterm Exam 1 ---- E84, Fall 2019}
\end{center}


\begin{comment}
\section*{Please fill out the feedback form}

{\bf Note: } To help me improve my teaching of the course in the future, I
would like to collect your feedback and comments. Please fill the following 
form {\bf anonymously} and turn it in {\bf separately} from your midterm 
exam. Your help is greatly appreciated!

\begin{itemize}
\item {\bf Feedback:} (circle one of the numbers)
\begin{itemize}

\item {\bf Lecture pace:} (1 too slow, 3 just right, 5 too fast)

\framebox[0.5in]{1}\framebox[0.5in]{2}\framebox[0.5in]{3}\framebox[0.5in]{4}\framebox[0.5in]{5}

\item {\bf Lecture clearity:} (1 Too sketchy, 3 just right, 5 too tedious)

\framebox[0.5in]{1}\framebox[0.5in]{2}\framebox[0.5in]{3}\framebox[0.5in]{4}\framebox[0.5in]{5}

\item {\bf Overall homework workload:} (1 too little, 3 just right, 5 too much)

\framebox[0.5in]{1}\framebox[0.5in]{2}\framebox[0.5in]{3}\framebox[0.5in]{4}\framebox[0.5in]{5}

\item {\bf Overall difficulty of course materials:} (1 too easy, 3 just right, 5 too hard)

\framebox[0.5in]{1}\framebox[0.5in]{2}\framebox[0.5in]{3}\framebox[0.5in]{4}\framebox[0.5in]{5}
\end{itemize} 

\item {\bf Comments:}
\begin{itemize}

\item {\bf What went well for your learning in the course? Be specific.}

\begin{tabular}{l}
.  \\
.  \\
.  \\
\end{tabular}
\vskip 5cm

\item {\bf What did not go well for your learning in the course? Be specific.}
\begin{tabular}{l}
.  \\
.  \\
.  \\
\end{tabular}
\vskip 0.9in

\item {\bf Indicate one thing you want to improve and how.}
\begin{tabular}{l}
.  \\
.  \\
.  \\
\end{tabular}
\vskip 0.9in

\end{itemize}

\end{itemize}
\end{comment}


\section*{Instructions}

{\bf Instructions}
\begin{itemize}
\item Take home, open everything except discussion. Due Thursday in class.
\item Mark your start and end times. Spend no more than 3 hours.
  (Don't click the problems until you are ready to start.)
\item If you want to have a hard copy of the exam, compare your print-out
  with the online version to make sure your copy is complete.
\item Mark your name clearly on top of each page, and indicate the total
  number of pages submitted. For example, 2/5 for the second of the total 
  five pages.
\item Show all steps towards getting the final answers. Box your final 
  answers. A final answer without evidence of the steps leading to it 
  will receive no credit.
\item You may consider using different methods to solve the same probem 
  (e.g., Problems 2 and 3) and compare results to make sure they are 
  consistent and correct.
\end{itemize}

\section*{The Problems (Don't click until you are ready to start)}
\begin{enumerate}

\item {\bf Problem 1. (20 points)} 
Show the relationship between the output voltage $V_o$ and the three input
voltages $V_1$, $V_2$ and $V_3$ of the circuit shown below, where all 
resistors have the same resistance value $R$. Extrapolate your result to 
cover the general case of $n$ inputs $V_i$, $i=1,\cdots,n$.

\htmladdimg{../midterm1d.gif}

%\begin{comment}
{\bf Solution:}

\[ \frac{V_1-V_0}{R}+\frac{V_2-V_0}{R}+\frac{V_3-V_0}{R}=\frac{V_0}{R}
,\;\;\;\;\; V_0=\frac{V_1+V_2+V_3}{4}	\]
In general
\[ V_0=\frac{1}{n+1}\sum_{i=1}^n V_i \]
%\end{comment}


\item {\bf Problem 2. (40 points)} 

In the figure below, $V_1=20V$, $V_2=V_3=10V$, $R_1=R_5=10\Omega$, 
$R_2=R_4=5\Omega$, $R_6=1.5\Omega$, $R_3=6\Omega$. Find voltage $V_{ab}$

\htmladdimg{../midterm1e.gif}

%\begin{comment}
{\bf Solution:} Use Thevenin's theorem. 
\begin{itemize}
\item Remove $R_3$ and $V_3$ as the load
\item Find open-circuit voltage $V_T$:
\[	V_{ab}=V_a-V_b=V_1\frac{R_5}{R_1+R_5}-V_2\frac{R_4}{R_2+R_4}
	=20 \frac{10}{10+10}-10 \frac{5}{5+5}=10-5=5 \]
\item Find $R_T$:
\[	R_T=R_1//R_5+R_6+R_2//R_4
	=\frac{R_1R_5}{R_1+R_5}+R_6+\frac{R_4R_2}{R_2+R_4}=9 \]
\item Connect load of $V_3$ and $R_3$, find current
\[	I=\frac{V_T-V_3}{R_T+R_3}=\frac{5-10}{9+6}=-\frac{1}{3} \]
\item Find $V_{ab}$
\[	V_{ab}=-\frac{1}{3}\times 6 +10=8V	\]
\end{itemize}
%\end{comment}

\item {\bf Problem 3. (40 points)} 

In the circuit below, $I_0=6A$, $V_0=5V$, $R_2=R_4=4\Omega$, $R_1=2\Omega$,
$R_3=8\Omega$. Find the current through $R_5=6\Omega$.

\htmladdimg{../midterm1f.gif}

%\begin{comment}
{\bf Solution:} Use superposition. First consider the current source $I_0$ 
only with $V_0=0$ (short-circuit). Convert the delta composed of the top three 
resistors ($R_1$, $R_2$ and $R_5$) to Y:
\[	R_a=\frac{2\times 4}{2+6+4}=\frac{2}{3},\;\;\;\;
	R_b=\frac{2\times 6}{2+6+4}=1,\;\;\;\;R_c=\frac{4\times 6}{2+6+4}=2 \]
where $R_a$ is in series with $I_0$, and $R_b$ and $R_c$ are in series with 
$R_4$ and $R_3$, respectively. Treating the two branches as two voltage dividers,
we find the voltage across $R_5$ is zero and therefore $I'=0$,

Next consider the voltage source $V_0$ only, with $I_0=0$ (open-circuit).
The total resistance of the loop is $15\Omega$, and the total current is 
$I_{total}=5V/15\Omega=1/3\;A$, and the current through $R_5$ can be  found
by current divider to be $I''=1/6\;A$. The current due to both $I_0$ and $V_0$
is therefore $I=I'+I''=1/6\;A$ 
%\end{comment}

\end{enumerate}
\end{document}





\item {\bf Problem 4. (25 points)}

\htmladdimg{../midterm1h.png}

In the circuit shown, $R=20 \Omega$, $C=1\,\mu F$, and $L=0.8\;mH$. 
The current source is $i_0(t)=\sqrt{2}\,\cos(50,000 t)$ Ampere. 
Find the following:
\begin{itemize}
\item current $i_L(t)$ through $L$ 
\item current $i_{RC}(t)$ through $R$ and $C$ in series
\item voltage $v_R(t)$ across $R$
\item voltage $v_C(t)$ across $C$
\item voltage $v_L(t)$ across $L$
\end{itemize}

\begin{comment}
{\bf Solution:} 
\[
\dot{I}=1
\]
\[
Z_R=20;\;\;\;Z_C=\frac{1}{j\omega C}=\frac{1}{j5\times 10^{-2}}=-j20;\;\;\;
Z_L=j\omega L=j\,40=40 e^{j\pi/2}
\]
\[
Z_{RC}=Z_R+Z_C=20-j20=20\sqrt{2}e^{j\pi/4}
\]
\[ 
Z_{total}=\frac{Z_{RC} Z_L}{Z_{RC}+Z_L}
=\frac{20\sqrt{2}e^{j\pi/4} 40 e^{j\pi/2}}{20-j20+j40}
=\frac{800\sqrt{2}e^{j\pi/4}}{20\sqrt{2}e^{j\pi/4}}=40
\]
\[
\dot{V}=\dot{I}Z_{total}=40
\]
\[ 
\dot{I}_{RC}=\frac{\dot{V}}{Z_{RC}}=\frac{40}{20\sqrt{2} e^{-j\pi/4}}=\sqrt{2}e^{j\pi/4}
=1+j,\;\;\;\;i_{RC}(t)=2\cos(\omega t+\pi/4)
\]
\[
\dot{I}_{L}=\frac{\dot{V}}{Z_L}=\frac{40}{40 e^{j\pi/2}}=e^{-j\pi/2}=-j,\;\;\;\;
i_L(t)=\sqrt{2}\cos(\omega t-\pi/2)
\]
\[
\dot{V}_R=\dot{I}_{RC} R=20(1+j)=20\sqrt{2}e^{j\pi/4},\;\;\;\;\;v_R(t)=40\cos(\omega t+\pi/4)
\]
\[
\dot{V}_C=\dot{I}_{RC} Z_C=-j 20(1+j)=20(1-j)=20\sqrt{2}e^{-j\pi/4},\;\;\;\;\;
v_C(t)=40\cos(\omega t-\pi/4)
\]
\[ 
\dot{V}_R+\dot{V}_C=40=\dot{V}
\]
\end{comment}

\end{enumerate}

\end{document}

