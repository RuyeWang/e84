\documentstyle[11pt]{article}
\usepackage{html}
\begin{document}
\begin{center}
{\Large \bf  Final Exam (Fall, 2015)}
\end{center}

\section{The Instructions}

\begin{itemize}
\item Take home, open book/notes/web
\item Mark your start and end times. Don't spend more than 3 hours.
  Do not click ``The exam problems'' until you are ready to take the 
  final.
\item Mark your name and question number clearly on top of each page.
  Indicate the total number of pages submitted.
\item When solving a problem, list all the steps. Concisely describe 
  in English what you are doing in each step, then show the calculation 
  and the result of the step. Box final your answers.
\item A final answer, even if correct, without evidence of the steps 
  leading to the answer will not receive credit.
\end{itemize}



\section{The Exam Problems (don't click until you take the final)}

\begin{enumerate}


\item {\bf Problem 1 (25 points)} 

  Find the output $V_{out}$ in terms of the four inputs $V_1$,
  $V_2$, $V_3$, and $V_4$. Assume $R_5=R_{f_1}$.

  \htmladdimg{../../lectures/figures/opamp2.png}

  \begin{comment}

  {\bf Solution:} Let $V$ be the output of the first op-amp.
  \[
  \frac{V_1}{R_1}+\frac{V_2}{R_2}+\frac{V}{R_{f_1}}=0,
  \;\;\;\;\;\;\mbox{i.e.,}\;\;\;\;
  V=-\frac{R_{f_1}}{R_1}V_1-\frac{R_{f_1}}{R_2}V_2
  \]
  \[
  \frac{V_{out}}{R_{f_2}}+\frac{V_3}{R_3}+\frac{V_4}{R_4}+\frac{V}{R_5}
  =\frac{V_{out}}{R_{f_2}}+\frac{V_3}{R_3}+\frac{V_4}{R_4}
  -\frac{1}{R_5}\left(\frac{R_{f_1}}{R_1}V_1-\frac{R_{f_1}}{R_2}V_2\right)=0
  \]
  Solving for $V_{out}$:
  \[
  V_{out}=\frac{R_{f_2}}{R_5}\frac{R_{f_1}}{R_1}V_1+\frac{R_{f_2}}{R_5}\frac{R_{f_1}}{R_2}V_2-\frac{R_{f_2}}{R_3}V_3-\frac{R_{f_2}}{R_4}V_4
  =\frac{R_{f_2}}{R_1}V_1+\frac{R_{f_2}}{R_2}V_2-\frac{R_{f_2}}{R_3}V_3-\frac{R_{f_2}}{R_4}V_4
  \]
  \end{comment}

\item {\bf Problem 2 (25 points)} 

  An op-amp circuit is given below:

  \htmladdimg{../../lectures/figures/opampMF.png}

  \begin{itemize}
  \item What is the order of the circuit? What kind of filter is it
    (high-pass, low-pass, band-pass, or band-stop)? 
  \item Find the frequency response function $H(j\omega)=V_{out}/V_{in}$ 
    of circuit in canonical form, in terms of the cut-off or corner 
    frequency $\omega_c$ if the circuit is first order, or natural 
    frequency $\omega_n$ and the quality factor $Q$ if it is second
    order.
  \item Given the voltage gain $|H(j\omega)|$ when (a) $\omega=0$, 
    and (b) $\omega\rightarrow\infty$.
  \end{itemize}

  \begin{comment}
    {\bf Solution:}

    Apply KCL to the point between $C_1$ and $C_2$ denoted by $V$:
    \[
    \frac{V_{in}-V}{1/j\omega C_1}+\frac{V_{out}-V}{1/j\omega C_2}
    =\frac{V}{R_2}+\frac{V}{1/j\omega C_1}
    \]
    i.e.,
    \[
    V_{in}j\omega C_1+V_{out}j\omega C_2
    =\left(\frac{1}{R_2}+2j\omega C_1+j\omega C_2\right) V
    =\left(\frac{1}{R_2}+j\omega(2C_1+C_2)\right) V
    \]
    Apply KCL to the inverting input of the op-amp which is virtually
    grounded:
    \[
    \frac{V}{1/j\omega C_1}+\frac{V_{out}}{R_1}=0
    \;\;\;\;\;\;\mbox{i.e.}\;\;\;\;\;\;
    V=-\frac{1}{j\omega R_1C_1}V_{out}
    \]
    Substitute into the first equation
    \begin{eqnarray}
      V_{in}j\omega C_1+V_{out}j\omega C_2
      &=&-\left(\frac{1}{R_2}+j\omega(2C_1+C_2)\right) \frac{1}{j\omega R_1C_1}V_{out}
      \nonumber\\
      &=&-\left(\frac{1}{j\omega R_1R_2C_1}+\frac{2C_1+C_2}{R_1C_1}\right) V_{out}
      \nonumber
    \end{eqnarray}
    Rearrange:
    \[
    V_{in}
    =-\frac{1}{j\omega C_1}\left(j\omega C_2+\frac{1}{j\omega R_1R_2C_1}+\frac{2C_1+C_2}{R_1C_1}\right) V_{out}
    =\left(\frac{C_2}{C_1}+\frac{1}{(j\omega)^2 R_1R_2C^2_1}
    +\frac{2C_1+C_2}{j\omega R_1C_1^2}\right) V_{out}
    \]
    The FRF can be found to be:
    \begin{eqnarray}
      H(j\omega)&=&\frac{V_{out}}{V_{in}}
      =-\left(\frac{C_2}{C_1}+\frac{1}{(j\omega)^2 R_1R_2C_1^2}
      +\frac{2C_1+C_2}{j\omega R_1C_1^2}\right)^{-1}
      =-\frac{(j\omega)^2R_1R_2C_1^2}{(j\omega)^2R_1R_2C_1C_2+j\omega(2C_1+C_2)R_2+1}
      \nonumber\\
      &=&-\frac{C_1}{C_2}\;\frac{(j\omega)^2}{(j\omega)^2+j\omega(2C_1+C_2)/R_1C_1C_2+1/R_1R_2C_1C_2}
      =-\frac{C_1}{C_2}\;\frac{(j\omega)^2}{(j\omega)^2+j\omega\omega_n/Q+\omega_n^2}
      \nonumber
    \end{eqnarray}
    where
    \[
    \omega_n=\frac{1}{\sqrt{R_1R_2C_1C_2}},\;\;\;\;\;\;
    \frac{2C_1+C_2}{R_1C_1C_2}=\frac{\omega_n}{Q}=\Delta\omega,
    \;\;\;\mbox{i.e.,}\;\;\;\;
    Q=\omega_n\frac{R_1C_1C_2}{2C_1+C_2}=\frac{\sqrt{R_1R_2C_1C_2}}{(2C_1+C_2)R_2}
    \]
    This is a 2nd-order high-pass filter, when $\omega=\infty$, we have
    \[
    H(j\omega)\big|_{\omega=\infty}=H(j\infty)=-\frac{C_1}{C_2}
    \]
    \end{comment}

\item {\bf Problem 3 (25 points)}

  The output $v_{out}(t)$ of the transistor circuit with a sinusoidal input
  is plotted as shown below. As you can see, $v_{out}(t)$ is distorted in 
  either of the two cases of (a) and (b). As the designer of the circuit, 
  you can change $R_B$, $R_C$ and/or $V_{cc}$ to avoid the distortion. 

  \htmladdimg{../final09s1.gif}

  \begin{itemize}
  \item What would you do to avoid distortion in (a) and why? 
  \item What would you do to avoid distortion in (b) and why? 
  \end{itemize}
  Sketch the input and output characteristic plots of the transistor
  circuit to visualize and explane how each of the two types of distortion 
  may be caused.

  \htmladdimg{../final09s2.gif}




  \begin{comment}
  {\bf Solution:}

  \begin{itemize}
  \item Reduce $R_B$, and/or increase $V_{cc}$.
  \item Increase $V_{cc}$ and/or reduce $R_C$ and/or increase $R_B$.
  \end{itemize}


\item {\bf Problem 3 (25 points)} 

The circuit below shows a simple means for obtaining improved bias
stability of the DC operating point of the transistor. As always,
assume $V_{BE}=0.7V$ when answering the following questions.

\htmladdimg{../midterm3e.gif}

\begin{itemize}
\item Explain qualitatively what happens if $I_C$ tends to rise as a
  result of an increased $\beta$.
\item Derive an expression for $I_C$ in terms of $R_B$, $R_C$, $\beta$
  and $V_{CC}$.
\item Find an approximation of the expression of $I_C$ when $\beta$ is 
  large enough, so that $I_C$ independent of $\beta$. In this case, how 
  are $R_C$ and $R_B$ related?
\item Find $R_C$ and $R_B$ so that the DC operating point is $V_{CE}=5V$ 
and $I_C=2mA$, when $\beta=100$ and $V_{CC}=10V$. 
\item Find $V_{CE}$ and $I_C$ for $\beta=50$, $\beta=100$, and $\beta=200$
	based on the resistances found above.
\end{itemize}

\item {\bf Problem 3. (30 points)} 

\htmladdimg{../midterm3a.gif}

The circuit shown below is a silicon transistor amplifier which takes one
input and generates two outputs. Assume $V_{CC}=20V$, $R_1=20K\Omega$,
$R_2=10K\Omega$, $R_C=R_E=500\Omega$, $\beta=100$. 

\begin{itemize}
\item Find $V_B$, $I_B$, $V_E$ and $V_C$, and the DC operating point in 
terms of $I_C$ and $V_{CE}$. 
\item In the figure provided, draw the load line, indicate the DC operating 
point, and find the corresponding $I_C$ and $V_{CE}$.
\item If the input voltage is such that it produces an AC component of the 
base current:
\[	i_b(t)=0.1 cos(\omega t) \; mA	\]
give the expression of the AC component of the two output voltages $v_1(t)$ 
at the emitter and $v_2(t)$ at the collector, and sketch their waveforms in 
the SAME plot provided below, where $v_{in}(t)=cos(\omega t)$ is also ploted.
(No need to be to the scale vertically, but do pay attention to the time
scale.)
\end{itemize}

\htmladdimg{../midterm3b.gif}
\htmladdimg{../midterm3c.gif}

% {\bf Solution:} 
% Apply Thevenin thm to base circuit to get $R_B=R_1 || R_2=6.7K$, $V_BB=6.7V$.
% \[ I_B=(V_{BB}-V_{BE})/(R_B+(\beta+1)R_E)=6V/(6.7K+101\times 0.5K)=0.106 mA \]
% \[ I_C=I_E=10.6 mA \]
% \[ V_E=0.5\times 10.6=5.3V,\;\;\;\; V_C=20-0.5\times 10.6=14.7V,\;\;\;\;
% 	V_{CE}=V_C-V_E=14.7-5.3=9.4V \]
% \[ v_1(t)=-5 cos(\omega t) V,\;\;\;\;v_1(t)=5 cos(\omega t) V	\]

\end{comment}

\item {\bf Problem 4 (25 points)} 
  \begin{itemize}
  \item Represent voltage $V_L$ across the load $R_L$ in terms of all
    parameters given in the circuit shown on the left of the figure below.
  \item Represent the voltage $V'_L$ across the load $R_L$ in terms of all
    parameters given in the circuit shown on the right of the figure below,
    in which a voltage-amplifier is inserted between the source and the load, 
    characterized by three parameters: (a) the input resistance $R_{in}$, (b)
    the output resistance $R_{out}$, and (c) the over-circuit voltage gain 
    $A_{oc}$. 
  \item Assuming $V_o=10\,V$, $R_o=6\,k\Omega$, $R_L=4\,k\Omega$, 
    $R_{in}=100\,k\Omega$, $R_{out}=1\,k\Omega$, and $A_{oc}=10$, find the 
    numerical value for $V_L$ and $V'_L$ in the two cases above.
  \end{itemize}
\htmladdimg{../SourceLoad.png}

\begin{comment}
{\bf Solution:}
\begin{itemize}
\item \[ V_L=V_o \frac{R_L}{R_o+R_L} \]
\item \[ V'_L=A_{oc}V_o \frac{R_{in}}{R_o+R_{in}}\frac{R_L}{R_{out}+R_L} \]
\item \[ V_L=V_o \frac{R_L}{R_o+R_L}=10\frac{4}{6+4}=4  \]
  \[ V'_L=A_{oc}V_o \frac{R_{in}}{R_o+R_{in}}\frac{R_L}{R_{out}+R_L} 
  =10\times 10\frac{100}{6+100}\frac{4}{1+4}=75.47 \]
\end{itemize}
\end{comment}

\end{enumerate}

\end{document}

