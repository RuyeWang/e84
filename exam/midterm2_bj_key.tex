\documentstyle[11pt]{article}
\usepackage{html}
\begin{document}
\begin{center}
{\Large \bf E84 Midterm Exam 2}
\end{center}


\begin{enumerate}

\item {\bf Problem 1. (35 points)} 

\htmladdimg{../midterm3a.gif}

The circuit shown below is a silicon transistor amplifier which takes one
input and generates two outputs. Assume $V_{CC}=20V$, $R_1=20K\Omega$,
$R_2=10K\Omega$, $R_C=R_E=500\Omega$, $\beta=100$. 

\begin{itemize}
\item Find $V_B$, $I_B$, $V_E$ and $V_C$, and the DC operating point in 
terms of $I_C$ and $V_{CE}$. 
\item In the figure provided, draw the load line, indicate the DC operating 
point, and find the corresponding $I_C$ and $V_{CE}$.
\item If the input voltage is such that it produces an AC component of the 
base current:
\[	i_b(t)=0.1 cos(\omega t) \; mA	\]
give the expression of the AC component of the two output voltages $v_1(t)$ 
at the emitter and $v_2(t)$ at the collector, and sketch their waveforms in 
the SAME plot provided below, where $v_{in}(t)=cos(\omega t)$ is also plotted.
(No need to be to the scale vertically, but do pay attention to the time
scale.)
\end{itemize}

\htmladdimg{../midterm3b.gif}
\htmladdimg{../midterm3c.gif}

 {\bf Solution:} 
 Apply Thevenin theorem to base circuit to get $R_B=R_1 || R_2=6.7K$, $V_{BB}=6.7V$.
 \[ I_B=(V_{BB}-V_{BE})/(R_B+(\beta+1)R_E)=6V/(6.7K+101\times 0.5K)=0.106 mA \]
 \[ I_C=I_E=10.6 mA \]
 \[ V_E=0.5\times 10.6=5.3V,\;\;\;\; V_C=20-0.5\times 10.6=14.7V,\;\;\;\;
 	V_{CE}=V_C-V_E=14.7-5.3=9.4V \]
 \[ v_1(t)=-5 cos(\omega t) V,\;\;\;\;v_1(t)=5 cos(\omega t) V	\]


\item {\bf Problem 2. (35 points)} 

Find the DC operating point $(I_C, V_{CE})$ of the transistor circuit 
given below, where $R_1=100K\Omega$, $R_2=300K\Omega$, $R_C=R_E=2K\Omega$, 
$V_{CC}=12V$, and $\beta=100$. If you find the DC operating point is not 
in the middle of the linear region of the output characteristic plot, 
modify $R_2$ so that the DC operating point is in the middle of the linear
region (to maximize the dynamic range of the AC output).

\htmladdimg{../../lectures/figures/transistorbiasingb.gif}

 {\bf Solution:} 
 
 Find the load line: 
 When $I_C=0$, $V_{CE}=V_{CC}=12V$, when $V_{CE}=0$, $I_C=V_{CC}/(R_C+R_E)=3mA$
 
 \[ R_{B}=\frac{R_1 R_2}{R_1+R_2}=75K,\;\;\;V_{BB}=12\frac{R_2}{R_1+R_2}=9V \]
 \[ I_B=\frac{V_{BB}-V_{BE}}{R_B+(\beta+1)R_E}=\frac{9-0.7}{75+202}=30\mu A \]
 \[ I_C=\beta I_B=3 mA, \;\;\;\;V_{CE}=V_{CC}-(R_C+R_E)I_C=0V \]
 The DC operating point is in saturation region.
 
 Now we modify $R_2$ to move Q-point to the middel point where $I_C=1.5mA$.
 \[ V_{BB}=12R_2/(100+R_2),\;\;\;\;R_B=100R_2/(100+R_2) \]
 We then plug these into 
 \[ I_C=\beta \frac{V_{BB}-V_{BE}}{(\beta+1) R_E+R_B}=1.5 \]
 and solve it for $R_2$ to get $R_2=55K$.


\item {\bf Problem 3. (30 points)} 
The output $v_{out}(t)$ of the transistor circuit with a sinusoidal input
is plotted. As you can see, $v_{out}(t)$ is distorted in either of the two
cases of (a) and (b). As the designer of the circuit, you can change $R_B$,
$R_C$ and/or $V_{cc}$ to avoid the distortion. 

\htmladdimg{../final09s1.gif}

\begin{itemize}
\item What would you do to avoid distortion in (a) and why? 
\item What would you do to avoid distortion in (b) and why? 
\end{itemize}

Keep your answer accurate, concise, and to the point.

\htmladdimg{../final09s2.gif}

{\bf Hint:} Draw the input and output characteristic plots of the transistor
circuit to visualize how each of the two types of distortion can be avoided.

{\bf Solution:}

\begin{itemize}
\item Reduce $R_B$.
\item Increase $V_{cc}$ and/or reduce $R_C$.
\end{itemize}


\end{itemize}
\end{document}

\item {\bf Problem 3. (20 points)} 
  Answer the following questions regarding the circuits shown in the figure
  below, where $V_o=1\,V$, $R_o=6\,k\Omega$, $R_L=4\,k\Omega$, 
  $R_{in}=1\,M\Omega$, $R_{out}=0.1\,k\Omega$, and $A_{oc}=10$.
  \begin{itemize}
  \item Represent the voltage $V_L$ across the load $R_L$ in terms of all 
    of the parameters given in the circuit shown on the left of the figure. 
    Then obtain the numerical value of $V_L$ by substituting the specific 
    values of the parameters into the expression.
  \item Repeat the above for the circuit shown on the right of the figure,
    in which a voltage-amplification circuit, a buffer, is inserted between 
    the source and the load, characterized by three parameters: (a) the
    input resistance $R_{in}$, (b) the output resistance $R_{out}$, and 
    (c) the open-circuit voltage gain $A_{oc}$. 
  \end{itemize}
\htmladdimg{../SourceLoad.png}

% {\bf Solution:}
% \begin{itemize}
% \item \[ V_L=V_o \frac{R_L}{R_o+R_L} \]
% \item \[ V'_L=A_{oc}V_o \frac{R_{in}}{R_o+R_{in}}\frac{R_L}{R_{out}+R_L} \]
% \item \[ V_L=V_o \frac{R_L}{R_o+R_L}=\frac{4}{6+4}=0.4  \]
%  \[ V'_L=A_{oc}V_o \frac{R_{in}}{R_o+R_{in}}\frac{R_L}{R_{out}+R_L} 
%  =10\frac{1000}{6+1000}\frac{4}{0.1+4}=9.7 \]




\item {\bf Problem 1. (33 points)} 

  An electric motor, modeled as an inductor and a resistor in series, has 
  a power factor of 0.85. The nameplate current is 10 Amps at 115 Volts 
  (60 Hz). 
  \begin{itemize}
  \item Find the apparent power, active power, and reactive power. 
  \item Find the inductance and resistance of the motor.
  \item Find the capacitance of a parallel shunt capacitor that can improve
    the power factor to 1.
  \item Find the capacitance if the power factor can be 0.9.
  \end{itemize}

%  {\bf Solution:} 
%  The apprarent power is $S=115V \times 10A = 1150 W$, the real power is
%  $P=S\cos \phi=S*0.85=977.5 W$, the reactive power is 
%  $P=S\sin \phi=S*0.527=605.8 W$. The impedance of the motor is
%  \[	Z=\frac{V}{I}=\frac{115}{10}=11.5\Omega \]
%  The inductance L and resistance R satisfy the following equations:
%  \[ \left\{ \begin{array}{l} (\omega L)^2+R^2=Z^2=11.5^2 \\
%    tan^{-1} \frac{\omega L}{R}=cos^{-1} 0.85 \end{array} \right. \]
%  Given $\omega=2\pi f=377\;rad/sec$, the quations can be solve for R and
%  L to get
%  \[	R=9.8\Omega,\;\;\;\;L=16\;mH,\;\;\;\;\omega L=6\Omega	\]
%  With the parallel capacitor C, the overall impedance is
%  \[	Z=(R+j\omega L)\; || \;(1/j\omega C)
%  =\frac{(R+j\omega L)/j\omega C}{(R+j\omega L)+1/j\omega C}
%  =\frac{R+j\omega L}{j\omega CR-\omega^2 LC+1}	\]
%  For $\angle Z=0$, we need to have
%  \[ \tan^{-1}\frac{\omega L}{R}=\tan^{-1}\frac{\omega RC}{1-\omega^2 LC},
%  \;\;\;\mbox{i.e.}\;\;\;	
%  \frac{\omega L}{R}=\frac{\omega RC}{1-\omega^2 LC}	\]
%  which can be solved for $C$ to get
%  \[	C=\frac{L}{R^2+\omega^2 L^2}=120\;\mu F	\]
%  For the power factor to be 0.9, or $\cos^{-1} 0.9=\phi=25.8$, we need
%  \[ \tan^{-1}\frac{\omega L}{R}-\tan^{-1}\frac{\omega RC}{1-\omega^2 LC}
%  =25.8,	\;\;\;\mbox{i.e.}\;\;\;	
%  \tan^{-1}\frac{\omega RC}{1-\omega^2 LC}=\tan^{-1}\frac{\omega L}{R}-25.8=
%  5.8 \]
% which can be solved to get $C=25.5\;\mu F$.


\item {\bf Problem 3. (34 points)} 



The circuit shown below is called Darlington transistor amplifier which is
composed of two transistors $T_1$ and $T_2$ with their collectors connected 
and the emitter of $T_1$ connected to the base of $T_2$. Assume $V_{CC}=20V$ 
and both transistors have $\beta=50$. 
\begin{enumerate}
\item Give the expressions of $I_{C1}$, $I_{E1}=I_{B2}$, $I_{C2}$, $I_{E2}$
  all in terms of $\beta$ and $I_{B1}$.
\item Given $R_C=1 k\Omega$, find $R_B$ so that the DC operating point $Q$
  is in the middle of the linear region of the output V-I characteristic plot.
\item Given $R_B= 4.6 M\Omega$ and the same $R_C$ as before, find the proper 
DC voltage $V_{cc}$ so that the Q point is in the middle of the linear region.
\end{enumerate}
For simplicity, assume for both transistors $v_{be}=0.7V$.

\htmladdimg{../midterm3g.gif}

%{\bf Solution:}

%\begin{enumerate}
%\item $I_{C1}=\beta I_{B1}$, $I_{E1}=I_{B2}=(\beta+1) I_{B1}$, 
%  $I_{C2}=\beta I_{B2}=\beta (\beta+1) I_{B1}$, 
%  $I_{E2}=(\beta+1) I_{B2}=(\beta+1)^2 I_{B1}=(\beta^2+2\beta+1)I_{B1}$, 

%\item In order for $V_C=V_{CC}-R_C (I_{C1}+I_{C2})=10V$, we need to have
%  $I_{C1}+I_{C2}=\beta I_{B1}+\beta (\beta+1) I_{B1}=2500 I_{B1}=10 mA$, i.e.,
%  $I_{B1}=10 mA/2600=3.8 \mu A$. 

%  $R_B=(V_{CC}-2V_{be})/I_{B1}=18.6/3.8=4.89 M\Omega$
%\item 
%  \[ \beta I_{B1}+\beta (\beta+1) I_{B1}
%  =(2\beta+\beta^2)\;\frac{V_{cc}-1.4}{4.6\times 10^6}
%  =2600\;\frac{V_{cc}-1.4}{4.6\times 10^6}=\frac{1}{2}\frac{V_{cc}}{10^3} \]
%  Solving this we get
%  \[ V_{cc}=12\;V \]
% \end{enumerate}
\end{enumerate}

\end{document}

