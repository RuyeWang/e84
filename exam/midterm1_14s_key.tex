\documentstyle[11pt]{article}
\usepackage{html}
\begin{document}
\begin{center}
{\Large \bf  Midterm Exam 1 ---- E84, Spring 2014}
\end{center}

\section*{E84 Midterm Exam 1}

{\bf Instructions}
\begin{itemize}
\item Take home, open everything except discussion. Due Monday in class.
\item There are in total four problems, 25 points each.
\item Mark your start and end times. Don't spend more than 3 hours.
\item Compare your print-out of the exam with the online version to make
	sure your hard copy is complete.
\item Mark your name and question number clearly on top of each page.
	Indicate the total number of pages submitted.
\item When solving a problem, list all the steps. In each step, describe 
	concisely what you are doing in English, then show the calculation 
	and the result of the step. A final answer, even if correct, without 
	evidence of the steps leading to the answer will not receive credit.
\end{itemize}

{\bf The Problems}
\begin{enumerate}

\item {\bf Problem 1. (25 points)} 
Show the relationship between the output voltage $V_o$ and the three input
voltages $V_1$, $V_2$ and $V_3$ of the circuit shown below, where all 
resistors have the same resistance value $R$. Extrapolate your result to 
cover the general case of $n$ inputs $V_i$, $i=1,\cdots,n$.

\htmladdimg{../midterm1d.gif}

{\bf Solution:}

\[ \frac{V_1-V_0}{R}+\frac{V_2-V_0}{R}+\frac{V_3-V_0}{R}=\frac{V_0}{R}
,\;\;\;\;\; V_0=\frac{V_1+V_2+V_3}{4}	\]
In general
\[ V_0=\frac{1}{n+1}\sum_{i=1}^n V_i \]

\item {\bf Problem 2. (25 points)} 

In the figure below, $V_1=20V$, $V_2=V_3=10V$, $R_1=R_5=10\Omega$, 
$R_2=R_4=5\Omega$, $R_6=1.5\Omega$, $R_3=6\Omega$. Find voltage $V_{ab}$

\htmladdimg{../midterm1e.gif}

{\bf Solution:} 

Method I based on Thevenin's theorem:
\begin{itemize}
\item Remove $R_3$ and $V_3$ as the load
\item Find open-circuit voltage $V_T$:
\[	V_{ab}=V_a-V_b=V_1\frac{R_5}{R_1+R_5}-V_2\frac{R_4}{R_2+R_4}
	=20 \frac{10}{10+10}-10 \frac{5}{5+5}=10-5=5 \]
\item Find $R_T$:
\[	R_T=R_1//R_5+R_6+R_2//R_4
	=\frac{R_1R_5}{R_1+R_5}+R_6+\frac{R_4R_2}{R_2+R_4}=9 \]
\item Connect load of $V_3$ and $R_3$, find current
\[	I=\frac{V_T-V_3}{R_T+R_3}=\frac{5-10}{9+6}=-\frac{1}{3} \]
\item Find $V_{ab}$
\[	V_{ab}=-\frac{1}{3}\times 6 +10=8V	\]
\end{itemize}

Method II based on voltage source conversion:
\begin{itemize}
\item Convert $V_1$ and $R_1$ treated as a voltage source to a current
  source $I_1=V_1/R_1=2$ and $R_1$, which is in parallel with $R_5$.
\item Find parallel combination $R_a=R_1 || R_5=5$.
\item Convert current source $I_1=2$ and $R_a=5$ to a voltage source
  of $V_a=10$ and $R_a=5$.
\item Convert $V_2$ and $R_2$ treated as a voltage source to a current 
  source $I_2=V_2/R_2=2$ and $R_2$, which is in parallel with $R_4$.
\item Find parallel combination $R_b=R_2 || R_4=2.5$.
\item Convert current source $I_2=2$ and $R_b=2.5$ to a voltage source
  of $V_a=5$ and $R_b=2.5$.
\item Find voltage across $R_3$ (voltage divider) to be 
  $(V_a+V_b+V_3) R_3/(R_3+R_a+R_6+R_b)=5\times 6/(6+5+1.5+2.5)=2$
  (+ on right).
\item Find $V_{ab}=10-2=8$.
\end{itemize}

Method III based on loop current method ($I_1$, $I_2$, and $I_3$ for the
three loops):

\begin{itemize}
\item Left loop:
  \[ R_1I_a+R_5(I_a-I_b)-V_1=0 \]
\item Middle loop:
  \[ R_3I_b+V_3+R_4(I_b-I_c)+R_6-R_5(I_b-I_a)=0 \]
\item Right loop:
  \[ R_2I_c+V_2+R_4(I_c-I_b)=0 \]
\item Solving the three equations to get
  \[ I_a=5/6,\;\;\;I_b=-1/3,\;\;\;\;I_c=-7/6 \]
\item Find $V_{ab}=V_3+R_3 I_3=10-6/3=8$
\end{itemize}


\item {\bf Problem 3. (25 points)} 

In the circuit below, $I_0=6A$, $V_0=5V$, $R_2=R_4=4\Omega$, $R_1=2\Omega$,
$R_3=8\Omega$. Find the current through $R_5=6\Omega$.

\htmladdimg{../midterm1f.gif}

{\bf Solution:} 

Method I based on superposition. 

\begin{itemize}
\item Consider the current source $I_0$ only with $V_0=0$ (short-circuit). 
  Convert the delta composed of the top three resistors ($R_1$, $R_2$ and 
  $R_5$) to Y:
  \[
  R_a=\frac{2\times 4}{2+6+4}=\frac{2}{3},\;\;\;\;
  R_b=\frac{2\times 6}{2+6+4}=1,\;\;\;\;R_c=\frac{4\times 6}{2+6+4}=2 
  \]
  where $R_a$ is in series with $I_0$, and $R_b$ and $R_c$ are in series 
  with $R_4$ and $R_3$, respectively. Treating the two branches as two 
  voltage dividers, we find the voltage across $R_5$ is zero and therefore 
  $I'=0$.
\item Consider the voltage source $V_0$ only, with $I_0=0$ (open-circuit).
  The total resistance of the loop is $15\Omega$, and the total current is 
  $I_{total}=5V/15\Omega=1/3\;A$, and the current through $R_5$ can be  found
  by current divider to be $I''=1/6\;A$. 
\item The current due to both $I_0$ and $V_0$ is therefore $I=I'+I''=1/6\;A$ 
\end{itemize}

Method II based on Thevenin's theorem. 
\begin{itemize}
\item Pull $R_5$ out, and treat the rest as a Thevenin voltage source
  with $V_T$ and $R_T$.
\itme Find $R_T=(R_1+R_2)||(R_3+R_4)=6||12=4$
\item Find $V_T$ by superposition:
  \begin{itemize}
  \item Due to $I_0=6$ alone ($V_0=0$): 
    \[ 
    I_{14}=I_0\frac{R_2+R_3}{(R_1+R_4)+(R_2+R_3)}=4,\;\;\;\;
    I_{23}=I_0\frac{R_1+R_4}{(R_1+R_4)+(R_2+R_3)}=8
    \]
    $V'_T=R_1I_{14}-R_2I_{23}=2\times 4-4\times 2=0$.
  \item Due to $V_0=5$ alone ($I_0=0$): 
    \[
    V''_T=V_0\frac{R_1+R_2}{(R_1+R_2)+(R_3+R_4)}=5\times \frac{6}{18}=\frac{5}{3}
    \]
  \item $V_T=V'_T+T''_T=5/3$
  \end{itemize}
\item Find current through $R_5$:
  \[
  I=\frac{V_T}{R_T+R_5}=\frac{5}{3}\times \frac{1}{4+6}=\frac{1}{6}
  \]
\end{itemize}

Method III based on loop current method:

\begin{itemize}
\item Top loop current $I_a$:
  \[   2(I_a-I_0)+4I_a+6(I_a-I_b)=0   \]
\item Bottom loop current $I_b$:
  \[   4(I_b-6)+6(I_b-I_a)-5+8I_b=0 \]
\item Solve these two equations to get $I_a$, $I_b$ and current
  through $R_5$:
  \[ I_a=\frac{13}{6},\;\;\;\;\;\;I_b=\frac{7}{3},\;\;\;\;\;
  I_b-I_a=\frac{14}{6}-\frac{13}{6}=\frac{1}{6} \]
\end{itemize}



\item {\bf Problem 4. (25 points)}

\htmladdimg{../midterm1h.png}

In the circuit shown, $R=20 \Omega$, $C=1\,\mu F$, and $L=0.8\;mH$. 
The current source is $i_0(t)=\sqrt{2}\,\cos(50,000 t)$ Ampere. 
Find the following:
\begin{itemize}
\item current $i_L(t)$ through $L$ 
\item current $i_{RC}(t)$ through $R$ and $C$ in series
\item voltage $v_R(t)$ across $R$
\item voltage $v_C(t)$ across $C$
\item voltage $v_L(t)$ across $L$
\end{itemize}

{\bf Solution:} 
\[
\dot{I}=1
\]
\[
Z_R=20;\;\;\;Z_C=\frac{1}{j\omega C}=\frac{1}{j5\times 10^{-2}}=-j20;\;\;\;
Z_L=j\omega L=j\,40=40 e^{j\pi/2}
\]
\[
Z_{RC}=Z_R+Z_C=20-j20=20\sqrt{2}e^{j\pi/4}
\]
\[ 
Z_{total}=\frac{Z_{RC} Z_L}{Z_{RC}+Z_L}
=\frac{20\sqrt{2}e^{j\pi/4} 40 e^{j\pi/2}}{20-j20+j40}
=\frac{800\sqrt{2}e^{j\pi/4}}{20\sqrt{2}e^{j\pi/4}}=40
\]
\[
\dot{V}=\dot{I}Z_{total}=40
\]
\[ 
\dot{I}_{RC}=\frac{\dot{V}}{Z_{RC}}=\frac{40}{20\sqrt{2} e^{-j\pi/4}}=\sqrt{2}e^{j\pi/4}
=1+j,\;\;\;\;i_{RC}(t)=2\cos(\omega t+\pi/4)
\]
\[
\dot{I}_{L}=\frac{\dot{V}}{Z_L}=\frac{40}{40 e^{j\pi/2}}=e^{-j\pi/2}=-j,\;\;\;\;
i_L(t)=\sqrt{2}\cos(\omega t-\pi/2)
\]
\[
\dot{V}_R=\dot{I}_{RC} R=20(1+j)=20\sqrt{2}e^{j\pi/4},\;\;\;\;\;v_R(t)=40\cos(\omega t+\pi/4)
\]
\[
\dot{V}_C=\dot{I}_{RC} Z_C=-j 20(1+j)=20(1-j)=20\sqrt{2}e^{-j\pi/4},\;\;\;\;\;
v_C(t)=40\cos(\omega t-\pi/4)
\]
\[ 
\dot{V}_R+\dot{V}_C=40=\dot{V}
\]

\end{enumerate}

\end{document}

