\documentstyle[11pt]{article}
\usepackage{html}
\begin{document}
\begin{center}
{\Large \bf  Midterm Exam 3 ---- E84, Spring, 2008}
\end{center}

\begin{itemize}
\item Due 5pm Tuesday 5/13.
\item Take home, open everything except discussion.
\item Mark your start and end times. Don't spend more than 3 hours.
\item Mark your name and question number clearly on top of each page.
	Indicate the total number of pages submitted.
\item When solving a problem, list all the steps. In each step, describe 
	what you are doing in English, then show the calculation and the 
	result of the step. A final answer, even if correct, without 
	evidence of the steps leading to the answer will not receive credit.
\end{itemize}

\begin{enumerate}

\item {\bf Problem 1. (33 points)} 

The circuit shown below contains an AC voltage source of 210V with frequency
50 Hz, an amp meter A, and an unknown component $Z$. In order to find the 
complex impedance of $Z$, a capacitor $C=75.8\mu F$ is added as shown. The 
amp meter reading is 3A before the switch is closed and it is 4A after the
switch is closed. Find the complex value of $Z$.

(Hint: (a) Assume an ideal amp meter with zero impedance, (b) it only reads 
the rms value of the current without reading the phase, (c) use phasor method,
(d) all values are steady state values, no need to consider transient values.)

\htmladdimg{../midterm3j.gif}

{\bf Solution:}
Let $\dot{V}=210\angle 0$. Find current through C to be
\[ \dot{I}_C=\dot{V} j\omega C=j314\times 75.8\times 10^{-6}\times 210\angle 0
=j5\;A=5\angle 90\;A \]
But
\[ \dot{I}=\dot{I}_Z+\dot{I}_C \]
with $I=4A$, $I_Z=3A$, and $I_C=5A$, i.e., these three currents form a 
right triangle with $\beta=\tan^{-1} 3/4=36.87^\circ$, i.e.,
\[ \dot{I}_Z=3\angle -36.87 \]
\[ \dot{Z}=\dot{V}/\dot{I}_Z=210\angle 0/3\angle -36.87 =70\angle 36.87
=56+j\;42\]

\htmladdimg{../midterm3k.gif}


%\item {\bf Problem 1. (20 points)} 
%
%Find the output voltage $V$ for each of the two circuits shown in the 
%figure below, using the line at the bottom as the reference (ground).
%
%\htmladdimg{../midterm3d.gif}
%
%{\bf Solution:} (a) 4.3V (b) 0.7V

\item {\bf Problem 2. (34 points)} 

The circuit below shows a simple means for obtaining improved bias
stability of the DC operating point of the transistor. As always,
assume $V_{BE}=0.7V$ when answering the following questions.

\htmladdimg{../midterm3e.gif}

\begin{enumerate}
\item Explain qualitatively what happens if $I_C$ tends to rise as a
result of an increased $\beta$.
\item Derive an expression for $I_C$ in terms of $R_B$ and $R_C$ and
$\beta$.
\item Find an approximation of the expression of $I_C$ if $\beta$ is 
large enough. To make $I_C$ independent of $\beta$, how $R_C$ and
$R_B$ should be related?
\item Find $R_C$ and $R_B$ so that the DC operating point is $V_{CE}=5V$ 
and $I_C=2mA$, when $\beta=100$ and $V_{CC}=10V$.
\item Find $V_{CE}$ and $I_C$ for $\beta=50$, $\beta=100$, and $\beta=200$
	based on the resistances found above.
\end{enumerate}

{\bf Solution:} 

\begin{itemize}
\item $\beta \uparrow \Longrightarrow I_C \uparrow \Longrightarrow V_C \downarrow 
  \Longrightarrow I_B \downarrow \Longrightarrow I_C \downarrow $
\item
\[ I_B=\frac{V_{CC}-0.7}{(\beta+1)R_C+R_B},\;\;\;\;\;
   I_C=\beta I_B=\frac{\beta(V_{CC}-0.7)}{(\beta+1)R_C+R_B}  \]
\item If $(\beta+1)R_C \gg R_B$, then $I_C \approx (V_{CC}-0.7)/R_C$, 
independent of $\beta$.
\item $I_C=2mA$, $V_C=5V$, $R_C=(V_{CC}-V_C)/I_C=5V/2mA=2.5K\Omega$,
$I_B=I_C/\beta=0.02mA$, $R_B=(5-0.7)/0.02=4.3/0.02=215K\Omega$.
\item 
When $\beta=50$:
\[	I_B=\frac{V_{CC}-0.7}{(\beta+1)R_C+R_B}=\frac{9.3}{51\times 2.5K+215}
	=0.027mA \]
\[ 	I_C=\beta I_B=1.36mA,\;\;\;V_C=V_{CC}-(\beta+1)I_B=6.6V \]
When $\beta=100$:
\[	I_B=\frac{V_{CC}-0.7}{(\beta+1)R_C+R_B}=\frac{9.3}{101\times 2.5K+215}
	=0.02mA \]
\[ 	I_C=\beta I_B=2 mA,\;\;\;V_C=V_{CC}-(\beta+1)I_B=5V \]
When $\beta=200$:
\[	I_B=\frac{V_{CC}-0.7}{(\beta+1)R_C+R_B}=\frac{9.3}{201\times 2.5K+215}
	=0.013mA \]
\[ 	I_C=\beta I_B=2.6mA,\;\;\;V_C=V_{CC}-(\beta+1)I_B=3.5V \]
\end{itemize}

\item {\bf Problem 3. (33 points)} 

\htmladdimg{../midterm3a.gif}

The circuit shown below is a silicon transistor amplifier which takes one
input $v_{in}(t)$ and generates two outputs $v_1(t)$ and $v_2(t)$. Assume 
$V_{CC}=20V$, $R_1=20K\Omega$, $R_2=10K\Omega$, $R_C=R_E=500\Omega$, $\beta=100$. 

\begin{enumerate}
\item Find $V_B$, $I_B$, $V_E$ and $V_C$, and the DC operating point in 
terms of $I_C$ and $V_{CE}$. 
\item In the figure provided, draw the load line, indicate the DC operating 
point, and find the corresponding $I_C$ and $V_{CE}$.
\item If the input voltage is such that it produces an AC component of the 
base current:
\[	i_b(t)=0.1 cos(\omega t) \; mA	\]
give the expression of the AC component of the two output voltages $v_1(t)$ 
at the emitter and $v_2(t)$ at the collector, and sketch their waveforms in 
the SAME plot provided below, where $v_{in}(t)=cos(\omega t)$ is also ploted.
(No need to be to the scale vertically, but do pay attention to the time
scale.)
\end{enumerate}

\htmladdimg{../midterm3b.gif}
\htmladdimg{../midterm3c.gif}

{\bf Solution:} 
Apply Thevenin thm to base circuit to get $R_B=R_1 || R_2=6.7K$, $V_{BB}=6.7V$.
\[ I_B=\frac{V_{BB}-V_{BE}}{R_B+(\beta+1)R_E}=\frac{6V}{6.7K+101\times 0.5K}
	=0.105 mA \]
\[ I_C=I_E=10.5 mA \]
\[ V_E=0.5\times 10.5=5.25V,\;\;\;\; V_C=20-0.5\times 10.6=14.75V,\;\;\;\;
	V_{CE}=V_C-V_E=14.75-5.25=9.5V \]
\[ v_1(t)=-5 cos(\omega t) V,\;\;\;\;v_1(t)=5 cos(\omega t) V	\]

\end{enumerate}

\end{document}

