\documentstyle[11pt]{article}
\usepackage{html}
\begin{document}
\begin{center}
{\Large \bf  Midterm Exam 2 ---- E84, Spring, 2011}
\end{center}

Consider the transistor circuit shown in the figure below with 
$V_{cc}=20V$, $R_B=129k\Omage$, $R_C=2k\Omega$. Assume $\beta=100$.
Answer the following questions and box each of your answers.
(Note, when a transistor is in the saturation region, its $V_{CE}$
can be assumed to be 0.2 V.)

\htmladdimg{../MidtermTwo11s.gif}

\begin{enumerate}
\item Sketch the output characteristic plot with four specific $I_B$
  values: 0, 0.05, 0.10, and 0.15 mA. Clearly label the horizontal
  axis for $V_{CE}$ with increment of 5 V, and the vertical axis for
  $I_C$ with increment of 2.5 mA.
  Find the DC operating point of the transistor in terms of $I_B$, 
  $I_C$ and $V_C$. 

  {\bf Solution:}  
  \[ I_B=(V_{cc}-V_{BE})/R_B=(20-0.7)/129=0.15 mA \]
  \[ I_C=\beta I_B=100\times 0.15=15 mA \]
  \[ V_C=V_{cc}-R_CI_C=20-2\times 15=-10 mA \]
  The transistor is saturated with $V_{CE}=0.2$ V, and $I_C\approx 10\;mA$

\item It is desirable for the DC operating point of the transistor to be 
  in the middle of the linear range of the output characteristic plot so
  that $V_C=V_{cc}/2=10$ V. Find the corresponding current $I_C$, and modify 
  $R_B$ so that the DC operating to be in the middle of the linear range.

  {\bf Solution:} $I_C=5$ mA.
  \[ I_B=(20-0.7)/R_B=I_C/\beta=0.05,\;\;\;\;\;\mbox{i.e.}\;\;\;\;\;
  R_B=(20-0.7)/0.05=386 \]

\item If the $V_{cc}$ is reduced from 20 V to 16 V and $R_B=300k\Omega$.
  Find $R_C$ for the DC operating point to be in the middle of the 
  linear range of the output characteristic plot. 

  {\bf Solution:}
  \[ I_B=(16-0.7)/300=0.051\;mA,\;\;\;\;\;\;I_C=100 I_C=5.1\;mA \]
  For this current to be at the middle, we need to have:
  \[ R_C=V_{cc}/10.2=16/10.2=1.569\;k\Omega \]  

\item If the $V_{cc}$ is reduced from 20 V to 12 V and $R_C=2\;k\Omega$.
  Modify $R_B$ so that the DC operating point is still in the middle of 
  the linear range.

  {\bf Solution:} When $V_{CE}=0$, $I_C=V_{CC}/R_C=12/2=6$ mA. At the middle
  point, $V_{CE}=6$ V and $I_C=3$ mA, and 
  \[ I_B=(12-0.7)/R_B=I_C/\beta=0.03,\;\;\;\;\;\mbox{i.e.}\;\;\;\;\;
  R_B=(12-0.7)/0.03=377 \]

\end{enumerate}

\end{document}
