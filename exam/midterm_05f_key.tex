\documentstyle[11pt]{article}
\usepackage{html}
\begin{document}
\begin{center}
{\Large \bf  Midterm Exam 1 ---- E84, Fall, 2005}
\end{center}

\section*{Please fill out the feedback form}

{\bf Note: } To help me improve my teaching of the course in the future, I would 
like to collect your feedback and comments. Please fill the following form 
{\bf anonymously} and turn it in {\bf separately} from your midterm exam. Your 
help is greatly appreciated!

\begin{itemize}
\item {\bf Feedback:} (circle one of the numbers)
\begin{itemize}

\item {\bf Lecture pace:} (1 too slow, 3 just right, 5 too fast)

\framebox[0.5in]{1}\framebox[0.5in]{2}\framebox[0.5in]{3}\framebox[0.5in]{4}\framebox[0.5in]{5}

\item {\bf Lecture clearity:} (1 Too tedious, 3 just right, 5 too sketchy)

\framebox[0.5in]{1}\framebox[0.5in]{2}\framebox[0.5in]{3}\framebox[0.5in]{4}\framebox[0.5in]{5}

\item {\bf Overall homework workload:} (1 too little, 3 just right, 5 too much)

\framebox[0.5in]{1}\framebox[0.5in]{2}\framebox[0.5in]{3}\framebox[0.5in]{4}\framebox[0.5in]{5}

\item {\bf Overall difficulty of course materials:} (1 too easy, 3 just right, 5 too hard)

\framebox[0.5in]{1}\framebox[0.5in]{2}\framebox[0.5in]{3}\framebox[0.5in]{4}\framebox[0.5in]{5}
\end{itemize} 

\item {\bf Comments:}
\begin{itemize}

\item {\bf What went well for your learning in the course? Be specific.}

\begin{tabular}{l}
.  \\
.  \\
.  \\
\end{tabular}
\vskip 5cm

\item {\bf What did not go well for your learning in the course? Be specific.}
\begin{tabular}{l}
.  \\
.  \\
.  \\
\end{tabular}
\vskip 0.9in

\item {\bf Indicate one thing you want to improve and how.}
\begin{tabular}{l}
.  \\
.  \\
.  \\
\end{tabular}
\vskip 0.9in

\end{itemize}

\end{itemize}


\section*{E84 Midterm Exam 1}

{\bf Instructions}
\begin{itemize}
\item Take home, open everything except discussion. Due Wednesday in class.
\item Mark your start and end times. Don't spend more than 3 hours.
\item Compare your print-out of the exam with the online version to make
	sure your hard copy is complete.
\item Mark your name and question number clearly on top of each page.
	Indicate the total number of pages submitted.
\item When solving a problem, list all the steps. In each step, describe 
	concisely what you are doing in English, then show the calculation 
	and the result of the step. A final answer, even if correct, without 
	evidence of the steps leading to the answer will not receive credit.
\end{itemize}

{\bf The Problems}
\begin{enumerate}

\item {\bf Problem 0. (20 points)} 
A network composed of 12 resistors each of $3\Omega$ is shown in the figure.
Find the resistance between its two terminals A and B. 

{\bf Hint:} If the voltage between two nodes in a network is zero, then the 
two nodes can be connected without changing the voltages and currents in the 
network. (As there is no current between the two nodes.)

\htmladdimg{../cube.gif}

{\bf Solution:} Due to the symmetry of the network, nodes a, b, and c are
at the same voltage and there they can be connected as a single node, the
same is true for nodes d, e, and f. The network between A and B is therefore
composed of two subnetworks each containing three resistors in parallel, 
and the third subnetwork containing six resistors in parallel, the three
subnetworks are then connected in series. Therefore the total resistance is
\[  (3//3//3) + (3//3//3//3//3//3)+ (3//3//3)=1+0.5+1=2.5 \]

\item {\bf Problem 1. (20 points)}

Treat the resistor on the right in the circuit below as the load of a
one-port network. Find the Thevenin model of this network represented
by $R_T$ and $V_T$ in terms of the components given in the circuit. 
Then find the Norton model $R_N$ and $I_N$ of the network. HInt: pay 
attention to the polarities labeled in the figure.

\htmladdimg{../midterm05f3.gif}

{\bf Solution:}
\begin{itemize}
\item Thevenin model: $R_T=R_1||R_2$, 
\[ V_T=I\frac{R_1 R_2}{R_1+R_2}-V\frac{R_2}{R_1+R_2}=(IR_1-V)\frac{R_2}{R_1+R_2} \]
\item Norton model: $R_N=R_T=R_1||R_2$, 
\[ I_N=I-\frac{V}{R_1} \]
\end{itemize}

\item {\bf Problem 2. (25 points)} 
Find the current $i(t)$ through the resistor of $5\Omega$. Assume the 
voltage and current sources, respectively, 
\[ v_0(t)=10\sqrt{2} \cos \; 2t\; V,\;\;\;\;\; 
   i_0(t)=2\sqrt{2} \cos(3t+45^\circ)\; A \]
\htmladdimg{../midterm05f1.gif}

{\bf Hint:} The two energy sources are at different frequencies. But
you can still use superposition theorem to find $i(t)$.

{\bf Solution:} Represent $v_0(t)$ and $i_0(t)$ as phasors:
\[ \dot{V}_0=10\angle 0^\circ V,\;\;\;\;\; \dot{I}_0=2 \angle 45^\circ A \]
Use superposition:
\begin{itemize}
\item Due to voltage source along with current source open,
\[ \dot{I}'=\frac{\dot{V}_0}{j2/3+j2||(-j0.5)+5}
           =\frac{10\angle 0^\circ}{j2/3-j2/3+5}=2\angle 0^\circ \; A \]
\[ i'(t)=2\sqrt{2} \cos 2t \; A \]
\item Due to current source along with voltage source shorted, 
\[ \dot{I}''=\dot{I}_0 \frac{j1+j3||(-j1/3)}{5+j1+j3||(-j1/3)}
   =2\angle 45^\circ \; \frac{j5/8}{5+j5/8}=0.25\angle 127.87^\circ \;A
\]
\[ i''(t)=0.25\sqrt{2} \cos(3t+127.87^\circ) \; A \]
\end{itemize}
The current $i(t)$ is therefore
\[ i(t)=i'(t)+i''(t)=2\sqrt{2} \cos 2t+0.25\sqrt{2} \cos(3t+127.87^\circ) \; A \]

\item {\bf Problem 4. (35 points)}

At $t=0$, S switches from a to b. Find the voltage $v(t)$ across the $8\Omega$ 
resistor for $t \ge 0$ as the complete response to this event. Assume the 
circuit is already in steady state when $t<0$.

\htmladdimg{../midterm05f2.gif}

{\bf Hints:}
\begin{itemize}
\item The voltage across a capacitor can not change instantaneously. 
  Similarly, the current through an inductor can not change instantaneously.
\item After S switches from a to b, the circuit has two current sources, the
  current through $L$ and the current source of 1.25A. Use superposition 
  theorem to find $v(t)$ with two components, one caused by the initial 
  current through $L$, the other caused by the current source.
\end{itemize}

{\bf Solution:} 
\begin{itemize}
\item Find voltage component $v'(t)$ of $v(t)$ caused by initial current in $L$:
  \begin{itemize}
  \item Find time constant: $\tau=L/R_{eq}=2/(2+10+8)=0.1$ second.
  \item Find current through $L$ at $t=0$: $I_L=20/10=2A$ (upward).
  \item Find $v'(0)=8\Omega \times 2A=16V$, $v(\infty)=0$.
  \item Find voltage component of $v(t)$ caused by this current:
  \[ v'(t)=v(\infty)+[v(0)-v(\infty)]e^{-t/\tau}=
  0+(16-0) e^{-t/\tau}=16 e^{-t/0.1}=16 e^{-10t} \]
\end{itemize}

\item Find voltage component $v''(t)$ of $v(t)$ caused by current source:
\begin{itemize}
\item Find initial value of the voltage across resistor $V(0)=1.25A\times 8 
  \Omega=-10V$. 
\item Find final value of the voltage across resistor. At steady state, the 
  inductor is short circuit:
  \[ v(\infty)=-1.25\times \frac{10+2}{10+2+8}\times 8=-6V \]
\item Find voltage component of $v(t)$ caused by current source:
  \[ v''(t)=v''(\infty)+[v''(0)-v''(\infty)] e^{-t/\tau}
           =-6+(-10+6) e^{-10t}=-6-4 e^{-10t} \]
\end{itemize}
\end{itemize}
The overall voltage $v(t)$ can be found to be
\[ v(t)=v'(t)+v''(t)=16 e^{-10t}+(-6-4 e^{-10t})=12 e^{-10t}-6 \]

\end{enumerate}

\end{document}

\item {\bf Problem 1. (33 points)} 
A DC generator of $V_0=78V$ is connected through two wires each of 
$0.1\Omega$ to two different loads, an electrical oven of $4\Omega$ 
and a re-chargeable battery with voltage (leftover) $60V$ and internal 
resistance $1\Omega$. The schematic of this circuit is shown in the 
figure below. Find
\begin{itemize}
	\item power consumption of the electrical oven
	\item power loss caused by the internal resistance of the battery
\end{itemize}

\htmladdimg{../midterm1c.gif}

{\bf Solution:}

 Use superposition principle to find I0 through the wires, I1 through
 the oven, and I2 through the battery:

 (1) generator alone: 4 || 1=4/5=0.8, total resistance is 0.1+0.1+0.8=1
 I0'=78/1=78, I1'=78 x 1/5=78/5, I2'=78 x 4/5=312/5

 (2) battery alone: 4 || 0.2=4/21, total resistance is 1+4/21=25/21
 I2''=60/(25/21)=60x21/25=252/5, I1''=(252/5) x 0.2/4.2=12/5,
 I0''=(252/5) x 4/4.2=240/5=48

 (3) total currents:
 I0=I0'-I0''=78-48=30, I1=I1'+I1''=78/5+12/5=90/5=18, 
 I2=I2'-I2''=312/5-252/5=60/5=12 (check: I1+I2=18+12=30=I0)
 power on oven: 18x18x4=1296, power on internal resistance: 12x12=144

\item {\bf Problem 2. (34 points)} 
Find the value of the load resistance $R_L$ for it to get maximum power 
from the voltage source.

\htmladdimg{../midterm1a.gif}

{\bf Solution:}

 Use Thevenin's theorem (assume all $R=1$, $V_0=1$)
 (0) Remove load $R_L$.
 (1) Find equivalent resistance: $R=1+1+1||(1+1)=8/3$
 (2) Find current through voltage source: $I=V/R=3/8$
	and currents through two branches: $I_1=2/8$, $I_2=1/8$
 (3) Find open-circuit voltage $V_T=3/8+1/8=1/2$
 (4) Find $R_T$ between a and b with $V_0$ short circuit:
 the 5 resistors form a bridge, convert a delta to a Y with 
 3 resistors with value $1 \times 1/(1+1+1)=1/3$, total resistance:
 $R_T=(1+1/3)||(1+1/3)+1/3=1$
 For $R_L$ as the load of the Thevenin voltage source ($V_T, R_T$)
 to get maximum power, we need $R_L=R_T=1$.

\item {\bf Problem 3. (33 points)} 
The components in the circuit below take the following values:
$R_1=1.5\Omega$, $R_2=1\Omega$, $L=1mH$, and $C=500\mu F$. The voltage
source is $v(t)=40\sqrt{2} cos(1000t)$. Find the two branch currents 
$i_C(t)$, $i_L(t)$ and the overall current $i(t)=i_C(t)+i_L(t)$.
% also find the voltages across the components $v_C(t)$ and $v_L(t)$.

{\bf Solution:}

 phasor representation of voltage source: $\dot(V)=40\angle{0^\circ}$
 $\omega=1000$, 
 impedances: $Z_L=j\omega L=j\omega L=j1$, $Z_C=1/j\omega C=-j2$
 $Z_{RC}=1-j2$, $Z_{LRC}=j1 || (1-j2)=(2+j)/(1-j)=0.5+j1.5$
 $Z_{total}=1.5+Z_{LRC}=2+j1.5=2.5\angle{36.9^\circ}$
 $\dot{I}=\dot{V}/Z_{total}=40\angle{0^\circ}/2.5\angle{36.9^\circ}
 =16\angle{-36.9^\circ}$
 Find branch currents by current divider:
$\dot{I}_L=\dot{I} Z_{RC}/(Z_L+Z_{RC})=\dot{I} (1-j2)/(j+1-j2)
 =(1.58\angle{-18.4^\circ})(16\angle{-36.9^\circ})
 =25.3\angle{-55.3^\circ}$
 $\dot{I}_{RC}=\dot{I} Z_L/(Z_L+Z_{RC})=\dot{I} j/(j+1-j2)
 =(0.707\angle{135^\circ})(16\angle{-36.9^\circ})
 =11.3\angle{98.1^\circ}$

\htmladdimg{../midterm1b.gif}

\end{enumerate}

\end{document}


\item {\bf Evaluations:} Please evaluate my teaching effectiveness in terms 
of the following (circle one of the numbers, 1 for poor, 7 for excellent):

\begin{itemize}
\item {\bf Lectures (pace, clarity, style, etc.):}

\framebox[0.5in]{1}\framebox[0.5in]{2}\framebox[0.5in]{3}\framebox[0.5in]{4}\framebox[0.5in]{5}\framebox[0.5in]{6}\framebox[0.5in]{7}

\item {\bf Individual help/Office hours (approachability, helpfulness, effectiveness, etc.):}

\framebox[0.5in]{1}\framebox[0.5in]{2}\framebox[0.5in]{3}\framebox[0.5in]{4}\framebox[0.5in]{5}\framebox[0.5in]{6}\framebox[0.5in]{7}

\item {\bf Assignments, labs and tests (amount, difficulty, etc.)}

\framebox[0.5in]{1}\framebox[0.5in]{2}\framebox[0.5in]{3}\framebox[0.5in]{4}\framebox[0.5in]{5}\framebox[0.5in]{6}\framebox[0.5in]{7}

\item {\bf Overall teaching effectiveness of E84:} 

\framebox[0.5in]{1}\framebox[0.5in]{2}\framebox[0.5in]{3}\framebox[0.5in]{4}\framebox[0.5in]{5}\framebox[0.5in]{6}\framebox[0.5in]{7}

\end{itemize}

