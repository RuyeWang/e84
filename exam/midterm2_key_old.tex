\documentstyle[11pt]{article}
\usepackage{html}
\begin{document}
\begin{center}
{\Large \bf  Midterm Exam 2 ---- E84, Fall, 2004}
\end{center}

\begin{itemize}
\item Take home, open everything except discussion.
\item Mark your start and end times. Don't spend more than 3 hours.
\item Due Wednesday in class.
\item Mark your name and question number clearly on top of each page.
	Indicate the total number of pages submitted.
\item When solving a problem, list all the steps. In each step, describe 
	what you are doing in English, then show the calculation and the 
	result of the step. A final answer, even if correct, without 
	evidence of the steps leading to the answer will not receive credit.
\end{itemize}

\begin{enumerate}

\item {\bf Problem 1. (33 points)} 

The circuit shown below takes an input $v_0(t)=sin(\omega t)$ and generates 
two outputs $v_1(t)$ and $v_2(t)$. 

\htmladdimg{../midterm2a.gif}

\begin{itemize}
\item For each of the two output ports, find the gain $g_i$ ($i=1,2$) 
defined as the ratio between the magnitudes of the output and input 
voltages (steady state), and the phase shift $\phi_i$ ($i=1,2$) defined
as angular difference between the output and the input
\[ g_i\stackrel{\triangle}{=}\frac{|\dot{V}_i|}{|\dot{V}_0|},\;\;\;\;\;
\phi_i\stackrel{\triangle}{=}\angle \dot{V}_i-\angle \dot{V}_0,\;\;\;\;
(i=1,2) \]
Both $g_i$ and $\phi_i$ are functions of $\tau=RC$ and $\omega$.

\item Sketch $g_1$ and $g_2$ as functions of frequency $\omega$.

\item This ciruit is used as an effort to separate the two frequency components 
in the input $v_0(t)=sin(50 t)+cos(150 t)$. Assume $R=1K\Omega$, find the 
capacitance $C$ so that the cross-over frequency between the two outputs 
$v_1(t)$ and $v_2(t)$ is in the middle between the two frequency components
of the input, i.e., $\omega_c=(50+150)/2=100$.

\item Give the expression of $v_1(t)$ and $v_2(t)$ as two functions of time.

\end{itemize}

{\bf Solution:} 
\[	g_1=\frac{1/j\omega C}{R+1/j\omega C}=\frac{1}{1+j\omega \tau},\;\;\;\;
g_2=\frac{R}{R+1/j\omega C}=\frac{j\omega \tau}{1+j\omega \tau} \]
\[	|g_1|=\frac{1}{\sqrt{1+(\omega \tau)^2}},\;\;\;\;
	\phi_1=-tan^{-1}(\omega \tau),\;\;\;\;
	|g_2|=\frac{\omega \tau}{\sqrt{1+(\omega \tau)^2}},\;\;\;
	\phi_1=\frac{\pi}{2}-tan^{-1}(\omega \tau) \]
At the cross-over frequency between the two outputs their gains are equal 
(both reduced to half-power point), i.e., $\omega \tau=1$, and 
$|g_1|=|g_2|=1/\sqrt{2}$. Since the cross-over freqnecy is $\omega_c=100$,
we have $\tau=RC=1/\omega_c=0.01$ and $C=\tau/R=0.01/1000=10\;\mu F$.

For frequency $\omega=50$, 
\[ |g_1|=\frac{1}{\sqrt{1+(\omega \tau)^2}}=\frac{1}{\sqrt{1+0.5^2}}=0.89,\;\;\;\;
|g_2|=\frac{\omega \tau}{\sqrt{1+(\omega \tau)^2}}=\frac{0.5}{\sqrt{1+0.5^2}}=0.45 \]
\[ \phi_1=-\tan^{-1}(50\times 0.01)=-26.5^\circ,\;\;\;\;
   \phi_2=90^\circ-\tan^{-1}(50\times 0.01)=90^\circ-26.5^\circ=73.5^\circ \]

For frequency $\omega=150$, 
\[ |g_1|=\frac{1}{\sqrt{1+(\omega \tau)^2}}=\frac{1}{\sqrt{1+1.5^2}}=0.55,\;\;\;\;
|g_2|=\frac{\omega \tau}{\sqrt{1+(\omega \tau)^2}}=\frac{1.5}{\sqrt{1+1.5^2}}=0.83 \]
\[ \phi_1=-\tan^{-1}(150\times 0.01)=-56.3^\circ,\;\;\;\;
   \phi_2=90^\circ-\tan^{-1}(150\times 0.01)=90^\circ-56.3^\circ=33.7^\circ \]

\[ v_1(t)=0.89\; sin(50 t-26.5^\circ)+0.55\;sin(150 t-56.3^\circ) \]
\[ v_2(t)=0.45\; sin(50 t+73.5^\circ)+0.83\;sin(150 t+33.7^\circ) \]


\item {\bf Problem 2. (33 points)} 

Find the current $\dot{I}$ in the bridge circuit shown below, assuming the input
voltage is $\dot{V}=10\angle 0^\circ$. (Hint: Remove $R$ as the load of the rest 
of the circuit, which can be treated as a voltage source by Thevenin's theorem.)

\htmladdimg{../midterm2b.gif}

{\bf Solution:}
First find $\dot{V}_T$:
\[	\dot{V}_a=\frac{20}{20+j20}\cdot 10\angle 0^\circ
	=\frac{20\times 10\angle 0^\circ}{20\sqrt{2} \angle 45^\circ}
	=5-j5
\]
\[	\dot{V}_b=\frac{-j10}{-j10-j10}\cdot 10\angle 0^\circ
	=\frac{10\times 10\angle 0^\circ}{20}=5+j0	\]
\[	\dot{V}_T=\dot{V_a}-\dot{V}_b=-j5=5\angle -90^\circ	\]
Then find $Z_T$:
Short circuit voltage source, the impedance between points a and b is
\[ Z_T=Z_{ab}=Z_{ac}||Z_{ad}+Z_{bc}||Z_{bd}=\frac{20(j20)}{20+j20}+\frac{-j10}{2} 
	=10+j10-j5=10+j5 \]
Finally find $\dot{I}$:
\[ \dot{I}=\frac{V_T}{R+Z_T}=\frac{5\angle -90^\circ}{5+10+j5}
	=0.32\angle (-108.4^\circ) A \]

Alternatively, we can convert the top half of the bridge from delta to Y to get
\[   R_a=8-j16,\;\;\;\;R_b=8+j4,\;\;\;\;R_c=-4-2j  \]
The two parallel branches are
\[  Z_{left}=20+8+j4=28+j4,\;\;\;\;Z_{right}=-j10-4-j2=-4-j12,\;\;\;\;
    Z_{left} || Z_{right}=2-j14 \]
The total impedance is
\[ Z_{total}=Z_a+Z_{left} || Z_{right}=8-j16+2-j14=10-j30 \]
The current is
\[ \dot{I}=\frac{\dot{V}}{Z_{total}}=\frac{10}{10-j30}=(1+j3)/10 \]
The two branch currents are (current divider):
\[ \dot{I}_{left}=\dot{I}\frac{Z_{right}}{Z_{left}+Z_{right}}=(3-j)/20,\;\;\;
   \dot{I}_{right}=\dot{I}\frac{Z_{left}}{Z_{left}+Z_{right}}=(-1+j7)/20  \]
The voltages across $R$ and $C$ are:
\[ V_R=20 I_{left}= 3-j,\;\;\;\; V_C=-j10 \; I_{right}=3.5+j0.5 \]
The voltage across and current through the middle resistor $R=5$ are, respectively
\[ V=V_R-V_C=-0.5-j1.5,\;\;\;\;
  I=V/R=(-0.5-j1.5)/5=-0.1-j0.3=-0.32 \angle 71.6^\circ \; A \]
The current is the same as the one obtained from the previous method.


\item {\bf Problem 3. (33 points)} In the circuit shown below, $V_0=10V$,
$R_1=20\Omega$, $R_2=30\Omega$, $L_1=0.05H$, $L_2=0.15H$. Switch $S_1$ 
closes at $t=0$, switch $S_2$ closes at $t=0.02$ second. Find currents 
$i_1(t)$ and $i_2(t)$ for the following two time periods:
\begin{itemize}
\item $0\le t < 0.02 s$
\item $t \ge 0.02\;s$ 
\end{itemize}
Sketch the plots of these two currents for $t \ge 0$.

{\bf Hints:} Use $f(0)$, $f(\infty)$ and $\tau$ (no need to solve DE!).
For second period, assume $t'=t-0.02$, and use $i(0.02)$ as the initial 
value. In the final expression, replace $t'$ by $t-0.02$.

\htmladdimg{../midterm2c.gif}

{\bf Solution:}
\begin{itemize}
\item During the period $0\le t < 0.02 s$, the two currents are always the same
	$i_1(t)=i_2(t)$. Initial value: $i_1(0)=i_2(0)=0$, steady state value:
\[ i_1(\infty)=\frac{V_0}{R_1+R_2}=\frac{10}{20+30}=0.2A	\]
\[ \tau=\frac{L}{R}=\frac{0.05+0.15}{20+30}=4\times 10^{-3}\;s 	\]
\[ i_1(t)=i_2(t)=i_1(\infty)+[i_1(0)-i_1(\infty)]e^{-t/\tau}
	=0.2+(0-0.2) e^{-t/0.004}=0.2(1-e^{-250t})	\]
In particular, when $t=0.02$, we have
\[ i_1(t)=i_2(t)=0.2(1-e^{-250\itmes 0.02})=0.2A	\]

\item During the period $t \ge 0.02\;s$, we have $i_1(0.02)=i_2(0.02)=0.2A$, 
	$i_1(\infty)=V_0/R_1=10/20=0.5A$, $i_2(\infty)=0A$,
	$\tau_1=L_1/R_1=0.05/20=2.5\times 10^{-3}\;s$, and 
	$\tau_2=L_2/R_2=0.15/30=5\times 10^{-3}\;s$. We have

\[	i_1(t)=i_1(\infty)+[i_1(0.02)-i_1(\infty)]e^{-t/\tau_1}
	=0.5+(0.2-0.5)e^{-t/0.0025}=0.5-0.3e^{-400(t-0.02)} \]
\[	i_2(t)=i_2(\infty)+[i_2(0.02)-i_2(\infty)]e^{-t/\tau_2}	
	=0.2 e^{-200(t-0.02)}	\]

\end{itemize}




\end{enumerate}

\end{document}

\item {\bf Problem 3. (33 points)} 

In the circuit shown in the figure, $V_0=6V$, $I_0=2A$, $R_1=6\Omega$, 
$R_2=3\Omega$, and $L=0.5H$. Assume the circuit has reached steady state.
Find the voltage $v(t)$ across $R_1$ and current $i(t)$ through $L$ as 
time functions after the switch is closed at $t=0$.


{\bf Solution:} 
$i(0)=V_0/R_1=6/6=1A$, $i(\infty)=V_0/R_1+I_0=1+2=3A$. 

Find time constant: $R=R_1R_2/(R_1+R_2)=3\times 6/(3+6)=2\Omega$,
$\tau=L/R=0.5/2=0.25S$. 

The current $i(t)$ is therefore:
\[ i(t)=i(\infty)+[i(0)-i(\infty)]e^{-t/\tau}=3+(1-3)e^{-t/0.25}
	=3-2e^{-4t} \;A \]
\[ v(t)=V_0-V_L(t)=V_0-L\frac{d}{dt}i(t)=6-0.5 \frac{d}{dt}(3-2e^{-4t})
	6-4e^{-4t} \]


\item {\bf Problem 3. (33 points)} 
The circuit in the figure shows a voltage source $V_0$ and $R_0$ and an
amplification circuit modeled by as a two-port network. Assume the two-port 
is represented by a Z-model ($Z_{11}, Z_{12}, Z_{21}, Z_{22}$). Find the 
expression for the load impedance $Z_L$ in terms of $R_0$ as well as the
four Z-parameters, for it to get maximum power from the voltage source.
(Hint: use Thevenin's theorem.)

{\bf Solution:} 
\begin{itemize}
\item First set up all equations:
\[ \left\{ \begin{array}{l} V_1=Z_{11}I_1+Z_{12}I_2 \\
	V_2=Z_{21}I_1+Z_{22}I_2 \end{array} \right.	\]
\[ \left\{ \begin{array}{l} V_1=V_0-R_0I_1 \\
	V_2=-R_L I_2 \end{array} \right.	\]
\item Use Thevenin's theorem

\begin{itemize}
\item Find $Z_{Th}$: assume $V_0=0$, equate equations 1 and 3 to get:
\[ V_1=Z_{11}I_1+Z_{12}I_2, \;\;\;\;\mbox{i.e.,}\;\;\;\;\;
I_1=-\frac{Z_{12}}{Z_{11}+R_0} I_2 \]
Substitute this $I_1$ in equation 2 to get:
\[ V_2=(-\frac{Z_{12}Z_{21}}{Z_{11}+R_0}+Z_{22}) I_2,\;\;\;\;\;
\mbox{i.e.,}\;\;\;\;\;
Z_{Th}=\frac{V_2}{I_2}=-\frac{Z_{12}Z_{21}}{Z_{11}+R_0}+Z_{22}	\]
\item Find $V_{Th}$:, assume $I_2=0$, we have
\[ \left\{\begin{array}{l} V_1=Z_{11}I_1 \\V_2=Z_{21}I_1\end{array}\right. \]
Substitute $V_1=V_0-R_0I_1$ into $V_1=Z_{11}I_1$ to get
\[	V_{Th}=V_2=V_0\frac{Z_{21}}{Z_{11}+R_0}	\]
\end{itemize}

For $R_L$ to get maximum power, we need to have
\[ R_L=Z_{Th}=Z_{22}-\frac{Z_{12}Z_{21}}{Z_{11}+R_0}	\]
and the current is
\[ I_l=\frac{V_{Th}}{Z_{Th}+R_L}=\frac{V_{Th}}{Z_{Th}}
	=\frac{2V_0Z_{21}}{2(Z_{22}Z_{11}-Z_{12}Z_{21}+Z_{22}R_0)} \]


\end{itemize}


\end{enumerate}

\end{document}

