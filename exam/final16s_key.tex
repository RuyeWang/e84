\documentstyle[11pt]{article}
\usepackage{html}
\begin{document}
\begin{center}
{\Large \bf E84 Final Exam}
\end{center}

\section*{Instruction (read this before you start)}

\begin{itemize}
\item Take home, open everything except discussion. Due 12 noon, May 11,
  in my office.
\item Mark your start and end times. Don't spend more than 4 hours.
\item Compare your print-out of the exam with the online version to make 
  sure your hard copy is complete.
\item Mark your name and question number clearly on top of each page.
  Indicate the total number of pages submitted.
\item You should be able to solve all problems by hand. If you feel you 
  have to rely on some software tools (calculator, Matlab, or any other
  ``solver''), you may be doing something wrong.
\item When solving a problem, list all the steps. In each step, concisely
  state what you are doing in English, then show the calculation and the 
  result of the step. 
\item A final answer, even if correct, without evidence of the steps 
  leading to the answer will not receive credit.
\item Box all answers.
\item Don't click the ``next'' link for the problems until you are ready 
  to take the exam.
\end{itemize}

\section*{The Final Exam (don't click until you are ready to take the exam)}

\begin{enumerate}

\item Answer the following questions:
  \begin{itemize}
  \item ({\bf 10 pts})
    The RMS values of the three AC voltages across R, L, and C are 
    known to be $V_1=4\,V$, $V_2=5\,V$, and $V_3=2\,V$, respectively. 
    Find the RMS voltage $V_0$ of the source. If the AC voltage source 
    is replaced by a DC voltage source $V_0=10\,V$, Find the three 
    voltages $V_1$, $V_2$, and $V_3$.

    \htmladdimg{../../lectures/figures/Problems10.png}

  \item ({\bf 10 pts})
    In the figure above, it is know that $R=10\Omega$, $L=100\,mH$,
    and $C=500\,\mu F$, and the current through the loop is known to be
    $i(t)=\cos(100\,t)\,A$, find all four voltages $v_0(t)$, $v_1(t)$, 
    $v_2(t)$, and $v_3(t)$.

    {\bf Solution:} 
    $V_0=5$. $V_1=V_2=0$, $V_3=10$.

    \[
    \left\{\begin{array}{lll}
    v_1(t)=R i(t)=10\cos(100\,t)\\
    v_2(t)=Z_L i(t)=j\omega L i(t)=100\times 0.1 \cos(100\,t+\pi/2)=10\cos(100\,t+\pi/2)\\
    v_3(t)=Z_C i(t)=1/(j\omega C) i(t)=1/(100\times 500\times 10^{-6})\cos(100\,t)=20\cos(100,\,t-\pi/2)
    \end{array}\right.
    \]
    \[
    V_0=V_1+V_2+V_3=10+10\angle\pi/2+20\angle(-\pi/2)=10+10\angle(-\pi/2)
    \]
    \[
    v_0(t)=10\sqrt{2}\cos(100t-\pi/4)
    \]

  \item ({\bf 10 pts})
    The RMS values of the three currents through R, L, and C measured 
    by the three ammeters $A_1$, $A_2$, and $A_3$ are $I_1=4\,A$, $I_2=5\,A$, 
    and $I_3=2\,A$, respectively. Find the RMS value of the source current 
    $I_0$. If the AC voltage source is replaced by a DC current source 
    $I_0=1\,A$, Find the three currents $I_1$, $I_2$, and $I_3$ measured 
    respectively by ammeters $A_1$, $A_2$, and $A_3$.

    \htmladdimg{../../lectures/figures/Problems11.png}

    {\bf Solution:} $I_0=5$. $I_2=1$, $I_1=I_3=0$.
    

  \end{itemize}

\item ({\bf 23 pts})
  In the circuit below, $R_1=20\Omega$, $R_2=10\Omega$, $R_3=30\Omega$,
  $L=10\,mH$, $I_0=5A$. The circuit is at steady state before the switch is 
  turned from position 1 to position 2 at $t=0$. Find the three voltages 
  $v_1(t)$, $v_2(t)$ and $v_L(t)$ across $R_1$, $R_2$ and $L$, respectively,
  for $t>0$ (the bottom wire is treated as ground).
  
  \htmladdimg{../../lectures/figures/Problems12.png}


  {\bf Solution:} $\tau=L/(R_2+R_3)=10^{-2}/40=2.5\times 10^{-4}\,s$
  \[
  i_L(0^-)=i_L(0^+)=I_0\frac{R_1}{R_1+R_3}=5A\frac{20}{50}=2A,\;\;\;\;\;
  i_L(\infty)=0,\;\;\;\;\;i_L(t)=2\,e^{-t/\tau}
  \]
  \[ 
  v_1(t)=5\times 20=100\,V,\;\;\;\; v_2(t)=R_2 i_L(t)=-20\,e^{-t/\tau},\;\;\;\;\;
  v_3(t)=R_3 i_L(t)=60\,e^{-t/\tau},\;\;\;\;\;
  v_L(t)=-80\,e^{-t/\tau}
  \]


\item ({\bf 23 pts})
  The circuit below shows a simple means for obtaining improved bias
  stability of the DC operating point of the transistor. As always,
  assume $V_{BE}=0.7V$ when answering the following questions.

  \htmladdimg{../../lectures/figures/midterm3e.gif}

  \begin{itemize}
  \item Complete the event chain below (in terms of $V_C$, $V_B$, $I_B$,
    etc.) to show qualitatively that $R_B$ connected to the collector $C$ 
    (instead of $V_{CC}$) introduces a negative feedback by which the DC
    operating point tends to be stabilized:
    \[
    I_C \uparrow \Longrightarrow \cdots\cdots\cdots \Longrightarrow I_C \downarrow     
    \]
  \item Derive an expression for $I_C$ in terms of $R_B$, $R_C$, $V_{CC}$
    and $\beta$.
  \item Assuming $\beta=100$, $V_{BE}=0.7$, and $V_{CC}=10V$, find $R_C$ 
    and $R_B$ so that $V_{CE}=5V$ and $I_C=2mA$.
  \item Confirm your design does put the DC operating point in the middle
    of the linear region.
  \end{itemize}


  {\bf Solution:}

  \[
  I_C \uparrow \Longrightarrow V_C \downarrow
  \Longrightarrow I_B \downarrow \Longrightarrow I_C \downarrow 
  \]

  \[
  I_B=\frac{V_{CC}-0.7}{(\beta+1)R_C+R_B},\;\;\;\;\;
  I_C=\beta I_B=\frac{\beta(V_{CC}-0.7)}{(\beta+1)R_C+R_B} 
  \]
  $I_C=2mA$, $I_B=I_C/\beta=0.02mA$, $V_C=5V$,
  $R_C=(V_{CC}-V_C)/(I_C+I_B)=5V/2.02mA=2.475K\Omega$,  
  $R_B=(5-0.7)/0.02=4.3/0.02=215K\Omega$.

  \[
  I_B=\frac{V_{CC}-0.7}{(\beta+1)R_C+R_B}=\frac{9.3}{101\times 2.5K+215}
  =0.02mA
  \]

  \[
  I_C=\beta I_B=2 mA,\;\;\;V_C=V_{CC}-(\beta+1)I_B=5V 
  \]

  \[
  I_C=\beta I_B=2.6mA,\;\;\;V_C=V_{CC}-(\beta+1)I_B=3.5V 
  \]

\item ({\bf 24 pts})
  Determine qualitatively what kind of filter the following op-amp
  circuit is. Is it a first or second order? a low-pass, high-pass, or 
  band-pass? 

  Find the frequency response function (FRF) of the circuit, and find the
  cut-off frequency $\omega_c$ of the filter in terms of the given circuit
  parameters $R_1$, $R_2$, $R$ and $C$, so that
  $|H(j\omega_c)|=|H(j\omega_p)|/\sqrt{2}$, where $\omega_p$ is the
  frequency at which $|H(j\omega_p)|$ reaches maximum.

  \htmladdimg{../../lectures/figures/Problems14.png}


  {\bf Solution:} Let $v^+\approx v^-=v$
  \[
  v=v_{out}\frac{R_1}{R_1+R_2},\;\;\;\;\;\;\;\;\;\;\frac{v_{in}-v}{R}
  =\frac{v}{1/j\omega C}=v\;j\omega C
  \]
  Substituting $v$ in the first equation into the second, we find
  \[
  H(j\omega)=\frac{v_{out}}{v_{in}}=\frac{R_1+R_2}{R_1}\;\frac{1}{j\omega RC+1}
  \]
  This is a first-order low-pass filter with cut-off frequency $\omega_c=1/RC$.


\end{enumerate}
\end{document}

