 \documentstyle[11pt]{article}
 \usepackage{html}
 \begin{document}
 \begin{center}
 {\Large \bf  E84 Midterm Exam 1 (Fall 2008)}
 \end{center}

 \section*{Please fill out the feedback form}

 {\bf Note: } I would like to collect your feedback and comments. Please fill
 the following form {\bf anonymously} and turn it in {\bf separately} from your 
 midterm exam. Your help is greatly appreciated.

 \begin{itemize}
 \item {\bf Feedback:} (circle one of the numbers)
 \begin{itemize}

 \item {\bf Lecture pace:} (1 too slow, 3 just right, 5 too fast)

 \framebox[0.5in]{1}\framebox[0.5in]{2}\framebox[0.5in]{3}\framebox[0.5in]{4}\framebox[0.5in]{5}

 \item {\bf Lecture clearity:} (1 Too tedious, 3 just right, 5 too sketchy)

 \framebox[0.5in]{1}\framebox[0.5in]{2}\framebox[0.5in]{3}\framebox[0.5in]{4}\framebox[0.5in]{5}

 \item {\bf Overall homework workload:} (1 too little, 3 just right, 5 too much)

 \framebox[0.5in]{1}\framebox[0.5in]{2}\framebox[0.5in]{3}\framebox[0.5in]{4}\framebox[0.5in]{5}

 \item {\bf Overall difficulty of course materials:} (1 too easy, 3 just right, 5 too hard)

 \framebox[0.5in]{1}\framebox[0.5in]{2}\framebox[0.5in]{3}\framebox[0.5in]{4}\framebox[0.5in]{5}
 \end{itemize} 

 \item {\bf Comments:}
 \begin{itemize}

 \item {\bf What went well for your learning in the course? Be specific.}

 \begin{tabular}{l}
 .  \\
 .  \\
 .  \\
 \end{tabular}
 \vskip 5cm

 \item {\bf What did not go well for your learning in the course? Be specific.}
 \begin{tabular}{l}
 .  \\
 .  \\
 .  \\
 \end{tabular}
 \vskip 0.9in

 \item {\bf Indicate one thing you want to improve and how.}
 \begin{tabular}{l}
 .  \\
 .  \\
 .  \\
 \end{tabular}
 \vskip 0.9in

 \end{itemize}

 \end{itemize}


 \section*{E84 Midterm Exam 1}

 {\bf Instructions}
 \begin{itemize}
 \item Take home, open everything except discussion. Due Wednesday in class.
 \item Mark your start and end times. Don't spend more than 2 hours.
 \item Compare your print-out of the exam with the online version to make
         sure your hard copy is complete.
 \item Mark your name and question number clearly on top of each page.
         Indicate the total number of pages submitted.
 \item When solving a problem, list all the steps. In each step, describe 
         concisely what you are doing in English, then show the calculation 
         and the result of the step. A final answer, even if correct, without 
         evidence of the steps leading to the answer will not receive credit.
 \end{itemize}

 \begin{enumerate}

 \item {\bf Problem 1. (30 pts)}

   Find the two currents labeled as $I_a$ and $I_b$ in the figure.

   \htmladdimg{../mdtm1_08b.gif}

   {\bf Solution:}

   By KCL (observation), $I_b=6+12=18A$. 

   Use loop current method to around the loop and get:
   \[ 18I_a+12(I_a+12)+6(I_a+12+6)=0,\;\;\;\;\mbox{i.e.,}\;\;\;\;
   36I_a+252=0 \]
   Solving this we get $I_a=-7$. 

   Assume currents going through 6$\Omega$, 12$\Omega$ and 18$\Omega$
   resistors are, respectively, $I_1$ (down), $I_2$ (down) and $I_3$ 
   (right). Using KCL and KVL, we get
   \[
   \left\{ \begin{array}{l}
     I_1+I_2=6\\I_2+I_3=-12\\6(I_1-2 I_2+3I_3)=0 \end{array} \right. 
   \]
   Solving this we get: $I_1=11$, $I_2=-5$, $I_3=I_a=-7$, $I_b=I_1-I_3=18$.

   Alternatively, one can use superposition theorem. First, turn the 6A current 
   off and get 
   \[
   I'_a=-12 \frac{6+12}{6+12+18}=-6 A 
   \]
   Second, turn the 12A off and get
   \[
   I''_a=-6 \frac{6}{6+12+18}=-1 A 
   \]
   Now we have $I_a=I'_a+I''_a=-7 A$. $I_b=18A$ is simply the sum of 6A and 12A.

\item {\bf Problem 2. (30 points)}

  Find the voltage $V$ across the 2$\Omega$ resistor (with assumed polarity shown),
  and the current $I$ through the 1$V$ voltage source (with assumed direction shown).

  \htmladdimg{../mdtm1_08a.gif}

  {\bf Solution:}

  Assume lower-left node is grounded as 0V. upper-left node is 5V, upper-right
  node is 1V, and lower-right node is 3V. Voltage across $2\Omega$ resistor is
  $1V-3V=-2V$. Current through $1\Omega$ resistor (right-ward) is 4A, current
  through $2\Omega$ resistor (up-ward) is 1A, Applying KCL to upper-rigt node,
  we get $I=5A$.

  Or, use superposition theorem:
  \begin{itemize}
    \item short the 1V and 2V sources, $V_1=-3V$, $I_1=3A+1.5A=4.5A$
    \item short the 1V and 3V sources, $V_1= 0V$, $I_1=  2A$
    \item short the 2V and 3V sources, $V_1= 1V$, $I_1=-(1+0.5)A=-1.5A$
  \end{itemize}
  Total: $V=-2V$, $I=5A$.

\item {\bf Problem 3. (40 pts)}

  The following two parts of the problem are independent, each worth 20 pts.
  \begin{itemize}
    \item Find the resistance $R$ so that the voltage across the resistor is 1V.
    \item Find the resistance $R$ so that the current through the resistor is 0.2A.
  \end{itemize}

  \htmladdimg{../mdtm1_08c.gif}
  
  {\bf Solution:} 

  The problems can be most easily solved by assuming the desired voltage/current
  and then figuring out the resistance $R$ needed. 

  Fiirst, if $V=1$ as desired, then the current (rightward) through the $6\Omega$ 
  resistor is $(5+3-1)V/6\Omega=7/6A$, and the current through $R$ (downward) is 
  $I=7/6-1=1/6\;A$, and we get $R=V/I=6\Omega$. 

  Second, if $I=0.2A$ as desired, then the current (rightward) through $6\Omega$
  resistor has to be $1.2A$ and the voltage drop across it is $7.2V$ (positive on
  the left). Now the voltage to the right of the $6\Omega$ resistor is $V=5+3-7.2=0.8V$,
  and $R=V/I=0.8/0.2=4\Omega$.

  Superposition method could also be used. 
  \begin{itemize}
    \item Only turn 3V voltage source on (5V shorted together with $3\Omega$ resistor, 1A
      current source is open), we get
      \[ I'=\frac{3}{6+R},\;\;\;\;\; V'=\frac{3R}{6+R} \]
    \item Only turn 5V voltage source on (3V shorted, 1A current source is open), we get
      \[ I''=\frac{5}{6+R},\;\;\;\;\; V''=\frac{5R}{6+R} \]
    \item Only turn 1A current source on (3V and 5V are both shorted), we get
      \[ I'''=\frac{-6}{6+R},\;\;\;\;\; V'''=\frac{-6R}{6+R} \]
  \end{itemize}
  Now we have:
  \[
  I=I'+I''+I'''=\frac{2}{6+R}=0.2A;\;\;\;\; V=V'+V''+V'''=\frac{2R}{6+R}=1V 
  \]
  Solving these for $R$, we get $R=4\Omega$ for the first equation and 
  $R=6\Omega$ for the second.

  Alternatively, we can use Thevenin's theorem or Norton's theorem by treating $R$
  as a load. In either case, we need to find the internal resistance of the rest
  of the circuit excluding R. When all energy sources are turned off, the internal 
  resistance is simply $R_T=6\Omega$. 
  \begin{itemize}
    \item Find open circuit voltage $V_{oc}$:

      Use superposition, when current source is open, $V'_{oc}=5V+3V=8V$; when 
      voltage sources are short, $V''_{oc}=-1A\times 6\Omega=-6V$. Therefore 
      $V_T=V_{oc}=V'_{oc}+V''_{oc}=2V$

    \item Find short circuit current $I_{sc}$:

      Use superposition, when current source is open, $I'_{sc}=8V/6\Omega=4/3\;A$;
      when voltage sources are short, $I''_{sc}=-1A$. Therefore 
      $I_N=I_{sc}=I'_{sc}+I''_{sc}=1/3\;A$, which is consistent with the previous 
      result: $V_T=V_{oc}=I_{sc} R_T=6/3 = 2V$.

  \end{itemize}
  The rest of the circuit is equivalent to a voltage source $V_T=2V$ in series with 
  a resistor $R=6\Omega$, for $V=1V$, we need $R=6\Omega$. For $I=0.2A$, we need 
  $R=4\Omega$.  

\end{itemize}

\end{document}

  

\item {\bf Problem 1. (33 points)}
  Find all node voltages in the circuit, where $R_1=200\Omega$, 
  $R_2=75\Omega$, $R_3=25\Omega$, $R_4=50\Omega$, $R_5=100\Omega$,
  $V=10V$, $I=0.2A$. Use both node voltage and loop current methods.

  \htmladdimg{../hw3b.gif}

%  {\bf Solution:}
%  \begin{itemize}
%    \item {\bf Node voltage:}
%      \[ \begin{array}{ll}
%	\mbox{left node:} & V_1/200+(V_1-V_2)/75=0.2 \\
%	\mbox{middle node:} & (V_2-V_1)/75+V_2/25+V_3/100=0 \\
%	\mbox{voltage source:} & V_3=V_2-V=V_2-10 \end{array} \]
%      Solving this we get:
%      \[ V_1=14.24,\;\;\;\;\;V_2=4.58,\;\;\;\;V_3=-5.42 \]
%    \item {\bf Loop current:}
%      first convert the current source $I=0.2A$ and the parallel resistor
%      $R_1=200\Omega$ into a voltage source with $V_1=40V$ in series with
%      $R_1=200\Omega$. Assume loop currents $I_a$ (left loop) and $I_b$
%      right loop and get loop current equations:
%      \[ \begin{array}{ll}
%	\mbox{left loop:} & (200+75+25)I_a-25I_b=40 \\
%	\mbox{right loop:} & -25I_a+(100+25)I_b=-10 \end{array} \right. \]
%      Solving these we get $I_a=0.129A$, $I_b=-0.054A$. 
%      $V_2=(I_a-I_b)R_3=0.183\times 25=4.58V$, same as previous result.
%  \end{itemize}

\item {\bf Problem 2. (33 points)} 
A DC generator of $V_0=78V$ is connected through two wires each of 
$0.1\Omega$ to two different loads, an electrical oven of $4\Omega$ 
and a re-chargeable battery with voltage (leftover) $60V$ and internal 
resistance $1\Omega$. The schematic of this circuit is shown in the 
figure below. Find
\begin{itemize}
	\item power consumption of the electrical oven
	\item power loss caused by the internal resistance of the battery
\end{itemize}

\htmladdimg{../midterm1c.gif}

%{\bf Solution:}

%{\bf Method 1} Use superposition principle to find I0 through the wires, 
%I1 through the oven, and I2 through the battery:

%\begin{itemize}
%\item (1) generator alone: $4 || 1=4/5=0.8$, total resistance is 
%	$0.1+0.1+0.8=1$, $I0'=78/1=78$, $I1'=78 \times 1/5=78/5=15.6$, 
%	$I2'=78 \times 4/5=312/5=62.4$

%\item (2) battery alone: $4 || 0.2=4/21$, total resistance is 
%	$1+4/21=25/21=1.19$, $I2''=60/(25/21)=60x21/25=252/5=50.4$, 
%	$I1''=(252/5) \times 0.2/4.2=12/5=2.4$,
% 	$I0''=(252/5) \times 4/4.2=240/5=48$

%\item (3) total currents:

% $I0=I0'-I0''=78-48=30$, $I1=I1'+I1''=78/5+12/5=90/5=18$, 

% $I2=I2'-I2''=312/5-252/5=60/5=12$ (check: I1+I2=18+12=30=I0)
% power on oven: $18\times 18\times 4=1296$, power on internal resistance: $12x12=144$
%\end{itemize}

%{\bf Method 2} Use Thevenin's theorem to find current through $R=4\Omega$.
%Pull the resistor out, treat the rest as a one-port network. 
%\begin{itemize}
%\item Find voltage $V_T$:
%\begin{itemize}
%\item 78V alone, $V'=78\times \frac{1}{1+0.1+0.1}=65V$
%\item 60V alone, $V''=60-60\times \frac{1}{1+0.1+0.1}=10V$
%\item Total voltage: $V=V'+V''=65+10=75V$
%\end{itemize}
%\item Find resistance $R_T=1 || (0.1+0.1)=1/6$.
%\item Find current through $R=4\Omega$:
%	$I_1=V_T/(R_T+4)=75/(1/6+4)=18$
%\item Power by $R=4\Omega$ is $18\times 18\times 4=1296$
%\item Voltage across $R=4\Omega$ is $18\times 4=72V$
%\item Voltage across $R=4\Omega$ is $72-60=12V$.
%\item Power by $R=1\Omega$ is $12\times 12=144 W$
%\end{itemize}

%{\bf Method 3} Solve the circuit by KVL. Let generator current be $I0$, current 
%through oven be $I1$, current through battery be $I2$, set up these equations:
%\[	I0-I1-I2=0,\;\;\;\;78-0.2I0-4I1=0,\;\;\;\;60+I2-4I1=0	\]
%Solve to get $I0=30$, $I1=18$, $I2=12$.

\item {\bf Problem 3. (34 points)} 
Find the value of the load resistance $R_L$ for it to get maximum power 
from the voltage source.

\htmladdimg{../midterm1a.gif}

%{\bf Solution:}

%Use Thevenin's theorem (assume all $R=1$, $V_0=1$)
%\begin{itemize}
%\item (0) Remove load $R_L$.
%\item (1) Find open-circuit voltage $V_T$:
%\begin{itemize}
%\item Find equivalent resistance: $R=1+1+1||(1+1)=8/3$
%\item Find current through voltage source: $I=V/R=3/8$
%	and currents through two branches: $I_1=2/8$, $I_2=1/8$
%\item Find $V_T=3/8+1/8=1/2$
%\end{itemize}
%\item (2) Find $R_T$ between a and b with $V_0$ short circuit:
%\begin{itemize}
%\item The 5 resistors form a bridge, convert a delta to a Y with 
% 3 resistors with value $1 \times 1/(1+1+1)=1/3$, 
%\item total resistance: $R_T=(1+1/3)||(1+1/3)+1/3=1$
%\end{itemize}
%\item (3) For $R_L$ as the load of the Thevenin voltage source 
%	($V_T, R_T$) to get maximum power, we need $R_L=R_T=1$.
%\end{itemize}

\end{enumerate}

\end{document}


\item {\bf Problem 3. (33 points)} 
The components in the circuit below take the following values:
$R_1=1.5\Omega$, $R_2=1\Omega$, $L=1mH$, and $C=500\mu F$. The voltage
source is $v(t)=40\sqrt{2} cos(1000t)$. Find the two branch currents 
$i_C(t)$, $i_L(t)$ and the overall current $i(t)=i_C(t)+i_L(t)$.
% also find the voltages across the components $v_C(t)$ and $v_L(t)$.

\htmladdimg{../midterm1b.gif}

{\bf Solution:}

phasor representation of voltage source: $\dot(V)=40\angle{0^\circ}$
$\omega=1000$, 

impedances:
\begin{itemize}
\item $Z_L=j\omega L=j1$, 
\item $Z_C=1/j\omega C=-j2$,
\item $Z_{RC}=1-j2=\sqrt{5}\angle{-63.43}$, 
\item$Z_{LRC}=j1 || (1-j2)=(2+j)/(1-j)=0.5+j1.5=1.58\angle{71.57}$,
\item $Z_{total}=1.5+Z_{LRC}=2+j1.5=2.5\angle{36.87^\circ}$,
\end{itemize}

Find total current:

$\dot{I}=\dot{V}/Z_{total}=40\angle{0^\circ}/2.5\angle{36.9^\circ}
=16\angle{-36.87^\circ}$

Find branch currents by current divider:

 $\dot{I}_L=\dot{I} Z_{RC}/(Z_L+Z_{RC})=\dot{I} (1-j2)/(j+1-j2)
 =(1.58\angle{-18.4^\circ})(16\angle{-36.9^\circ})
 =25.3\angle{-55.3^\circ}$,

 $\dot{I}_{RC}=\dot{I} Z_L/(Z_L+Z_{RC})=\dot{I} j/(j+1-j2)
 =(0.707\angle{135^\circ})(16\angle{-36.9^\circ})
 =11.3\angle{98.1^\circ}$

\end{enumerate}

\end{document}


\item {\bf Evaluations:} Please evaluate my teaching effectiveness in terms 
of the following (circle one of the numbers, 1 for poor, 7 for excellent):

\begin{itemize}
\item {\bf Lectures (pace, clarity, style, etc.):}

\framebox[0.5in]{1}\framebox[0.5in]{2}\framebox[0.5in]{3}\framebox[0.5in]{4}\framebox[0.5in]{5}\framebox[0.5in]{6}\framebox[0.5in]{7}

\item {\bf Individual help/Office hours (approachability, helpfulness, effectiveness, etc.):}

\framebox[0.5in]{1}\framebox[0.5in]{2}\framebox[0.5in]{3}\framebox[0.5in]{4}\framebox[0.5in]{5}\framebox[0.5in]{6}\framebox[0.5in]{7}

\item {\bf Assignments, labs and tests (amount, difficulty, etc.)}

\framebox[0.5in]{1}\framebox[0.5in]{2}\framebox[0.5in]{3}\framebox[0.5in]{4}\framebox[0.5in]{5}\framebox[0.5in]{6}\framebox[0.5in]{7}

\item {\bf Overall teaching effectiveness of E84:} 

\framebox[0.5in]{1}\framebox[0.5in]{2}\framebox[0.5in]{3}\framebox[0.5in]{4}\framebox[0.5in]{5}\framebox[0.5in]{6}\framebox[0.5in]{7}

\end{itemize}

