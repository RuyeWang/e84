\documentstyle[11pt]{article}
\usepackage{html}
\begin{document}
\begin{center}
{\Large \bf  E84 Midterm Exam 2 (Fall 2008)}
\end{center}

\section*{E84 Midterm Exam 2}

{\bf Instructions}
\begin{itemize}
\item Take home, open notes, feel free to use a calculator, but not any software 
  package such as Multisim. 
\item Mark your start and end times. Don't spend more than 3 hours. Due on 
  Monday in class.
\item Compare your print-out of the exam with the online version to make
  sure your hard copy is complete.
\item Mark your name and question number clearly on top of each page.
  Indicate the total number of pages submitted.
\item When solving a problem, list all the steps. In each step, indicate
  concisely what you are doing in English, then show the calculation 
  and the result of for the step. Box the final answer.
  A final answer, even if correct, without evidence of the steps leading
  to it will receive ZERO credit.
\end{itemize}

\begin{itemize}

\item {\bf Problem 1. (33 pts)}

In the circuit shown in the figure, $R_1=R_2=5\,\Omega$, $C_1=0.01\,F$,
$C_2=0.03\,F$, $L=1\,H$, and the input voltage is 
$v_{in}(t)=10+20\cos 5 t+30 \cos 10 t$ V. Find the output voltage $v_{out}(t)$
across $R_2$. (Hint: the circuit is linear, therefore the superposition
principal applies.)

\htmladdimg{../mdtm2_08c.gif}

%{\bf Solution:} Use superposition theorem. 
%\begin{itemize}
%  \item DC component: 
%    \[ v'_{out}(t)=10\times \frac{5}{5+5}=5\;V \]
%  \item When $\omega=5$, the impedance of the middle branch is:
%    \[ j5 // (-j 20)+(-j 20/3)=0 \]
%    i.e., this branch is at resonance with zero impedance, with $v''_{out}(t)=0$.
%  \item When $\omega=10$, the impedance of the middle branch is:
%    \[ j10 // (-j 10)+(-j 10/3)=\infty \]
%    i.e., there is no current through this branch at this frequency,
%    and the output is:
%    \[ v'''_{out}(t)=v_{in}(t)\frac{5}{5+5}=15\cos 10 t\; V \]
%\end{itemize}
%The overall output is: $v_{out}(t)=v'_{out}(t)+v'''_{out}(t)=5+15 \cos 10 t\;V$

\item {\bf Problem 2. (33 pts)}

The circuit in the figure below is composed of three resistors
$R_1=R_2=R_3=1k\Omega$, and an ideal inductor $L=1\;mH$. The current source 
is 6 mA. The switch is open when $t<0$ and the circuit has reached steady
state. The switch is then closed at $t=0$. 

\htmladdimg{../mdtm2_08a.gif}

\begin{itemize}

\item Fill out the following table for the initial and final values of various
variables in the circuit (12 pts).

\begin{tabular}{c||cccc|cccc} \hline
  time $t$&$i_L(t)$&$i_1(t)$&$i_2(t)$&$i_3(t)$&$v_L(t)$&$v_1(t)$&$v_2(t)$&$v_3(t)$\\ \hline \hline
  $0^-$    & & & & & & & & \\ \hline
  $0^+$    & & & & & & & & \\ \hline
  $\infty$ & & & & & & & & \\ \hline
\end{tabular}

where $v_i(t)$ and $i_i(t)$ and the voltage and current associated with $R_i$
for $i=1,2,3$, respectively. Their polarities and directions are shown in the figure.

\item Find the time constant $\tau=L/R$ of the circuit, and give the expressions of
$i_L(t)$ and $v_L(t)$ associated with the inductor $L$ (6 pts).

\item Sketch the waveforms of all eight functions. (16 pts)

\end{itemize}

%{\bf Solution:} 
%
%\begin{itemize}
%\item
%  \begin{tabular}{c||cccc|cccc} \hline
%    time $t$&$i_L(t)$&$i_1(t)$&$i_2(t)$&$i_3(t)$&$v_L(t)$&$v_1(t)$&$v_2(t)$&$v_3(t)$\\ \hline \hline
%    $0^-$    &4&4&2&2&0&4&2&2 \\ \hline
%    $0^+$    &4&3&3&2&2&3&3&2 \\ \hline
%    $\infty$ &6&3&3&0&0&3&3&0 \\ \hline
%  \end{tabular}

%\item $R=R_3=1$, $\tau=L/R=10^{-6}$ second.
%  \[ i_L(t)=i_L(\infty)+[i_L(0)-i_L(\infty)]e^{-t/\tau}
%  = 6+(4-6) e^{-t/0.001}=6-2 e^{-1000 t}\;\;(mA) \]
%  $i_L(t)$ increases from 4 to 6 exponentially.

%  \[ v_L(t)=L \frac{d}{dt} i_(t)=-2L \frac{d}{dt}e^{-t/\tau}
%  =2 e^{-t/\tau}\;\;V \]
%  $v_L(t)$ jumps from 0 to 2V at t=0 and then decreases to 0 again exponentially.

%\item 
%  $i_1(t)$ jumps from 4 mA to 3 mA, $v_1(t)$ jumps from 4V to 3V
%  $i_2(t)$ jumps from 2 mA to 3 mA, $v_2(t)$ jumps from 2V to 3V
%  $i_3(t)$ decreases from 2 mA to 0 exponentially.
%  $v_3(t)$ decreases from 2 V to 0 exponentially.
%\end{itemize}

\item {\bf Problem 3. (34 pts)}

In the circuit shown in the following figure, $R_1=10 \Omega$, $R_2=2 \Omega$,
$R_3=8 \Omega$, $L=2 H$, $V=20 V$, $I=1.25 A$. The switch is in position a and
the circuit has reached steady state, until the moment $t=0$ when the switch is
turned to position b. Determine the voltage $v(t)$ across $R_3$ as the response
of the system to the change of position of the switch.

Hint: $v(t)$ is the superposition of $v'(t)$ and $v''(t)$ responding to two processes
respectively:
(a) the voltage drop across $R_3$ due to the initial current through $L$ alone after
the switch is turned from a to b; and (b) the voltage across $R_3$ as the complete 
response to the current source $I$ alone after the switch turned from a to b.

\htmladdimg{../mdtm2_08b.gif}

Now solve the problem in the following steps by superposition theorem:
\begin{itemize}
  \item Find $v'(t)$ across $R_3$ due to $i_L(t)$ alone:
    \begin{enumerate}
    \item (2 pts) Find the steady state current $i_L(0^-)$ through the inductor $L$ 
      before the switch is turned from a to b at $t=0$;
    \item (2 pts) Find the current $i_L(0^+)$ through $L$ right after the switch is 
      turned from a to b at $t=0$;
    \item (3 pts) Find the voltage $v'(0^+)$ across $R_3$ due to $i_L(0^+)$;
    \item (3 pts) Find the time constant $\tau$ after $t=0$;
    \item (3 pts) Find the $v'(t)$ across $R_3$ corresponding to the decay of
      $i_L(t)$.
    \item (3 pts) Sketch $v'(t)$.
    \end{enumerate}
  \item Find $v''(t)$ across $R_3$ due to current source $I=1.25 A$:
    \begin{enumerate}
    \item (3 pts) Find the initial condition for the voltage $v''(0^+)$ across 
      $R_3$ due to $I$ right after $t=0$;
    \item (4 pts) Find the steady state response $v''(\infty)$ across $R_3$ due 
      to the $I$ alone;
    \item (4 pts) Find the complete response $v''(t)$ due to the current source
      $I$ alone.
    \item (3 pts) Sketch $v''_t(t)$.
    \end{enumerate}
  \item (4 pts) Find and sketch the total voltage $v(t)=v'(t)+v''(t)$ across $R_3$.
\end{itemize}

%{\bf Solution:}

%\begin{itemize}
%  \item
%    \begin{enumerate}
%    \item $i_L(0^-)=20V/10\Omega=2A$ (upward)
%    \item $i_L(0^+)=i_L(0^-)=2A$ as current through L does not change instantly.
%    \item $v'(0^+)=i_L(0^+)R_3=2\times 8=16 V$
%    \item $\tau=L/R=2H/(2+10+8)\Omega=0.1$ sec.
%    \item Due to current $i_L(t)=2 e^{-t/\tau}A$, the voltage $v(0^+)$ across $R_3$ is
%      \[ v'(t)=v'(0^+) e^{-t/tau}=16 e^{-10t} \]
%      \item sketch
%    \end{enumerate}
%  \item
%    \begin{enumerate}
%    \item $v''(0^+)=-1.25\times 8=-10 V$
%    \item $v''(\infty)=-1.25\times\frac{8+12}{8\times 12}=-6 V$
%    \item \[ v''(t)=v''(\infty)+[v''(0^+)-v''(\infty)]e^{-t/\tau}=
%      -6+(-10+6)e^{-10t}=-6-4e^{-10t} \]
%    \item sketch
%    \end{enumerate}
%  \item \[ v(t)=v'(t)+v''(t)=16 e^{-10t}+(-6-4e^{-10t})=12e^{-10t}-6 \]
%\end{itemize}

\end{itemize}
\end{document}
