\documentstyle[11pt]{article}
\usepackage{html}
\begin{document}
\begin{center}
{\Large \bf  Midterm Exam 3 ---- E84, Fall, 2004}
\end{center}

\begin{itemize}
\item Take home, open everything except discussion.
\item Mark your start and end times. % Don't spend more than 3 hours.
\item Due Monday in class.
\item Mark your name and question number clearly on top of each page.
	Indicate the total number of pages submitted.
\item When solving a problem, list all the steps. In each step, describe 
	what you are doing in English, then show the calculation and the 
	result of the step. A final answer, even if correct, without 
	evidence of the steps leading to the answer will not receive credit.
\end{itemize}

\begin{enumerate}

\item {\bf Problem 1. (20 points)} 

Find the output voltage $V$ for each of the two circuits shown in the 
figure below, using the line at the bottom as the reference (ground).

\htmladdimg{../midterm3d.gif}

{\bf Solution:} (a) 4.3V (b) 0.7V

\item {\bf Problem 2. (40 points)} 

The circuit below shows a simple means for obtaining improved bias
stability of the DC operating point of the transistor. As always,
assume $V_{BE}=0.7V$ when answering the following questions.

\htmladdimg{../midterm3e.gif}

\begin{itemize}
\item Explain qualitatively what happens if $I_C$ tends to rise as a
result of an increase in $\beta$.
\item Derive an expression for $I_C$ in terms of $R_B$ and $R_C$ and
$\beta$.
\item Find an approximation of the expression of $I_C$ if $\beta$ is 
large enough. To make $I_C$ independent of $\beta$, how $R_C$ and
$R_B$ should be related?
\item Find $R_C$ and $R_B$ so that the DC operating point is $V_{CE}=5V$ 
and $I_C=2mA$, when $\beta=100$ and $V_{CC}=10V$.
\item Find $V_{CE}$ and $I_C$ for $\beta=50$, $\beta=100$, and $\beta=200$
	based on the resistances found above.
\end{itemize}

{\bf Solution:} 

\begin{itemize}
\item $\beta \uparrow \Longrightarrow V_C \downarrow \Longrightarrow 
	I_B \downarrow \Longrightarrow I_C \uparrow $
\item
\[ I_B=\frac{V_{CC}-0.7}{(\beta+1)R_C+R_B},\;\;\;\;\;
   I_C=\beta I_B=\frac{\beta(V_{CC}-0.7)}{(\beta+1)R_C+R_B}  \]
\item If $(\beta+1)R_C \gg R_B$, then $I_C \approx (V_{CC}-0.7)/R_C$, 
independent of $\beta$.
\item $I_C=2mA$, $V_C=5V$, $R_C=(V_{CC}-V_C)/I_C=5V/2mA=2.5K\Omega$,
$I_B=I_C/\beta=0.02mA$, $R_B=(5-0.7)/0.02=4.3/0.02=215K\Omega$.
\item 
When $\beta=50$:
\[	I_B=\frac{V_{CC}-0.7}{(\beta+1)R_C+R_B}=\frac{9.3}{51\times 2.5K+215}
	=0.027mA \]
\[ 	I_C=\beta I_B=1.36mA,\;\;\;V_C=V_{CC}-(\beta+1)I_B=6.6V \]
When $\beta=100$:
\[	I_B=\frac{V_{CC}-0.7}{(\beta+1)R_C+R_B}=\frac{9.3}{101\times 2.5K+215}
	=0.02mA \]
\[ 	I_C=\beta I_B=2 mA,\;\;\;V_C=V_{CC}-(\beta+1)I_B=5V \]
When $\beta=200$:
\[	I_B=\frac{V_{CC}-0.7}{(\beta+1)R_C+R_B}=\frac{9.3}{201\times 2.5K+215}
	=0.013mA \]
\[ 	I_C=\beta I_B=2.6mA,\;\;\;V_C=V_{CC}-(\beta+1)I_B=3.5V \]
\end{itemize}

\item {\bf Problem 3. (40 points)} 

\htmladdimg{../midterm3a.gif}

The circuit shown below is a silicon transistor amplifier which takes one
input and generates two outputs. Assume $V_{CC}=20V$, $R_1=20K\Omega$,
$R_2=10K\Omega$, $R_C=R_E=500\Omega$, $\beta=100$. 

\begin{enumerate}
\item Find $V_B$, $I_B$, $V_E$ and $V_C$, and the DC operating point in 
terms of $I_C$ and $V_{CE}$. 
\item In the figure provided, draw the load line, indicate the DC operating 
point, and find the corresponding $I_C$ and $V_{CE}$.
\item If the input voltage is such that it produces an AC component of the 
base current:
\[	i_b(t)=0.1 cos(\omega t) \; mA	\]
give the expression of the AC component of the two output voltages $v_1(t)$ 
at the emitter and $v_2(t)$ at the collector, and sketch their waveforms in 
the SAME plot provided below, where $v_{in}(t)=cos(\omega t)$ is also ploted.
(No need to be to the scale vertically, but do pay attention to the time
scale.)
\end{enumerate}

\htmladdimg{../midterm3b.gif}
\htmladdimg{../midterm3c.gif}

{\bf Solution:} 
Apply Thevenin thm to base circuit to get $R_B=R_1 || R_2=6.7K$, $V_BB=6.7V$.
\[ I_B=\frac{V_{BB}-V_{BE}}{R_B+(\beta+1)R_E}=\frac{6V}{6.7K+101\times 0.5K}
	=0.105 mA \]
\[ I_C=I_E=10.5 mA \]
\[ V_E=0.5\times 10.5=5.25V,\;\;\;\; V_C=20-0.5\times 10.6=14.75V,\;\;\;\;
	V_{CE}=V_C-V_E=14.75-5.25=9.5V \]
\[ v_1(t)=-5 cos(\omega t) V,\;\;\;\;v_1(t)=5 cos(\omega t) V	\]

\item {\bf Problem 4. (20 points)} 
  \begin{itemize}
  \item Find expression for the voltage $V_L$ across the load $R_L$ in terms 
    of all parameters given in the circuit shown on the left of the figure 
    below.
  \item Find expression for the voltage $V'_L$ across the load $R_L$ in terms 
    of all parameters given in the circuit shown on the right of the figure 
    below, in which a voltage-amplification circuit, a buffer, is inserted 
    between the source and the load, characterized by three parameters: 
    (a) the input resistance $R_{in}$, (b) the output resistance $R_{out}$, 
    and (c) the over-circuit voltage gain $A_{oc}$. 
  \item Assuming $V_o=10\,V$, $R_o=6\,k\Omega$, $R_L=4\,k\Omega$, 
    $R_{in}=100\,k\Omega$, $R_{out}=1\,k\Omega$, and $A_{oc}=10$, find the 
    numerical value for $V_L$ and $V'_L$ in the two cases above.
  \end{itemize}
\htmladdimg{../SourceLoad.png}

{\bf Solution:}
\begin{itemize}
\item \[ V_L=V_o \frac{R_L}{R_o+R_L} \]
\item \[ V'_L=A_{oc}V_o \frac{R_{in}}{R_o+R_{in}}\frac{R_L}{R_{out}+R_L} \]
\item \[ V_L=V_o \frac{R_L}{R_o+R_L}=10\frac{4}{6+4}=4  \]
  \[ V'_L=A_{oc}V_o \frac{R_{in}}{R_o+R_{in}}\frac{R_L}{R_{out}+R_L} 
  =10\times 10\frac{100}{6+100}\frac{4}{1+4}=75.47 \]
\end{itemize}

\end{enumerate}

\end{document}

