\documentstyle[11pt]{article}
\usepackage{html}
\begin{document}
\begin{center}
{\Large \bf  Midterm Exam 3}
% ---- E84, Fall, 2004}
\end{center}

\begin{itemize}
\item Take home, open everything except discussion.
\item Mark your start and end times. Don't spend more than 3 hours.
\item Mark your name and question number clearly on top of each page.
  Indicate the total number of pages submitted.
\item When solving a problem, list all the steps. In each step, describe 
  what you are doing in English, then show the calculation and the 
  result of the step. 
%        A final answer, even if correct, without evidence of the steps 
%        leading to the answer will not receive credit.
\end{itemize}

\begin{enumerate}

\item {\bf Problem 1. (25 points)} 

Find the output voltage $V$ for each of the two circuits shown in the 
figure below, using the line at the bottom as the reference (ground).
Assume the voltage drop across a diode is 0.7 V when it is forward
biased.

\htmladdimg{../midterm3d.gif}

\item {\bf Problem 2. (25 points)} 

The circuit below shows a simple means for obtaining improved bias
stability of the DC operating point of the transistor. As always,
assume $V_{BE}=0.7V$ when answering the following questions.

\htmladdimg{../midterm3e.gif}

\begin{itemize}
\item Explain qualitatively what happens if $I_C$ tends to rise as a
  result of an increased $\beta$.
\item Derive an expression for $I_C$ in terms of $R_B$, $R_C$, $\beta$
  and $V_{CC}$.
\item Find an approximation of the expression of $I_C$ when $\beta$ is 
  large enough, so that $I_C$ independent of $\beta$. In this case, how 
  are $R_C$ and $R_B$ related?
\item Find $R_C$ and $R_B$ so that the DC operating point is $V_{CE}=5V$ 
and $I_C=2mA$, when $\beta=100$ and $V_{CC}=10V$. 
\item Find $V_{CE}$ and $I_C$ for $\beta=50$, $\beta=100$, and $\beta=200$
	based on the resistances found above.
\end{itemize}

\item {\bf Problem 3. (25 points)} 

\htmladdimg{../midterm3a.gif}

The circuit shown below is a silicon transistor amplifier which takes one
input and generates two outputs. Assume $V_{CC}=20V$, $R_1=20K\Omega$,
$R_2=10K\Omega$, $R_C=R_E=500\Omega$, $\beta=100$. 

\begin{itemize}
\item Find $V_B$, $I_B$, $V_E$ and $V_C$, and the DC operating point in 
terms of $I_C$ and $V_{CE}$. 
\item In the figure provided, draw the load line, indicate the DC operating 
point, and find the corresponding $I_C$ and $V_{CE}$.
\item If the input voltage is such that it produces an AC component of the 
base current:
\[	i_b(t)=0.1 cos(\omega t) \; mA	\]
give the expression of the AC component of the two output voltages $v_1(t)$ 
at the emitter and $v_2(t)$ at the collector, and sketch their waveforms in 
the SAME plot provided below, where $v_{in}(t)=cos(\omega t)$ is also plotted.
(No need to be to the scale vertically, but do pay attention to the time
scale.)
\end{itemize}

\htmladdimg{../midterm3b.gif}
\htmladdimg{../midterm3c.gif}

 {\bf Solution:} 
 Apply Thevenin theorem to base circuit to get $R_B=R_1 || R_2=6.7K$, $V_BB=6.7V$.
 \[ I_B=(V_{BB}-V_{BE})/(R_B+(\beta+1)R_E)=6V/(6.7K+101\times 0.5K)=0.106 mA \]
 \[ I_C=I_E=10.6 mA \]
 \[ V_E=0.5\times 10.6=5.3V,\;\;\;\; V_C=20-0.5\times 10.6=14.7V,\;\;\;\;
 	V_{CE}=V_C-V_E=14.7-5.3=9.4V \]
 \[ v_1(t)=-5 cos(\omega t) V,\;\;\;\;v_1(t)=5 cos(\omega t) V	\]


\item {\bf Problem 4. (25 points)} 
  Answer the following questions regarding the circuits shown in the figure
  below, where $V_o=1\,V$, $R_o=6\,k\Omega$, $R_L=4\,k\Omega$, 
  $R_{in}=1\,M\Omega$, $R_{out}=0.1\,k\Omega$, and $A_{oc}=10$.
  \begin{itemize}
  \item Represent the voltage $V_L$ across the load $R_L$ in terms of all 
    of the parameters given in the circuit shown on the left of the figure. 
    Then obtain the numerical value of $V_L$ by substituting the specific 
    values of the parameters into the expression.
  \item Repeat the above for the circuit shown on the right of the figure,
    in which a voltage-amplification circuit, a buffer, is inserted between 
    the source and the load, characterized by three parameters: (a) the
    input resistance $R_{in}$, (b) the output resistance $R_{out}$, and 
    (c) the open-circuit voltage gain $A_{oc}$. 
  \end{itemize}
\htmladdimg{../SourceLoad.png}

{\bf Solution:}
\begin{itemize}
\item \[ V_L=V_o \frac{R_L}{R_o+R_L} \]
\item \[ V'_L=A_{oc}V_o \frac{R_{in}}{R_o+R_{in}}\frac{R_L}{R_{out}+R_L} \]
\item \[ V_L=V_o \frac{R_L}{R_o+R_L}=\frac{4}{6+4}=0.4  \]
  \[ V'_L=A_{oc}V_o \frac{R_{in}}{R_o+R_{in}}\frac{R_L}{R_{out}+R_L} 
  =10\frac{1000}{6+1000}\frac{4}{0.1+4}=9.7 \]
\end{itemize}

\end{enumerate}

\end{document}

