\documentstyle[11pt]{article}
\usepackage{html}
\begin{document}
\begin{center}
{\Large \bf  Final Exam ---- E84, Fall, 2005}
\end{center}

\begin{itemize}
\item Take home, open everything except discussion.
\item Mark your start and end times. % Don't spend more than 3 hours.
\item Due Thursday 5pm.
\item Mark your name and question number clearly on top of each page.
	Indicate the total number of pages submitted.
\item When solving a problem, itemize all the steps. In each step, 
  describe what you are doing, then show the calculation and the result 
  of the step. A final answer, even if correct, without evidence of the
  steps leading to the answer will not receive credit.
\end{itemize}

\begin{enumerate}

\item {\bf Problem 1. (30 points)} 

Assume $V_{CC}=12V$, $\beta=100$, $V_{BE}=0.7V$ when the base-emitter PN 
junction is forward biased. 
\begin{enumerate}
\item What value should $R_B$ be to achieve the desired $I_B=0.02mA$?
With this $I_B$, what value should $R_C$ be to achieve the desired output
voltage $V_{CE}=6V$ (in the middle of the linear range of the output 
characteristic plot)?
\item If the measured effective (RMS) voltage of the output $v_{out}(t)$ 
  (the AC component of $V_{CE}$) is $0.6 V$ and the voltage gain of the
  circuit is known to be $A_V=-200$ (negative sign for 180 degree phase 
  shift), what is the effective voltage of the input $v_{in}(t)$ (AC
  component of $V_{BE}$)? What is the input resistance of the transistor
  $r_{be}=\Delta V_{BE}/\Delta I_B$, the reciprical of the slope of the 
  input characteristic plot, at the current DC operating point $V_{BE}=0.7$?
\end{enumerate}

\htmladdimg{../final05f_a.gif}

% {\bf Solution:} $(V_{CC}-V_{BE})/I_B=11.7/0.02=565K\Omega$. For 
% $V_{CE}=6V$, the corresponding $R_C$ can be found to be $R_C=3K\Omega$
% by solving $I_{C}=(V_{CC}-V_{CE})/R_C=6/R_C=\beta I_B=2 mA$.

% If $A_V=-200$ and AC component of $v_{out}$ is $0.6 V$, the AC component 
% of $V_{in}$ is $0.6/200=3 mV$. The corresponding AC component of $I_C$ 
% can be found as $0.6 V/3K\Omega=0.2 mA$, and the AC component of $i_b$
% is $i_b=i_c/\beta=0.002 mA$. Therefore the slope of the input 
% characteristic curve is $0.002 mA/3 mV$, i.e., the equivalent resistance
% $r_{be}$ is $3 mV/0.002 mA=1.5 K\Omega$.


\item {\bf Problem 2. (35 points)} 

A transistor circuit and the output characteristic plot of the transistor 
in the circuit are shown below. Assume $R_B=15 K\Omega$, $R_C=3 K\Omega$, 
and $V_{BB}=1 V$. Assume $V_{BE}=0.7 V$ when the base-emitter PN-junction 
is forward biased, and $V_{CE}=0.3 V$ when the transistor is saturated. 

\begin{itemize}
\item When $V_{CC}=12 V$, find the operating point in terms of $V_C$ 
  and $I_C$, and the dynamic range of output voltage, i.e., the 
  peak-to-peak value of an AC voltage without distortion. 

\item Then answer the same questions above assuming $V_{CC}=6 V$. 
\end{itemize}

\htmladdimg{../final05f_b.gif}
\htmladdimg{../final05f_c.gif}

% {\bf Solution:} $I_B=(1-0.7)V /15 K\Omega=20 \mu A$, $I_C=2 mA$, 
% When $V_{CC}=12 V$, $V_C=V_{CC}-R_C I_C=12V-3 K\Omega\; 2 mA=6 V$. 
% $2(6-0.3)=2\times 5.7=11.4V$.

% When $V_{CC}=6 V$, $V_C=V_{CC}-R_C I_C=6V-1.5 K\Omega 2 mA=3 V$. 
% $2(3-0.3)=2\times 2.7=5.4V$.

\item {\bf Problem 3. (35 points)}

In the transistor amplification circuit shown below, $V_{CC}=12V$, 
$\beta=100$. Assume the AC component of the input current is 
$i_b(t)=0.05\cos \omega t mA$. Sketch the output characteristic plot 
of the transistor circuit with the load line, and the current $i_c(t)$,
output voltage $v_{out}(t)$ on the same plot, for each of the 
following three cases:
\begin{enumerate}
\item $R_B=110 K\Omega$, $R_C=1.2 K\Omega$;
\item $R_B=110 K\Omega$, $R_C=0.6 K\Omega$;
\item $R_B=220 K\Omega$, $R_C=1.2 K\Omega$.
\end{enumerate}
Commont on each of the three cases in terms of both amplification and 
distortion.

\htmladdimg{../final05f_d.gif}

% {\bf Solution:}

% \begin{enumerate}
% \item $R_B=110 K\Omega$, $R_C=1.2 K\Omega$; 

% The two points of the load line are:
% $(I_C=0, V_{CE}=V_{CC}=12V)$ and $(I_C=V_{CC}/R_C=10 mA, V_{CE}=0)$.
% The DC component of base current is $(12-0.7)/110=0.1 mA$, and the
% correspondingly $i_c(t)=10 mA$. The corresponding output voltage is 
% $v_c=V_{CC}-R_C i_c=12 - 1.2\items 10=0 V$, the transistor is in the
% saturation region of its output characteristic plot. 

% At the negative peak of input current $i_b=0.05 mA$, 
% $i_c=\beta i_b=5 mA$ and $v_c=V_{CC}-R_C i_c=12-1.2\times 5=6V$. 

% At the positive peak of input current $i_b=0.15 mA$, 
% $i_c=\beta i_b=15 mA$ and $v_c=V_{CC}-R_C i_c=12-1.2\times 15=-6V$, 
% indicating the transistor is deeply saturated, the actually $v_c=0.2$.

% \item $R_B=110 K\Omega$, $R_C=0.6 K\Omega$;
% The two points of the load line are:
% $(I_C=0, V_{CE}=V_{CC}=12V)$ and $(I_C=V_{CC}/R_C=20 mA, V_{CE}=0)$.
% The DC component of $i_b$ and $i_c$ are the same as above.
% When $i_b=0.1 mA$ and $i_c=10 mA$, the corresponding output voltage
% is $v_c=V_{CC}-R_C i_c=12 - 0.6\items 10=6 V$, in the middle of the linear
% region of the output characteristic plot.

% At the negative peak of input current $i_b=0.05 mA$, 
% $i_c=\beta i_b=5 mA$ and $v_c=V_{CC}-R_C i_c=12-0.6\times 5=9V$. 

% At the positive peak of input current $i_b=0.15 mA$, 
% $i_c=\beta i_b=15 mA$ and $v_c=V_{CC}-R_C i_c=12-0.6\times 15=3V$. 

% \item $R_B=220 K\Omega$, $R_C=1.2 K\Omega$.

% The two points of the load line are:
% $(I_C=0, V_{CE}=V_{CC}=12V)$ and $(I_C=V_{CC}/R_C=10 mA, V_{CE}=0)$.
% The DC component of base current is $(12-0.7)/220=0.05 mA$, and the
% correspondingly $i_c(t)=5 mA$. The corresponding output voltage is 
% $v_c=V_{CC}-R_C i_c=12 - 1.2\items 5=6 V$, the transistor is in the
% middle of the linear region of its output characteristic plot. 

% At the negative peak of input current $i_b=0.05 mA$, 
% $i_c=\beta i_b=0 mA$ and $v_c=V_{CC}-R_C i_c=12V$. 

% At the positive peak of input current $i_b=0.05 mA$, 
% $i_c=\beta i_b=10 mA$ and $v_c=V_{CC}-R_C i_c=12-1.2\times 10=0V$. 

% \end{enumerate}
% Case 1 has severe distortion, case 3 has better amplification without
% distortion.


\end{enumerate}

\end{document}

