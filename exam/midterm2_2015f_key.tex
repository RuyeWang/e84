\documentstyle[11pt]{article}
\usepackage{html}
\begin{document}
\begin{center}
{\Large \bf E84 Additional practice problems}
\end{center}
\begin{enumerate}

\item A crystal oscillator is an electronic oscillator circuit that uses
  the mechanical resonance of a vibrating crystal of piezoelectric material
  to create an electrical signal with a precise frequency. The crystal can
  be modeled by the RLC circuit shown in the figure. Assuming $R$ can be
  approximated to be zero.
  \begin{itemize}
    \item Find the parallel resonant frequency $\omega_p$ at which the 
      impedance of the model is maximized.    
    \item Find the series resonant frequency $\omega_s$ at which the 
      impedance of the model is minimized.
  \end{itemize}

  \htmladdimg{../../lectures/figures/Problems0.png}

  {\bf Solution} The total admittance is
  \begin{eqnarray}
    Y(\omega)&=&\frac{1}{j\omega L+1/j\omega C}+j\omega C_0
    =\frac{j\omega C}{1-\omega^2LC}+j\omega C_0
    =\frac{j\omega C+j\omega C_0(1-\omega^2LC)}{1-\omega^2LC}
    \nonumber
  \end{eqnarray}
  \begin{itemize}
    \item The magnitude of $Y(\omega)$ is minimized to zero if
      \[
      j\omega C+j\omega C_0(1-\omega^2LC)=0
      \]
      i.e.,
      \[
      \omega^2=\frac{C+C_0}{L(CC_0)},\;\;\;\;\;\mbox{or}
      \;\;\;\;\omega=\frac{1}{{\sqrt{LC_p}}
      \]
      where 
      \[ 
      C_p=C||C_0=\frac{CC_0}{C+C_0}
      \]
      This is the parallel resonant frequency.
    \item The magnitude of $Y(\omega)$ is maximized to infinity if
      $\omega=1/\sqrt{LC}$ and the denominator becomes zero. This is
      the series resonant frequency.
  \end{itemize}

\item In the circuit below, $v_S(t)=A\cos(10^6 t)$ with some unknown
  peak value $A$, $R=10\Omega$, and $L=10\,\mu H$. The the RMS value
  of $v_2$ acorss $R_2$ is measured to be 10 V. It is also known that
  $v_s(t)$ and $i_s(t)$ to be in phase. 
  \begin{itemize}
  \item Find $C$.
  \item Find the RMS values of $v_C(t)$ and $v_L(t)$.
  \item Find the RMS values of $v_{ab}$ and $v_{RC}=V_{RL}$.
  \item Find the peak value $A$ of $v_s(t)$.
  \end{itemize}

  \htmladdimg{../../lectures/figures/Problems9.png}

  {\bf Solution} 


  \htmladdimg{../../lectures/figures/Problems8.png}

  As $v_s(t)$ and $i_s(t)$ are in phase (zero angular difference), 
  the admittance of the parallel combinatiion of the RL and RC 
  branches is real with imaginary part equal to zero:
  \[
  Y(\omega)=Y_{RL}(\omega)+Y_{RC}(\omega)
  =\frac{1}{R_1+j\omega L}+\frac{1}{R_2+1/j\omega C}
  =\frac{1}{10+j10^6 \times 10^{-5}}+\frac{1}{10-j/10^6C}
  \]
  For $Im(Y)=0$, we need to have $Z_C(\omega)=1/\omega C=10$, i.e., 
  \[
  C=\frac{1}{10\omega}=(10\times 10^6)^{-1}=10^{-7} =0.1 \mu F
  \]
  The impedance of the parallel combination of the RL and RC branches
  is
  \[
  Z(\omega)=\frac{1}{Y(\omega)}=Z_{RL}(\omega)||Z_{RC}(\omega)
  =\frac{(10+j10)(10-j10)}{10+j10+10-j10}
  =10
  \]
  As $|R|=|Z_C|=10 \Omega$, $|\dot{V}_C|=|\dot{V}_R|=10V$. But as 
  they are $\pi/2$ apart in phase, we have $|\dot{v}_{RC}|=10\sqrt{2}$.
  We also see that $|\dot{V}_{RL}|=10\sqrt{2}$. However, their 
  phase difference is $\pi/2$, and $|\dot{V}_{ab}|=10\sqrt{2}$. 

  The currents through RC and RL branches are:
  \[
  |\dot{I}_{RC}|=\frac{|\dot{V}_{RC}|}{|Z_{RC}|}=\frac{10\sqrt{2}}{\sqrt{10^2+10^2}}
  =1,
  \;\;\;\;\;\;
  |\dot{I}_{RL}|=\frac{|\dot{V}_{RL}|}{|Z_{RL}|}=\frac{10\sqrt{2}}{\sqrt{10^2+10^2}}
  =1
  \]
  But their phase difference is $\pi/2$, we have
  \[
  |\dot{I}_s|=|\dot{I}_{RL}+\dot{I}_{RC}|=\sqrt{2}
  \]
  The voltage across $R_1$ is $V_1=RI_s=10\sqrt{2}$, and
  \[
  \dot{V}_s=\dot{V}_1+\dot{V}_{cd}=10\sqrt{2}+10\sqrt{2}=20\sqrt{2}
  \]
  The peak value is therefore $A=\sqrt{2} V_s=40\,V$
  
  
\item In the circuit $V=9\,V$, $R_1=10\,k\Omega$, $R_2=R_3=3\,k\Omega$, 
  $R_4=1.5\,k\Omega$, $L=100\,mH$ and $C=1\,\mu F$. Find $i_L(t)$,
  $v_L(t)$, $i_C(t)$, and $v_C(t)$ after the switch is closed at $t=0$. 
  (Assume the circuit is in steady state before the switch is closed.)

  \htmladdimg{../../lectures/figures/Problems3.png}

  {\bf Solution}
  \[
  i_L(0)=\frac{9}{3||3+1.5}\frac{3}{3+3}=1.5 \,mA,
  \;\;\;\;\;\;\;  i_L(\infty)=9/1.5=6\,mA
  \]
  \[
  \tau_{RL}=L/R=L/(R_3||R_4)=10^{-1}/10^3=10^{-4}
  \]
  \[
  i_L(t)=i_L(\infty)+[i_L(0)-i_L(\infty)]e^{-t/\tau_{RL}}}=6-4.5 e^{-1000t} mA
  \]
  \[
  v_L(t)=L\frac{d}{dt}i_L(t)=0.1\times 10^4\,4.5 e^{-10000t} mV =4.5 e^{-10000t} V
  \]

  \[
  v_C(0)=4.5\,V,\;\;\;\;v_C(\infty)=0
  \]

  \[
  \tau_{RC}=RC=10^3\times 10^{-6}=10^{-3}
  \]
  
  \[
  v_C(t)=v_C(\infty)+[v_C(0)-v_C(\infty)]e^{-t/\tau_{RC}}=4.5 e^{-100 t}V
  \]
  \[
  i_C(t)=C\frac{d}{dt}v_C(t)=10^{-6}\,(-100)\, 4.5e^{-100t} A=-0.45 e^{-100t}\,mA
  \]

\item In the circuit below, the filter composed of $L$, $C_1$ and $C_2$
  between the source $v(t)$ and the load $R_L=100\Omega$ is to pass the 
  fundamental frequency $\omega_0=1000$ without attenuation but completely 
  block the 2nd harmonic $2\omega_0=2000$. Given $L=25\,mH$, find $C_1$
  and $C_2$.

  \htmladdimg{../../lectures/figures/Problems2.png}

  {\bf Solution} Find the total impedance of the filter branch:
  \begin{eqnarray}
    Z(\omega)&=&\frac{j\omega L/j\omega C_1}{j\omega L+1/j\omega C_1}+\frac{1}{j\omega C_2}
    =\frac{j\omega L}{1-\omega^2LC_1}-\frac{j}{\omega C_2}
    \nonumber\\
    &=&j\frac{\omega^2LC_2+\omega^2LC_1-1}{(1-\omega^2LC_1)\omega C_2}
    =j\frac{\omega^2L(C_1+C_2)-1}{(1-\omega^2LC_1)\omega C_2}
    \nonumber
  \end{eqnarray}
  When the denominator is zero, i.e., $\omega^2=1/LC_1$, we get 
  $|Z(\omega)|=\infty$, the filter is an open circuit. Therefore, to 
  completely block $2\omega_0=2000$, $C_1$ needs to satisfy:
  \[
  \frac{1}{\sqrt{LC_1}}=2\times 10^3,\;\;\;\;\mbox{i.e.,}\;\;\;\;
  C_1=\frac{1}{4\times 10^6 \times 25\times 10^-3}=10^{-5}\,F=10\,\mu F
  \]
  When the numerator is zero, i.e., $\omega^2L(C_1+C_2)=1$,  $|Z(\omega)|=0$,
  the filter is a short circuit. Therefore, to pass $\omega_0=1000$ without
  attenuation, $C_1+C_2$ nees to satisfy
  \[
  C_1+C_2=\frac{1}{\omega_0^2L}=\frac{1}{10^6\times 25\times 10^{-3}}=40\times 10^{-6}=40\,\mu F
  \]
  i.e.,
  \[
  C_2=40\,\mu F-C_1=40\,\mu F-10\,\mj F=30\,\mu F
  \]

\item In the circuit below, $R_1=3\,k\Omega$, $R_2=2\,k\Omega$, $C=1\,\mu F$, 
  $L=10\,mH$, $I_0=5\,mA$,and $v_0(t)=10\sin(10^4t)\,V$. Find the 
  current $i(t)$ through $R_2$.

  \htmladdimg{../../lectures/figures/Problems5.png}
    
  {\bf Solution} Use superposition. First let $v_0(t)=0$. 
  \[
  i'(t)=\frac{R_1}{R_1+R_2}I_0=\frac{2}{2+3}\;3\,mA=3\,mA
  \]
  Next let $I_0=0$ (open-circuit). 
  \[
  \frac{1}{\omega C}=\frac{1}{10^4\times 10^{-6}}=10^2,
  \;\;\;\;\;
  \omega L=10^4\times 10^{-2}=10^2
  \]
  As $j\omega L-1/j\omega C=0$, the $L$ and $C$ form a series resonant
  circuit with zero impedance, we have
  \[
  i''(t)=\frac{v_0(t)}{R_2}=5\,\sin(10^4t)\,mA
  \]
  \[
  i(t)=i'(t)+i''(t)=3+5=8\,mA
  \]

\item In the circuit below, $v_S(t)=A\cos(10^4 t)$, $R_1=R_2=R_3=20\,\Omega$.
  It is also known that the RMS values of the three voltages $v_1(t),\,v_2(t)$ 
  and $v_3(t)$ across respectively $R_1$, $R_2$ and $R_3$ are the same: 
  $V_1=V_2=V_3=200\,V$. Find:
  \begin{itemize}
    \item $v(t)$ across the RL and RC branches
    \item voltages $v_L(t)$ and $v_C(t)$ accross $C$ and $L$, respectively,
    \item currents $i_L(t)$ and $i_C(t)$ through $C$ and $L$, respectively,
    \item the RMS value of $v_S(t)$ across and the RMS value of $i(t)$ through 
      the voltage source,
    \item $C$ and $L$.
  \end{itemize}

  \htmladdimg{../../lectures/figures/Problems5.png}
    
  {\bf Solution} 
  
  As $V_1=V_2=V_3=200\,V$, the three currents through $R_1$, $R_2$ and 
  $R_3$ are also equal:
  \[
  I_1=I_2=I_3=\frac{200\,V}{20\,\Omega}=10\,A
  \]
  But as $\dot{I}_1=\dot{I}_2+\dot{I}_3$, they form an equilateral trangle:
  \[
  \dot{I}_1=10\angle 0,\;\;\;\;\;
  \dot{I}_2=\frac{\dot{V}}{R+jX}=10\angle (-\pi/3),\;\;\;\;\; 
  \dot{I}_3=\frac{\dot{V}}{R-jX}=10\angle (\pi/3),
  \]
  where $\dot{V}$ is the phasor voltage across the RC and RL branches.
  The phasor voltages across the three resistors $R_1$, $R_2$, and $R_3$
  are respectively:
  \[
  \dot{V_2}=R\dot{I}_2=200\angle (-\pi/3),\;\;\;\;\; 
  \dot{V_3}=R\dot{I}_3=200\angle (\pi/3),\;\;\;\;
  \dot{V_1}=R\dot{I}=200
  \]

  Also, as $I_2=I_3$, we have
  \[
  I_2=|\dot{I}_2|=\bigg|\frac{\dot{V}_{LC}}{R_2+j\omega L}\bigg|
  =I_3=|\dot{I}_3|=\bigg|\frac{\dot{V}_{LC}}{R_3+1/j\omega C}\bigg|
  \]
  or
  \[
  R_2^2+(\omega L)^2=R_3^2+(1/\omega C)^2
  \]
  but as $R_2=R_3$, we must have $\omega L=1/\omega C$, and
  \[
  V_L=I_2\omega L=I_3/\omega C=V_C,\;\;\;\;\;\;\mbox{or}\;\;\;\;\;
  \omega L=1/\omega C=X
  \]
  The phasor representations of these voltages are shown below:  

  \htmladdimg{../../lectures/figures/Problems6.png}

  As $\dot{V}_2+\dot{V}_L=\dot{V}$ and $\dot{V}_3+\dot{V}_C=\dot{V}$,
  also as $\dot{V}_L$ is leading $\dot{V}_2$ by $\pi/2$ and
  $\dot{V}_C$ is lagging $\dot{V}_3$ by $\pi/2$, we also get
  \[
  \dot{V}_L=200\sqrt{3}\angle(\pi/6),\;\;\;\;
  \dot{V}_C=200\sqrt{3}\angle(-\pi/6)
  \]
  so that
  \[
  \dot{V}_2+\dot{V}_L=\dot{V}_3+\dot{V}_C=\dot{V}=\sqrt{200^2+(200\sqrt{3})^2}=400
  \]
  In time domain, we have:
  \[
  i_2=i_L=10\sqrt{2}\cos(\omega t-\pi/3),\;\;\;\;
  v_2=200\sqrt{2}\cos(\omega t-\pi/3),\;\;\;\;
  v_L=200\sqrt{6}\cos(\omega t+\pi/6)
  \]
  \[
  i_3=i_C=10\sqrt{2}\cos(\omega t+\pi/3),\;\;\;\; 
  v_3=200\sqrt{2}\cos(\omega t+\pi/3),\;\;\;\;
  v_C=200\sqrt{6}\cos(\omega t-\pi/6)
  \]
  \[
  i(t)=10\sqrt{2}\cos(\omega t),\;\;\;\;\;v(t)=400\sqrt{2}\cos(\omega t)
  \]
  The RMS value of $v_S(t)$ is 
  \[
  V_S=V_1+V=200+400=600\,V
  \]
  \]
  \[
  X_L=\big|j\omega L\big|=\bigg|\frac{\dot{V}_L}{\dot{I}_2}\bigg|=\frac{200\sqrt{3}}{10}=20\sqrt{3}\;\Omega,
  \;\;\;\;\;\;
  X_C=\bigg|\frac{1}{j\omega C}\bigg|=\bigg|\frac{\dot{V}_C}{\dot{I}_3}\bigg|=20\sqrt{3}\;\Omega
  \]
  As $\omega=10^4$, we have
  \[
  L=\frac{X}{\omega}=2\sqrt{3}\times 10^{-3}=3.46\,mH,\;\;\;\;
  C=\frac{1}{\omega X}=\frac{10^{-4}}{20\sqrt{3}}=\frac{10^{-5}}{2\sqrt{3}}
    =2.89\times 10^{-6}=2.89\,\mu F
  \]

\item  In the circuit below, $R_1=3\Omega$, $R_2=6\Omega$, $R_3=2\Omega$, 
  $L=0.5\,H$, $V_1=6V$, $V_2=3V$. The circuit is in steady state when 
  $t<0$. Find current $i_3(t)$ through $R_3$ when switch is closed at
  $t=0$.

  \htmladdimg{../../lectures/figures/Problems7.png}

  {\bf Solution:} Current through $L$ at $t=0$ is
  \[
  i_L(0^-)=i_L(0^+)=3/6=0.5\,A
  \]
  We have $\tau=L/(R_1||R_2||R_3)=0.5/(3||6||2)=0.5\,s$ and $i_3(\infty)=0$.
  To find $i_3(0^+)$ after the switch is closed at $t=0$, we use 
  superposition
  \begin{itemize}
  \item $V_1=6V$ alone: Total current through $V_1$:
    \[
    \frac{V_1}{R_1+R_2||R_3}=\frac{6}{3+1.5}=\frac{4}{3}
    \]
    By current divider:
    \[
    i'_3(0^+)=\frac{4}{3}\;\frac{2}{2+6}=\frac{4\times 6}{3\times 8}=1\;\;\;\;\;\mbox{(down)}
    \]
  \item $V_2$ alone: 
    \[
    i''_3(0^+)=\frac{V_2}{R_1||R_2+R_3}=\frac{3}{2+2}=\frac{3}{4}=0.75\;\;\;\;\;\mbox{(down)}
    \]
  \item $i_L(0^+)$ alone:
    \[
    i'''_3(0^+)=i_L(0^+)\frac{R_1||R_2}{R_3+R_1||R_2}=0.5\frac{2}{2+2}=0.25\;\;\;\;\;\mbox{(up)}
    \]
  \end{itemize}
  \[
  i_3(0^+)=i'_3(0^+)+i''_3(0^+)+i'''_3(0^+)=1+0.75-0.25=1.5\,A
  \]
  \[
  i_3(t)=0-(1.5-0)e^{-t/\tau}=1.5e^{-2t}\,A
  \]

  Alternative method: $i_L(0)=0.5\,A$, applying KCL to the middle point a:
  \[
  \frac{V_a}{R_3}+\frac{V_a-V_1-V_2}{R_1}+\frac{V_a-V_2}{R_2}+i_L
  =\frac{V_a}{2}+\frac{V_a-9}{3}+\frac{V_a-3}{6}+0.5=0
  \]
  Solving this we get $V_a=3\,V$, and $i_3(0^+)=V_a/R_3=3/2=1.5\,A$, same
  as above.

\end{enumerate}
\end{document}

