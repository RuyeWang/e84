\documentstyle[12pt]{article}
\usepackage{html}
% \usepackage{graphics}  
\begin{document}

\section*{E84 Lab 1: Multimeter}

%{\bf Design on paper due 2/27 Monday in class}

%{\bf Final project due by 3/5 Monday in lab}

\begin{itemize}

  \item Study the schematic diagram of a multimeter shown in the figure
    below carefully to understand how the multimeter measures both DC and
    AC voltages, DC currents, and resistances at each of the positions of 
    the rotary switch. 

  \item R24 is a potentiometer, a resistor of $10\;K\Omega$ but with an 
    additional terminal which is a piece of metal that can slide along the
    resistor, so that the resistance between the 3rd terminal and any of 
    the other two is variable between 0 and 10 $K\Omega$. See
    \htmladdnormallink{here}{http://en.wikipedia.org/wiki/Potentiometer}
    for more detailed explanation. This potentiometer is needed to calibrate
    the Ohmmeter. Specifically, before taking the measurement of a resistance,
    the potentiometer needs to be adjusted so that the needle points to 
    zero on the resistance scale when the two leads of the multimeter touch
    each other for zero resistance.
    
  \item Understand how the rotary switch works. It is represented by the 
    24 circles on top of the diagram and the two layers of rectangular shapes
    above. At each of the positions, the corresponding circle is connected to
    both of the rectangles, while all remaining circles are not connected to
    anything. If you still have difficulty understanding this, it should be
    most helpful if you take a look at the physical parts and the printed 
    circuit board (PCB) in the kit.

  \item Do all parts specified below. Turn in your answer on paper with all 
    the resistance values before the first deadline. 

  \item Assemble the multimeter in the lab, verify every single position 
    to see if the meter works as expected.
    
\end{itemize}

\htmladdimg{../M1250Schematic1a.png}



\begin{enumerate}
\item {\bf Meterhead assembly:}

  \begin{itemize}

  \item 
    When a voltage of $0.1\;V$ or a current of $50\;\mu A$ is applied to 
    the position marked ``$0.1V/50\mu A$'', the meter shows a full scale display.
    Find the total resistance of the internal circuitry in the multimeter at this
    position. This information is necessary for understanding the DC voltage and
    current measurement.

%    {\bf Solution:} $R_M=0.1\;V/40\;\mu A=2\;k\Omega$

  \item 
    Find the needed voltage and current for a full scale display of the meter
    head assembly (the meter head in series with $R8=680\;\Omega$, in parallel 
    with both $R26=31\;k\Omega$ and $R24+R25=10k+18k=28\;k\Omega$, excluding
    $R7=240\;\Omega$). 

    Also find the over all resistance $R_M$ of the meter head assembly. This 
    information is necessary for understanding the AC voltage and resistance 
    measurement.

    \begin{comment}
    {\bf Solution:} 

    The meter-head assembly excluding $R7=240$ is $2000-240=1760\;\Omega=1.76\;k\Omega$.

    As a current $I_M=0.05\;mA$ is needed for a full scale display, the 
    corresponding voltage across the meter-head assembly is 
    $V_M=I_M R_M=1.76\times 0.05=0.088\;V$. 
    \end{comment}

  \item 
    Find the resistance $R_m$ of the meter-head in series with R8 (excluding R31 
    and R24 + R25).

    \begin{comment}
    {\bf Solution:}
    \[
    \frac{1}{28}+\frac{1}{31}+\frac{1}{R_m}=\frac{1}{1.76},
    \;\;\;\;\;\;   R_m=2\;k\Omega
    \]
    \end{comment}
  \end{itemize}

\item {\bf DC Voltage measurement}
  
  Find the values of R1 through R6 for the 6 scales of DC voltage measurement
  so that these resistors will indeed allow a full display with $I_m=50\;\mu A$ 
  at the marked DC voltage for each of the 6 position. Note that $R13=3\;k\Omega$ 
  is in series with $R7=240\;\Omega$ and the meter-head assembly with total 
  resistance $R13+R7+R_M=5\;k\Omega$. 
  
  \begin{comment}
  
  {\bf Solution:}

  As shown in Figure 4, the circuit for each scale of the DC voltage measurement 
  is a voltage divider with an internal resistance for the meter-head:
  \begin{itemize}
  \item 0.5V
    \[
    0.5\;\frac{5}{R6+5}=0.25,\;\;\;\;\;\;\;R6=5\;k\Omega
    \]
  \item 2.5V
    \[
    2.5\;\frac{5}{R5+R6+5}=0.25,\;\;\;\;\;\;\;R5=40\;k\Omega
    \]
  \item 10V
    \[
    10\;\frac{5}{R4+R5+R6+5}=0.25,\;\;\;\;\;\;\;R4=150\;k\Omega
    \]
  \item 50V
    \[
    50\;\frac{5}{R3+R4+R5+R6+5}=0.25,\;\;\;\;\;\;\;R3=800\;k\Omega
    \]
  \item 250V
    \[
    250\;\frac{5}{R2+R3+R4+R5+R6+5}=0.25,\;\;\;\;\;\;\;R2=4\;M\Omega
    \]
  \item 1000V
    \[
    1000\;\frac{5}{R1+R2+R3+R4+R5+R6+5}=0.25,\;\;\;\;\;\;\;R1=15\;M\Omega
    \]
  \end{itemize}
  \end{comment}


\item {\bf AC Voltage measurement}

  Based on the given values of R14 through R17 for AC voltage measurement
  (Figure 5 of the manual of the kit) find the total resistance $R_{DM}$
  of the meter-head assembly in series with the diode D1. Further, find 
  the resistance $R_D$ and voltage drop $V_D$ of the diode D1 when the 
  meter-head has a full scale display. 

  Find the values of R14 through R17 so that they will indeed allow the 
  meter-head to have a full scale display (with an average DC current of 
  $I_m=50\;\mu A$ going through the meter-head), when the AC voltages of
  10V, 50V, 250V and 1000V (RMS values) are applied to the corresponding 
  scale positions.

  \begin{comment}
  {\bf Solution:}

  Pick any two of the four scales, e.g., 10V and 50V, and get
  \[
  10\;\frac{R_{AV}}{R17+R_{AV}}=50\;\frac{R_{AV}}{R17+R16+R_{AV}}
  \]
  i.e.,
  \[
  10\;\frac{R_{AV}}{83.3+R_{AV}}=50\;\frac{R_{AV}}{443.3+R_{AV}}
  \]
  Solving for $R_{AV}$ we get $R_{AV}=6.7\;k\Omega$. Alternatively, we can 
  pick another pair of 250V and 1000V, and get
  \[
  250\;\frac{R_{AV}}{R17+R16+R15+R_{AV}}=1000\;\frac{R_{AV}}{R17+R16+R15+R14+R_{AV}},
  \]
  i.e.,
  \[
  250\;\frac{R_{AV}}{2243.3+R_{AV}}=1000\;\frac{R_{AV}}{8993.3+R_{AV}}
  \]
  Solving for $R_{AV}$ we again get $R_{AV}=6.7\;k\Omega$. 
  
  As we know $R_M=1.76\;k\Omega$, the resistance of D1 must be 
  $R_D=6.7-1.76=4.94\;k\Omega$. Correspondingly, at the full display with
  $I_m=50\;\mu A$, the voltages across D1 is
  $V_D=I_m R_D=0.05\times 4.94=0.247\;V$. As $V_m=0.088\;V$, the total
  voltage across the meter-head and D1 is $V_D+V_m=0.335\;V$.

  Any AC voltage to be measured needs to be rectified (half-wave) and then
  converted to the average DC value. When the AC voltage is $1\;V$, its
  peak value is $V_p=\sqrt{2}=1.4142\;V$, the average DC value after
  half-wave rectification is 
  \[
  V_{AV}=\sqrt{2}\times \frac{2}{\pi}\times\frac{1}{2}
  =\frac{\sqrt{2}}{\pi}\approx\frac{1.4142}{3.1416}=0.4502\;V
  \]
  We can show that in each of the four AC voltage scales, the meter-head
  has a full scale display:
  \begin{itemize}
  \item 10V
    \[
    I_m=\frac{10\times 0.4502}{6.7+83.3}=\frac{4.5}{90}=0.05\;mA
    \]
  \item 50V
    \[
    I_m=\frac{50\times 0.4502}{6.7+83.3+360}=\frac{22.51}{450}=0.05\;mA
    \]
  \item 250V
    \[
    I_m=\frac{250\times 0.4502}{6.7+83.3+360+1800}=\frac{112.55}{2250}=0.05\;mA
    \]
  \item 1000V
    \[
    I_m=\frac{1000\times 0.4502}{6.7+83.3+360+1800+6750}=\frac{450}{9000}=0.05\;mA
    \]
  \end{itemize}
  \end{comment}

\item {\bf DC current measurement}

  The circuit shown in Figure 6 of the manual of the kit has some major
  mistake! Correct it.

  Find the values of R12 and R11 so that the meter-head will have a full
  scale display (with an current of $I_m=50\;\mu A$), when the DC currents 
  of 2.5 mA and 25 mA are applied to the corresponding scale positions.  

  Use the given values of R9=0.97 $\Omega$ and R10=0.04 $\Omega$, find current
  $I_m$ through the meter-head when the applied DC current is 10 A. For this
  current to be exactly $0.05\;\mu A$, what value should R10 take?
  
  \begin{comment}
  {\bf Solution:}

  The circuit for each scale of the DC current measurement is a current divider
  with an internal resistance for the meter-head:

  \begin{itemize}
  \item 2.5 mA
    \[
    I_m=2.5\;\frac{R12}{R12+5000}=0.05,\;\;\;\;\;\;\;R12=102
    \]
  \item 25 mA
    \[
    I_m=25\;\frac{R11}{R11+5000}=0.05,\;\;\;\;\;\;\;R11=10
    \]
  \item 250 mA
    \[
    I_m=250\;\frac{R9+R10}{R9+R10+5000}=0.05,\;\;\;\;\;\;\;\;\;R9+R10=1
    \]
  \end{itemize}
%  \[
%  I_m=10000\;\frac{0.04}{5000+0.97}=0.08\;mA
%  \]
  \[
  I_m=10000\;\frac{R10}{5000+0.97+R10}=0.05\;mA,\;\;\;\;\;
  R10=0.025\;\Omega
  \]
  \end{comment}

\item {\bf Resistance measurement}

  \begin{itemize}
  \item Before taking the measurement of a given external resistance $R$, 
    the multimeter needs to be calibrated by adjusting R24 (assume the bottom 
    part is x and the top part is 10-x), so that the meter-head shows a full 
    scale display, when $R=0$ (with the two leads short circuited). Determine 
    the value of this x. 

  \item Find the value of R19 for the scale of the $\times 1\;k\Omega$
    resistance measurement, so that when an external resistance $R=20\;k\Omega$
    being measured, the display of the meter is exactly halfway (in the middle),
    i.e., the current through the meter is $25\;\mu A$.

  \end{itemize}

  \begin{comment}
  {\bf Solution:}
  
  \begin{itemize}
  \item Calibration:

    The resistance of the meter-head $R_m$ in parallel with R26 is:
    \[
    R_m||R26=2||31=1.88\;k\Omega
    \]
    When $R=0$, the total load of the 3V battery is 
    \[
    R_t=44+(18+x)||(10-x+1.88)=44+\frac{(18+x)(11.88-x)}{(18+x)+(11.88-x)}
    =\frac{1528.5-6.12\;x-x^2}{29.88}
    \]
    The total current through the 3V battery is:
    \[
    I_t=\frac{3V}{R_t}=\frac{3\times 29.88}{1528.5-6.12\;x-x^2}
    \]
    The current through the meter in parallel with R26 needs to be 0.05 $\mu A$:
    \[
    I_m=I_t\;\frac{18+x}{29.88}
    =\frac{3\times 29.88}{1528.5-6.12\;x-x^2}\;\frac{18+x}{29.88}
    =\frac{3(18+x)}{1528.5-6.12\;x-x^2}=0.05
    \]
    Solving for a positive value of $x$ we get $x=6.2\;k\Omega$. Now we have
    $18+x=24.2$, $10-x=3.8$. 

    The total resistance is
    \[
    R_t=\frac{1528.5-6.12\;x-x^2}{29.88}=48.6\;k\Omega
    \]
    and the total current is
    \[
    I_t=\frac{3}{R_t}=\frac{3}{48.6}=0.0617
    \]

    \htmladdimg{../multimeter2a.png}

  \item $\times 1\;k\Omega$ scale, $R=20\;k\Omega$
  
    The total current through the battery is
    \[
    \frac{3}{R+R19||R_t}
    \]
    The current through $R23=44\;k\Omega$ (current divider) need to be $I_t/2$    
    \[
    \frac{3}{R+R19||R_t}\;\frac{R19}{R19+R_t}=\frac{I_t}{2}
    \]
    Substituting $R=20\;k\Omega$, $R_t=48.6\;k\Omega$ and $I_t=0.0617\;mA$, 
    and solving for $R19$ we get $R19=33.96\approx 34\;k\Omega$.    
  \end{comment}

\end{enumerate}

\end{document}




{\bf How to find the polarity of a diode? (not necessarily trivial!)}

When the voltage at the anode (labeled by a triangle) of a diode is higher
than that of the cathode, the diode has a very low resistance (close to a short
circuit, called forward biased). If the polarity is reversed, the resistance 
of the diode becomes very high (close to an open circuit, called reverse biased).
The polarity of the diode can therefore be found by checking its resistance 
using a multimeter.

However, we need to know which lead of the multimeter (when used as an ohmmeter) 
is positive and which is negative. In most analog multimeters, such as the one 
you are building, the positive lead (marked by +) is connected to the negative 
end of the internal battery, while the other lead (marked by -, or COM for common)
is connected to positive end of the battery. So if the measured resistance of the 
diode is low, the end connected to the COM lead of the multimeter is the anode
(triangle).

However, the polarities of digital multimeters may be the opposite, i.e., the 
lead marked by + may be connected to the positive end of the internal battery.
In this case, the polarity of the diode can be determined in the opposite way
compared to the method above using an analog multimeter. 

Moreover, some digital multimeters have a particular position for diode measurement. 

Read \htmladdnormallink{this page}{http://www.allaboutcircuits.com/vol_3/chpt_3/2.html}
for more detailed discussions.

{\bf How to solder?} 

If you have never done soldering before, you should find 
\htmladdnormallink{this}{http://fourier.eng.hmc.edu/e84/labs/N0SS_SolderNotesV6.pdf}
and \htmladdnormallink{this}{http://fourier.eng.hmc.edu/e84/labs/soldering_rev3.pdf}
useful.


  
\end{document}


	

	

