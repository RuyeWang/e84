\documentstyle[12pt]{article}
\usepackage{html}
\textwidth 6.0in
\topmargin -0.5in
\oddsidemargin -0in
\evensidemargin -0.5in
% \usepackage{graphics}  
\begin{document}

\section*{E84 Lab 2: Construction of a Radio Receiver}

%{\bf Due 5 pm, last day of class (Friday, May 1)}

The schematic diagram of a transistor radio receiver is shown in the figure
below:
\htmladdimg{../radio.gif}

\begin{itemize}
\item Pick up the hardware of the radio receiver kit from the stock room,
  including all the components and the construction manual.



\item Study the diagram above carefully to understand how the receiver
  works, specifically, superheterodyne receiver, frequency mixer, push-pull
  output amplifier, etc. As we do not have enough time in class for in-depth
  discussions of these concepts, refer to the following web sites for to gain
  a better understanding:
  \begin{itemize}
    \item \htmladdnormallink{Radio Receiver}{http://en.wikipedia.org/wiki/Receiver_(radio)}
    \item \htmladdnormallink{http://en.wikipedia.org/wiki/Superheterodyne_receiver}{Superheterodyne receiver}
    \item \htmladdnormallink{Superheterodyne Receiver}{http://en.wikipedia.org/wiki/Superheterodyne_receiver}
    \item \htmladdnormallink{Frequency Mixer}{http://en.wikipedia.org/wiki/Frequency_mixer}
    \item \htmladdnormallink{Rectification}{http://en.wikipedia.org/wiki/Rectifier}
    \item \htmladdnormallink{Push-Pull Amplifier}{http://en.wikipedia.org/wiki/Push-pull_output}
  \end{itemize}
    
\item Predict the DC operating point $(I_c, V_{ce})$ and $V_b$ and $V_e$ for 
  transistors $Q_3$ and $Q_4$.

\item Construct the entire circuit. Verify the predicted values in the previous 
  part by actual measurement of the circuit.

\end{document}


	

	

