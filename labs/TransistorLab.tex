\documentstyle[12pt]{article}
\usepackage{html}
% \usepackage{graphics}  
\begin{document}

{\Large \bf E84 Lab: Transistor}

%{\bf Final construction due 5 pm, last day of class, paper submission due one week earlier}


\section{Transistor Testing}

Use the 
\htmladdnormallink{Type 575 semiconductor curve tracer}{http://en.wikipedia.org/wiki/Semiconductor_curve_tracer}
in the electronic lab to test 
\htmladdnormallink{2N3904}{http://www.fairchildsemi.com/ds/2N/2N3904.pdf}
transistors in common emitter mode. 
Observe and take pictures of both the input and output characteristic plots. 


{\bf Important note:} To make sure you do not burn your transistor, 
(a) study the datasheet of the transistor to determine the leads for E, B and C,
(b) do not apply $V_{CC}$ voltage higher than 30 V.

\htmladdimg{../2n3904.gif}

{\bf Initial control knob settings before you start:}
\begin{itemize}
\item {\bf Display control} (top-right)
  \begin{itemize}
  \item Vertical axis (for $I_B$ or $I_C$): 1 mA/Div
  \item Horizontal axis ($V_{BE}$ or ``base voltage''): 0.1 or 0.5 V/Div
  \item Horizontal axis ($V_{CE}$ or ``collector voltage''): 2 V/Div
  \end{itemize}
\item {\bf Collector voltage} (lower-left)
  \begin{itemize}
  \item Peak voltage range: x1 (be careful to use x10 as maximum voltage is 200 V!)
  \item Polarity: + for npn transistors
  \item Peak Volt ($V_{CC}$): 10 to 20 Volts
  \item Dissipation limiting resistor ($R_C$): 1 k$\Omega$
  \end{itemize}
\item {\bf Base voltage} (lower-right)
  \begin{itemize}
  \item Repetitive/Single scan: 

    single for input characteristic plot, repetitive for output characteristic plot.
  \item Steps/family: 12
  \item Polarity: +
  \item Steps/Sec: 240
  \item Series resistor ($R_B$) (no relevance here)
  \item Step selection: $0.005 mA = 5 \mu A$
  \end{itemize}
\end{itemize}

{\bf What to submit:}

\begin{itemize}
\item Take screenshots of both the input and output characteristic plots 
  observed on the semiconductor curve tracer with the following clear 
  labeled (with meaningful increment):
  \begin{itemize}
  \item The voltage (horizontal axis) $V_{BE}$ or $V_{CE}$
  \item The current (vertical axis) $I_B$ or $I_C$
  \item For the output plot, each one of the 12 $I_B$ values in the family
  \end{itemize}
\item Determine the value $\beta=I_C/I_B$ (also referred to as the 
  ``forward transfer current ratio'' and denoted by $h_f$ or $h_{FE}$ on
  the \htmladdnormallink{2N3904 datasheet}{http://www.fairchildsemi.com/ds/2N/2N3904.pdf}
  (also see \htmladdnormallink{class notes}{http://fourier.eng.hmc.edu/e84/lectures/ch4/node7.html})
  from the output plot for each of the 5 transistors.
\item Determine $V_{BE}$ from the input plot when $I_B$ is within the range of
  0.01 to 0.1 mA, i.e., $I_C$ is within the range of $0.01\;\beta$ to $0.1\;\beta$.
\item Describe how the load line changes in the output plot when $R_C$ and $V_{CC}$
  increases/decreases.
\end{itemize}


\section{Transistor Circuit Construction}


\begin{enumerate}

\item Design and build the voltage amplifier based on fixed biasing
  shown on the left in the figure below. Determine the value of $R_B$,
  $R_C$, and $V_{CC}$ to achieve maximum voltage amplification and 
  minimum distortion by setting the DC operating point to be in the 
  middle of the linear range of the output characteristic plot. 
  
  \begin{itemize}
  \item Calculate the DC operating point on the output characteristic 
    plot ($I_{C}$ and $V_{CE}$) based on your design, and compare it
    to the actual one measured from the actual circuit. 

  \item Test the circuit by a sinusoidal signal of 50 mV peak-to-peak 
    amplitude from signal generator (with output resistance 
    $R_S=50\,\Omega$) and an oscilloscope (with input resistance
    $R_{in}=1\,M\Omega$) to monitor the input and output signals both 
    before and after the amplification and to observe the voltage gain 
    and waveform distortion. Compare the predicted gain based on the
    small signal model with the actual one.

  \item Observe the polarity inversion and the signal distortion. 
    Increase the amplitude of the input sinusoid from 50 mV to 1 V 
    and observe the output in terms of the amount of distortion and 
    clipping at both the positive and negative peaks of the sinusoid.
  \end{itemize}

\item Repeat the above for the amplifier based on self biasing shown 
  on the right of the figure below. Determine the values of $R_1$, 
  $R_2$, $R_C$, and $R_E$ to achieve maximum voltage amplification 
  and minimum distortion by setting the DC operating point to be in 
  the middle of the linear range of the output characteristic plot. 

  \begin{itemize}
  \item Calculate the DC operating point on the output characteristic 
    plot ($I_{C}$ and $V_{CE}$) based on your design, and compare it
    to the actual one measured from the actual circuit. 

  \item Test the circuit by a sinusoidal signal of 50 mV peak-to-peak 
    amplitude from signal generator and an oscilloscope monitor the 
    input and output signals both before and after the amplification 
    and to observe the voltage gain and waveform distortion. Compare 
    the predicted gain based on the small signal model with the actual 
    one.

  \item Observe the voltage gain as a function of the by-pass capacitor 
    in parallel with $R_E$, and the capacitors at both the input and 
    output ports, by trying different $C$ values. 

  \item Observe the voltage gain as a function of the signal frequency.
    Generate a Bode plot of the magnitude of the voltage gain for the 
    frequency range of 10 Hz to 1 MHz. Find the maximum voltage gain 
    $A_{max}$ and the freqnecy range in which this maximum gain is 
    achieved.

  \item Observe the polarity inversion and the signal distortion. 
    Increase the amplitude of the input sinusoid from 50 mV to 1 V 
    and observe the output in terms of the amount of distortion and 
    clipping at both the positive and negative peaks of the sinusoid.
  \end{itemize}

\item Connect a load resistor $R_L$ to the amplificaiton circuit above, 
  measure the output voltage across $R_L$ as its value varies from 
  $1\,M\Omega$ to $10\,\Omega$ in decade scale. Then build an emitter
  follower and insert it in between the amplification cirucit and the 
  load $R_L$. Re-measure the output voltage across $R_L$ when its values 
  varies as before. Explain what you observe.

\end{itemize}

\htmladdimg{../../lectures/figures/ACamplification0a.png}.
\htmladdimg{../../lectures/figures/ACamplification1.gif}.
\htmladdimg{../../lectures/figures/EmitterFollower.png}.

{\bf What to submit:}

A text document that discusses concisely each of the items specified 
above, including 
\begin{itemize}
\item the design process by which you determin the circuit parameters
  (e.g., resistance values).
\item comparison of the predicted and measured DC operating points.
\item comparison of the predicted and measured AC voltage gains.
\item pictures of the screenshots of the input/output voltages 
  of the circuits required in the lab. 
\end{itemize}


\end{document}


	

	

