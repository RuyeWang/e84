\documentstyle[12pt]{article}
\usepackage{html}
% \usepackage{graphics}  
\begin{document}

{\Large \bf E84 Lab 2: Construction of a Radio Receiver}

%{\bf Final construction due 5 pm, last day of class, paper submission due one week earlier}

Pick up the hardware of the radio receiver kit from the stock room,
including all the components and the construction manual.

\section{Part 1: Transistor Circuits}

\subsection{Transistor Testing}

Different types of transistors are used in the 
\htmladdnormallink{AM/FM radio receiver kit}{../amfm108TK.pdf}
of this lab, including five
\htmladdnormallink{2N3904}{http://www.fairchildsemi.com/ds/2N/2N3904.pdf}
transistors (also available in the stock room). Use the 
\htmladdnormallink{Type 575 semiconductor curve tracer}{http://en.wikipedia.org/wiki/Semiconductor_curve_tracer}
in the electronic lab to test these transistors in common emitter mode. 
Observe and take pictures of both the input and output characteristic plots. 


{\bf Important note:} To make sure you do not burn your transistor, 
(a) study the datasheet of the transistor to determine the leads for E, B and C,
(b) do not apply $V_{CC}$ voltage higher than 30 V.

\htmladdimg{../2n3904.gif}

{\bf Initial control knob settings before you start:}
\begin{itemize}
\item {\bf Display control} (top-right)
  \begin{itemize}
  \item Vertical axis (for $I_B$ or $I_C$): 1 mA/Div
  \item Horizontal axis ($V_{BE}$ or ``base voltage''): 0.1 or 0.5 V/Div
  \item Horizontal axis ($V_{CE}$ or ``collector voltage''): 2 V/Div
  \end{itemize}
\item {\bf Collector voltage} (lower-left)
  \begin{itemize}
  \item Peak voltage range: x1 (be careful to use x10 as maximum voltage is 200 V!)
  \item Polarity: + for npn transistors
  \item Peak Volt ($V_{CC}$): 10 to 20 Volts
  \item Dissipation limiting resistor ($R_C$): 1 k$\Omega$
  \end{itemize}
\item {\bf Base voltage} (lower-right)
  \begin{itemize}
  \item Repetitive/Single scan: 

    single for input characteristic plot, repetitive for output characteristic plot.
  \item Steps/family: 12
  \item Polarity: +
  \item Steps/Sec: 240
  \item Series resistor ($R_B$) (no relevance here)
  \item Step selection: $0.005 mA = 5 \mu A$
  \end{itemize}
\end{itemize}

{\bf What to submit:}

\begin{itemize}
\item Take screenshots of both the input and output characteristic plots 
  observed on the semiconductor curve tracer with the following clear 
  labeled (with meaningful increment):
  \begin{itemize}
  \item The voltage (horizontal axis) $V_{BE}$ or $V_{CE}$
  \item The current (vertical axis) $I_B$ or $I_C$
  \item For the output plot, each one of the 12 $I_B$ values in the family
  \end{itemize}
\item Determine the value $\beta=I_C/I_B$ (also referred to as the 
  ``forward transfer current ratio'' and denoted by $h_f$ or $h_{FE}$ on
  the \htmladdnormallink{2N3904 datasheet}{http://www.fairchildsemi.com/ds/2N/2N3904.pdf}
  (also see \htmladdnormallink{class notes}{http://fourier.eng.hmc.edu/e84/lectures/ch4/node7.html})
  from the output plot for each of the 5 transistors.
\item Determine $V_{BE}$ from the input plot when $I_B$ is within the range of
  0.01 to 0.1 mA, i.e., $I_C$ is within the range of $0.01\;\beta$ to $0.1\;\beta$.
\item Describe how the load line changes in the output plot when $R_C$ and $V_{CC}$
  increases/decreases.
\end{itemize}


\subsection{Transistor Circuit Construction}

Design and build the two basic transistor voltage amplifiers shown in the 
figure below to achieve maximum voltage amplification and minimum distortion 
by setting the DC operating point to be in the middle of the linear range of
the output characteristic plot. Use $V_{CC}\le 20\;V$. 

Test the circuit by a sinusoidal signal of 50 mV peak-to-peak amplitude from
a signal generator (with output resistance $R_S=50\,\Omega$) and an oscilloscope
(with input resistance $R_{in}=1\,M\Omega$) to monitor the input and output 
signals both before and after the amplification and to observe the voltage 
gain and waveform distortion. 

Calculate the DC operating point based on your design,
and compare it with the actual circuit. Display both input and output signals 
(sinusoidal) on the oscilloscope (use both input channels). 

Specifically do the following:
\begin{itemize}
\item Determine the values of $R_1$, $R_2$, $R_C$, and $R_E$ so that (a) the 
  DC operating point is in the middle of the load line for minimum distortion
  and (b) the voltage gain is maximized. 

\item Measure the DC operating point ($I_{C}$ and $V_{CE}$) experimentally and
  compare it with the predicted based on your design.

\item Observe the voltage gain as a function of the by-pass capacitor in 
  parallel with $R_E$, and the capacitors at both the input and output ports,
  by trying different $C$ values. 

\item Observe the voltage gain as a function of the signal frequency.
  Generate a Bode plot of the magnitude of the voltage gain for the 
  frequency range of 10 Hz to 1 MHz. Find the maximum voltage gain $A_{max}$
  and the freqnecy $f_{max}$ at which this maximum gain is achieved.

\item Observe the polarity inversion and the signal distortion. Increase
  the amplitude of the input sinusoid from 50 mV to 1 V and observe the 
  output in terms of the amount of distortion and clipping at both the 
  positive and negative peaks of the sinusoid.

\end{itemize}

{\bf What to submit:}

A text document including pictures of the screenshots to discuss concisely
each of the items specified above.

\htmladdimg{../../lectures/figures/ACamplification0a.png}.
\htmladdimg{../../lectures/figures/ACamplification1.gif}.


\section{Part 2: Understanding Radio Reception}

Read and understand the following topics all related to this radio receiver lab. You 
are encouraged to search the Internet to find additional (possibly more thorough) 
descriptions of the topics.

\begin{itemize}
\item \htmladdnormallink{Radio Receiver}{http://en.wikipedia.org/wiki/Receiver_(radio)}
\item \htmladdnormallink{Superheterodyne Receiver}{http://en.wikipedia.org/wiki/Superheterodyne_receiver}
\item \htmladdnormallink{Frequency Mixer}{http://en.wikipedia.org/wiki/Frequency_mixer}
  (also \htmladdnormallink{here}{http://fourier.eng.hmc.edu/e84/lectures/ch4/node10.html})
\item \htmladdnormallink{AM/FM demodulation}{http://www.electronics-tutorials.com/receivers/fm-radio-receivers2.htm}
\item \htmladdnormallink{Foster-Seeley}{http://en.wikipedia.org/wiki/Foster-Seeley_discriminator}
\item \htmladdnormallink{Ratio FM detector}{http://www.radio-electronics.com/info/receivers/fm_demod/ratio-foster-seeley-fm-detector-discriminator.php}
\item \htmladdnormallink{Rectification}{http://en.wikipedia.org/wiki/Rectifier}
\item \htmladdnormallink{Amplifier}{http://en.wikipedia.org/wiki/Class_C_amplifier#Class_C}
\item \htmladdnormallink{Push-Pull Amplifier}{http://en.wikipedia.org/wiki/Push-pull_output}
\end{itemize}

{\bf What to submission:}

Write a short paragraph to explain the principle of each of the following three 
topics in your own words. Make sure you explain the goal as well as the method. 
Be concise and to the point. Use no more than one single sided page.

\begin{itemize}
\item Superheterodyne receiver
\item AM demodulation
\item FM demodulation
\end{itemize}

\section{Part 3: Construction of the radio receiver}


\end{document}


	

	

