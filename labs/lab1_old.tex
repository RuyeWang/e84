\documentstyle[12pt]{article}
\usepackage{html}
\textwidth 6.0in
\topmargin -0.5in
\oddsidemargin -0in
\evensidemargin -0.5in
% \usepackage{graphics}  
\begin{document}

\section*{E84 Lab 1: Design of a Multimeter}

%{\bf Design on paper due 2/18 Wednesday in class}

%{\bf Final project due by 3/4  Wednesday}


The schematic diagram of a multimeter is shown in the figure below:
\htmladdimg{../multimeter1.gif}

\begin{itemize}
  \item Pick up the hardware of the multimeter kit from the stock room,
    including all the components (resistors, potentiometer, meter head, 
    rotary switch, batteries, etc.). 

  \item Study the diagram above carefully to understand how the meter 
    measures the voltages, currents and resistances at each of positions 
    of the rotary positions. The rectangular shape to the left of the meter 
    head (a circle with an arrow in it) represents two parallel diodes 
    with opposite polarity for the protection of the meter head. You don't
    need to worry about this. 

  \item The part marked by 10KBVR is a potentiometer, a resistor of 10
    $K\Omega$ but with an additional terminal which is a piece of metal 
    that can slide along the resistor, so that the resistance between the 
    3rd terminal and any of the other two is variable between 0 and 10 
    $K\Omega$. See
    \htmladdnormallink{here}{http://en.wikipedia.org/wiki/Potentiometer}
    for more detailed explanation. This potentiometer is needed to calibrate
    the Ohmmeter. Specifically, before taking the measurement of a resistance,
    the potentiometer needs to be adjusted so that the needle points to 
    zero on the resistance scale when the two leads of the multimeter touch
    each other for zero resistance.
    
  \item Understand how the rotary switch works. It is represented by the 
    18 circles on top of the diagram and the two layers of rectangular shapes.
    At each of the 18 positions, the corresponding circle on top is connected
    to both of the rectangles, while all remaining circles are not connected
    to anything. If you still have difficulty understanding this, it should
    be most helpful if you take a look at the physical parts and the printed 
    circuit board (PCB) in the kit. The diagram provided above may contain
    minor errors. Whenever in doubt, take a look at the physical parts of
    the hardware.

  \item Find the resistance for each of the 13 resistors with their values 
    erased.

  \item Turn in your design on paper with all the resistance values before
    the first deadline (to be arranged). After the deadline the solution key
    and the complete diagram will be available 
    \htmladdnormallink{here}{../solution1/index.html}

  \item Assemble the multimeter in the lab, verify every single position 
    to see if the meter works as expected.
    
\end{itemize}

{\bf How to find the polarity of a diode? (not necessarily trivial!)}

When the voltage at the anode (labeled by a triangle) of a diode is higher
than that of the cathode, the diode has a very low resistance (close to a short
circuit, called forward biased). If the polarity is reversed, the resistance 
of the diode becomes very high (close to an open circuit, called reverse biased).
The polarity of the diode can therefore be found by checking its resistance 
using a multimeter.

However, we need to know which lead of the multimeter (when used as an ohmmeter) 
is positive and which is negative. In most analog multimeters, such as the one 
you are building, the positive lead (marked by +) is connected to the negative 
end of the internal battery, while the other lead (marked by -, or COM for common)
is connected to positive end of the battery. So if the measured resistance of the 
diode is low, the end connected to the COM lead of the multimeter is the anode
(triangle).

However, the polarities of digital multimeters may be the opposite, i.e., the 
lead marked by + may be connected to the positive end of the internal battery.
In this case, the polarity of the diode can be determined in the opposite way
compared to the method above using an analog multimeter. 

Moreover, some digital multimeters have a particular position for diode measurement. 

Read \htmladdnormallink{this page}{http://www.allaboutcircuits.com/vol_3/chpt_3/2.html}
for more detailed discussions.

{\bf How to solder?} 

If you have never done soldering before, you should find 
\htmladdnormallink{this web}{http://www.elecraft.com/TechNotes/N0SS_SolderNotes/N0SS_SolderNotesV6.pdf}
useful.

\end{document}


	

	

