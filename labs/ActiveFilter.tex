\documentstyle[12pt]{article}
\usepackage{html}
% \usepackage{graphics}  
\begin{document}

{\Large \bf E84 Lab: Active Filters}


\begin{itemize}

\item Derive the frequency response functions of the second 
  order Sallen-Key low-pass and high-pass filters discussed 
  \htmladdnormallink{here}{http://fourier.eng.hmc.edu/e84/lectures/ActiveFilters/node3.html}.

\item Derive the frequency response function of the active twin-T 
  filter discussed \htmladdnormallink{here}{http://fourier.eng.hmc.edu/e84/lectures/ActiveFilters/node4.html}.

  Here $R_4$ and $R_5$ are the top and bottom parts of a potentiameter
  which controls the amount of feedback from the output to the 
  twin-T filter:
  \[
  V_{feedback}=\frac{R_5}{R_4+R_5}V_{out}=\alpha V_{out},\;\;\;\;\;\;  \alpha=R_5/(R_4+R_5)
  \]
  The middle point between $R_4$ and $R_5$ is the wiper position of 
  the potentiameter that can move from top ($R_4=0$, $\alpha=1$) to 
  bottem ($R_5=\alpha=0$).

  Express the frequency response function (FRF) of the circuit in the 
  canonical 2nd order form and determine $\omega_n$ and $Q$ in terms 
  of R, C, and $\alpha$. 

\item Select component values for your twin-T filter circuit so that 
  the 60 Hz powerline noise is attenuated by 20 dB or more, while 
  frequencies below 50 and above 70 Hz should be attenuated by less 
  than 3 dB. The width of the twin-T notch filter is configurable using 
  the potentiometer. Use $C=270\,pF$. Make a Bode plot and show your 
  design meets these specifications.

\item Select component values for the Sallen-Key filter so that it 
  can be used as an antialias low-pass filter that have a passband 
  of 0-140 Hz and a stopband of 1 KHz and above to allow for 2 KHz 
  sampling. The attenuation should be less than 3 dB in the passband 
  but >30 dB in the stopband. Make a Bode plot to show your design 
  meets these specifications.

\item Construct the filter circuit that combines the twin-T filter 
  and the Sallen-Key filter you designed above. Test and verify the
  circuit to show that its FRF satisfies the specifications.

  \htmladdimg{../TwinTLab.png}

\item Test your EKG circuit with and without the filter, and show that 
  the filter eliminates 60 Hz (which may or may not be detectable, 
  depending on the specific location in the lab).


\end{itemize}


What to turn in:
\begin{enumerate}
  \item Hand analysis of your design of both filters in terms of the 
    component value selection. The Bode plots of their FRFs.
  \item Multisim Bode plots showing that your filter circuit agrees 
    with the hand analysis and meets the specifications.
  \item Measurement results of the circuit showing that its FRF
    matches the simulation and the specifications. Pay special 
    attention to frequencies of interest such as the shape of the 
    notch and the transition between pass and stop band.
  \item Test results of the filter circuit with the EKG signals.
\end{enumerate}

\end{document}


 

 

