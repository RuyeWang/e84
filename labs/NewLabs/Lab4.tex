\documentstyle[12pt]{article}
\usepackage{html}
\textwidth 6.0in
\topmargin -0.5in
\oddsidemargin -0in
\evensidemargin -0.5in
% \usepackage{graphics}  
\begin{document}

{\bf \Large E84 Lab 4: Design, Build, and Test a Multimeter}

In this lab you are to use either of the two multimeter kits (AMK200 and
M1250) available in the stockroom to build a multimeter. The lab contains 
the following two  parts:
\begin{itemize}
\item {\bf Part I: The warm-up problems}

\begin{enumerate}
\item Measurement of a physical process by instruments may be tricky 
  due to the 
  \htmladdnormallink{inevitable interfere}}{http://en.wikipedia.org/wiki/Observer_effect_(physics)}
  on the process being caused by the instruments (remember what you learned 
  in quantum mechanics?). The figure below shows two possible configurations 
  for the measurement of the voltage across and the current through the load. 

  \htmladdimg{../hw1b.gif}

  What are required of the ammeter and the voltmeter to minimize their
  influences on the measurements? 

  % The ammeter should have minimum (ideally 0) impedance while the voltmeter
  % should have maximum (ideally infinity) impedance. 

  How would the ammeter and the voltmeter affect the measurement of the
  current and the voltage in either of the configurations (a and b)?

  % In (a) the voltmeter will by-pass some current so that the actual current
  % through the load is smaller than the reading of the ammeter. In (b) the
  % ammeter will cause some voltage drop and the actual voltage across the load
  % is lower than the reading of the voltmeter. 

\item Design a multimeter that can measure both DC and AC voltage, DC current,
  and resistance with different scales. Specifically, you are given an analog 
  meter $A$ with a needle display, which reaches full scale when a DC current 
  of $I=100\;\mu A=10^{-4}\;A$ goes through it. The internal resistance of the
  meter is 10 Ohms. In addition, you need some multi-position rotary switches 
  to select different scales for each of the three types of measurements, and 
  resistors with any values needed in your design.

  \begin{itemize}
  \item DC Voltage measurement: DC voltages in these ranges can be measured
    0-2.5, 0-10, 0-50, and 0-250 (all in volts). Use a 4-position rotary switch
    to select one of the four ranges as shown in the figure below. For example, 
    when the range of 0-10 is selected, the needle display will reach full scale 
    when the voltage being measured is 10 V. The circuit is shown below. Determine 
    all resistances labeled.

    \htmladdimg{../multimeterV.gif}

    %      {\bf Solutionn:} $R_1=25\;K\Omega$, $R_2=75\;K\Omega$, $R_3=400\;K\Omega$
    %      and $R_4=2\;M\Omega$.
  \item AC Voltage measurement: To measure an AC voltage (in terms of its
    RMS value), it first needs to be converted into a DC voltage. This can 
    be achieved by a diode which only allows the current to pass in one
    direction (along the arrow) but not the other. This process is called
    rectification. The diode will also cause a voltage drop of 0.7 volt 
    along the direction. The actual reading of the meter reflects the 
    average value of the rectified current, as show in the figure below.

    Find the resistance $R$ so that when the incoming AC voltage is $V=10$ 
    volt (RMS), the meter shows a full scale display.

    \htmladdimg{../HalfWaveRectifier4.png}

    \htmladdimg{../multimeterACV.gif}


%      {\bf Solution:} The peak value of voltage is $10\times \sqrt{2}=14.14$,
%      which is reduced (due to voltage drop of the diode) to $14.14-0.7=13.44$.
%      The average value of the rectified voltage is $13.44/\pi=4.28$. For the
%      meter to have a full scale display, the current need to be
%      \[ 10^{-4}=\frac{4.28}{R},\;\;\;\;\;\;\mbox{i.e.}\;\;\;\;\;\;R=42.8\,k\Omega \]
%      The internal resistance of $10\,\Omega$ can be neglected.

  \item DC current measurement: measure currents in these ranges (all in mA):
    0-0.5, 0-2.5, 0-10, 0-50. Use a 4-position rotary switch to select one 
    of the four ranges as shown in the figure below. For example, when the 
    range of 0-10 is selected, the needle display will reach full scale when 
    a 10 mA current is measured. Determine all resistances labeled. Use
    $R_0=1\;K\Omega$.

    \htmladdimg{../multimeterA1a.gif}

%      {\bf Solution:} Voltage across input is $V=0.1\;mA\times 1\;K\Omega=100\;mV$.
%      Therefore 
%      \[ R_1=100/(0.5-0.1)=250\;\Omega \]
%      \[ R_2=100/(2.5-0.1)=41.67\;\Omega \]
%      \[ R_1=100/(10-0.1)=10.1\;\Omega \]
%      \[ R_1=100/(50-0.1)=2\;\Omega \]

  \item Resistance measurement: The circuit for resistance measurement is
    provided as shown below, where $V_1=1.5V$. Determine the values for the 
    resistors labeled as $R_0$, $R_1$, $R_{10}$, $R_{100}$ and $R_{1000}$ 
    and $V_2$ so that the needle display of the meter is full scale 
    ($I=100\;\mu A$) when the resistor $R=0$ being measured (between the 
    two leads labeled + and -) is zero, or half scale ($I=50\;\mu A$) when 
    the value of $R$ and the position of he two synchronized rotary switches
    are given in each of the four case shown in the table:

    \begin{tabular}{l|llll} \hline
      positions  & $\times 1$ & $\times 10$ & $\times 100$ & $\times 1K$ \\
      $R$ values & 20$\Omega$ & 200$\Omega$ & 2000$\Omega$ & 20 $K\Omega$ 
    \end{tabular}

    \htmladdimg{../multimeterR.gif}

%      {\bf Solution:}
%
%      \begin{itemize}
%	\item First determine $R_0$: when $R=0$, we get 
%	  $R_0=1.5V/0.1\;mA=15\;K\Omega$.
%	\item When $R=20\;\Omega$, the current through meter $A$ should be:
%	  \begin{eqnarray}
%	    I&=&\frac{V_1}{R+R_1||R_0}\frac{R_1}{R_1+R_0}
%	  =\frac{V_1}{R+\frac{R_1R_0}{R_1+R_0}}\frac{R_1}{R_1+R_0}
%	  =V_1\frac{R_1}{RR_1+RR_0+R_1R_0} 	  \nonumber \\
%	  &=&5\times 10^{-4}\;A 
%	  \nonumber
%	  \end{eqnarray}
%	  Given $V_1=1.5V$, $R_0=15K$ and $R=20\Omega$, we can solve this 
%	  equation to get 
%	  \[ R_1=\frac{R_0RI}{V_1-I(R+R_0)}=20\Omega \]
%	\item When $R=200\;\Omega$, solving the above equation we get 
%	  $R_{10}=202.7\Omega$.
%	\item When $R=2\;K\Omega$, solving the above equation we get 
%	  $R_{10}=2307.7\Omega$.
%	\item When $R=20\;K\Omega$, we need to determine $V_2$ and 
%	  $R_{1k}$ so that $I=5\items 10^{-5} A$, and also when $R=0$, 
%	  $I=10^{-4}A$. $R_{1k}$ and $V_1+V_2$ can be found by solving
%	  these equations:
%	  \[ \left\{ \begin{array}{l}
%	    (V_1+V_2)/(R_{1k}+R_0)=10^{-4}A \\
%	    (V_1+V_2)/(R+R_{1k}+R_0)=5\times 10^{-5}A
%	  \end{array}\right. \]
%	  Solving we get $V_1+V_2=2V$, $R_{1k}+R_0=20\;K\Omega$, i.e.,
%	  $R_{1k}=5K\Omega$, $V_2=0.5V$
%      \end{itemize}
     
  \end{itemize}

\end{enumerate}


\item {\bf Part II: Design, build, and test the multimeter}

  You can use either of the two different multimeter models available in the 
  stockroom. They are very similar to each other, in terms of their circuits, 
  and the degree of difficulty to build. In this part, you need to understand 
  how the DC voltage and current are measured, and come up with the values of
  the resistors used for measuring DC voltage and current. (You are NOT required
  to do any calculation for the AC voltage or resistance measurement.)

  It will be considered as an honor code violation if you copy the values of 
  these resistors from the schematics of these multimeters which you may be 
  able to find on the Internet. Show your own derivation and keep at least 2
  digits after the decimal point in the values of the resistors.
  

  \begin{itemize}
  \item \htmladdnormallink{AMK-200}{http://fourier.eng.hmc.edu/e84/labs/AMK200/node1.html}

    Find the values of the resistors used for DC voltage and current measurements
    (values not shown in the diagram).

  \item \htmladdnormallink{M1250}{http://fourier.eng.hmc.edu/e84/labs/M1250/node1.html}

    Find the values of $R_1$ through $R_6$ for DC voltage measurement,
    and the values of $R_{10}$, $R_{11}$, and $R_{12}$ for DC current measurement.
  \end{itemize}
  
  Then, build and test the multimeter, which you can keep and use in the future!
  Make sure it works correctly for all measurements.

\end{itemize}



\section*{E84 Lab 1: Design of a Multimeter}

%{\bf Design on paper due 2/27 Monday in class}

%{\bf Final project due by 3/5 Monday in lab}


The schematic diagram of a multimeter is shown in the figure below:
\htmladdimg{../MultimeterA2.gif}

\begin{itemize}
  \item Pick up the hardware of the multimeter kit from the stock room,
    including all the components (resistors, potentiometer, meter head, 
    rotary switch, batteries, etc.). 

  \item Study the diagram above carefully to understand how the meter 
    measures the voltages, currents and resistances at each of the positions 
    of the rotary switch. The rectangular shape to the left of the meter 
    head (a circle with an arrow in it) represents two parallel diodes 
    with opposite polarity for the protection of the meter head. You don't
    need to worry about this. 

  \item The part marked by 10KBVR is a potentiometer, a resistor of 10
    $K\Omega$ but with an additional terminal which is a piece of metal 
    that can slide along the resistor, so that the resistance between the 
    3rd terminal and any of the other two is variable between 0 and 10 
    $K\Omega$. See
    \htmladdnormallink{here}{http://en.wikipedia.org/wiki/Potentiometer}
    for more detailed explanation. This potentiometer is needed to calibrate
    the Ohmmeter. Specifically, before taking the measurement of a resistance,
    the potentiometer needs to be adjusted so that the needle points to 
    zero on the resistance scale when the two leads of the multimeter touch
    each other for zero resistance.
    
  \item Understand how the rotary switch works. It is represented by the 
    18 circles on top of the diagram and the two layers of rectangular shapes
    on the bottom. At each of the 18 positions, the corresponding circle on 
    top is connected to both of the rectangles, while all remaining circles 
    are not connected to anything. If you still have difficulty understanding 
    this, it should be most helpful if you take a look at the physical parts 
    and the printed circuit board (PCB) in the kit.

%  \item The diagram provided above contain some errors. One major error is
%    in the circuit for DC current measurement (top-left corner), where the 
%    three resistors for 250 mA (1 Ohm), 25 mA and 2.5 mA should all be in 
%    parallel, instead of in series as shown in the diagram. You can verify 
%    this by looking at the actual PCB. The second error has to do with the
%    rotary switch. The rectangle of the lower right corner of the diagram
%    should be farther extended to the left as it is needed for all three
%    resistance measurement scales of x1, x10 and x1k Ohms. In general,
%    whenever in doubt, take a look at the physical parts of the hardware.

  \item Find the resistance for each of the 13 resistors with their values 
    erased.

  \item Turn in your design on paper with all the resistance values before
    the first deadline. 
%    After the deadline the solution key and the complete diagram will be 
%    available \htmladdnormallink{here}{../solution1/index.html}

  \item Assemble the multimeter in the lab, verify every single position 
    to see if the meter works as expected.
    
\end{itemize}

{\bf Hints} 

Following these steps:

\begin{itemize}
\item Study the meter head assembly circuitry and find out the needed voltage 
  across the meter in series with the $190\; \Omega$ resistor for the current 
  through the meter to be $43 \;\mu A$, needed for a full scale display. Then 
  find the currents through the two parallel branches with the resistances of 
  $32\; k\Omega$ and $10+18=28\;k\Omega$. Next find the total current needed 
  by this meter assembly composed of all three parallel branches for a 
  full-scale meter display. This current can be used through out the calculation 
  in the following steps.

\item Find the resistances needed to measure the currents and voltages at all
  scales.

\item Find out what values are needed for the upper and lower resistances of the 
  potentiometer during the calibration of the ohmmeter, i.e. when the two leads
  of the meter are directly connected (short circuit with zero resistance) for a 
  full-scale display.

\item Find the resistances needed to measure an unknown resistor value at all
  four scales. Note that the middle value of the resistance measurement is 20, 
  200, 20k, or 200k $\Omega$ for each of the scales, and the corresponding 
  current through the meter assembly should be half of the current needed for 
  a full scale display.

\end{itemize}




{\bf How to find the polarity of a diode? (not necessarily trivial!)}

When the voltage at the anode (labeled by a triangle) of a diode is higher
than that of the cathode, the diode has a very low resistance (close to a short
circuit, called forward biased). If the polarity is reversed, the resistance 
of the diode becomes very high (close to an open circuit, called reverse biased).
The polarity of the diode can therefore be found by checking its resistance 
using a multimeter.

However, we need to know which lead of the multimeter (when used as an ohmmeter) 
is positive and which is negative. In most analog multimeters, such as the one 
you are building, the positive lead (marked by +) is connected to the negative 
end of the internal battery, while the other lead (marked by -, or COM for common)
is connected to positive end of the battery. So if the measured resistance of the 
diode is low, the end connected to the COM lead of the multimeter is the anode
(triangle).

However, the polarities of digital multimeters may be the opposite, i.e., the 
lead marked by + may be connected to the positive end of the internal battery.
In this case, the polarity of the diode can be determined in the opposite way
compared to the method above using an analog multimeter. 

Moreover, some digital multimeters have a particular position for diode measurement. 

Read \htmladdnormallink{this page}{http://www.allaboutcircuits.com/vol_3/chpt_3/2.html}
for more detailed discussions.

{\bf How to solder?} 

If you have never done soldering before, you should find 
\htmladdnormallink{this}{http://fourier.eng.hmc.edu/e84/labs/N0SS_SolderNotesV6.pdf}
and \htmladdnormallink{this}{http://fourier.eng.hmc.edu/e84/labs/soldering_rev3.pdf}
useful.

\end{document}


	

	

