
\documentstyle[12pt]{article}
\usepackage{html}
\begin{document}


\section*{Filter design and testing}

Read 
\htmladdnormallink{this page}{http://fourier.eng.hmc.edu/e84/lectures/ch3/node17.html}
to make sure you are prepared before carrying out the lab.

\begin{enumerate}

\item Design an RLC series band-pass (BP) filter with the available components R, L and 
  C in the lab so that the passing band is centered around 7.5 kHz and bandwidth is about 
  1.3 kHz. Derive its frequency response function (FRF) $H(\omega)$, and use Matlab to 
  generate a linear plot of its magnitude as a function of frequency (from 0 to 20 kHz), 
  also make a Bode plot (from 0 to 100 kHz) using your own code. Feed a sinusoidal signal 
  $x(t)=\cos(2\pi ft)$ as the input to the filter and find the gain at a set of discrete
  frequencies including the resonant frequency (e.g., $f=1, 2, 5, 7.5, 10, 2\; kHz$). 
  Compare your results with your analysis and simulation.

\item Treat the same circuit as a low-pass (LP) filter (voltage across C as the output).
  Adjust $R$ properly so that $\zeta\ge 1/\sqrt{2}$, and the FRF does not peak. 
  Generate a linear and Bode plots of the magnitude of its FRF, same as what you did 
  before for the BP filter. Then feed a square wave signal as the input to the filter 
  and observe and describe the waveform of the output.

\item Repeat the previous part when the same circuit is treated as a high-pass (HP) 
  filter (voltage across L as the output).

\item Build the band-pass and band-stop filters discussed in class 
  (example at the bottom of \htmladdnormallink{this page}{http://fourier.eng.hmc.edu/e84/lectures/ch3/node17.html}),
  using the same R, L and C in previous last two parts. Geernate the linear and Bode plots 
  as before. Feed a sinusoidal signal as the input to the filter and find the gain at a
  set of discrete frequencies (including the resonant frequency). Compare your results 
  with your analysis and simulation.

\end{enumerate}
\end{document}
  
