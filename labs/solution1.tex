\documentstyle[12pt]{article}
\usepackage{html}
\textwidth 6.0in
\topmargin -0.5in
\oddsidemargin -0in
\evensidemargin -0.5in
% \usepackage{graphics}  
\begin{document}

\section*{E84 Lab 1: Design of a Multimeter (Solution)}

{\bf Solution:}

The complete schematic diagram of the multimeter:

\htmladdimg{../multimeter.gif}
    
The following is the calculation of all 13 resistors in the diagram.
Assume the negative input is ground (common input).

\begin{itemize}
\item {\bf The meter head assembly:}

  The resistance in meter branch is $1.8+0.19=1.99\; K\Omega$. The voltage across 
  this branch when $I_0=0.043\; mA$ (full scale) is $V_0=1.99\times 0.043=0.0856
  \approx 0.09\;V$. The current through the 32 K resistor is $0.0856/32=0.0027\;mA$.
  The combined current through these parallel branches is $0.043+0.0027=0.0457;\mA$, 
  and the resistance of these parallel branches is $R_0=32||1.99=1.8735\;K$. 

  With another parallel branch of $10+18=28\;K\Omega$ added, the resistance
  of the meter head assembly becomes $32||28||1.99=1.756\;K\Omega$. For the meter
  head to get the full scale current $0.043\;mA$, the total current to these three
  parallel branches needs to be
  \[ V_0(\frac{1}{32}+\frac{1}{1.99}+\frac{1}{28})=0.0487 \approx 0.05\;mA \]

\item {\bf DC current measurement:}
  \begin{itemize}
  \item 250 mA scale:
      
    At this scale, the total resistance through the meter head branch is
    \[ 3,000+240+1,756=4,996\approx 5,000\;\Omega \]
    There is a 1 $\Omega$ resistor in parallel with the current being measured.
    According to the current division, the current through the meter head is:
    \[ 250 \frac{1}{4,996+1}=250 \frac{1}{5000}=0.05\;mA \]
    corresponding to the full-scale current through the meter head.
      
  \item Other scales:

    Assume the resistor needed for measuring current $I$ at the scale is $R$, for
    the meter to be at full scale, we need to have:
    \[ I \frac{R}{4,996+R}=0.05\;mA \]
    Solving this we get:
    \[ R=\frac{5,000\times 0.05}{I+0.05}=\frac{250}{I+0.05} \]
    When $I=250,\;25,\;2.5\,mA$, we get respectively,
    $R=1,\;10,\;98\;\Omega$.

  \end{itemize}

\item {\bf DC voltage measurement:}

  \begin{itemize}
  \item 0.5V scale:

    At this scale, the total resistance in series with the voltage being measured 
    is $4.9+3+0.24=8.14\;K\Omega$. This forms a voltage divider with the meter head
    assembly, and the voltage across it is:
    \[ 0.5V \frac{1.756}{8.14+1.756} =0.5V \frac{1.756}{9.896}=0.089\approx 0.09\;V \]
    corresponding to the full-scale current through the meter head.

  \item Other scales:

    Assume the resistor needed for measuring voltage $V$ at the scale is $R$, for 
    the meter to be at full scale, we need to have:
    \[ V\frac{1.756}{9.896+R}=V_0=0.09\;V \]
    Solving this we get:
    \[ R=\frac{1.756V-9.896\times 0.09}{0.09}  \]
    When $V=2.5,\;10,\;50,\;250,\;1000$, we get respectively:
    $R=38.9,\;185.2,\;965.7,\;4,868,\;19,501$ in $K\Omega$, which can be converted
    to the resistances needed for the configuration in the schematic:
    $39,\;185-39=146,\;966-185=781,\;4,868-966=3,902,\;14,633$

  \end{itemize}

\item {\bf AC voltage measurement:}

  \begin{itemize}

  \item 10V scale:
    At this scale, the total resistance in series with the voltage being measured 
    is $83\;K\Omega$. This forms a voltage divider with the meter head assembly, 
    and the voltage across it is:
    \[ 10 \frac{1.756}{83+1.756}=10V \frac{1.756}{84.756}\approx 0.21V \]
    which corresponds to full scale display.

  \item Other scales:

    Assume the resistor needed for measuring voltage $V$ at the scale is $R$, for 
    the meter to be at full scale, we need to have:
    \[ 50 \frac{1.756}{R+1.756}=0.207V \]
    Solving this we get:
    \[ R=\frac{1.756V-0.35}{0.207} \]
    When $V=10,\;50,\;250,\;50,\;1000$, we get respectively:
    $R=83,\;423,\;2,119,\;8,481$ in $K\Omega$, which can be converted to the 
    resistances needed for the configuration in the schematic:
    $83,\;423-83=340,\;2,119-423=1,696,\;8,481-2,119=6362\;K\Omega$.

  \end{itemize}

\item {\bf Resistance measurement:}

  \begin{itemize}
  \item To calibrate the meter, we connect the two leads for zero resistance 
  and the current through the meter head assembly should be $0.05$ mA for 
  the pointer to indicate $0 \Omega$.
  The voltage across the meter head assembly is
  \[ V'_0=I_0\times R_0=0.05\times 1.8735=0.0937\;V \]
  Assume the lower resistance of the potentiometer is $x$ K, then the upper 
  one is $10-x$ K. 
  The voltage of the middle lead of the potentiometer (with respect to ground) is:
  \[ V_p=0.05\times (1.8735+10-x)=0.05\times (11.8735-x)  \]
  The current through $R_2=44$ K should be:
  \[ 0.05+\frac{V_p}{18+x}=\frac{3-V_p}{R_2}=\frac{3-V_p}{44} \]
  i.e.
  \[ (x+62)V_p=(x+62)(0.05\times (11.8735-x))=0.8(x+18) \]
  i.e.
  \[ x^2+66.1265\,x-448.157=0 \]
  Solving this we get $x=6.1966\approx 6.2\;K\Omega$.
  The resistance along the meter head branch is $R_1=11.8735-x=11.8735-6.1966=5.6769$ K,
  the resistance of the parallel branch is $R_1=18+x=24.1966$ K, and the resistance
  of their parallel combination is $5.6769||24.1966=4.6\; K\Omega$.
  
  \htmladdimg{../multimeter2.gif}

\item When the resistance being measured is 20 $\Omega$ (x1, x10, x1k) and the 
  switch is at the corresponding position for resistance measurement, the current 
  through the meter head assembly should be 0.025 mA so that the needle points to
  the middle value 20 of the scale for resistance. The voltage at the middle lead 
  of the potentiometer is:
  \[ V_p=5.6769\times 0.025=0.1419\;V \]
  The current through 18 K resistor is
  \[ I_2=V_p/24.1966=0.0059\;mA \]
  The combined current is $.0.025+0.0059=0.0309\approx 0.031$ mA.

  The resistance of two parallel branches (meter and $18 K$ resistor) is:
  \[ R_1 || R_2=5.6769||24.1966=4.598\;K \]
  The series combination with $R_3=44\;K$ is $R_3+R_1 || R_2\approx 48.6\;K$.
  Let $R$ be the resistor needed to measure resistance $r$ (in positions x1, 
  x10 and x1k), then this equation needs to be satisfied:

  \[ \frac{3}{48.6||R+r}\frac{R}{48.6+R}=\frac{3R}{48.6R+r(48.6+R)}=0.031 \]
  Solving for $R$, we get
  \[ R=\frac{0.031\times 48.6R}{3-0.031\times (48.6+R)} \]
  Use $r=0.02,\; 0.2,\; 20$, all in $K\Omega$, we can obtain, respectively,
  $R=20\Omega,\; 200\Omega,\; 34\,K\Omega$. These values are very close to the
  actual resistances used $R=17.8\Omega,\;195\Omega,\;31\;K\Omega$.
  

\end{itemize}

